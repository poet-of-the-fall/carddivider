% Basic settings for this card set
\renewcommand{\cardcolor}{seaside}
\renewcommand{\cardextension}{Update Pack}
\renewcommand{\cardextensiontitle}{Seaside}
\renewcommand{\seticon}{seaside.png}

\clearpage
\newpage
\section{\cardextension \ - \cardextensiontitle \ (Rio Grande Games 2022)}

\begin{tikzpicture}
	\card
	\cardstrip
	\cardbanner{banner/orange.png}
	\cardicon{icons/coin.png}
	\cardprice{3}
	\cardtitle{Affe}
	\cardcontent{Dies gilt auch für Karten, die der Spieler in Zügen anderer Mitspieler nimmt, wie z.B. einen \emph{FLUCH} in deinem Zug durch die \emph{HEXE}.}
\end{tikzpicture}
\hspace{-0.6cm}
\begin{tikzpicture}
	\card
	\cardstrip
	\cardbanner{banner/goldorange.png}
	\cardicon{icons/coin.png}
	\cardprice{3}
	\cardtitle{Astrolabium}
	\cardcontent{Durch das \emph{ASTROLABIUM} erhältst du +1 Kauf und +\coin[1] in dem Zug, indem du es spielst, und in deinem nächsten Zug ebenfalls +1 Kauf und +\coin[1].}
\end{tikzpicture}
\hspace{-0.6cm}
\begin{tikzpicture}
	\card
	\cardstrip
	\cardbanner{banner/white.png}
	\cardicon{icons/coin.png}
	\cardprice{3}
	\cardtitle{Seekarte}
	\cardcontent{Wenn du eine gleiche Karte wie die aufgedeckte im Spiel hast, einschließlich der gespielten \emph{SEEKARTE} oder einer Dauerkarte, die du aus einem früheren Zug im Spiel hast, nimm die aufgedeckte Karte auf deine Hand. Ansonsten legst du die aufgedeckte Karte oben auf deinen Nachziehstapel zurück.}
\end{tikzpicture}
\hspace{-0.6cm}
\begin{tikzpicture}
	\card
	\cardstrip
	\cardbanner{banner/orange.png}
	\cardicon{icons/coin.png}
	\cardprice{4}
	\cardtitle{Blockade}
	\cardcontent{Die genommene Karte kommt aus dem Vorrat und wird zur Seite gelegt. Lege sie auf die \emph{BLOCKADE}, damit du weißt, für welche Karte die \emph{BLOCKADE} gilt. Falls die genommene Karte aus irgendeinem Grund nicht zur Seite gelegt bleibt (z.B. weil du sie mit einem \emph{WACHTURM} aus \emph{Blütezeit} entsorgst), wird die Blockade im aktuellen Zug abgeräumt. Ist die Karte zur Seite gelegt, nimmst du sie in deinem nächsten Zug auf deine Hand. Bis dahin gilt: Nehmen Mitspieler eine gleiche Karte in ihren Zügen, nehmen sie zusätzlich einen \emph{FLUCH}.}
\end{tikzpicture}
\hspace{-0.6cm}
\begin{tikzpicture}
	\card
	\cardstrip
	\cardbanner{banner/orange.png}
	\cardicon{icons/coin.png}
	\cardprice{4}
	\cardtitle{\scriptsize{Gezeitenbecken}}
	\cardcontent{Wenn du diese Karte spielst, ziehst du 3 Karten und erhältst +1 Aktion, aber zu Beginn deines nächsten Zuges musst du 2 Karten ablegen. Wenn du nur 1 Karte auf deiner Hand hast, legst du diese eine Karte ab. Und wenn du keine Karten auf deiner Hand hast, legst du keine ab. Wenn du mehrere Dauerkarten im Spiel hast, deren Anweisungen zu Beginn deines nächsten Zuges wirken, kannst du sie beliebig sortieren. Hast du zum Beispiel vier \emph{GEZEITENBECKEN} und eine \emph{WERFT}, könntest du alle deine Karten mit den \emph{GEZEITENBECKEN} ablegen und dann Karten für die \emph{WERFT} ziehen.}
\end{tikzpicture}
\hspace{-0.6cm}
\begin{tikzpicture}
	\card
	\cardstrip
	\cardbanner{banner/orange.png}
	\cardicon{icons/coin.png}
	\cardprice{4}
	\cardtitle{Seefahrerin}
	\cardcontent{Wenn du eine Dauerkarte in dem Zug nimmst, in dem du die \emph{SEEFAHRERIN} spielst, ist das Spielen der Dauerkarte optional. Diese Karte wirkt kumulativ: Wenn du zwei \emph{SEEFAHRERINNEN} spielst, darfst du bis zu zwei genommene Dauerkarten spielen. Allerdings kannst du nicht mit zwei \emph{SEEFAHRERINNEN} ein und dieselbe Dauerkarte zweimal nacheinander spielen.
	
	\medskip

	Die \emph{SEEFAHRERIN} betrifft alle von dir genommenen Dauerkarten, also z.B. gekaufte oder mit Karten wie der \emph{WERKSTATT} genommene. Nimmst du eine Dauerkarte in deiner Kaufphase, kannst du sie durch die \emph{SEEFAHRERIN} spielen, obwohl du in deiner Kaufphase bist. Gibt dir eine solche Karte +Aktionen, darfst du dadurch keine Aktionskarten in deiner Kaufphase spielen. Wenn du dadurch Geldkarten ziehst, darfst du sie nur spielen, wenn du noch keine Karten gekauft hast. Die Fähigkeit Dauerkarten zu spielen, hast du nur in dem Zug, in dem du die \emph{SEEFAHRERIN} spielst. In deinem nächsten Zug erhältst du einfach nur +\coin[2] und darfst eine Karte aus deiner Hand entsorgen. }
\end{tikzpicture}
\hspace{-0.6cm}
\begin{tikzpicture}
	\card
	\cardstrip
	\cardbanner{banner/orange.png}
	\cardicon{icons/coin.png}
	\cardprice{5}
	\cardtitle{\footnotesize{Korsarenschiff}}
	\cardcontent{Das entsorgte \emph{SILBER} oder \emph{GOLD} gibt in seinem Zug trotzdem für den Spieler, der es entsorgt.

	\medskip

	Wenn du \emph{SILBER} und \emph{GOLD} in einem Zug spielst und vom Angriff des \emph{KORSAREN} betroffen bist, wird nur eine Karte davon (die zuerst gespielte) entsorgt. }
\end{tikzpicture}
\hspace{-0.6cm}
\begin{tikzpicture}
	\card
	\cardstrip
	\cardbanner{banner/orange.png}
	\cardicon{icons/coin.png}
	\cardprice{5}
	\cardtitle{Meerhexe}
	\cardcontent{Wenn du diese Karte spielst, ziehst du 2 Karten und jeder Mitspieler nimmt einen \emph{FLUCH}. Zu Beginn deines nächsten Zuges ziehst du 2 Karten und legst dann 2 Karten ab.}
\end{tikzpicture}
\hspace{-0.6cm}
\begin{tikzpicture}
	\card
	\cardstrip
	\cardbanner{banner/orangeblue.png}
	\cardicon{icons/coin.png}
	\cardprice{5}
	\cardtitle{Piratin}
	\cardcontent{Du kannst diese Karte spielen, wenn du eine Geldkarte nimmst, oder wenn ein Mitspieler eine Geldkarte nimmt. Spielst du diese Karte während des Zuges eines Mitspielers, ist dein nächster Zug der Zug, in dem du durch die \emph{PIRATIN} eine Geldkarte nimmst. Die Geldkarte, die du nimmst, kommt aus dem Vorrat und du nimmst sie direkt auf deine Hand.}
\end{tikzpicture}
\hspace{-0.6cm}
\begin{tikzpicture}
	\card
	\cardstrip
	\cardbanner{banner/white.png}
	\cardtitle{Platzhalter\quad}
\end{tikzpicture}
\hspace{0.6cm}
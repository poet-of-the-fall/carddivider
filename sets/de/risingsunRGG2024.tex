% Basic settings for this card set
\renewcommand{\cardcolor}{risingsun}
\renewcommand{\cardextension}{Erweiterung XV}
\renewcommand{\cardextensiontitle}{Rising Sun}
\renewcommand{\seticon}{risingsun.png}

\clearpage
\newpage
\section{\cardextension \ - \cardextensiontitle}

\begin{tikzpicture}
	\card
	\cardstrip
	\cardbanner{banner/white.png}
	\cardicon{icons/hex.png}
	\cardprice{5}
	\cardtitle{Bergschrein}
	\cardcontent{Diese Karte kostet \hex[5]; siehe \emph{Schulden} bei \enquote{Neue Regeln}. Es ist egal, wer eine Aktionskarte entsorgt hat und wann, solange eine Aktionskarte im Müll liegt. Die Aktionskarte im Müll kann auch eine Karte sein, die du gerade durch das Spielen von \emph{BERGSCHREIN} entsorgt hast.}
\end{tikzpicture}
\hspace{-0.6cm}
\begin{tikzpicture}
	\card
	\cardstrip
	\cardbanner{banner/white.png}
	\cardicon{icons/hex.png}
	\cardprice{6}
	\cardtitle{Daimyo}
	\cardcontent{Dies ist nicht optional. Was auch immer deine nächste Aktionskarte ist, die keine Befehlskarte ist, durch \emph{DAIMYO} spielst du sie erneut. Dies tust du sogar, wenn die Karte sich selbst entsorgt hat. Befehlskarten wie z.B. \emph{DAIMYO} spielst du nicht erneut. \emph{DAIMYO} wartet auf eine Aktionskarte, die keine Befehlskarte ist (bzw. bewirkt nichts weiter, wenn du deinen Zug beendest, ohne eine solche Karte gespielt zu haben). Wenn du zwei \emph{DAIMYOS} spielst, und dann z.B. einen \emph{HANDWERKER}, spielst du den \emph{HANDWERKER} insgesamt dreimal - einmal normal plus je einmal für jeden \emph{DAIMYO}. \emph{DAIMYO} kostet \hex[6], siehe \emph{Schulden} bei \enquote{Neue Regeln}.}
\end{tikzpicture}
\hspace{-0.6cm}
\begin{tikzpicture}
	\card
	\cardstrip
	\cardbanner{banner/white.png}
	\cardicon{icons/hex.png}
	\cardprice{8}
	\cardtitle{Künstlerin}
	\cardcontent{Diese Karte kostet \hex[8], siehe \emph{Schulden} bei \enquote{Neue Regeln}. Sie zählt selbst mit. Für diese Karte zählen auch Karten aus anderen Zügen, die noch im Spiel sind, wie \emph{SAMURAI} aus einem früheren Zug.}
\end{tikzpicture}
\hspace{-0.6cm}
\begin{tikzpicture}
	\card
	\cardstrip
	\cardbanner{banner/white.png}
	\cardicon{icons/coin.png}
	\cardprice{2}
	\cardtitle{\footnotesize{Fischhändlerin}}
	\cardcontent{Siehe \emph{Schattenkarten} bei \enquote{Neue Regeln}. Wenn du \emph{FISCHHÄNDLERIN} spielst, erhältst du +1 Kauf und +\coin[1]}
\end{tikzpicture}
\hspace{-0.6cm}
\begin{tikzpicture}
	\card
	\cardstrip
	\cardbanner{banner/white.png}
	\cardicon{icons/coin.png}
	\cardprice{2}
	\cardtitle{\footnotesize{Schlangenhexe}}
	\cardcontent{Deine Hand aufzudecken, wenn alle deine Handkarten unterschiedliche Namen haben, ist optional. Wenn du es tust, legst du \emph{SCHLANGENHEXE} auf ihren Stapel zurück, dann nehmen sich alle anderen Personen einen \emph{FLUCH}. Wenn du aus irgendeinem Grund \emph{SCHLANGENHEXE} nicht auf ihren Stapel zurücklegen kannst, nehmen die anderen Personen keinen \emph{FLUCH}. Du deckst deine Hand erst auf, nachdem du +1 Karte durch die \emph{SCHLANGENHEXE} gezogen hast. Die Bedingung „alle deine Handkarten verschieden" ist auch erfüllt, wenn du keine Handkarten hast.}
\end{tikzpicture}
\hspace{-0.6cm}
\begin{tikzpicture}
	\card
	\cardstrip
	\cardbanner{banner/white.png}
	\cardicon{icons/coin.png}
	\cardprice{3}
	\cardtitle{\footnotesize{Aristrokratin}}
	\cardcontent{Es kommt darauf an, wie viele \emph{ARISTOKRATINNEN} du im Spiel hast, nicht wie viele du in diesem Zug gespielt hast. Spielst du z.B. \emph{DAIMYO} und dann \emph{ARISTOKRATIN}, erhältst du zweimal +3 Aktionen. Hast du 0, 9 oder 10 \emph{ARISTOKRATINNEN} im Spiel, passiert nichts.}
\end{tikzpicture}
\hspace{-0.6cm}
\begin{tikzpicture}
	\card
	\cardstrip
	\cardbanner{banner/white.png}
	\cardicon{icons/coin.png}
	\cardprice{3}
	\cardtitle{Erdkeller}
	\cardcontent{Die \hex[] nimmst du auch, wenn du schon hast \hex[] siehe \emph{Schulden} bei \enquote{Neue Regeln}.}
\end{tikzpicture}
\hspace{-0.6cm}
\begin{tikzpicture}
	\card
	\cardstrip
	\cardbanner{banner/white.png}
	\cardicon{icons/coin.png}
	\cardprice{3}
	\cardtitle{Handwerker}
	\cardcontent{Du nimmst eine Karte, auch wenn du bereits \hex[] hattest, siehe \emph{Schulden} bei \enquote{Neue Regeln}.}
\end{tikzpicture}
\hspace{-0.6cm}
\begin{tikzpicture}
	\card
	\cardstrip
	\cardbanner{banner/orange.png}
	\cardicon{icons/coin.png}
	\cardprice{3}
	\cardtitle{Flussboot}
	\cardcontent{Wählt bei der Spielvorbereitung eine Aktionskarte, die keine Dauerkarte ist, \coin[5] kostet und in diesem Spiel nicht genutzt wird, und legt ein Exemplar davon zur Seite. Zum Auslosen einer solchen Karte könnt ihr die Platzhalterkarten benutzen. Erfordert diese Karte eine bestimmte Spielvorbereitung, führt diese Spielvorbereitung ebenfalls durch. Wenn du \emph{FLUSSBOOT} spielst, spielst du dadurch die zur Seite gelegte Karte zu Beginn deines nächsten Zuges. Dadurch wird die zur Seite gelegte Karte nicht bewegt; sie bleibt zur Seite gelegt, sogar wenn auf ihr Anweisungen stehen, aufgrund derer sie bewegt wurde, \emph{FLUSSBOOT} wird normalerweise in der Aufräumphase deines nächsten Zuges abgelegt, bleibt aber so lange im Spiel wie die Karte, die du durch sie spielst, was manchmal länger dauert (wie z.B. bei \emph{VOGELFREIE} aus \emph{Dark Ages}, benutzt für eine Dauerkarte).}
\end{tikzpicture}
\hspace{-0.6cm}
\begin{tikzpicture}
	\card
	\cardstrip
	\cardbanner{banner/white.png}
	\cardicon{icons/coin.png}
	\cardprice{4}
	\cardtitle{Dichterin}
	\cardcontent{Karten mit \hex[] in ihren Kosten kosten nicht \enquote{\coin[3] oder weniger}. Die aufgedeckte Karte legst du oben auf deinen Nachziehstapel zurück, wenn sie nicht auf deine Hand genommen wird.}
\end{tikzpicture}
\hspace{-0.6cm}
\begin{tikzpicture}
	\card
	\cardstrip
	\cardbanner{banner/white.png}
	\cardicon{icons/coin.png}
	\cardprice{4}
	\cardtitle{Flussschrein}
	\cardcontent{Es ist egal, ob du in deiner Aktionsphase Karten genommen hast, es zählt hier nur die Kaufphase. Spielst du mehrere \emph{FLUSSSCHREINE}, nimmst du für jeden eine Karte, vorausgesetzt, du hast in deiner Kaufphase keine Karte genommen. Das Entsorgen von Karten mit dem \emph{FLUSSSCHREIN} ist optional; du nimmst auch eine Karte, wenn du keine Karten entsorgt hast. Hast du mehrere Kaufphasen in deinem Zug, wie z.B. durch \emph{FORTSETZUNG}, nimmst du durch \emph{FLUSSSCHREIN} nur dann eine Karte, wenn du in keiner dieser Kaufphasen eine Karte genommen hast.}
\end{tikzpicture}
\hspace{-0.6cm}
\begin{tikzpicture}
	\card
	\cardstrip
	\cardbanner{banner/white.png}
	\cardicon{icons/coin.png}
	\cardprice{4}
	\cardtitle{Gasse}
	\cardcontent{Siehe \emph{Schattenkarten} bei \enquote{Neue Regeln}. Wenn du \emph{GASSE} spielst, ziehst du eine Karte, erhältst +1 Aktion und legst eine Karte ab.}
\end{tikzpicture}
\hspace{-0.6cm}
\begin{tikzpicture}
	\card
	\cardstrip
	\cardbanner{banner/white.png}
	\cardicon{icons/coin.png}
	\cardprice{4}
	\cardtitle{Ninja}
	\cardcontent{Siehe \emph{Schattenkarten} bei \enquote{Neue Regeln}. Wenn du \emph{NINJA} spielst, ziehst du eine Karte, dann legt jede andere Person so viele Handkarten ab, bis sie nur noch 3 Handkarten hat.}
\end{tikzpicture}
\hspace{-0.6cm}
\begin{tikzpicture}
	\card
	\cardstrip
	\cardbanner{banner/white.png}
	\cardicon{icons/coin.png}
	\cardprice{4}
	\cardtitle{\scriptsize{Rustikales Dorf}}
	\cardcontent{Zuerst wird +1 \suntoken ausgeführt, was eine Prophezeiung auslösen kann. Dann ziehst du +1 Karte und erhältst +2 Aktionen und du darfst 2 Karten ablegen (optional einschließlich der gerade gezogenen) für zusätzlich +1 Karte.}
\end{tikzpicture}
\hspace{-0.6cm}
\begin{tikzpicture}
	\card
	\cardstrip
	\cardbanner{banner/white.png}
	\cardicon{icons/coin.png}
	\cardprice{4}
	\cardtitle{Wandel}
	\cardcontent{Denke daran, dass du \hex[] zu einem beliebigen Zeitpunkt während deines Zuges tilgen kannst. Dadurch kannst du manchmal steuern, wie \emph{WANDEL} wirkt. Hast du mindestens einen \hex[], gibt \emph{WANDEL} dir +\coin[3]: wenn nicht, entsorgst du eine Handkarte, nimmst eine beliebige Karte, die mehr \coin[] kostet, und nimmst \hex[] in der Höhe der	Differenz in \coin[]. Du könntest z.B. ein \emph{KUPFER} entsorgen, eine \emph{PROVINZ} nehmen und \hex[8] nehmen. Du darfst keine Karte nehmen, die gleich viel oder weniger \coin[] kostet. Hierbei werden andere, spezielle Aspekte von Kosten ignoriert; du könntest z.B. ein \emph{ANWESEN} entsorgen und einen \emph{ALCHEMISTEN} nehmen (aus \emph{Alchemisten}), der \coin[3] und \potion kostet.}
\end{tikzpicture}
\hspace{-0.6cm}
\begin{tikzpicture}
	\card
	\cardstrip
	\cardbanner{banner/white.png}
	\cardicon{icons/coin.png}
	\cardprice{5}
	\cardtitle{Goldmine}
	\cardcontent{Du darfst ein \emph{GOLD} nehmen, auch wenn du schon \hex[] hast, siehe \emph{Schulden} bei \enquote{Neue Regeln}. Du kannst kein \emph{GOLD} nehmen, ohne \hex[4] zu erhalten.}
\end{tikzpicture}
\hspace{-0.6cm}
\begin{tikzpicture}
	\card
	\cardstrip
	\cardbanner{banner/white.png}
	\cardicon{icons/coin.png}
	\cardprice{5}
	\cardtitle{\tiny{Kaiserl. Gesandter}}
	\cardcontent{Die \hex[] nimmst du auch, wenn du schon \hex[] hattest. siehe \emph{Schulden} bei \enquote{Neue Regeln}.
	}
\end{tikzpicture}
\hspace{-0.6cm}
\begin{tikzpicture}
	\card
	\cardstrip
	\cardbanner{banner/white.png}
	\cardicon{icons/coin.png}
	\cardprice{5}
	\cardtitle{Kitsune}
	\cardcontent{Zuerst wird +1 \suntoken ausgeführt, wodurch eine Prophezeiung ausgelöst werden kann. Wähle dann zwei verschiedene Optionen und führe sie in der aufgeführten Reihenfolge aus.}
\end{tikzpicture}
\hspace{-0.6cm}
\begin{tikzpicture}
	\card
	\cardstrip
	\cardbanner{banner/white.png}
	\cardicon{icons/coin.png}
	\cardprice{5}
	\cardtitle{Reismakler}
	\cardcontent{Wenn du eine Karte entsorgst, die sowohl Geld- als auch Aktionskarte ist, ziehst du erst +2 Karten und dann +5 Karten. Entsorgst du eine Karte, die weder Geld- noch Aktionskarte ist, ziehst du keine Karten.}
\end{tikzpicture}
\hspace{-0.6cm}
\begin{tikzpicture}
	\card
	\cardstrip
	\cardbanner{banner/white.png}
	\cardicon{icons/coin.png}
	\cardprice{5}
	\cardtitle{Ronin}
	\cardcontent{Siehe \emph{Schattenkarten} bei \enquote{Neue Regeln}. Wenn du diese Karte spielst, ziehst du einzeln nacheinander Karten, bis du 7 Handkarten hast. Ggf. mischst du deinen Ablagestapel, falls der Nachziehstapel nicht ausreicht. Hast du bereits 7 oder mehr Handkarten, ziehst du keine.}
\end{tikzpicture}
\hspace{-0.6cm}
\begin{tikzpicture}
	\card
	\cardstrip
	\cardbanner{banner/white.png}
	\cardicon{icons/coin.png}
	\cardprice{5}
	\cardtitle{Sänfte}
	\cardcontent{Die \hex[] nimmst du auch, wenn du schon \hex[] hast; siehe \emph{Schulden} bei \enquote{Neue Regeln}.}
\end{tikzpicture}
\hspace{-0.6cm}
\begin{tikzpicture}
	\card
	\cardstrip
	\cardbanner{banner/white.png}
	\cardicon{icons/coin.png}
	\cardprice{5}
	\cardtitle{Tanuki}
	\cardcontent{Siehe \emph{Schattenkarten} bei \enquote{Neue Regeln}. Wenn du diese Karte spielst, entsorgst du eine deiner Handkarten und nimmst eine Karte, die bis zu \coin[2] mehr kostet als die eben entsorgte, so als wenn du \emph{UMBAU} spielst.}
\end{tikzpicture}
\hspace{-0.6cm}
\begin{tikzpicture}
	\card
	\cardstrip
	\cardbanner{banner/white.png}
	\cardicon{icons/coin.png}
	\cardprice{5}
	\cardtitle{Teehaus}
	\cardcontent{Zuerst wird +1 \suntoken ausgeführt, was eine Prophezeiung auslösen kann. Dann ziehst du +1 Karte und erhältst +1 Aktion und +\coin[2]}
\end{tikzpicture}
\hspace{-0.6cm}
\begin{tikzpicture}
	\card
	\cardstrip
	\cardbanner{banner/orange.png}
	\cardicon{icons/coin.png}
	\cardprice{6}
	\cardtitle{Samurai}
	\cardcontent{Wenn du \emph{SAMURAI} spielst, legt jede andere Person so viele Handkarten ab, bis sie nur noch 3 hat. Danach bleibt \emph{SAMURAI} im Spiel und du erhältst durch ihn +\coin[1] zu Beginn jeder deiner Züge bis zum Spielende. Die anderen Personen müssen aber nicht erneut Karten ablegen.}
\end{tikzpicture}
\hspace{-0.6cm}
\begin{tikzpicture}
	\card
	\cardstrip
	\cardbanner{banner/gold.png}
	\cardicon{icons/coin.png}
	\cardprice{7}
	\cardtitle{Reis}
	\cardcontent{Hast du z.B. \emph{DAIMYO}, \emph{SÄNFTE}, \emph{FISCHHÄNDLERIN}, drei \emph{KUPFER} und \emph{REIS} im Spiel, sind die Kartentypen AKTION, BEFEHL, SCHATTEN und GELD, also erhältst du durch \emph{REIS} +\coin[4].}
\end{tikzpicture}
\hspace{-0.6cm}
\begin{tikzpicture}
	\card
	\cardstrip
	\cardbanner{banner/blue.png}
	\cardtitle{\scriptsize{Prophezeiungen (1/7)\quad\quad}}
	\cardcontent{\emph{Bürokratie:} Dies gilt auch für Karten, die von Nicht-Vorrats-Stapeln genommen werden (wie z. B. \emph{KOSTBARKEITEN}, aus \emph{Plünderer}). Karten mit \hex[] oder anderen Symbolen wie \potion in den Kosten kosten nicht \coin[0], du erhältst deshalb durch sie ein \emph{KUPFER}.
	
	\medskip
	
	\emph{Entwicklung:} Dies ist nicht optional. Alle genommenen Karten legst du auf deinen Nachziehstapel. Dies gilt auch für Karten, die du nimmst, wenn du nicht am Zug bist.
	
	\medskip
	
	\emph{Erleuchtung:} Geldkarten zählen in allen Belangen als Aktionskarten. Wenn du z.B. \emph{REISMAKLER} benutzt, um ein \emph{KUPFER} zu entsorgen, ist es eine Aktionskarte und weiterhin eine Geldkarte, also ziehst du insgesamt 7 Karten. Geldkarten können in der Kaufphase immer noch normal gespielt werden, aber wenn sie in der Aktionsphase gespielt werden, produzieren sie +1 Karte und +1 Aktion anstelle dessen, was sie normalerweise tun. Du kannst diese Geldkarten auf die Seite drehen, um dich daran zu erinnern, dass sie kein \coin[] eingebracht haben. Da Geldkarten Aktionskarten sind, können sie mit Wegen (aus \emph{Menagerie}) benutzt werden, um etwas anderes zu erhalten als +1 Karte und +1 Aktion. \emph{WEGELAGERER} (aus \emph{Verbündete}) kann nicht verhindern, dass deine erste Geldkarte für +1 Karte und +1 Aktion in einer Aktionsphase benutzt wird.}
\end{tikzpicture}
\hspace{-0.6cm}
\begin{tikzpicture}
	\card
	\cardstrip
	\cardbanner{banner/blue.png}
	\cardtitle{\scriptsize{Prophezeiungen (2/7)\quad\quad}}
	\cardcontent{\emph{Florierender Handel:} Die Senkung der Kosten gilt für alle Karten überall, auch für die im Vorrat, Handkarten und Karten in den Nachziehstapeln. Sie wirkt kumulativ mit anderen Dingen, die Kosten senken, wie \emph{BRÜCKE} (aus \emph{Intrige}). Wenn du in deiner Kaufphase noch Aktionen (das Spielrecht, nicht den Kartentyp) übrighast, darfst du sie stattdessen als Käufe verwenden. Wenn du z.B. überhaupt keine Aktionskarte gespielt hast, bleibt dir eine Aktion (das Spielrecht), die du nicht genutzt hast, und diese kannst du als zusätzlichen Kauf nutzen. Wichtig sind hierfür zur Verfügung stehende Aktionen (ungenutzte Spielrechte), nicht Aktionskarten; du erhältst normalerweise ein Aktions-Spielrecht pro Zug, die Anzahl kannst du mit Karten wie \emph{RUSTIKALES DORF} erhöhen.}
\end{tikzpicture}
\hspace{-0.6cm}
\begin{tikzpicture}
	\card
	\cardstrip
	\cardbanner{banner/blue.png}
	\cardtitle{\scriptsize{Prophezeiungen (3/7)\quad\quad}}
	\cardcontent{\emph{Göttlicher Wind:} Die 10 Königreich-Vorratsstapel, die ihr in diesem Spiel benutzt, werden entfernt, genau wie ein elfter Stapel, falls einer hinzugefügt worden ist (wie z.B. der Bannstapel der \emph{JUNGEN HEXE} aus \emph{Reiche Ernte}). \emph{RUINEN} (aus \emph{Dark Ages}), Tränke (aus \emph{Alchemisten}) sowie \emph{PLATIN} und \emph{KOLONIE} (aus \emph{Blütezeit}) werden nicht entfernt. Teilt 10 neue Königreichkarten aus. Führt die für diese neuen Karten erforderliche Spielvorbereitung aus, wie z.B. das Auslegen von Tränken, falls nötig. Teilt keine Erbstücke (aus \emph{Nocturne}) aus. Entscheidet nicht erneut, ob ihr Unterschlüpfe (aus \emph{Dark Ages}) verwendet oder \emph{PLATIN} und \emph{KOLONIE}. Teilt einen Verbündeten (aus \emph{Verbündete}) aus, falls ihr mit einer Kontaktkarte spielt und noch keinen habt. Die entfernten Stapel spielen nicht mehr mit. Sie zählen nicht mehr als leere Stapel, wenn sie leer sind, und es können keine Karten dorthin zurückgelegt werden. Ihr könnt aber mit den Karten von diesen Stapeln weiterspielen. Marker (wie z.B. die Marker vom \emph{LEHRER}, aus \emph{Abenteuer}) auf den entfernten Stapeln werden ebenfalls entfernt. Merkmale (aus \emph{Plünderer}) und \emph{OBELISK} (aus \emph{Empires}) beeinflussen immer noch ihren entfernten Stapel und der Bann (der \emph{JUNGEN HEXE}, aus \emph{Reiche Ernte}) ist immer noch der Bann. \emph{SUCHE} (aus \emph{Plünderer}) wird nicht ausgelöst, wenn Stapel entfernt werden. \enquote{In-Spielen-mit-dieser-Karte}-Fähigkeiten, wie z.B. bei der \emph{SCHAMANIN} (aus \emph{Plünderer}) gelten auch für die entfernten Stapel weiter.}
\end{tikzpicture}
\hspace{-0.6cm}
\begin{tikzpicture}
	\card
	\cardstrip
	\cardbanner{banner/blue.png}
	\cardtitle{\scriptsize{Prophezeiungen (4/7)\quad\quad}}
	\cardcontent{\emph{Großer Anführer:} Da jede Aktionskarte, die du spielst, dir mindestens +1 Aktion bringt, wirst du immer alle deine Aktionskarten spielen können, allerdings nicht bei Ausnahmen wie dem \emph{VERSCHNEITEN DORF} (aus \emph{Menagerie}).
	
	\medskip
	
	\emph{Gütiger Kaiser:} Du nimmst eine beliebige Aktionskarte aus dem Vorrat auf deine Hand unabhängig von den Kosten. Dies ist nicht optional. Wenn die letzte Sonne entfernt wurde, gilt dies sofort, d.h. mitten in der Ausführung der Omenkarte.
	
	\medskip

	\emph{Gute Ernte:} Wenn du z.B. 4 \emph{KUPFER} und ein \emph{SILBER} spielst, würdest du insgesamt +2 Käufe und +\coin[2] von der \emph{GUTEN ERNTE} erhalten. Wenn du im gleichen Zug eine Geldkarte gespielt hast, bevor du die letzte Sonne von der \emph{GUTEN ERNTE} entfernst, gibt dir dies nicht rückwirkend +1 Kauf und +\coin[1].}
\end{tikzpicture}
\hspace{-0.6cm}
\begin{tikzpicture}
	\card
	\cardstrip
	\cardbanner{banner/blue.png}
	\cardtitle{\scriptsize{Prophezeiungen (5/7)\quad\quad}}
	\cardcontent{\emph{Herannahende Armee:} Ihr fügt in der Spielvorbereitung einen zusätzlichen Angriffskartenstapel hinzu, auch wenn unter den normalen 10 Königreichkartenstapeln bereits eine Angriffskarte ist. Für Geteilte Stapel (aus \emph{Verbündete} und \emph{Empires}) gilt: Ein Stapel ist ein Angriffs-Stapel, wenn die Platzhalterkarte für diesen Stapel eine Angriffskarte ist. Erfordert der hinzugefügte Angriffs-Stapel eigene Vorbereitungen, setzt auch diese um. Der hinzugefügte Stapel ist ein normaler Königreichkarten-Stapel, von ihm könnt ihr Karten nehmen wie von anderen Stapeln. Die Spielvorbereitung macht ihr vor Beginn des Spiels, darum beeinflusst sie das Spiel, auch wenn die Prophezeiung niemals eintrifft. Sobald die Prophezeiung eingetroffen ist, erhältst du +\coin[1] für jede Angriffskarte, die du spielst. Für Dauer-Angriffskarten gilt dies nur in dem Zug, in dem die Dauer-Angriffskarte gespielt wurde. \\
	+\coin[1] ist das letzte, was die Angriffskarte auslöst, wenn sie gespielt wurde. Du erhältst +\coin[1], auch wenn du die Anweisungen auf der Angriffskarte nicht befolgt hast, wenn du z.B. einen Weg (aus \emph{Menagerie}) benutzt hast.
		
	\medskip
	
	\emph{Krankheit:} Du kannst dich dafür entscheiden, einen \emph{FLUCH} zu nehmen, auch wenn der \emph{FLUCH}-Stapel leer ist, oder Karten abzulegen, auch wenn du weniger als 3 Handkarten hast (in dem Fall legst du alle ab, die du hast).}
\end{tikzpicture}
\hspace{-0.6cm}
\begin{tikzpicture}
	\card
	\cardstrip
	\cardbanner{banner/blue.png}
	\cardtitle{\scriptsize{Prophezeiungen (6/7)\quad\quad}}
	\cardcontent{\emph{Panik:} Dies macht Geldkarten zu einmalig benutzbaren Karten. Durch diese Prophezeiung können Geldkarten auf Nicht-Vorratsstapel zurückgelegt werden, aber keine Karten ohne Stapel, wie z.B. Erbstücke (aus \emph{Nocturne}), \emph{KOSTBARKEITEN} (aus \emph{Plünderer}) werden oben auf den Stapel zurückgelegt.
	
	\medskip
	
	\emph{Schnelle Expansion:} Dies ist nicht optional. Du kannst das Spielen jeder dieser Karten und das Abarbeiten anderer Effekte zu Beginn deines nächsten Zuges in beliebiger Reihenfolge durchführen.
	
	\medskip
	
	\emph{Strenger Winter:} Dies gilt für Vorratsstapel und Nicht-Vorratsstapel. Karten, die aus dem Müll genommen wurden, beeinflussen den Stapel, von dem die Karte stammt, sofern es einen gibt. \emph{KUPFER} und \emph{ANWESEN} sind \enquote{von} ihren Stapeln, auch wenn dies Karten sind, mit denen du in das Spiel gestartet bist. Wenn du nicht am Zug bist, bringt dein Nehmen einer Karte weder \hex[] auf dessen Stapel noch entfernt es \hex[] von dessen Stapel.}
\end{tikzpicture}
\hspace{-0.6cm}
\begin{tikzpicture}
	\card
	\cardstrip
	\cardbanner{banner/blue.png}
	\cardtitle{\scriptsize{Prophezeiungen (7/7)\quad\quad}}
	\cardcontent{\emph{Wachstum:} Dies kann zu Kettenreaktionen führen. Wenn du z.B. \emph{REIS} nimmst und dadurch \emph{WACHSTUM} auslöst, könntest du z.B. \emph{GOLD} nehmen, das wiederum \emph{WACHSTUM} auslöst; wenn du nun ein \emph{SILBER} nimmst, kannst du die Kette mit \emph{ANWESEN} beenden. Dies ist nicht optional. Wenn du eine Geldkarte nimmst, musst du eine Karte mit geringeren Kosten nehmen, wenn du kannst.
	
	\medskip
	
	\emph{Wartezeit:} Statt in der Aufräumphase ungespielte Karten abzulegen, legst du sie zur Seite und fügst sie zu Beginn deines nächsten Zuges zu deinen Handkarten hinzu.}
\end{tikzpicture}
\hspace{-0.6cm}
\begin{tikzpicture}
	\card
	\cardstrip
	\cardbanner{banner/white.png}
	\cardtitle{Ereignisse (1/2)\quad}
	\cardcontent{\emph{Fortsetzen:} Du kannst dieses Ereignis nur einmal pro Zug nutzen. Wenn du es kaufst, nimmst du eine Aktionskarte, die bis zu \coin[4] kostet und keine Angriffskarte ist. Du kehrst in deine Aktionsphase zurück und spielst die genommene Aktionskarte. Dafür brauchst du keine deiner Aktionen deines Zuges. Du erhältst auch +1 Aktion und +1 Kauf. Dass du in deine Aktionsphase zurückkehrst, führt nicht zu einer Wiederholung der \enquote{Zu Beginn des Zuges}-Fähigkeiten. Wenn aber deine Kaufphase danach erneut stattfindet, werden \enquote{Zu Beginn deiner Kaufphase}- Fähigkeiten erneut ausgeführt.
	
	\medskip
	
	\emph{Anhäufen:} Eine Dauerkarte im Spiel, die du in einem früheren Zug gespielt hast, verhindert, dass du durch ANHÄUFEN eine Aktionskarte nehmen kannst. Karten, die du in diesem Zug gespielt hast, die aber nicht mehr im Spiel sind, wie z.B. \emph{PFERD} (aus \emph{Menagerie}), verhindern das nicht.
	
	\medskip
	
	\emph{Askese:} Du könntest z.B. zusätzliche \coin[3] zahlen - also \coin[5] insgesamt - und 3 Handkarten entsorgen.
	
	\medskip
	
	\emph{Kredit:} Du kannst hierdurch keine Karten mit \hex[] in ihren Kosten nehmen.
	
	\medskip
	
	\emph{Voraussicht:} Die Karte wird deiner Hand hinzugefügt, nachdem du die nächste Hand gezogen hast.}
\end{tikzpicture}
\hspace{-0.6cm}
\begin{tikzpicture}
	\card
	\cardstrip
	\cardbanner{banner/white.png}
	\cardtitle{Ereignisse (2/2)\quad}
	\cardcontent{\tiny{\emph{Kampfübung:} Wenn du mit diesem Ereignis eine Dauerkarte spielst, könnte es eine gute Idee sein, sie schräg zu legen. So kannst du dir merken, dass du sie zweimal gespielt hast.
	
	\medskip
	
	\emph{Kintsugi:} Du musst dir merken, ob du in diesem Spiel schon ein \emph{GOLD} genommen hast. Sobald du ein \emph{GOLD} genommen hast, nimmst du durch \emph{KINTSUGI} eine Karte, auch wenn du das \emph{GOLD} nicht mehr hast.
	
	\medskip
	
	\emph{Seehandel:} Zähle zunächst, wie viele Aktionskarten du im Spiel hast. Ziehe so viele Karten und entsorge dann Handkarten bis zu dieser Anzahl. Das Ziehen der Karten ist nicht optional, aber das Entsorgen ist optional. Wenn du keine Aktionskarten im Spiel hast, ziehst du keine Karten und kannst dann keine entsorgen. Wenn du durch dieses Ereignis Karten ziehst, ist es in dieser Kaufphase zu spät, um weitere Geldkarten zu spielen.
	
	\medskip
	
	\emph{Tribut empfangen:} Die Aktionskarten, die du nimmst, müssen alle unterschiedliche Namen haben und müssen andere Karten sein als die, die du im Spiel hast. Du nimmst sie einzeln nacheinander, in beliebiger Reihenfolge. Du musst nicht die volle Anzahl von drei Karten nehmen.

	\medskip
	
	\emph{Sammeln:} Du nimmst 3 Karten in der angegebenen Reihenfolge, keine davon ist optional. Wenn du eine davon nicht nehmen kannst - z.B. weil keine Karte im Vorrat genau \coin[4] kostet - nimmst du trotzdem die anderen.}}
\end{tikzpicture}
\hspace{-0.6cm}
\begin{tikzpicture}
	\card
	\cardstrip
	\cardbanner{banner/white.png}
	\cardtitle{\scriptsize{Spielvorbereitung (1/2)}\qquad}
	\cardcontent{Zum Spielen mit \emph{DOMINION Rising Sun} benötigt ihr ein \emph{Basisspiel} oder das \emph{Basiskarten-Set}. Legt alle Basiskarten (\emph{KUPFER}, \emph{SILBER}, \emph{GOLD} (+ggf. \emph{PLATIN}), \emph{ANWESEN}, \emph{HERZOGTUM}, \emph{PROVINZ} (+ggf. \emph{KOLONIE}) sowie die \emph{FLÜCHE} und die Müllkarte (bzw. das Mülltableau aus der \emph{DOMINION 2. Edition}) wie gewohnt als Teil des Vorrats in die Tischmitte.}
\end{tikzpicture}
\hspace{-0.6cm}
\begin{tikzpicture}
	\card
	\cardstrip
	\cardbanner{banner/white.png}
	\cardtitle{\scriptsize{Spielvorbereitung (2/2)}\qquad}
	\cardcontent{\tiny{\emph{Ereignisse \& Prophezeiungen}\\
	Zusätzlich zu den Königreichkarten gibt es \emph{Ereignisse} (siehe \enquote{Neue Regeln} \rightarrow \emph{Ereignisse}) und \emph{Prophezeiungen} (siehe \enquote{Neue Regeln} \rightarrow \emph{Prophezeiungen}). \emph{Ereignisse} haben exakt die gleiche Funktion wie z.B. in \emph{Abenteuer} und \emph{Plünderer}. \emph{Prophezeiungen sind Regeln, die ggf. zur Anwendung kommen, sobald alle Sonnenmarker von ihnen entfernt wurden.}
	\\
	In Spielen mit einer oder mehreren \emph{Omenkarten} legt eine \emph{Prophezeiung} aus. Verwendet nur eine einzige Prophezeiung, egal wie viele Omenkarten ihr verwendet. Legt \emph{5/8/10/12/13 Sonnenmarker} auf die Prophezeiung, wenn ihr \emph{2/3/4/5/6 Personen} seid.
	\\
	Ist die Prophezeiung \emph{Herannahende Armee}, legt einen zusätzlichen Angriffs-Königreichkartenstapel in den Vorrat (auch wenn dort schon einer liegt). Siehe \enquote{Neue Regeln} \rightarrow \emph{Prophezeiungen}.
	\\
	In Spielen mit \emph{Flussboot} wählt bei der Spielvorbereitung eine Aktionskarte, die keine Dauerkarte ist, \coin[5] kostet und in diesem Spiel nicht genutzt wird, und legt ein Exemplar davon zur Seite.
	\\
	Zieht \emph{Ereignisse} zufällig aus einem Stapel (dieser kann auch \emph{Landmarken} (aus \emph{Empires}), \emph{Ereignisse} aus anderen Erweiterungen, \emph{Projekte} (aus \emph{Renaissance}), \emph{Wege} (aus \emph{Menagerie}) und / oder \emph{Merkmale} (aus \emph{Plünderer}) enthalten) oder mischt sie (trotz ihrer unterschiedlichen Rückseite) in die Platzhalterkarten ein. Deckt ihr eine dieser Karten auf, legt ihr sie neben dem Vorrat bereit. Diese Karten gehören nicht zum Vorrat. Wir empfehlen, pro Spiel maximal 2 dieser Karten zu verwenden (darunter maximal 1 Weg).}}
\end{tikzpicture}
\hspace{-0.6cm}
\begin{tikzpicture}
	\card
	\cardstrip
	\cardbanner{banner/white.png}
	\cardtitle{\scriptsize{Neue Regeln (1/6)}\qquad}
	\cardcontent{\tiny{Es gelten die Basisspielregeln mit folgenden Erweiterungen:

	\medskip
	
	\emph{Schulden}\\
	In \emph{Rising Sun} gibt es \emph{Schulden}, die erstmals in \emph{Empires} erschienen sind. Die \emph{Schulden} kennzeichnet ihr mit den Schuldenmarkern, das Symbol \hex[] zeigt die Höhe der \emph{Schulden}.
	\begin{itemize}
		\item Hast du Schuldenmarker, darfst du keine Karten, Ereignisse oder Projekte (aus \emph{Renaissance}) kaufen; Schuldenmarker haben keinen weiteren Effekt, z.B. bewirken sie nichts bei Spielende.
		\item Kaufst du eine Karte oder ein Ereignis mit \hex[] in den Kosten, erhältst du entsprechend viele Schuldenmarker.
		\item Durch einen Effekt mit +\hex[] nimmst du entsprechend viele Schuldenmarker. Z.B. bedeutet +\hex[2], dass du 2 Schuldenmarker nimmst.
		\item Du kannst jederzeit in deinem Zug Schuldenmarker entfernen, indem du \coin[1] pro Schuldenmarker zahlst. Dies verbraucht keine Aktion und keinen Kauf und kann mehrfach im eigenen Zug gemacht werden. Hierfür darfst du nicht außerhalb der Reihe Geldkarten spielen.
		\item \hex[]-Beträge sind nicht \coin[]-Beträge. Rechnungen mit \coin[]-Beträgen beeinflussen keine \hex[]-Beträge.
		\item Einige Karten betreffen Kosten in einem Bereich. \enquote{Bis zu \coin[4]} heißt z.B. \enquote{\coin[0], \coin[1], \coin[2], \coin[3] oder \coin[4]}. Dies schließt keine Kosten mit \hex[] darin ein.
	\end{itemize}}}
\end{tikzpicture}
\hspace{-0.6cm}
\begin{tikzpicture}
	\card
	\cardstrip
	\cardbanner{banner/white.png}
	\cardtitle{\scriptsize{Neue Regeln (2/6)}\qquad}
	\cardcontent{\tiny{\emph{Schulden (Fortsetzung)}\\
	\begin{itemize}
		\item Einige Karten vergleichen Kosten. Eine Karte, die \hex[8] kostet, kostet mehr als eine Karte, die \hex[6] kostet, genauso wie eine Karte mit Kosten \coin[8] mehr kostet als eine Karte mit Kosten \coin[6]. Schulden und \coin[] sind aber nicht vergleichbar. Bei einer Karte mit den Kosten \coin[4] und einer Karte mit den Kosten \hex[6] kostet keine von beiden mehr als die andere. \hex[6] kostet aber mehr als \coin[0]; in allen einfachen \hex[]-Kosten ist eine implizite \coin[0] enthalten; \hex[6] kosten deshalb gleich viel \coin[] wie \coin[0] und mehr \hex[].
		\item Du darfst nicht ohne Veranlassung Schuldenmarker nehmen.
		\item Du darfst nicht mit \hex[ ]überzahlen (für \emph{Die Gilden}).
		\item \hex[ ]ist nicht auf die enthaltenen Schuldenmarker beschränkt. Verwendet Ersatzmaterial, falls die Marker ausgehen.
	\end{itemize}

	\medskip

	\emph{Schattenkarten}\\
	In \emph{Rising Sun} gibt es 5 \emph{Schattenkarten}. Sie haben alle unterschiedliche Rückseiten, und du kannst sie aus deinem Nachziehstapel spielen.
	\begin{itemize}
		\item Wenn du \emph{Schattenkarten} mischst, lege sie unter deinen Nachziehstapel. Hast du mehrere \emph{Schattenkarten}, können sie in beliebiger Reihenfolge unter deinem Nachziehstapel liegen. Sie können auch mit anderen Karten gemischt sein, die du explizit unter deinen Nachziehstapel gelegt hast, wie z.B. \emph{Vorherbestimmte} Karten (aus \emph{Plünderer}).
	\end{itemize}}}
\end{tikzpicture}
\hspace{-0.6cm}
\begin{tikzpicture}
	\card
	\cardstrip
	\cardbanner{banner/white.png}
	\cardtitle{\scriptsize{Neue Regeln (3/6)}\qquad}
	\cardcontent{\tiny{\emph{Schattenkarten (Fortsetzung)}\\
	\begin{itemize}
		\item Wenn du möchtest, kannst du deine \emph{Schattenkarten} unter deinem Nachziehstapel seitlich drehen, damit du dich leicht daran erinnerst, dass sie dort sind.
		\item \emph{Schattenkarten} bleiben nicht notwendigerweise unten in deinem Nachziehstapel; sie werden nur dort platziert, nachdem du sie gemischt hast.
		\item \emph{Schattenkarten} werden nicht unter den Nachziehstapel gelegt, wenn du sie ziehst oder zu sonst irgendeinem anderen Zeitpunkt, als wenn du sie gemischt hast.
		\item Du kannst die Rückseiten der Karten in deinem Nachziehstapel jederzeit durchschauen und damit sehen, wo die \emph{Schattenkarten} sind.
		\item Immer wenn du normalerweise eine Aktionskarte spielen kannst, kannst du eine \emph{Schattenkarte} aus deinem Nachziehstapel spielen. Es ist nicht wichtig, wo in deinem Nachziehstapel sie liegt. Du spielst sie genau so, als ob du sie aus deiner Hand spielen würdest; sie kommt ins Spiel und du folgst ihren Anweisungen.
		\item Wenn eine Karte wie \emph{THRONSAAL} dich anweist, eine Karte aus deiner Hand zu spielen, kannst du diese Gelegenheit nutzen und eine \emph{Schattenkarte} aus deinem Nachziehstapel spielen.
		\item Du kannst \emph{Schattenkarten} aus deinem Nachziehstapel spielen, als wären sie auf deiner Hand. Sie sind aber nicht auf deiner Hand. Du kannst sie nicht für andere Zwecke, die Handkarten nutzen, aus deinem Nachziehstapel nehmen. Z.B. kannst du sie nicht durch die Anweisung einer \emph{GASSE} ablegen (außer sie ist tatsächlich auf deiner Hand).
	\end{itemize}}}
\end{tikzpicture}
\hspace{-0.6cm}
\begin{tikzpicture}
	\card
	\cardstrip
	\cardbanner{banner/white.png}
	\cardtitle{\scriptsize{Neue Regeln (4/6)}\qquad}
	\cardcontent{\tiny{\emph{Omenkarten und Prophezeiungen}\\
	In \emph{Rising Sun} gibt es \emph{Omenkarten} und \emph{Prophezeiungen}. \emph{Prophezeiungen} sind Regeln, die ggf. im Spiel zur Anwendung kommen. Mit \emph{Omenkarten} kannst du die Zeit herunterzählen, bis eine \emph{Prophezeiung} wirksam wird.
	\begin{itemize}
		\item In jedem Spiel mit einer oder mehreren \emph{Omenkarten} teilt eine \emph{Prophezeiung} für sie aus. Verwendet nur eine einzige \emph{Prophezeiung}, egal wie viele \emph{Omenkarten} ihr verwendet.
		\item Legt auf die \emph{Prophezeiung} 5 Sonnenmarker bei 2 Personen, 8 Sonnenmarker bei 3 Personen, 10 bei 4 Personen, 12 bei 5 Personen und 13 bei 6 Personen.
		\item \enquote{+1 \suntoken} bedeutet, dass du einen Sonnenmarker von der \emph{Prophezeiung} entfernst. Wenn dies der letzte Marker war, kommt der Regeltext auf der \emph{Prophezeiung} sofort und für den Rest des Spiels zur Anwendung.
		\item \enquote{+1 \suntoken} steht immer als Erstes auf \emph{Omenkarten}, vor allem anderen, was die Karte macht.
		\item \enquote{+1 \suntoken} hat keine weiteren Auswirkungen, nachdem alle Marker entfernt sind.
		\item Der Text auf \emph{Prophezeiungen} ist wirkungslos, bis der letzte Sonnenmarker von ihr entfernt wurde.
	\end{itemize}}}
\end{tikzpicture}
\hspace{-0.6cm}
\begin{tikzpicture}
	\card
	\cardstrip
	\cardbanner{banner/white.png}
	\cardtitle{\scriptsize{Neue Regeln (5/6)}\qquad}
	\cardcontent{\tiny{\emph{Ereignisse}\\
	In \emph{Rising Sun} gibt es \emph{Ereignisse}, die erstmals in \emph{Abenteuer} erschienen sind. In deiner Kaufphase kannst du ein \emph{Ereignis} statt einer anderen Karte erwerben (dies verbraucht 1 Kauf). Du bezahlst die Kosten, die auf dem \emph{Ereignis} stehen, und dessen Effekt tritt sofort ein.
	\begin{itemize}
		\item \emph{Ereignisse} sind keine Königreichkarten. Sie liegen lediglich aus und liefern einen Effekt, den du kaufen kannst. Es gibt keine Möglichkeit, dass du ein Ereignis nehmen kannst oder dass ein Ereignis in deinem Kartensatz ist.
		\item Der Erwerb eines \emph{Ereignisses} verbraucht 1 Kauf. Normalerweise kannst du entweder eine Karte kaufen oder ein \emph{Ereignis} erwerben. Wenn du 2 Käufe hast, wie z.B. nach dem Spielen der \emph{FISCHHÄNDLERIN}, kannst du zwei Karten kaufen oder zwei \emph{Ereignisse} erwerben oder eine Karte und ein \emph{Ereignis}, in beliebiger Reihenfolge.
		\item \emph{Ereignisse} können in einem Zug mehrmals erworben werden, wenn du genügend Käufe und \coin[] dafür verfügbar hast.
		\item Nachdem du ein \emph{Ereignis} erworben hast, darfst du in dieser Kaufphase keine weiteren Geldkarten spielen, es sei denn, ein \emph{Ereignis} oder eine Karte erlaubt dir dies explizit.
		\item Der Erwerb eines \emph{Ereignisses} ist kein Kauf einer Karte und löst deshalb nicht Karten wie den \emph{FEILSCHER} (aus \emph{Hinterland}) aus.
		\item Die Kosten von \emph{Ereignissen} werden nicht durch Karten wie \emph{Florierender Handel} beeinflusst.
	\end{itemize}}}
\end{tikzpicture}
\hspace{-0.6cm}
\begin{tikzpicture}
	\card
	\cardstrip
	\cardbanner{banner/white.png}
	\cardtitle{\scriptsize{Neue Regeln (6/6)}\qquad}
	\cardcontent{\emph{Dauerkarten}\\
	In \emph{Rising Sun} gibt es einige Dauerkarten (wie schon in \emph{Seaside} und den Erweiterungen ab \emph{Abenteuer}). Die orangefarbenen Dauerkarten beinhalten Anweisungen, die in späteren Zügen umgesetzt werden. Du legst sie nicht in der Aufräumphase des Zuges ab, in dem du sie gespielt hast, sondern sie bleiben bis zur Aufräumphase des Zuges, in dem ihre letzte Anweisung ausgeführt wird, im Spiel. Wird eine Dauerkarte mehrfach gespielt (z.B. durch \emph{DAIMYO}), bleibt die verursachende Karte ebenfalls so lange im Spiel, bis die Dauerkarte abgelegt wird. So kannst du verfolgen, dass diese Dauerkarte mehrmals gespielt wurde. Um anzuzeigen, dass eine Dauerkarte in der aktuellen Aufräumphase noch nicht abgelegt wird, wird sie in eine eigene Reihe oberhalb der restlichen gespielten Karten gelegt.\\
	Hast du zu Beginn deines Zuges mehrere Dauerkarten im Spiel, deren Anweisungen zu diesem Zeitpunkt ausgeführt werden sollen, darfst du die Reihenfolge selbst bestimmen, in der du sie abhandelst.}
\end{tikzpicture}
\hspace{-0.6cm}
\begin{tikzpicture}
	\card
	\cardstrip
	\cardbanner{banner/white.png}
	\cardtitle{\footnotesize{Neue Anweisungen}\qquad}
	\cardcontent{\enquote{+1 \suntoken} bedeutet, dass du einen Sonnenmarker von der \emph{Prophezeiung} entfernst. Wenn dies der letzte Marker war, kommt der Regeltext auf der \emph{Prophezeiung} sofort und für den Rest des Spiels zur Anwendung.}
\end{tikzpicture}
\hspace{-0.6cm}
\begin{tikzpicture}
	\card
	\cardstrip
	\cardbanner{banner/white.png}
	\cardtitle{\scriptsize{Empfohlene 10er Sätze\qquad}}
	\cardcontent{\emph{Richtung Osten (+ \underline{Prophezeiung}):}\\
	\underline{Entwicklung}, Fischhändlerin, Flussschrein, Gasse, Handwerker, Künstlerin, Reis, Rustikales Dorf, Sänfte, Samurai, Teehaus

	\smallskip

	\emph{Dämmerung einer Ära (+ \underline{Prophezeiung} \& \underline{\underline{Ereignis}}):}\\
	\underline{Gütiger Kaiser}, \underline{\underline{Kampfübung}}, Aristokratin, Bergschrein, Daimyo, Goldmine, Kaiserlicher Gesandter, Kitsune, Ninja, Reismakler, Ronin, Wandel

	\bigskip

	\emph{Sprung vorwärts (+ \textit{Basisspiel} \& \underline{Prophezeiung}):}\\
	\underline{Schnelle Expansion}, Bergschrein, Erdkeller, Flussboot (mit Markt), Goldmine, Tanuki, \textit{Schmiede}, \textit{Thronsaal}, \textit{Töpferei}, \textit{Vorbotin}, \textit{Werkstatt}

	\smallskip

	\emph{Geld zum Verbrennen (+ \textit{Basisspiel} \& \underline{Prophezeiung} \& \underline{\underline{Ereignis}}):}\\
	\underline{Panik}, \underline{\underline{Sammeln}}, Dichterin, Handwerker, Ronin, Schlangenhexe, Wandel, \textit{Bürokrat}, \textit{Händlerin}, \textit{Jahrmarkt}, \textit{Keller}, \textit{Wilddiebin}}
\end{tikzpicture}
\hspace{-0.6cm}
\begin{tikzpicture}
	\card
	\cardstrip
	\cardbanner{banner/white.png}
	\cardtitle{\scriptsize{Empfohlene 10er Sätze\qquad}}
	\cardcontent{\emph{Das Puzzle lösen (+ \textit{Intrige} \& \underline{Prophezeiung}):}\\
	\underline{Erleuchtung}, Bergschrein, Flussboot (mit Anbau), Goldmine, Künstlerin, Ronin, \textit{Eisenhütte}, \textit{Herumtreiberin}, \textit{Höflinge}, \textit{Verschwörer}, \textit{Wunschbrunnen}

	\smallskip

	\emph{Kalte Berechnung (+ \textit{Intrige} \& \underline{Prophezeiung} \& \underline{\underline{Ereignis}}):}\\
	\underline{Strenger Winter}, \underline{\underline{Anhäufen}}, Handwerker, Ninja, Schlangenhexe, Tanuki, Teehaus, \textit{Armenviertel}, \textit{Baron}, \textit{Diplomatin}, \textit{Geheimgang}, \textit{Herzog}

	\bigskip

	\emph{Invasionsflotte (+ \textit{Seaside} \& \underline{Prophezeiung}):}\\
	\underline{Herannahende Armee}, Flussboot (mit Basar), Gasse, Kitsune, Ninja, Reismakler, \textit{Außenposten}, \textit{Blockade}, \textit{Korsarenschiff}, \textit{Müllverwerter}, \textit{Schatzkarte}, \textit{Seekarte}

	\smallskip

	\emph{Inselbewohner (+ \textit{Seaside} \& \underline{Prophezeiung} \& \underline{\underline{Ereignis}}):}\\
	\underline{Großer Anführer}, \underline{\underline{Kintsugi}}, Flussschrein, Handwerker, Kaiserlicher Gesandter, Reis, Schlangenhexe, \textit{Gezeitenbecken}, \textit{Hafen}, \textit{Karawane}, \textit{Meerhexe}, \textit{Piratin}}
\end{tikzpicture}
\hspace{-0.6cm}
\begin{tikzpicture}
	\card
	\cardstrip
	\cardbanner{banner/white.png}
	\cardtitle{\scriptsize{Empfohlene 10er Sätze\qquad}}
	\cardcontent{\emph{Überholspur (+ \textit{Alchemist} \& \underline{Prophezeiung}):}\\
	\underline{Entwicklung}, Erdkeller, Fischhändlerin, Flussboot (mit Lehrling), Kaiserlicher Gesandter, Rustikales Dorf, Samurai, \textit{Alchemist}, \textit{Golem}, \textit{Universität}, \textit{Weinberg}

	\smallskip

	\emph{Faules Verderben (+ \textit{Alchemist} \& \underline{Prophezeiung} \& \underline{\underline{Ereignis}}):}\\
	\underline{Wartezeit}, \underline{\underline{Tribut empfangen}}, Aristokratin, Gasse, Kitsune, Ninja, Sänfte, Wandel, \textit{Apotheker}, \textit{Kräuterkundiger}, \textit{Vertrauter}, \textit{Verwandlung}

	\bigskip
	
	\emph{Handel auf dem Fluss (+ \textit{Blütezeit} \& \underline{Prophezeiung}):}\\
	\underline{Florierender Handel}, Erdkeller, Flussboot (mit Stadt), Flussschrein, Handwerker, Sänfte, \textit{Amboss}, \textit{Großer Markt}, \textit{Kristallkugel}, \textit{Sammelsurium}, \textit{Waffenkiste}

	\smallskip

	\emph{Herbsternte (+ \textit{Blütezeit} \& \underline{Prophezeiung} \& \underline{\underline{Ereignis}}):}\\
	\underline{Gute Ernte}, \underline{\underline{Fortsetzen}}, Aristokratin, Kaiserlicher Gesandter, Ninja, Rustikales Dorf, Wandel, \textit{Bank}, \textit{Geldanlage}, \textit{Magnatin}, \textit{Steinbruch}, \textit{Wachturm}}
\end{tikzpicture}
\hspace{-0.6cm}
\begin{tikzpicture}
	\card
	\cardstrip
	\cardbanner{banner/white.png}
	\cardtitle{\scriptsize{Empfohlene 10er Sätze\qquad}}
	\cardcontent{\emph{Wintersonnenwende (+ \textit{Reiche Ernte / Die Gilden} \& \underline{Prophezeiung}):}\\
	\underline{Strenger Winter}, Daimyo, Goldmine, Schlangenhexe, Teehaus, Wandel, \textit{Bäcker}, \textit{Berater}, \textit{Herold}, \textit{Leuchtenmacher}, \textit{Steuereintreiber}

	\smallskip

	\emph{Aus den Schatten (+ \textit{Reiche Ernte / Die Gilden} \& \underline{Prophezeiung} \& \underline{\underline{Ereignis}}):}\\
	\underline{Schnelle Expansion}, \underline{\underline{Sammeln}}, Dichterin, Fischhändlerin, Gasse, Künstlerin, Tanuki, \textit{Arzt}, \textit{Bauerndorf}, \textit{Ernte}, \textit{Füllhorn}, \textit{Menagerie}

	\bigskip

	\emph{Schnelle Hände (+ \textit{Hinterland} \& \underline{Prophezeiung}):}\\
	\underline{Entwicklung}, Gasse, Ronin, Rustikales Dorf, Samurai, Schlangenhexe, \textit{Feilscher}, \textit{Hexenkessel}, \textit{Komplott}, \textit{Oase}, \textit{Weberin}

	\smallskip

	\emph{Papierkram (+ \textit{Hinterland} \& \underline{Prophezeiung} \& \underline{\underline{Ereignis}}):}\\
	\underline{Bürokratie}, \underline{\underline{Voraussicht}}, Flussboot (mit Hexenhütte), Flussschrein, Handwerker, Kaiserlicher Gesandter, Tanuki, \textit{Gewürzhändler}, \textit{Grenzdorf}, \textit{Radmacherin}, \textit{Tunnel}, \textit{Wegkreuzung}}
\end{tikzpicture}
\hspace{-0.6cm}
\begin{tikzpicture}
	\card
	\cardstrip
	\cardbanner{banner/white.png}
	\cardtitle{\scriptsize{Empfohlene 10er Sätze\qquad}}
	\cardcontent{\emph{Pandemie (+ \textit{Dark Ages} \& \underline{Prophezeiung}):}\\
	\underline{Krankheit}, Aristokratin, Bergschrein, Fischhändlerin, Reismakler, Sänfte, \textit{Katakomben}, \textit{Landstreicher}, \textit{Lumpensammler}, \textit{Prozession}, \textit{Schurke}

	\smallskip

	\emph{Entfernte Horden (+ \textit{Dark Ages} \& \underline{Prophezeiung} \& \underline{\underline{Ereignis}}):}\\
	\underline{Herannahende Armee}, \underline{\underline{Askese}}, Dichterin, Reis, Ronin, Samurai, Schlangenhexe, \textit{Eremit}, \textit{Jagdgründe}, \textit{Knappe}, \textit{Mundraub}, \textit{Ritter}, \textit{Waffenkammer}

	\bigskip

	\emph{Streuner (+ \textit{Abenteuer} \& \underline{Prophezeiung} \& \underline{\underline{Ereignis}}):}\\
	\underline{Florierender Handel}, \underline{\underline{\textit{Ball}}}, Dichterin, Kaiserlicher Gesandter, Ronin, Sänfte, Tanuki, \textit{Amulett}, \textit{Gefolgsmann}, \textit{Geizhals}, \textit{Karawanenwächter}, \textit{Kundschafter}

	\smallskip

	\emph{Heldenreise (+ \textit{Abenteuer} \& \underline{Prophezeiung} \& \underline{\underline{Ereignis}}):}\\
	\underline{Wartezeit}, \underline{\underline{Kintsugi}}, Aristokratin, Erdkeller, Kitsune, Künstlerin, Samurai, \textit{Duplikat}, \textit{Ferne Lande}, \textit{Kunsthandwerker}, \textit{Page}, \textit{Verlies}}
\end{tikzpicture}
\hspace{-0.6cm}
\begin{tikzpicture}
	\card
	\cardstrip
	\cardbanner{banner/white.png}
	\cardtitle{\scriptsize{Empfohlene 10er Sätze\qquad}}
	\cardcontent{\emph{Sommerschlösser (+ \textit{Empires} \& \underline{Prophezeiung} \& \underline{\underline{Ereignis}}):}\\
	\underline{Gütiger Kaiser}, \underline{\underline{\textit{Museum}}}, Aristokratin, Flussschrein, Reis, Schlangenhexe, Wandel, \textit{Forum}, \textit{Patrizier / Handelsplatz}, \textit{Schlösser}, \textit{Stadtviertel}, \textit{Vermögen}

	\smallskip

	\emph{Weggefegt (+ \textit{Empires} \& \underline{Prophezeiung} \& \underline{\underline{Ereignis}}):}\\
	\underline{Göttlicher Wind}, \underline{\underline{Seehandel}}, Bergschrein, Erdkeller, Kitsune, Künstlerin, Reismakler, \textit{Krone}, \textit{Lehnsherr}, \textit{Tempel}, \textit{Wagenrennen}, \textit{Zauber}

	\bigskip

	\emph{Erstklassiger Reis (+ \textit{Nocturne} \& \underline{Prophezeiung}):}\\
	\underline{Wachstum}, Daimyo, Gasse, Reis, Samurai, Teehaus, \textit{Getreuer Hund}, \textit{Schäferin}, \textit{Seliges Dorf}, \textit{Teufelswerkstatt}, \textit{Verfluchtes Dorf}

	\smallskip

	\emph{Dunkle Ecken (+ \textit{Nocturne} \& \underline{Prophezeiung} \& \underline{\underline{Ereignis}}):}\\
	\underline{Krankheit}, \underline{\underline{Anhäufen}}, Flussschrein, Goldmine, Künstlerin, Sänfte, Tanuki, \textit{Attentäter}, \textit{Heiliger Hain}, \textit{Konklave}, \textit{Wechselbalg}, \textit{Werwolf}}
\end{tikzpicture}
\hspace{-0.6cm}
\begin{tikzpicture}
	\card
	\cardstrip
	\cardbanner{banner/white.png}
	\cardtitle{\scriptsize{Empfohlene 10er Sätze\qquad}}
	\cardcontent{\emph{Haufenweise Geld (+ \textit{Renaissance} \& \underline{Prophezeiung} \& \underline{\underline{Ereignis}}):}\\
	\underline{Bürokratie}, \underline{\underline{\textit{Gildenhaus}}}, Flussboot (mit Seher), Gasse, Goldmine, Kitsune, Reismakler, \textit{Experiment}, \textit{Gelehrte}, \textit{Goldmünze}, \textit{Schatzmeisterin}, \textit{Versteck}

	\smallskip

	\emph{Frischer Anfang (+ \textit{Renaissance} \& \underline{Prophezeiung} \& \underline{\underline{Ereignis}}):}\\
	\underline{Göttlicher Wind}, \underline{\underline{Tribut empfangen}}, Daimyo, Dichterin, Ninja, Teehaus, Wandel, \textit{Bildhauerin}, \textit{Erfinder}, \textit{Frachtschiff}, \textit{Grenzposten}, \textit{Patron}

	\bigskip

	\emph{Zum Ochsen werden (+ \textit{Menagerie} \& \underline{Prophezeiung} \& \underline{\underline{Ereignis}}):}\\
	\underline{Erleuchtung}, \underline{\underline{\textit{Weg des Ochsen}}}, Aristokratin, Dichterin, Fischhändlerin, Künstlerin, Samurai, \textit{Kamelzug}, \textit{Pferdestall}, \textit{Schlachtross}, \textit{Viehmarkt}, \textit{Wanderin}

	\smallskip

	\emph{Alternativen (+ \textit{Menagerie} \& \underline{Prophezeiung} \& \underline{\underline{Ereignis}}):}\\
	\underline{Panik}, \underline{\underline{Seehandel}}, Daimyo, Gasse, Handwerker, Ronin, Rustikales Dorf, \textit{Hexenzirkel}, \textit{Kopfgeldjägerin}, \textit{Koppel}, \textit{Nachschub}, \textit{Schlitten}}
\end{tikzpicture}
\hspace{-0.6cm}
\begin{tikzpicture}
	\card
	\cardstrip
	\cardbanner{banner/white.png}
	\cardtitle{\scriptsize{Empfohlene 10er Sätze\qquad}}
	\cardcontent{\emph{Profi-Händler (+ \textit{Verbündete} \& \underline{Prophezeiung} \& \underline{\underline{Ereignis}}):}\\
	\underline{Erleuchtung}, \underline{\underline{\textit{Handwerkergilde}}}, Daimyo, Fischhändlerin, Flussboot (mit Barbar), Reismakler, Rustikales Dorf, \textit{Augurinnen}, \textit{Botin}, \textit{Jägerin}, \textit{Tausch}, \textit{Vertrag}

	\smallskip

	\emph{Fieberhaftes Basteln (+ \textit{Verbündete} \& \underline{Prophezeiung} \& \underline{\underline{Ereignis}}):}\\
	\underline{Schnelle Expansion}, \underline{\underline{Kredit}}, Aristokratin, Fischhändlerin, Handwerker, Schlangenhexe, Teehaus, \textit{Aufwiegler}, \textit{Hauptstadt}, \textit{Konflikte}, \textit{Marquis}, \textit{Wirtin}

	\bigskip

	\emph{Von Beute begraben (+ \textit{Plünderer} \& \underline{Prophezeiung} \& \underline{\underline{Ereignis}}):}\\
	\underline{Wachstum}, \underline{\underline{\textit{Aufblühen}}}, Dichterin, Goldmine, Ninja, Reismakler, Tanuki, \textit{Erste Maatin}, \textit{Flaggschiff}, \textit{Königstruhe}, \textit{Anhänger}, \textit{Vorarbeiter}

	\smallskip

	\emph{Glänzende Dinge (+ \textit{Plünderer} \& \underline{Prophezeiung} \& \underline{\underline{Ereignis}}):}\\
	\underline{Gute Ernte}, \underline{\underline{Kredit}}, Daimyo, Erdkeller, Flussschrein, Reis, Sänfte, \textit{Juwelen-Ei}, \textit{Meuchlerin}, \textit{Pilger}, \textit{Schiffsjunge}, \textit{Werkzeug}}
\end{tikzpicture}
\hspace{-0.6cm}
\begin{tikzpicture}
	\card
	\cardstrip
	\cardbanner{banner/white.png}
	\cardtitle{Platzhalter\quad}
\end{tikzpicture}
\hspace{0.6cm}
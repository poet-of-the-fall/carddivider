% Basic settings for this card set
\renewcommand{\cardcolor}{basicgame}
\renewcommand{\cardextension}{Edition II}
\renewcommand{\cardextensiontitle}{Das Basisspiel}

\clearpage
\newpage
\section{\cardextension \ - \cardextensiontitle}

\begin{tikzpicture}
	\card
	\cardstrip
	\cardbanner{banner/blue.png}
	\cardicon{banner/coin.png}
	\cardprice{2}
	\cardtitle{Burggraben}
	\cardcontent{Spielt ein anderer Spieler eine Angriffskarte (mit der Aufschrift AKTION -- ANGRIFF), kannst du die Karte \emph{BURGGRABEN} vorzeigen, falls du sie in diesem Moment auf der Hand hast. In diesem Fall bist \emph{du} von der Auswirkung des Angriffst nicht betroffen, d.h. du musst mit der \emph{HEXE} keine Fluchkarte nehmen usw. Haben mehrer Spieler einen \emph{BURGGRABEN} auf der Hand, dürfen diese auch eingesetzt und vorgezeigt werden. Danach nehmen die Spieler ihre Karte zurück auf die Hand.
	\\
	\medskip
	\\
	Der Spieler, der den Angriff gespielt hat, darf unabhängig davon, ob ein oder mehrere \emph{BURGGRÄBEN} gespielt wurden, die weiteren Anweisungen seiner Aktionskarte ausführen. Der \emph{BURGGRABEN} darf auch in der eigenen Aktionsphase gespielt werden -- dann ziehst du 2 Karten nach.}
\end{tikzpicture}
\hspace{-1cm}
\begin{tikzpicture}
	\card
	\cardstrip
	\cardbanner{banner/white.png}
	\cardicon{banner/coin.png}
	\cardprice{2}
	\cardtitle{Kapelle}
	\cardcontent{Die ausgespielte \emph{KAPELLE} selber darf nicht entsorgt werden, da sie sich nicht mehr auf der Hand befindet. Weitere \emph{KAPELLEN} auf der Hand dürfen entsorgt werden.}
\end{tikzpicture}
\hspace{-1cm}
\begin{tikzpicture}
	\card
	\cardstrip
	\cardbanner{banner/white.png}
	\cardicon{banner/coin.png}
	\cardprice{2}
	\cardtitle{Keller}
	\cardcontent{Der ausgespielte \emph{KELLER} selber darf nicht abgelegt werden, da er sich nicht mehr in deiner Hand befindet. Sage an wie viele Karten du ablegst und lege diese auf deinen Ablagestapel. Danach ziehst du die gleiche Anzahl Karten vom Nachziehstapel. Sollte während dieses Vorgangs der Nachziehstapel aufgebraucht werden, wird dein Ablagestapel zusammen mit den soeben abgelegten Karten gemischt und als neuer Nachziehstapel bereitgelegt.}
\end{tikzpicture}
\hspace{-1cm}
\begin{tikzpicture}
	\card
	\cardstrip
	\cardbanner{banner/white.png}
	\cardicon{banner/coin.png}
	\cardprice{3}
	\cardtitle{Dorf}
	\cardcontent{Spielst du mehrere \emph{DÖRFER} hintereinander, zählst du am besten laut mit, wie viele Aktionen du noch spielen darfst, damit du den Überblick behältst.}
\end{tikzpicture}
\hspace{-1cm}
\begin{tikzpicture}
	\card
	\cardstrip
	\cardbanner{banner/white.png}
	\cardicon{banner/coin.png}
	\cardprice{3}
	\cardtitle{Händlerin}
	\cardcontent{DIe ziehst 1 Karte und erhältst + 1 Akton. Wenn du in diesen Zug vor dem Ausspeilen dieser \emph{HÄNDLERIN} noch kein Silber ausgespielt hast, erhältst du für das erste danach ausgespielte Silber +\coin{1}. Für jedes weitere ausgespielte Silber erhältst du keinen zusätzlichen Bonus. Hast du mherer \emph{HÄNDLERINNEN} ausgespielt, erhältst du pro \emph{HÄNDLERIN} +\coin{1}.}
\end{tikzpicture}
\hspace{-1cm}
\begin{tikzpicture}
	\card
	\cardstrip
	\cardbanner{banner/white.png}
	\cardicon{banner/coin.png}
	\cardprice{3}
	\cardtitle{Vasall}
	\cardcontent{Ist die aufgedeckte KArte eine Aktionskarte (auch ggf. kombineirte), \emph{darfst} du sie sofort ausspielen. Wenn du sie ausspielst, legst du sie in deinen SPielebreich und führst sofort die Anweisung darauf aus. Dafür benötigst du keine zusätzliche Aktion. Das Ausspielen der AKtionskarte verbraucht auch keien freie oder zusätzliche Aktion, die du durch das Ausspielen anderer Karten bereits gesammelt hast.}
\end{tikzpicture}
\hspace{-1cm}
\begin{tikzpicture}
	\card
	\cardstrip
	\cardbanner{banner/white.png}
	\cardicon{banner/coin.png}
	\cardprice{3}
	\cardtitle{Vorbotin}
	\cardcontent{Du ziehst 1 Karte und erhälts + 1 Aktion. Schau dir deinen Ablagestapel an. DU \emph{darfst} eine KArte daraus auswählen und oben auf deinen nachziehstapel legen. Die restlichen Karten (oder alle) legst du in beliebiger Reihenfolge zurück auf den Ablagestapel. Ist dein Ablagestapel leer, passiert nichts.}
\end{tikzpicture}
\hspace{-1cm}
\begin{tikzpicture}
	\card
	\cardstrip
	\cardbanner{banner/white.png}
	\cardicon{banner/coin.png}
	\cardprice{3}
	\cardtitle{Werkstatt}
	\cardcontent{Nimm dir eine Karte aus dem Vorrat und lege diese sofort auf deinen Ablagestapel. Du kansnt weder Geldkarten noch zusätzlich über Aktionskarten erhaltenes Geld oder Münzen (bei Erweiterungen mit Münzen) einsetzen, um den angegebenen Betrag auf der Karte zu erhöhen.}
\end{tikzpicture}
\hspace{-1cm}
\begin{tikzpicture}
	\card
	\cardstrip
	\cardbanner{banner/white.png}
	\cardicon{banner/coin.png}
	\cardprice{4}
	\cardtitle{Bürokrat}
	\cardcontent{Ist dein Nachziehstapel aufgebraucht, wenn du diese Karte spielst, legst du die Silberkarte verdeckt ab. Sie bildet dann deinen Nachziehstapel. Das Gleiche gilt für alle Mitspieler, die eine Punktekarte verdeckt auf den Nachziehstapel legen müssen.}
\end{tikzpicture}
\hspace{-1cm}
\begin{tikzpicture}
	\card
	\cardstrip
	\cardbanner{banner/white.png}
	\cardicon{banner/coin.png}
	\cardprice{4}
	\cardtitle{Geldverleiher}
	\cardcontent{\emph{Errata:} Der Kartentext ist falsch, es sollte \enquote{Du \emph{darfst} ein Kupfer aus der Hand entsorgen. Wenn du das tust: + \coin{3}.} heißen.
	\\
	\medskip
	\\
	Wenn du kein Kupfer zum entsorgen auf der Hand hast, erhältst du kein zusätzliches Geld für die Kaufphase.}
\end{tikzpicture}
\hspace{-1cm}
\begin{tikzpicture}
	\card
	\cardstrip
	\cardbanner{banner/white.png}
	\cardicon{banner/coin.png}
	\cardprice{4}
	\cardtitle{Miliz}
	\cardcontent{Deine Mitspieler müssen Karten aus ihrer Hand ablegen, bis sie nur noch 3 Karten auf der Hand haben. Spieler, die zum Zeitpunkt des Angriffs bereits 3 oder weniger Karten auf der Hand haben, müssen keine weiteren Karten ablegen.}
\end{tikzpicture}
\hspace{-1cm}
\begin{tikzpicture}
	\card
	\cardstrip
	\cardbanner{banner/white.png}
	\cardicon{banner/coin.png}
	\cardprice{4}
	\cardtitle{Schmiede}
	\cardcontent{Du musst 3 Karten vom Nachziehstapel ziehen und auf die Hand nehmen.}
\end{tikzpicture}
\hspace{-1cm}
\begin{tikzpicture}
	\card
	\cardstrip
	\cardbanner{banner/white.png}
	\cardicon{banner/coin.png}
	\cardprice{4}
	\cardtitle{Thronsaal}
	\cardcontent{\emph{Errata:} Der Kartentext ist falsch, es sollte \enquote{Du \emph{darfst} eine beliebige Aktionskarte aus der Hand zweimal ausspielen.} heißen.
	\\
	\medskip
	\\
	Wähle eine Aktionskarte aus deiner Hand und spiele sie zweimal aus, d.h. du legst die AKtionskarte aus, führst die Anweisungen der Karte komplett aus, nimmst die Karte zurück auf die Hand, legst sie noch einmal aus und führst die Anweisungen erneut aus. Für das doppelte ausspielen dieser Aktionskarte muss der Spieler keine zusätzliche Aktion (+1 Aktion) zur Verfügung haben -- sie ist sozusagen \enquote{kostenlos}. Legst du zwei \emph{THRONSAAL}-Karten aus, darfst du zuerst eine Aktion doppelt ausführen und dann eine andere Aktion ebenfalls doppelt ausführen. Du darfst aber nicht ein und und dieselbe Aktion vier mal ausführen. 
	\\
	\medskip
	\\
	Erlaubt die doppelt ausgespielte Karte +1 Aktion (z.B. der \emph{MARKT}), hast du nach der vollständigen Ausführung des \emph{THRONSAALS} zwei weitere Aktionen zur Verfügung. Hättest du zwei \emph{MARKT}-Karten regulär hintereinander ausgespielt, bleibt dir nur noch eine zusätzliche Aktion zur Verfügung, da das Ausspielen der zweiten Marktkarte schon die zusätzliche Aktion der ersten Karte aufgebraucht hätte. Beim \emph{THRONSAAL} ist es besonders wichtig, laut die verbleibenden Anzahlen an Aktionen mitzuzählen. Du darfst keine weitere Aktion ausspielen, bevor der \emph{THRONSAAL} komplett abgearbeitet ist.}
\end{tikzpicture}
\hspace{-1cm}
\begin{tikzpicture}
	\card
	\cardstrip
	\cardbanner{banner/white.png}
	\cardicon{banner/coin.png}
	\cardprice{4}
	\cardtitle{Umbau}
	\cardcontent{Der ausgespielte \emph{UMBAU} selber darf nicht entsorgt werden, da er sich nicht mehr in der Hand befindet. Weitere \emph{UMBAU}-Karten in deiner Hand dürfen entsorgt werden. Wenn du keine Karte zum Entsorgen auf der Hand hast, darfst du dir auch keine neue Karte nehmen. Die neue Karte, die du dir nimmst, darf maximal bis zu \coin{2} mehr als die entsorgte Karte kosten. Der Betrag darf weder durch weitere Geldkarten, Münzen oder zusätzliches Geld von anderen Aktionskarten erhöht werden. Die neue Karte kann die gleiche Karte sein, wie die, die du entsorgt hast. Lege die neue Karte auf deinen Ablagestapel.}
\end{tikzpicture}
\hspace{-1cm}
\begin{tikzpicture}
	\card
	\cardstrip
	\cardbanner{banner/white.png}
	\cardicon{banner/coin.png}
	\cardprice{4}
	\cardtitle{Wilddiebin}
	\cardcontent{Du ziehst 1 Karte, erhältst + 1 Aktion und +\coin{1}. Dann schaust du, wie viele Vorratsstapel (Fluch-, Geld-, Punkte- und Aktionskarten) bereits leer sind. Ist kein Stapel leer, musst du keine Handkarten ablegen. Ist ein Stapel leer, legst du 1 Handkarte ab usw. Wenn du nicht so viele Karten auf der Hand hast, wie Vorratsstapel leer sind, legst du so viele Karten ab, wie du kannst.}
\end{tikzpicture}
\hspace{-1cm}
\begin{tikzpicture}
	\card
	\cardstrip
	\cardbanner{banner/white.png}
	\cardicon{banner/coin.png}
	\cardprice{5}
	\cardtitle{Banditin}
	\cardcontent{Zuerst nimmst du ein Gold vom Vorrat und legst es auf deinen Ablagestapel. Dann deckt jeder Mitspieler -- beginnend bei deinem linken Mitspieler -- die obersten zwei Karten seine Nachziehstapels auf. Deckt ein SPieler zwei Geldkarten (auch ggf. kombiniert) außer Kupfer aus, muss er \emph{eine} davon entsorgen. Dabei darf er selbst entscheiden, welche Geldkarte er entsorgt. Die andere Geldkarte wird -- genauso wie alle anderen Karten -- abgelegt. Deckt ein Spieler eine Geldkarte außer Kupfer sowie eine andere Karte (z.B. ein Kupfer oder eine beliebige Aktionskarte) auf, wird diese Geldkarte entsorgt. Die andere aufgedeckte Karte wird abgelegt.}
\end{tikzpicture}
\hspace{-1cm}
\begin{tikzpicture}
	\card
	\cardstrip
	\cardbanner{banner/white.png}
	\cardicon{banner/coin.png}
	\cardprice{5}
	\cardtitle{Bibliothek}
	\cardcontent{Aktionskarten darfst du zur Seite legen, sobald du sie ziehst, musst dies aber nicht tun. Hast du bereits 7 oder mehr Karten auf der Hand, wenn du die \emph{BIBLIOTHEK} ausspielst, ziehst du keine Karte nach. Wenn dein Nachziehstapel während des Ziehens aufgebraucht ist, mischst du den Ablagestapel, mischst aber die zur Seite gelegten Aktionskarten nicht mit ein. Diese werden erst auf den Ablagestapel gelegt, sobald du 7 Karten auf der Hand hast. Sollten die Karten nicht reichen, ziehst du nur so viele Karten wie möglich.}
\end{tikzpicture}
\hspace{-1cm}
\begin{tikzpicture}
	\card
	\cardstrip
	\cardbanner{banner/white.png}
	\cardicon{banner/coin.png}
	\cardprice{5}
	\cardtitle{Hexe}
	\cardcontent{Wenn du die \emph{HEXE} spielst und nicht mehr genügend Fluchkarten vorrätig sind, werden diese im Uhrzeigersinn (beginnend bei deinem linken Nachbarn) verteilt. Die Mitspieler legen die Fluchkarten sofort auf ihren Ablagestapel. Du ziehst immer 2 Karten von deinem Nachziehstapel, auch wenn keine Fluchkarten mehr im Vorrat sind.}
\end{tikzpicture}
\hspace{-1cm}
\begin{tikzpicture}
	\card
	\cardstrip
	\cardbanner{banner/white.png}
	\cardicon{banner/coin.png}
	\cardprice{5}
	\cardtitle{Jahrmarkt}
	\cardcontent{Spielst du mehrer \emph{JAHRMÄRKTE} hintereinander, zählst du am besten laut mit, wie viele Aktionen du noch ausspielen darfst, damit du den Überblick behältst.}
\end{tikzpicture}
\hspace{-1cm}
\begin{tikzpicture}
	\card
	\cardstrip
	\cardbanner{banner/white.png}
	\cardicon{banner/coin.png}
	\cardprice{5}
	\cardtitle{Laboratorium}
	\cardcontent{Du \emph{musst} zuerst zwei Karten vom Nachziehstapel auf die Hand nehmen. Dann \emph{darfst} du eine weiter Aktionskarte ausspielen.}
\end{tikzpicture}
\hspace{-1cm}
\begin{tikzpicture}
	\card
	\cardstrip
	\cardbanner{banner/white.png}
	\cardicon{banner/coin.png}
	\cardprice{5}
	\cardtitle{Markt}
	\cardcontent{Du musst eine Karte vom Nachziehstapel auf die Hand nehmen. Du \emph{darfst} in der Aktionsphase eine weiter Aktionskarte ausspielen. Du \emph{darfst} in der Kaufphase einen zusätzlichen Kauf tätigen und hast dafür ein zusätzliches Geld zur Verfügung.}
\end{tikzpicture}
\hspace{-1cm}
\begin{tikzpicture}
	\card
	\cardstrip
	\cardbanner{banner/white.png}
	\cardicon{banner/coin.png}
	\cardprice{5}
	\cardtitle{Mine}
	\cardcontent{\emph{Errata:} Der Kartentext ist falsch, es sollte \enquote{Du \emph{darfst} eine beliebige Geldkarte aus der Hand entsorgen. Nimm eine Geldkarte vom Vorrat auf die Hand, die bis zu \coin{3} mehr kostet.} heißen.
	\\
	\medskip
	\\
	Normalerweise entsorgst du ein Kupfer und nimmst dir dafür ein Silber, oder du entsorgst ein Silber und nimmst dir ein Gold. Du kannst dir aber auch eine gleichwertige oder billigere Karte nehmen. Die neue Karte nimmst du sofort auf die Hand und darfst sie noch während deines Zuges einsetzen. Wer keine Geldkarte zum Entsorgen hat, erhält keine neue Karte.}
\end{tikzpicture}
\hspace{-1cm}
\begin{tikzpicture}
	\card
	\cardstrip
	\cardbanner{banner/white.png}
	\cardicon{banner/coin.png}
	\cardprice{5}
	\cardtitle{\scriptsize{Ratsversammlung}}
	\cardcontent{Jeder Spieler \emph{muss} eine Karte von seinem Nachziehstapel auf die Hand nehmen.}
\end{tikzpicture}
\hspace{-1cm}
\begin{tikzpicture}
	\card
	\cardstrip
	\cardbanner{banner/white.png}
	\cardicon{banner/coin.png}
	\cardprice{5}
	\cardtitle{Torwächterin}
	\cardcontent{Du ziehst 1 Karte und erhältst + 1 Aktion. Dann ziehst du dir die obersten 2 Karten deines Nachziehstapels an. Du kannst beide Karten entsorgen, beide Karten ablegen oder sie in beliebiger Reihenfolge zurück auf den Nachziehstapel legen. Du kannst aber auch eine entsorgten und eine ablegen, oder eine entsorgen und die anderen zurück auf den Nachziehstapel legen, oder eine ablegen und die anderen zurücklegen.}
\end{tikzpicture}
\hspace{-1cm}
\begin{tikzpicture}
	\card
	\cardstrip
	\cardbanner{banner/white.png}
	\cardicon{banner/coin.png}
	\cardprice{6}
	\cardtitle{Töpferei}
	\cardcontent{Nimm eine Karte vom Vorrat, die zu diesem Zeitpunkt maximal \coin{5} kostet. Du darfst kein zusätzliches \coin{\enspace} einsetzen, um dir eine teurere Karte zu nehmen. Außer \coin{\enspace} darf die Karte keine zusätzlichen Kosten enthalten.
	\\
	\medskip
	\\
	DU darfst dir zum Beispiel keine Karte mit \potion (aus Alchemie) oder \hex (aus Empires) in den Kosten nehmen. Die genommene Karte nimmst du direkt auf die Hand. Anschließend legst du eine beliebige Handkarte (das kann die gerade genommene oder eine andere sein) oben auf den Nachziehstapel.}
\end{tikzpicture}
\hspace{-1cm}
\begin{tikzpicture}
	\card
	\cardbanner{banner/green.png}
	\cardicon{banner/coin.png}
	\cardprice{4}
	\cardtitle{Gärten}
	\cardcontent{DIese Karte ist die einzige Punktekarte unter den Königreichkarten. SIe hat bis zum Ende des Spiels keine Funktion. Bei der Wertung des Spiels erhält der Spieler, der die Karte in seinem Kartensatz (Nachziehstapel, Handkarten und Ablagestapel) hat, für jeweils 10 Karten einen Siegpunkt. Es wird imemr abgerundet, d.h. 39 Karten ergeben 3 Siegpunkte, ebenso wie 31 Karten 3 Spiegpunkte ergeben. Wer mehrere \emph{GÄRTEN} besitzt, erhält für jeden \emph{GARTEN} die entsprechende Anzahl an Siegpunkten.}
\end{tikzpicture}
\hspace{-1cm}
\begin{tikzpicture}
	\card
	\cardstrip
	\cardbanner{banner/white.png}
	\cardicon{}
	\cardprice{}
	\cardtitle{\scriptsize{Empfohlene 10er Sätze\qquad}}
	\cardcontent{\emph{Erstes Spiel:}
		\\
		Burggraben, Dorf, Händlerin, Keller, Markt, Miliz, Mine, Schmiede, Umbau, Werkstatt
		\\
		\smallskip
		\\
		\emph{Verzerrte Größen:}
		\\
		Banditin, Bürokrat, Gärtner, Hexe, Jahrmarkt, Kapelle, Thronsaal, Töpferei, Torwächterin, Werkstatt
		\\
		\smallskip
		\\
		\emph{Schleichweg:}
		\\
		Bürokrat, Dorf, Geldverleiher, Laboratorium, Jahrmarkt, Ratsversammlung, Töpferei, Torwächterin, Vasall, Vorbotin
		\\
		\smallskip
		\\
		\emph{Kunststück:}
		\\
		Bibliothek, Gärten, Jahrmarkt, Keller, Miliz, Ratsversammlung, Schmiede, Thronsaal, Vorbotin, Wilddiebin
		\\
		\smallskip
		\\
		\emph{Verbesserungen:}
		\\
		Burggraben, Geldverleiher, Händlerin, Hexe, Keller, Markt, Mine, Töpferei, Umbau, Wilddiebin
		\\
		\smallskip
		\\
		\emph{Silber \& Gold:}
		\\
		Bandit, Bürokrat, Geldverleiher, Händlerin, Kapelle, Laboratorium, Mine, Thronsaal, Vasall, Vorbotin
		\\
	}
\end{tikzpicture}
\hspace{1cm}
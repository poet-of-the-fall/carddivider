% Basic settings for this card set
\renewcommand{\cardcolor}{menagerie}
\renewcommand{\cardextension}{Erweiterung XII}
\renewcommand{\cardextensiontitle}{Menagerie}
\renewcommand{\seticon}{menagerie.png}

\clearpage
\newpage
\section{\cardextension \ - \cardextensiontitle \ (Rio Grande Games 2020)}

\begin{tikzpicture}
	\card
	\cardstrip
	\cardbanner{banner/white.png}
	\cardicon{icons/coin.png}
	\cardprice{}
	\cardtitle{}
	\cardcontent{}
\end{tikzpicture}
\hspace{-0.6cm}
\begin{tikzpicture}
	\card
	\cardstrip
	\cardbanner{banner/white.png}
	\cardtitle{\scriptsize{Spielvorbereitung (1/3)}\qquad}
	\cardcontent{\underline{\emph{Ereignisse}}
	\\
	Zusätzlich zu den Königreichkarten gibt es \emph{Ereignisse}. Sie haben exakt die gleiche Funktion wie in \emph{Abenteuer} (und \emph{Empires}) und können wahrend des Spiels erworben werden (siehe NEUE REGELN).

	\smallskip

	Wir empfehlen, pro Spiel maximal insgesamt 2 \emph{Projekte} (aus \emph{Renaissance}), \emph{Landmarken} (aus \emph{Empires}) oder \emph{Ereignisse} (aus \emph{Abenteuer}, \emph{Empires} und/oder \emph{Menagerie}) zu verwenden.}
\end{tikzpicture}
\hspace{-0.6cm}
\begin{tikzpicture}
	\card
	\cardstrip
	\cardbanner{banner/white.png}
	\cardtitle{\scriptsize{Spielvorbereitung (2/3)}\qquad}
	\cardcontent{\tiny{\underline{\emph{Wege}}
	\\
	Zusätzlich zu den Königreichkarten und den \emph{Ereignissen} gibt es \emph{Wege}, deren Anweisungen während des Spiels den gespielten Aktionskarten zusätzliche Möglichkeiten geben (siehe NEUE REGELN).
	
	\smallskip

	Wir empfehlen, pro Spiel maximal 1 \emph{Weg} zu verwenden.
	
	\smallskip

	Zieht \emph{Ereignisse} und \emph{Wege} zufällig aus einem Stapel (dieser kann auch die \emph{Landmarken} (aus \emph{Empires}), \emph{Ereignisse} (aus \emph{Abenteuer} oder \emph{Empires}) und/oder \emph{Projekte} (aus \emph{Renaissance}) enthalten) oder mischt sie (trotz ihrer unterschiedlichen Rückseite) in die Platzhalterkarten ein. Deckt ihr ein \emph{Ereignis} oder einen \emph{Weg} auf, legt das \emph{Ereignis} bzw. den \emph{Weg} neben dem Vorrat bereit. \emph{Ereignisse} und \emph{Wege} gehören nicht zum Vorrat. Deckt so lange Karten auf, bis ihr 10 Königreichkarten und (empfohlen) maximal 2 \emph{Ereignisse}, \emph{Landmarken}, \emph{Projekte} oder \emph{Wege} (wir empfehlen maximal 1 \emph{Weg} pro Spiel) aufgedeckt habt. Legt die überzähligen \emph{Ereignisse}, \emph{Wege}, \emph{Landmarken} und \emph{Projekte} in die Schachtel zurück -- sie kommen in diesem Spiel nicht zum Einsatz.
	\\
	\emph{Ereignisse} und \emph{Wege} können nicht als Bannstapel für die \emph{JUNGE HEXE} (aus \emph{Reiche Ernte}) genutzt werden.
	\\
	Jedes \emph{Ereignis} und jeder \emph{Weg} ist nur 1x Spiel enthalten.
	\\
	Wenn ihr die Karte \emph{Weg der Maus} verwendet, legt eine nicht verwendete Aktionskarte mit den Kosten \coin[2] oder \coin[3] bereit und trefft die fur diese Karte notwendigen Vorbereitungen.}}
\end{tikzpicture}
\hspace{-0.6cm}
\begin{tikzpicture}
	\card
	\cardstrip
	\cardbanner{banner/white.png}
	\cardtitle{\scriptsize{Spielvorbereitung (3/3)}\qquad}
	\cardcontent{\underline{\emph{Exil-Tableaus und Pferde}}
	\\
	Wenn ihr mindestens eine Karte bzw. einen \emph{Weg} oder ein \emph{Ereignis} verwendet, die sich auf das \emph{Exil} bezieht bzw. darauf, dass eine Karte verbannt werden soll, erhält jeder Spieler ein \emph{Exil}-Tableau: 
	\\ 
	\emph{Königreichkarten: DEPOT - HEXENZIRKEL - KAMELZUG - KARDINAL - KOPFGELDJÄGERIN - VERTREIBUNG - WACHE - ZUFLUCHTSORT 
	\\ 
	Ereignisse: ENKLAVE - INVESTITION - TRANSPORT - VERBANNUNG 
	\\ 
	Wege: WEG DES KAMELS - WEG DES WURMS.}
	
	\medskip

	Wenn ihr mindestens eine Karte verwendet, die sich auf das \emph{Pferd} bezieht, legt den \emph{Pferde}-Stapel neben dem Vorrat bereit: 
	\\
	\emph{Königreichkarten: HERBERGE - KAVALLERIE - KOPPEL - NACHSCHUB - PFERDESTALL - SCHLITTEN - SCHROTT - STALLBURSCHE
	\\
	Ereignisse: AUSRITT - FORDERUNG - GUTES GESCHÄFT - STAMPEDE.}
	\\
	\emph{Pferde} gehören nicht zum Vorrat.}
\end{tikzpicture}
\hspace{-0.6cm}
\begin{tikzpicture}
	\card
	\cardstrip
	\cardbanner{banner/white.png}
	\cardtitle{\footnotesize{Neue Regeln (1/5)}\qquad}
	\cardcontent{}
\end{tikzpicture}
\hspace{-0.6cm}
\begin{tikzpicture}
	\card
	\cardstrip
	\cardbanner{banner/white.png}
	\cardtitle{\scriptsize{Empfohlene 10er Sätze\qquad}}
	\cardcontent{\emph{Name:}\\
	Karten ...

	\smallskip

	\emph{Name:}\\
	Karten ...

	\smallskip

	\emph{Name:}\\
	Karten ...

	\smallskip

	\emph{Name:}\\
	Karten ...

	\smallskip

	\emph{Name:}\\
	Karten ...

	\smallskip

	\emph{Name:}\\
	Karten ...}
\end{tikzpicture}
\hspace{-0.6cm}
\begin{tikzpicture}
	\card
	\cardstrip
	\cardbanner{banner/white.png}
	\cardtitle{Platzhalter\quad}
\end{tikzpicture}
\hspace{0.6cm}
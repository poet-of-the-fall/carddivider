% Basic settings for this card set
\renewcommand{\cardcolor}{menagerie}
\renewcommand{\cardextension}{Erweiterung XII}
\renewcommand{\cardextensiontitle}{Menagerie}
\renewcommand{\seticon}{menagerie.png}

\clearpage
\newpage
\section{\cardextension \ - \cardextensiontitle \ (Rio Grande Games 2020)}

\begin{tikzpicture}
	\card
	\cardstrip
	\cardbanner{banner/white.png}
	\cardicon{icons/coin.png}
	\cardprice{}
	\cardtitle{}
	\cardcontent{}
\end{tikzpicture}
\hspace{-0.6cm}
\begin{tikzpicture}
	\card
	\cardstrip
	\cardbanner{banner/white.png}
	\cardtitle{\scriptsize{Spielvorbereitung (1/3)}\qquad}
	\cardcontent{\underline{\emph{Ereignisse}}
	\\
	Zusätzlich zu den Königreichkarten gibt es \emph{Ereignisse}. Sie haben exakt die gleiche Funktion wie in \emph{Abenteuer} (und \emph{Empires}) und können wahrend des Spiels erworben werden (siehe NEUE REGELN).

	\smallskip

	Wir empfehlen, pro Spiel maximal insgesamt 2 \emph{Projekte} (aus \emph{Renaissance}), \emph{Landmarken} (aus \emph{Empires}) oder \emph{Ereignisse} (aus \emph{Abenteuer}, \emph{Empires} und/oder \emph{Menagerie}) zu verwenden.}
\end{tikzpicture}
\hspace{-0.6cm}
\begin{tikzpicture}
	\card
	\cardstrip
	\cardbanner{banner/white.png}
	\cardtitle{\scriptsize{Spielvorbereitung (2/3)}\qquad}
	\cardcontent{\tiny{\underline{\emph{Wege}}
	\\
	Zusätzlich zu den Königreichkarten und den \emph{Ereignissen} gibt es \emph{Wege}, deren Anweisungen während des Spiels den gespielten Aktionskarten zusätzliche Möglichkeiten geben (siehe NEUE REGELN).
	
	\smallskip

	Wir empfehlen, pro Spiel maximal 1 \emph{Weg} zu verwenden.
	
	\smallskip

	Zieht \emph{Ereignisse} und \emph{Wege} zufällig aus einem Stapel (dieser kann auch die \emph{Landmarken} (aus \emph{Empires}), \emph{Ereignisse} (aus \emph{Abenteuer} oder \emph{Empires}) und/oder \emph{Projekte} (aus \emph{Renaissance}) enthalten) oder mischt sie (trotz ihrer unterschiedlichen Rückseite) in die Platzhalterkarten ein. Deckt ihr ein \emph{Ereignis} oder einen \emph{Weg} auf, legt das \emph{Ereignis} bzw. den \emph{Weg} neben dem Vorrat bereit. \emph{Ereignisse} und \emph{Wege} gehören nicht zum Vorrat. Deckt so lange Karten auf, bis ihr 10 Königreichkarten und (empfohlen) maximal 2 \emph{Ereignisse}, \emph{Landmarken}, \emph{Projekte} oder \emph{Wege} (wir empfehlen maximal 1 \emph{Weg} pro Spiel) aufgedeckt habt. Legt die überzähligen \emph{Ereignisse}, \emph{Wege}, \emph{Landmarken} und \emph{Projekte} in die Schachtel zurück -- sie kommen in diesem Spiel nicht zum Einsatz.
	\\
	\emph{Ereignisse} und \emph{Wege} können nicht als Bannstapel für die \emph{JUNGE HEXE} (aus \emph{Reiche Ernte}) genutzt werden.
	\\
	Jedes \emph{Ereignis} und jeder \emph{Weg} ist nur 1x Spiel enthalten.
	\\
	Wenn ihr die Karte \emph{Weg der Maus} verwendet, legt eine nicht verwendete Aktionskarte mit den Kosten \coin[2] oder \coin[3] bereit und trefft die fur diese Karte notwendigen Vorbereitungen.}}
\end{tikzpicture}
\hspace{-0.6cm}
\begin{tikzpicture}
	\card
	\cardstrip
	\cardbanner{banner/white.png}
	\cardtitle{\scriptsize{Spielvorbereitung (3/3)}\qquad}
	\cardcontent{\underline{\emph{Exil-Tableaus und Pferde}}
	\\
	Wenn ihr mindestens eine Karte bzw. einen \emph{Weg} oder ein \emph{Ereignis} verwendet, die sich auf das \emph{Exil} bezieht bzw. darauf, dass eine Karte verbannt werden soll, erhält jeder Spieler ein \emph{Exil}-Tableau: 
	\\ 
	\emph{Königreichkarten: DEPOT - HEXENZIRKEL - KAMELZUG - KARDINAL - KOPFGELDJÄGERIN - VERTREIBUNG - WACHE - ZUFLUCHTSORT 
	\\ 
	Ereignisse: ENKLAVE - INVESTITION - TRANSPORT - VERBANNUNG 
	\\ 
	Wege: WEG DES KAMELS - WEG DES WURMS.}
	
	\medskip

	Wenn ihr mindestens eine Karte verwendet, die sich auf das \emph{Pferd} bezieht, legt den \emph{Pferde}-Stapel neben dem Vorrat bereit: 
	\\
	\emph{Königreichkarten: HERBERGE - KAVALLERIE - KOPPEL - NACHSCHUB - PFERDESTALL - SCHLITTEN - SCHROTT - STALLBURSCHE
	\\
	Ereignisse: AUSRITT - FORDERUNG - GUTES GESCHÄFT - STAMPEDE.}
	\\
	\emph{Pferde} gehören nicht zum Vorrat.}
\end{tikzpicture}
\hspace{-0.6cm}
\begin{tikzpicture}
	\card
	\cardstrip
	\cardbanner{banner/white.png}
	\cardtitle{\footnotesize{Neue Regeln (1/10)}\qquad}
	\cardcontent{\tiny{\emph{Es gelten die Basisspielregeln mit folgenden Änderungen:}
	
	\smallskip

	\emph{Exil-Tableaus / Karten verbannen}
	\\
	In Menagerie gibt es \emph{Exil}-Tableaus, auf die du Karten verbannen und von denen du sie wieder zurückholen kannst.
	\\
	Es gibt Anweisungen auf Aktionskarten, \emph{Ereignissen} und \emph{Wegen}, mit denen du (oder die Mitspieler, wie beim \emph{HEXENZIRKEL}) Karten verbannst. Das heißt, du legst sie auf dein eigenes \emph{Exil}-Tableau (bzw. beim \emph{HEXENZIRKEL} die Mitspieler auf ihr eigenes \emph{Exil}-Tableau). \enquote{Karten im \emph{Exil}} sind Karten aut dem eigenen \emph{Exil}-Tableau. Eine Karte aus dem Vorrat zu verbannen bedeutet nicht, dass du sie nimmst, und löst keine \enquote{wenn du ... nimmst}-Effekte aus.

	\smallskip

	Wenn du eine Karte nimmst, darfst du alle anderen Karten von deinem \emph{Exil}-Tableau ablegen, die den gleichen Namen haben wie die genommene Karte. Wenn du z.B. zwei \emph{SILBER} in deinem \emph{Exil} hast und ein \emph{SILBER} nimmst, darfst du die beiden \emph{SILBER} von deinem \emph{Exil}-Tableau auf deinen Ablagestapel ablegen. Du darfst sie auch im \emph{Exil} liegen lassen, aber du darfst nicht eins davon ablegen. Eine Karte vom \emph{Exil}-Tableau abzulegen bedeutet nicht, dass du sie nimmst, sondern ist ein \enquote{Ablegen einer Karte}. Wenn es zu einem anderen Zeitpunkt als in der Aufräumphase stattfindet, kann es den \emph{TUNNEL} (aus \emph{Hinterland}) oder den \emph{DORFANGER} auslösen.

	\smallskip

	Karten auf dem eigenen \emph{Exil}-Tableau liegen offen und sind für Mitspieler einsehbar, gehören aber dir; zähle sie bei Spielende, bzw. wenn die Anzahl deiner Karten relevant ist, zu deinem Gesamtergebnis hinzu.}}
\end{tikzpicture}
\hspace{-0.6cm}
\begin{tikzpicture}
	\card
	\cardstrip
	\cardbanner{banner/white.png}
	\cardtitle{\footnotesize{Neue Regeln (2/10)}\qquad}
	\cardcontent{\tiny{\emph{Wege}
	\\
	In Menagerie gibt es \emph{Wege}. Sie sind keine Königreichkarten und können nicht gekauft oder wie \emph{Ereignisse} erworben werden. Vor Spielbeginn entscheiden die Spieler, mit wie vielen \emph{Wegen} gespielt wird. Wir empfehlen, pro Spiel maximal 1 \emph{Weg} zu verwenden. Die \emph{Wege} werden neben dem Vorrat bereitgelegt, gehören aber nicht zum Vorrat.
	\\
	Jeder \emph{Weg} gibt Aktionskarten eine alternative Möglichkeit: Entweder spielst du die Aktionskarte gemäß ihrer eigenen Anweisung oder du führst stattdessen beim Spielen der Karte die Anweisungen einer ausliegenden \emph{Wege}-Karte aus.

	\smallskip

	Um den Überblick zu behalten, ist es ratsam, die Aktionskarte um 90 Grad zu drehen, die mit einem \emph{Weg} benutzt wurde. So kannst du dir merken, dass du den \emph{Weg} benutzt hast anstatt der Anweisung auf der Aktionskarte.

	\smallskip

	Eine Aktionskarte mit den Anweisungen einer \emph{Wege}-Karte zu spielen bedeutet, dass du nichts tust, wozu dich die Aktionskarte beim Spielen anweist. Wenn du z.B. ein \emph{PFERD} spielst und dich entscheidest, stattdessen die Anweisung vom \emph{Weg des Schafes} (mit der Anweisung +\coin[2]) zu verwenden, erhältst du +\coin[2] und nicht +2 Karten und +1 Aktion (die Anweisung des \emph{PFERDES}) und legst das \emph{PFERD} nicht auf seinen Stapel zurück.
	
	\smallskip

	Anweisungen unterhalb der Trennlinie sind nicht von der \emph{Wege}-Karte betroffen; sie finden weiterhin wie dort erwähnt statt. Wenn du z.B. eine \emph{FERNSTRASSE} (aus \emph{Hinterland}) spielst und den \emph{Weg des Schafes} benutzt, erhältst du +\coin[2] und während die \emph{FERNSTRASSE} dann im Spiel ist, kosten andere Karten wegen ihrer Fähigkeit weniger.}}
\end{tikzpicture}
\hspace{-0.6cm}
\begin{tikzpicture}
	\card
	\cardstrip
	\cardbanner{banner/white.png}
	\cardtitle{\footnotesize{Neue Regeln (3/10)}\qquad}
	\cardcontent{\tiny{\textit{Zusatzliche Hinweise}
	\\
	\begin{itemize}
		\item Auf einigen \emph{Wegen} steht \enquote{diese Karte}. Damit ist die Aktionskarte gemeint, die du mit der Anweisung des \emph{Weges} spielst, nicht die \emph{Wege}-Karte selbst. Der \emph{Weg der Schildkröte} sagt z.B. \enquote{Lege diese Karte zur Seite.}: Wenn du einen \emph{MARKT} spielst, aber den \emph{Weg der Schildkröte} benutzt, legst du den \emph{MARKT} zur Seite.
		\item Wenn eine \emph{ZAUBERIN} (aus \emph{Empires}) dich betrifft, darfst du die Anweisungen der ersten gespielten Aktionskarte nicht ausführen. Du darfst aber statt der Anweisungen auf der Aktionskarte die Anweisungen auf einer \emph{Wege}-Karte ausführen. Dann bekommst du aber nicht +1 Karte und +1 Aktion.
		\item Wenn eine Aktionskarte zu einem ungewöhnlichen Zeitpunkt gespielt werden kann, wie z.B. der \emph{HIRTENHUND}, kann stattdessen ein \emph{Weg} benutzt werden.
		\item Wenn du eine Aktionskarte mehrmals spielst, z.B. mit dem \emph{DRAHTZIEHER}, darfst du bei jedem Spielen wählen, ob du einen \emph{Weg} benutzen möchtest oder nicht.
		\item Ist deine gespielte Karte eine \emph{Dauerkarte}, bleibt sie nur im Spiel, wenn du sie mindestens einmal als sie selbst gespielt hast, also nicht stattdessen einen \emph{Weg} benutzt hast. Bleibt sie im Spiel, musst du dir für deinen nächsten Zug merken, wie oft du sie als sie selbst gespielt hast.
		\item Löst das Spielen einer Aktionskarte \enquote{Zuerst}-Handlungen aus (z.B. weil es eine Angriffskarte ist, auf die die Mitspieler mit \emph{BURGGRABEN} reagieren können, oder weil die zuvor gespielte Karte ein \emph{BRENNOFEN} war), werden diese abgehandelt, bevor du die Entscheidung triffst, ob du statt der Aktionskarte einen \emph{Weg} benutzt oder nicht.
		\item Die Marker aus \emph{Abenteuer} gelten auch, wenn du statt der Anweisungen der Aktionskarte eines entsprechend markierten Vorratsstapels eine \emph{Wege}-Karte benutzt.
	\end{itemize}
	\\
	}}
\end{tikzpicture}
\hspace{-0.6cm}
\begin{tikzpicture}
	\card
	\cardstrip
	\cardbanner{banner/white.png}
	\cardtitle{\footnotesize{Neue Regeln (4/10)}\qquad}
	\cardcontent{\emph{Pferde}
	\\
	In \emph{Menagerie} gibt es einen Stapel mit \emph{Pferde}-Karten, und es gibt Aktionskarten sowie \emph{Ereignisse}, mit denen du \emph{Pferde} erhalten kannst. Der \emph{Pferde}-Stapel gehört nicht zum Vorrat. Du darfst nur dann eine \emph{Pferde}-Karte von diesem Stapel nehmen, wenn eine Karte oder ein Ereignis dich anweist, dass du ein \emph{Pferd} nehmen sollst, aber nicht durch Karten wie die \emph{FALKNERIN} oder die \emph{VERTREIBUNG}.

	\smallskip

	\begin{itemize}
		\item \enquote{Nimm ein Pferd} bedeutet, dass du dir ein \emph{Pferd} vom \emph{Pferde}-Stapel nimmst. Wenn du angewiesen wirst, ein \emph{Pferd} zu nehmen, und es ist kein \emph{Pferd} mehr auf dem \emph{Pferde} Stapel, nimmst du kein \emph{Pferd}.
		\item Wenn du ein \emph{Pferd} spielst, erhältst du +2 Karten, +1 Aktion und legst das \emph{Pferd} zurück auf den \emph{Pferde}-Stapel. Wenn du eine Karte wie den \emph{DRAHTZIEHER} dazu verwendest, um ein \emph{Pferd} mehrmals zu spielen, erhältst du +2 Karten und +1 Aktion für jedes Mal, auch wenn du das \emph{Pferd} nur einmal zurücklegen kannst.
	\end{itemize}
	\\
	}
\end{tikzpicture}
\hspace{-0.6cm}
\begin{tikzpicture}
	\card
	\cardstrip
	\cardbanner{banner/white.png}
	\cardtitle{\footnotesize{Neue Regeln (5/10)}\qquad}
	\cardcontent{\tiny{\emph{Ereignisse}
	\\ 
	In \emph{Menagerie} gibt es \emph{Ereignisse}, die erstmals in \emph{Abenteuer} erschienen sind. In deiner Kaufphase kannst du ein \emph{Ereignis} statt einer anderen Karte erwerben (dies verbraucht 1 Kauf). Du bezahlst die Kosten, die auf dem \emph{Ereignis} stehen, und dessen Effekt tritt sofort ein.

	\smallskip

	\emph{Ereignisse} sind keine Königreichkarten. Sie liegen lediglich aus und liefern einen Effekt, den du kaufen kannst. Es gibt keine Möglichkeit, dass du ein \emph{Ereignis} nehmen kannst oder dass ein \emph{Ereignis} in deinem Kartensatz ist. Der Erwerb eines \emph{Ereignisses} verbraucht 1 Kauf. Normalerweise kannst du entweder eine Karte kaufen oder ein \emph{Ereignis} erwerben. Wenn du 2 Käufe hast, wie z.B. nach dem Spielen des \emph{ZUFLUCHTSORTS}, kannst du bis zu zwei Karten kaufen oder zwei \emph{Ereignisse} erwerben oder eine Karte und ein \emph{Ereignis}, in beliebiger Reihenfolge. 
	\\
	\emph{Ereignisse} können in einem Zug mehrmals erworben werden, wenn du genügend Käufe und \coin[ ] dafür verfügbar hast, es sei denn, das \emph{Ereignis} sagt etwas anderes wie \emph{VERZWEIFLUNG}. 
	\\
	Nachdem du ein \emph{Ereignis} erworben hast, darfst du in dieser Kaufphase keine weiteren Geldkarten spielen, es sei denn, ein \emph{Ereignis} oder eine Karte erlaubt dir dies explizit. Der Erwerb eines \emph{Ereignisses} ist kein Kauf einer Karte und löst deshalb nicht
	Karten wie den \emph{FEILSCHER} (aus \emph{Hinterland}) aus.
	\\
	Die Kosten von \emph{Ereignissen} werden nicht durch Karten wie die \emph{BRÜCKE} (aus \emph{Intrige}) beeinflusst.}}
\end{tikzpicture}
\hspace{-0.6cm}
\begin{tikzpicture}
	\card
	\cardstrip
	\cardbanner{banner/white.png}
	\cardtitle{\footnotesize{Neue Regeln (6/10)}\qquad}
	\cardcontent{\emph{Reaktionskarten}
	\\
	In \emph{Menagerie} gibt es fünf \emph{Reaktionskarten}. Normalerweise werden \emph{AKTIONS-REAKTIONS-Karten} wie jede andere Aktionskarte gespielt, um die Anweisungen über der Trennlinie zu nutzen. Die \emph{REAKTION} der Karte wird zu einem anderen Zeitpunkt, der unter der Trennlinie angegeben ist, ausgelöst und führt NICHT dazu, dass die Karte gespielt, sondern in der Regel nur aufgedeckt wird. Vier der \emph{Reaktionskarten} aus \emph{Menagerie} können - zusätzlich zum \enquote{normalen} Spielen in der Aktionsphase eines Spielers - zu einem ungewöhnlichen Zeitpunkt als \emph{REAKTION} gespielt werden: \emph{SCHWARZE KATZE},   \emph{FALKNERIN}, \emph{HIRTENHUND} und \emph{DORFANGER}. 

	\smallskip

	Das Spielen einer dieser \emph{Reaktionskarten} -- ausgelöst durch die Anweisung unter der Trennlinie -- bringt sie ins Spiel, als wenn sie normal gespielt wurde, verbraucht aber keine Aktion. Wenn du eine Karte in dem Zug eines Mitspielers spielst, legst du sie in der Aufräumphase jenes Zuges ab, außer es ist eine Dauerkarte, bei der noch weitere Anweisungen ausstehen.
	
	\smallskip

	Wenn du eine \emph{Reaktionskarte} -- ausgelöst durch die Anweisung unter der Trennlinie -- spielst, darfst du statt der Anweisung der \emph{Reaktionskarte} einen \emph{Weg} nutzen.}
\end{tikzpicture}
\hspace{-0.6cm}
\begin{tikzpicture}
	\card
	\cardstrip
	\cardbanner{banner/white.png}
	\cardtitle{\footnotesize{Neue Regeln (7/10)}\qquad}
	\cardcontent{\emph{Reaktionskarten -- Forsetzung}
	\\
	Wenn das Spielen einer dieser \emph{Reaktionskarten} -- ausgelöst durch die Anweisung unter der Trennlinie -- dazu führt, dass du eine weitere \emph{Reaktionskarte} ziehst, die sofort verwendet werden kann, darfst du sie benutzen usw. Hast du z.B. eine \emph{SCHWARZE KATZE} auf der Hand und ein Mitspieler nimmt eine \emph{PROVINZ}, darfst du sie spielen. Ziehst du dabei eine weitere \emph{SCHWARZE KATZE}, darfst du sie ebenfalls spielen usw.

	\smallskip

	Wenn mehrere Spieler, die nicht am Zug sind, etwas zur gleichen Zeit machen möchten -- wie z.B. \emph{Reaktionskarten} spielen --, beginnt der (ausgehend vom Spieler, der am Zug ist) in Zugreihenfolge nachfolgende Spieler und führt eine Aktivität aus (z.B. spielt er eine \emph{Reaktionskarte}), dann folgen reihum die weiteren Spieler. Dies kann etwas daran ändern, was die einzelnen Spieler machen möchten. Nachdem ein Spieler ggf. eine entsprechende Aktivität 
	ausgeführt hat, startet ihr erneut ausgehend vom Spieler, der am Zug ist, und ermittelt, wer etwas machen möchte (z.B. eine weitere \emph{Reaktionskarte} spielen) usw. 
	
	\smallskip
	
	Manchmal ergibt sich eine Situation, in der mit einer \emph{Reaktionskartereagiert} werden darf, und diese \emph{Reaktionskarte} ergibt eine weitere Situation, in der mit einer \emph{Reaktionskarte} reagiert werden kann. Löst alle \emph{Reaktionskarten} für die neue
	Situation auf und geht dann zurück und löst die erste \emph{Reaktionskarte} auf.}
\end{tikzpicture}
\hspace{-0.6cm}
\begin{tikzpicture}
	\card
	\cardstrip
	\cardbanner{banner/white.png}
	\cardtitle{\footnotesize{Neue Regeln (8/10)}\qquad}
	\cardcontent{\emph{Kosten mit einem *}
	\\
	Bei einigen Karten sind die Kosten mit einem * markiert, der die Spieler an etwas erinnern soll.

	\smallskip

	Das \emph{PFERD} hat z.B. einen * bei seinen Kosten, kostet aber \coin[3] und kann daher mit einen \emph{UMBAU} (aus dem \emph{Basisspiel}) entsorgt werden, um dafür ein um \coin[2] teureres \emph{HERZOGTUM} zu nehmen usw. Für \emph{Anweisungen}, die verschiedene Kosten vergleichen, hat das \emph{PFERD} die gleichen Kosten wie andere Karten, die \coin[3] kosten. Beim \emph{PFERD} soll der * lediglich daran erinnern, dass du \emph{PFERDE} nicht kaufen kannst, weil sie nicht zum Vorrat gehören.
	
	\smallskip

	Beim \emph{VIEHMARKT} erinnert der * daran, dass du die Karte auf eine andere Art und Weise kaufen kannst.

	\smallskip

	\emph{SCHLACHTROSS}, \emph{FISCHER} und \emph{WANDERIN} haben einen *, weil ihre Kosten sich während eines Zuges verändern können. Dies kann manchmal zu verschachtelten Situationen führen.}
\end{tikzpicture}
\hspace{-0.6cm}
\begin{tikzpicture}
	\card
	\cardstrip
	\cardbanner{banner/white.png}
	\cardtitle{\footnotesize{Neue Regeln (9/10)}\qquad}
	\cardcontent{\emph{Kosten mit einem * -- Fortsetzung}
	\\
	Wenn sich die Kosten dieser Karten ändern, ändern sie sich auf allen Karten dieses Namens, überall und für alle Zwecke. Wenn du z.B. eine der Karten durch \emph{UMBAU} entsorgst, gelten die geänderten Kosten, nicht die auf der Karte aufgedruckten. Hat ein Mitspieler eine Anweisung, für die es wichtig ist, welche Kosten seine Karte während deines Zuges hat (wie z.B. die Promo-Karte \emph{GOUVERNEUR}), nutzt er die gleichen, geänderten Kosten wie du. Kosten können nicht unter \coin[0] sinken.
	
	\smallskip
	
	\emph{SCHLACHTROSS} und \emph{FISCHER} werden auch von anderen Dingen beeinflusst, die ihre Kosten ändern, wie der \emph{BRÜCKE} (aus \emph{Intrige}).
	
	\smallskip
	
	Kosten können sich während der Ausführung von Anweisungen ändern. Wichtig ist, den Anweisungen auf den Karten in ihrer Reihenfolge zu folgen.}
\end{tikzpicture}
\hspace{-0.6cm}
\begin{tikzpicture}
	\card
	\cardstrip
	\cardbanner{banner/white.png}
	\cardtitle{\footnotesize{Neue Regeln (10/10)}\qquad}
	\cardcontent{\emph{Dauerkarten}
	\\
	In \emph{Menagerie} gibt es vier \emph{Dauerkarten} (wie schon in \emph{Seaside}, \emph{Abenteuer} und \emph{Renaissance}). Die orangefarbenen \emph{Dauerkarten} beinhalten Anweisungen, die in spüteren Zügen umgesetzt werden. Sie werden nicht in der Aufräumphase des Zuges abgelegt, in dem sie gespielt wurden, sondern bleiben bis zur Aufräumphase des Zuges, in dem die letzte Anweisung ausgeführt wird, im Spiel. Wird eine \emph{Dauerkarte} mehrfach gespielt (z.B. durch den \emph{THRONSAAL} aus dem \emph{Basisspiel}), bleibt die verursachende Karte ebenfalls so lange im Spiel, bis die \emph{Dauerkarte} abgelegt wird. Um anzuzeigen, dass eine \emph{Dauerkarte} in der aktuellen Aufräumphase noch nicht abgelegt wird, wird sie in eine eigene Reihe oberhalb der restlichen gespielten Karten gelegt. Die Anweisungen in späteren Zügen finden meistens zu Beginn des nächsten eigenen Zuges statt. Bei mehreren im Spiel befindlichen \emph{Dauerkarten} darf der Spieler die Reihenfolge selbst bestimmen, in der er sie abhandelt.
	
	\smallskip

	Hinweis: Wird eine gespielte \emph{Dauerkarte} mit der Anweisung eines \emph{Weges} genutzt, bleibt sie nur über den aktuellen Zug hinaus im Spiel, wenn sie mindestens 1x mit ihrer eigentlichen Anweisung genutzt wurde (siehe NEUE REGELN, Wege).}
\end{tikzpicture}
\hspace{-0.6cm}
\begin{tikzpicture}
	\card
	\cardstrip
	\cardbanner{banner/white.png}
	\cardtitle{\footnotesize{Anweisungen (1/3)}\qquad}
	\cardcontent{}
\end{tikzpicture}
\hspace{-0.6cm}
\begin{tikzpicture}
	\card
	\cardstrip
	\cardbanner{banner/white.png}
	\cardtitle{\footnotesize{Anweisungen (2/3)}\qquad}
	\cardcontent{}
\end{tikzpicture}
\hspace{-0.6cm}
\begin{tikzpicture}
	\card
	\cardstrip
	\cardbanner{banner/white.png}
	\cardtitle{\footnotesize{Anweisungen (3/3)}\qquad}
	\cardcontent{}
\end{tikzpicture}
\hspace{-0.6cm}
\begin{tikzpicture}
	\card
	\cardstrip
	\cardbanner{banner/white.png}
	\cardtitle{\scriptsize{Empfohlene 10er Sätze\qquad}}
	\cardcontent{\emph{Name:}\\
	Karten ...

	\smallskip

	\emph{Name:}\\
	Karten ...

	\smallskip

	\emph{Name:}\\
	Karten ...

	\smallskip

	\emph{Name:}\\
	Karten ...

	\smallskip

	\emph{Name:}\\
	Karten ...

	\smallskip

	\emph{Name:}\\
	Karten ...}
\end{tikzpicture}
\hspace{-0.6cm}
\begin{tikzpicture}
	\card
	\cardstrip
	\cardbanner{banner/white.png}
	\cardtitle{Platzhalter\quad}
\end{tikzpicture}
\hspace{0.6cm}
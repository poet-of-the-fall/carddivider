% Basic settings for this card set
\renewcommand{\cardcolor}{allies}
\renewcommand{\cardextension}{Erweiterung XIII}
\renewcommand{\cardextensiontitle}{Verbündete}
\renewcommand{\seticon}{allies.png}

\clearpage
\newpage
\section{\cardextension \ - \cardextensiontitle \ (Rio Grande Games 2022)}

\begin{tikzpicture}
	\card
	\cardstrip
	\cardbanner{banner/white.png}
	\cardicon{icons/coin.png}
	\cardprice{2}
	\cardtitle{Schmeichler}
	\cardcontent{Du kannst diese Karte unabhängig davon spielen, wie viele Karten noch auf deiner Hand sind. Wenn du diese Karte spielst und dann mindestens drei Karten auf deiner Hand hast, legst du drei Karten ab und erhältst +\coin[3]. Hast du eine oder zwei Karten, legst du diese ab und erhältst +\coin[3]. Hast du keine Karte, erhältst du nicht die +\coin[3]. Wenn du einen \emph{SCHMEICHLER} nimmst oder entsorgst, erhältst du \emph{+2 Gefallen}. Du kannst diese sofort einlösen, z.B. für die Anweisung auf dem \emph{STADTSTAAT}.}
\end{tikzpicture}
\hspace{-0.6cm}
\begin{tikzpicture}
	\card
	\cardstrip
	\cardbanner{banner/gold.png}
	\cardicon{icons/coin.png}
	\cardprice{2}
	\cardtitle{Tand}
	\cardcontent{Wähle zwei der vier Optionen. Die ersten drei Optionen sind einfache \enquote{\emph{+1}}-Anweisungen, die letzte Option ist der komplette Rest. Du könntest also z.B. die Option \enquote{\emph{+1 Kauf}} wählen und die Option \enquote{wenn du in diesem Zug eine Karte nimmst, darfst du sie auf deinen Nachziehstapel legen.}}
\end{tikzpicture}
\hspace{-0.6cm}
\begin{tikzpicture}
	\card
	\cardstrip
	\cardbanner{banner/white.png}
	\cardicon{icons/coin.png}
	\cardprice{3}
	\cardtitle{\footnotesize{Händlerlager}}
	\cardcontent{Hast du mehrere \emph{HÄNDLERLAGER} im Spiel, darfst du wählen, wie viele du auf deinen Nachziehstapel legen möchtest.}
\end{tikzpicture}
\hspace{-0.6cm}
\begin{tikzpicture}
	\card
	\cardstrip
	\cardbanner{banner/orange.png}
	\cardicon{icons/coin.png}
	\cardprice{3}
	\cardtitle{Importeurin}
	\cardcontent{Zu Beginn des Spiels erhält jeder Spieler fünf Gefallen-Marker statt einem. Die \emph{IMPORTEURIN} bietet keine Möglichkeit, während des Spiels mehr Gefallen zu erhalten.}
\end{tikzpicture}
\hspace{-0.6cm}
\begin{tikzpicture}
	\card
	\cardstrip
	\cardbanner{banner/white.png}
	\cardicon{icons/coin.png}
	\cardprice{3}
	\cardtitle{\footnotesize{Untergebener}}
	\cardcontent{Wenn du diese Karte spielst, erhältst du einfach \emph{+1 Karte}, \emph{+1 Aktion} und \emph{+1 Gefallen.}}
\end{tikzpicture}
\hspace{-0.6cm}
\begin{tikzpicture}
	\card
	\cardstrip
	\cardbanner{banner/white.png}
	\cardicon{icons/coin.png}
	\cardprice{3}
	\cardtitle{Wächter}
	\cardcontent{Mische, falls es nötig ist. Wenn du selbst nach dem Mischen weniger als fünf Karten hast, schau sie alle an.}
\end{tikzpicture}
\hspace{-0.6cm}
\begin{tikzpicture}
	\card
	\cardstrip
	\cardbanner{banner/white.png}
	\cardicon{icons/coin.png}
	\cardprice{4}
	\cardtitle{Botin}
	\cardcontent{Ist dein Nachziehstapel leer, mische erst. Nachdem du abgelegt hast, darfst du ausnahmsweise tun, was sonst nicht erlaubt ist: deinen Ablagestapel durchsehen. Wenn im Ablagestapel keine Aktions- oder Geldkarte enthalten ist, verfällt die Anweisung. Du bist nicht verpflichtet, eine spielbare Karte zu spielen.}
\end{tikzpicture}
\hspace{-0.6cm}
\begin{tikzpicture}
	\card
	\cardstrip
	\cardbanner{banner/orange.png}
	\cardicon{icons/coin.png}
	\cardprice{4}
	\cardtitle{\tiny{Königliche Galeere}}
	\cardcontent{Es ist optional, hiermit eine Aktionskarte zu spielen, die keine Dauerkarte ist. Spielst du eine, führst du die Karte komplett aus und legst sie dann zur Seite. Ist die Karte irgendwohin bewegt worden (z.B. wenn sie sich selbst entsorgt hat), kannst du sie nicht zur Seite legen und die \emph{KÖNIGLICHE GALEERE} wird in diesem Zug normal abgelegt. Legst du die Karte zur Seite, bleibt die \emph{KÖNIGLICHE GALEERE} während dieses Zuges im Spiel, und zu Beginn deines nächsten Zuges spielst du die Karte erneut. Die \emph{KÖNIGLICHE GALEERE} und die zur Seite gelegte Karte werden dann beide in diesem nächsten Zug abgelegt.\\
	Eine Karte durch die \emph{KÖNIGLICHE GALEERE} zu spielen verbraucht keine Aktion, auch wenn das Spielen der \emph{KÖNIGLICHEN GALEERE} selbst dies tut.}
\end{tikzpicture}
\hspace{-0.6cm}
\begin{tikzpicture}
	\card
	\cardstrip
	\cardbanner{banner/white.png}
	\cardicon{icons/coin.png}
	\cardprice{4}
	\cardtitle{Mittelsmann}
	\cardcontent{Wenn du z.B. ein \emph{ANWESEN} entsorgst, das \coin[2] kostet, kannst du wählen zwischen \emph{+2 Karten} oder \emph{+2 Aktionen} oder +\coin[2] oder \emph{+2 Gefallen}. Wenn du eine Karte mit \potion oder \hex in ihren Kosten entsorgst (aus anderen \emph{DOMINION}-Erweiterungen), erhältst du nichts für diese Symbole.}
\end{tikzpicture}
\hspace{-0.6cm}
\begin{tikzpicture}
	\card
	\cardstrip
	\cardbanner{banner/white.png}
	\cardicon{icons/coin.png}
	\cardprice{4}
	\cardtitle{Ortschaft}
	\cardcontent{Du wählst einfach aus, ob du \emph{+1 Karte} und \emph{+2 Aktionen} erhalten möchtest oder \emph{+1 Kauf} und +\coin[2].}
\end{tikzpicture}
\hspace{-0.6cm}
\begin{tikzpicture}
	\card
	\cardstrip
	\cardbanner{banner/white.png}
	\cardicon{icons/coin.png}
	\cardprice{4}
	\cardtitle{Schreinerin}
	\cardcontent{Schaue zuerst, ob es leere Vorratsstapel gibt. Gibt es keine, erhältst du \emph{+1 Aktion} und nimmst eine Karte, die bis zu \coin[4] kostet. Gibt es einen oder mehrere leere Vorratsstapel, entsorgst du stattdessen eine Karte aus deiner Hand und nimmst eine Karte, die bis zu \coin[2] mehr kostet als die Karte, die du entsorgt hast.}
\end{tikzpicture}
\hspace{-0.6cm}
\begin{tikzpicture}
	\card
	\cardstrip
	\cardbanner{banner/white.png}
	\cardicon{icons/coin.png}
	\cardprice{4}
	\cardtitle{Wirtin}
	\cardcontent{Zuerst erhältst du \emph{+1 Aktion} und wählst dann eine der drei Optionen, danach führst du die gewählte Option aus. Entweder erhältst du \emph{+1 Karte}; oder \emph{+3 Karten} und du legst 3 Karten ab; oder \emph{+5 Karten} und du legst 6 Karten ab.}
\end{tikzpicture}
\hspace{-0.6cm}
\begin{tikzpicture}
	\card
	\cardstrip
	\cardbanner{banner/white.png}
	\cardicon{icons/coin.png}
	\cardprice{5}
	\cardtitle{Aufwiegler}
	\cardcontent{Falls du den \emph{AUFWIEGLER} spielst, gilt für den Rest des Zuges seine Anweisung. Jedes Mal, wenn du eine Angriffskarte nimmst, muss jeder Mitspieler Handkarten ablegen, bis er nur noch drei Karten auf der Hand hat.\\
	Deckst du einen \emph{BURGGRABEN} auf, wenn jemand den \emph{AUFWIEGLER} gespielt hat, bist du von dem Angriff nicht betroffen. Du kannst den \emph{BURGGRABEN} aber nicht erst dann aufdecken, wenn später eine Angriffskarte genommen wird.}
\end{tikzpicture}
\hspace{-0.6cm}
\begin{tikzpicture}
	\card
	\cardstrip
	\cardbanner{banner/white.png}
	\cardicon{icons/coin.png}
	\cardprice{5}
	\cardtitle{Barbar}
	\cardcontent{Wenn du z.B. einen \emph{VERTRAG} hiermit entsorgst, könntest du eine \emph{KÖNIGLICHE GALEERE} nehmen, weil beides Dauerkarten sind, oder ein \emph{SILBER}, weil beides Geldkarten sind, oder einen \emph{SCHMEICHLER}, weil beides Kontaktkarten sind.\\
	Kostet die entsorgte Karte \coin[3] oder mehr, musst du eine billigere Karte nehmen, wenn du kannst. Gibt es keine billigeren Karten mit einen gemeinsamen Kartentyp, nimmst du einfach keine Karte. Der Angriff trifft die Mitspieler in Zugreihenfolge, beginnend mit dem Spider links von dir - das kann wichtig sein.}
\end{tikzpicture}
\hspace{-0.6cm}
\begin{tikzpicture}
	\card
	\cardstrip
	\cardbanner{banner/white.png}
	\cardicon{icons/coin.png}
	\cardprice{5}
	\cardtitle{\scriptsize{Gildemeisterin}}
	\cardcontent{Wind eine Fähigkeit eines Verbündeten durch das Nehmen von Karten ausgelöst (z.B. \emph{NOMADENSTAMM}), kannst du dafür den Gefallen benutzen, den du gerade erhalten hast.}
\end{tikzpicture}
\hspace{-0.6cm}
\begin{tikzpicture}
	\card
	\cardstrip
	\cardbanner{banner/white.png}
	\cardicon{icons/coin.png}
	\cardprice{5}
	\cardtitle{Hauptstadt}
	\cardcontent{Ziehe zuerst eine Karte und erhalte \emph{+2 Aktionen}. Entscheide dann, ob du für +\coin[2] zwei Karten ablegen möchtest. Du darfst dich auch für das Ablegen entscheiden, wenn du weniger als zwei Karten auf der Hand hast. Dann legst du alle Handkarten ab, die du hast, aber du erhältst nur +\coin[2] wenn du tatsächlich zwei Karten abgelegt hast. Entscheide dann, ob du \coin[2] für \emph{+2 Karten} bezahlen möchtest.\\
	Die \coin[2] kannst du durch das Ablegen mit der \emph{HAUPTSTADT} erhalten haben oder auf eine andere Art und Weise, wie z.B. durch einen \emph{BARBAR}, den du vorher im Zug gespielt hast. Du darfst aber keine Geldkarten spielen, um die \coin[2] zu bezahlen.}
\end{tikzpicture}
\hspace{-0.6cm}
\begin{tikzpicture}
	\card
	\cardstrip
	\cardbanner{banner/white.png}
	\cardicon{icons/coin.png}
	\cardprice{5}
	\cardtitle{Jägerin}
	\cardcontent{Wähle von den drei Karten eine Aktionskarte, dann eine Geldkarte und dann eine Punktekarte. Karten mit mehreren Kartentypen kannst du für einen beliebigen, passenden Typ wählen. Sind die aufgedeckten Karten z.B. \emph{BURG}, \emph{TAND} und \emph{SILBER}, müsstest du die \emph{BURG} als Aktionskarte nehmen, könntest zwischen \emph{SILBER} und \emph{TAND} als Geldkarte wählen und würdest keine Punktekarte auf die Hand nehmen. Dann würdest du die Geldkarte ablegen, die du nicht gewählt hast.}
\end{tikzpicture}
\hspace{-0.6cm}
\begin{tikzpicture}
	\card
	\cardstrip
	\cardbanner{banner/white.png}
	\cardicon{icons/coin.png}
	\cardprice{5}
	\cardtitle{Markthalle}
	\cardcontent{Es kommt daraut an, wie viel die Karte tatsächlich in dem Moment kostet, in dem du sie nimmst. Kosten Karten z.B. \coin[1] weniger durch die \emph{BRÜCKE} (aus \emph{Intrige}), führt das Nehmen eines \emph{SILBERS} nicht zu \emph{+1 Kauf}, das Nehmen eines \emph{HERZOGTUMS} aber schon.}
\end{tikzpicture}
\hspace{-0.6cm}
\begin{tikzpicture}
	\card
	\cardstrip
	\cardbanner{banner/white.png}
	\cardicon{icons/coin.png}
	\cardprice{5}
	\cardtitle{Spezialistin}
	\cardcontent{Zuerst darfst du eine Aktions- oder Geldkarte aus deiner Hand spielen. Hast du das getan, wählst du nach der kompletten Ausführung des Spielens der Karte, ob du sie noch einmal spielen möchtest oder eine gleiche Karte nehmen möchtest. Du kannst die Karte sogar dann noch einmal spielen, wenn sie nicht mehr im Spiel ist. Du kannst dich auch dann dafür entscheiden, eine gleiche Karte zu nehmen, wenn es keine gleichen Karten mehr zu nehmen gibt. Dann nimmst du nichts. Durch die \emph{SPEZIALISTIN} kannst du nur Karten aus dem Vorrat nehmen.}
\end{tikzpicture}
\hspace{-0.6cm}
\begin{tikzpicture}
	\card
	\cardstrip
	\cardbanner{banner/white.png}
	\cardicon{icons/coin.png}
	\cardprice{5}
	\cardtitle{Tausch}
	\cardcontent{Zuerst erhaltst du \emph{+1 Karte} und \emph{+1 Aktion}. Dann darfst du optional eine Aktionskarte aus deiner Hand auf ihren Stapel legen. Wenn du dies tust, nimm eine Aktionskarte aus dem Vorrat auf deine Hand, die bis zu \coin[5] kostet. Die Karte, die du nimmst, darf nicht die gleiche Karte sein wie die Karte, die du zurückgelegt hast.\\
	Die Karte zurückzulegen ist kein Entsorgen und löst keine \enquote{Wenn du eine Karte entsorgst}- Anweisungen aus. Die Karte zu nehmen ist aber ein Nehmen und löst \enquote{Wenn du diese Karte nimmst}-Anweisung aus.}
\end{tikzpicture}
\hspace{-0.6cm}
\begin{tikzpicture}
	\card
	\cardstrip
	\cardbanner{banner/white.png}
	\cardicon{icons/coin.png}
	\cardprice{5}
	\cardtitle{\footnotesize{Umgestaltung}}
	\cardcontent{Entsorge zuerst eine Karte aus deiner Hand. Dann wähle zwischen \emph{+1 Karte} und \emph{+1 Aktion} oder ob du eine Karte nehmen möchtest, die bis zu \coin[2] mehr kostet als die entsorgte Karte.}
\end{tikzpicture}
\hspace{-0.6cm}
\begin{tikzpicture}
	\card
	\cardstrip
	\cardbanner{banner/white.png}
	\cardicon{icons/coin.png}
	\cardprice{5}
	\cardtitle{\scriptsize{Unterhändlerin}}
	\cardcontent{Ziehe zuerst drei Karten und schaue dann, ob du für das Ziehen der drei Karten mischen musstest. Wenn du gemischt hast, erhältst du \emph{+1 Aktion} und \emph{+2 Gefallen}. Das Mischen zählt nur dann als Mischen, wenn mindestens eine Karte in deinem Ablagestapel war.}
\end{tikzpicture}
\hspace{-0.6cm}
\begin{tikzpicture}
	\card
	\cardstrip
	\cardbanner{banner/goldorange.png}
	\cardicon{icons/coin.png}
	\cardprice{5}
	\cardtitle{Vertrag}
	\cardcontent{Wenn du eine Karte zur Seite legst, bleibt der \emph{VERTRAG} bis zur Aufräumphase deines nächsten Zuges im Spiel. Legst du keine Karte zur Seite, wird der \emph{VERTRAG} in der Aufräumphase dieses Zuges abgelegt. Legst du eine Karte zur Seite, musst du sie zu Beginn deines nächsten Zuges spielen. Die zur Seite gelegte Karte liegt offen.}
\end{tikzpicture}
\hspace{-0.6cm}
\begin{tikzpicture}
	\card
	\cardstrip
	\cardbanner{banner/orange.png}
	\cardicon{icons/coin.png}
	\cardprice{5}
	\cardtitle{Wegelagerer}
	\cardcontent{\tiny{Du ziehst auch dann drei Karten, wenn der \emph{WEGELAGERER} nicht aus dem Spiel abgelegt werden kann. Wenn du z.B. den \emph{THRONSAAL} (aus dem \emph{Basisspiel}) auf den \emph{WEGELAGERER} anwendest, legst du ihn nur einmal ab, ziehst aber trotzdem sechs Karten. Das Ablegen des \emph{WEGELAGERERS} geschieht zuerst, es ist also sogar möglich, diesen \emph{WEGELAGERER} durch den Effekt der \emph{+3 Karten} zu ziehen.\\
	Der Angriff sorgt dafür, dass in allen Zügen der Mitspieler die jeweils zuerst gespielte Geldkarte keine Auswirkungen hat. Bei Extrazügen gilt dies auch für jeden einzelnen Extrazug. Ist die zuerst gespielte Geldkarte eines Mitspielers z.B. ein \emph{KUPFER}, bringt dies kein \coin. Dies wirkt nicht kumulativ: Spielen mehrere Spieler hintereinander einen \emph{WEGELAGERER} oder ein Spieler mehrere \emph{WEGELAGERER}, hat trotzdem nur jeweils eine einzige Geldkarte in allen Zügen der Mitspieler keine Auswirkungen.\\
	Die Geldkarte hat sogar dann keine Auswirkungen, wenn sie gleichzeitig eine Aktionskarte ist (wie \emph{KRONE} aus \emph{Empires}). Dadurch bewirkt die Geldkarte nicht das, was sie normalerweise beim Ausspielen bewirkt, es verhindert aber nicht die Anweisungen unter der Trennlinie (wie beim \emph{VERMÖGEN} aus \emph{Empires}).\\
	Ist die Geldkarte gleichzeitig eine Aktionskarte, kann ein Weg (aus \emph{Menagerie}) trotzdem auf sie angewandt werden und eine \emph{ZAUBERIN} (aus \emph{Empires}) kann sie trotzdem betreffen. Der Spieler, der die Geldkarte gespielt hat, entscheidet, welche Anweisung zutrifft.}}
\end{tikzpicture}
\hspace{-0.6cm}
\begin{tikzpicture}
	\card
	\cardstrip
	\cardbanner{banner/white.png}
	\cardicon{icons/coin.png}
	\cardprice{6}
	\cardtitle{Marquis}
	\cardcontent{Auch wenn du nicht die komplette Menge an Karten ziehen konntest, legst du trotzdem danach Handkarten ab, bis du nur noch 10 Karten auf der Hand hast.}
\end{tikzpicture}
\hspace{-0.6cm}
\begin{tikzpicture}
	\card
	\cardstrip
	\cardbanner{banner/white.png}
	\cardicon{icons/coin.png}
	\cardprice{\scriptsize{3-6}}
	\cardtitle{Augurinnen}
	\cardcontent{\emph{Kräutersammlerin:} Dass du die Karten deines Nachziehstapels auf deinen Ablagestapel legst, löst keine \enquote{Wenn du diese Karte ablegst}-Anweisungen aus wie beim \emph{TUNNEL} (aus \emph{Hinterland}). Eine Geldkarte von deinem Nachziehstapel zu spielen, ist optional, genauso wie das Rotieren der Augurinnen. 
	
	\medskip

	\emph{Altardienerin:} Beide Anweisungen sind optional: Du kannst eine von beiden ausführen oder beide oder keine von beiden. Du erhältst nur dann ein \emph{GOLD}, wenn du tatsächlich eine Aktions- oder Punktekarte aus deiner Hand entsorgt hast. Du nimmst nur dann eine Augurinnenkarte, wenn du tatsächlich die \emph{ALTARDIENERIN} entsorgt hast. Wenn du hierdurch eine Augurinnenkarte nimmst, nimmst du die aktuell oben liegende Karte der Augurinnen, auch wenn das eine weitere \emph{ALTARDIENERIN} ist. 
	
	\medskip
	
	\emph{Meisterhexe:} Nenne eine Karte und decke die oberste Karte deines Nachziehstapels auf. Ist sie die genannte, nimmt jeder Mitspieler einen \emph{FLUCH}. Du nimmst die Karte unabhängig davon auf deine Hand, ob sie die von dir genannte ist oder nicht. 
	
	\medskip

	\emph{Prophetin:} Wenn dein Nachziehstapel nach dem Ziehen leer ist, wird die erste Karte, die du zurücklegst, zur obersten Karte des Stapels.}
\end{tikzpicture}
\hspace{-0.6cm}
\begin{tikzpicture}
	\card
	\cardstrip
	\cardbanner{banner/white.png}
	\cardicon{icons/coin.png}
	\cardprice{\scriptsize{3-6}}
	\cardtitle{Bastionen}
	\cardcontent{\tiny{\begin{Spacing}{1}
	\emph{Zelt:} Wenn du mehrere \emph{ZELTE} im Spiel hast, darfst du entscheiden, wie viele davon du auf deinen Nachziehstapel legen möchtest. 
	
	\medskip
	
	\emph{Garnison:} Auf dieser Karte können nur Marker liegen, wenn sie im Spiel ist. Wenn sie aus dem Spiel geht, liegen keine Marker mehr darauf. Du kannst die Marker verwenden, die du sonst bei den Kontaktkarten nutzt, um Gefallen zu erhalten. Hier haben die Marker aber nur die Bedeutung, die auf der Karte \emph{GARNISON} beschrieben ist. Wenn du den \emph{THRONSAAL} auf die \emph{GARNISON} anwendest und dann 3 Karten nimmst, legst du insgesamt 6 Marker darauf ab. Und im nächsten Zug ziehst du 6 Karten, nicht 12, denn du kannst die Marker nur einmal entfernen. 
	
	\medskip

	\emph{Hügelfort:} Beende zuerst vollständig das Nehmen einer Karte, die bis zu \coin[4] kostet. Wähle dann aus, ob du sie deinen Handkarten hinzufügen oder stattdessen \emph{+1 Karte} und \emph{+1 Aktion} erhalten möchtest. Wenn die Karte sich nicht mehr dort befindet, wohin du sie genommen hast (normalerweise dein Ablagestapel), kannst du sie nicht auf die Hand nehmen, falls du diese Option wählst. Wurde die Karte in deinem Ablagestapel verdeckt, kannst du sie trotzdem auf die Hand nehmen. 
	
	\medskip

	\emph{Burg:} Wenn du +\coin[3] wählst, wird die \emph{BURG} in diesem Zug abgelegt. Wenn du die \emph{+3 Karten} im nächsten Zug wählst, legst du die \emph{BURG} zur Seite bis zur Aufräumphase dieses nächsten Zuges (das passiert auch, wenn du beides mit der \emph{ÄLTESTEN} wählst).
	\end{Spacing}}}
\end{tikzpicture}
\hspace{-0.6cm}
\begin{tikzpicture}
	\card
	\cardstrip
	\cardbanner{banner/white.png}
	\cardicon{icons/coin.png}
	\cardprice{\scriptsize{2-5}}
	\cardtitle{Bürger}
	\cardcontent{\tiny{\begin{Spacing}{1}
	\emph{Ausruferin:} Wähle zuerst, ob du +\coin[2] erhalten möchtest oder ein \emph{SILBER} oder \emph{+1 Karte} und \emph{+1 Aktion}. Unabhängig davon, was du gewählt hast, darfst du dich danach entscheiden, ob du die Bürger rotierst oder nicht. \\

	\emph{Eisenschmied:} Du ziehst entweder Karten, bis du 6 Karten auf der Hand hast; oder du ziehst 2 Karten; oder du ziehst eine Karte und erhältst \emph{+1 Aktion}. \\

	\emph{Müller:} Wenn du (nach dem Mischen) weniger als 4 Karten in deinem Nachziehstapel hast, siehst du alle Karten deines Nachziehstapels an. \\

	\emph{Älteste:} Du darfst auch eine Aktionskarte spielen, die dich nicht aus verschiedenen Optionen wählen lässt. Du führst sie einfach normal aus. Spielst du aber eine Karte mit einer Anweisung, die dich aus verschiedenen Optionen wählen lässt, darfst du optional eine zusätzliche Option wählen.\\
	Wenn du z.B. den \emph{GRAF} (aus \emph{Dark Ages}) spielst, kannst du dich entscheiden, nur eine Option der oberen \enquote{Wähle eins}-Anweisung zu wählen, aber zwei Optionen der unteren \enquote{Wähle eins}-Option. Wählst du mehrere Optionen, führst du deren Anweisungen in der Reihenfolge auf der Karte aus.\\
	Wenn du z.B. mit einer \emph{ÄLTESTEN} einen \emph{EISENSCHMIED} spielst und dich entscheidest, die Anweisungen \enquote{Zieh, bis du 6 Karten auf der Hand hast,} und \enquote{\emph{+1 Karte} und \emph{+1 Aktion}} zu nutzen, ziehst du erst, bis du 6 Karten auf deiner Hand hast, und erhältst dann \emph{+1 Karte} und \emph{+1 Aktion}.\\
	Wenn du z.B. mit einer \emph{ÄLTESTEN} einen \emph{HÖFLING} (aus \emph{Intrige}) spielst, wählst du eine zusätzliche Option, aber nicht eine zusätzliche Option pro Kartentyp.\\
	Die \emph{ÄLTESTE} betrifft nicht alle Wahlmöglichkeiten auf Karten, bei denen etwas zu wählen ist, nur die Auswahl zwischen mehreren Anweisungen. Die \emph{WERKSTATT} z.B. gibt dir die Wahl, welche Karte du nimmst. Wenn du mit einer \emph{ÄLTESTEN} die \emph{WERKSTATT} spielst, bewirkt die \emph{ÄLTESTE} nichts darüber hinaus.
	\end{Spacing}}}
\end{tikzpicture}
\hspace{-0.6cm}
\begin{tikzpicture}
	\card
	\cardstrip
	\cardbanner{banner/white.png}
	\cardicon{icons/coin.png}
	\cardprice{\scriptsize{3-6}}
	\cardtitle{Irrfahrten}
	\cardcontent{\emph{Alte Landkarte:} Alle Anweisungen werden in der aufgelisteten Reihenfolge ausgeführt. Zuerst erhältst du \emph{+1 Karte}, dann \emph{+1 Aktion}. Danach legst du eine Karte ab und erhältst erneut \emph{+1 Karte}. Danach wählst du, ob du die Irrfahrten rotierst oder nicht. 
	
	\medskip

	\emph{Seereise:} Dies hält dich nicht davon ab, Karten zu spielen, die nicht auf deiner Hand sind. Wenn z.B. deine dritte gespielte Karte ein \emph{GOLEM} (aus \emph{Die Alchimisten}) ist, kannst du damit trotzdem die entsprechenden zwei Karten spielen, die zur Seite gelegt sind. Wenn du in einem Zug mit der \emph{SEEREISE} den \emph{THRONSAAL} auf eine andere Karte anwendest, zählen sowohl der \emph{THRONSAAL} als auch die andere Karte als aus deiner Hand gespielt, aber das erneute Spielen der Karte mit dem \emph{THRONSAAL} zählt nicht zusätzlich. Dies beschränkt das Spielen von allen Kartentypen, auch Geldkarten wie \emph{KUPFER}.
	
	\medskip

	\emph{Versunkener Schatz:} Gibt es im Vorrat keine solche Aktionskarte, nimmst du keine. 
	
	\medskip

	\emph{Ferne Küste:} Ein \emph{ANWESEN} zu nehmen ist nicht optional. Ist der \emph{ANWESEN}-Stapel leer, erhältst du trotzdem \emph{+2 Karten} und \emph{+1 Aktion}.}
\end{tikzpicture}
\hspace{-0.6cm}
\begin{tikzpicture}
	\card
	\cardstrip
	\cardbanner{banner/white.png}
	\cardicon{icons/coin.png}
	\cardprice{\scriptsize{3-6}}
	\cardtitle{Konflikte}
	\cardcontent{\tiny{\begin{Spacing}{1}
	\emph{Schlachtplan:} Zuerst erhältst du \emph{+1 Karte} und \emph{+1 Aktion}, dann darfst du eine Angriffskarte aus deiner Hand aufdecken, um eine Karte zu ziehen, und dann darfst du einen beliebigen Vorratsstapel rotieren.\\
	Bei vielen Stapeln macht das Rotieren keinen sinnvollen Unterschied. Es kann aber für gemischte Stapel wichtig sein oder für die Schlösser (aus \emph{Empires}) oder Ruinen und Ritter (aus \emph{Dark Ages}). 
		
	\smallskip

	\emph{Bogenschützin:} Wenn es den Spielern wichtig ist, decken sie die Karten in Zugreihenfolge auf. Jeder Mitspieler, der 5 oder mehr Karten auf der Hand hat, wählt eine aus, die er geheim hält, und deckt die restlichen Karten auf. Du wählst jeweils eine der aufgedeckten Karten, die der betreffende Spieler ablegt. 
		
	\smallskip

	\emph{Kriegsherr:} Dies hält dich nicht davon ab, Karten zu spielen, die nicht auf deiner Hand sind. Z.B. kannst du mit \emph{GOLEM} (aus \emph{Die Alchimisten}) trotzdem die zwei Karten spielen, die zur Seite gelegt werden, unabhängig davon, wie viele gleiche Karten davon im Spiel sind.\\
	Betrifft der \emph{KRIEGSHERR} dich, kannst du mit dem \emph{THRONSAAL} keine Karte aus deiner Hand spielen, von der du zwei oder mehr gleiche Karten im Spiel hast. Aber du kannst mit dem \emph{THRONSAAL} eine Karte spielen, von der du nur ein Exemplar im Spiel hast, und dann diese Karte wieder spielen, auch wenn du dann zwei gleiche Karten im Spiel hast. Dies betrifft nur Aktionskarten. Es betrifft z.B. nicht \emph{KUPFER}. 
		
	\smallskip

	\emph{Territorium:} Sind z.B. in deinem Kartensatz drei \emph{ANWESEN}, eine \emph{PROVINZ} und ein \emph{TERRITORIUM} ist das \emph{TERRITORIUM} 3 \victorypoint wert. Wenn das Nehmen des \emph{TERRITORIUMS} dazu führt, dass der Konflikte-Stapel leer ist, zählt dies für die Menge \emph{GOLD}, die du erhältst.
	\end{Spacing}}}
\end{tikzpicture}
\hspace{-0.6cm}
\begin{tikzpicture}
	\card
	\cardstrip
	\cardbanner{banner/white.png}
	\cardicon{icons/coin.png}
	\cardprice{\scriptsize{3-6}}
	\cardtitle{Zauberer}
	\cardcontent{\tiny{\begin{Spacing}{1}\vspace{1em}
	\emph{Zauberschüler:} Das Rotieren der Zauberer ist optional, aber das Entsorgen einer Karte ist obligatorisch. Wenn du eine Geldkarte entsorgst, erhältst du \emph{+1 Gefallen} und legst den \emph{ZAUBERSCHÜLER} auf deinen Nachziehstapel, dies ist obligatorisch. Das bedeutet, du kannst denselben \emph{ZAUBERSCHÜLER} in diesem Zug wieder ziehen und ihn wieder spielen. \\
	Wenn du eine Nicht-Geldkarte entsorgst, bleibt der \emph{ZAUBERSCHÜLER} im Spiel und wird in der Aufräumphase wie andere Karten abgelegt. \\

	\emph{Beschwörer:} Diese Karte kehrt in jedem Zug auf deine Hand zurück, solange du sie weiterhin spielst. \\

	\emph{Hexenmeister:} Jeder Mitspieler nennt eine Karte und deckt die oberste Karte seines Nachziehstapels auf. Jeder Spieler, bei dem die aufgedeckte Karte nicht die von ihm genannte ist, nimmt einen \emph{FLUCH}. Unabhängig davon, ob dies der Fall ist, legt jeder Mitspieler die aufgedeckte Karte wieder auf seinen Nachziehstapel. Wenn du also den \emph{HEXENMEISTER} zweimal in einem Zug spielst, kennen die Mitspieler wahrscheinlich beim zweiten Spielen die Karte. \\

	\emph{Lich:} Einen Zug aussetzen bedeutet, dass du beim nächsten Mal, wenn du einen Zug machen würdest, keinen machst. In dem Zug passiert dann nichts: keine \enquote{Zu Beginn deines Zuges}-Anweisungen und keine Phasen. Das Spiel wird wie üblich mit dem Spieler links von dir fortgeführt. Du kannst einen zusätzlichen Zug aussetzen, wie z.B. einen von der \emph{SEEREISE}. Ausgesetzte Züge zählen trotzdem für die Entscheidung bei Gleichstand, als ob der Spieler nicht ausgesetzt hätte. Spielst du mehrere \emph{LICH}, setzt du mehrere Züge aus.\\
	Wenn du einen \emph{LICH} entsorgst, legst du ihn aus dem Müll auf deinen Ablagestapel, was keine Anweisungen für das Nehmen von Karten auslöst. Dann nimmst du eine Karte, die weniger als ein \emph{LICH} kostet, aus dem Müll - dies löst Anweisungen für das Nehmen von Karten aus. Du musst eine billigere Karte nehmen, falls möglich.
	\end{Spacing}}}
\end{tikzpicture}
\hspace{-0.6cm}
\begin{tikzpicture}
	\card
	\cardstrip
	\cardbanner{banner/white.png}
	\cardtitle{Verbündete (1/5)\qquad}
	\cardcontent{\tiny{\begin{Spacing}{1}
	\emph{Architektengilde:} Dies wirkt nur einmal pro Nehmen, sie kann sich aber selbst auslösen: Du könntest z.B. eine \emph{PROVINZ} nehmen, zwei Gefallen einlösen, um ein \emph{GOLD} zu nehmen (billiger als die \emph{PROVINZ}), und dann 2 Gefallen einlösen, um ein \emph{LABORATORIUM} zu nehmen (billiger als \emph{GOLD}). 
	
	\smallskip

	\emph{Bankiersverbund:} Du löst hierfür keine Gefallen ein. Du erhältst nur +\coin je nach der Anzahl an Gefallen, die auf deinem Gefallen-Tableau liegen. 
	
	\smallskip

	\emph{Bergvolk:} Du brauchst die kompletten 5 Gefallen, um dies zu nutzen. 
	
	\smallskip

	\emph{Bund der Ladenbesitzer:} Du löst hierfür keine Gefallen ein. Nach jedem Spielen einer Kontaktkarte erhältst du +\coin[1], wenn du 5 oder mehr Gefallen hast, und zusätzlich dazu \emph{+1 Aktion} und \emph{+1 Kauf}, wenn du 10 oder mehr Gefallen hast. In Partien mit mehreren Kontaktkarten gilt dieser Bonus für alle Kontaktkarten, auch wenn nur eine von ihnen benutzt wurde, um die Gefallen zu erhalten. 
	
	\smallskip

	\emph{Erfinderfamilie:} Hiermit darfst du keine Marker auf Punktestapel legen. Du darfst aber hiermit Marker auf gemischte Stapel legen, die Punktekarten enthalten, falls die Platzhalterkarte keine Punktekarte ist. Dies bedeutet, dass du hiermit Marker auf die 6 gemischten Stapel aus \emph{Verbündete} legen kannst, aber nicht auf die Schlösser aus \emph{Empires}. Die Wirkung ist kumulativ. Zwei Marker auf einem Stapel bedeuten, dass die Karten dieses Stapels \coin[2] weniger kosten. Dies reduziert die Kosten nicht unter \coin[0]. Hierdurch kosten Karten immer für alle Spieler weniger, nicht nur für den Spieler, der den Marker legt.
	\end{Spacing}}}
\end{tikzpicture}
\hspace{-0.6cm}
\begin{tikzpicture}
	\card
	\cardstrip
	\cardbanner{banner/white.png}
	\cardtitle{Verbündete (2/5)\qquad}
	\cardcontent{\tiny{\begin{Spacing}{1}\vspace{1em}
	\emph{Freimaurerloge:} Immer wenn du mischst, kannst du Gefallen einlösen, um die Karten anzuschauen und bis zu zwei Karten pro Gefallen zu wählen, die du auf deinen Ablagestapel legst. Mische die übrigen Karten wie üblich, aber mische diese Karten nicht ein. Du darfst deine Karten nicht anschauen, wenn du nicht mindestens einen Gefallen einlöst.\\
	Bevor du dich entscheidest, ob du Gefallen einlöst, kannst du die restlichen Karten des Nachziehstapels anschauen, die du vor dem Mischen ziehen wirst. Nachdem du eines Gefallen eingelöst und die Karten angeschaut hast, darfst du immer noch weitere Gefallen einlösen.\\
	Die \emph{UNTERHÄNDLERIN} und der \emph{UNTERGEBENE} können dazu führen, dass du mischen musst, bevor du von ihnen Gefallen erhältst. Die Gefallen, die du noch nicht hast, können für dieses Mischen nicht verwendet werden. 
	
	\smallskip

	\emph{Friedlicher Kult:} Löse die beliebig vielen Gefallen alle auf einmal ein. Wähle dann alle Karten aus, die du entsorgen möchtest und entsorge sie. Führe dann die Dinge in beliebiger Reihenfolge aus, die durch das Entsorgen der Karten ausgelöst werden. 

	\smallskip

	\emph{Gefolgschaft der Schreiber:} Du kannst dies nur einmal pro Spielen einer Aktionskarte machen. Führe die Aktionskarte vollständig aus. Wenn du dann weniger als 4 Karten au deiner Hand hast, darfst du einen Gefallen für \emph{+1 Karte} einlösen. 
	
	\smallskip

	\emph{Handwerkergilde:} Die Karte nimmst du direkt auf deinen Nachziehstapel.
	
	\smallskip
	
	\emph{Hexenbund:} Nachdem du das Spielen einer Kontaktkarte vollständig ausgeführt hast, darfst du 3 Gefallen einlösen, damit jeder Mitspieler einen \emph{FLUCH} nimmt. Dies ist auch mit Gefallen möglich, die du gerade eben erst durch das Spielen dieser Kontaktkarte erhalten hast. Dies gilt nicht als Spielen einer Angriffskarte und kann nicht mit einem \emph{BURGGRABEN} abgewehrt werden.
	\end{Spacing}}}
\end{tikzpicture}
\hspace{-0.6cm}
\begin{tikzpicture}
	\card
	\cardstrip
	\cardbanner{banner/white.png}
	\cardtitle{Verbündete (3/5)\qquad}
	\cardcontent{\tiny{\begin{Spacing}{1}\vspace{1em}
	\emph{Höhlenbewohner:} Zu Beginn deines Zuges darfst du einen Gefallen einlösen. Wenn du das tust, legst du eine Karte ab und ziehst dann eine Karte. Dann darfst du einen weiteren Gefallen einlösen, um eine weitere Karte abzulegen und eine weitere Karte zu ziehen, und so weiter, bis du aufhörst. Gefallen einzulösen.\\
	Du ziehst auch dann eine Karte, wenn du keine ablegen konntest. 
		
	\smallskip

	\emph{Höhlenhafen:} Du könntest z.B. zwei Gefallen einlösen, um ein \emph{KUPFER} und ein \emph{SILBER} auf der Hand zu behalten, deine restlichen Handkarten und alle deine weiteren Karten aus dem Spiel ablegen (wie üblich), dann eine neue Hand mit 5 Karten ziehen und sie zu \emph{KUPFER} und \emph{SILBER} hinzufügen. Wenn du aus irgendeinem Grund nicht 5 Karten ziehst (z.B. durch einen \emph{AUSSENPOSTEN} aus \emph{Seaside}) ändert der \emph{HÖHLENHAFEN} dies nicht. Du ziehst weiterhin so viele Karten, wie du normalerweise hättest ziehen müssen, die einbehaltenen Karten werden dabei nicht angerechnet. 
		
	\smallskip

	\emph{Holzarbeitergilde:} Hiermit kannst du eine Aktionskarte mit beliebigen Kosten nehmen, auch Aktionskarten mit \potion oder \hex in den Kosten. Du nimmst nur eine Aktionskarte, wenn du eine entsorgt hast. 
		
	\smallskip

	\emph{Hütte der Fallensteller:} Ist dein Nachziehstapel leer, wird die Karte zur einzigen Karte in deinem Nachziehstapel.
	
	\smallskip
	
	\emph{Inselvolk:} Dies kann nie dazu führen, dass du zwei \emph{INSELVOLK}-Züge in Folge machst.
		
	\smallskip

	\emph{Marktstädte:} Führe zuerst das Spielen der Aktionskarte vollständig aus, bevor du entscheidest, ob du einen Gefallen einlösen möchtest, um eine weitere zu spielen.
	\end{Spacing}}}
\end{tikzpicture}
\hspace{-0.6cm}
\begin{tikzpicture}
	\card
	\cardstrip
	\cardbanner{banner/white.png}
	\cardtitle{Verbündete (4/5)\qquad}
	\cardcontent{\tiny{\begin{Spacing}{1}
	\emph{Nomadenstamm:} Es kommt darauf an, wie viel die Karte in dem Moment kostet, in dem du sie nimmst, nicht darauf, wie viel sie normalerweise kostet.\\
	Diese Karte wirkt nur einmal pro Nehmen.\\
	Du kannst einen Gefallen einlösen und hast dann die Wahl zwischen \emph{+1 Karte}, \emph{+1 Aktion} oder \emph{+1 Kauf}. 
		
	\smallskip

	\emph{Schäfer der Hochebene:} Wenn du z.B. fünf Gefallen, zwei \emph{ANWESEN} und einen \emph{BURGGRABEN} hast, kannst du drei Paare bilden, das ergibt 6\victorypoint. 
		
	\smallskip

	\emph{Stadtstaat:} Wenn du in deiner Kaufphase eine Aktionskarte nimmst (wie zB. durch einen Kauf). kannst du sie mit Hilfe des \emph{STADTSTAATS} spielen. Wenn du durch sie \emph{+ Aktionen} erhältst, erlaubt dir der \emph{STADTSTAAT} nicht, mehr Aktionskarten in deiner Kaufphase zu spielen. Wenn sie dich Geldkarten ziehen lässt, kannst du jene nur spielen, wenn du in dieser Kaufphase (außer mit einem als Geldkarte gespielten \emph{SCHWARZMARKT}, z.B. durch den \emph{KAPITALISMUS} (aus \emph{Renaissance})) noch nichts gekauft hast. Mit dem \emph{STADTSTAAT} kannst du nur eine Karte spielen, die sich immer noch dort befindet, wohin sie genommen wurde (normalerweise der Ablagestapel) - auch wenn sie im Ablagestapel durch andere Karten verdeckt wurde.\\
	Der \emph{STADTSTAAT} wirkt nur während deiner Züge.
	\end{Spacing}}}
\end{tikzpicture}
\hspace{-0.6cm}
\begin{tikzpicture}
	\card
	\cardstrip
	\cardbanner{banner/white.png}
	\cardtitle{Verbündete (5/5)\qquad}
	\cardcontent{\tiny{\begin{Spacing}{1}
	\emph{Sterndeuterorden:} Immer wenn du mischst, kannst du Gefallen einlösen, um die Karten anzuschauen und eine Karte pro eingelöstem Gefallen zu wählen, die du nach oben legst. Mische die übrigen Karten wie üblich.\\
	Du darfst deine Karten nicht anschauen, wenn du nicht mindestens einen Gefallen einlöst. Bevor du dich entscheidest, ob du Gefallen einlöst, kannst du die restlichen Karten des Nachziehstapels anschauen, die du vor dem Mischen ziehen wirst. Wenn du z.B. am Ende des Zuges mischst und zwei Karten übrig hattest, kannst du diese Karten anschauen, dann entscheiden, ob du Gefallen einlöst und welche Karten du nach oben legen möchtest.\\
	Nachdem du einen Gefallen eingelöst und die Karten angeschaut hast, darfst du weitere Gefallen einlösen.\\
	Die \emph{UNTERHÄNDLERIN} und der \emph{UNTERGEBENE} können dazu führen, dass du mischen musst, bevor du Gefallen erhältst. Die Gefallen, die du noch nicht hast, können für dieses Mischen nicht verwendet werden.
		
	\smallskip

	\emph{Taschendiebe:} Zu Beginn jedes deiner Züge entscheidest du, ob du einen Gefallen einlösen möchtest. Löst du keinen ein, legst du Handkarten ab, bis du nur noch 4 auf deiner Hand hast. Du brauchst keinen Gefallen einzulösen, wenn du schon 4 oder weniger Karten auf der Hand hattest.\\
	Dies ist keine gespielte Angriffskarte und kann nicht vom \emph{BURGGRABEN} abgewehrt werden. 
		
	\smallskip

	\emph{Waldbewohner:} Du kannst dies nur einmal pro Zug machen. 
		
	\smallskip

	\emph{Wüstenführer:} Nachdem du deine Handkarten abgelegt und fünf Karten gezogen hast, darfst du einen weiteren Gefallen einlösen, um es noch einmal zu tun, und dies beliebig oft wiederholen.
	\end{Spacing}}}
\end{tikzpicture}
\hspace{-0.6cm}
\begin{tikzpicture}
	\card
	\cardstrip
	\cardbanner{banner/white.png}
	\cardtitle{\footnotesize{Spielvorbereitung\qquad}}
	\cardcontent{\underline{\emph{Verbündete, Gefallen-Tableaus \& Marker}}\\
	Verwendet ihr mindestens eine Karte mit dem Kartentyp KONTAKT, nehmt ihr genau einen Verbündeten ins Spiel. Außerdem erhält jeder Spieler ein Gefallen-Tableau und je einen Marker für sein Tableau. Legt die restlichen Marker neben dem Vorrat bereit (siehe NEUE REGELN -> Verbündete, Kontaktkarten \& Gefallen). 

	\medskip

	\underline{\emph{Gemischte Stapel}}\\
	In \emph{DOMINION Verbündete} gibt es sechs gemischte Stapel mit jeweils vier verschiedenen Karten (à 4 Exemplaren). Zu Spielbeginn sind die Karten in jedem gemischten Stapel nach ihren Kosten sortiert, die kleinsten Kosten oben, die größten Kosten unten. Die Augurinnen enthalten z.B. zu Beginn 4 \emph{KRÄUTERSAMMLERINNEN} obenauf, darunter 4 \emph{ALTARDIENERINNEN}, dann 4 \emph{MEISTERHEXEN} und ganz unten 4 \emph{PROPHETINNEN} (siehe NEUE REGELN -> Gemischte Stapel).}
\end{tikzpicture}
\hspace{-0.6cm}
\begin{tikzpicture}
	\card
	\cardstrip
	\cardbanner{banner/white.png}
	\cardtitle{\footnotesize{Neue Regeln (1/3)}\qquad}
	\cardcontent{Es gelten die Basisspielregeln mit folgenden Änderungen:

	\medskip
	
	\emph{\underline{Dauerkarten}}\\
	In \emph{Verbündete} gibt es neun Dauerkarten (wie schon in \emph{Seaside} und den Erweiterungen ab \emph{Abenteuer}). Die orangefarbenen Dauerkarten beinhalten Anweisungen, die in späteren Zügen umgesetzt werden. Sie werden nicht in der Aufräumphase des Zuges abgelegt, in dem sie gespielt wurden, sondern bleiben bis zur Aufräumphase des Zuges, in dem die letzte Anweisung ausgeführt wird, im Spiel. Wird eine Dauerkarte mehrfach gespielt (z.B. durch die \emph{SPEZIALISTIN}), bleibt die verursachende Karte ebenfalls so lange im Spiel, bis die Dauerkarte abgelegt wird. Um anzuzeigen, dass eine Dauerkarte in der aktuellen Aufräumphase noch nicht abgelegt wird, wird sie in eine eigene Reihe oberhalb der restlichen gespielten Karten gelegt. Hast du mehrere Dauerkarten im Spiel, darfst du die Reihenfolge selbst bestimmen, in der du sie abhandelst.}
\end{tikzpicture}
\hspace{-0.6cm}
\begin{tikzpicture}
	\card
	\cardstrip
	\cardbanner{banner/white.png}
	\cardtitle{\footnotesize{Neue Regeln (2/3)}\qquad}
	\cardcontent{\tiny{\begin{Spacing}{1}\vspace{1em}
	\emph{\underline{Verbündete, Kontaktkarten \& Gefallen}}\\
	Verbündete sind jeweils nur 1x im Spiel enthalten. Sie sind keine Königreichkarten und erlauben es, Gefallen einzulösen. Kontaktkarten sind Königreichkarten, mit denen man Gefallen-Marker erhalten kann. Wenn ihr mit mindestens einer Karte spielt, die den Kartentyp KONTAKT hat (\emph{SCHMEICHLER}, \emph{TAND}, \emph{IMPORTEURIN}, \emph{UNTERGEBENER}, \emph{ZAUBERSCHÜLER}, \emph{MITTELSMANN}, \emph{GILDEMEISTERIN}, \emph{UNTERHÄNDLERIN}, \emph{VERTRAG}), verwendet genau einen beliebigen \emph{Verbündeten} für diese Partie. Sollten Kontaktkarten irgendwie anders am Spiel beteiligt sein, z.B. im \emph{SCHWARZMARKT}- Deck (von der Promo-Karte \emph{SCHWARZMARKT}) oder als Bannstapel der \emph{JUNGEN HEXE} (aus \emph{Reiche Ernte}), dann fügt auch genau einen Verbündeten zum Spiel hinzu (wenn ihr das nicht schon getan habt).\\
	Falls ihr die verwendeten Karten zufällig über die Platzhalterkarten bestimmt, seht ihr in den Typ-Zeilen der Platzhalterkarten, ob dort KONTAKT auftaucht. Ihr könnt weiterhin so viele \emph{Ereignisse} (aus \emph{Abenteuer}, \emph{Empires} und/oder \emph{Menagerie}), \emph{Landmarken} (aus \emph{Empires}), \emph{Projekte} (aus \emph{Renaissance}) und/oder Wege (aus \emph{Menagerie}) verwenden, wie ihr ohne Verbündete und Kontakte verwendet hättet.\\
	Die Marker repräsentieren die Gefallen. Sie werden auf ein Gefallen-Tableau gelegt, um sie von den Talern und Dorfbewohnern (aus anderen Erweiterungen) zu unterscheiden, die auf ihren eigenen Tableaus abgelegt werden. Erhältst du durch eine Karte \enquote{\emph{+1 Gefallen}}, lege einen Marker auf dein Gefallen-Tableau. Um einen Gefallen einzulösen, entferne einen Marker von deinem Gefallen-Tableau. \\
	Gefallen können erst ab dem ersten Zug des Spiels benutzt werden, nicht davor. Gefallen einzulösen ist immer optional und kann immer nur ein einziges Mal gemacht werden, wenn es von einem Verbündeten ausgelöst wird, außer auf der Karte steht \enquote{Wiederhole das beliebig oft.}
	\end{Spacing}}}
\end{tikzpicture}
\hspace{-0.6cm}
\begin{tikzpicture}
	\card
	\cardstrip
	\cardbanner{banner/white.png}
	\cardtitle{\footnotesize{Neue Regeln (3/3)}\qquad}
	\cardcontent{\tiny{\begin{Spacing}{1}\vspace{1em}
	\emph{\underline{Gemischte Stapel \& Rotieren}}\\
	In \emph{DOMINION Verbündete} gibt es sechs gemischte Stapel mit jeweils vier verschiedenen Karten (à 4 Exemplaren). Zu Spielbeginn werden die Karten in jedem gemischten Stapel nach ihren Kosten sortiert. Der Augurinnen-Stapel enthält z.B. zu Beginn 4 \emph{KRÄUTERSAMMLERINNEN} obenauf, darunter 4 \emph{ALTARDIENERINNEN}, dann 4 \emph{MEISTERHEXEN} und ganz unten 4 \emph{PROPHETINNEN}. Diese Sortierung kann sich im Spielverlauf ändern, z.B. durch \emph{TAUSCH}. Wie bei den gemischten Stapeln in \emph{DOMINION Empires} kann auch in dieser Erweiterung nur die jeweils oberste Karte der gemischten Stapel gekauft oder genommen werden. Du darfst die Karten der gemischten Stapel zu jeder Zeit durchschauen, ohne die Reihenfolge zu ändern. \\
	\emph{Rotieren:} Die zu Spielbeginn jeweils oberste Karte der gemischten Stapel har eine Anweisung, mit der man den Stapel (oder mit \emph{SCHLACHTPLAN}: einen beliebigen Stapel) \enquote{rotieren} kann. Einen Stapel zu rotieren bedeutet, die oberste Karte und alle mit ihr identischen Karten direkt darunter vom Stapel zu nehmen und sie unter den Stapel zu legen. Wenn z.B. drei \emph{KRÄUTERSAMMLERINNEN} als Oberstes im Augurinnen-Stapel liegen und darunter die \emph{ALTARDIENERINNEN}, legst du die drei \emph{KRÄUTERSAMMLERINNEN} nach unten und die \emph{ALTARDIENERINNEN} sind jetzt obenauf.\\
	Bei Anweisungen auf anderen Karten (oder Ereignissen usw.), die sich auf einen Stapel beziehen (z.B. den Kartentyp eines Stapels), ist die entsprechende Platzhalterkarte relevant. Bei Anweisungen, die sich auf eine konkrete Karte beziehen (z.B. Anweisungen, die die Kosten einer Karte reduzieren) ist die jeweilige Karte relevant. \emph{TRAINING} (aus \emph{Abenteuer}) z.B. lässt dich einen Marker auf einen Aktions-Vorratsstapel legen, wodurch du +\coin[1] erhältst, wenn du eine Karte von diesem Stapel spielst. Du kannst den Marker auf die \emph{Irrfahrten} legen, dann erhältst du auch +\coin[1], wenn du einen \emph{VERSUNKENEN SCHATZ} spielst.
	\end{Spacing}}}
\end{tikzpicture}
\hspace{-0.6cm}
\begin{tikzpicture}
	\card
	\cardstrip
	\cardbanner{banner/white.png}
	\cardtitle{\footnotesize{Neue Anweisungen}\qquad}
	\cardcontent{\emph{Gefallen erhalten/einlösen:}\\
	Hat eine Karte die Anweisung \enquote{\emph{+1 Gefallen}}, erhältst du einen Gefallen und legst einen Marker auf dein Gefallen-Tableau. Wenn du 1 Gefallen einlöst, entferne einen Marker von deinem Tableau.

	\bigskip

	\emph{Einen Stapel rotieren:}\\
	Einen Stapel zu rotieren bedeutet, dass du die oberste Karte und alle gleichen Karten direkt darunter vom Stapel nimmst und sie unter den Stapel legst. }
\end{tikzpicture}
\hspace{-0.6cm}
\begin{tikzpicture}
	\card
	\cardstrip
	\cardbanner{banner/white.png}
	\cardtitle{\scriptsize{Empfohlene 10er Sätze\qquad}}
	\cardcontent{\emph{Immer diese Entscheidungen} (\underline{Verbündete}):\\
	\underline{Stadtstaat}, Botin, Bürger, Händlerlager, Jägerin, Königliche Galeere, Marquis, Tand, Umgestaltung, Wegelagerer, Wirtin

	\smallskip

	\emph{Blick in die Zukunft} (\underline{Verbündete}):\\
	\underline{Sterndeuterorden}, Aufwiegler, Augurinnen, Barbar, Markthalle, Ortschaft, Schreinerin, Spezialistin, Untergebener, Unterhändlerin, Wächter

	\smallskip

	\emph{Verbündete für Anfänger} (\textit{Basisspiel (2. Edition)} + \underline{Verbündete}):\\
	\underline{Handwerkergilde}, Hauptstadt, Irrfahrten, Markthalle, Mittelsmann, Schmeichler, \textit{Gärten}, \textit{Markt}, \textit{Umbau}, \textit{Vasall}, \textit{Vorbotin}

	\smallskip

	\emph{Verfeindete Ladenbesitzer} (\textit{Basisspiel (2. Edition)} + \underline{Verbündete}):\\
	\underline{Bund der Ladenbesitzer}, Gildemeisterin, Konflikte, Königliche Galeere, Ortschaft, Unterhändlerin, \textit{Banditin}, \textit{Burggraben}, \textit{Geldverleiher}, \textit{Händlerin}, \textit{Laboratorium}

	\smallskip

	\emph{Dunkle Geschäfte} (\textit{Intrige (2. Edition)} + \underline{Verbündete}):\\
	\underline{Hexenbund}, Botin, Bürger, Jägerin, Mittelsmann, Vertrag, \textit{Adelige}, \textit{Geheimgang}, \textit{Herumtreiberin}, \textit{Höflinge}, \textit{Verwalter}
	
	\smallskip
	
	\emph{Fußvolk} (\textit{Intrige (2. Edition)} + \underline{Verbündete}):\\
	\underline{Schäfer der Hochebene} Händlerlager, Tausch, Untergebener, Wirtin, Zauberer, \textit{Austausch}, \textit{Baron}, \textit{Handlanger}, \textit{Patrouille}, \textit{Verschwörer}}
\end{tikzpicture}
\hspace{-0.6cm}
\begin{tikzpicture}
	\card
	\cardstrip
	\cardbanner{banner/white.png}
	\cardtitle{\scriptsize{Empfohlene 10er Sätze\qquad}}
	\cardcontent{\emph{Vorausschauendes Denken} (\textit{Seaside} + \underline{Verbündete}):\\
	\underline{Höhlenbewohner}, Gildemeisterin, Irrfahrten, Königliche Galeere, Wächter, Wegelagerer, \textit{Beutelschneider}, \textit{Eingeborenendorf}, \textit{Lagerhaus}, \textit{Schmuggler}, \textit{Taktiker}

	\smallskip

	\emph{Schatzsuche} (\textit{Seaside} + \underline{Verbündete}):\\
	\underline{Marktstädte}, Bastionen, Marquis, Ortschaft, Unterhändlerin, Tausch, \textit{Ausguck}, \textit{Außenposten}, \textit{Hafen}, \textit{Schatzkammer}, \textit{Schatzkarte}

	\smallskip

	\emph{Rekursion} (\textit{Alchemisten} + \underline{Verbündete}):\\
	\underline{Höhlenhafen}, Barbar, Händlerlager, Importeurin, Markthalle, Umwandlung, Zauberer, \textit{Alchemist}, \textit{Golem}, \textit{Lehrling}, \textit{Vision}

	\smallskip

	\emph{Erfindungsboom} (\textit{Blütezeit (mit Platin und Kolonie)} + \underline{Verbündete}):\\
	\underline{Erfinderfamilie}, Augurinnen, Hauptstadt, Importeurin, Schreinerin, Tand, \textit{Ausbau}, \textit{Gesindel}, \textit{Königshof}, \textit{Steinbruch}, \textit{Talisman}
	
	\smallskip
	
	\emph{Schmeicheln macht reich} (\textit{Blütezeit (mit Platin und Kolonie)} + \underline{Verbündete}):\\
	\underline{Bankiersverbund}, Irrfahrten, Marquis, Mittelsmann, Ortschaft, Schmeichler, \textit{Bank}, \textit{Gewölbe}, \textit{Handelsroute}, \textit{Münzer}, \textit{Stadt}}
\end{tikzpicture}
\hspace{-0.6cm}
\begin{tikzpicture}
	\card
	\cardstrip
	\cardbanner{banner/white.png}
	\cardtitle{\scriptsize{Empfohlene 10er Sätze\qquad}}
	\cardcontent{\emph{Sammelleidenschaft} (\textit{Reiche Ernte} + \textit{Die Gilden} + \underline{Verbündete}):\\
	\underline{Holzarbeitergilde}, Bastionen Konflikte, Markthalle, Vertrag, Wächter, \textit{Berater}, \textit{Festplatz}, \textit{Menagerie}, \textit{Platz}, \textit{Treibjagd}

	\smallskip

	\emph{Walderkunder} (\textit{Reiche Ernte} + \textit{Die Gilden} + \underline{Verbündete}):\\
	\underline{Waldbewohner}, Augurinnen, Königliche Galeere, Unterhändlerin, Wächter, Wirtin, \textit{Bäcker}, \textit{Bauerndorf}, \textit{Harlekin}, \textit{Leuchtenmacher}, \textit{Wandergeselle}

	\smallskip

	\emph{Längster Tunnel} (\textit{Hinterland} + \underline{Verbündete}):\\
	\underline{Gefolgschaft der Schreiber}, Hauptstadt, Schreinerin, Tand, Vertrag, Wirtin, \textit{Feilscher}, \textit{Fruchtbares Land}, \textit{Lebenskünstler}, \textit{Markgraf}, \textit{Tunnel}

	\smallskip

	\emph{Expertise} (\textit{Hinterland} + \underline{Verbündete}):\\
	\underline{Freimaurerloge}, Barbar, Bürger, Spezialistin, Untergebener: Wegelagerer, \textit{Fernstraße}, \textit{Gasthaus}, \textit{Gewürzhändler}, \textit{Grenzdorf}, \textit{Wegkreuzung}
	
	\smallskip
	
	\emph{Ernste Angelegenheiten} (\textit{Dark Ages (mit Unterschlüpfen)} + \underline{Verbündete}):\\
	\underline{Höhlenbewohner}, Barbar, Mittelsmann, Vertrag, Wegelagerer, Zauberer, \textit{Armenhaus}, \textit{Banditenlager}, \textit{Bettler}, \textit{Grabräuber}, \textit{Mundraub}

	\smallskip

	\emph{Rattenhändler} (\textit{Dark Ages (mit Unterschlüpfen)} + \underline{Verbündete}):\\
	\underline{Wüstenführer}, Aufwiegler, Bürger Importeurin, Unterhändlerin, Tausch, \textit{Graf}, \textit{Knappe}, \textit{Leichenkarren}, \textit{Ratten}, \textit{Ritter}}
\end{tikzpicture}
\hspace{-0.6cm}
\begin{tikzpicture}
	\card
	\cardstrip
	\cardbanner{banner/white.png}
	\cardtitle{\scriptsize{Empfohlene 10er Sätze\qquad}}
	\cardcontent{\emph{Achtung, Taschendiebe} (\textit{Abenteuer} + \underline{Verbündete} + \underline{\textit{Ereignisse}}):\\
	\underline{\textit{Mission}}, \underline{Taschendiebe}, Augurinnen, Spezialistin, Tand, Umwandlung, Wirtin, \textit{Duplikat}, \textit{Geizhals}, \textit{Kunsthandwerker}, \textit{Schatz}, \textit{Verlorene Stadt}

	\smallskip

	\emph{Vollendete Zukunft} (\textit{Abenteuer} + \underline{Verbündete} + \underline{\textit{Ereignisse}}):\\
	\underline{Marktstädte}, \underline{\textit{Seeweg}}, Aufwiegler, Bastionen, Marquis, Schmeichler, Wächter, \textit{Ausrüstung}, \textit{Geisterwald}, \textit{Hafenstadt}, \textit{Karawanenwächter}, \textit{Transformation}

	\smallskip

	\emph{Inselimperium} (\textit{Empires} + \underline{Verbündete} + \underline{\textit{Landmarken}}):\\
	\underline{Inselvolk}, \underline{\textit{Obstgarten}}, Bastionen, Schmeichler, Spezialistin, Tausch, Vertrag, \textit{Bauernmarkt}, \textit{Siedler/Emsiges Dorf}, \textit{Stadtviertel}, \textit{Wilde Jagd}, \textit{Zauberin}
	
	\smallskip
	
	\emph{Krieg der Schlösser} (\textit{Empires} + \underline{Verbündete} + \underline{\textit{Ereignisse}}):\\
	\underline{Hütte der Fallensteller}, \underline{\textit{Siegeszug}}, Hauptstadt, Importeurin, Jägerin, Konflikte, Schreinerin, \textit{Katapult/Felsen}, \textit{Krone}, \textit{Patrizier/Handelsplatz}, \textit{Schlösser}, \textit{Zauber}

	\smallskip

	\emph{Liebe und Tod} (\textit{Nocturne} + \underline{Verbündete}):\\
	\underline{Friedlicher Kult}, Augurinnen, Jägerin, Schmeichler, Schreinerin, Tand, \textit{Getreuer Hund}, \textit{Götze}, \textit{Konklave}, \textit{Sündenpfuhl}, \textit{Totenbeschwörer}

	\smallskip

	\emph{Spiel's noch einmal, Sam} (\textit{Nocturne} + \underline{Verbündete}):\\
	\underline{Holzarbeitergilde}, Botin, Jägerin, Königliche Galeere, Tausch, Zauberer, \textit{Attentäter}, \textit{Fährtensucher}, \textit{Folterknecht}, \textit{Kobold}, \textit{Seliges Dorf}}
\end{tikzpicture}
\hspace{-0.6cm}
\begin{tikzpicture}
	\card
	\cardstrip
	\cardbanner{banner/white.png}
	\cardtitle{\scriptsize{Empfohlene 10er Sätze\qquad}}
	\cardcontent{\emph{Fließband} (\textit{Renaissance} + \underline{Verbündete} + \underline{\textit{Projekte}}):\\
	\underline{\textit{Erkundung}}, \underline{Nomadenstamm}, Botin, Bürger, Importeurin, Umwandlung, Zauberer, \textit{Bergdorf}, \textit{Experiment}, \textit{Gewürze}, \textit{Patron}, \textit{Schatzmeisterin}

	\smallskip

	\emph{Zeitalter der Schreiber} (\textit{Renaissance} + \underline{Verbündete} + \underline{\textit{Projekte}}):\\
	\underline{\textit{Finsterer Plan}}, \underline{Gefolgschaft der Schreiber}, Hauptstadt, Irrfahrten, Markthalle, Spezialistin, Untergebener, \textit{Alte Hexe}, \textit{Erfinder}, \textit{Forscherin}, \textit{Schauspieltruppe}, \textit{Schwarzer Meister}
	
	\smallskip
	
	\emph{Weise Eulen} (\textit{Menagerie} + \underline{Verbündete} + \underline{\textit{Wege}}):\\
	\underline{Architektengilde}, \underline{\textit{Weg der Eule}}, Barbar, Händlerlager, Marquis, Ortschaft, Zauberer, \textit{Herberge}, \textit{Jagdhütte}, \textit{Kopfgeldjägerin}, \textit{Schwarze Katze}, \textit{Viehmarkt}

	\smallskip

	\emph{Bergkönige} (\textit{Menagerie} + \underline{Verbündete} + \underline{\textit{Ereignisse}}):\\
	\underline{Bergvolk}, \underline{\textit{Plackerei}}, Aufwiegler, Bastionen, Botin, Gildemeisterin, Mittelsmann, \textit{Hexenzirkel}, \textit{Lastkahn}, \textit{Nachschub}, \textit{Schrott}, \textit{Verschneites Dorf}}
\end{tikzpicture}
\hspace{-0.6cm}
\begin{tikzpicture}
	\card
	\cardstrip
	\cardbanner{banner/white.png}
	\cardtitle{Platzhalter\quad}
\end{tikzpicture}
\hspace{0.6cm}
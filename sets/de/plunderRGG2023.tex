% Basic settings for this card set
\renewcommand{\cardcolor}{plunder}
\renewcommand{\cardextension}{Erweiterung XIV}
\renewcommand{\cardextensiontitle}{Plünderer}
\renewcommand{\seticon}{plunder.png}

\clearpage
\newpage
\section{\cardextension \ - \cardextensiontitle}

\begin{tikzpicture}
	\card
	\cardstrip
	\cardbanner{banner/orange.png}
	\cardicon{icons/coin.png}
	\cardprice{2}
	\cardtitle{Grotte}
	\cardcontent{Du kannst zum Beispiel 3 deiner Handkarten zur Seite legen, diese zu Beginn deines nächsten Zuges ablegen und dann 3 Karten ziehen.}
\end{tikzpicture}
\hspace{-0.6cm}
\begin{tikzpicture}
	\card
	\cardstrip
	\cardbanner{banner/gold.png}
	\cardicon{icons/coin.png}
	\cardprice{2}
	\cardtitle{Juwelen-Ei}
	\cardcontent{Du darfst diese Karte nur entsorgen, wenn eine andere Karte die Anweisung gibt, dass du das tun darfst/musst. Entsorgst du ein \emph{JUWELEN-EI}, nimmst du die Kostbarkeit, unabhängig davon, wer die Karte gespielt hat, die die Anweisung zur Entsorgung gegeben hat.}
\end{tikzpicture}
\hspace{-0.6cm}
\begin{tikzpicture}
	\card
	\cardstrip
	\cardbanner{banner/goldorange.png}
	\cardicon{icons/coin.png}
	\cardprice{2}
	\cardtitle{Käfig}
	\cardcontent{Du nimmst die Karten nach dem Ziehen deiner Kartenhand für den nächsten Zug auf deine Hand. Wenn du zum Beispiel in einem Zug zwei \emph{ANWESEN} und zwei \emph{KUPFER} zur Seite legst und du in einem späteren Zug eine \emph{PROVINZ} kaufst, entsorgst du den \emph{KÄFIG} und nimmst am Ende deines Zuges die zwei \emph{ANWESEN} und zwei \emph{KUPFER} auf deine Hand.}
\end{tikzpicture}
\hspace{-0.6cm}
\begin{tikzpicture}
	\card
	\cardstrip
	\cardbanner{banner/white.png}
	\cardicon{icons/coin.png}
	\cardprice{2}
	\cardtitle{Schamanin}
	\cardcontent{In Spielen, in denen die \emph{SCHAMANIN} genutzt wird, muss jede:r Spieler:in zu Beginn jedes ihrer/seiner Züge bis zum Spielende eine Karte aus dem Müll nehmen, die bis zu \coin[6] kostet. Das ist nicht optional. Nur wenn keine solche Karte im Müll enthalten ist, musst du keine nehmen. Das gilt auch im ersten Zug, was beispielsweise beim \emph{TOTENBESCHWÖRER} (aus \emph{Nocturne}) relevant ist. Das gilt auch, wenn kein:e Spieler:in jemals eine \emph{SCHAMANIN} nimmt. Die genommene Karte wird auf den Ablagestapel der/des jeweiligen Spielerin/Spielers gelegt. Die Karte gilt als \enquote{genommen} und löst ggf. andere Ettekte aus, die sich auf \enquote{genommene} Karten beziehen. Nimmst du zum Beispiel ein \emph{ANWESEN}, löst dies den Effekt des \emph{KÄFIGS} aus.}
\end{tikzpicture}
\hspace{-0.6cm}
\begin{tikzpicture}
	\card
	\cardstrip
	\cardbanner{banner/orange.png}
	\cardicon{icons/coin.png}
	\cardprice{2}
	\cardtitle{Suche}
	\cardcontent{Wenn du eine \emph{SUCHE} mit Hilfe des \emph{THRONSAALS} (aus dem \emph{Basisspiel}) zweimal spielst, bleibt jener zusammen mit der \emph{SUCHE} so lange im Spiel, bis ein Vorratsstapel leer wird, dann entsorgst du die \emph{SUCHE}, nimmst zwei Kostbarkeiten und legst dann den \emph{THRONSAAL} in diesem Zug ab.}
\end{tikzpicture}
\hspace{-0.6cm}
\begin{tikzpicture}
	\card
	\cardstrip
	\cardbanner{banner/orangeblue.png}
	\cardicon{icons/coin.png}
	\cardprice{3}
	\cardtitle{\scriptsize{Blinder Passagier}}
	\cardcontent{Du darfst diese Karte spielen, wenn ein:e Spieler:in (auch du selbst) eine Dauerkarte nimmt.}
\end{tikzpicture}
\hspace{-0.6cm}
\begin{tikzpicture}
	\card
	\cardstrip
	\cardbanner{banner/orange.png}
	\cardicon{icons/coin.png}
	\cardprice{3}
	\cardtitle{\scriptsize{Einsamer Schrein}}
	\cardcontent{Dieser Effekt kann auch in fremden Zügen ausgelöst werden, auch wenn die/der Spieler:in dann nichts entsorgen kann oder möchte. Sie/Er muss keine Karte(n) entsorgen, der \emph{EINSAME SCHREIN} wird aber trotzdem in diesem Zug abgelegt.}
\end{tikzpicture}
\hspace{-0.6cm}
\begin{tikzpicture}
	\card
	\cardstrip
	\cardbanner{banner/orange.png}
	\cardicon{icons/coin.png}
	\cardprice{3}
	\cardtitle{Sirene}
	\cardcontent{Wenn du eine \emph{SIRENE} nimmst, entsorge sie oder eine Aktionskarte aus deiner Hand. Wenn du (z.B. durch die Anweisung auf einer anderen Karte) die \emph{SIRENE} vor dem Entsorgen bereits von dort, woher du sie genommen hast, bewegt hast (z.B. indem du sie mit Hilfe der \emph{Insignien} auf deinen Nachziehstapel gelegt hast), wird die \emph{SIRENE} nicht entsorgt (aber du kannst eine andere Aktionskarte aus deiner Hand entsorgen).}
\end{tikzpicture}
\hspace{-0.6cm}
\begin{tikzpicture}
	\card
	\cardstrip
	\cardbanner{banner/orange.png}
	\cardicon{icons/coin.png}
	\cardprice{3}
	\cardtitle{Vorarbeiter}
	\cardcontent{Der \emph{VORARBEITER} kann seine Fähigkeit Runde um Runde wieder verwenden, solange du in jedem Zug eine Karte nimmst, die zum Zeitpunkt des Nehmens genau \coin[5] kostet. Solltest du bereits vor dem Spielen des \emph{VORARBEITERS} eine Karte genommen haben, die genau \coin[5] kostet, reicht dies nicht.}
\end{tikzpicture}
\hspace{-0.6cm}
\begin{tikzpicture}
	\card
	\cardstrip
	\cardbanner{banner/white.png}
	\cardicon{icons/coin.png}
	\cardprice{4}
	\cardtitle{Abenteurerin}
	\cardcontent{Befolge erst alle Anweisungen der gespielten Geldkarte, bevor du die restlichen aufgedeckten Karten auf deinen Nachziehstapel zurücklegst. Wenn du zum Beispiel eine \emph{FIGURINE} aufdeckst und spielst, sind die beiden Karten, die du ziehst, nicht dieselben, die du dir bereits angeschaut hast.}
\end{tikzpicture}
\hspace{-0.6cm}
\begin{tikzpicture}
	\card
	\cardstrip
	\cardbanner{banner/white.png}
	\cardicon{icons/coin.png}
	\cardprice{4}
	\cardtitle{Ausgesetzter}
	\cardcontent{Kartentypen sind die Begriffe unten auf der Karte. Wenn du zum Beispiel einen \emph{KÄFIG} entsorgst, ziehst du 4 Karten - 2 pro Kartentyp, den die entsorgte Karte hat (Geld + Dauer).}
\end{tikzpicture}
\hspace{-0.6cm}
\begin{tikzpicture}
	\card
	\cardstrip
	\cardbanner{banner/orange.png}
	\cardicon{icons/coin.png}
	\cardprice{4}
	\cardtitle{Flaggschiff}
	\cardcontent{Die nächste von dir gespielte Aktionskarte, die nicht den Typ BEFEHL hat, erneut zu spielen, ist nicht optional. Sie wird auch erneut gespielt, wenn die gespielte Karte selbst entsorgt wird und auch, wenn dies nicht in deinem Zug erfolgt. Aktionskarten, die den Typ BEFEHL enthalten, sind von der Anweisung des \emph{FLAGGSCHIFFS} nicht betroffen. Spielst du zum Beispiel 2 \emph{FLAGGSCHIFFE} und dann ein \emph{HAFENDORF}, spielst du das \emph{HAFENDORF} drei Mal - einmal normal und jeweils einmal pro \emph{FLAGGSCHIFF}.}
\end{tikzpicture}
\hspace{-0.6cm}
\begin{tikzpicture}
	\card
	\cardstrip
	\cardbanner{banner/goldorange.png}
	\cardicon{icons/coin.png}
	\cardprice{4}
	\cardtitle{Gondel}
	\cardcontent{Wenn du dich entscheidest, die +\coin[2] in diesem Zug zu nutzen, legst du die \emph{GONDEL} am Ende dieses Zuges ab. Wenn du dich entscheidest, die +\coin[2] erst in deinem nächsten Zug zu nutzen, bleibt die Karte bis dahin im Spiel. Wenn du die \emph{GONDEL} mehrfach spielst, z.B. mit Hilfe der \emph{KÖNIGSTRUHE}, entscheidest du jedes Mal, ob du die +\coin[2] jetzt oder in deinem nächsten Zug nutzt. Die Karte bleibt nur im Spiel, wenn du dich mindestens 1x dafür entscheidest, die +\coin[2] in deinem nächsten Zug zu nutzen (in diesem Fall bleibt auch die \emph{KÖNIGSTRUHE} solange im Spiel).}
\end{tikzpicture}
\hspace{-0.6cm}
\begin{tikzpicture}
	\card
	\cardstrip
	\cardbanner{banner/white.png}
	\cardicon{icons/coin.png}
	\cardprice{4}
	\cardtitle{Hafendorf}
	\cardcontent{Du bekommst nur +\coin[1], wenn die nächste von dir gespielte Aktionskarte dir selbst mindestens +\coin[1] gibt. Es reicht nicht, zum Beispiel mit Hilfe des Ereignisses \emph{TRAINING} (aus \emph{Abenteuer}) +\coin[1] für das Ausspielen einer Aktionskarte zu bekommen. Wenn du einen \emph{Weg} nutzt, um mit Hilfe einer Aktionskarte +\coin zu erhalten, bekommst du den Bonus. Es ist okay, wenn du das erhaltene +\coin[1] zum Beispiel beim \emph{GESCHICHTENERZÄHLER} (aus \emph{Abenteuer}) nicht mehr hast. + 1 Taler (aus \emph{Gilden} und \emph{Renaissance}) gilt nicht als +\coin[1]. +\coin[0] reicht nicht aus, um den Bonus zu erhalten. Wenn du ein \emph{HAFENDORF} mithilfe eines \emph{THRONSAALS} (aus dem \emph{Basisspiel}) zweimal spielst und dann eine \emph{MILIZ} (aus dem \emph{Basisspiel}), erhältst du für das erste Spielen des \emph{HAFENDORFS} nichts (denn du hast danach das \emph{HAFENDORF} gespielt), für das zweite Spielen bekommst du aber den Bonus von +\coin[1].}
\end{tikzpicture}
\hspace{-0.6cm}
\begin{tikzpicture}
	\card
	\cardstrip
	\cardbanner{banner/blue.png}
	\cardicon{icons/coin.png}
	\cardprice{4}
	\cardtitle{\scriptsize{Kartenzeichnerin}}
	\cardcontent{Wenn du (auch nach dem Mischen deines Ablagestapels) weniger als vier Karten im Nachziehstapel hast, siehst du dir nur jene Karten an. Wenn ein:e Spieler:in (auch du selbst) eine Punktekarte (auch ggf. kombinierte) nimmt, darfst du diese Karte aus deiner Hand spielen. Wenn du eine \emph{KARTENZEICHNERIN} als Reaktion spielst, darfst du auch sofort eine weitere \emph{KARTENZEICHNERIN} spielen, selbst wenn du diese gerade erst durch das Spielen der vorhergehenden \emph{KARTENZEICHNERIN} erhalten hast.}
\end{tikzpicture}
\hspace{-0.6cm}
\begin{tikzpicture}
	\card
	\cardstrip
	\cardbanner{banner/orange.png}
	\cardicon{icons/coin.png}
	\cardprice{4}
	\cardtitle{\footnotesize{Landungstrupp}}
	\cardcontent{Die Karte bleibt so lange im Spiel, bis die erste in einem Zug von dir gespielte Karte eine Geldkarte (auch ggf. kombinierte) ist. Befolge erst alle Anweisungen der Geldkarte, bevor du den \emph{LANDUNGSTRUPP} auf deinen Nachziehstapel legst. Ist die gespielte Geldkarte beispielsweise eine \emph{FIGURINE}, ziehst du zwei Karten und legst dann den \emph{LANDUNGSTRUPP} auf deinen Nachziehstapel.}
\end{tikzpicture}
\hspace{-0.6cm}
\begin{tikzpicture}
	\card
	\cardstrip
	\cardbanner{banner/orange.png}
	\cardicon{icons/coin.png}
	\cardprice{4}
	\cardtitle{Schiffsjunge}
	\cardcontent{Du kannst einen \emph{SCHIFFSJUNGEN} entsorgen, um einen \emph{SCHIFFSJUNGEN} zu nehmen.}
\end{tikzpicture}
\hspace{-0.6cm}
\begin{tikzpicture}
	\card
	\cardstrip
	\cardbanner{banner/gold.png}
	\cardicon{icons/coin.png}
	\cardprice{4}
	\cardtitle{\footnotesize{Schmelztiegel}}
	\cardcontent{Wenn du zum Beispiel ein \emph{ANWESEN} entsorgst, das \coin[2] kostet, erhältst du +\coin[2]. Zusätzliche \potion oder \hex in den Kosten werden nicht berücksichtigt.}
\end{tikzpicture}
\hspace{-0.6cm}
\begin{tikzpicture}
	\card
	\cardstrip
	\cardbanner{banner/goldorange.png}
	\cardicon{icons/coin.png}
	\cardprice{4}
	\cardtitle{Seil}
	\cardcontent{Wenn du diese Karte spielst, erhältst du +\coin[1] und +1 Kauf. Zu Beginn deines nächsten Zuges ziehst du zuerst eine Karte und darfst dann eine deiner Handkarten entsorgen.}
\end{tikzpicture}
\hspace{-0.6cm}
\begin{tikzpicture}
	\card
	\cardstrip
	\cardbanner{banner/white.png}
	\cardicon{icons/coin.png}
	\cardprice{4}
	\cardtitle{Sumpfhütten}
	\cardcontent{Die \emph{SUMPFHÜTTEN} selbst zählen beim Zählen deiner Karten mit, ebenso wie Dauerkarten, die aus vorherigen Zügen im Spiel sind. Auch Geldkarten zählen mit, z.B. Geld-Dauerkarten aus vorherigen Zügen oder Geldkarten, die z.B. durch eine \emph{ABENTEURERIN} ins Spiel gekommen sind. Zur Seite gelegte Karten, sowie Karten z.B. auf dem \emph{QUARTIERMEISTER} zählen dagegen nicht. Sollte die Anzahl nicht durch drei teilbar sein, runde ab. Hast du zum Beispiel 8 Karten im Spiel, ziehst du 2 Karten.}
\end{tikzpicture}
\hspace{-0.6cm}
\begin{tikzpicture}
	\card
	\cardstrip
	\cardbanner{banner/goldorange.png}
	\cardicon{icons/coin.png}
	\cardprice{4}
	\cardtitle{Überfluss}
	\cardcontent{Diese kombinierte Geld- und Dauerkarte spielst du normalerweise in deiner Kaufphase. Du erhältst dafür aber nicht direkt \coin, sondern die Karte bleibt so lange im Spiel, bis du das nächste Mal eine Aktionskarte nimmst (egal ob du sie kaufst oder auf andere Weise nimmst). Dann erhältst du +1 Kauf und +\coin[3]. Sollte dies nicht in deinem Zug passieren, nutzen dir die +\coin[3] und +1 Kauf normalerweise nichts.}
\end{tikzpicture}
\hspace{-0.6cm}
\begin{tikzpicture}
	\card
	\cardstrip
	\cardbanner{banner/gold.png}
	\cardicon{icons/coin.png}
	\cardprice{4}
	\cardtitle{Werkzeug}
	\cardcontent{Du darfst jede Karte nehmen, von der ein:e Spieler:in (auch du selbst) eine gleiche im Spiel hat - auch Dauerkarten. \emph{WERKZEUG} selbst ist durch sein Spielen im Spiel, d.h. du darfst auch ein \emph{WERKZEUG} nehmen.}
\end{tikzpicture}
\hspace{-0.6cm}
\begin{tikzpicture}
	\card
	\cardstrip
	\cardbanner{banner/gold.png}
	\cardicon{icons/coin.png}
	\cardprice{5}
	\cardtitle{Anhänger}
	\cardcontent{Zähle auch diese Karte selbst bei der Ausführung der Anweisung mit. Hast du zum Beispiel drei \emph{KUPFER} und eine \emph{GONDEL} aus deinem letzten Zug sowie einen \emph{ANHÄNGER} im Spiel, erhältst du +\coin[3].}
\end{tikzpicture}
\hspace{-0.6cm}
\begin{tikzpicture}
	\card
	\cardstrip
	\cardbanner{banner/white.png}
	\cardicon{icons/coin.png}
	\cardprice{5}
	\cardtitle{\footnotesize{Bergbaustraße}}
	\cardcontent{Die Geldkarte zu spielen ist optional. Die Anweisung ist kumulativ, d.h. wenn du zwei \emph{BERGBAUSTRASSEN} spielst, darfst du zweimal eine Geldkarte nach dem Nehmen spielen. Dieselbe Geldkarte darfst du aber nicht zweimal mit Hilfe von zwei \emph{BERGBAUSTRASSEN} spielen. Es ist egal, wie du die Geldkarte nimmst, ob du sie kaufst oder auf andere Weise nimmst, auch wenn das zum Beispiel in deiner Aktionsphase passiert.}
\end{tikzpicture}
\hspace{-0.6cm}
\begin{tikzpicture}
	\card
	\cardstrip
	\cardbanner{banner/white.png}
	\cardicon{icons/coin.png}
	\cardprice{5}
	\cardtitle{Erste Maatin}
	\cardcontent{Wenn du durch eine so gespielte Aktionskarte eine gleiche Karte ziehst, darfst du jene spielen usw. Du kannst mit Hilfe einer \emph{ERSTEN MAATIN} auch eine oder mehrere \emph{ERSTE MAATINNEN} spielen. Sei bei der Abhandlung deiner gespielten Karten achtsam und verliere nicht den Überblick, welche \emph{ERSTE MAATIN} du gerade abhandelst, so wie du das beim \emph{THRONSAAL} (aus dem \emph{Basisspiel}) machen musst. Wenn du keine Aktionskarten spielen kannst oder willst, ziehst du trotzdem, bis du 6 Karten auf deiner Hand hast.}
\end{tikzpicture}
\hspace{-0.6cm}
\begin{tikzpicture}
	\card
	\cardstrip
	\cardbanner{banner/gold.png}
	\cardicon{icons/coin.png}
	\cardprice{5}
	\cardtitle{Figurine}
	\cardcontent{Diese Geldkarte spielst du normalerweise in deiner Kaufphase. Sie gibt dir + 2 Karten. So gezogene Aktionskarten können in der Kaufphase normalerweise nicht gespielt werden (wenn sie nicht auch Geldkarten sind); du darfst aber eine Aktionskarte für +1 Kauf und +\coin[1] ablegen.}
\end{tikzpicture}
\hspace{-0.6cm}
\begin{tikzpicture}
	\card
	\cardstrip
	\cardbanner{banner/orange.png}
	\cardicon{icons/coin.png}
	\cardprice{5}
	\cardtitle{Fregatte}
	\cardcontent{Die Anweisung, seine Handkarten auf 4 zu reduzieren, tritt jedes Mal in Kraft, wenn ein:e Mitspieler:in eine Aktionskarte gespielt hat. Das kann auch später im selben Zug sein (z.B. durch den \emph{BLINDEN PASSAGIER}). Nur wer die Aktionskarte gespielt hat, ist betroffen und folgt erst allen Anweisungen der gespielten Karte, bevor sie/er seine Handkarten reduziert.}
\end{tikzpicture}
\hspace{-0.6cm}
\begin{tikzpicture}
	\card
	\cardstrip
	\cardbanner{banner/white.png}
	\cardicon{icons/coin.png}
	\cardprice{5}
	\cardtitle{Gauner}
	\cardcontent{Dieser Effekt ist kumulativ, d.h. wenn du zwei \emph{GAUNER} spielst, kannst du bis zu zwei Geldkarten, die du ablegst, zur Seite legen und sie nach dem Ziehen deiner Kartenhand zum Ende deines Zuges auf deine Hand nehmen.}
\end{tikzpicture}
\hspace{-0.6cm}
\begin{tikzpicture}
	\card
	\cardstrip
	\cardbanner{banner/orange.png}
	\cardicon{icons/coin.png}
	\cardprice{5}
	\cardtitle{Langschiff}
	\cardcontent{Wenn du diese Karte spielst, erhältst du +2 Aktionen und zu Beginn deines nächsten Zuges +2 Karten.}
\end{tikzpicture}
\hspace{-0.6cm}
\begin{tikzpicture}
	\card
	\cardstrip
	\cardbanner{banner/orange.png}
	\cardicon{icons/coin.png}
	\cardprice{5}
	\cardtitle{Mannschaft}
	\cardcontent{Diese Karte zu Beginn deines nächsten Zuges auf deinen Nachziehstapel zu legen, ist nicht optional.}
\end{tikzpicture}
\hspace{-0.6cm}
\begin{tikzpicture}
	\card
	\cardstrip
	\cardbanner{banner/orange.png}
	\cardicon{icons/coin.png}
	\cardprice{5}
	\cardtitle{Meuchlerin}
	\cardcontent{Alle Karten mit dem Kartentyp Kostbarkeit sind selbst Geldkarten, die \coin[5] oder mehr kosten. Das heißt, wenn ein:e Spieler:in (auch du selbst) eine Kostbarkeit nimmt, während du eine \emph{MEUCHLERIN} im Spiel hast, nimmst du eine weitere Kostbarkeit und legst die \emph{MEUCHLERIN} dann in der nächsten Aufräumphase ab.}
\end{tikzpicture}
\hspace{-0.6cm}
\begin{tikzpicture}
	\card
	\cardstrip
	\cardbanner{banner/white.png}
	\cardicon{icons/coin.png}
	\cardprice{5}
	\cardtitle{Pilger}
	\cardcontent{Die Karte, die du auf deinen Nachziehstapel legst, muss keine von denen sein, die du gerade gezogen hast.}
\end{tikzpicture}
\hspace{-0.6cm}
\begin{tikzpicture}
	\card
	\cardstrip
	\cardbanner{banner/orange.png}
	\cardicon{icons/coin.png}
	\cardprice{5}
	\cardtitle{\scriptsize{Quartiermeister}}
	\cardcontent{Diese Karte bleibt für den Rest des Spiels im Spiel. In jedem deiner Züge nimmst du entweder eine Karte, die bis zu \coin[4] kostet und legst sie zur Seite (auf den \emph{QUARTIERMEISTER}) oder du nimmst eine der Karten auf deinem \emph{QUARTIERMEISTER} auf deine Hand. Wenn du 2 \emph{QUARTIERMEISTER} spielst, halte die Karten darauf getrennt, du darfst diese nicht mischen.

	\smallskip

	Wenn du einen \emph{QUARTIERMEISTER} mit Hilfe des \emph{THRONSAALS} (aus dem \emph{Basisspiel}) zweimal spielst, hast du nur einen \emph{QUARTIERMEISTER}, auf den du Karten legen oder von dort nehmen kannst. Du kannst dann aber zweimal pro Zug entscheiden, ob du eine Karte, die bis zu \coin[4] kostet, nimmst oder eine von deinem \emph{QUARTIERMEISTER} auf deine Hand nimmst.}
\end{tikzpicture}
\hspace{-0.6cm}
\begin{tikzpicture}
	\card
	\cardstrip
	\cardbanner{banner/gold.png}
	\cardicon{icons/coin.png}
	\cardprice{5}
	\cardtitle{Silbermine}
	\cardcontent{Du kannst ein \emph{SILBER} nehmen, aber auch jede andere Geldkarte aus dem Vorrat (auch ggf. kombinierte), die weniger als die \emph{SILBERMINE} kostet, z.B. \emph{GONDEL}, \emph{JUWELEN-El} usw.}
\end{tikzpicture}
\hspace{-0.6cm}
\begin{tikzpicture}
	\card
	\cardstrip
	\cardbanner{banner/gold.png}
	\cardicon{icons/coin.png}
	\cardprice{5}
	\cardtitle{Spitzhacke}
	\cardcontent{Wenn du nach dem Spielen der \emph{SPITZHACKE} noch mindestens 1 Karte auf deiner Hand hast, musst du eine entsorgen. Zeige die genommene Kostbarkeit deinen
	Mitspieler:innen.}
\end{tikzpicture}
\hspace{-0.6cm}
\begin{tikzpicture}
	\card
	\cardstrip
	\cardbanner{banner/goldorange.png}
	\cardicon{icons/coin.png}
	\cardprice{5}
	\cardtitle{\tiny{Vergrabener Schatz}}
	\cardcontent{Wenn du diese Karte nimmst, spiele sie - das ist nicht optional.}
\end{tikzpicture}
\hspace{-0.6cm}
\begin{tikzpicture}
	\card
	\cardstrip
	\cardbanner{banner/orange.png}
	\cardicon{icons/coin.png}
	\cardprice{5}
	\cardtitle{\footnotesize{Vergrößerung}}
	\cardcontent{Eine Karte aus deiner Hand zu Beginn deines nächsten Zuges zu entsorgen ist nicht optional.}
\end{tikzpicture}
\hspace{-0.6cm}
\begin{tikzpicture}
	\card
	\cardstrip
	\cardbanner{banner/white.png}
	\cardicon{icons/coin.png}
	\cardprice{5}
	\cardtitle{\tiny{Wohlhabendes Dorf}}
	\cardcontent{Zu den drei Geldkarten mit unterschiedlichen Namen zählen auch Geld-Dauerkarten, die du aus vergangenen Zügen im Spiel hast, sowie Kostbarkeiten selber.}
\end{tikzpicture}
\hspace{-0.6cm}
\begin{tikzpicture}
	\card
	\cardstrip
	\cardbanner{banner/gold.png}
	\cardicon{icons/coin.png}
	\cardprice{6}
	\cardtitle{\tiny{Sack voll Kostbark.}}
	\cardcontent{Wenn du diese Karte spielst, erhältst du +\coin[1], +1 Kauf und du nimmst eine Kostbarkeit.}
\end{tikzpicture}
\hspace{-0.6cm}
\begin{tikzpicture}
	\card
	\cardstrip
	\cardbanner{banner/gold.png}
	\cardicon{icons/coin.png}
	\cardprice{7}
	\cardtitle{Königstruhe}
	\cardcontent{Wenn du eine \emph{KÖNIGSTRUHE} durch eine \emph{KÖNIGSTRUHE} dreimal spielst, kannst du drei weitere Geldkarten jeweils dreimal spielen. Wenn du eine kombinierte Geld-Dauerkarte mit Hilfe der \emph{KÖNIGSTRUHE} dreimal spielst, bleibt die \emph{KÖNIGSTRUHE} so lange wie die gespielte Geld-Dauerkarte im Spiel.}
\end{tikzpicture}
\hspace{-0.6cm}
\begin{tikzpicture}
	\card
	\cardstrip
	\cardbanner{banner/gold.png}
	\cardicon{icons/coin.png}
	\cardprice{7*}
	\cardtitle{\tiny{Kostbarkeiten (1/3)}}
	\cardcontent{\emph{Amphore:} Du kannst entscheiden, ob du die +1 Kauf und +\coin[3] jetzt oder zu Beginn deines nächsten Zuges erhältst. Wenn du dich entscheidest, die +1 Kauf und +\coin[3] in diesem Zug zu erhalten, legst du die \emph{AMPHORE} am Ende dieses Zuges ab.
	\\
	Wenn du dich entscheidest, die +1 Kauf und +\coin[3] erst in deinem nächsten Zug zu erhalten, bleibt die Karte bis dahin im Spiel. Wenn du die \emph{AMPHORE} mehrfach nutzt, z.B. mit Hilfe der \emph{KÖNIGSTRUHE}, entscheidest du jedes Mal, ob du die +1 Kauf und +\coin[3] jetzt oder in deinem nächsten Zug erhältst. Die Karte bleibt nur im Spiel, wenn du dich mindestens 1x dafür entscheidest, die +1 Kauf und +\coin[3] in deinem nächsten Zug zu erhalten (in diesem Fall bleibt auch die \emph{KÖNIGSTRUHE} solange im Spiel).
	
	\medskip
	
	\emph{Dublonen:} Wenn du diese Karte nimmst, nimm ein \emph{GOLD}.
	
	\medskip
	
	\emph{Endloser Kelch:} Wenn du diese Karte einmal gespielt hast, bleibt sie bis zum Spielende im Spiel.
	
	\medskip
	
	\emph{Galionsfigur:} Wenn du diese Karte spielst, erhältst du +\coin[3] und zu Beginn deines nächsten Zuges +2 Karten.
	
	\medskip

	\emph{Hammer:} Wenn du diese Karte spielst, erhältst du +\coin[3] und nimmst eine Karte, die bis zu \coin[4] kostet. Dies ist nicht optional.}
\end{tikzpicture}
\hspace{-0.6cm}
\begin{tikzpicture}
	\card
	\cardstrip
	\cardbanner{banner/gold.png}
	\cardicon{icons/coin.png}
	\cardprice{7*}
	\cardtitle{\tiny{Kostbarkeiten (2/3)}}
	\cardcontent{\emph{Insignien:} Wenn du mehrere Karten nimmst, kannst du Insignien auf keine, alle oder einige davon anwenden, also beliebig viele davon auf deinen Nachziehstapel legen.
	
	\medskip
	
	\emph{Juwelen:} Wenn du diese Karte spielst, erhältst du +\coin[3] sowie +1 Kauf und du legst diese Karte zu Beginn deines nächsten Zuges unter deinen Nachziehstapel.
	
	\medskip
	
	\emph{Kugel:} Schau zuerst deinen Nachrichstape durch und entscheide dann, ob du eine Geld-
	oder eine Aktionskarte daraus spielst oder ob du +1 Kauf und +\coin[3] erhältst.
	
	\medskip
	
	\emph{Rätselschatulle:} Wenn du eine Karte zur Seite legst, legst du die \emph{RÄTSELSCHATULLE} trotzdem ganz normal am Ende dieses Zuges ab. Die zur Seite gelegte Karte nimmst du nach dem Ziehen deiner nächsten Kartenhand auf deine Hand.
	
	\medskip
	
	\emph{Schild:} Wenn eine Mitspielerin eine Angriffskarte spielt, darfst du diese Karte aus deiner Hand aufdecken, genau wie beim \emph{BURGGRABEN} (aus dem \emph{Basisspiel}), du bist dann von dem Angriff nicht betroffen. Du tust dies, bevor die Angriffskarte ausgeführt wird und du kannst diesen \emph{SCHILD} gegen mehrere Angriffskarten pro Zug einsetzen, solange du ihn auf deiner Hand hast. Der \emph{SCHILD} bleibt auf deiner Hand und du kannst ihn in deinem Zug spielen, um +1 Kauf und +\coin[3]) zu erhalten.}
\end{tikzpicture}
\hspace{-0.6cm}
\begin{tikzpicture}
	\card
	\cardstrip
	\cardbanner{banner/gold.png}
	\cardicon{icons/coin.png}
	\cardprice{7*}
	\cardtitle{\tiny{Kostbarkeiten (3/3)}}
	\cardcontent{\emph{Schwert:} Dies ist eine Angriffskarte, Spieler:innen können mit Karten wie dem \emph{BURGGRABEN} (aus dem \emph{Basisspiel}) und/oder dem \emph{SCHILD} reagieren.
	
	\medskip
	
	\emph{Sextant:} Du kannst alle 5 Karten zurücklegen oder alle 5 Karten ablegen, oder eine beliebige Mischung davon.
	
	\medskip
	
	\emph{Stab:} Eine Aktionskarte aus deiner Hand zu spielen ist optional.
	
	\medskip
	
	\emph{Zauberrolle:} Du kannst diese Karte in deiner Aktions- oder Kaufphase spielen. Wenn du sie in deiner Aktionsphase spielst, benötigt sie 1 Aktion. Die so genommene Karte zu spielen, verbraucht keine Aktion.
	
	\medskip
	
	\emph{Zuchtziege:} Eine Karte zu entsorgen ist optional.}
\end{tikzpicture}
\hspace{-0.6cm}
\begin{tikzpicture}
	\card
	\cardstrip
	\cardbanner{banner/purple.png}
	\cardtitle{Merkmale (1/4)\quad}
	\cardcontent{\emph{Aufdringlich:} Dieser Effekt ist nicht optional.
	
	\medskip
	
	\emph{Benachbart:} Jedes Mal, wenn du eine Karte nimmst, die aus dem markierten Stapel stammt (= \emph{Benachbarte} Karte), erhältst du +1 Kauf.
	
	\medskip
	
	\emph{Billig:} Alle Karten des markierten Stapels (= \emph{Billige} Karten) kosten für das gesamte Spiel für alle Belange (auch bei der Wertung) \coin[1] weniger. Die Kosten können nicht unter \coin[0] fallen. Kosten wie \potion und \hex werden nicht beeinflusst. So hat \emph{Billig} auf die Karten des \emph{INGENIEURIN}-Stapels (aus \emph{Empires}) keinen Einfluss. Der Effekt ist erst nach Spielstart (d.h. noch nicht in der Spielvorbereitung) anwendbar, so dass zum Beispiel keine Karte, die ursprünglich \coin[4] kostet, für den \emph{BANN-Stapel} (aus \emph{Reiche Ernte}) genutzt werden kann.
	
	\medskip
	
	\emph{Eilig:} Wenn die Karten des markierten Stapels (= \emph{Eilige} Karten) normalerweise nicht gespielt werden können (z.B. \emph{TERRITORIUM} aus \emph{Verbündete}), kommen sie trotzdem in deinem nächsten Zug ins Spiel, haben aber keinen Effekt und werden am Ende des Zuges abgelegt.
	
	\medskip
	
	\emph{Freundlich:} Du därfst pro Zug nur eine \emph{Freundliche} Karte auf diese Weise ablegen.}
\end{tikzpicture}
\hspace{-0.6cm}
\begin{tikzpicture}
	\card
	\cardstrip
	\cardbanner{banner/purple.png}
	\cardtitle{Merkmale (2/4)\quad}
	\cardcontent{\emph{Fromm:} Jedes Mal, wenn du eine Karte nimmst, die aus dem markierten Stapel stammt (= \emph{Fromme} Karte), darfst du eine Karte aus deiner Hand entsorgen.
	
	\medskip
	
	\emph{Geduldig:} Du kannst mehrere \emph{Geduldige} Karten auf einmal zur Seite legen, du darfst zu Beginn deines nächsten Zuges entscheiden, in welcher Reihenfolge du sie spielst. Wenn die Karten des markierten Stapels (= \emph{Geduldige} Karten) keinen Effekt beim Spielen haben (z.B. \emph{TERRITORIUM} aus \emph{Verbündete}), kannst du sie trotzdem zur Seite legen, im nächsten Zug spielen und in der Aufräumphase des nächsten Zuges ablegen.
	
	\medskip
	
	\emph{Geerbt:} Alle Spielerinnen entscheiden - beginnend bei der/dem Startspieler:in - welche Karte sie durch eine \emph{Geerbte} Karte ersetzen wollen. Ersetzte \emph{KUPFER} werden auf den Vorratsstapel zurückgelegt, ersetzte \emph{ANWESEN} und andere Karten (z.B. \emph{UNTERSCHLÜPFE} aus \emph{Dark Ages}) werden aus dem Spiel genommen. Wenn der markierte Stapel ein gemischter Stapel ist (aus \emph{Empires} oder \emph{Verbündete}), nehmen die Spieler:innen in Zugreihenfolge eine Karte vom Stapel. So bekommen in einem 6 Personen-Spiel mit den \emph{BÜRGERN} (aus \emph{Verbündete}) als \emph{Geerbter} Stapel die ersten 4 Spieler:innen eine \emph{AUSRUFERIN}, die zwei anderen Spieler:innen den \emph{EISENSCHMIED}. Jene Karten gelten nicht als \enquote{genommen}.}
\end{tikzpicture}
\hspace{-0.6cm}
\begin{tikzpicture}
	\card
	\cardstrip
	\cardbanner{banner/purple.png}
	\cardtitle{Merkmale (3/4)\quad}
	\cardcontent{\emph{Inspirierend:} Wenn du eine Karte, die aus dem markierten Stapel stammt (= \emph{Inspirierende} Karte), spielst, darfst du, nachdem du sie ausgeführt hast, eine Aktionskarte aus deiner Hand spielen, von der du keine gleiche im Spiel hast. Mit Hilfe von \emph{Inspirierend} kannst du normalerweise keine andere \emph{Inspirierende} Karte spielen, außer es handelt sich um einen gemischten Stapel - dann kannst du eine Karte mit anderem Namen vom gleichen Stapel spielen. Dauerkarten, die noch aus vorherigen Runden im Spiel sind, gelten als \enquote{im Spiel}, Karten, die du aus dem Spiel genommen hast, z.B. das \emph{BERGWERK} (aus \emph{Intrige}), das sich selbst entsorgt hat, befinden sich nicht im Spiel.
		
	\medskip
	
	\emph{Reich:} Jedes Mal, wenn du eine Karte nimmst, die aus dem markierten Stapel stammt (= \emph{Reiche} Karte), nimm ein \emph{SILBER}.
		
	\medskip
	
	\emph{Scheu:} Du kannst pro Zug nur 1 \emph{Scheue} Karte auf diese Weise ablegen.
	
	\medskip
	
	\emph{Unermüdlich:} Dies ist nicht optional. Du legst die zur Seite gelegte Karte erst nach dem Ziehen deiner Kartenhand auf deinen Nachziehstapel.

	\medskip
	
	\emph{Verflucht:} Wenn du eine Karte vom markierten Stapels nimmst (= \emph{Verfluchte} Karte), nimm eine Kostbarkeit und einen \emph{FLUCH}. Wenn kein \emph{FLUCH} mehr im Vorrat ist, nimmst du nur eine Kostbarkeit.}
\end{tikzpicture}
\hspace{-0.6cm}
\begin{tikzpicture}
	\card
	\cardstrip
	\cardbanner{banner/purple.png}
	\cardtitle{Merkmale (4/4)\quad}
	\cardcontent{\emph{Vorherbestimmt:} Jedes Mal, wenn du Karten mischst, darf du beliebig viele \emph{Vorherbestimmte} Karten davon herausnehmen (decke sie dabei zur Kontrolle auf). Mische dann die übrigen Karten und lege die herausgenommenen Karten auf und/oder unter die gemischten
	Karten.
	\\
	Wenn zum Beispiel von den zu mischenden Karten 5 Karten \emph{Vorherbestimmte} Karten sind, kannst du 3 herausnehmen, die restlichen Karten mischen und dann 2 \emph{Vorherbestimmte} Karten nach oben und eine nach unten legen. In Spielen mit \emph{Vorherbestimmt} darfst du, bevor du mischst, die zu mischenden Karten anschauen, auch wenn du sicher weißt, dass keine \emph{Vorherbestimmten} Karten darin sind.
	
	\medskip
	
	\emph{Waghalsig:} Wenn du eine Karte spielst, die aus dem markigrten Stapel stammt (= \emph{Waghalsige} Karte), führe sie komplett aus und dann führe sie noch einmal komplett aus. Wenn du sie dann aus dem Spiel ablegst (wenn es sich um eine Dauerkarte handelt, kann das in einem späteren Zug sein), lege sie auf ihren Stapel zurück.
	\\
	Wenn du die Anweisungen der Karten nicht ausführst, z.B. weil du einen \emph{Weg} (aus \emph{Menagerie}) nutzt, führst du die Karte nicht ein zweites Mal aus, legst sie aber trotzdem auf ihren Stapel zurück, wenn du sie aus dem Spiel ablegst.}
\end{tikzpicture}
\hspace{-0.6cm}
\begin{tikzpicture}
	\card
	\cardstrip
	\cardbanner{banner/white.png}
	\cardtitle{Ereignisse (1/4)\quad}
	\cardcontent{\emph{Vergraben:} Sobald du dieses Ereignis erwirbst, ist die Ausführung seiner Fähigkeit nicht optional. 
	
	\medskip
	
	\emph{Ausweichen:} Wenn du in diesem Zug nicht mischst, bewirkt dieses Ereignis nichts. Wenn du mischst, schaust du zuerst deinen Ablagestapel durch, nimmst bis zu drei Karten heraus, mischst den Rest und legst die herausgenommenen Karten auf deinen Ablagestapel. \emph{Ausweichen} ist kumulativ und kann mehrfach pro Zug genutzt werden. Wenn du das Ereignis zum Beispiel dreimal nutzt, kannst du bis zu 9 Karten herausnehmen. Du kannst dadurch ggf. so viele Karten in deinem Ablagestapel lassen, dass du nicht genügend Karten zum Ziehen deiner Kartenhand auf deinem Nachziehstapel hast – dann ziehst du nur die möglichen Karten, denn das bewirkt nicht, dass du erneut mischen musst.
	
	\medskip
	
	\emph{Beeilung:} Wenn du \emph{Beeilung} zweimal hintereinander erwirbst und dann eine Aktionskarte nimmst, darfst du sie trotzdem nur einmal spielen. Du kannst aber eine \emph{Beeilung} erwerben, eine Aktionskarte kaufen, jene spielen und dann eine weitere \emph{Beeilung} erwerben usw.
	
	\medskip
	
	\emph{Riskieren:} Du bekommst die Kostbarkeit nur, wenn du eine Aktionskarte aus deiner Hand entsorgt hast.}
\end{tikzpicture}
\hspace{-0.6cm}
\begin{tikzpicture}
	\card
	\cardstrip
	\cardbanner{banner/white.png}
	\cardtitle{Ereignisse (2/4)\quad}
	\cardcontent{
	\emph{Zustellung:} Dieses Ereignis mehr als einmal pro Zug zu erwerben bringt keine zusätzlichen Vorteile. Die zur Seite gelegten Karten nimmst du nach dem Ziehen deiner Kartenhand auf deine Hand.
	
	\medskip
	
	\emph{Brandschatzung:} Wenn du nicht drei Karten abgelegt hast, nimmst du keine Kostbarkeit.
	
	\medskip
	
	\emph{Herumsuchen:} Du kannst entweder eine Karte aus deiner Hand entsorgen oder ein \emph{ANWESEN} aus dem Müll nehmen. Nur wenn du ein \emph{ANWESEN} genommen hast, nimmst du auch eine Karte, die bis zu \coin[5] kostet, aus dem Vorrat.
	
	\medskip
	
	\emph{In See stechen:} Mit dem Erwerb dieses Ereignisses kehrst du unmittelbar in deine Aktionsphase zurück. \enquote{Zu Beginn deines Zuges}-Anweisungen werden dadurch nicht ausgelöst; aber \enquote{Zu Beginn deiner Kaufphase}-Anweisungen werden, sobald du die Aktionsphase beendet hast und erneut in deine Kaufphase übergehst, ausgelöst.

	\medskip
	
	\emph{Spiegeln:} Dieses Ereignis ist kumulativ. Wenn du z.B. Spiegeln zuerst dreimal in deinem Zug erwirbst und dann eine Aktionskarte, nimmst du insgesamt vier Exemplare dieser Aktionskarte (bzw. so viele, wie noch im Vorrat vorhanden sind, falls das weniger sind).}
\end{tikzpicture}
\hspace{-0.6cm}
\begin{tikzpicture}
	\card
	\cardstrip
	\cardbanner{banner/white.png}
	\cardtitle{Ereignisse (3/4)\quad}
	\cardcontent{\emph{Vorbereitung:} Sobald du deine Handkarten aufgedeckt zur Seite gelegt hast, musst du zu Beginn deines nächsten Zuges alle jene Aktions- und Geldkarten spielen. Das ist verpflichtend.
	
	\smallskip
	
	\emph{Mahlstrom:} Eine Karte zu entsorgen ist nicht optional für deine Mitspieler:innen. Sie müssen eine entsorgen, wenn sie 5 oder mehr Karten in ihrer Hand haben. Da es sich nicht um einen durch eine Angriffskarte ausgelösten Effekt handelt, können deine Mitspielerinnen nicht mit Reaktionskarten wie \emph{BURGGRABEN} (aus dem \emph{Basisspiel}) oder \emph{SCHILD} reagieren.
	
	\smallskip
	
	\emph{Reise:} Du kannst dieses Ereignis nur einmal pro Zug erwerben. Wenn du das tust und der vorherige Zug nicht deiner war, legst du deine Karten aus dem Spiel nicht ab - deine Handkarten legst du ab - und führst nach Zugende einen weiteren Zug aus. Der weitere Zug ist ein ganz normaler Zug, mit der Ausnahme, dass er nicht für die Endwertung im Falle eines Gleichstands mitzählt. Die Karten, die im Spiel bleiben, sind einfach nur im Spiel, du kannst ihre Anweisungen nicht erneut nutzen (z.B. gibt dir ein \emph{KUPFER} kein \coin[1]). Karten mit einer \enquote{Während diese Karte im Spiel ist}-Anweisung können genutzt werden und die Karten sind relevant für Anweisungen, die sich auf Karten, die sich im Spiel befinden, beziehen (z.B. \emph{SUMPFHÜTTEN}). Ansonsten bedeutet es nur, dass du die Karten für deinen weiteren Zug nicht ziehen kannst. Karten, die sowieso im Spiel geblieben wären (z.B. \emph{LANGSCHIFF}), bleiben aus diesem Grund im Spiel und machen, was sie normalerweise getan hätten.}
\end{tikzpicture}
\hspace{-0.6cm}
\begin{tikzpicture}
	\card
	\cardstrip
	\cardbanner{banner/white.png}
	\cardtitle{Ereignisse (4/4)\quad}
	\cardcontent{\emph{Plünderung:} Du nimmst einfach nur eine Kostbarkeit.
		
	\medskip
	
	\emph{Aufblühen:} Nimm zuerst eine Kostbarkeit. Dann kannst du - eine nach der anderen - entscheiden, ob und welche Geldkarte mit unterschiedlichem Namen du nimmst. Führe eventuelle durch das Nehmen ausgelöste Anweisungen der jeweils genommenen Karte komplett aus, bevor du die nächste nimmst usw. Du musst keine Geldkarten (außer der Kostbarkeit) nehmen. Du könntest zum Beispiel eine \emph{GONDEL} nehmen, ihre \enquote{Wenn du diese Karte nimmst}-Anweisung ausführen, um einen \emph{GAUNER} zu spielen, dann ein \emph{GOLD} nehmen, dann ein \emph{SILBER} nehmen und dann stoppen.
	
	\medskip
	
	\emph{Invasion:} Du führst die Anweisungen in der angegebenen Reihenfolge aus. Eine Aktionskarte zu spielen ist optional, der Rest ist verpflichtend.}
\end{tikzpicture}
\hspace{-0.6cm}
\begin{tikzpicture}
	\card
	\cardstrip
	\cardbanner{banner/white.png}
	\cardtitle{\scriptsize{Spielvorbereitung (1/2)}\qquad}
	\cardcontent{Zum Spielen mit \emph{DOMINION Plünderer} benötigt ihr ein \emph{DOMINION-Basisspiel} oder das \emph{Basiskarten-Set}. Legt alle Basiskarten aus dem \emph{Basisspiel} (\emph{KUPFER}, \emph{SILSER}, \emph{GOLD}  (+ ggf. \emph{PLATIN}), \emph{ANWESEN}, \emph{HERZOGTUM} \emph{PROVINZ} (+ ggf. \emph{KOLONIE}) sowie \emph{FLÜCHE} und die Müllkarte bzw. das Mülltableau) wie gewohnt als Teil des Vorrat in die Tischmitte.

	\medskip

	\emph{Kostbarkeiten}\\
	Verwendet ihr mindestens 1 Karte, die sich auf \emph{Kostbarkeiten} bezieht, mischt alle Karten mit dem Typ \emph{KOSTBARKEIT} und legt sie als gemischten Stapel mit der Bildseite nach unten neben dem Vorrat bereit. \emph{Kostbarkeits-Karten} gehören nicht zum Vorrat.}
\end{tikzpicture}
\hspace{-0.6cm}
\begin{tikzpicture}
	\card
	\cardstrip
	\cardbanner{banner/white.png}
	\cardtitle{\scriptsize{Spielvorbereitung (2/2)}\qquad}
	\cardcontent{\emph{Ereignisse \& Merkmale}\\
	Zusätzlich zu den Königreichkarten gibt es \emph{Ereignisse} (siehe NEUE REGELN \rightarrow Ereignisse) und \emph{Merkmale} (siehe NEUE REGELN \rightarrow Merkmale). Wir empfehlen, pro Spiel maximal insgesamt 2 \emph{Projekte} (aus \emph{Renaissance}), \emph{Landmarken} (aus \emph{Empires}), \emph{Ereignisse} (aus \emph{Abenteuer}, \emph{Empires}, \emph{Menagerie} und/oder \emph{Plünderer}), \emph{Wege} (aus \emph{Menagerie}) sowie \emph{Merkmale} (aus \emph{Plünderer}) zu verwenden.
	
	\smallskip
	
	Zieht \emph{Ereignisse} und \emph{Merkmale} zufällig aus einem Stapel (dieser kann auch die \emph{Landmarken} (aus \emph{Empires}), \emph{Ereignisse} (aus \emph{Abenteuer}, \emph{Empires} und/oder \emph{Menagerie}), \emph{Projekte} (aus \emph{Renaissance}) und/oder \emph{Wege} (aus \emph{Menagerie}) enthalten) oder mischt sie (trotz ihrer unterschiedlichen Rückseite) in die Platzhalterkarten ein. Deckt ihr ein \emph{Ereignis} oder ein \emph{Merkmal} auf, legt das \emph{Ereignis} bzw. das \emph{Merkmal} neben dem Vorrat bereit. \emph{Ereignisse} und \emph{Merkmale} gehören nicht zum Vorrat.
	
	\smallskip
	
	Pro \emph{Merkmal}, das ihr im Spiel verwendet, wählt aus den 10 Königreichkartenstapeln des Spiels einen beliebigen Geld- oder Aktionskartenstapel aus und legt das \emph{Merkmal} so unter jenen Stapel, dass der \emph{Merkmal}-Text lesbar ist. \emph{Ereignisse} und \emph{Merkmale} können keine Stapel ersetzen, sie werden auch nicht als Stapel behandelt.}
\end{tikzpicture}
\hspace{-0.6cm}
\begin{tikzpicture}
	\card
	\cardstrip
	\cardbanner{banner/white.png}
	\cardtitle{\scriptsize{Neue Regeln (1/4)}\qquad}
	\cardcontent{Es gelten die Basisspielregeln mit folgenden Erweiterungen:

	\medskip

	\emph{Kostbarkeiten}\\
	Es gibt 15 verschiedene \emph{Kostbarkeiten}, die jeweils 2x vorhanden sind. Die Anweisung \enquote{Nimm eine \emph{Kostbarkeit}} bedeutet, du nimmst die oberste Karte des verdeckten \emph{Kostbarkeits-Stapels}. Wenn du eine \emph{Kostbarkeit} nimmst, zeige sie deinen Mitspieler:innen und lege sie dann auf deinen Ablagestapel. Es ist nicht erlaubt, den \emph{Kostbarkeits-Stapel} während des Spiels durchzusehen. Der \emph{Kostbarkeits-Stapel} gehört nicht zum Vorrat – die Spieler:innen dürfen davon keine Karten kaufen und nur Karten davon nehmen, wenn eine Anweisung sich ausdrücklich auf \emph{Kostbarkeiten} bezieht.}
\end{tikzpicture}
\hspace{-0.6cm}
\begin{tikzpicture}
	\card
	\cardstrip
	\cardbanner{banner/white.png}
	\cardtitle{\scriptsize{Neue Regeln (2/4)}\qquad}
	\cardcontent{\tiny{\emph{Merkmale}\\
	\emph{Merkmale} sind jeweils nur 1x im Spiel enthalten. Sie sind keine Königreichkarten und beeinflussen nur einen einzelnen Geld- oder Aktionskartenstapel, der vor dem Spiel ausgewählt wird. Pro \emph{Merkmal}, das ihr im Spiel verwendet, wählt aus den 10 Königreichkarten-Stapeln des Spiels einen beliebigen Geld- oder Aktionskartenstapel aus und legt das \emph{Merkmal} so unter jenen Stapel, dass der \emph{Merkmal}-Text lesbar ist.
	
	\smallskip
	
	Während des Spiels werden die Karten von jenem Stapel beeinflusst – die Art und Weise steht auf dem jeweiligen \emph{Merkmal}. 
	\begin{itemize}
		\item \emph{Merkmale} sind keine Königreichkarten und können niemals gekauft oder genommen werden.
		\item \emph{Merkmale} können nur auf Königreichkartenstapel angewendet werden, niemals auf Basiskarten wie z.B. \emph{SILBER} oder die \emph{RUINEN} (aus \emph{Dark Ages}).
		\item Es dürfen niemals 2 \emph{Merkmale} unter denselben Stapel gelegt werden.
		\item Die Anweisungen auf \emph{Merkmalen} enthalten immer den Namen des \emph{Merkmals} und beziehen sich auf Karten des Stapels, auf den das \emph{Merkmal} angewendet wird, z.B. sind alle Karten des Stapels, unter dem das \emph{Merkmal} \emph{Fromm} liegt, \emph{Fromme} Karten.
		\item \emph{Merkmale} können auch auf gemischte Stapel (aus \emph{Empires} und \emph{Verbündete}) angewendet werden und betreffen dann alle Karten jenes Stapels.
		\item Ist ein Stapel mit einem \emph{Merkmal} leer, werden die Karten trotzdem noch durch das \emph{Merkmal} beeinflusst.
	\end{itemize}}}
\end{tikzpicture}
\hspace{-0.6cm}
\begin{tikzpicture}
	\card
	\cardstrip
	\cardbanner{banner/white.png}
	\cardtitle{\scriptsize{Neue Regeln (3/4)}\qquad}
	\cardcontent{\tiny{\emph{Ereignisse}\\
	In \emph{Plünderer} gibt es \emph{Ereignisse}, die erstmals in \emph{Abenteuer} erschienen sind. In deiner Kaufphase kannst du ein \emph{Ereignis} statt einer anderen Karte erwerben (dies verbraucht 1 Kauf). Du bezahlst die Kosten, die auf dem \emph{Ereignis} stehen, und dessen Effekt tritt sofort ein.
	\begin{itemize}
		\item \emph{Ereignisse} sind keine Königreichkarten. Sie liegen lediglich aus und liefern einen Effekt, den du kaufen kannst. Es gibt keine Möglichkeit, dass du ein \emph{Ereignis} nehmen kannst oder dass ein \emph{Ereignis} in deinem Kartensatz ist.
		\item Der Erwerb eines \emph{Ereignisses} verbraucht 1 Kauf. Normalerweise kannst du entweder eine Karte kaufen oder ein \emph{Ereignis} erwerben. Wenn du 2 Käufe hast, wie zB. nach dem Spielen des \emph{SACKES VOLL KOSTBARKEITEN}, kannst du zwei Karten kaufen oder zwei \emph{Ereignisse} erwerben oder eine Karte und ein \emph{Ereignis}, in beliebiger Reihenfolge.
		\item \emph{Ereignisse} können in einem Zug mehrmals erworben werden, wenn du genügend Käufe dafür verfügbar hast.
		\item Nachdem du ein \emph{Ereignis} erworben hast, darfst du in dieser Kaufphase keine weiteren Geldkarten spielen, es sei denn, eine Anweisung erlaubt dir dies explizit.
		\item Der Erwerb eines \emph{Ereignisses} ist kein Kauf einer Karte und löst deshalb nicht Karten wie den \emph{FEILSCHER} (aus \emph{Hinterland}) aus.
		\item Die Kosten von \emph{Ereignissen} werden nicht durch Karten wie die \emph{BRÜCKE} (aus \emph{Intrige}) beeinflusst.
	\end{itemize}}}
\end{tikzpicture}
\hspace{-0.6cm}
\begin{tikzpicture}
	\card
	\cardstrip
	\cardbanner{banner/white.png}
	\cardtitle{\scriptsize{Neue Regeln (4/4)}\qquad}
	\cardcontent{\tiny{\emph{Dauerkarten}\\
	In \emph{Plünderer} gibt es zwanzig \emph{Dauerkarten} (wie schon in \emph{Seaside} und den Erweiterungen ab \emph{Abenteuer}) bei den Königreichkarten und vier Dauerkarten bei den \emph{Kostbarkeiten}. Die orangefarbenen Dauerkarten beinhalten Anweisungen, die in späteren Zügen umgesetzt werden. Du legst sie nicht in der Aufräumphase des Zuges ab, in dem du sie gespielt hast, sondern sie bleiben bis zur Aufräumphase des Zuges, in dem die letzte Anweisung ausgeführt wird, im Spiel. Wird eine Dauerkarte mehrfach gespielt (z.B. durch das \emph{FLAGGSCHIFF}), bleibt die verursachende Karte ebenfalls so lange im Spiel, bis die Dauerkarte abgelegt wird. Um anzuzeigen, dass eine Dauerkarte in der aktuellen Aufräumphase noch nicht abgelegt wird, wird sie in eine eigene Reihe oberhalb der restlichen gespielten Karten gelegt. Hast du zu Beginn deines Zuges mehrere Dauerkarten im Spiel, deren Anweisungen zu dem Zeitpunkt ausgeführt werden sollen, darfst du die Reihenfolge selbst bestimmen, in der du sie abhandelst.
	
	\smallskip
	
	Viele Dauerkarten in \emph{Plünderer} sind Geldkarten. Sie sind wie normale Geldkarten, mit der Ausnahme, dass sie so lange im Spiel bleiben, bis ihre letzte Anweisung abgeschlossen ist. Einige Dauerkarten in \emph{Plünderer} tun etwas, sobald eine bestimmte Situation das \enquote{nächste Mal} eintritt. Das kann im gleichen Zug sein oder viele Züge später. Die entsprechende Dauerkarte bleibt so lange im Spiel, bis diese spezielle Situation eintritt. Zum Beispiel könntest du einen \emph{EINSAMEN SCHREIN} und 2 \emph{KUPFER} spielen, ein \emph{SILBER} kaufen und dann sofort 2 Karten aus deiner Hand entsorgen – dann legst du den \emph{EINSAMEN SCHREIN} in diesem Zug ab. Oder du kaufst stattdessen einen \emph{BLINDEN PASSAGIER} und lässt den \emph{EINSAMEN SCHREIN} bis zu deinem nächsten Zug im Spiel.}}
\end{tikzpicture}
\hspace{-0.6cm}
\begin{tikzpicture}
	\card
	\cardstrip
	\cardbanner{banner/white.png}
	\cardtitle{\scriptsize{Empfohlene 10er Sätze\qquad}}
	\cardcontent{\emph{Treibgut:}\\
	Abenteurerin, Bergbaustraße, Einsamer Schrein, Erste Maatin (\rightarrow \underline{Eilig}), Juwelen-Ei, Landungstrupp, Schmelztiegel, Silbermine, Wohlhabendes Dorf, Überfluss

	\smallskip

	\emph{Strandgut:}\\
	\underline{Vorbereitung}, Blinder Passagier, Gondel, Grotte (\rightarrow \underline{Fromm}), Langschiff, Mannschaft, Meuchlerin, Quartiermeister, Spitzhacke, Suche, Sirene

	\smallskip

	\emph{Erste Plünderung} (+ \textit{Basisspiel (2. Edition)}):\\
	Ausgesetzter, Fregatte, Hafendorf (\rightarrow \underline{Unermüdlich}), Pilger, Sack voll Kostbarkeiten, \textit{Burggraben}, \textit{Keller}, \textit{Markt}, \textit{Mine}, \textit{Vasall}

	\smallskip

	\emph{Voodoo} (+ \textit{Basisspiel (2. Edition)}):\\
	\underline{Mahlstrom}, Erste Maatin, Flaggschiff, Gondel, Schamanin, Vergrabener Schatz, \textit{Bürokrat}, \textit{Geldverleiher}, \textit{Jahrmarkt} (\rightarrow \underline{Verflucht}), \textit{Töpferei}, \textit{Umbau}

	\smallskip

	\emph{Eier aufschlagen} (+ \textit{Intrige (2. Edition)}):\\
	Ausgesetzter (\rightarrow \underline{Waghalsig}), Juwelen-Ei, Kartenzeichnerin, Quartiermeister, Vergrabener Schatz, \textit{Adlige}, \textit{Austausch}, \textit{Bergwerk}, \textit{Harem}, \textit{Höflinge}

	\smallskip

	\emph{Landratten} (+ \textit{Intrige (2. Edition)}):\\
	\underline{Ausweichen}, Abenteurerin, Anhänger, Meuchlerin, Pilger, Wohlhabendes Dorf, \textit{Geheimgang}, \textit{Handlanger}, \textit{Mühle}, \textit{Verschwörer} (\rightarrow Freundlich), \textit{Wunschbrunnen}}
\end{tikzpicture}
\hspace{-0.6cm}
\begin{tikzpicture}
	\card
	\cardstrip
	\cardbanner{banner/white.png}
	\cardtitle{\scriptsize{Empfohlene 10er Sätze\qquad}}
	\cardcontent{\emph{Weinrote Meere} (+ \textit{Seaside (2. Edition)}):\\
	Fregatte (\rightarrow Billig), Käfig, Schiffsjunge, Seil, Vergrößerung, \textit{Astrolabium}, \textit{Fischerdorf}, \textit{Karawane}, \textit{Meerhexe}, \textit{Seefahrerin}

	\smallskip

	\emph{Schatzinseln} (+ \textit{Seaside (2. Edition)}):\\
	\underline{In See stechen} Blinder Passagier, Langschiff, Mannschaft, Vergrabener Schatz, Überfluss, \textit{Ausguck}, \textit{Insel}, \textit{Korsarenschiff}, \textit{Schatzkarte} (\rightarrow Geerbt), \textit{Seekarte}

	\smallskip

	\emph{Speziallieferung} (+ \textit{Die Alchemisten}):\\
	\underline{Zustellung}, Bergbaustraße, Flaggschiff, Gauner, Juwelen-Ei, Sumpfhütten, Werkzeug, \textit{Alchemist}, \textit{Apotheker}, \textit{Golem} (\rightarrow Verflucht), \textit{Verwandlung}

	\smallskip

	\emph{Hübscher Schmuck} (+ \textit{Blütezeit (2. Edition)}):\\
	Figurine, Juwelen-Ei, Königstruhe, Seil (\rightarrow \underline{Vorherbestimmt}), Silbermine, \textit{Bank}, \textit{Brautkrone}, \textit{Geldanlage}, \textit{Kristallkugel}, \textit{Waffenkiste}

	\smallskip

	\emph{Gekauftes Glück} (+ \textit{Blütezeit (2. Edition)}):\\
	\underline{Plünderung}, Anhänger, Bergbaustraße, Blinder Passagier, Käfig, Sumpfhütten, \textit{Amboss}, \textit{Arbeiterdorf}, \textit{Bischof}, \textit{Buchhalterin}, \textit{Magnatin} (\rightarrow \underline{Aufdringlich})}
\end{tikzpicture}
\hspace{-0.6cm}
\begin{tikzpicture}
	\card
	\cardstrip
	\cardbanner{banner/white.png}
	\cardtitle{\scriptsize{Empfohlene 10er Sätze\qquad}}
	\cardcontent{\emph{Herolde \& Abenteurer} (+ \textit{Reiche Ernte / Die Gilden}):\\
	Abenteurerin (\rightarrow \underline{InspirierendInspirierend}), Anhänger, Flaggschiff, Schiffsjunge, Spitzhacke, \textit{Arzt}, \textit{Festplatz}, \textit{Herold}, \textit{Hellseherin}, \textit{Steinmetz}

	\smallskip

	\emph{Durch den Sumpf} (+ \textit{Reiche Ernte / Die Gilden}):\\
	\underline{Reise}, Käfig, Pilger (\rightarrow \underline{Geduldig}), Sumpfhütten, Vorarbeiter, Werkzeug, \textit{Bäcker}, \textit{Füllhorn}, \textit{Kaufmannsgilde}, \textit{Menagerie}, \textit{Weiler}

	\smallskip

	\emph{Wüstenträume} (+ \textit{Hinterland (2. Edition)}):\\
	Anhänger, Grotte, Hafendorf, Kartenzeichnerin, Vergrößerung, \textit{Feilscher}, \textit{Nomaden} (\rightarrow \underline{Waghalsig}), \textit{Oase}, \textit{Souk}, \textit{Weberin}

	\smallskip

	\emph{Wikinger-Komplott} (+ \textit{Hinterland (2. Edition)}):\\
	\underline{Herumsuchen}, Fregatte, Mannschaft, Schiffsjunge, Schmelztiegel, Wohlhabendes Dorf, \textit{Berserker} (\rightarrow \underline{Reich}), \textit{Hexenkessel}, \textit{Katzengold}, \textit{Komplott}, \textit{Stallungen}

	\smallskip

	\emph{Vaters Ratten} (+ \textit{Dark Ages}):\\
	Ausgesetzter, Erste Maatin, Schamanin, Seil, Suche, \textit{Armenhaus}, \textit{Knappe}, \textit{Landstreicher}, \textit{Leichenkarren}, \textit{Ratten} (\rightarrow \underline{Geerbt})

	\smallskip

	\emph{Verwüstungen} (+ \textit{Dark Ages}):\\
	\underline{Invasion}, Gauner, Grotte, Königstruhe, Meuchlerin, Vergrößerung, \textit{Eisenhändler}, \textit{Falschgeld}, \textit{Lagerraum}, \textit{Mundraub} (\rightarrow \underline{Unermüdlich}), \textit{Raubzug}}
\end{tikzpicture}
\hspace{-0.6cm}
\begin{tikzpicture}
	\card
	\cardstrip
	\cardbanner{banner/white.png}
	\cardtitle{\scriptsize{Empfohlene 10er Sätze\qquad}}
	\cardcontent{\emph{Segel setzen} (+ \textit{Abenteuer}):\\
	\textit{\underline{Überfahrt}}, Abenteurerin, Erste Maatin, Figurine, Kartenzeichnerin, Suche, \textit{Ferne Lande}, \textit{Hafenstadt}, \textit{Kunsthandwerker} (\rightarrow \underline{Geduldig}), \textit{Rattenfänger}, \textit{Schatz}

	\smallskip

	\emph{Schneller Job} (+ \textit{Abenteuer}):\\
	\underline{Beeilung}, Blinder Passagier, Einsamer Schrein, Quartiermeister, Sumpfhütten, Werkzeug, \textit{Ausrüstung}, \textit{Gefolgsmann}, \textit{Geisterwald}, \textit{Königliche Münzen}, \textit{Weinhändler} (\rightarrow \underline{Scheu})

	\smallskip

	\emph{Stadtgründer} (+ \textit{Empires}):\\
	\textit{\underline{Museum}}, Fregatte, Schmelztiegel, Vorarbeiter, Werkzeug, Überfluss, \textit{Bauernmarkt}, \textit{Gärtnerin}, \textit{Patrizier / Handelsplatz} (\rightarrow \underline{Benachbart}), \textit{Stadtviertel}, \textit{Wilde Jagd}

	\smallskip

	\emph{Viel, viel, viel} (+ \textit{Empires}):\\
	\underline{Aufblühen}, Bergbaustraße, Figurine (\rightarrow \underline{Freundlich}), Landungstrupp, Seil, Wohlhabendes Dorf, \textit{Gladiator / Festung}, \textit{Krone}, \textit{Opfer}, \textit{Zauber}, \textit{Zauberin}}
\end{tikzpicture}
\hspace{-0.6cm}
\begin{tikzpicture}
	\card
	\cardstrip
	\cardbanner{banner/white.png}
	\cardtitle{\scriptsize{Empfohlene 10er Sätze\qquad}}
	\cardcontent{\emph{Nacht der Plünderung} (+ \textit{Nocturne}):\\
	Anhänger, Figurine, Sack voll Kostbarkeiten, Schiffsjunge, Vorarbeiter (\rightarrow \underline{Fromm}), \textit{Getreuer Hund}, \textit{Krypta}, \textit{Seliges Dorf}, \textit{Tragischer Held}, \textit{Werwolf}
	
	\smallskip
	
	\emph{Skelettinsel} (+ \textit{Nocturne}):\\
	\underline{Brandschatzung}, Einsamer Schrein, Königstruhe, Langschiff, Meuchlerin, Pilger, \textit{Attentäter}, \textit{Fährtensucher}, \textit{Geisterstadt}, \textit{Götze} (\rightarrow \underline{Eilig}), \textit{Teufelswerkstatt}
	
	\smallskip
	
	\emph{Kreislauf des Lebens} (+ \textit{Renaissance}):\\
	\textit{\underline{Kathedrale}}, Fregatte, Juwelen-Ei, Schamanin, Spitzhacke, Suche, \textit{Experiment}, \textit{Freibeuterin}, \textit{Gelehrte} (\rightarrow \underline{Inspirierend}), \textit{Patron}, \textit{Schauspieltruppe}

	\smallskip

	\emph{Herr der Spiegel} (+ \textit{Renaissance}):\\
	\underline{Spiegeln}, Gauner, Gondel, Quartiermeister, Schmelztiegel, Vorarbeiter, \textit{Fahnenträger}, \textit{Frachtschiff}, \textit{Gewürze} (\rightarrow \underline{Vorherbestimmt}), \textit{Grenzposten}, \textit{Seher}}
\end{tikzpicture}
\hspace{-0.6cm}
\begin{tikzpicture}
	\card
	\cardstrip
	\cardbanner{banner/white.png}
	\cardtitle{\scriptsize{Empfohlene 10er Sätze\qquad}}
	\cardcontent{\emph{Nach Hause gehen} (+ \textit{Menagerie}):\\
	\textit{\underline{Weg des Eichhörnchens}}, Einsamer Schrein, Gondel, Kartenzeichnerin, Landungstrupp, Silbermine, \textit{Brennofen}, \textit{Lastkahn}, \textit{Nachschub}, \textit{Verschneites Dorf}, \textit{Wache} (\rightarrow \underline{Billig})
	
	\smallskip
	
	\emph{Groß rauskommen} (+ \textit{Menagerie}):\\
	\underline{Riskieren}, Grotte, Hafendorf, Sack voll Kostbarkeiten, Sirene, Vergrößern, \textit{Drahtzieher}, \textit{Hirtenhund}, \textit{Kamelzug}, \textit{Viehmarkt} (\rightarrow \textit{Benachbart}), \textit{Zufluchtsort}
	
	\smallskip
	
	\emph{Schiffskameraden} (+ \textit{Verbündete}):\\
	\textit{\underline{Höhlenbewohner}}, Flaggschiff, Hafendorf, Mannschaft, Sack voll Kostbarkeiten, Schamanin, \textit{Bastionen}, \textit{Mittelsmann}, \textit{Schmeichler} (\rightarrow \underline{Aufdringlich}), \textit{Umgestaltung}, \textit{Wirtin}
	
	\smallskip
	
	\emph{Vergraben \& vergessen} (+ \textit{BVerbündete}):\\
	\textit{Vergraben}, Ausgesetzter, Königstruhe, Landungstrupp, Vergrabener Schatz, Überfluss, \textit{Aufwiegler}, \textit{Botin}, \textit{Händlerlager} (\rightarrow \underline{Scheu}), \textit{Irrfahrten}, \textit{Wegelagerer}}
\end{tikzpicture}
\hspace{-0.6cm}
\begin{tikzpicture}
	\card
	\cardstrip
	\cardbanner{banner/white.png}
	\cardtitle{Platzhalter\quad}
\end{tikzpicture}
\hspace{0.6cm}
% Basic settings for this card set
\renewcommand{\cardcolor}{prosperity}
\renewcommand{\cardextension}{Erweiterung III}
\renewcommand{\cardextensiontitle}{Blütezeit}
\renewcommand{\seticon}{prosperity.png}

\clearpage
\newpage
\section{\cardextension \ - \cardextensiontitle \ 2. Edition (Rio Grande Games 2022)}

\begin{tikzpicture}
	\card
	\cardstrip
	\cardbanner{banner/blue.png}
	\cardicon{icons/coin.png}
	\cardprice{3}
	\cardtitle{Wachturm}
	\cardcontent{Diese Karte ist eine kombinierte Aktions- und Reaktionskarte. Sie kann in der Aktionsphase gespielt werden (Anweisung über der Trennlinie) oder als Reaktion auf die unter der Trennlinie angegebene Situation.

	\medskip

	Spielst du den \emph{WACHTURM} in deiner Aktionsphase, ziehst du solange Karten nach, bis du 6 Karten auf der Hand hast. Hast du bereits 6 oder mehr Handkarten, ziehst du keine
	Karten nach.

	\medskip

	Wenn du den \emph{WACHTURM} auf der Hand hast und eine Karte nimmst (in deinem eigenen Zug oder während des Zuges eines Mitspielers), darfst du ihn als Reaktion aufdecken und die genommene Karte entweder entsorgen oder auf deinen Nachziehstapel legen. Anschließend nimmst du den \emph{WACHTURM} wieder auf die Hand. Nimmst du anschließend eine oder mehrere weitere Karten (durch die gleiche oder eine andere Anweisung bzw. durch einen Kauf), kannst du den \emph{WACHTURM} erneut aufdecken - solange du ihn auf der Hand hast. Hast du den \emph{WACHTURM} in deiner nächsten Aktionsphase noch immer auf der Hand, darfst du ihn spielen.}
\end{tikzpicture}
\hspace{-0.6cm}
\begin{tikzpicture}
	\card
	\cardstrip
	\cardbanner{banner/gold.png}
	\cardicon{icons/coin.png}
	\cardprice{3}
	\cardtitle{Amboss}
	\cardcontent{Eine Geldkarte abzulegen, ist optional. Wenn du eine ablegst, nimmst du eine Karte aus dem Vorrat, die bis zu \coin[4] kostet, auf deinen Ablagestapel.}
\end{tikzpicture}
\hspace{-0.6cm}
\begin{tikzpicture}
	\card
	\cardstrip
	\cardbanner{banner/white.png}
	\cardicon{icons/coin.png}
	\cardprice{4}
	\cardtitle{Arbeiterdorf}
	\cardcontent{Du \emph{musst} eine Karte ziehen, bekommst + 2 Aktionen und einen zusätzlichen Kauf für die Kaufphase.}
\end{tikzpicture}
\hspace{-0.6cm}
\begin{tikzpicture}
	\card
	\cardstrip
	\cardbanner{banner/white.png}
	\cardicon{icons/coin.png}
	\cardprice{4}
	\cardtitle{Bischof}
	\cardcontent{Du erhältst +\coin[1] und einen \victorypointtoken-\emph{}. Dann musst du eine Handkarte entsorgen, wenn du mindestens 1 Karte auf der Hand hast. Lege halb so viele \victorypointtoken-\emph{Marker} vor dir ab, wie die entsorgte Karte kostet. Ungerade Kosten werden abgerundet. Andere Kosten wie \potion (aus \emph{Alchemisten}) oder \hex (aus \emph{Empires}) spielen keine Rolle. So erhältst du je \emph{2} \victorypointtoken-\emph{Marker} für ein entsorgtes \emph{ARBEITERDORF} (Kosten: \coin[4]), ebenso wie für ein \emph{GESINDEL} (Kosten: \coin[5]) oder einen \emph{GOLEM} (aus \emph{Die Alchemisten}, Kosten: \coin[4] und \potion).

	\medskip 

	Jeder Mitspieler darf eine Handkarte entsorgen, erhält dafür aber keine Punktemarker.}
\end{tikzpicture}
\hspace{-0.6cm}
\begin{tikzpicture}
	\card
	\cardstrip
	\cardbanner{banner/gold.png}
	\cardicon{icons/coin.png}
	\cardprice{4}
	\cardtitle{Brautkrone}
	\cardcontent{Wenn du später im Zug mehrere Karten nimmst, nachdem du die \emph{BRAUTKRONE} gespielt hast, darfst du eine beliebige Anzahl davon auf deinen Nachziehstapel legen. Dies betrifft Karten, die du kaufst oder auf eine andere Art und Weise nimmst, wie z.B. durch die \emph{WAFFENKISTE}.

	\medskip 

	Wenn du eine \emph{BRAUTKRONE} mit einer \emph{BRAUTKRONE} spielst, darfst du 2 weitere Geldkarten aus deiner Hand je zweimal spielen, nicht etwa eine Geldkarte viermal.}
\end{tikzpicture}
\hspace{-0.6cm}
\begin{tikzpicture}
	\card
	\cardstrip
	\cardbanner{banner/blue.png}
	\cardicon{icons/coin.png}
	\cardprice{4}
	\cardtitle{Buchhalterin}
	\cardcontent{Für Spieler mit einem leeren Nachziehstapel wird die abgelegte Karte zur einzigen Karte ihres Nachziehstapels.

	\medskip

	Zu Beginn deines Zuges darfst du beliebig viele \emph{BUCHHALTERINNEN} aus deiner Hand einzeln nacheinander spielen, ohne dafür Aktionen zu verwenden.}
\end{tikzpicture}
\hspace{-0.6cm}
\begin{tikzpicture}
	\card
	\cardstrip
	\cardbanner{banner/white.png}
	\cardicon{icons/coin.png}
	\cardprice{4}
	\cardtitle{Denkmal}
	\cardcontent{Du erhältst +\coin[2] und legst einen \victorypointtoken-\emph{Marker} in deinen Spielbereich.}
\end{tikzpicture}
\hspace{-0.6cm}
\begin{tikzpicture}
	\card
	\cardstrip
	\cardbanner{banner/gold.png}
	\cardicon{icons/coin.png}
	\cardprice{4}
	\cardtitle{Geldanalge}
	\cardcontent{Du musst eine Karte entsorgen. Nur falls die \emph{GELDANLAGE} deine letzte Handkarte war, entsorgst du keine Karte. Dann entscheidest du dich entweder für +\coin[1] oder dafür, die \emph{GELDANLAGE} zu entsorgen. Wenn du sie entsorgst, deckst du deine Handkarten auf und erhältst +\emph{1} \victorypoint pro Geldkarte mit unterschiedlichem Namen auf deiner Hand. Wenn du z.B. 2 \emph{KUPFER} und ein \emph{SILBER} aufdeckst, erhältst du +\emph{2} \victorypoint. Du darfst die aufgedeckten Geldkarten weiterhin spielen, nachdem du die \emph{GELDANLAGE} ausgeführt hast.}
\end{tikzpicture}
\hspace{-0.6cm}
\begin{tikzpicture}
	\card
	\cardstrip
	\cardbanner{banner/gold.png}
	\cardicon{icons/coin.png}
	\cardprice{4}
	\cardtitle{Steinbruch}
	\cardcontent{Alle Aktionskarten (auch kombinierte) kosten in diesem Zug \coin[2] weniger.
	Dies betrifft alle Aktionskarten, d.h. auch Handkarten, Karten in den Ablage- und Nachziehstapeln usw. Der Effekt ist kumulativ, d.h. mit einem zweiten \emph{STEINBRUCH} oder anderen Aktionskarten, die die Kosten von Karten reduzieren, können die Kosten weiter gesenkt werden, jedoch nie auf weniger als 0.}
\end{tikzpicture}
\hspace{-0.6cm}
\begin{tikzpicture}
	\card
	\cardstrip
	\cardbanner{banner/white.png}
	\cardicon{icons/coin.png}
	\cardprice{5}
	\cardtitle{Gesindel}
	\cardcontent{Ziehe drei Karten nach. Anschließend muss jeder Mitspieler (beginnend bei deinem linken Nachbarn) die obersten drei Karten seines Nachziehstapels aufdecken und alle aufgedeckten Geldkarten sowie Aktionskarten (auch kombinierte), ablegen. Alle anderen aufgedeckten Karten legt er in einer Reihenfolge seiner Wahl zurück auf den Nachziehstapel.}
\end{tikzpicture}
\hspace{-0.6cm}
\begin{tikzpicture}
	\card
	\cardstrip
	\cardbanner{banner/white.png}
	\cardicon{icons/coin.png}
	\cardprice{5}
	\cardtitle{Gewölbe}
	\cardcontent{Ziehe 2 Karten nach. Lege anschließend beliebig viele Handkarten (auch 0) ab. Du darfst auch Karten ablegen, die du gerade erst nachgezogen hast. Für jede abgelegte Karte erhältst du +\coin[1].

	\medskip

	Jeder Mitspieler darf 2 Handkarten ablegen und eine Karte nachziehen. Falls ein Mitspieler nur 1 Handkarte hat, darf er diese zwar ablegen, jedoch keine Karte nachziehen.}
\end{tikzpicture}
\hspace{-0.6cm}
\begin{tikzpicture}
	\card
	\cardstrip
	\cardbanner{banner/gold.png}
	\cardicon{icons/coin.png}
	\cardprice{6}
	\cardtitle{Hort}
	\cardcontent{Du nimmst nur dann ein \emph{GOLD}, wenn du die Punktekarte (auch kombinierte) kaufst, aber nicht, wenn du sie eine andere Art und Weise nimmst, wie z.B. mit der \emph{WAFFENKISTE}. Das \emph{GOLD} kommt aus dem Vorrat, wenn dort kein \emph{GOLD} mehr ist, erhältst du keins. Hast du zwei \emph{HORTE} im Spiel, nimmst du dir pro gekaufter Punktekarte zwei \emph{GOLD} usw. Kaufst du in einem Spielzug zwei oder mehr Punktekarten, nimmst du dir für jede Punktekarte entsprechend viele \emph{GOLD} vom Vorrat. Du erhältst auch \emph{GOLD}, wenn du die gekaufte Punktekarte im gleichen Spielzug wieder entsorgst.}
\end{tikzpicture}
\hspace{-0.6cm}
\begin{tikzpicture}
	\card
	\cardstrip
	\cardbanner{banner/gold.png}
	\cardicon{icons/coin.png}
	\cardprice{5}
	\cardtitle{Kristallkugel}
	\cardcontent{Wenn du dich gegen alle diese Alternativen entscheidest, musst du die Karte auf deinen Nachziehstapel zurücklegen.

	\medskip

	Wenn du durch die \emph{KRISTALLKUGEL} eine Aktionskarte während deiner Kaufphase spielst, durch die du +Aktionen erhältst, darfst du trotzdem keine weiteren Aktionskarten in deiner Kaufphase spielen; wenn du durch die Aktionskarte Geldkarten erhältst, darfst du diese noch spielen.}
\end{tikzpicture}
\hspace{-0.6cm}
\begin{tikzpicture}
	\card
	\cardstrip
	\cardbanner{banner/white.png}
	\cardicon{icons/coin.png}
	\cardprice{5}
	\cardtitle{Magnatin}
	\cardcontent{Wenn du z.B. zwei \emph{KUPFER} und ein \emph{SILBER} auf der Hand hast, ziehst du 3 Karten.}
\end{tikzpicture}
\hspace{-0.6cm}
\begin{tikzpicture}
	\card
	\cardstrip
	\cardbanner{banner/white.png}
	\cardicon{icons/coin.png}
	\cardprice{5}
	\cardtitle{Münzer}
	\cardcontent{Du darfst eine Geldkarte aus deiner Hand aufdecken. Wenn du das tust, nimm dir eine Karte mit gleichem Namen vom Vorrat. Ist keine entsprechende Karte im Vorrat vorhanden, erhältst du nichts. Nimm die aufgedeckte Geldkarte zurück auf die Hand.

	\medskip

	Wenn du den \emph{MÜNZER} nimmst, musst du alle Geldkarten, die du im Spiel hast, sofort entsorgen (Ausnahme: Dauerkarten). Die \coin, die die Geldkarten beim Spielen eingebracht haben, gehen durch das Entsorgen nicht verloren, du darfst sie also für weitere Käufe in diesem Zug noch verwenden. Wenn du einen \emph{MÜNZER} nimmst, während eine \emph{BRAUTKRONE} im Spiel ist, darfst du den \emph{MÜNZER} auf deinen Nachziehstapel legen, und zwar unabhängig davon, ob du die \emph{BRAUTKRONE} vorher oder hinterher entsorgst.

	\medskip

	Beachte, dass du alle Geldkarten, die du in deiner Kaufphase verwenden möchtest, vor deinem ersten Kauf spielen musst. Du kannst nicht 5 \emph{KUPFER} spielen, den \emph{MÜNZER} kaufen, diese Geldkarten entsorgen und dann weiteres Geld auslegen. Sobald eine Karte gekauft wurde, dürfen keinen weiteren Geldkarten mehr gespielt werden.}
\end{tikzpicture}
\hspace{-0.6cm}
\begin{tikzpicture}
	\card
	\cardstrip
	\cardbanner{banner/gold.png}
	\cardicon{icons/coin.png}
	\cardprice{5}
	\cardtitle{\footnotesize{Sammelsurium}}
	\cardcontent{Du erhältst +1 \victorypoint für jede Aktionskarte, die du kaufst oder auf eine andere Art und Weise nimmst. Mehrere \emph{SAMMELSURIEN} wirken kumulativ. Wenn du z.B. 2 \emph{SAMMELSURIEN} im Spiel hast und ein \emph{DORF} kaufst, erhältst du +2 \victorypoint.}
\end{tikzpicture}
\hspace{-0.6cm}
\begin{tikzpicture}
	\card
	\cardstrip
	\cardbanner{banner/white.png}
	\cardicon{icons/coin.png}
	\cardprice{5}
	\cardtitle{Stadt}
	\cardcontent{Du erhältst +1 Karte sowie +2 Aktionen. Wenn noch kein Vorratsstapel leer ist, passiert nichts weiter. Wenn genau 1 Vorratsstapel leer ist, erhältst du nochmal +1 Karte.
	Wenn 2 oder mehr Vorratsstapel leer sind, erhältst du stattdessen +1 Karte, +\coin[1] und +1 Kauf.}
\end{tikzpicture}
\hspace{-0.6cm}
\begin{tikzpicture}
	\card
	\cardstrip
	\cardbanner{banner/gold.png}
	\cardicon{icons/coin.png}
	\cardprice{5}
	\cardtitle{Waffenkiste}
	\cardcontent{Mit der ersten \emph{WAFFENKISTE}, die du in einem Zug spielst, kannst du nicht die vom Mitspieler genannte Karte nehmen. Mit der zweiten \emph{WAFFENKISTE} darfst du die dann genannte Karte nicht nehmen und die Karte, die der Mitspieler bei der ersten \emph{WAFFENKISTE} genannt hat, usw. Die Karte nimmst du aus dem Vorrat, und legst sie auf deinen Ablagestapel. Du darfst die genannten Karten trotzdem auf andere Arten und Weisen nehmen, aber nicht mit einer \emph{WAFFENKISTE}.

	\medskip

	Dein Mitspieler muss keine Karte aus dem Vorrat nennen; aber mit der \emph{WAFFENKISTE} nimmst du eine Karte vom Vorrat auf deinen Ablagestapel.}
\end{tikzpicture}
\hspace{-0.6cm}
\begin{tikzpicture}
	\card
	\cardstrip
	\cardbanner{banner/white.png}
	\cardicon{icons/coin.png}
	\cardprice{5}
	\cardtitle{\scriptsize{Wunderheilerin}}
	\cardcontent{Diese Karte macht aus \emph{FLÜCHEN} während des gesamten Spiels und in allen Situationen Geldkarten, als ob im unteren Bereich der Karte \enquote{\emph{FLUCH - GELD}} stünde. Sie können in der Kaufphase für +\coin[1] gespielt werden. Sie werden entsorgt, wenn du einen \emph{MÜNZER} nimmst und sie werden in deinen Handkarten bei einer \emph{MAGNATIN} mitgezählt. Mit den \emph{HÖFLINGEN} (aus \emph{Intrige 2. Edition}) erhältst du zwei Wahlmöglichkeiten, wenn du einen \emph{FLUCH} aufdeckst, usw. 
	
	\medskip

	\emph{FLÜCHE} sind aber auch weiterhin \emph{FLÜCHE} und zählen -1 \victorypoint bei Spielende.}
\end{tikzpicture}
\hspace{-0.6cm}
\begin{tikzpicture}
	\card
	\cardstrip
	\cardbanner{banner/white.png}
	\cardicon{icons/coin.png}
	\cardprice{6}
	\cardtitle{\scriptsize{Großer Markt}}
	\cardcontent{Du erhältst +1 Karte, +1 Aktion, +1 Kauf sowie +\coin[2].

	\medskip

	Wenn du diese Karte kaufen möchtest, darfst du zu diesem Zeitpunkt kein \emph{KUPFER} im Spiel haben. Wenn du zu einem früheren Zeitpunkt in deinem Zug \emph{KUPFER} im Spiel hattest, dieses aber entsorgt hast, darfst du den \emph{GROSSEN MARKT} kaufen. Kannst du den \emph{GROSSEN MARKT} auf andere Art nehmen, darfst du das jederzeit tun, auch wenn du \emph{KUPFER} im Spiel hast.}
\end{tikzpicture}
\hspace{-0.6cm}
\begin{tikzpicture}
	\card
	\cardstrip
	\cardbanner{banner/white.png}
	\cardicon{icons/coin.png}
	\cardprice{7}
	\cardtitle{Ausbau}
	\cardcontent{Entsorge eine beliebige Handkarte und nimm eine Karte vom Vorrat, die bis zu \coin[3] mehr kostet als die entsorgte Karte. Du darfst den Betrag nicht mit zusätzlichem \coin erhöhen. Hast du keine Karte auf der Hand, die du entsorgen kannst, darfst du dir keine Karte vom Vorrat nehmen. Den gespielten \emph{AUSBAU} selbst darfst du nicht entsorgen, da er sich nicht mehr auf deiner Hand befindet.}
\end{tikzpicture}
\hspace{-0.6cm}
\begin{tikzpicture}
	\card
	\cardstrip
	\cardbanner{banner/gold.png}
	\cardicon{icons/coin.png}
	\cardprice{7}
	\cardtitle{Bank}
	\cardcontent{Diese Karte ist eine Geldkarte mit einem variablen Wert: Pro Geldkarte (inklusive dieser / auch kombinierte Geldkarten), die du im Spiel hast, ist sie \coin[1] wert. Spielst du die \emph{BANK} als erste Geldkarte in deinem Zug, ist sie genau \coin[1] wert. Spielst du dagegen z.B. zuerst ein \emph{GOLD}, ein \emph{SILBER} und zwei \emph{KUPFER} und dann die \emph{BANK}, ist die \emph{BANK} \coin[5] wert. Spielst du im Anschluss noch eine \emph{BANK}, bleibt die erste \emph{BANK} \coin[5] wert, die zweite ist \coin[6] wert.}
\end{tikzpicture}
\hspace{-0.6cm}
\begin{tikzpicture}
	\card
	\cardstrip
	\cardbanner{banner/white.png}
	\cardicon{icons/coin.png}
	\cardprice{7}
	\cardtitle{Königshof}
	\cardcontent{Diese Karte ist ähnlich dem \emph{THRONSAAL} (aus dem \emph{Basisspiel}) - mit dem Unterschied, dass du die Aktionskarte, die du aus deiner Hand wählst, dreimal (statt zweimal) spielst. Lege die Aktionskarte ins Spiel, führe die Anweisungen darauf komplett aus, dann noch einmal und schließlich ein drittes Mal. Du kannst den Königshof auch spielen, ohne eine Aktionskarte dreimal zu spielen. Die Anweisungen werden auch dann dreimal ausgeführt (soweit möglich), wenn die Karte in der ersten oder zweiten Ausführung aus dem Spiel entfernt wird (z.B. weil sie sich selbst entsorgt). Für das dreimalige Spielen der Aktionskarte benötigst du keine Aktionen. Du darfst zwischen dem dreimaligen Spielen der Aktionskarte keine andere Aktion spielen, außer die Aktionskarte selbst gibt dazu die Anweisung. Wenn du mit einem \emph{KÖNIGSHOF} einen \emph{KÖNIGSHOF} spielst, wird der zweite \emph{KÖNIGSHOF} dreimal ausgeführt. Du darfst also erst eine Handkarte dreimal ausführen, danach eine weitere Handkarte (auch eine, die du mit der ersten gezogen hast) dreimal ausführen und anschließend eine dritte Handkarte dreimal ausführen. Du kannst aber nicht eine einzige Handkarte neunmal ausführen.}
\end{tikzpicture}
\hspace{-0.6cm}
\begin{tikzpicture}
	\card
	\cardstrip
	\cardbanner{banner/white.png}
	\cardicon{icons/coin.png}
	\cardprice{7}
	\cardtitle{\footnotesize{Kunstschmiede}}
	\cardcontent{Egal ob du keine Karte entsorgst (\coin[0] insgesamt) oder z.B. drei Karten, die jeweils \coin[2] kosten (\coin[6] insgesamt) - du \emph{musst} eine Karte vom Vorrat nehmen, die genau so viel kostet wie die entsorgten Karten zusammen gekostet haben, außer es ist keine entsprechende Karte im Vorrat vorhanden. Entsorgst du keine Karten und ist z.B. der \emph{KUPFER}-Stapel leer und es sind keine anderen Karten mit \coin[0]-Kosten im Vorrat, musst du dir einen \emph{FLUCH} vom Vorrat nehmen, der ebenfalls \coin[0] kostet. Andere Kosten (wie \potion aus \emph{Alchemisten} oder \hex aus \emph{Empires}) haben für die \emph{KUNSTSCHMIEDE} keine Auswirkung. Du darfst auch keine Karte nehmen, die andere Kosten enthält.}
\end{tikzpicture}
\hspace{-0.6cm}
\begin{tikzpicture}
	\card
	\cardstrip
	\cardbanner{banner/white.png}
	\cardicon{icons/coin.png}
	\cardprice{8*}
	\cardtitle{Hausiererin}
	\cardcontent{Diese Karte ist eine Karte mit variablen Kosten (siehe NEUE REGELN). Du erhältst +1 Karte, +1 Aktion sowie +\coin[1].

	\medskip

	Wenn du diese Karte in deiner Kaufphase kaufst, kostet sie für jede Aktionskarte, die du im Spiel hast, \coin[2] weniger, niemals allerdings weniger als \coin[0]. Kaufst du eine Karte außerhalb der Kaufphase (z.B. durch den \emph{SCHWARZMARKT}), kostet die \emph{HAUSIERERIN} \coin[8], egal ob du weitere Aktionskarten im Spiel hast oder nicht. Aktionskarten, die durch den \emph{THRONSAAL} (aus dem \emph{Basisspiel}) oder den \emph{KÖNIGSHOF} mehrfach gespielt wurden, sind trotzdem jeweils nur einmal im Spiel und reduzieren die Kosten einer \emph{HAUSIERERIN} um \coin[2].}
\end{tikzpicture}
\hspace{-0.6cm}
\begin{tikzpicture}
	\card
	\cardstrip
	\cardbanner{banner/gold.png}
	\cardicon{icons/coin.png}
	\cardprice{9}
	\cardtitle{Platin}
	\cardcontent{Diese Karte ist eine Basiskarte und keine Königreichkarte. Spielt ihr ausschließlich mit Königreichkarten aus \emph{Blütezeit}, wird diese Karte zusätzlich zu den Basis-Geldkarten \emph{KUPFER}, \emph{SILBER} und \emph{GOLD} in der Spielvorbereitung in den Vorrat gelegt. Bei Spielen mit Königreichkarten aus verschiedenen Editionen oder Erweiterungen entscheidet vor Spielbeginn, ob ihr \emph{PLATIN} (und damit auch die \emph{KOLONIE}) in den Vorrat legen wollt oder nicht (siehe SPIELVORBEREITUNG).}
\end{tikzpicture}
\hspace{-0.6cm}
\begin{tikzpicture}
	\card
	\cardstrip
	\cardbanner{banner/green.png}
	\cardicon{icons/coin.png}
	\cardprice{11}
	\cardtitle{Kolonien}
	\cardcontent{Diese Karte ist eine Basiskarte und keine Königreichkarte. Spielt ihr ausschließlich mit Königreichkarten aus \emph{Blütezeit}, wird diese Karte zusätzlich zu den Basis-Punktekarten \emph{ANWESEN}, \emph{HERZOGTUM} und \emph{PROVINZ} in der Spielvorbereitung in den Vorrat gelegt. Bei Spielen mit Königreichkarten aus verschiedenen Editionen oder Erweiterungen entscheidet vor Spielbeginn, ob ihr die \emph{KOLONIE} (und damit auch das \emph{PLATIN}) in den Vorrat legen wollt oder nicht. Achtet darauf, dass in diesem Fall das Spiel auch endet, wenn der Vorratsstapel \emph{KOLONIE} leer ist (siehe SPIELVORBEREITUNG).}
\end{tikzpicture}
\hspace{-0.6cm}
\begin{tikzpicture}
	\card
	\cardstrip
	\cardbanner{banner/white.png}
	\cardtitle{\scriptsize{Spielvorbereitung}\qquad}
	\cardcontent{Zum Spielen mit \emph{DOMINION-Blütezeit} benötigt ihr ein \emph{DOMINION-Basisspiel} oder das \emph{Basiskarten-Set}. Legt alle Basiskarten aus dem \emph{Basisspiel} (\emph{KUPFER}, \emph{SILSER}, \emph{GOLD} und \emph{ANWESEN}, \emph{HERZOGTUM} \emph{PROVINZ}, sowie \emph{FLÜCHE} und die Müllkarte bzw. das Mülltableau) wie gewohnt als Vorrat in die Tischmitte.

	\medskip

	Zusätzlich können die neuen Basiskarten \emph{PLATIN} (\emph{12 Karten}) und \emph{KOLONIE} (bei \emph{mind. 3 Spielern 12 Karten}, bei \emph{2 Spielern 8 Karten}) zum Vorrat in die Tischmitte gelegt werden. Dabei werden \emph{PLATIN} und \emph{KOLONIE} entweder gemeinsam oder gar nicht verwendet. Wir empfehlen, die neuen Basiskarten nur zu nutzen, wenn mindestens 4 Königreichkarten aus \emph{DOMINION-Blütezeit} verwendet werden. Alternativ könnt ihr die 10 Platzhalterkarten der Königreichkarten mischen und die oberste aufdecken. Ist sie aus \emph{DOMINION-Blütezeit}, verwendet ihr \emph{PLATIN} und \emph{KOLONIE} im Vorrat, ansonsten nicht.

	\medskip

	Neben dem Vorrat werden die Punktemarken, für alle gut erreichbar, bereitgelegt.
	
	\medskip
	
	Es gelten die Basisspielregeln für das Spielende mit folgender Ergänzung:

	\medskip

	In Spielen mit der neuen Punktekarte \emph{KOLONIE} endet das Spiel am Ende des Zuges eines Spielers auch, wenn der Vorratsstapel \emph{KOLONIE} aufgebraucht ist.}
\end{tikzpicture}
\hspace{-0.6cm}
\begin{tikzpicture}
	\card
	\cardstrip
	\cardbanner{banner/white.png}
	\cardtitle{\scriptsize{Neue Regeln (1/2)}\qquad}
	\cardcontent{Es gelten die Basisspielregeln mit folgenden Erweiterungen:

	\medskip

	\emph{In der Kaufphase}: Alle Geldkarten (auch kombinierte und Geldkarten mit zusätzlichen Anweisungen), die in einem Zug eingesetzt werden sollen, müssen \emph{vor dem ersten Kauf} und \emph{einzeln} gespielt werden. Der Spieler darf keine weiteren Geldkarten spielen, sobald er einen Kauf getätigt hat.

	\medskip
	
	\emph{Geldkarten mit zusätzlichen Anweisungen}: Die neuen Geldkarten \emph{AMBOSS}, \emph{BANK}, \emph{BRAUTKRONE}, \emph{GELDANLAGE}, \emph{HORT}, \emph{KRISTALLKUGEL}, \emph{SAMMELSURIUM}, \emph{STEINBRUCH}, und \emph{WAFFENKISTE} beinhalten zusätzliche Anweisungen, die beim Spielen (oder einem auf der Karte beschriebenen anderen Zeitpunkt) beachtet und ausgeführt werden müssen. Die Karten sind Königreichkarten und gehören nicht zu den Basis-Geldkarten, die in jedem Spiel benutzt werden. Trotzdem sind sie von Karteneffekten betroffen, die sich allgemein auf Geldkarten beziehen.}
\end{tikzpicture}
\hspace{-0.6cm}
\begin{tikzpicture}
	\card
	\cardstrip
	\cardbanner{banner/white.png}
	\cardtitle{\scriptsize{Neue Regeln (2/2)}\qquad}
	\cardcontent{\emph{Besonderheiten beim Kauf}: Die aufgedruckten Kosten enthalten bei manchen Karten nicht alle Informationen, die zum Kauf der Karte notwendig sind. Diese Karten haben einen * rechts oben an den Kosten, um dies anzuzeigen. Wer z.B. die \emph{HAUSIERERIN} kauft, muss dafür \coin[8] bezahlen. Durch vorher gespielte Aktionskarten kannst du diesen Wert allerdings reduzieren, d.h. du musst weniger dafür bezahlen. Das Sternchen * markiert diese Karte als Karte mit variablen Kosten.

	\smallskip

	Der \emph{GROSSE MARKT} hat eine extra Bedingung, die erfüllt sein muss, um ihn kaufen zu können: Der Spieler darf kein \emph{KUPFER} im Spiel haben.

	\smallskip

	Beachtet, dass es in anderen Erweiterungen (z.B. \emph{Reiche Ernte}) ebenfalls Karten mit
	* an den Kosten gibt. Diese haben allerdings eine andere Funktion.}
\end{tikzpicture}
\hspace{-0.6cm}
\begin{tikzpicture}
	\card
	\cardstrip
	\cardbanner{banner/white.png}
	\cardtitle{\footnotesize{Anweisungen (1/3)}\qquad}
	\cardcontent{\emph{+1}\victorypoint: Diese Anweisung gibt es auf den Karten \emph{BISCHOF}, \emph{DENKMAL}, \emph{GELDANLANGE} und \emph{SAMMELSURIUML}. Spielst du eine solche Karte aus, nimmst du dir entsprechend viele Punktemarker vom Vorrat. Du legst die Marker in deinen Spielbereich. Dort verbleiben sie bis zum Ende des Spiels und werden dann zu den anderen erspielten Siegpunkten addiert.

	\medskip

	\emph{Im Spiel}: Geld- und Aktionskarten, die du offen in deinem Spielbereich vor dir liegen hast (auch Dauerkarten z.B. aus \emph{Seaside}), befinden sich im Spiel, bis sie abgelegt werden. Nicht im Spiel befinden sich zur Seite gelegte und entsorgte Karten sowie alle Handkarten, Karten im Vorrat und in den Nachzieh- und Ablagestapeln. Auch Reaktionskarten, die als Reaktion aufgedeckt werden, befinden sich nicht im Spiel.

	\medskip

	\emph{Ablegen}: Karten werden immer von der Hand abgelegt, sofern nicht anders auf der Karte angegeben. Abgelegte Karten kommen offen auf den eigenen Ablagestapel. Legst du mehrere Karten gleichzeitig ab, musst du diese den Mitspielern nicht zeigen. Ggf. musst du aber die Anzahl der abgelegten Karten \enquote{nachweisen}, z.B. beim \emph{KELLER} (aus dem \emph{Basisspiel}). Lediglich die oberste Karte des Ablagestapels muss immer sichtbar sein.}
\end{tikzpicture}
\hspace{-0.6cm}
\begin{tikzpicture}
	\card
	\cardstrip
	\cardbanner{banner/white.png}
	\cardtitle{\footnotesize{Anweisungen (2/3)}\qquad}
	\cardcontent{\emph{Aufdecken}: Du deckst die Karte(n) auf, zeigst sie allen Mitspielern, führst eventuelle Arweisungen aus und legst sie dort hin zurück, von wo du sie hast. Eine aufgedeckte Handkarte wird wieder zurück auf die Hand genommen.

	\medskip

	\emph{Diese Karte}: Enthält eine Karte eine Anweisung, die sich auf \enquote{diese Karte} bezieht, ist normalerweise die Karte gemeint auf der die Anweisung steht, keine andere Karte, auf die innerhalb der Anweisung Bezug genommen wird. Eine Ausnahme sind die \emph{Wege} (aus \emph{Menagerie}), bei denen mit \enquote{diese Karte} die Aktionskarte gemeint ist, anstelle derer die Anweisung der Wege Karte ausgeführt wird.

	\smallskip

	Beispiel: Der Kartentext des \emph{WACHTURMS} besagt: \enquote{Wenn du eine Karte nimmst, darfst du diese Karte aus deiner Hand aufdecken, um entweder jene Karte zu entsorgen oder sie auf deinen Nachziehstapel zu legen.} Dies bedeutet, dass du die \emph{WACHTURM}-Karte aus deiner Hand aufdecken darfst, wenn du eine andere Karte nimmst.

	\medskip

	\emph{Jene Karten}: Enthält eine Karte eine Anweisung, die sich auf \enquote{jene Karten} bezieht, sind immer die Karten gemeint, auf die auf einer Karte (\enquote{dieser Karte}) Bezug genommen wird - es ist niemals die gerade genutzte Karte gemeint.}
\end{tikzpicture}
\hspace{-0.6cm}
\begin{tikzpicture}
	\card
	\cardstrip
	\cardbanner{banner/white.png}
	\cardtitle{\footnotesize{Anweisungen (3/3)}\qquad}
	\cardcontent{\emph{Eine gleiche Karte}: Nur Karten, die exakt denselben Namen tragen, gelten als gleiche Karten.

	\medskip

	\emph{Entsorgen}: Entsorgst du Karten, legst du sie offen auf den Müllstapel bzw. auf die Müllkarte, falls noch keine Karte entsorgt wurde. Entsorgte Karten können nicht wieder gekauft oder genommen werden, es sei denn eine Karte erlaubt dies.

	\medskip

	\emph{Nehmen}: Karten, die du durch Kauf oder eine Anweisung auf einer anderen Karte nimmst, nimmst du physisch an dich und fügst sie dadurch deinem Kartensatz hinzu. Genommene Karten legst du (soweit nicht anders auf der Karte angegeben) auf den Ablagestapel.}
\end{tikzpicture}
\hspace{-0.6cm}
\begin{tikzpicture}
	\card
	\cardstrip
	\cardbanner{banner/white.png}
	\cardtitle{\scriptsize{Empfohlene 10er Sätze\qquad}}
	\cardcontent{\emph{Anfänger:}\\
	Arbeiterdorf, Ausbau, Bank, Brautkrone, Buchhalterin, Denkmal, Gesindel, Kristallkugel, Magnatin, Wachturm

	\smallskip

	\emph{Freundliche Interaktion:}\\
	Arbeiterdorf, Bischof, Brautkrone, Gewölbe, Hausiererin, Hort, Kunstschmiede, Sammelsurium, Stadt, Waffenkiste

	\smallskip

	\emph{Das große Geld} (+ \textit{Basisspiel (2. Edition)}):\\
	Bank, Brautkrone, Großer Markt, Kristallkugel, Münzer, \textit{Geldverleiher}, \textit{Laboratorium}, \textit{Mine}, \textit{Töpferei}, \textit{Vorbotin}

	\smallskip

	\emph{Die Armee des Königs} (+ \textit{Basisspiel (2. Edition)}):\\
	Ausbau, Gesindel, Gewölbe, Königshof, Sammelsurium, \textit{Bürokrat}, \textit{Burggraben}, \textit{Dorf}, \textit{Händlerin}, \textit{Ratsversammlung}

	\smallskip

	\emph{Wege zum Sieg} (+ \textit{Die Intrige (2. Edition)}):\\
	Bischof, Denkmal, Hausiererin, Magnatin, Sammelsurium, \textit{Anbau}, \textit{Armenviertel}, \textit{Baron}, \textit{Handlanger}, \textit{Harem}

	\smallskip

	\emph{Die glücklichen Sieben} (+ \textit{Die Intrige (2. Edition)}):\\
	Ausbau, Bank, Brautkrone, Königshof, Kunstschmiede, \textit{Anbau}, \textit{Baron}, \textit{Bergwerk}, \textit{Patrouille}, \textit{Wunschbrunnen}}
\end{tikzpicture}
\hspace{-0.6cm}
\begin{tikzpicture}
	\card
	\cardstrip
	\cardbanner{banner/white.png}
	\cardtitle{\scriptsize{Empfohlene 10er Sätze\qquad}}
	\cardcontent{\emph{Explodierendes Königreich} (+ \textit{Seaside (2. Edition)}):\\
	Bischof, Großer Markt, Königshof, Stadt, Steinbruch, \textit{Ausguck}, \textit{Außenposten}, \textit{Fischerdorf}, \textit{Taktiker}, \textit{Werft}

	\smallskip

	\emph{Piratenbucht} (+ \textit{Seaside (2. Edition)}):\\
	Geldanalge, Hort, Magnatin, Münzer, Wunderheilerin, \textit{Affe}, \textit{Astrolabium}, \textit{Eingeborenendorf}, \textit{Korsarenschiff}, \textit{Schatzkammer}

	\smallskip

	\emph{Berufsausbildung} (+ \textit{Die Alchemisten}):\\
	Amboss, Arbeiterdorf, Bischof, Hausiererin, Münzer, Wunderheilerin, \textit{Leherling}, \textit{Universität}, \textit{Vertrauter}, \textit{Weinberg}

	\smallskip

	\emph{Umleitungen} (+ \textit{Reiche Ernte / Die Gilden}):\\
	Buchhalterin, Hort, Kristallkugel, Kunstschmiede, Magnatin, \textit{Bauerndorf}, \textit{Füllhorn}, \textit{Harlekin}, \textit{Nachbau}, \textit{Turnier}

	\smallskip

	\emph{Steinhauer} (+ \textit{Reiche Ernte / Die Gilden}):\\
	Ausbau, Großer Markt, Stadt, Steinbruch, Wunderheilerin, \textit{Bäcker}, \textit{Hellseherin}, \textit{Kaufmannsgilde}, \textit{Leuchtenmacher}, \textit{Metzger}}
\end{tikzpicture}
\hspace{-0.6cm}
\begin{tikzpicture}
	\card
	\cardstrip
	\cardbanner{banner/white.png}
	\cardtitle{\scriptsize{Empfohlene 10er Sätze\qquad}}
	\cardcontent{\emph{Müllabfuhr} (+ \textit{Dark Ages (mit Unterschlüfpen)}):\\
	Amboss, Kristallkugel, Magnatin, Stadt, Waffenkiste, \textit{Falschgeld}, \textit{Knappe}, \textit{Marktplatz}, \textit{Mundraub}, \textit{Raubzug}

	\smallskip

	\emph{Ganovenehre} (+ \textit{Dark Ages (mit Unterschlüpfen)}):\\
	Hort, Kunstschmiede, Sammelsurium, Steinbruch, Wachturm, \textit{Banditenlager}, \textit{Knappe}, \textit{Marodeur}, \textit{Prozession}, \textit{Schurke}

	\smallskip

	\emph{Der Letzte Wille} (+ \textit{Abenteuer}):\\
	Bischof, Denkmal, Gewölbe, Magnatin, Sammelsurium, \textit{Hafenstadt}, \textit{Königliche Münzen}, \textit{Kurier}, \textit{Relikt}, \textit{Verlies}, \textit{\underline{Erbschaft}}

	\smallskip

	\emph{Think Big} (+ \textit{Abenteuer}):\\
	Ausbau, Hausiererin, Hort, Königshof, Waffenkiste, \textit{Ferne Lande}, \textit{Gefolgsmann}, \textit{Geizhals}, \textit{Geschichtenerzähler}, \textit{Riese}, \textit{\underline{Ball}}, \textit{\underline{Überfahrt}}

	\smallskip

	\emph{Big Times} (+ \textit{Empires}):\\
	Bank, Brautkrone, Geldanalge, Großer Markt, Kunstschmiede, \textit{Gladiator/Reichtum}, \textit{Königlicher Schmied}, \textit{Patrizier/Handelsplatz}, \textit{Vermögen}, \textit{Villa}, \textit{\underline{Beherrschen}}, \textit{\underline{Obelisk}}

	\smallskip

	\emph{Geschmiedete Tore} (+ \textit{Empires}):\\
	Amboss, Hausiererin, Münzer, Sammelsurium, Waffenkiste, \textit{Feldlager/Diebesgut}, \textit{Gärtnerin}, \textit{Stadtviertel}, \textit{Wagenrennen}, \textit{Wilde Jagd}, \textit{\underline{Basilika}}, \textit{\underline{Palast}}}
\end{tikzpicture}
\hspace{-0.6cm}
\begin{tikzpicture}
	\card
	\cardstrip
	\cardbanner{banner/white.png}
	\cardtitle{\scriptsize{Empfohlene 10er Sätze\qquad}}
	\cardcontent{\emph{Schätze der Nacht} (+ \textit{Nocturne}):\\
	Brautkrone, Geldanalge, Kristallkugel, Waffenkiste, Wunderheilerin, \textit{Krypta}, \textit{Nachtwache}, \textit{Plünderer}, \textit{Vampirin}, \textit{Wächterin}

	\smallskip

	\emph{Ein Tag beim Pferderennen} (+ \textit{Nocturne}):\\
	Amboss, Bischof, Buchhalterin, Hausiererin, Wachturm, \textit{Druidin (Geschenk des Sumpfes, Geschenk des Flusses, Geschenk des Waldes)}, \textit{Folterknecht}, \textit{Friedhof}, \textit{Seliges Dorf}, \textit{Tragischer Held}

	\smallskip

	\emph{Traumtänzer} (+ \textit{Renaissance}):\\
	Arbeiterdorf, Denkmal, Gewölbe, Wachturm, Wunderheilerin, \textit{Alte Hexe}, \textit{Frachtschiff}, \textit{Gelehrter}, \textit{Priester}, \textit{Zepter}, \textit{\underline{Akademie}}

	\smallskip

	\emph{Geld regiert die Welt} (+ \textit{Renaissance}):\\
	Bank, Geldanalge, Gesindel, Großer Markt, Stadt, \textit{Forscherin}, \textit{Patron}, \textit{Schatzmeisterin}, \textit{Schwarzer Meister}, \textit{Versteck}, \textit{\underline{Kapitalismus}}, \textit{\underline{Zitadelle}}}
\end{tikzpicture}
\hspace{-0.6cm}
\begin{tikzpicture}
	\card
	\cardstrip
	\cardbanner{banner/white.png}
	\cardtitle{\scriptsize{Empfohlene 10er Sätze\qquad}}
	\cardcontent{\emph{Zeitlich limitiertes Angebot} (+ \textit{Menagerie}):\\
	Amboss, Arbeiterdorf, Großer Markt, Hausiererin, Münzer, \textit{Fischer}, \textit{Nachschub}, \textit{Schlachtross}, \textit{Vertreibung}, \textit{Wanderin}, \textit{\underline{Verzweiflung}}, \textit{\underline{Weg des Frosches}}

	\smallskip

	\emph{Otter-Chaos} (+ \textit{Menagerie}):\\
	Buchhalterin, Denkmal, Stadt, Steinbruch, Waffenkiste, \textit{Drahtzieher}, \textit{Jagdhütte}, \textit{Kamelzug}, \textit{Koppel}, \textit{Viehmarkt}, \textit{\underline{Reiche Ernte}}, \textit{\underline{Weg des Otters}}
	
	\smallskip

	\emph{Erfinderboom} (+ \textit{Verbündete}):\\
	Amboss, Ausbau, Gesindel, Königshof, Steinbruch, \textit{Augurinnen}, \textit{Hauptstadt}, \textit{Importeurin}, \textit{Schreinerin}, \textit{Tand}, \textit{\underline{Erfinderfamilie}}

	\smallskip

	\emph{Schmeicheln macht reich} (+ \textit{Verbündete}):\\
	Bank, Buchhalterin, Geldanalge, Gewölbe, Stadt, \textit{Irrfahrten}, \textit{Marquis}, \textit{Mittelsmann}, \textit{Ortschaft}, \textit{Schmeichler}, \textit{\underline{Bankiersverbund}}}
\end{tikzpicture}
\hspace{-0.6cm}
\begin{tikzpicture}
	\card
	\cardstrip
	\cardbanner{banner/white.png}
	\cardtitle{Platzhalter\quad}
\end{tikzpicture}
\hspace{0.6cm}

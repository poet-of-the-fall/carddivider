% Basic settings for this card set
\renewcommand{\cardcolor}{alchemy}
\renewcommand{\cardextension}{Erweiterung II}
\renewcommand{\cardextensiontitle}{Die Alchemisten}
\renewcommand{\seticon}{alchemy.png}

\clearpage
\newpage
\section{\cardextension \ - \cardextensiontitle \ (Rio Grande Games 2015)}

\begin{tikzpicture}
	\card
	\cardstrip
	\cardbanner{banner/green.png}
	\cardicon{icons/potion.png}
	\cardtitle{Weinberg}
	\cardcontent{Diese Karte ist eine Punktekarte und hat bis zum Ende des Spiels keine Funktion. Bei Spielende erhält der Spieler, der diese Karte in seinem Kartensatz (Nachziehstapel, Ablagestapel und Handkarten) hat, für jeweils 3 Aktionskarten (auch kombinierte Aktionskarten) 1 Siegpunkt. Es wird immer abgerundet, d.h. wer z.B. 12, 13 oder 14 Aktionskarten besitzt, erhält 4 Siegpunkte. Wer mehrere \emph{WEINBERGE} besitzt, erhält für jeden \emph{WEINBERG} die entsprechende Anzahl Siegpunkte.}
\end{tikzpicture}
\hspace{-0.6cm}
\begin{tikzpicture}
	\card
	\cardstrip
	\cardbanner{banner/white.png}
	\cardicon{icons/potion.png}
	\cardtitle{Verwandlung}
	\cardcontent{Wenn du diese Karte ausspielst und noch mindestens eine Karte auf der Hand hast, musst du eine Handkarte entsorgen. Wenn du keine Karte oder einen \emph{FLUCH} entsorgst, erhältst du nichts. Entsorgst du eine Aktions-, Punkte- oder Geldkarte, erhältst du den jeweiligen Bonus. Entsorgst du eine kombinierte Karte, erhältst du den Bonus beider Kartentypen. Sollte keine entsprechende Karte mehr im Vorrat sein, erhältst du nichts.}
\end{tikzpicture}
\hspace{-0.6cm}
\begin{tikzpicture}
	\card
	\cardstrip
	\cardbanner{banner/white.png}
	\cardicon{icons/coin.png}
	\cardprice{2}
	\cardtitle{\scriptsize{Kräuterkundiger}}
	\cardcontent{Wenn du den \emph{KRÄUTERKUNDIGEN} in der Aufräumphase ablegst, darfst du eine Geldkarte, die vor dir ausliegt, oben auf den Nachziehstapel legen. Ist der Nachziehstapel leer, legst du die Geldkarte auf den leeren Platz; sie ist dann die einzige Karte im Nachziehstapel. Wenn du mehrere \emph{KRÄUTERKUNDIGE} im Spiel hast und ablegst, darfst du für jeden eine ausliegende Geldkarte auf den Nachziehstapel legen.}
\end{tikzpicture}
\hspace{-0.6cm}
\begin{tikzpicture}
	\card
	\cardstrip
	\cardbanner{banner/white.png}
	\cardicon{icons/coin.png}
	\cardprice{2}
	\cardiconaddition{icons/potion.png}
	\cardtitle{\quad \footnotesize{Apotheker}}
	\cardcontent{Sollte der Nachziehstapel während des Aufdeckens aufgebraucht werden, mischst du deinen Ablagestapel und legst ihn als neuen Nachziehstapel bereit. Hast du dann trotzdem nicht genug Karten im Nachziehstapel, um 4 Karten aufzudecken, deckst du nur so viele Karten auf, wie möglich. Alle aufgedeckten \emph{KUPFER} und \emph{TRÄNKE} nimmst du auf die Hand und legst die anderen aufgedeckten Karten zurück auf den Nachziehstapel.}
\end{tikzpicture}
\hspace{-0.6cm}
\begin{tikzpicture}
	\card
	\cardstrip
	\cardbanner{banner/white.png}
	\cardicon{icons/coin.png}
	\cardprice{2}
	\cardiconaddition{icons/potion.png}
	\cardtitle{\quad \footnotesize{Universität}}
	\cardcontent{Du darfst eine Aktionskarte vom Vorrat nehmen, die bis zu \coin[5] kostet. Du darfst allerdings keine Karte nehmen, deren Kosten einen \potion beinhalten.}
\end{tikzpicture}
\hspace{-0.6cm}
\begin{tikzpicture}
	\card
	\cardstrip
	\cardbanner{banner/white.png}
	\cardicon{icons/coin.png}
	\cardprice{2}
	\cardiconaddition{icons/potion.png}
	\cardtitle{\quad Vision}
	\cardcontent{Alle Spieler, auch du selbst, decken die oberste Karte ihres Nachziehstapels auf. Für jeden Spieler entscheidest du separat, ob er die Karte ablegt oder zurück auf seinen Nachziehstapel legt. 

	\medskip

	Danach deckst du so lange Karten von deinem Nachziehstapel auf, bis du eine Karte, die \emph{keine} Aktionskarte ist, aufgedeckt hast (kombinierte Aktionskarten sind auch Aktionskarten). Nimm alle gerade aufgedeckten Karten auf die Hand. Hast du nur Aktionskarten aufgedeckt und dein Nachziehstapel ist aufgebraucht, mischst du deinen Ablagestapel und ziehst weiter, bis du eine Karte aufdeckst, die keine Aktionskarte ist. Findest du auch in diesem Stapel nur Aktionskarten, nimmst du alle Karten auf die Hand.}
\end{tikzpicture}
\hspace{-0.6cm}
\begin{tikzpicture}
	\card
	\cardstrip
	\cardbanner{banner/white.png}
	\cardicon{icons/coin.png}
	\cardprice{3}
	\cardiconaddition{icons/potion.png}
	\cardtitle{\quad Alchemist}
	\cardcontent{Wenn du zusätzlich zu dieser Karte einen \emph{TRANK} im Spiel hast, darfst du diese Karte in der Aufräumphase zurück auf den Nachziehstapel legen, statt sie abzulegen. Wenn du mindestens 1 \emph{TRANK} im Spiel hast, darfst du beliebig viele ausgespielte \emph{ALCHEMISTEN} zurück auf den Nachziehstapel legen.}
\end{tikzpicture}
\hspace{-0.6cm}
\begin{tikzpicture}
	\card
	\cardstrip
	\cardbanner{banner/gold.png}
	\cardicon{icons/coin.png}
	\cardprice{3}
	\cardiconaddition{icons/potion.png}
	\cardtitle{\quad \tiny{Stein der Weisen}}
	\cardcontent{Diese Karte ist eine Geldkarte mit einem variablen Wert und gehört nicht zu den Basiskarten (wie \emph{KUPFER} oder \emph{TRANK}), sondern zu den Königreichkarten. Ausgespielt wird sie aber – wie andere Geldkarten auch – in der Kaufphase. 

	\medskip

	\emph{Wichtig:} Geldkarten dürfen in der Kaufphase nur ausgespielt werden, bevor du die erste Karte kaufst (neue Regeln, S. 4).

	\medskip

	Zähle die Karten, die du in diesem Moment im Nachzieh- und Ablagestapel hast (Summe). Pro volle 5 Karten erhöht sich der Geldwert des \emph{STEIN DER WEISEN} für die Kaufphase in diesem Zug um \emph{1}. Spielst du mehrere \emph{STEIN DER WEISEN} aus, hat jede Karte den entsprechenden Geldwert. Du darfst beim Durchzählen der Karten deines Nachzieh- und Ablagestapels diese weder anschauen, noch ihre Reihenfolge verändern.}
\end{tikzpicture}
\hspace{-0.6cm}
\begin{tikzpicture}
	\card
	\cardstrip
	\cardbanner{banner/white.png}
	\cardicon{icons/coin.png}
	\cardprice{3}
	\cardiconaddition{icons/potion.png}
	\cardtitle{\quad \footnotesize{Vertrauter}}
	\cardcontent{Jeder Mitspieler, beginnend bei deinem linken Nachbarn, muss einen \emph{FLUCH} vom Vorrat nehmen und ihn ablegen. Wird der Vorrat an \emph{FLÜCHEN} dabei aufgebraucht, erhalten die Spieler, für die kein \emph{FLUCH} mehr vorhanden ist, nichts. }
\end{tikzpicture}
\hspace{-0.6cm}
\begin{tikzpicture}
	\card
	\cardstrip
	\cardbanner{banner/white.png}
	\cardicon{icons/coin.png}
	\cardprice{4}
	\cardiconaddition{icons/potion.png}
	\cardtitle{\quad Golem}
	\cardcontent{Decke solange Karten von deinem Nachziehstapel auf, bis du 2 Aktionskarten aufgedeckt hast, die keine \emph{GOLEMS} sind. Alle aufgedeckten \emph{GOLEMS} und Karten, die keine Aktionskarten sind, legst du ab. Hast du, auch nach dem Mischen des Ablagestapels, nur 1 oder gar keine Aktionskarte aufgedeckt, spielt du entsprechend weniger Aktionskarten aus. Du musst die aufgedeckten Aktionskarten ausspielen, darfst allerdings die Reihenfolge selbst bestimmen. Du darfst die Aktionskarten nicht auf die Hand nehmen. Anweisungen, die sich auf Handkarten beziehen, haben keine Auswirkungen auf die aufgedeckten Aktionskarten. Ist eine der aufgedeckten Karten z.B. ein \emph{THRONSAAL}, darfst du eine Karte aus der Hand auswählen und diese zwei Mal ausspielen, du darfst aber nicht die anderen aufgedeckten Karten dafür auswählen, da sie sich nicht auf deiner Hand befinden.}
\end{tikzpicture}
\hspace{-0.6cm}
\begin{tikzpicture}
	\card
	\cardstrip
	\cardbanner{banner/white.png}
	\cardicon{icons/coin.png}
	\cardprice{5}
	\cardtitle{Lehrling}
	\cardcontent{Entsorge eine Handkarte. Wenn du mindestens eine Handkarte hast, musst du eine Karte entsorgen. Pro \coin[X], das die entsorgte Karte kostet, ziehst du eine Karte nach. Wenn die entsorgte Karte außerdem \potion kostet, ziehst du weitere 2 Karten nach.}
\end{tikzpicture}
\hspace{-0.6cm}
\begin{tikzpicture}
	\card
	\cardstrip
	\cardbanner{banner/white.png}
	\cardicon{icons/coin.png}
	\cardprice{6}
	\cardiconaddition{icons/potion.png}
	\cardtitle{\quad \scriptsize{Besessenheit}}
	\cardcontent{\tiny{Zuerst spielst du deinen aktuellen Zug regulär zu Ende, bevor dein linker Nachbar einen Extra-Zug ausführen muss. Da die \emph{BESESSENHEIT} keine Angriffskarte ist, kann sich der Mitspieler nicht gegen den Extra-Zug \enquote{wehren}. 

	\medskip

	Zu Beginn des Extra-Zuges zeigt dein linker Nachbar dir seine Handkarten. Du entscheidest in diesem Zug alles für den Mitspieler – welche Aktionskarten und Geldkarten er spielt und welche Karten er kauft, nimmt, entsorgt etc. Du darfst alle Karten sehen, die auch der Mitspieler sieht – d. h. Handkarten, nachgezogene und angesehene Karten sowie die Karten, die der Mitspieler in der Aufräumphase des Extra-Zuges nachzieht.

	\medskip

	Alle Karten, die er nehmen, kaufen oder auf andere Art erhalten würde, erhältst stattdessen du und legst sie auf deinen Ablagestapel. Das betrifft auch Karten, die er auf die Hand nehmen oder anderweitig ablegen müsste. Alle Münzen (z. B. aus \emph{Die Gilden}) und Geldmarker (z. B. aus \emph{Seaside}), die der Spieler im \emph{BESESSENHEITS}-Zug erhält, bekommst du nicht. 

	\medskip

	Alle Karten, die der Spieler entsorgen müsste, werden separat neben den Müllstapel gelegt. Für weitere Anweisungen, die sich auf entsorgte Karten beziehen, gilt die Karte während des Extra-Zuges als entsorgt. Am Ende des Extra-Zuges legt der Mitspieler diese auf seinen eigenen Ablagestapel. 

	\medskip

	Alle Karten, die der Mitspieler während des Extra-Zuges in den Vorrat zurücklegen muss, werden tatsächlich in den Vorrat zurückgelegt.}}
\end{tikzpicture}
\hspace{-0.6cm}
\begin{tikzpicture}
	\card
	\cardstrip
	\cardbanner{banner/gold.png}
	\cardicon{icons/coin.png}
	\cardprice{4}
	\cardtitle{Trank}
\end{tikzpicture}
\hspace{-0.6cm}
\begin{tikzpicture}
	\card
	\cardstrip
	\cardbanner{banner/white.png}
	\cardtitle{\scriptsize{Empfohlene 10er Sätze\qquad}}
	\cardcontent{\emph{Verbotene Künste} (Alchemisten + \textit{Basisspiel}):\\
	Besessenheit, Lehrling, Universität, Vertrauter, \textit{Dieb}, \textit{Gärten}, \textit{Keller}, \textit{Laboratorium}, \textit{Ratsversammlung}, \textit{Thronsaal}

	\smallskip

	\emph{Quacksalber:} (Alchemisten + \textit{Basisspiel}):\\
	Alchemist, Apotheker, Golem, Kräuterkundiger, Verwandlung, \textit{Jahrmarkt}, \textit{Kanzler}, \textit{Keller}, \textit{Miliz}, \textit{Schmiede}

	\smallskip

	\emph{Chemiestunde:} (Alchemisten + \textit{Basisspiel}):\\
	Alchemist, Golem, Stein der Weisen, Universität, \textit{Burggraben}, \textit{Bürokrat}, \textit{Hexe}, \textit{Holzfäller}, \textit{Markt}, \textit{Umbau}

	\smallskip

	\emph{Diener:} (Alchemisten + \textit{Die Intrige}):\\
	Besessenheit, Golem, Verwandlung, Vision, Weinberg, \textit{Große Halle}, \textit{Handlanger}, \textit{Lakai}, \textit{Verschwörer}, \textit{Verwalter}

	\smallskip

	\emph{Geheime Forschungen:} (Alchemisten + \textit{Die Intrige}):\\
	Kräuterkundiger, Stein der Weisen, Universität, Vertrauter, \textit{Adlige}, \textit{Armenviertel}, \textit{Brücke}, \textit{Kerkermeister}, \textit{Lakai}, \textit{Maskerade}

	\smallskip

	\emph{Tröpfe, Tränke, Trottel:} (Alchemisten + \textit{Die Intrige}):\\
	Apotheker, Golem, Lehrling, Vision, \textit{Adlige}, \textit{Baron}, \textit{Eisenhütte}, \textit{Handelsposten}, \textit{Kupferschmied}, \textit{Wunschbrunnen}}
\end{tikzpicture}
\hspace{0.6cm}

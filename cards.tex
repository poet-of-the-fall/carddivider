\documentclass[a4paper]{article}
\usepackage[margin=0cm,top=2cm]{geometry}
\usepackage[utf8]{inputenc}
\usepackage[german]{babel}
\usepackage{microtype}
\usepackage{graphicx}
\usepackage{color}
\usepackage{xcolor}
\usepackage{tikz}
\usepackage{rotating}
\usepackage[scaled]{helvet}
\usepackage[babel,german=quotes]{csquotes}
\usepackage{fancyhdr}

%   Colors for basic elements
\definecolor{framebg}{RGB}{0,0,0}
\definecolor{contentbg}{RGB}{255,255,255}

%   Colors to distinguish the extensions
\definecolor{basicgame}{RGB}{152,170,149}
\definecolor{intrigue}{RGB}{225,106,85}
\definecolor{seaside}{RGB}{100,145,189}
\definecolor{alchemy}{RGB}{255,245,158}
\definecolor{prosperity}{RGB}{125,195,170}
\definecolor{cornucopia}{RGB}{154,199,119}
\definecolor{hinterlands}{RGB}{175,159,104}
\definecolor{darkages}{RGB}{198,58,69}
\definecolor{guilds}{RGB}{225,111,159}
\definecolor{adventures}{RGB}{199,186,152}
\definecolor{empires}{RGB}{105,181,103}
\definecolor{nocturne}{RGB}{59,75,189}
\definecolor{promo}{RGB}{241,177,59}

%  Change emph command to writing text bold
\DeclareTextFontCommand{\emph}{\boldmath\bfseries}

% Change section numbering to Alph
\renewcommand\thesection{\Alph{section}}

% Define Heading
\pagestyle{fancy}
\fancyhf{}
\fancyhead[C]{Dominion Card Dividers - Created by Danilo Gasdzik based on the works of Eiko Wagenknecht. \\ Dominion Kartentrenner - Erstellt von Danilo Gasdzik basierend auf der Arbeit von Eiko Wagenknecht. \\ \leftmark}
\fancyfoot[L]{\vspace{-8em}\tiny{\hspace{1cm}Entlang der schwarzen Linien ausschneiden. Je nach gewünschter\\\hspace{1cm}Größe der Kartentrenner die obere oder untere Linie wählen.}}
\fancyfoot[C]{\vspace{-8em}\thepage}
\fancyfoot[R]{\vspace{-8em}\tiny{Cut along the black edge. For your desired size choose the upper or lower line.\hspace{1cm}}}
\renewcommand{\headrulewidth}{0pt}

% Enum without indent
\newcounter{mycounter}  
\newenvironment{noindlist}
{\begin{list}{\arabic{mycounter}.~~}{\usecounter{mycounter} \labelsep=0em \labelwidth=2em \leftmargin=2em \itemindent=0em \itemsep=-0.5em \topsep=1em}}
{\end{list}}

% Itemize without spacing
\usepackage{enumitem}
\setlist[itemize]{leftmargin=2em, itemsep=-1em, topsep=1em}

% Shortcuts for often used icons
\newcommand{\coin}[1]{\includegraphics[height=1em]{banner/coin.png}\hspace{-0.7em}\raisebox{0.25em}{\tiny{#1}}\hspace{0.3em}}
\newcommand{\hex}[1]{\includegraphics[height=1em]{banner/hex.png}\hspace{-0.75em}\raisebox{0.25em}{\tiny{\textcolor{white}{#1}}}\hspace{0.3em}}
\newcommand{\victorypointtoken}[1]{\includegraphics[height=1em]{banner/victorypointtoken.png}\hspace{-0.65em}\raisebox{0.25em}{\tiny{\textcolor{white}{#1}}}\hspace{0.3em}}
\newcommand{\victorypoint}{\includegraphics[height=1em]{banner/victorypoint.png}}
\newcommand{\potion}{\includegraphics[height=1em]{banner/potion.png}}
\newcommand{\negativecardmarker}{\includegraphics[height=1em]{banner/negativecardmarker.png}}
\newcommand{\negativecoinmarker}{\includegraphics[height=1em]{banner/negativecoinmarker.png}}

%   Draw cards in portrait format
%   Card in portrait
%   ---------------------------------------

%   TikZ/PGF size settings for cards
\pgfmathsetmacro{\cardwidth}{6.3cm}
\pgfmathsetmacro{\cardheight}{11.1cm}
\pgfmathsetmacro{\bannerwidth}{3.2cm}
\pgfmathsetmacro{\iconwidth}{0.5cm}
\pgfmathsetmacro{\iconoffset}{0.175cm}
\pgfmathsetmacro{\stripwidth}{0.2cm}
\pgfmathsetmacro{\stripheight}{0.7cm}
\pgfmathsetmacro{\contentoffset}{0.25cm}
\pgfmathsetmacro{\contentoffsettop}{1.6cm}
\pgfmathsetmacro{\contentwidth}{\cardwidth-\contentoffset-\contentoffset}
\pgfmathsetmacro{\contentheight}{\cardheight-\contentoffset-\contentoffsettop}
\pgfmathsetmacro{\cardoffset}{0.5cm}

%   Stylings for elements
\tikzset{
    %   runde Ecken für die Karten
    cardcorners/.style={
        rounded corners=0.4cm
    }
}


%   Create card with black background, white box and extension on top right
\newcommand{\card}{
	\draw ([xshift=0.1cm,yshift=-\cardoffset]0,0) -- ([xshift=0.1cm,yshift=\cardheight+\cardoffset]0,0);
	\draw ([xshift=\cardwidth-0.1cm,yshift=-\cardoffset]0,0) -- ([xshift=\cardwidth-0.1cm,yshift=\cardheight+\cardoffset]0,0);
	\draw ([xshift=-\cardoffset,yshift=0.1cm]0,0) -- ([xshift=\cardwidth+\cardoffset,yshift=0.1cm]0,0);
	\draw ([xshift=-\cardoffset,yshift=\cardheight-0.1cm]0,0) -- ([xshift=\cardwidth+\cardoffset,yshift=\cardheight-0.1cm]0,0);
	\draw ([xshift=-\cardoffset,yshift=\cardheight-0.3cm]0,0) -- ([xshift=\cardwidth+\cardoffset,yshift=\cardheight-0.3cm]0,0);
	\node [anchor=south west,fill=framebg,minimum width=\cardwidth,minimum height=\cardheight] (main) at (0,0) {};
	\node [anchor=south west,fill=contentbg,minimum width=\contentwidth,minimum height=\contentheight,cardcorners] (content) at ([yshift=\contentoffset,xshift=\contentoffset]0,0) {};
	\node [anchor=south east,\cardcolor,minimum width=\stripwidth,minimum height=\stripheight] at ([xshift=-0.6*\stripwidth,yshift=\contentoffset]content.north east) {\scriptsize{\textsc{\cardextension}}};
	\node [anchor=south east,\cardcolor,minimum width=\stripwidth,minimum height=\stripheight] at ([xshift=-0.6*\stripwidth]content.north east) {\scriptsize{\textsc{\cardextensiontitle}}};
}

%   Add banner on top left
\newcommand{\cardbanner}[1]{
	\node [anchor=south west] (cardbanner) at ([yshift=0.2*\contentoffset]content.north west) {
		\includegraphics[width=\bannerwidth]{#1}
	};
}

%   Add icon to banner
\newcommand{\cardicon}[1]{
	\node [anchor=south west] (cardicon) at ([xshift=\iconoffset,yshift=\iconoffset]cardbanner.south west) {
		\includegraphics[width=\iconwidth]{#1}
	};
}

%   Add price to icon
\newcommand{\cardprice}[1]{
	\node [] at (cardicon) {
		\small{
			\textsc{#1}
		}
	};
}


%   Add second icon
\newcommand{\cardiconaddition}[1]{
	\node [anchor=south west] (cardiconaddition) at ([xshift=\iconwidth]cardicon.south west) {
		\includegraphics[width=\iconwidth]{#1}
	};
}

%   Add price to second icon
\newcommand{\cardpriceaddition}[1]{
	\node [] at (cardiconaddition) {
		\small{
			\textsc{#1}
		}
	};
}

%   Add title of the card to banner
\newcommand{\cardtitle}[1]{
	\node [] at ([xshift=0.55*\iconwidth,yshift=1.1*\iconwidth]cardbanner.south) {
		\small{
			\textsc{#1}
		}
	};
}

%   Add colored strip on top right
\newcommand{\cardstrip}{
	\node [anchor=north east,fill=\cardcolor,minimum width=\stripwidth,minimum height=\stripheight] at ([xshift=-\contentoffset]main.north east) {};
}

%   Write content into white area
\newcommand{\cardcontent}[1]{
	\node [] at (content) {
		\rotatebox{270}{
			\parbox[t][\contentwidth][c]{\contentheight}{
				\begin{minipage}{\contentheight}
					\centering
					\begin{minipage}{0.9\textwidth}
						\raggedright
						\scriptsize{
							\textsf{#1}
						}
					\end{minipage}
				\end{minipage}
			}
		}
	};
}
%   Draw cards in landscape format
%\input{format/tikzcardslandscape.tex}

% Basic settings for this card set will be overwritten on each card set
\newcommand{\cardcolor}{}
\newcommand{\cardextension}{}
\newcommand{\cardextensiontitle}{}

\begin{document}
	\begin{center}
		\begin{minipage}{0.75\textwidth}
			\Huge 
			Dominion Kartentrenner
			\normalsize
			\tableofcontents
		\end{minipage}
	    	
	    % Basic settings for this card set
\renewcommand{\cardcolor}{}
\renewcommand{\cardextension}{}
\renewcommand{\cardextensiontitle}{}
\renewcommand{\seticon}{empty.png}

\clearpage
\newpage
\section{Anleitung und Grundausstattung}

\begin{tikzpicture}
	\card
	\cardbanner{banner/white.png}
	\cardtitle{Anleitung (1)\quad}
	\cardcontent{\tiny{\emph{Spielablauf:} Dominion wird zugweise gespielt. Der Spieler an der Reihe hat am Beginn seines Zuges normalerweise 5 Karten auf der Hand. Er führt nun seinen Zug aus, der aus den 3 folgenden Phasen besteht, die immer in dieser Reihenfolge gespielt werden müssen.

	\medskip

	\emph{1. Phase:} Aktion - Der Spieler \emph{darf} Aktionskarten ausspielen.

	\emph{2. Phase:} Kauf - Der Spieler \emph{darf} Karten kaufen.

	\emph{3. Phase:} Aufräumen - Der Spieler \emph{muss} alle  ausgespielten \emph{und} alle Handkarten offen auf seinen Ablagestapel legen und \emph{sofort} 5 Karten für den nächsten Zug nachziehen.

	\medskip

	Die 1. und die 2. Phase darf, die 3. Phase muss gespielt werden. Wenn der Spieler seinen Zug beendet hat, ist der nächste Spieler an der Reihe. Das Spiel verläuft in dieser Weise bis Spielende.
	
	\medskip

	\begin{tabbing}
		Stapel der anderen Spieler: \= Links \kill
		Eigener Ablagestapel: \> Darf weder durchgezählt noch durchgesehen werden. \\
		Eigener Nachziehstapel: \> Darf durchgezählt, nicht aber durchgesehen werden. \\
		Stapel der anderen Spieler: \> Dürfen weder durchgezählt noch durchgesehen werden. \\
		Stapel im Vorrat: \> Dürfen jederzeit durchgezählt und durchgesehen werden. \\
		Müllstapel: \> Darf jederzeit durchgezählt und durchgesehen werden. \\
	\end{tabbing}}}
\end{tikzpicture}
\hspace{-0.6cm}
\begin{tikzpicture}
	\card
	\cardbanner{banner/white.png}
	\cardtitle{Anleitung (2)\quad}
	\cardcontent{\tiny{\begin{tabbing}
	Spielende:xxx \=  18x Provinzen,xxx \= 12 andere Punktekarten,xxx \= 50 Flüche,xxx \= Geld: \kill
	2 Spieler: \> 8x Provinzen, \> 8 andere Punktekarten, \> 10 Flüche, \> Geld: 46 K, 40 S, 30 G\\
	3 Spieler: \> 12x Provinzen, \> 12 andere Punktekarten, \> 20 Flüche, \> Geld: 39 K, 40 S, 30 G\\
	4 Spieler: \> 12x Provinzen, \> 12 andere Punktekarten, \> 30 Flüche, \> Geld: 32 K, 40 S, 30 G\\
	5 Spieler: \> 15x Provinzen, \> 12 andere Punktekarten, \> 40 Flüche, \> Geld: 85 K, 80 S, 60 G\\
	6 Spieler: \> 18x Provinzen, \> 12 andere Punktekarten, \> 50 Flüche, \> Geld: 78 K, 80 S, 60 G\\
	Spielende: \> Provinzstapel, Kolonienstapel (Dominion – Blütezeit) \emph{oder} \\
					\>3 Stapel (1 - 4 Spieler) bzw. 4 Stapel (5 - 6 Spieler) aus dem Vorrat leer \\
	\end{tabbing}
	Punktekarten aus Erweiterungen werden in Anzahl der \enquote{anderen Punktekarten} ausgelegt. Zur Spielende-Bedingung zählen alle Karten im Vorrat, also auch Fluch-, Geld- und Punktekarten, nicht jedoch z. B. der Müllstapel.

	\medskip

	\emph{Platin und Kolonie (Dominion – Blütezeit):} Die im Spiel befindlichen Königreich-Platzhalterkarten werden gemischt. Ist die erste gezogene Königreichskarte eine Karte aus der Blütezeit-Erweiterung, so wird mit Platin und Kolonie gespielt, ansonsten ohne.

	\medskip

	\emph{Unterschlupf-Karten (Dominion – Dark Ages):} Die im Spiel befindlichen Königreich-Platzhalterkarten werden gemischt. Ist die erste gezogene Königreichskarte eine Karte aus der Dark Ages-Erweiterung, so startet jeder Spieler mit 7 Kupfer, 1 Hütte, 1 Totenstadt und 1 Verfallenes Anwesen, andernfalls erhält jeder Spieler 7 Kupfer und 3 Anwesen.}}
\end{tikzpicture}
\hspace{-0.6cm}
\begin{tikzpicture}
	\card
	\cardbanner{banner/white.png}
	\cardtitle{Platzhalter\quad}
\end{tikzpicture}
\hspace{-0.6cm}
\begin{tikzpicture}
	\card
	\cardbanner{banner/gold.png}
	\cardicon{icons/coin.png}
	\cardprice{0}
	\cardtitle{Kupfer}
\end{tikzpicture}
\hspace{-0.6cm}
\begin{tikzpicture}
	\card
	\cardbanner{banner/gold.png}
	\cardicon{icons/coin.png}
	\cardprice{3}
	\cardtitle{Silber}
\end{tikzpicture}
\hspace{-0.6cm}
\begin{tikzpicture}
	\card
	\cardbanner{banner/gold.png}
	\cardicon{icons/coin.png}
	\cardprice{6}
	\cardtitle{Gold}
\end{tikzpicture}
\hspace{-0.6cm}
\begin{tikzpicture}
	\card
	\cardbanner{banner/green.png}
	\cardicon{icons/coin.png}
	\cardprice{2}
	\cardtitle{Anwesen}
\end{tikzpicture}
\hspace{-0.6cm}
\begin{tikzpicture}
	\card
	\cardbanner{banner/green.png}
	\cardicon{icons/coin.png}
	\cardprice{5}
	\cardtitle{Herzogtum}
\end{tikzpicture}
\hspace{-0.6cm}
\begin{tikzpicture}
	\card
	\cardbanner{banner/green.png}
	\cardicon{icons/coin.png}
	\cardprice{8}
	\cardtitle{Provinz}
\end{tikzpicture}
\hspace{-0.6cm}
\begin{tikzpicture}
	\card
	\cardbanner{banner/purple.png}
	\cardicon{icons/coin.png}
	\cardprice{0}
	\cardtitle{Fluch}
\end{tikzpicture}	
\hspace{0.6cm}

	    \input{sets/de/basic1.tex}
	    % Basic settings for this card set
\renewcommand{\cardcolor}{basicgame}
\renewcommand{\cardextension}{Special Edition}
\renewcommand{\cardextensiontitle}{Das Basisspiel}
\renewcommand{\seticon}{basic1.png}

\clearpage
\newpage
\section{\cardextension \ - \cardextensiontitle \ (Rio Grande Games 2013)}

\begin{tikzpicture}
	\card
	\cardstrip
	\cardbanner{banner/blue.png}
	\cardicon{icons/coin.png}
	\cardprice{2}
	\cardtitle{Burggraben}
	\cardcontent{Spielt ein anderer Spieler eine Angriffskarte (mit der Aufschrift AKTION -- ANGRIFF), kannst du die Karte \emph{BURGGRABEN} vorzeigen, falls du sie in diesem Moment auf der Hand hast. In diesem Fall bist du von den Auswirkungen des Angriffs nicht betroffen, d.h. du musst bei der \emph{HEXE} keine Fluchkarte nehmen usw. Haben mehrere Spieler einen \emph{BURGGRABEN} auf der Hand, dürfen diese auch eingesetzt und vorgezeigt werden. Danach nehmen die Spieler ihre Karte zurück auf die Hand. Der Spieler, der den Angriff gespielt hat, darf unabhängig davon, ob ein oder mehrere \emph{BURGGRÄBEN} gespielt werden, die weiteren Anweisungen seiner Aktionskarte ausführen. Der \emph{BURGGRABEN} darf auch in der eigenen Aktionsphase gespielt werden – dann ziehst du 2 Karten nach.}
\end{tikzpicture}
\hspace{-0.6cm}
\begin{tikzpicture}
	\card
	\cardstrip
	\cardbanner{banner/white.png}
	\cardicon{icons/coin.png}
	\cardprice{2}
	\cardtitle{Keller}
	\cardcontent{Der ausgespielte \emph{KELLER} selbst darf nicht abgelegt werden, da er sich nicht mehr in deiner Hand befindet. Sage an, wie viele Karten du ablegst und lege diese auf deinen Ablagestapel. Danach ziehst du die gleiche Anzahl Karten vom Nachziehstapel. Sollte während dieses Vorgangs der Nachziehstapel aufgebraucht werden, wird dein Ablagestapel zusammen mit den soeben abgelegten Karten gemischt und als neuer Nachziehstapel bereitgelegt.}
\end{tikzpicture}
\hspace{-0.6cm}
\begin{tikzpicture}
	\card
	\cardstrip
	\cardbanner{banner/white.png}
	\cardicon{icons/coin.png}
	\cardprice{3}
	\cardtitle{Dorf}
	\cardcontent{Spielst du mehrere \emph{DÖRFER} hintereinander, zählst du am besten laut mit, wie viele Aktionen du noch ausspielen darfst, damit du den Überblick behältst.}
\end{tikzpicture}
\hspace{-0.6cm}
\begin{tikzpicture}
	\card
	\cardstrip
	\cardbanner{banner/white.png}
	\cardicon{icons/coin.png}
	\cardprice{3}
	\cardtitle{Werkstatt}
	\cardcontent{Nimm dir eine Karte aus dem Vorrat und lege diese sofort auf deinen Ablagestapel. Du kannst weder Geldkarten noch zusätzlich über Aktionskarten erhaltenes Geld oder Münzen (bei Erweiterungen mit Münzen) einsetzen, um den angegebenen Betrag auf der Karte zu erhöhen.}
\end{tikzpicture}
\hspace{-0.6cm}
\begin{tikzpicture}
	\card
	\cardstrip
	\cardbanner{banner/white.png}
	\cardicon{icons/coin.png}
	\cardprice{3}
	\cardtitle{Holzfäller}
	\cardcontent{In der Kaufphase darfst du 1 zusätzliche Karte kaufen, also insgesamt 2 Käufe tätigen. Für deine Käufe stehen dir in diesem Zug insgesamt 2 Geld zusätzlich zur Verfügung.}
\end{tikzpicture}
\hspace{-0.6cm}
\begin{tikzpicture}
	\card
	\cardstrip
	\cardbanner{banner/white.png}
	\cardicon{icons/coin.png}
	\cardprice{4}
	\cardtitle{Schmiede}
	\cardcontent{Du musst 3 Karten von deinem Nachziehstapel ziehen und auf die Hand nehmen.}
\end{tikzpicture}
\hspace{-0.6cm}
\begin{tikzpicture}
	\card
	\cardstrip
	\cardbanner{banner/white.png}
	\cardicon{icons/coin.png}
	\cardprice{4}
	\cardtitle{Umbau}
	\cardcontent{Der ausgespielte \emph{UMBAU} selbst darf nicht entsorgt werden, da er sich nicht mehr in deiner Hand befindet. Weitere \emph{UMBAU}-Karten in deiner Hand dürfen entsorgt werden. Wenn du keine Karte zum Entsorgen auf der Hand hast, darfst du dir auch keine neue Karte nehmen. Die neue Karte, die du dir nimmst, darf maximal bis zu \coin[2] mehr als die entsorgte Karte kosten. Der Betrag darf weder durch weitere Geldkarten, Münzen oder zusätzliches Geld von anderen Aktionskarten erhöht werden. Die neue Karte kann die gleiche Karte sein wie die, die du entsorgt hast. Lege die neue Karte auf deinen Ablagestapel.}
\end{tikzpicture}
\hspace{-0.6cm}
\begin{tikzpicture}
	\card
	\cardstrip
	\cardbanner{banner/white.png}
	\cardicon{icons/coin.png}
	\cardprice{4}
	\cardtitle{Miliz}
	\cardcontent{Deine Mitspieler müssen Karten aus ihrer Hand ablegen, bis sie nur noch 3 Karten auf der Hand haben. Spieler, die zum Zeitpunkt des Angriffs bereits 3 oder weniger Karten auf der Hand haben, müssen keine weiteren Karten ablegen.}
\end{tikzpicture}
\hspace{-0.6cm}
\begin{tikzpicture}
	\card
	\cardstrip
	\cardbanner{banner/white.png}
	\cardicon{icons/coin.png}
	\cardprice{5}
	\cardtitle{Markt}
	\cardcontent{Du musst eine Karte vom Nachziehstapel auf die Hand nehmen. Du \emph{darfst} in der Aktionsphase eine weitere Aktionskarte ausspielen. Du \emph{darfst} in der Kaufphase einen zusätzlichen Kauf tätigen und hast dafür ein zusätzliches Geld zur Verfügung.}
\end{tikzpicture}
\hspace{-0.6cm}
\begin{tikzpicture}
	\card
	\cardstrip
	\cardbanner{banner/white.png}
	\cardicon{icons/coin.png}
	\cardprice{5}
	\cardtitle{Mine}
	\cardcontent{\emph{Errata:} Der Kartentext ist falsch, es sollte \enquote{Du \emph{darfst} eine beliebige Geldkarte aus der Hand entsorgen. Nimm eine Geldkarte vom Vorrat auf die Hand, die bis zu \coin[3] mehr kostet.} heißen.

	\medskip

	Normalerweise entsorgst du ein Kupfer und nimmst dir dafür ein Silber, oder du entsorgst ein Silber und nimmst dir ein Gold. Du kannst dir aber auch eine gleichwertige oder billigere Karte nehmen. Die neue Karte nimmst du sofort auf die Hand und darfst sie noch während deines Zuges einsetzen. Wer keine Geldkarte zum Entsorgen hat, erhält keine neue Karte.}
\end{tikzpicture}
\hspace{-0.6cm}
\begin{tikzpicture}
	\card
	\cardstrip
	\cardbanner{banner/white.png}
	\cardicon{icons/coin.png}
	\cardprice{6}
	\cardtitle{Abenteurer}
	\cardcontent{Sollte dein Nachziehstapel während des Aufdeckens aufgebraucht werden, mische deinen Ablagestapel. Die bereits aufgedeckten Karten werden nicht mit gemischt, sondern bleiben zunächst offen liegen. Solltest du auch mit Hilfe des neuen Nachziehstapels nicht genügend Geldkarten aufdecken, bekommst du nur diejenigen Geldkarten, die du aufgedeckt hast.}
\end{tikzpicture}
\hspace{-0.6cm}
\begin{tikzpicture}
	\card
	\cardstrip
	\cardbanner{banner/white.png}
	\cardicon{icons/coin.png}
	\cardprice{5}
	\cardtitle{Bibliothek}
	\cardcontent{Aktionskarten darfst du zur Seite legen, sobald du sie ziehst, musst dies aber nicht tun. Hast du bereits 7 oder mehr Karten auf der Hand, wenn du die \emph{BIBLIOTHEK} ausspielst, ziehst du keine Karten nach. Wenn dein Nachziehstapel während des Ziehens aufgebraucht ist, mischst du den Ablagestapel, mischst aber die zur Seite gelegten Aktionskarten nicht mit ein. Diese werden erst auf den Ablagestapel gelegt, sobald du 7 Karten auf der Hand hast. Sollten die Karten nicht reichen, ziehst du nur so viele Karten wie möglich.}
\end{tikzpicture}
\hspace{-0.6cm}
\begin{tikzpicture}
	\card
	\cardstrip
	\cardbanner{banner/white.png}
	\cardicon{icons/coin.png}
	\cardprice{4}
	\cardtitle{Bürokrat}
	\cardcontent{Ist dein Nachziehstapel aufgebraucht, wenn du diese Karte spielst, legst du die Silberkarte verdeckt ab. Sie bildet dann deinen Nachziehstapel. Das Gleiche gilt für alle Mitspieler, die eine Punktekarte verdeckt auf den eigenen Nachziehstapel legen müssen.}
\end{tikzpicture}
\hspace{-0.6cm}
\begin{tikzpicture}
	\card
	\cardstrip
	\cardbanner{banner/white.png}
	\cardicon{icons/coin.png}
	\cardprice{4}
	\cardtitle{Dieb}
	\cardcontent{Jeder Mitspieler legt die beiden aufgedeckten Karten zunächst offen vor sich ab. Wer nur noch 1 Karte im Nachziehstapel hat, legt diese vor sich ab und mischt erst dann seinen Ablagestapel. Hat ein Spieler nach dem Mischen noch immer nicht genug Karten, deckt er nur so viele auf wie möglich. Hat ein Spieler 2 Geldkarten offen liegen, wählst du eine davon aus, die der Spieler entsorgen muss. Die andere Karte legt er auf seinen Ablagestapel. Hat ein Spieler 1 Geldkarte offen liegen, muss er diese entsorgen. Hat ein Spieler keine Geldkarte aufgedeckt, muss er keine Karte entsorgen. Von den auf diese Weise entsorgten Karten darfst du eine beliebige Anzahl nehmen.}
\end{tikzpicture}
\hspace{-0.6cm}
\begin{tikzpicture}
	\card
	\cardstrip
	\cardbanner{banner/white.png}
	\cardicon{icons/coin.png}
	\cardprice{4}
	\cardtitle{Festmahl}
	\cardcontent{Du nimmst dir eine beliebige Karte aus dem Vorrat, die höchstens \coin[5] kostet und legst sie sofort auf deinen Ablagestapel. Du darfst den Betrag weder durch weitere Geldkarten, Münzen oder zusätzliches Geld von anderen Aktionskarten erhöhen. Spielst du das \emph{FESTMAHL} direkt nach dem \emph{THRONSAAL}, erhältst du 2 Karten, obwohl du das \emph{FESTMAHL} nur einmal entsorgen kannst.}
\end{tikzpicture}
\hspace{-0.6cm}
\begin{tikzpicture}
	\card
	\cardstrip
	\cardbanner{banner/green.png}
	\cardicon{icons/coin.png}
	\cardprice{4}
	\cardtitle{Gärten}
	\cardcontent{Diese Karte ist die einzige Punktekarte unter den Königreichkarten. Sie hat bis zum Ende des Spiels keine Funktion. Bei der Wertung des Spiels erhält der Spieler, der diese Karte in seinem Kartensatz (Nachziehstapel, Handkarten und Ablagestapel) hat, für jeweils 10 Karten einen Siegpunkt. Es wird immer abgerundet, d.h. 39 Karten ergeben 3 Siegpunkte, ebenso wie 31 Karten 3 Siegpunkte ergeben. Wer mehrere \emph{GÄRTEN} besitzt, erhält jeden \emph{GARTEN} die entsprechende Anzahl an Siegpunkten.}
\end{tikzpicture}
\hspace{-0.6cm}
\begin{tikzpicture}
	\card
	\cardstrip
	\cardbanner{banner/white.png}
	\cardicon{icons/coin.png}
	\cardprice{4}
	\cardtitle{\footnotesize{Geldverleiher}}
	\cardcontent{\emph{Errata:} Der Kartentext ist falsch, es sollte \enquote{Du \emph{darfst} ein Kupfer aus der Hand entsorgen. Wenn du das tust: + \coin[3].} heißen.

	\medskip

	Wenn du kein Kupfer zum Entsorgen auf der Hand hast, erhältst du kein zusätzliches Geld für die Kaufphase.}
\end{tikzpicture}
\hspace{-0.6cm}
\begin{tikzpicture}
	\card
	\cardstrip
	\cardbanner{banner/white.png}
	\cardicon{icons/coin.png}
	\cardprice{5}
	\cardtitle{Hexe}
	\cardcontent{Wenn du die \emph{HEXE} spielst und nicht mehr genügend Fluchkarten vorrätig sind, werden diese im Uhrzeigersinn (beginnend mit deinem linken Nachbarn) verteilt. Die Mitspieler legen die Fluchkarten sofort auf ihren Ablagestapel. Du ziehst immer 2 Karten von deinem Nachziehstapel, auch wenn keine Fluchkarten mehr im Vorrat sind.}
\end{tikzpicture}
\hspace{-0.6cm}
\begin{tikzpicture}
	\card
	\cardstrip
	\cardbanner{banner/white.png}
	\cardicon{icons/coin.png}
	\cardprice{5}
	\cardtitle{Jahrmarkt}
	\cardcontent{Spielst du mehrere \emph{JAHRMÄRKTE} hintereinander, zählst du am besten laut mit, wie viele Aktionen du noch ausspielen darfst, damit du den Überblick behältst.}
\end{tikzpicture}
\hspace{-0.6cm}
\begin{tikzpicture}
	\card
	\cardstrip
	\cardbanner{banner/white.png}
	\cardicon{icons/coin.png}
	\cardprice{3}
	\cardtitle{Kanzler}
	\cardcontent{Legst du deinen Nachziehstapel auf deinen Ablagestapel, musst du dies tun, bevor du eine weitere Aktion ausspielst oder zur Kaufphase übergehst. Du darfst deinen Nachziehstapel nicht einsehen, bevor du ihn ablegst.}
\end{tikzpicture}
\hspace{-0.6cm}
\begin{tikzpicture}
	\card
	\cardstrip
	\cardbanner{banner/white.png}
	\cardicon{icons/coin.png}
	\cardprice{2}
	\cardtitle{Kapelle}
	\cardcontent{Die ausgespielte \emph{KAPELLE} selbst darf nicht entsorgt werden, da sie sich nicht mehr auf der Hand befindet. Weitere \emph{KAPELLEN} auf der Hand dürfen entsorgt werden.}
\end{tikzpicture}
\hspace{-0.6cm}
\begin{tikzpicture}
	\card
	\cardstrip
	\cardbanner{banner/white.png}
	\cardicon{icons/coin.png}
	\cardprice{5}
	\cardtitle{\footnotesize{Laboratorium}}
	\cardcontent{Du \emph{musst} zuerst zwei Karten vom Nachziehstapel auf die Hand nehmen. Dann \emph{darfst} du eine weitere Aktionskarte ausspielen.}
\end{tikzpicture}
\hspace{-0.6cm}
\begin{tikzpicture}
	\card
	\cardstrip
	\cardbanner{banner/white.png}
	\cardicon{icons/coin.png}
	\cardprice{5}
	\cardtitle{\scriptsize{Ratsversammlung}}
	\cardcontent{Jeder Mitspieler \emph{muss} eine Karte von seinem Nachziehstapel auf die Hand nehmen.}
\end{tikzpicture}
\hspace{-0.6cm}
\begin{tikzpicture}
	\card
	\cardstrip
	\cardbanner{banner/white.png}
	\cardicon{icons/coin.png}
	\cardprice{4}
	\cardtitle{Spion}
	\cardcontent{\emph{Zuerst} musst du eine Karte vom Nachziehstapel auf die Hand nehmen. Dann \emph{muss} jeder Spieler (auch du) die oberste Karte seines Nachziehstapels aufdecken. Du entscheidest für jeden Spieler, ob die aufgedeckte Karte auf den Nachziehstapel zurück- oder auf den Ablagestapel abgelegt werden soll. Spieler, deren Nachziehstapel aufgebraucht ist, mischen ihren Ablagestapel und legen ihn als neuen Nachziehstapel bereit. Nur wer weder einen Nachzieh- noch einen Ablagestapel hat, braucht keine Karte aufzudecken. Ist den Mitspielern die Reihenfolge des Aufdeckens wichtig, deckst du zuerst auf – die anderen Spieler folgen im Uhrzeigersinn.}
\end{tikzpicture}
\hspace{-0.6cm}
\begin{tikzpicture}
	\card
	\cardstrip
	\cardbanner{banner/white.png}
	\cardicon{icons/coin.png}
	\cardprice{4}
	\cardtitle{Thronsaal}
	\cardcontent{\emph{Errata:} Der Kartentext ist falsch, es sollte \enquote{Du \emph{darfst} eine beliebige Aktionskarte aus der Hand zweimal ausspielen.} heißen.

	\medskip

	Wähle eine Aktionskarte aus deiner Hand und spiele sie zweimal aus, d. h. du legst die Aktionskarte aus, führst die Anweisungen der Karte komplett aus, nimmst die Karte zurück auf die Hand, legst sie noch einmal aus und führst die Anweisungen erneut aus. Für das doppelte Ausspielen dieser Aktionskarte muss der Spieler keine zusätzlichen Aktionen (+1 Aktion) zur Verfügung haben – sie ist sozusagen \enquote{kostenlos}. Legst du zwei \emph{THRONSAAL}-Karten aus, darfst du zuerst eine Aktion doppelt ausführen und dann eine andere Aktion ebenfalls doppelt ausführen. Du darfst aber nicht ein und dieselbe Aktion viermal ausführen.

	\medskip

	Erlaubt die doppelt ausgespielte Karte +1 Aktion (z.B. der \emph{MARKT}), hast du nach der vollständigen Ausführung des \emph{THRONSAALS} zwei weitere Aktionen zur Verfügung. Hättest du zwei \emph{MARKT}-Karten regulär hintereinander ausgespielt, bliebe dir nur noch eine zusätzliche Aktion zur Verfügung, da das Ausspielen der zweiten Marktkarte schon die zusätzliche Aktion der ersten Karte aufgebraucht hätte. Beim \emph{THRONSAAL} ist es besonders wichtig, laut die verbleibende Anzahl an Aktionen mitzuzählen. Du darfst keine weitere Aktion ausspielen, bevor der \emph{THRONSAAL} komplett abgearbeitet ist.}
\end{tikzpicture}
\hspace{-0.6cm}
\begin{tikzpicture}
	\card
	\cardstrip
	\cardbanner{banner/white.png}
	\cardicon{icons/coin.png}
	\cardprice{3}
	\cardtitle{\footnotesize{Schwarzmarkt}}
	\cardcontent{\tiny{Habt ihr den \emph{SCHWARZMARKT} als eine der 10 Königreichkarten-Sätze ausgewählt, müsst ihr vor Spielbeginn zusätzlich einen verdeckten \emph{SCHWARZMARKT}-Stapel bilden.

	\medskip

	\emph{Der SCHWARZMARKT-Stapel ...}
	... darf nur aus Königreichkarten bestehen, die nicht im Spiel sind, sich also nicht im Vorrat be nden. Er muss mindestens 15 Karten umfassen.

	\medskip

	Ihr einigt euch darauf, welche Königreichkarten ihr im \emph{SCHWARZMARKT}-Stapel haben wollt. Das können auch mehr als 15 Karten sein, ja sogar alle (nicht verwendeten) König- reichkarten dürfen im \emph{SCHWARZMARKT}-Stapel vorkommen.

	\medskip

	Dann kommt von jeder ausgewählten Königreichkarte eine Karte in den \emph{SCHWARZMARKT}-Stapel. Alle Spieler dürfen sich noch mal die Karten im Stapel anschauen. Danach wird der \emph{SCHWARZMARKT}-Stapel gemischt und als verdeckter Stapel neben dem Vorrat bereitgelegt. Dieser Stapel gehört nicht zum Vorrat.

	\medskip

	\emph{Der SCHWARZMARKT-Kauf}
	Er  findet in der Aktionsphase statt, d.h. schon vor der eigentlichen Kaufphase. Zunächst deckt der Spieler die obersten drei Karten des \emph{SCHWARZMARKT}-Stapels auf. Dann darf er beliebig viele Geldkarten ausspielen und eine der drei aufgedeckten Karten kaufen, wenn das Geld (ausgespielte Geldkarten und Geldwerte auf Aktionskarten) dazu reicht. Er darf auch darauf verzichten, eine der aufgedeckten Karten zu kaufen. Nicht gekaufte Karten werden in beliebiger Reihenfolge zurück unter den \emph{SCHWARZMARKT}-Stapel gelegt. Die ausgespielten Geldkarten bleiben zunächst offen liegen. Nicht verwendetes Geld, Geldwerte und Münzen darf der Spieler noch in der Kaufphase seines Zuges nutzen.

	\smallskip}}
\end{tikzpicture}
\hspace{-0.6cm}
\begin{tikzpicture}
	\card
	\cardstrip
	\cardbanner{banner/white.png}
	\cardtitle{\scriptsize{Empfohlene 10er Sätze\qquad}}
	\cardcontent{\emph{Erstes Spiel:}\\
	Burggraben, Dorf, Holzfäller, Keller, Markt, Miliz, Mine, Schmiede, Umbau, Werkstatt

	\smallskip

	\emph{Großes Geld:}\\
	Abenteurer, Bürokrat, Festmahl, Geldverleiher, Kanzler, Kapelle, Laboratorium, Mark, Mine, Thronsaal

	\smallskip

	\emph{Interaktion:}\\
	Bibliothek, Burggraben, Bürokrat, Dieb, Dorf, Jahrmarkt, Kanzler, Miliz, Ratsversammlung, Spion

	\smallskip

	\emph{Im Wandel:}\\
	Dieb, Dorf, Festmahl, Gärten, Hexe, Holzfäller, Kapelle, Keller, Laboratorium, Werkstatt

	\smallskip

	\emph{Dorfplatz:}\\
	Bibliothek, Bürokrat, Dorf, Holzfäller, Jahrmarkt, Keller, Markt, Schmiede, Thronsaal, Umbau}
\end{tikzpicture}
\hspace{0.6cm}

		% Basic settings for this card set
\renewcommand{\cardcolor}{basicgame}
\renewcommand{\cardextension}{Second Edition}
\renewcommand{\cardextensiontitle}{Das Basisspiel}
\renewcommand{\seticon}{basic2.png}

\clearpage
\newpage
\section{\cardextension \ - \cardextensiontitle \ (Rio Grande Games 2017)}

\begin{tikzpicture}
	\card
	\cardstrip
	\cardbanner{banner/blue.png}
	\cardicon{icons/coin.png}
	\cardprice{2}
	\cardtitle{Burggraben}
	\cardcontent{Spielt ein anderer Spieler eine Angriffskarte (mit der Aufschrift AKTION -- ANGRIFF), darfst du den \emph{BURGGRABEN} aus deiner Hand vorzeigen, \emph{bevor} der Spieler die Anweisung(en) der Angriffskarte ausführt. In diesem Fall bist \emph{du} davon nicht betroffen, d.h. du musst bei der \emph{HEXE} keine Fluchkarte nehmen usw. Haben mehrere Spieler einen \emph{BURGGRABEN} auf der Hand, dürfen diese auch eingesetzt und vorgezeigt werden. Danach nehmen die Spieler ihre Karte zurück auf die Hand. Der Spieler, der den Angriff gespielt hat, muss unabhängig davon, ob ein oder mehrere \emph{BURGGRÄBEN} gezeigt wurden, die weiteren Anweisungen seiner Aktionskarte ausführen. 

	\medskip

	Der \emph{BURGGRABEN} darf auch in der eigenen Aktionsphase gespielt werden – dann ziehst du 2 Karten nach.}
\end{tikzpicture}
\hspace{-0.6cm}
\begin{tikzpicture}
	\card
	\cardstrip
	\cardbanner{banner/white.png}
	\cardicon{icons/coin.png}
	\cardprice{2}
	\cardtitle{Kapelle}
	\cardcontent{Die ausgespielte \emph{KAPELLE} selbst darf nicht entsorgt werden, da sie sich nicht mehr auf der Hand befindet. Weitere \emph{KAPELLEN} auf der Hand dürfen entsorgt werden.}
\end{tikzpicture}
\hspace{-0.6cm}
\begin{tikzpicture}
	\card
	\cardstrip
	\cardbanner{banner/white.png}
	\cardicon{icons/coin.png}
	\cardprice{2}
	\cardtitle{Keller}
	\cardcontent{Der ausgespielte \emph{KELLER} selbst darf nicht abgelegt werden, da er sich nicht mehr in deiner Hand befindet. Sage an, wie viele Karten du ablegst und lege diese auf deinen Ablagestapel. Danach ziehst du die gleiche Anzahl Karten vom Nachziehstapel. Sollte während dieses Vorgangs der Nachziehstapel aufgebraucht werden, wird dein Ablagestapel zusammen mit den soeben abgelegten Karten gemischt und als neuer Nachziehstapel bereitgelegt.}
\end{tikzpicture}
\hspace{-0.6cm}
\begin{tikzpicture}
	\card
	\cardstrip
	\cardbanner{banner/white.png}
	\cardicon{icons/coin.png}
	\cardprice{3}
	\cardtitle{Dorf}
	\cardcontent{Spielst du mehrere \emph{DÖRFER} hintereinander, zählst du am besten laut mit, wie viele Aktionen du noch ausspielen darfst, damit du den Überblick behältst.}
\end{tikzpicture}
\hspace{-0.6cm}
\begin{tikzpicture}
	\card
	\cardstrip
	\cardbanner{banner/white.png}
	\cardicon{icons/coin.png}
	\cardprice{3}
	\cardtitle{Händlerin}
	\cardcontent{Du ziehst 1 Karte und erhältst + 1 Aktion. Wenn du in diesem Zug vor dem Ausspielen dieser \emph{HÄNDLERIN} noch kein Silber ausgespielt hast, erhältst du für das erste danach ausgespielte Silber +\coin[1]. Für jedes weitere ausgespielte Silber erhältst du keinen zusätzlichen Bonus. Hast du mehrere \emph{HÄNDLERINNEN} ausgespielt, erhältst du pro \emph{HÄNDLERIN} +\coin[1].}
\end{tikzpicture}
\hspace{-0.6cm}
\begin{tikzpicture}
	\card
	\cardstrip
	\cardbanner{banner/white.png}
	\cardicon{icons/coin.png}
	\cardprice{3}
	\cardtitle{Vasall}
	\cardcontent{Ist die aufgedeckte Karte eine Aktionskarte (auch ggf. kombinierte), \emph{darfst} du sie sofort ausspielen. Wenn du sie ausspielst, legst du sie in deinen Spielbereich und führst sofort die Anweisungen darauf aus. Dafür benötigst du keine zusätzliche Aktion. Das Ausspielen der Aktionskarte verbraucht auch keine freie oder zusätzliche Aktion, die du durch das Ausspielen anderer Karten bereits gesammelt hast.}
\end{tikzpicture}
\hspace{-0.6cm}
\begin{tikzpicture}
	\card
	\cardstrip
	\cardbanner{banner/white.png}
	\cardicon{icons/coin.png}
	\cardprice{3}
	\cardtitle{Vorbotin}
	\cardcontent{Du ziehst 1 Karte und erhältst + 1 Aktion. Schau dir deinen Ablagestapel an. Du \emph{darfst} eine Karte daraus auswählen und oben auf deinen Nachziehstapel legen. Die restlichen Karten (oder alle) legst du in beliebiger Reihenfolge zurück auf den Ablagestapel. Ist dein Ablagestapel leer, passiert nichts.}
\end{tikzpicture}
\hspace{-0.6cm}
\begin{tikzpicture}
	\card
	\cardstrip
	\cardbanner{banner/white.png}
	\cardicon{icons/coin.png}
	\cardprice{3}
	\cardtitle{Werkstatt}
	\cardcontent{Nimm dir eine Karte aus dem Vorrat und lege diese sofort auf deinen Ablagestapel. Du kannst weder Geldkarten noch zusätzlich über Aktionskarten erhaltenes Geld oder Münzen (bei Erweiterungen mit Münzen) einsetzen, um den angegebenen Betrag auf der Karte zu erhöhen.}
\end{tikzpicture}
\hspace{-0.6cm}
\begin{tikzpicture}
	\card
	\cardstrip
	\cardbanner{banner/white.png}
	\cardicon{icons/coin.png}
	\cardprice{4}
	\cardtitle{Bürokrat}
	\cardcontent{Ist dein Nachziehstapel aufgebraucht, wenn du diese Karte spielst, legst du die Silberkarte verdeckt ab. Sie bildet dann deinen Nachziehstapel. Das Gleiche gilt für alle Mitspieler, die eine Punktekarte verdeckt auf den eigenen Nachziehstapel legen müssen.}
\end{tikzpicture}
\hspace{-0.6cm}
\begin{tikzpicture}
	\card
	\cardstrip
	\cardbanner{banner/green.png}
	\cardicon{icons/coin.png}
	\cardprice{4}
	\cardtitle{Gärten}
	\cardcontent{Diese Karte ist die einzige Punktekarte unter den Königreichkarten. Sie hat bis zum Ende des Spiels keine Funktion. Bei der Wertung des Spiels erhält der Spieler, der diese Karte in seinem Kartensatz (Nachziehstapel, Handkarten und Ablagestapel) hat, für jeweils 10 Karten einen Siegpunkt. Es wird immer abgerundet, d. h. 39 Karten ergeben  3 Siegpunkte, ebenso wie 31 Karten 3 Siegpunkte ergeben. Wer mehrere \emph{GÄRTEN} besitzt, erhält für jeden \emph{GARTEN} die entsprechende Anzahl an Siegpunkten. }
\end{tikzpicture}
\hspace{-0.6cm}
\begin{tikzpicture}
	\card
	\cardstrip
	\cardbanner{banner/white.png}
	\cardicon{icons/coin.png}
	\cardprice{4}
	\cardtitle{\footnotesize{Geldverleiher}}
	\cardcontent{\emph{Errata:} Der Kartentext ist falsch, es sollte \enquote{Du \emph{darfst} ein Kupfer aus der Hand entsorgen. Wenn du das tust: + \coin[3].} heißen.

	\medskip

	Wer kein Kupfer auf der Hand hat oder keines entsorgen will, erhält kein zusätzliches Geld für die Kaufphase.}
\end{tikzpicture}
\hspace{-0.6cm}
\begin{tikzpicture}
	\card
	\cardstrip
	\cardbanner{banner/white.png}
	\cardicon{icons/coin.png}
	\cardprice{4}
	\cardtitle{Miliz}
	\cardcontent{Deine Mitspieler müssen Karten aus ihrer Hand ablegen, bis sie nur noch 3 Karten auf der Hand haben. Spieler, die zum Zeitpunkt des Angriffs bereits 3 oder weniger Karten auf der Hand haben, müssen keine weiteren Karten ablegen.}
\end{tikzpicture}
\hspace{-0.6cm}
\begin{tikzpicture}
	\card
	\cardstrip
	\cardbanner{banner/white.png}
	\cardicon{icons/coin.png}
	\cardprice{4}
	\cardtitle{Schmiede}
	\cardcontent{Du musst 3 Karten vom Nachziehstapel ziehen und auf die Hand nehmen.}
\end{tikzpicture}
\hspace{-0.6cm}
\begin{tikzpicture}
	\card
	\cardstrip
	\cardbanner{banner/white.png}
	\cardicon{icons/coin.png}
	\cardprice{4}
	\cardtitle{Thronsaal}
	\cardcontent{\emph{Errata:} Der Kartentext ist falsch, es sollte \enquote{Du \emph{darfst} eine beliebige Aktionskarte aus der Hand zweimal ausspielen.} heißen.

	\smallskip

	Spiele den \emph{THRONSAAL} aus, spiele die ausgewählte Aktionskarte aus, führe die Anweisungen vollständig aus und führe sie dann noch einmal vollständig aus \emph{(nimm sie dazwischen NICHT wieder auf die Hand)}. Die Aktionskarte muss beim zweiten Mal nicht mehr im Spiel sein (z.B. weil sie sich selbst entsorgt hat); du führst ihre Abweisungen dann trotzdem so vollständig wie möglich aus. Für das doppelte Ausspielen dieser Aktionskarte muss der Spieler keine zusätzlichen Aktionen (+1 Aktion) zur Verfügung haben – sie ist sozusagen \enquote{kostenlos}. Legst du zwei \emph{THRONSAAL}-Karten aus, darfst du zuerst eine Aktion doppelt ausführen und dann eine andere Aktion ebenfalls doppelt ausführen. Du darfst aber nicht ein und dieselbe Aktion viermal ausführen. 

	Erlaubt die doppelt ausgespielte Karte +1 Aktion (z. B. der \emph{MARKT}), hast du nach der vollständigen Ausführung des \emph{THRONSAALS} zwei weitere Aktionen zur Verfügung. Hättest du zwei \emph{MARKT}-Karten regulär hintereinander ausgespielt, bliebe dir nur noch eine zusätzliche Aktion zur Verfügung, da das Ausspielen der zweiten Marktkarte schon die zusätzliche Aktion der ersten Karte aufgebraucht hätte. Beim \emph{THRONSAAL} ist es besonders wichtig, laut die verbleibende Anzahl an Aktionen mitzuzählen. Du darfst keine weitere Aktion ausspielen, bevor der \emph{THRONSAAL} komplett abgearbeitet ist.}
\end{tikzpicture}
\hspace{-0.6cm}
\begin{tikzpicture}
	\card
	\cardstrip
	\cardbanner{banner/white.png}
	\cardicon{icons/coin.png}
	\cardprice{4}
	\cardtitle{Umbau}
	\cardcontent{Der ausgespielte \emph{UMBAU} selbst darf nicht entsorgt werden, da er sich nicht mehr in deiner Hand befindet. Weitere \emph{UMBAU}-Karten in deiner Hand dürfen entsorgt werden. Wenn du keine Karte zum Entsorgen auf der Hand hast, darfst du dir auch keine neue Karte nehmen. Die neue Karte darf maximal bis zu \coin[2] mehr als die entsorgte Karte kosten. Der Betrag darf weder durch weitere Geldkarten, Münzen oder zusätzliches Geld von anderen Aktionskarten erhöht werden. Die neue Karte kann die gleiche Karte sein wie die, die du entsorgt hast. Lege die neue Karte auf deinen Ablagestapel.}
\end{tikzpicture}
\hspace{-0.6cm}
\begin{tikzpicture}
	\card
	\cardstrip
	\cardbanner{banner/white.png}
	\cardicon{icons/coin.png}
	\cardprice{4}
	\cardtitle{Wilddiebin}
	\cardcontent{Du ziehst 1 Karte, erhältst + 1 Aktion und +\coin[1]. Dann schaust du, wie viele Vorratsstapel (Fluch-, Geld-, Punkte- und Aktionskarten) bereits leer sind. Ist kein Stapel leer, musst du keine Handkarten ablegen. Ist ein Stapel leer, legst du 1 Handkarte ab usw. Wenn du nicht so viele Karten auf der Hand hast, wie Vorratsstapel leer sind, legst du so viele Karten ab, wie du kannst.}
\end{tikzpicture}
\hspace{-0.6cm}
\begin{tikzpicture}
	\card
	\cardstrip
	\cardbanner{banner/white.png}
	\cardicon{icons/coin.png}
	\cardprice{5}
	\cardtitle{Banditin}
	\cardcontent{Zuerst nimmst du ein Gold vom Vorrat und legst es auf deinen Ablagestapel. Dann deckt jeder Mitspieler – beginnend bei deinem linken Mitspieler – die obersten zwei Karten seines Nachziehstapels auf. Deckt ein Spieler zwei Geldkarten (auch ggf. kombinierte) außer Kupfer auf, muss er \emph{eine} davon entsorgen. Dabei darf er selbst entscheiden, welche Geldkarte er entsorgt. Die andere Geldkarte wird – genauso wie alle anderen Karten – abgelegt. Deckt ein Spieler eine Geldkarte außer Kupfer sowie eine andere Karte (z. B. ein Kupfer oder eine beliebige Aktionskarte) auf, wird diese Geldkarte entsorgt. Die andere aufgedeckte Karte wird abgelegt.}
\end{tikzpicture}
\hspace{-0.6cm}
\begin{tikzpicture}
	\card
	\cardstrip
	\cardbanner{banner/white.png}
	\cardicon{icons/coin.png}
	\cardprice{5}
	\cardtitle{Bibliothek}
	\cardcontent{Aktionskarten darfst du zur Seite legen, sobald du sie ziehst, musst dies aber nicht tun. Hast du bereits 7 oder mehr Karten auf der Hand, wenn du die \emph{BIBLIOTHEK} ausspielst, ziehst du keine Karten nach. Wenn dein Nachziehstapel während des Ziehens aufgebraucht ist, mischst du den Ablagestapel, mischst aber die zur Seite gelegten Aktionskarten nicht mit ein. Diese werden erst auf den Ablagestapel gelegt, sobald du  das Ziehen beendet hast. Sollten die Karten nicht reichen, ziehst du nur so viele Karten wie möglich.}
\end{tikzpicture}
\hspace{-0.6cm}
\begin{tikzpicture}
	\card
	\cardstrip
	\cardbanner{banner/white.png}
	\cardicon{icons/coin.png}
	\cardprice{5}
	\cardtitle{Hexe}
	\cardcontent{Wenn du die \emph{HEXE} spielst und nicht mehr genügend Fluchkarten vorrätig sind, werden diese im Uhrzeigersinn (beginnend mit deinem linken Nachbarn) verteilt. Die Mitspieler legen die Fluchkarten sofort auf ihren Ablagestapel. Du ziehst immer 2 Karten von deinem Nachziehstapel, auch wenn keine Fluchkarten mehr im Vorrat sind.}
\end{tikzpicture}
\hspace{-0.6cm}
\begin{tikzpicture}
	\card
	\cardstrip
	\cardbanner{banner/white.png}
	\cardicon{icons/coin.png}
	\cardprice{5}
	\cardtitle{Jahrmarkt}
	\cardcontent{Spielst du mehrere \emph{JAHRMÄRKTE} hintereinander, zählst du am besten laut mit, wie viele Aktionen du noch ausspielen darfst, damit du den Überblick behältst.}
\end{tikzpicture}
\hspace{-0.6cm}
\begin{tikzpicture}
	\card
	\cardstrip
	\cardbanner{banner/white.png}
	\cardicon{icons/coin.png}
	\cardprice{5}
	\cardtitle{\footnotesize{Laboratorium}}
	\cardcontent{Du \emph{musst} zuerst zwei Karten vom Nachziehstapel auf die Hand nehmen. Dann \emph{darfst} du eine weitere Aktionskarte ausspielen.}
\end{tikzpicture}
\hspace{-0.6cm}
\begin{tikzpicture}
	\card
	\cardstrip
	\cardbanner{banner/white.png}
	\cardicon{icons/coin.png}
	\cardprice{5}
	\cardtitle{Markt}
	\cardcontent{Du musst eine Karte vom Nachziehstapel auf die Hand nehmen. Du \emph{darfst} in der Aktionsphase eine weitere Aktionskarte ausspielen. Du \emph{darfst} in der Kaufphase einen zusätzlichen Kauf tätigen und hast dafür ein zusätzliches Geld zur Verfügung.}
\end{tikzpicture}
\hspace{-0.6cm}
\begin{tikzpicture}
	\card
	\cardstrip
	\cardbanner{banner/white.png}
	\cardicon{icons/coin.png}
	\cardprice{5}
	\cardtitle{Mine}
	\cardcontent{\emph{Errata:} Der Kartentext ist falsch, es sollte \enquote{Du \emph{darfst} eine beliebige Geldkarte aus der Hand entsorgen. Nimm eine Geldkarte vom Vorrat auf die Hand, die bis zu \coin[3] mehr kostet.} heißen.

	\medskip

	Normalerweise entsorgst du ein Kupfer und nimmst dir dafür ein Silber, oder du entsorgst ein Silber und nimmst dir ein Gold. Du kannst dir aber auch eine gleichwertige oder billigere Karte nehmen. Die neue Karte nimmst du sofort auf die Hand und darfst sie noch während deines Zuges einsetzen. Wer keine Geldkarte zum Entsorgen hat oder keine entsorgen will, darf keine Geldkarte nehmen.}
\end{tikzpicture}
\hspace{-0.6cm}
\begin{tikzpicture}
	\card
	\cardstrip
	\cardbanner{banner/white.png}
	\cardicon{icons/coin.png}
	\cardprice{5}
	\cardtitle{\scriptsize{Ratsversammlung}}
	\cardcontent{Jeder Mitspieler \emph{muss} eine Karte von seinem Nachziehstapel auf die Hand nehmen.}
\end{tikzpicture}
\hspace{-0.6cm}
\begin{tikzpicture}
	\card
	\cardstrip
	\cardbanner{banner/white.png}
	\cardicon{icons/coin.png}
	\cardprice{5}
	\cardtitle{\footnotesize{Torwächterin}}
	\cardcontent{Du ziehst 1 Karte und erhältst + 1 Aktion. Dann siehst du dir die obersten 2 Karten deines Nachziehstapels an. Du kannst beide Karten entsorgen, beide Karten ablegen oder sie in beliebiger Reihenfolge zurück auf den Nachziehstapel legen. Du kannst aber auch eine entsorgen und eine ablegen, oder eine entsorgen und die andere zurück auf den Nachziehstapel legen, oder eine ablegen und die andere zurücklegen.}
\end{tikzpicture}
\hspace{-0.6cm}
\begin{tikzpicture}
	\card
	\cardstrip
	\cardbanner{banner/white.png}
	\cardicon{icons/coin.png}
	\cardprice{6}
	\cardtitle{Töpferei}
	\cardcontent{Nimm eine Karte vom Vorrat, die zu diesem Zeitpunkt maximal \coin[5] kostet. Du darfst kein zusätzliches \coin einsetzen, um dir eine teurere Karte zu nehmen. Außer \coin darf die Karte keine zusätzlichen Kosten enthalten. 

	\medskip

	Du darfst dir zum Beispiel keine Karte mit \potion (aus Alchemie) oder \hex (aus Empires) in den Kosten nehmen. Die genommene Karte nimmst du direkt auf die Hand. Anschließend legst du eine beliebige Handkarte (das kann die gerade genommene oder eine andere sein) oben auf deinen Nachziehstapel.}
\end{tikzpicture}
\hspace{-0.6cm}
\begin{tikzpicture}
	\card
	\cardstrip
	\cardbanner{banner/white.png}
	\cardtitle{\scriptsize{Empfohlene 10er Sätze\qquad}}
	\cardcontent{\emph{Erstes Spiel:}\\
		Burggraben, Dorf, Händlerin, Keller, Markt, Miliz, Mine, Schmiede, Umbau, Werkstatt
	
		\smallskip
	
		\emph{Verzerrte Größen:}\\
		Banditin, Bürokrat, Gärtner, Hexe, Jahrmarkt, Kapelle, Thronsaal, Töpferei, Torwächterin, Werkstatt
	
		\smallskip
	
		\emph{Schleichweg:}\\
		Bürokrat, Dorf, Geldverleiher, Laboratorium, Jahrmarkt, Ratsversammlung, Töpferei, Torwächterin, Vasall, Vorbotin
	
		\smallskip
	
		\emph{Kunststück:}\\
		Bibliothek, Gärten, Jahrmarkt, Keller, Miliz, Ratsversammlung, Schmiede, Thronsaal, Vorbotin, Wilddiebin
	
		\smallskip
	
		\emph{Verbesserungen:}\\
		Burggraben, Geldverleiher, Händlerin, Hexe, Keller, Markt, Mine, Töpferei, Umbau, Wilddiebin
	
		\smallskip
	
		\emph{Silber \& Gold:}\\
		Bandit, Bürokrat, Geldverleiher, Händlerin, Kapelle, Laboratorium, Mine, Thronsaal, Vasall, Vorbotin}
\end{tikzpicture}
\hspace{0.6cm}

		% Basic settings for this card set
\renewcommand{\cardcolor}{intrigue}
\renewcommand{\cardextension}{Edition II}
\renewcommand{\cardextensiontitle}{Die Intrige}

\clearpage
\newpage
\section{\cardextension \ - \cardextensiontitle}

\begin{tikzpicture}
	\card
	\cardstrip
	\cardbanner{banner/whitegreen.png}
	\cardicon{banner/coin.png}
	\cardprice{6}
	\cardtitle{Adelige}
	\cardcontent{Diese Karte ist \emph{zugleich} eine Aktions- und eine Punktekarte. Wenn du sie ausspielst, kannst du wählen, 3 Karten nachzuziehen \emph{oder} 2 zusätzliche Aktionen zu erhalten. Die beiden Anweisungen können jedoch nicht geteilt und gemischt werden. Bei Spielende sind die Adeligen 2 Punkte wert. Im Spiel zu 3. und zu 4. werden 12 Karten verwendet, im Spiel zu 2. werden 8 Karten verwendet.}
\end{tikzpicture}
\hspace{-1cm}
\begin{tikzpicture}
	\card
	\cardstrip
	\cardbanner{banner/white.png}
	\cardicon{banner/coin.png}
	\cardprice{5}
	\cardtitle{Anbau}
	\cardcontent{Du ziehst zuerst eine Karte. Danach \emph{musst} du eine Karte aus deiner Hand entsorgen und dann eine Karte nehmen, die genau 1 Geld mehr kostet als die entsorgte Karte. Ist keine solche Karte im Vorrat, erhältst du keine Karte, musst jedoch trotzdem eine entsorgen. Wenn du keine Karte zum Entsorgen hast, entsorgst du keine und nimmst dir keine Karte.}
\end{tikzpicture}
\hspace{-1cm}
\begin{tikzpicture}
	\card
	\cardstrip
	\cardbanner{banner/white.png}
	\cardicon{banner/coin.png}
	\cardprice{3}
	\cardtitle{Armenviertel}
	\cardcontent{Du erhältst 2 zusätzliche Aktionen. Dann \emph{musst} du deine Kartenhand vorzeigen. Wenn du keine Aktionskarten auf der Hand hast (auch kombinierte Aktions-/Punktekarten sind Aktionskarten), musst du 2 Karten nachziehen. Sollte die erste der gezogenen Karten eine Aktionskarte sein, ziehst du trotzdem eine zweite Karte.}
\end{tikzpicture}
\hspace{-1cm}
\begin{tikzpicture}
	\card
	\cardstrip
	\cardbanner{banner/white.png}
	\cardicon{banner/coin.png}
	\cardprice{5}
	\cardtitle{Austausch}
	\cardcontent{Entsorge zuerst eine Karte aus deiner Hand. Dann nimmst du dir eine Karte vom Vorrat, die maximal 2 mehr kostet als die entsorgte Karte. Eine Karte kostet nur dann maximal 2 mehr, wenn die restlichen Kosten (z. B. TRANK aus Empires oder SCHULDEN aus Alchemie) gleich oder niedriger sind. Wenn die genommene Karte eine Aktions- und/oder Geldkarte ist, legst du die Karte oben auf deinen Nachziehstapel. Ansonsten legst du die Karte auf den Ablagestapel. Ist die genommene Karte eine Punktekarte, nimmt sich jeder Mitspieler – beginnend bei deinem linken Mitspieler – einen Fluch. Ist die genommene Karte eine Punktekarte sowie eine Aktions- oder Geldkarte (z. B. MÜHLE), legst du die Karte oben auf deinen Nachziehstapel und jeder Mitspieler muss sich einen Fluch nehmen.}
\end{tikzpicture}
\hspace{-1cm}
\begin{tikzpicture}
	\card
	\cardstrip
	\cardbanner{banner/white.png}
	\cardicon{banner/coin.png}
	\cardprice{4}
	\cardtitle{Baron}
	\cardcontent{Du musst kein Anwesen ablegen, auch wenn du eines auf der Hand hast. Wenn du jedoch keines ablegst, musst du dir ein Anwesen nehmen, so lange noch welche im Vorrat sind. Du kannst nicht nur den +1 Kauf nutzen und die übrigen Anweisungen ignorieren.}
\end{tikzpicture}
\hspace{-1cm}
\begin{tikzpicture}
	\card
	\cardstrip
	\cardbanner{banner/white.png}
	\cardicon{banner/coin.png}
	\cardprice{4}
	\cardtitle{Bergwerk}
	\cardcontent{Du ziehst immer eine Karte nach und erhältst 2 zusätzliche Aktionen. Dann musst du entscheiden, ob du das Bergwerk entsorgst, bevor du weitere Aktionen ausspielst oder in die anderen Phasen übergehst. Wenn du das Bergwerk auf einen Thronsaal spielst, kannst du die Karte nur einmal entsorgen (d. h., du erhältst insgesamt +2 Karten, +4 Aktionen aber nur +2 Geld ).}
\end{tikzpicture}
\hspace{-1cm}
\begin{tikzpicture}
	\card
	\cardstrip
	\cardbanner{banner/white.png}
	\cardicon{banner/coin.png}
	\cardprice{4}
	\cardtitle{Brücke}
	\cardcontent{Die Kosten sind für alle Belange um 1 Geld reduziert. Wenn du z. B. ein Bergwerk ausspielst, danach eine Brücke, dann eine Eisenhütte, könntest du dir für die Eisen- hütte ein Herzogtum nehmen (kostet durch die Brücke nur noch 4 Geld). Die Karten der Spieler (Handkarten, Nachziehstapel und Ablagestapel) sind auch betroffen. Der Effekt ist kumulativ. Wenn du die Brücke auf einen Thronsaal spielst, sind die Kosten der Karten für diesen Zug um 2 Geld reduziert. Die Kosten sinken niemals unter 0 Geld. Wenn du eine Brücke und dann einen Anbau ausspielst, kannst du ein Kupfer entsorgen (das immer noch 0 Geld kostet) und dir einen Handlanger dafür nehmen (kostet durch die Brücke nur noch 1 Geld).}
\end{tikzpicture}
\hspace{-1cm}
\begin{tikzpicture}
	\card
	\cardstrip
	\cardbanner{banner/white.png}
	\cardicon{banner/coin.png}
	\cardprice{2}
	\cardtitle{Burghof}
	\cardcontent{Du ziehst 3 Karten und nimmst diese auf die Hand bevor du eine Karte auf den Nachziehstapel legst. Die Karte, die du auf den Nachziehstapel legst, muss keine der 3 gerade gezogenen Karten sein.}
\end{tikzpicture}
\hspace{-1cm}
\begin{tikzpicture}
	\card
	\cardstrip
	\cardbanner{banner/blue.png}
	\cardicon{banner/coin.png}
	\cardprice{4}
	\cardtitle{Diplomatin}
	\cardcontent{Diese Karte ist eine Aktions- und Reaktionskarte. Wird sie als Aktion in der Aktionsphase ausgespielt, nimmst du 2 Karten. Hast du dann 5 oder weniger Karten auf der Hand, erhältst du außerdem + 2 Aktionen. Spielt ein Mitspieler eine Angriffskarte aus und du hast zu diesem Zeitpunkt 5 oder mehr Karten auf der Hand, darfst du diese Karte – bevor der ausgespielte Angriff ausgeführt wird – aus der Hand aufdecken. Wenn du das tust, nimmst du diese DIPLOMATIN wieder auf die Hand, ziehst 2 Karten und legst dann 3 Karten (auch möglich inklusive dieser DIPLOMATIN) ab. Hast du dann immer noch 5 oder mehr Karten sowie eine DIPLOMATIN auf der Hand, darfst du die DIPLOMATIN noch einmal aufdecken – und dies so oft wiederholen wie du möchtest und die Bedingung der 5 oder mehr Karten auf der Hand erfüllt ist. Erst dann wird der Angriff ausgeführt. Hast du mehrere Reaktionskarten auf der Hand, mit denen du auf das Ausspielen einer Angriffskarte reagieren kannst, darfst du diese nacheinander in beliebiger Reihenfolge aufdecken.}
\end{tikzpicture}
\hspace{-1cm}
\begin{tikzpicture}
	\card
	\cardstrip
	\cardbanner{banner/white.png}
	\cardicon{banner/coin.png}
	\cardprice{4}
	\cardtitle{Eisenhütte}
	\cardcontent{Du nimmst dir eine Karte vom Vorrat und legst sie auf deinen Ablagestapel. Je nach Kartentyp der genommenen Karte erhältst du einen Bonus. Nimmst du eine Karte mit kombiniertem Kartentyp, z. B. Große Halle erhältst du +1 Aktion (weil die Große Halle eine Aktionskarte ist) und +1 Karte (weil die Große Halle auch eine Punktekarte ist).}
\end{tikzpicture}
\hspace{-1cm}
\begin{tikzpicture}
	\card
	\cardstrip
	\cardbanner{banner/white.png}
	\cardicon{banner/coin.png}
	\cardprice{4}
	\cardtitle{Geheimgang}
	\cardcontent{Du ziehst 2 Karten und erhältst + 1 Aktion. Dann nimmst du eine beliebige Karte aus deiner Hand (auch ggf. eine, die du gerade gezogen hast) und legst sie an eine beliebige Stelle in deinen Nachziehstapel. Du darfst sie oben drauf, unten drunter oder irgendwo in die Mitte legen. Du darfst dabei die Karten deines Nachziehstapels zählen, aber nicht ansehen. Befinden sich keine Karten in deinem Nachziehstapel, wird die zurückgelegte Karte zur einzigen Karte in deinem Nachziehstapel.}
\end{tikzpicture}
\hspace{-1cm}
\begin{tikzpicture}
	\card
	\cardstrip
	\cardbanner{banner/white.png}
	\cardicon{banner/coin.png}
	\cardprice{5}
	\cardtitle{Handelsposten}
	\cardcontent{Wenn du 2 oder mehr Karten auf der Hand hast, \emph{musst} du 2 Karten entsorgen und dir dafür ein Silber nehmen. Du nimmst das Silber direkt auf die Hand und kannst es auch in der Kaufphase verwenden. Wenn kein Silber mehr im Vorrat ist, erhältst du kein Silber, musst jedoch trotzdem 2 Karten entsorgen. Wenn du nur 1 Karte auf der Hand hast, musst du diese entsorgen, erhältst jedoch kein Silber. Wenn du keine Karte mehr auf der Hand hast, kannst du nichts entsorgen und erhältst auch kein Silber.}
\end{tikzpicture}
\hspace{-1cm}
\begin{tikzpicture}
	\card
	\cardstrip
	\cardbanner{banner/white.png}
	\cardicon{banner/coin.png}
	\cardprice{2}
	\cardtitle{Handlanger}
	\cardcontent{Wähle 2 verschiedene Anweisungen. Du darfst nicht eine Anweisung zweimal wählen. Du musst zuerst beide Anweisungen auswählen und sie dann erst (in jeder möglichen Reihenfolge) ausführen. Du kannst nicht eine Karte nachziehen und dann erst die zweite Anweisung wählen.}
\end{tikzpicture}
\hspace{-1cm}
\begin{tikzpicture}
	\card
	\cardstrip
	\cardbanner{banner/goldgreen.png}
	\cardicon{banner/coin.png}
	\cardprice{6}
	\cardtitle{Harem}
	\cardcontent{Diese Karte ist zugleich eine Geld- und eine Punktekarte. Du kannst sie in der Kaufphase spielen, genau wie ein Silber. Bei Spielende ist der Harem 2 Punkte wert. Im Spiel zu 3. und zu 4. werden 12 Karten verwendet, im Spiel zu 2. werden 8 Karten verwendet.}
\end{tikzpicture}
\hspace{-1cm}
\begin{tikzpicture}
	\card
	\cardstrip
	\cardbanner{banner/white.png}
	\cardicon{banner/coin.png}
	\cardprice{2}
	\cardtitle{Herumtreiberin}
	\cardcontent{Die Karte, die du entsorgst oder vom Müllstapel nimmst, muss den Typ AKTION beinhalten, d. h. sie kann auch eine kombinierte Aktionskarte (z. B. MÜHLE) sein. Genommene Karten werden auf den Ablagestapel gelegt – es sei denn, auf der Karte steht etwas anderes. Wird eine Karte entsorgt, die einen speziellen Effekt beim Entsorgen hat, tritt dieser ein.}
\end{tikzpicture}
\hspace{-1cm}
\begin{tikzpicture}
	\card
	\cardstrip
	\cardbanner{banner/green.png}
	\cardicon{banner/coin.png}
	\cardprice{5}
	\cardtitle{Herzog}
	\cardcontent{Diese Karte hat bis zum Ende des Spiels keine Funktion. Bei Spielende ist der Herzog 1 Punkt pro Herzogtum in Handkarten, Nachziehstapel und Ablagestapel wert. Im Spiel zu 3. und zu 4. werden 12 Karten verwendet, im Spiel zu 2. werden 8 Karten verwendet.}
\end{tikzpicture}
\hspace{-1cm}
\begin{tikzpicture}
	\card
	\cardstrip
	\cardbanner{banner/white.png}
	\cardicon{banner/coin.png}
	\cardprice{5}
	\cardtitle{Höflinge}
	\cardcontent{Decke eine Karte aus deiner Hand auf. Zähle dann die Typen, denen diese Karte angehört – also AKTION, GELD, REAKTION, ANGRIFF, PUNKTE, FLUCH etc. Pro Typ, dem die Karte angehört, entscheidest du dich für eine der vier angegebenen Optionen. Dabei darfst du keine der Optionen doppelt auswählen. Wenn du zum Beispiel eine PATROUILLE (AKTION) aufdeckst, darfst du eine Option auswählen, deckst du einen KARAWANENWÄCHTER aus Abenteuer (AKTION – DAUER – REAKTION) auf, darfst du 3 unterschiedliche Optionen wählen. Entscheidest du dich für das Gold, legst du dieses auf den Ablagestapel. Kannst du keine Handkarte aufdecken, erhältst du nichts.}
\end{tikzpicture}
\hspace{-1cm}
\begin{tikzpicture}
	\card
	\cardstrip
	\cardbanner{banner/white.png}
	\cardicon{banner/coin.png}
	\cardprice{5}
	\cardtitle{Kerkermeister}
	\cardcontent{Jeder Mitspieler, beginnend mit dem Spieler links vom Angreifer, muss sich eine der beiden Anweisungen wählen und diese dann ausführen. Ein Spieler kann wählen, 2 Karten abzulegen, auch wenn er weniger als 2 Karten auf der Hand hat. Hat er nur eine Karte auf der Hand legt er diese ab. Hat er keine Karte mehr auf der Hand, muss er auch keine ablegen. Ein Spieler kann wählen einen Fluch zu nehmen, auch wenn keine Fluchkarten mehr im Vorrat sind. In diesem Fall nimmt er keinen Fluch. Fluchkarten nehmen die Spieler direkt auf die Hand.}
\end{tikzpicture}
\hspace{-1cm}
\begin{tikzpicture}
	\card
	\cardstrip
	\cardbanner{banner/white.png}
	\cardicon{banner/coin.png}
	\cardprice{5}
	\cardtitle{Lakai}
	\cardcontent{Zunächst entscheidest du dich für eine der der beiden Anweisungen. Entweder erhältst du +2 virtuelles Geld \emph{oder} du wählst die zweite Anweisung für den Angriff. In diesem Fall sind nur Spieler mit 5 oder mehr Karten auf der Hand betroffen. Wehrt ein Spieler den Angriff mit einem Burggraben ab, darf er weder Karten nachziehen, noch muss er Karten ablegen. Ein Spieler kann auf den Angriff mit der Geheimkammer reagieren, auch wenn er weniger als 5 Karten auf der Hand hat. Anschließend hast du +1 Aktion, unabhängig davon, welche der beiden Anweisungen du gewählt hast.}
\end{tikzpicture}
\hspace{-1cm}
\begin{tikzpicture}
	\card
	\cardstrip
	\cardbanner{banner/white.png}
	\cardicon{banner/coin.png}
	\cardprice{3}
	\cardtitle{Maskerade}
	\cardcontent{Du ziehst zuerst 2 Karten. Dann wählen alle Spieler gleichzeitig eine Karte aus ihrer Hand und legen diese verdeckt zwischen sich und den Spieler zu ihrer Linken. Erst dann nehmen alle Spieler die Karten, die sie vom Spieler rechts bekommen haben auf. Die Spieler wählen also zuerst, welche Karte sie weiter geben, bevor sie sehen, welche Karte sie bekommen. Am Ende darfst nur du eine Karte aus deiner Hand entsorgen. Die Maskerade ist kein Angriff. Die übrigen Spieler dürfen also keine Reaktionskarten aus ihrer Hand vorzeigen um sich zu schützen.}
\end{tikzpicture}
\hspace{-1cm}
\begin{tikzpicture}
	\card
	\cardstrip
	\cardbanner{banner/whitegreen.png}
	\cardicon{banner/coin.png}
	\cardprice{4}
	\cardtitle{Mühle}
	\cardcontent{Diese Karte ist eine kombinierte Aktions- und Punktekarte. Als Punktekarte bringt sie beim Zählen der Punkte 1. Spielst du die MÜHLE als Aktionskarte aus, ziehst du 1 Karte und erhältst + 1 Aktion. Du darfst 2 Karten aus deiner Hand ablegen. Wenn du das tust, erhältst du + 2 . Tust du das nicht (weil du zum Beispiel nicht genügend Karten auf der Hand hast), erhältst du nichts. Nur, wenn du nicht mehr als eine Karte auf der Hand hast, darfst du genau eine Karte ablegen, erhältst dafür aber kein GELD.}
\end{tikzpicture}
\hspace{-1cm}
\begin{tikzpicture}
	\card
	\cardstrip
	\cardbanner{banner/white.png}
	\cardicon{banner/coin.png}
	\cardprice{5}
	\cardtitle{Patrouille}
	\cardcontent{Ziehe zuerst 3 Karten. Decke dann die obersten 4 Karten deines Nachziehstapels auf. So aufgedeckte Punktekarten (auch ggf. kombinierte) und Flüche nimmst du alle auf die Hand. Die restlichen Karten legst du in beliebiger Reihenfolge zurück auf den Nachziehstapel.}
\end{tikzpicture}
\hspace{-1cm}
\begin{tikzpicture}
	\card
	\cardstrip
	\cardbanner{banner/white.png}
	\cardicon{banner/coin.png}
	\cardprice{3}
	\cardtitle{Trickser}
	\cardcontent{Jeder Mitspieler, beginnend mit dem Spieler links vom Angreifer, muss die oberste Karte von seinem Nachziehstapel aufdecken. Er muss diese Karte entsorgen und du wählst eine Karte aus dem Vorrat, die das gleiche kostet. Diese Karte nimmt sich der Spieler und legt sie bei sich ab. Ist im Vorrat keine Karte, mit den gleichen Kosten, erhält der Spieler nichts, muss jedoch die Karte trotzdem entsorgen. Entsorgt er z. B. ein Kupfer, kannst du einen Fluch auswählen, den er nehmen muss. Du kannst auch die selbe Karte wählen, die er entsorgt hat. Die gewählte Karte muss im Vorrat zur Verfügung stehen. Du kannst also keine Karte aus einem leeren Stapel oder vom Müll wählen. Sind keine Karten mehr im Vorrat, die genau so viel kosten, wie die entsorgte Karte, erhält der Spieler nichts. Deckt ein Spieler den Burggraben aus seiner Hand auf, muss er keine Karte vom Nachziehstapel aufdecken und entsorgen und erhält auch keine Karte.}
\end{tikzpicture}
\hspace{-1cm}
\begin{tikzpicture}
	\card
	\cardstrip
	\cardbanner{banner/white.png}
	\cardicon{banner/coin.png}
	\cardprice{4}
	\cardtitle{Verschwörer}
	\cardcontent{Du überprüfst die Bedingung ob du +1 Karte und +1 Aktion erhältst wenn du den Verschwörer ausgespielt hast. Wenn die Bedingung später im Zug erfüllt wird, überprüfst du die Bedingung nicht rückwirkend. Wird eine Karte auf den Thronsaal gespielt, zählt der Thronsaal selbst als gespielte Aktionskarte und die darauf gespielte Aktionskarte zusätzlich zweimal als gespielte Aktionskarte. Wenn du z. B. den Verschwörer auf den Thronsaal spielst, ist der Thronsaal die erste Aktionskarte, der zuerst ausgespielte Verschwörer ist die zweite Aktionskarte (du erhältst also keine +1 Karte und keine +1 Aktion). Wenn du den Verschwörer zum zweiten mal ausspielst, hast du 3 Aktionskarten ausgespielt und erhältst +1 Karte und +1 Aktion.}
\end{tikzpicture}
\hspace{-1cm}
\begin{tikzpicture}
	\card
	\cardstrip
	\cardbanner{banner/white.png}
	\cardicon{banner/coin.png}
	\cardprice{3}
	\cardtitle{Verwalter}
	\cardcontent{Wenn du dich entscheidest, 2 Karten zu entsorgen und 2 oder mehr Karten auf der Hand hast, musst du genau 2 Karten entsorgen. Wenn du dich entscheidest, 2 Karten zu entsorgen, aber aber nur 1 Karte auf der Hand hast musst du diese Karte entsorgen. Du kannst die verschiedenen Anweisungen nicht mischen, du musst wählen: \emph{entweder} +2 Karten \emph{oder} +2 Geld \emph{oder} 2 Karten entsorgen.}
\end{tikzpicture}
\hspace{-1cm}
\begin{tikzpicture}
	\card
	\cardstrip
	\cardbanner{banner/white.png}
	\cardicon{banner/coin.png}
	\cardprice{3}
	\cardtitle{Wunschbrunnen}
	\cardcontent{Du ziehst zuerst eine Karte nach. Dann benennst du eine Karte (z. B. „Kupfer“, nicht „Geld“) und deckst die oberste Karte von deinem Nachziehstapel auf. Wenn es sich um die benannte Karte handelt, nimmst du sie auf die Hand. Wenn nicht, legst du sie zurück auf den Nachziehstapel.}
\end{tikzpicture}
\hspace{-1cm}
\begin{tikzpicture}
	\card
	\cardstrip
	\cardbanner{banner/white.png}
	\cardtitle{\scriptsize{Empfohlene 10er Sätze\qquad}}
	\cardcontent{\emph{Siegestanz:}
		\\
		Adlige, Austausch, Baron, Eisenhütte, Harem, Herzog, Höflinge, Maskerade, Mühle, Patrouille
		\\
		\smallskip
		\\
		\emph{Verschwörung:}
		\\
		Bergwerk, Eisenhütte, Geheimgang, Handelsposten, Handlanger, Herumtreiberin, Kerkermeister, Trickser, Verschwörer, Verwalter 
		\\
		\smallskip
		\\
		\emph{Beste Wünsche:}
		\\
		Anbau, Armenviertel, Baron, Burghof, Diplomatin, Geheimgang, Herzog, Kerkermeister, Verschwörer, Wunschbrunnen
		\\
	}
\end{tikzpicture}
\hspace{-1cm}
	    % Basic settings for this card set
\renewcommand{\cardcolor}{intrigue}
\renewcommand{\cardextension}{Erweiterung}
\renewcommand{\cardextensiontitle}{Die Intrige}
\renewcommand{\seticon}{intrigue1.png}

\clearpage
\newpage
\section{\cardextension \ - \cardextensiontitle \ (Rio Grande Games 2014)}

\begin{tikzpicture}
	\card
	\cardstrip
	\cardbanner{banner/white.png}
	\cardicon{icons/coin.png}
	\cardprice{2}
	\cardtitle{Burghof}
	\cardcontent{Du ziehst 3 Karten von deinem Nachziehstapel und nimmst sie auf die Hand. Dann wählst du eine beliebige Karte aus deiner Hand und legst sie verdeckt auf den Nachziehstapel.}
\end{tikzpicture}
\hspace{-0.6cm}
\begin{tikzpicture}
	\card
	\cardstrip
	\cardbanner{banner/blue.png}
	\cardicon{icons/coin.png}
	\cardprice{2}
	\cardtitle{\footnotesize{Geheimkammer}}
	\cardcontent{Wenn du diese Karte in deiner eigenen Aktionsphase ausspielst, legst du eine beliebige Anzahl Karten (auch 0 Karten) aus deiner Hand ab. \emph{Danach} erhältst du +\coin[1] pro abgelegte Karte. 

	\medskip

	Spielt ein anderer Spieler eine Angriffskarte aus, darfst du diese Karte vorzeigen, wenn du sie in diesem Moment auf der Hand hast, auch wenn dich der Angriff nicht betrifft. Wenn du das tust, ziehst du zuerst 2 Karten nach und legst dann 2 beliebige Karten aus deiner Hand verdeckt zurück auf den Nachziehstapel. Du darfst auch die \emph{GEHEIMKAMMER} selbst auf den Nachziehstapel legen, da sich die Karte auch nach dem Vorzeigen auf deiner Hand befindet. Du darfst pro Angriff so viele Reaktionskarten vorzeigen, wie du möchtest. So kannst du zum Beispiel zuerst die \emph{GEHEIMKAMMER} abwickeln und danach einen \emph{BURGGRABEN} vorzeigen, um den Angriff abzuwehren. Die \emph{GEHEIMKAMMER} selbst wehrt einen Angriff nicht ab.}
\end{tikzpicture}
\hspace{-0.6cm}
\begin{tikzpicture}
	\card
	\cardstrip
	\cardbanner{banner/white.png}
	\cardicon{icons/coin.png}
	\cardprice{2}
	\cardtitle{Handlanger}
	\cardcontent{Du darfst von den vier Anweisungen der Karte \emph{genau 2} auswählen und diese für deinen Zug nutzen. Du musst zwei verschiedene Anweisungen wählen und darfst nicht z. B. zwei Karten ziehen oder 2 zusätzliche Käufe tätigen. Du musst dich sofort entscheiden, welche zwei Anweisungen du nutzen möchtest. Du darfst nicht eine Karte nachziehen und dich erst dann entscheiden, welche zweite Anweisung du ausführen möchtest.}
\end{tikzpicture}
\hspace{-0.6cm}
\begin{tikzpicture}
	\card
	\cardstrip
	\cardbanner{banner/white.png}
	\cardicon{icons/coin.png}
	\cardprice{3}
	\cardtitle{Armenviertel}
	\cardcontent{Du darfst in deiner Aktionsphase zwei zusätzliche Aktionen ausführen. Zeige deine Kartenhand vor. Wenn du keine Aktionskarte (oder Aktions-/Punktekarte) auf der Hand hast, ziehst du zwei Karten nach. Sollten sich darunter Aktionskarten befinden, darfst du diese auch gleich nutzen.}
\end{tikzpicture}
\hspace{-0.6cm}
\begin{tikzpicture}
	\card
	\cardstrip
	\cardbanner{banner/whitegreen.png}
	\cardicon{icons/coin.png}
	\cardprice{3}
	\cardtitle{Große Halle}
	\cardcontent{Diese Karte ist eine kombinierte Aktions- und Punktekarte. Sie kann in der Aktionsphase eingesetzt werden und bringt \emph{zusätzlich} bei Spielende 1 Siegpunkt. Wenn du diese Karte ausspielst, ziehst du eine Karte und darfst eine zusätzliche Aktion ausspielen.}
\end{tikzpicture}
\hspace{-0.6cm}
\begin{tikzpicture}
	\card
	\cardstrip
	\cardbanner{banner/white.png}
	\cardicon{icons/coin.png}
	\cardprice{3}
	\cardtitle{Maskerade}
	\cardcontent{Ziehe zwei Karten. Anschließend wählen alle Spieler (einschließlich dir selbst) eine beliebige Karte aus ihrer Hand und legen sie verdeckt neben ihrem linken Nachbarn ab. Wenn alle Karten verteilt sind, nimmt jeder Spieler die erhaltene Karte auf die Hand. Da die \emph{MASKERADE} keine Angriffskarte ist, dürfen die anderen Spieler keine Reaktionskarten vorzeigen. Danach darfst du eine Karte aus deiner Hand entsorgen.}
\end{tikzpicture}
\hspace{-0.6cm}
\begin{tikzpicture}
	\card
	\cardstrip
	\cardbanner{banner/white.png}
	\cardicon{icons/coin.png}
	\cardprice{3}
	\cardtitle{Trickser}
	\cardcontent{Alle Mitspieler müssen die oberste Karte ihres Nachziehstapels aufdecken und diese entsorgen. Du wählst für jeden Mitspieler jeweils eine Karte aus dem Vorrat, die genauso viel kostet, wie die entsorgte Karte und gibst sie dem Mitspieler. Dieser legt die neue Karte auf seinen Ablagestapel. Befindet sich keine Karte mit den gleichen Kosten im Vorrat, muss der Mitspieler seine Karte trotzdem entsorgen, erhält dafür aber keine neue Karte. Du darfst dem Mitspieler auch eine gleiche Karte wie die, die er entsorgt hat, zurückgeben. }
\end{tikzpicture}
\hspace{-0.6cm}
\begin{tikzpicture}
	\card
	\cardstrip
	\cardbanner{banner/white.png}
	\cardicon{icons/coin.png}
	\cardprice{3}
	\cardtitle{Verwalter}
	\cardcontent{Du wählst von den drei Anweisungen der Karte \emph{genau 1} aus und führst diese dann komplett aus. Wenn du dich entscheidest, zwei Karten zu entsorgen, du aber nur eine Karte auf der Hand hast, musst du diese entsorgen.}
\end{tikzpicture}
\hspace{-0.6cm}
\begin{tikzpicture}
	\card
	\cardstrip
	\cardbanner{banner/white.png}
	\cardicon{icons/coin.png}
	\cardprice{3}
	\cardtitle{\scriptsize{Wunschbrunnen}}
	\cardcontent{Zuerst ziehst du eine Karte. Du darfst in der Aktionsphase eine zusätzliche Aktion ausführen. Nenne eine Karte (z.B. \emph{KUPFER}) und decke die oberste Karte deines Nachziehstapels auf. Handelt es sich dabei um die von dir genannte Karte, nimmst du sie auf die Hand. Ansonsten legst du sie zurück auf den Nachziehstapel.}
\end{tikzpicture}
\hspace{-0.6cm}
\begin{tikzpicture}
	\card
	\cardstrip
	\cardbanner{banner/white.png}
	\cardicon{icons/coin.png}
	\cardprice{4}
	\cardtitle{Baron}
	\cardcontent{Du \emph{darfst} ein Anwesen (sofern du gerade eins auf der Hand hast) ablegen und erhältst dafür +\coin[4] für die Kaufphase. Wenn du das nicht tun kannst (weil du kein Anwesen auf der Hand hast) oder willst, musst du dir ein Anwesen nehmen, solange noch welche im Vorrat sind.}
\end{tikzpicture}
\hspace{-0.6cm}
\begin{tikzpicture}
	\card
	\cardstrip
	\cardbanner{banner/white.png}
	\cardicon{icons/coin.png}
	\cardprice{4}
	\cardtitle{Bergwerk}
	\cardcontent{Du ziehst zuerst eine Karte nach und \emph{darfst} dann diese Karte entsorgen, bevor du ggf. weitere Aktionen ausspielst. Du erhältst dafür +\coin[2]. Wenn du das \emph{BERGWERK} auf einen \emph{THRONSAAL} spielst, erhältst du den Bonus für das Entsorgen nur einmal, da du die Karte nur einmal entsorgen kannst. Die anderen Anweisungen (+ 1 Karte sowie + 2 Aktionen) werden durch den \emph{THRONSAAL} dagegen verdoppelt.}
\end{tikzpicture}
\hspace{-0.6cm}
\begin{tikzpicture}
	\card
	\cardstrip
	\cardbanner{banner/white.png}
	\cardicon{icons/coin.png}
	\cardprice{4}
	\cardtitle{Brücke}
	\cardcontent{Die Kosten aller Karten (auch Handkarten, Karten aus den Nachzieh- und Ablagestapeln) werden in diesem Spielzug für alle Belange um \coin[1] reduziert (nicht aber unter \coin[0]). Dieser Effekt ist kumulativ, d. h. die Kosten pro Karte können durch das Ausspielen bestimmter Karten (z. B. den \emph{THRONSAAL}) auch um \coin[2] oder mehr reduziert werden.}
\end{tikzpicture}
\hspace{-0.6cm}
\begin{tikzpicture}
	\card
	\cardstrip
	\cardbanner{banner/white.png}
	\cardicon{icons/coin.png}
	\cardprice{4}
	\cardtitle{Eisenhütte}
	\cardcontent{Du nimmst dir eine beliebige Karte vom Vorrat, die maximal \coin[4] kostet. Durch das Ausspielen bestimmter Aktionskarten (z. B. die \emph{BRÜCKE}) können die Kosten der Karten reduziert werden.

	\medskip

	Je nachdem, ob du dich für eine Aktions-, Geld- oder Punktekarte entschieden hast, erhältst du einen anderen Bonus. Solltest du dich für eine kombinierte Karte entscheiden, erhältst du die Boni beider Kartentypen.}
\end{tikzpicture}
\hspace{-0.6cm}
\begin{tikzpicture}
	\card
	\cardstrip
	\cardbanner{banner/white.png}
	\cardicon{icons/coin.png}
	\cardprice{4}
	\cardtitle{\footnotesize{Kupferschmied}}
	\cardcontent{Mit dieser Karte erhöhst du den Wert aller in diesem Zug gespielten \emph{KUPFER} um +\coin[1]. Der Effekt ist kumulativ, d. h. durch das Ausspielen anderer Aktionskarten (z. B. den \emph{THRONSAAL} oder einen weiteren \emph{KUPFERSCHMIED}) kann der Wert weiter erhöht werden.}
\end{tikzpicture}
\hspace{-0.6cm}
\begin{tikzpicture}
	\card
	\cardstrip
	\cardbanner{banner/white.png}
	\cardicon{icons/coin.png}
	\cardprice{4}
	\cardtitle{Späher}
	\cardcontent{Sollten für das Aufdecken der vier Karten nicht genügend Karten im Nachziehstapel sein, ziehst du zunächst die restlichen Karten und mischst dann deinen Ablagestapel neu, ohne die bereits aufgedeckten Karten mit einzumischen. Sollten auch dann nicht genügend Karten zur Verfügung stehen, ziehst du nur so viele Karten wie möglich. Du musst alle Punktekarten auf die Hand nehmen, die restlichen Karten legst du in beliebiger Reihenfolge auf den Nachziehstapel. Diese musst du deinen Mitspielern nicht zeigen. Kombinierte Aktions-/Punktekarten sind auch Punktekarten. }
\end{tikzpicture}
\hspace{-0.6cm}
\begin{tikzpicture}
	\card
	\cardstrip
	\cardbanner{banner/white.png}
	\cardicon{icons/coin.png}
	\cardprice{4}
	\cardtitle{Verschwörer}
	\cardcontent{Wenn du zu dem Zeitpunkt an dem du den \emph{VERSCHWÖRER} spielst, bereits mindestens 3 Aktionskarten (inklusive diesem \emph{VERSCHWÖRER}) ausgespielt hast, erhältst du den Bonus. Wenn du erst im weiteren Verlauf deiner Aktionsphase diese Bedingung erfüllst, erhältst du den Bonus nicht. Aktionskarten, die z. B. durch den \emph{THRONSAAL} doppelt ausgespielt werden dürfen, gelten als 2 ausgespielte Karten.}
\end{tikzpicture}
\hspace{-0.6cm}
\begin{tikzpicture}
	\card
	\cardstrip
	\cardbanner{banner/white.png}
	\cardicon{icons/coin.png}
	\cardprice{5}
	\cardtitle{Anbau}
	\cardcontent{Nachdem du dir eine Karte genommen und eine zusätzliche Aktion erhalten hast, musst du eine Karte aus deiner Hand entsorgen sofern du noch Handkarten hast. Du nimmst dir dafür eine Karte vom Vorrat, die \emph{genau} \coin[1] mehr kostet als die entsorgte Karte. Wenn keine solche Karte vorhanden ist, musst du zwar eine Karte entsorgen, erhältst aber keine Karte vom Vorrat. }
\end{tikzpicture}
\hspace{-0.6cm}
\begin{tikzpicture}
	\card
	\cardstrip
	\cardbanner{banner/white.png}
	\cardicon{icons/coin.png}
	\cardprice{5}
	\cardtitle{\footnotesize{Handelsposten}}
	\cardcontent{Wenn du nur eine Karte auf der Hand hast, musst du sie entsorgen, erhältst dafür aber kein Silber. Wenn du zwei oder mehr Karten auf der Hand hast, musst du genau zwei Karten entsorgen und nimmst dir dafür ein Silber direkt auf die Hand. Sollte kein Silber mehr im Vorrat sein, musst du die Karten trotzdem entsorgen, erhältst aber kein Silber. }
\end{tikzpicture}
\hspace{-0.6cm}
\begin{tikzpicture}
	\card
	\cardstrip
	\cardbanner{banner/green.png}
	\cardicon{icons/coin.png}
	\cardprice{5}
	\cardtitle{Herzog}
	\cardcontent{Diese Karte ist die einzige reine Punktekarte unter den Königreichkarten. Sie hat bis zum Ende des Spiels keine Funktion. Bei Spielende erhält der Spieler, der diese Karte in seinem Kartensatz (Nachziehstapel, Handkarten und Ablagestapel) hat, für jedes \emph{HERZOGTUM} im Kartensatz 1 Siegpunkt. Wer mehrere \emph{HERZÖGE} besitzt, erhält für jeden \emph{HERZOG} die entsprechende Anzahl Siegpunkte.} 
\end{tikzpicture}
\hspace{-0.6cm}
\begin{tikzpicture}
	\card
	\cardstrip
	\cardbanner{banner/white.png}
	\cardicon{icons/coin.png}
	\cardprice{5}
	\cardtitle{\footnotesize{Kerkermeister}}
	\cardcontent{Jeder Mitspieler (beginnend bei deinem linken Nachbarn) muss entweder zwei Karten ablegen oder einen \emph{FLUCH} vom Stapel nehmen. Ein Spieler kann sich entscheiden, die Karten abzulegen, auch wenn er nur eine oder gar keine Karte auf der Hand hat. Er legt dann nur so viele Karten ab, wie er kann. Er kann sich auch entscheiden, einen \emph{FLUCH} zu nehmen, wenn es keine \emph{FLÜCHE} mehr im Vorrat gibt.}
\end{tikzpicture}
\hspace{-0.6cm}
\begin{tikzpicture}
	\card
	\cardstrip
	\cardbanner{banner/white.png}
	\cardicon{icons/coin.png}
	\cardprice{5}
	\cardtitle{Lakai}
	\cardcontent{Du entscheidest dich für eine der beiden Optionen: Entweder erhältst du +\coin[2] in diesem Zug oder du legst alle deine Handkarten ab und ziehst vier neue Karten nach. Wenn du die zweite Option wählst, müssen außerdem alle Mitspieler, die fünf oder mehr Karten auf der Hand haben (alle anderen sind nicht betroffen), diese ablegen und ebenfalls vier Karten nachziehen. Jeder Spieler (auch wenn er von dem Angriff nicht betroffen ist) kann eine oder mehrere Reaktionskarten vorzeigen, wenn du den \emph{LAKAIEN} spielst.}
\end{tikzpicture}
\hspace{-0.6cm}
\begin{tikzpicture}
	\card
	\cardstrip
	\cardbanner{banner/white.png}
	\cardicon{icons/coin.png}
	\cardprice{5}
	\cardtitle{Saboteur}
	\cardcontent{Jeder Mitspieler (beginnend bei deinem linken Nachbarn) muss solange Karten von seinem Nachziehstapel aufdecken, bis er eine Karte aufdeckt, die mindestens \emph{3} kostet (\emph{Achtung:} Die Kosten einer Karte können durch die \emph{BRÜCKE} verringert werden). Er muss diese Karte sofort entsorgen und darf sich dafür eine Karte aus dem Vorrat nehmen, die mindestens \emph{2} weniger kostet. Die restlichen aufgedeckten Karten werden abgelegt. Sollte im gesamten restlichen Nachziehstapel keine Karte mit passendem Wert vorhanden sein, wird der Ablagestapel ohne die bereits aufgedeckten Karten gemischt und zum neuen Nachziehstapel. Sollte dann immer noch keine passende Karte zu finden sein, legt der Spieler alle Karten ab und nichts passiert.}
\end{tikzpicture}
\hspace{-0.6cm}
\begin{tikzpicture}
	\card
	\cardstrip
	\cardbanner{banner/white.png}
	\cardicon{icons/coin.png}
	\cardprice{5}
	\cardtitle{Tribut}
	\cardcontent{Dein linker Mitspieler muss die obersten beiden Karten seines Nachziehstapels aufdecken und ablegen. Da der \emph{TRIBUT} keine Angriffskarte ist, kann sich der Mitspieler nicht gegen diese Anweisung wehren. Der Spieler, der ein \emph{TRIBUT} ausgespielt hat, erhält für die erste Karte den genannten Bonus. Für die zweite Karte erhält der Spieler nur dann ein weiteres Mal den Bonus, wenn nicht die gleiche Karte wie zuvor aufgedeckt wurde. Kombinierte Karten bringen dem Spieler auch doppelte Boni.}
\end{tikzpicture}
\hspace{-0.6cm}
\begin{tikzpicture}
	\card
	\cardstrip
	\cardbanner{banner/whitegreen.png}
	\cardicon{icons/coin.png}
	\cardprice{6}
	\cardtitle{Adelige}
	\cardcontent{Diese Karte ist eine kombinierte Aktions- und Punktekarte. Sie kann in der Aktionsphase eingesetzt werden und bringt außerdem bei Spielende 2 Siegpunkte. Wenn du diese Karte ausspielst, musst du dich entscheiden, ob du entweder 3 Karten ziehst oder  2 weitere Aktionen ausspielen willst. Du darfst die Anweisungen aber nicht mischen oder aufteilen. Wenn du die Karte das nächste Mal auf der Hand hast und ausspielst, darfst du natürlich eine andere Wahl treffen.}
\end{tikzpicture}
\hspace{-0.6cm}
\begin{tikzpicture}
	\card
	\cardstrip
	\cardbanner{banner/goldgreen.png}
	\cardicon{icons/coin.png}
	\cardprice{6}
	\cardtitle{Harem}
	\cardcontent{Diese Karte ist eine kombinierte Geld- und Punktekarte. Sie wird während des Zugs wie eine normale Geldkarte eingesetzt und bringt außerdem bei Spielende 2 Siegpunkte.}
\end{tikzpicture}
\hspace{-0.6cm}
\begin{tikzpicture}
	\card
	\cardstrip
	\cardbanner{banner/white.png}
	\cardtitle{\scriptsize{Empfohlene 10er Sätze\qquad}}
	\cardcontent{\emph{Siegestanz:}\\
	Adlige, Anbau, Brücke, Eisenhütte, Große Halle, Handlanger, Harem, Herzog, Maskerade, Späher 

	\smallskip

	\emph{Geheime Pläne:}\\
	Armenviertel, Eisenhütte, Handelsposten, Handlanger, Harem, Saboteur, Tribut, Trickser, Verschwörer, Verwalter

	\smallskip

	\emph{Beste Wünsche:}\\
	Anbau, Armenviertel, Burghof, Handelsposten, Kerkermeister, Kupferschmied, Maskerade, Späher, Verwalter, Wunschbrunnen

	\smallskip

	\emph{Demontage} (Intrige + \textit{Basisspiel}):\\
	Bergwerk, Brücke, Geheimkammer, Kerkermeister, Trickser, Saboteur, \textit{Dieb}, \textit{Spion}, \textit{Thronsaal}, \textit{Umbau}

	\smallskip

	\emph{Eine Hand voll} (Intrige + \textit{Basisspiel}):\\
	Adlige, Burghof, Kerkermeister, Lakai, Verwalter, \textit{Bürokrat}, \textit{Kanzler}, \textit{Miliz}, \textit{Mine}, \textit{Ratsversammlung}

	\smallskip

	\emph{Untergebene} (Intrige + \textit{Basisspiel}):\\
	Adlige, Baron, Lakai, Maskerade, Verwalter, Handlanger, \textit{Bibliothek}, \textit{Hexe}, \textit{Jahrmarkt}, \textit{Keller}}
\end{tikzpicture}
\hspace{-0.6cm}

	    \input{sets/de/seaside.tex}
	    % Basic settings for this card set
\renewcommand{\cardcolor}{seaside}
\renewcommand{\cardextension}{Erweiterung I}
\renewcommand{\cardextensiontitle}{Seaside}
\renewcommand{\seticon}{seaside.png}

\clearpage
\newpage
\section{\cardextension \ - \cardextensiontitle \ (Rio Grande Games 2014)}

\begin{tikzpicture}
	\card
	\cardstrip
	\cardbanner{banner/white.png}
	\cardicon{icons/coin.png}
	\cardprice{2}
	\cardtitle{\tiny{Eingeborenendorf}}
	\cardcontent{Wenn du dein erstes \emph{EINGEBORENENDORF} nimmst oder kaufst, erhältst du ein Eingeborenen-Tableau und legst es vor dir ab. 

	\medskip

	Immer wenn du ein \emph{EINGEBORENENDORF} ausspielst, wählst du genau eine der beiden Anweisungen und führst sie wenn möglich aus. Du darfst eine Anweisung auch wählen, wenn du sie nicht ausführen kannst. Karten, die du auf das Tableau legst, werden immer verdeckt abgelegt. Du darfst dir jederzeit die Karten auf deinem Tableau ansehen. 

	\medskip

	Die ausgespielte Aktionskarte \emph{EINGEBORENENDORF} legst du in der Aufräumphase ab. Alle Karten auf dem Tableau gehören auch zum Kartensatz eines Spielers. Alle Karten auf den Tableaus werden bei Spielende mit berücksichtigt.}
\end{tikzpicture}
\hspace{-0.6cm}
\begin{tikzpicture}
	\card
	\cardstrip
	\cardbanner{banner/white.png}
	\cardicon{icons/coin.png}
	\cardprice{2}
	\cardtitle{Embargo}
	\cardcontent{Wenn du das \emph{EMBARGO} in deiner Aktionsphase ausspielst, musst du es entsorgen und einen Embargomarker auf einen beliebigen Vorratsstapel (Königreichkarten \emph{und} Geldkarten sind erlaubt) legen. Um die +\coin[2] in der Kaufphase nicht zu \enquote{vergessen}, empfehlen wir, das entsorgte \emph{EMBARGO} zunächst separat neben den Müllstapel zu legen und erst in der Aufräumphase endgültig zu entsorgen.

	\medskip
Wenn du das \emph{EMBARGO} auf einen \emph{THRONSAAL} folgend ausspielst, legst du 2 Embargomarker. Du kannst sie auf denselben oder unterschiedliche Stapel legen. Wenn keine Embargomarker mehr vorhanden sind, benutze einen geeigneten Ersatz (z.B. echte Geldmünzen). Auf jeden Vorratsstapel dürfen beliebig viele Embargomarker gelegt werden. 

	\medskip

	Spieler, die Karten von einem Vorratsstapel mit einem oder mehreren Embargomarkern kaufen, müssen pro Marker eine Fluchkarte nehmen. Wer Karten von einem Stapel mit Embargomarker(n) auf eine andere Weise nimmt (z. B. durch den \emph{SCHMUGGLER}), nimmt \emph{keine} Fluchkarte. Wenn keine Fluchkarten mehr vorrätig sind, haben die Marker keinen Effekt.}
\end{tikzpicture}
\hspace{-0.6cm}
\begin{tikzpicture}
	\card
	\cardstrip
	\cardbanner{banner/orange.png}
	\cardicon{icons/coin.png}
	\cardprice{2}
	\cardtitle{Hafen}
	\cardcontent{Der \emph{HAFEN} ist eine Dauerkarte. Lege eine Handkarte verdeckt auf den \emph{HAFEN}. Diese und der \emph{HAFEN} werden in der Aufräumphase nicht abgelegt. Nimm zu Beginn deines nächsten Zuges die zur Seite gelegte Karte auf die Hand. Lege den \emph{HAFEN} in der Aufräumphase ab.}
\end{tikzpicture}
\hspace{-0.6cm}
\begin{tikzpicture}
	\card
	\cardstrip
	\cardbanner{banner/orange.png}
	\cardicon{icons/coin.png}
	\cardprice{2}
	\cardtitle{Leuchtturm}
	\cardcontent{Der \emph{LEUCHTTURM} ist eine Dauerkarte. Solange der \emph{LEUCHTTURM} offen vor dir liegt (im Spielbereich oder darüber), bist du grundsätzlich nicht betroffen, wenn Mitspieler Angriffskarten ausspielen (sogar wenn du das möchtest). Selber ausgespielte Angriffskarten werden vom \emph{LEUCHTTURM} nicht abgewehrt. Auf Angriffe von Mitspielern darfst du weiterhin zusätzlich Reaktionskarten ausspielen. Lege den \emph{LEUCHTTURM} in der Aufräumphase des nächsten Zuges ab.}
\end{tikzpicture}
\hspace{-0.6cm}
\begin{tikzpicture}
	\card
	\cardstrip
	\cardbanner{banner/white.png}
	\cardicon{icons/coin.png}
	\cardprice{2}
	\cardtitle{\footnotesize{Perlentaucher}}
	\cardcontent{Zieh die unterste Karte des Nachziehstapels so hervor, dass du die benachbarte Karte nicht sehen kannst. Schaue sie dir an und lege sie dann verdeckt oben auf den Nachziehstapel \emph{oder} zurück unter den Nachziehstapel.}
\end{tikzpicture}
\hspace{-0.6cm}
\begin{tikzpicture}
	\card
	\cardstrip
	\cardbanner{banner/white.png}
	\cardicon{icons/coin.png}
	\cardprice{3}
	\cardtitle{Ausguck}
	\cardcontent{Sieh dir erst alle 3 Karten an, bevor du die Anweisungen ausführst. Solltest du weniger als 3 Karten im Nachziehstapel haben, auch nachdem du ggf. den Ablagestapel gemischt hast, führst du die Anweisungen der Reihenfolge nach aus. Anweisungen, für die es keine Karten mehr im Stapel gibt, entfallen.}
\end{tikzpicture}
\hspace{-0.6cm}
\begin{tikzpicture}
	\card
	\cardstrip
	\cardbanner{banner/white.png}
	\cardicon{icons/coin.png}
	\cardprice{3}
	\cardtitle{Botschafter}
	\cardcontent{Wähle eine beliebige Handkarte und zeige sie deinen Mitspielern. Du darfst dann bis zu 2 dieser Karten von deiner Hand zurück in den Vorrat legen. Jeder Mitspieler nimmt sich anschließend eine solche Karte aus dem Vorrat (reihum im Uhrzeigersinn, beginnend beim linken Mitspieler). Der ausgespielte Botschafter kann nicht in den Vorrat zurückgelegt werden.}
\end{tikzpicture}
\hspace{-0.6cm}
\begin{tikzpicture}
	\card
	\cardstrip
	\cardbanner{banner/orange.png}
	\cardicon{icons/coin.png}
	\cardprice{3}
	\cardtitle{Fischerdorf}
	\cardcontent{Das \emph{FISCHERDORF} ist eine Dauerkarte. Du \emph{darfst} 2 weitere Aktionen ausführen und erhältst für die Kaufphase +\coin[1]. 

	\medskip

	In deinem nächsten Zug \emph{darfst} du eine weitere Aktion ausführen und erhältst für die Kaufphase +\coin[1].}
\end{tikzpicture}
\hspace{-0.6cm}
\begin{tikzpicture}
	\card
	\cardstrip
	\cardbanner{banner/white.png}
	\cardicon{icons/coin.png}
	\cardprice{3}
	\cardtitle{Lagerhaus}
	\cardcontent{Ziehe 3 Karten und spiele dann eine Aktionskarte. Danach legst du 3 Handkarten ab. Wenn du weniger als 3 Karten auf der Hand hast, legst du alle Handkarten ab.}
\end{tikzpicture}
\hspace{-0.6cm}
\begin{tikzpicture}
	\card
	\cardstrip
	\cardbanner{banner/white.png}
	\cardicon{icons/coin.png}
	\cardprice{3}
	\cardtitle{Schmuggler}
	\cardcontent{Hat der rechts von dir sitzende Mitspieler in seinem letzten Zug eine Karte mit Kosten von \coin[6] oder weniger genommen, gekauft oder auf andere Art erhalten, nimmst du dir eine gleiche Karte vom Vorrat. Hat der Spieler mehrere Karten genommen, darfst du wählen, welche du nimmst. Da der \emph{SCHMUGGLER} keine Angriffskarte ist, dürfen keine Reaktionskarten ausgespielt werden.}
\end{tikzpicture}
\hspace{-0.6cm}
\begin{tikzpicture}
	\card
	\cardstrip
	\cardbanner{banner/white.png}
	\cardicon{icons/coin.png}
	\cardprice{4}
	\cardtitle{\scriptsize{Beutelschneider}}
	\cardcontent{Alle Mitspieler müssen eine Kupferkarte aus der Hand ablegen. Da der Beutelschneider eine Angriffskarte ist, dürfen die Mitspieler mit einer Reaktionskarte auf diesen Angriff reagieren.}
\end{tikzpicture}
\hspace{-0.6cm}
\begin{tikzpicture}
	\card
	\cardstrip
	\cardbanner{banner/whitegreen.png}
	\cardicon{icons/coin.png}
	\cardprice{4}
	\cardtitle{Insel}
	\cardcontent{Die \emph{INSEL} ist eine kombinierte Aktions- und Punktekarte. Sie kann in der Aktionsphase eingesetzt werden und bringt zusätzlich bei Spielende 2 Punkte. Wenn du deine erste \emph{INSEL} nimmst oder kaufst, erhältst du ein Insel-Tableau und legst es vor dir ab. 

	\medskip

	Immer wenn du eine \emph{INSEL} ausspielst, legst du die ausgespielte \emph{INSEL} und eine beliebige Handkarte offen auf dein Insel-Tableau. Dort verbleiben sie bis zum Spielende. Wenn du mindestens eine Karte auf der Hand hast, musst du eine Handkarte auf dein Insel-Tableau legen. Wenn du keine Karte auf der Hand hast, nachdem du die \emph{INSEL} ausgespielt hast, legst du nur die \emph{INSEL} auf das Tableau. 

	\medskip

	Bei Spielende nimmst du alle Karten vom Insel-Tableau zu deinen Karten.}
\end{tikzpicture}
\hspace{-0.6cm}
\begin{tikzpicture}
	\card
	\cardstrip
	\cardbanner{banner/orange.png}
	\cardicon{icons/coin.png}
	\cardprice{4}
	\cardtitle{Karawane}
	\cardcontent{Die \emph{KARAWANE} ist eine Dauerkarte. Sie wird in der Aufräumphase nicht abgelegt. Ziehe zu Beginn des nächsten Zuges eine Karte und lege die \emph{KARAWANE} in der Aufräumphase dieses Zuges ab.}
\end{tikzpicture}
\hspace{-0.6cm}
\begin{tikzpicture}
	\card
	\cardstrip
	\cardbanner{banner/white.png}
	\cardicon{icons/coin.png}
	\cardprice{4}
	\cardtitle{\footnotesize{Müllverwerter}}
	\cardcontent{Du musst eine Karte entsorgen, sofern du eine auf der Hand hast. Entsprechend der Kosten der entsorgten Karte erhältst du für die Kaufphase +\emph{X}. Wenn du keine Karte entsorgen kannst, erhältst du kein zusätzliches Geld.}
\end{tikzpicture}
\hspace{-0.6cm}
\begin{tikzpicture}
	\card
	\cardstrip
	\cardbanner{banner/white.png}
	\cardicon{icons/coin.png}
	\cardprice{4}
	\cardtitle{Navigator}
	\cardcontent{Schau dir die obersten 5 Karten deines Nachziehstapels an. Sind nicht genügend Karten im Stapel, mischst du deinen Ablagestapel und legst ihn verdeckt unter deinen Nachziehstapel. Sind es nun immer noch weniger als 5 Karten, schaust du dir alle an und legst sie dann entweder ab oder in einer beliebigen Reihenfolge zurück auf den Nachziehstapel.}
\end{tikzpicture}
\hspace{-0.6cm}
\begin{tikzpicture}
	\card
	\cardstrip
	\cardbanner{banner/white.png}
	\cardicon{icons/coin.png}
	\cardprice{4}
	\cardtitle{\footnotesize{Piratenschiff}}
	\cardcontent{Wenn du dein erstes \emph{PIRATENSCHIFF} nimmst oder kaufst, erhältst du ein Piratenschiff-Tableau und legst es vor dir ab. 

	\medskip

	Immer wenn du ein \emph{PIRATENSCHIFF} ausspielst, wählst du \emph{eine} der beiden Anweisungen: 

	\smallskip

	Entweder die \emph{erste Anweisung}: Alle Mitspieler decken die beiden obersten Karten ihres Nachziehstapels auf. Dann entsorgen sie jeweils eine Geldkarte nach deiner Wahl. Hat ein Mitspieler keine Geldkarte aufgedeckt, entsorgt er keine Karte. Die restlichen aufgedeckten Karten werden abgelegt. Wird mindestens eine Karte entsorgt, erhältst du einen \emph{Geldmarker} und legst ihn auf dein Tableau; 

	\smallskip

	oder die \emph{zweite Anweisung}: Du erhältst pro \emph{Geldmarker} auf deinem Piratenschiff-Tableau in der Kaufphase +\coin[1].

	\medskip

	Nach der Nutzung in der Kaufphase verbleiben die Geldmarker auf dem Tableau und können beim erneuten Ausspielen eines \emph{PIRATENSCHIFFES} wieder eingesetzt werden. Mitspieler können auf das Ausspielen eines \emph{PIRATENSCHIFFES} mit Reaktionskarten reagieren, auch wenn du die zweite Anweisung wählst, die deine Mitspieler nicht direkt betrifft.}
\end{tikzpicture}
\hspace{-0.6cm}
\begin{tikzpicture}
	\card
	\cardstrip
	\cardbanner{banner/white.png}
	\cardicon{icons/coin.png}
	\cardprice{4}
	\cardtitle{Schatzkarte}
	\cardcontent{Nur wenn du zusätzlich zu der ausgespielten \emph{SCHATZKARTE} noch eine weitere auf der Hand hast und beide entsorgst, erhältst du 4 Gold. Sollten weniger als  4 Gold im Vorrat sein, nimmst du dir so viele Goldkarten wie vorhanden sind. Lege alle auf diese Weise erhaltenen Goldkarten verdeckt auf den Nachziehstapel. Solltest du nur eine \emph{SCHATZKARTE} auf der Hand haben und diese ausspielen, musst du diese Karte entsorgen, erhältst aber nichts dafür.}
\end{tikzpicture}
\hspace{-0.6cm}
\begin{tikzpicture}
	\card
	\cardstrip
	\cardbanner{banner/white.png}
	\cardicon{icons/coin.png}
	\cardprice{4}
	\cardtitle{Seehexe}
	\cardcontent{Sollte der Nachziehstapel eines Mitspielers leer sein, mischt er seinen Ablagestapel und legt die oberste Karte des neuen Nachziehstapels ab. Hat ein Spieler keine Karten mehr in seinem Nachziehstapel, kann er zwar keine Karte ablegen, nimmt sich aber trotzdem eine Fluchkarte. Beginnend mit dem Mitspieler links von dem Spieler, der die \emph{SEEHEXE} ausgespielt hat, nimmt sich jeder Mitspieler einen \emph{FLUCH} vom Vorrat. Sollten nicht mehr genügend Fluchkarten für alle Spieler vorhanden sein, werden die restlichen in o. g. Reihenfolge verteilt. }
\end{tikzpicture}
\hspace{-0.6cm}
\begin{tikzpicture}
	\card
	\cardstrip
	\cardbanner{banner/orange.png}
	\cardicon{icons/coin.png}
	\cardprice{5}
	\cardtitle{\footnotesize{Aussenposten}}
	\cardcontent{Der \emph{AUSSENPOSTEN} ist eine Dauerkarte, die bis zum Ende des nächsten Zuges (Extrazug) im Spiel bleibt und erst in der Aufräumphase des nächsten Zuges (Extrazug) abgelegt wird. Der \emph{AUSSENPOSTEN} kommt erst in der Aufräumphase des Zuges, in dem er ausgespielt wird, zum Einsatz. Du ziehst in diesem Fall nur 3 statt 5 Karten nach und führst den Extrazug \emph{sofort} aus. 

	\medskip

	Wenn du den \emph{AUSSENPOSTEN} zusammen mit weiteren Dauerkarten ausgespielt hast, kommen die \enquote{Zu Beginn deines nächsten Zuges}-Anweisungen der Dauerkarten in deinem Extrazug zum Einsatz. Spielst du in deinem Extrazug einen weiteren \emph{AUSSENPOSTEN}, erhältst du keinen weiteren Extrazug. Am Ende deines Extrazuges legst du den \emph{AUSSENPOSTEN} ab und ziehst 5 Karten nach.}
\end{tikzpicture}
\hspace{-0.6cm}
\begin{tikzpicture}
	\card
	\cardstrip
	\cardbanner{banner/white.png}
	\cardicon{icons/coin.png}
	\cardprice{5}
	\cardtitle{Bazar}
	\cardcontent{Du \emph{musst} eine Karte nachziehen, \emph{darfst} 2 weitere Aktionen ausführen und erhältst für die Kaufphase +\coin[2].}
\end{tikzpicture}
\hspace{-0.6cm}
\begin{tikzpicture}
	\card
	\cardstrip
	\cardbanner{banner/white.png}
	\cardicon{icons/coin.png}
	\cardprice{5}
	\cardtitle{Entdecker}
	\cardcontent{Wenn du eine Provinz aus der Hand aufdeckst, erhältst du ein Gold. Wenn du das nicht tun kannst (weil du keine Provinz auf der Hand hast) oder willst (weil du deine Provinz nicht zeigen möchtest), erhältst du ein Silber. Nimm das Gold oder Silber auf die Hand.}
\end{tikzpicture}
\hspace{-0.6cm}
\begin{tikzpicture}
	\card
	\cardstrip
	\cardbanner{banner/white.png}
	\cardicon{icons/coin.png}
	\cardprice{5}
	\cardtitle{\footnotesize{Geisterschiff}}
	\cardcontent{Deine Mitspieler müssen Karten aus ihrer Hand verdeckt auf den Nachziehstapel legen, bis sie nur noch 3 Karten auf der Hand haben. Welche Karten sie auf den Nachziehstapel legen, entscheiden die Mitspieler selbst. Spieler, die zum Zeitpunkt des Angriffs bereits 3 oder weniger Karten auf der Hand haben, müssen keine Karten auf den Nachziehstapel legen. }
\end{tikzpicture}
\hspace{-0.6cm}
\begin{tikzpicture}
	\card
	\cardstrip
	\cardbanner{banner/orange.png}
	\cardicon{icons/coin.png}
	\cardprice{5}
	\cardtitle{\footnotesize{Handelsschiff}}
	\cardcontent{Das \emph{HANDELSSCHIFF} ist eine Dauerkarte. Du erhältst für deine Kaufphase +\coin[2]. 

	\medskip

	Zu Beginn deines nächsten Zuges erhältst du +\coin[2] für die Kaufphase. Lege das \emph{HANDELSSCHIFF} in der Aufräumphase dieses Zuges ab.}
\end{tikzpicture}
\hspace{-0.6cm}
\begin{tikzpicture}
	\card
	\cardstrip
	\cardbanner{banner/white.png}
	\cardicon{icons/coin.png}
	\cardprice{5}
	\cardtitle{\footnotesize{Schatzkammer}}
	\cardcontent{Wenn du eine \emph{SCHATZKAMMER} spielst und in diesem Zug \emph{keine} Punktekarte gekauft hast, \emph{darfst} du die ausgespielte \emph{SCHATZKAMMER} in der Aufräumphase zurück auf den Nachziehstapel legen. 

	\medskip

	Wenn du mehrere \emph{SCHATZKAMMERN} ausgespielt hast, darfst du auch diese \emph{SCHATZKAMMERN} auf den Nachziehstapel zurücklegen. Wenn du eine Punktekarte auf andere Art nimmst bzw. erhältst (d. h. \emph{nicht} kaufst), darfst du \emph{SCHATZKAMMERN} zurück auf den Nachziehstapel legen. 

	\medskip

	Wenn du deine ausgespielte \emph{SCHATZKAMMER} gern zurücklegen möchtest, das aber in der Aufräumphase vergisst und die Karte bereits auf den Ablagestapel gelegt hast, darfst du dies nachträglich \emph{nicht} rückgängig machen.}
\end{tikzpicture}
\hspace{-0.6cm}
\begin{tikzpicture}
	\card
	\cardstrip
	\cardbanner{banner/orange.png}
	\cardicon{icons/coin.png}
	\cardprice{5}
	\cardtitle{Taktiker}
	\cardcontent{Der \emph{TAKTIKER} ist eine Dauerkarte. Sobald du diese Karte ausspielst, legst du alle Handkarten ab. Nur wenn du auf diese Weise mindestens eine Handkarte abgelegt hast, ziehst du zu Beginn deines nächsten Zuges 5 Karten. Außerdem erhältst du dann im nächsten Zug eine zusätzliche Aktion und einen zusätzlichen Kauf. 

	\medskip

	\emph{Grundsätzlich gilt: Nur wenn du mindestens eine Handkarte ablegen kannst, erhältst du den Bonus im nächsten Zug.}

	\medskip

	Wenn du den \emph{TAKTIKER} auf einen \emph{THRONSAAL} spielst, erhältst du den Bonus im nächsten Zug nur einmal, da du beim zweiten Ausspielen des \emph{TAKTIKERS} keine Handkarte mehr auf der Hand hast und damit die Bedingung nicht erfüllst. }
\end{tikzpicture}
\hspace{-0.6cm}
\begin{tikzpicture}
	\card
	\cardstrip
	\cardbanner{banner/orange.png}
	\cardicon{icons/coin.png}
	\cardprice{5}
	\cardtitle{Werft}
	\cardcontent{ Die \emph{WERFT} ist eine Dauerkarte. Du \emph{musst} sofort 2 Karten nachziehen und \emph{darfst} einen weiteren Kauf tätigen. 

	\medskip

	Zu Beginn deines nächsten Zuges (nicht vorher) \emph{musst} du wieder 2 Karten ziehen und \emph{darfst} einen weiteren Kauf tätigen.}
\end{tikzpicture}
\hspace{-0.6cm}
\begin{tikzpicture}
	\card
	\cardstrip
	\cardbanner{banner/white.png}
	\cardtitle{\scriptsize{Empfohlene 10er Sätze\qquad}}
	\cardcontent{\emph{Auf hoher See:}\\
	Ausguck, Bazar, Embargo, Entdecker, Hafen, Insel, Karawane, Piratenschiff, Schmuggler, Werft

	\smallskip

	\emph{Vergrabene Schätze:}\\
	Außenposten, Beutelschneider, Botschafter, Fischerdorf, Lagerhaus, Leuchtturm, Perlentaucher, Schatzkarte, Taktiker, Werft

	\smallskip

	\emph{Schiffswracks:}\\
	Eingeborenen, Geisterschiff, Handelsschiff, Lagerhaus, Leuchtturm, Perlentaucher, Schatzkammer, Schmuggler, Seehexe

	\smallskip

	\emph{Griff nach den Sternen} (Seaside + \textit{Basisspiel}):\\
	Ausguck, Beutelschneider, Geisterschiff, Schatzkarte, Seehexe, \textit{Abenteurer}, \textit{Dorf}, \textit{Keller}, \textit{Ratsversammlung}, \textit{Spion}

	\smallskip

	\emph{Wiederholungen} (Seaside + \textit{Basisspiel}):\\
	Außenposten, Entdecker, Karawane, Perlentaucher, Piratenschiff, Schatzkammer, \textit{Jahrmarkt}, \textit{Kanzler}, \textit{Miliz}, \textit{Werkstatt}

	\smallskip

	\emph{Geben und Nehmen} (Seaside + \textit{Basisspiel}):\\
	Botschafter, Fischerdorf, Hafen, Insel, Müllverwerter, Schmuggler, \textit{Bibliothek}, \textit{Geldverleiher}, \textit{Hexe}, \textit{Markt}}
\end{tikzpicture}
\hspace{0.6cm}

	    \input{sets/de/alchemy.tex}
		% Basic settings for this card set
\renewcommand{\cardcolor}{alchemy}
\renewcommand{\cardextension}{Erweiterung II}
\renewcommand{\cardextensiontitle}{Die Alchemisten}
\renewcommand{\seticon}{alchemy.png}

\clearpage
\newpage
\section{\cardextension \ - \cardextensiontitle \ (Rio Grande Games 2015)}

\begin{tikzpicture}
	\card
	\cardstrip
	\cardbanner{banner/green.png}
	\cardicon{icons/potion.png}
	\cardtitle{Weinberg}
	\cardcontent{Diese Karte ist eine Punktekarte und hat bis zum Ende des Spiels keine Funktion. Bei Spielende erhält der Spieler, der diese Karte in seinem Kartensatz (Nachziehstapel, Ablagestapel und Handkarten) hat, für jeweils 3 Aktionskarten (auch kombinierte Aktionskarten) 1 Siegpunkt. Es wird immer abgerundet, d.h. wer z.B. 12, 13 oder 14 Aktionskarten besitzt, erhält 4 Siegpunkte. Wer mehrere \emph{WEINBERGE} besitzt, erhält für jeden \emph{WEINBERG} die entsprechende Anzahl Siegpunkte.}
\end{tikzpicture}
\hspace{-0.6cm}
\begin{tikzpicture}
	\card
	\cardstrip
	\cardbanner{banner/white.png}
	\cardicon{icons/potion.png}
	\cardtitle{Verwandlung}
	\cardcontent{Wenn du diese Karte ausspielst und noch mindestens eine Karte auf der Hand hast, musst du eine Handkarte entsorgen. Wenn du keine Karte oder einen \emph{FLUCH} entsorgst, erhältst du nichts. Entsorgst du eine Aktions-, Punkte- oder Geldkarte, erhältst du den jeweiligen Bonus. Entsorgst du eine kombinierte Karte, erhältst du den Bonus beider Kartentypen. Sollte keine entsprechende Karte mehr im Vorrat sein, erhältst du nichts.}
\end{tikzpicture}
\hspace{-0.6cm}
\begin{tikzpicture}
	\card
	\cardstrip
	\cardbanner{banner/white.png}
	\cardicon{icons/coin.png}
	\cardprice{2}
	\cardtitle{\scriptsize{Kräuterkundiger}}
	\cardcontent{Wenn du den \emph{KRÄUTERKUNDIGEN} in der Aufräumphase ablegst, darfst du eine Geldkarte, die vor dir ausliegt, oben auf den Nachziehstapel legen. Ist der Nachziehstapel leer, legst du die Geldkarte auf den leeren Platz; sie ist dann die einzige Karte im Nachziehstapel. Wenn du mehrere \emph{KRÄUTERKUNDIGE} im Spiel hast und ablegst, darfst du für jeden eine ausliegende Geldkarte auf den Nachziehstapel legen.}
\end{tikzpicture}
\hspace{-0.6cm}
\begin{tikzpicture}
	\card
	\cardstrip
	\cardbanner{banner/white.png}
	\cardicon{icons/coin.png}
	\cardprice{2}
	\cardiconaddition{icons/potion.png}
	\cardtitle{\quad \footnotesize{Apotheker}}
	\cardcontent{Sollte der Nachziehstapel während des Aufdeckens aufgebraucht werden, mischst du deinen Ablagestapel und legst ihn als neuen Nachziehstapel bereit. Hast du dann trotzdem nicht genug Karten im Nachziehstapel, um 4 Karten aufzudecken, deckst du nur so viele Karten auf, wie möglich. Alle aufgedeckten \emph{KUPFER} und \emph{TRÄNKE} nimmst du auf die Hand und legst die anderen aufgedeckten Karten zurück auf den Nachziehstapel.}
\end{tikzpicture}
\hspace{-0.6cm}
\begin{tikzpicture}
	\card
	\cardstrip
	\cardbanner{banner/white.png}
	\cardicon{icons/coin.png}
	\cardprice{2}
	\cardiconaddition{icons/potion.png}
	\cardtitle{\quad \footnotesize{Universität}}
	\cardcontent{Du darfst eine Aktionskarte vom Vorrat nehmen, die bis zu \coin[5] kostet. Du darfst allerdings keine Karte nehmen, deren Kosten einen \potion beinhalten.}
\end{tikzpicture}
\hspace{-0.6cm}
\begin{tikzpicture}
	\card
	\cardstrip
	\cardbanner{banner/white.png}
	\cardicon{icons/coin.png}
	\cardprice{2}
	\cardiconaddition{icons/potion.png}
	\cardtitle{\quad Vision}
	\cardcontent{Alle Spieler, auch du selbst, decken die oberste Karte ihres Nachziehstapels auf. Für jeden Spieler entscheidest du separat, ob er die Karte ablegt oder zurück auf seinen Nachziehstapel legt. 

	\medskip

	Danach deckst du so lange Karten von deinem Nachziehstapel auf, bis du eine Karte, die \emph{keine} Aktionskarte ist, aufgedeckt hast (kombinierte Aktionskarten sind auch Aktionskarten). Nimm alle gerade aufgedeckten Karten auf die Hand. Hast du nur Aktionskarten aufgedeckt und dein Nachziehstapel ist aufgebraucht, mischst du deinen Ablagestapel und ziehst weiter, bis du eine Karte aufdeckst, die keine Aktionskarte ist. Findest du auch in diesem Stapel nur Aktionskarten, nimmst du alle Karten auf die Hand.}
\end{tikzpicture}
\hspace{-0.6cm}
\begin{tikzpicture}
	\card
	\cardstrip
	\cardbanner{banner/white.png}
	\cardicon{icons/coin.png}
	\cardprice{3}
	\cardiconaddition{icons/potion.png}
	\cardtitle{\quad Alchemist}
	\cardcontent{Wenn du zusätzlich zu dieser Karte einen \emph{TRANK} im Spiel hast, darfst du diese Karte in der Aufräumphase zurück auf den Nachziehstapel legen, statt sie abzulegen. Wenn du mindestens 1 \emph{TRANK} im Spiel hast, darfst du beliebig viele ausgespielte \emph{ALCHEMISTEN} zurück auf den Nachziehstapel legen.}
\end{tikzpicture}
\hspace{-0.6cm}
\begin{tikzpicture}
	\card
	\cardstrip
	\cardbanner{banner/gold.png}
	\cardicon{icons/coin.png}
	\cardprice{3}
	\cardiconaddition{icons/potion.png}
	\cardtitle{\quad \tiny{Stein der Weisen}}
	\cardcontent{Diese Karte ist eine Geldkarte mit einem variablen Wert und gehört nicht zu den Basiskarten (wie \emph{KUPFER} oder \emph{TRANK}), sondern zu den Königreichkarten. Ausgespielt wird sie aber – wie andere Geldkarten auch – in der Kaufphase. 

	\medskip

	\emph{Wichtig:} Geldkarten dürfen in der Kaufphase nur ausgespielt werden, bevor du die erste Karte kaufst (neue Regeln, S. 4).

	\medskip

	Zähle die Karten, die du in diesem Moment im Nachzieh- und Ablagestapel hast (Summe). Pro volle 5 Karten erhöht sich der Geldwert des \emph{STEIN DER WEISEN} für die Kaufphase in diesem Zug um \emph{1}. Spielst du mehrere \emph{STEIN DER WEISEN} aus, hat jede Karte den entsprechenden Geldwert. Du darfst beim Durchzählen der Karten deines Nachzieh- und Ablagestapels diese weder anschauen, noch ihre Reihenfolge verändern.}
\end{tikzpicture}
\hspace{-0.6cm}
\begin{tikzpicture}
	\card
	\cardstrip
	\cardbanner{banner/white.png}
	\cardicon{icons/coin.png}
	\cardprice{3}
	\cardiconaddition{icons/potion.png}
	\cardtitle{\quad \footnotesize{Vertrauter}}
	\cardcontent{Jeder Mitspieler, beginnend bei deinem linken Nachbarn, muss einen \emph{FLUCH} vom Vorrat nehmen und ihn ablegen. Wird der Vorrat an \emph{FLÜCHEN} dabei aufgebraucht, erhalten die Spieler, für die kein \emph{FLUCH} mehr vorhanden ist, nichts. }
\end{tikzpicture}
\hspace{-0.6cm}
\begin{tikzpicture}
	\card
	\cardstrip
	\cardbanner{banner/white.png}
	\cardicon{icons/coin.png}
	\cardprice{4}
	\cardiconaddition{icons/potion.png}
	\cardtitle{\quad Golem}
	\cardcontent{Decke solange Karten von deinem Nachziehstapel auf, bis du 2 Aktionskarten aufgedeckt hast, die keine \emph{GOLEMS} sind. Alle aufgedeckten \emph{GOLEMS} und Karten, die keine Aktionskarten sind, legst du ab. Hast du, auch nach dem Mischen des Ablagestapels, nur 1 oder gar keine Aktionskarte aufgedeckt, spielt du entsprechend weniger Aktionskarten aus. Du musst die aufgedeckten Aktionskarten ausspielen, darfst allerdings die Reihenfolge selbst bestimmen. Du darfst die Aktionskarten nicht auf die Hand nehmen. Anweisungen, die sich auf Handkarten beziehen, haben keine Auswirkungen auf die aufgedeckten Aktionskarten. Ist eine der aufgedeckten Karten z.B. ein \emph{THRONSAAL}, darfst du eine Karte aus der Hand auswählen und diese zwei Mal ausspielen, du darfst aber nicht die anderen aufgedeckten Karten dafür auswählen, da sie sich nicht auf deiner Hand befinden.}
\end{tikzpicture}
\hspace{-0.6cm}
\begin{tikzpicture}
	\card
	\cardstrip
	\cardbanner{banner/white.png}
	\cardicon{icons/coin.png}
	\cardprice{5}
	\cardtitle{Lehrling}
	\cardcontent{Entsorge eine Handkarte. Wenn du mindestens eine Handkarte hast, musst du eine Karte entsorgen. Pro \coin[X], das die entsorgte Karte kostet, ziehst du eine Karte nach. Wenn die entsorgte Karte außerdem \potion kostet, ziehst du weitere 2 Karten nach.}
\end{tikzpicture}
\hspace{-0.6cm}
\begin{tikzpicture}
	\card
	\cardstrip
	\cardbanner{banner/white.png}
	\cardicon{icons/coin.png}
	\cardprice{6}
	\cardiconaddition{icons/potion.png}
	\cardtitle{\quad \scriptsize{Besessenheit}}
	\cardcontent{\tiny{Zuerst spielst du deinen aktuellen Zug regulär zu Ende, bevor dein linker Nachbar einen Extra-Zug ausführen muss. Da die \emph{BESESSENHEIT} keine Angriffskarte ist, kann sich der Mitspieler nicht gegen den Extra-Zug \enquote{wehren}. 

	\medskip

	Zu Beginn des Extra-Zuges zeigt dein linker Nachbar dir seine Handkarten. Du entscheidest in diesem Zug alles für den Mitspieler – welche Aktionskarten und Geldkarten er spielt und welche Karten er kauft, nimmt, entsorgt etc. Du darfst alle Karten sehen, die auch der Mitspieler sieht – d. h. Handkarten, nachgezogene und angesehene Karten sowie die Karten, die der Mitspieler in der Aufräumphase des Extra-Zuges nachzieht.

	\medskip

	Alle Karten, die er nehmen, kaufen oder auf andere Art erhalten würde, erhältst stattdessen du und legst sie auf deinen Ablagestapel. Das betrifft auch Karten, die er auf die Hand nehmen oder anderweitig ablegen müsste. Alle Münzen (z. B. aus \emph{Die Gilden}) und Geldmarker (z. B. aus \emph{Seaside}), die der Spieler im \emph{BESESSENHEITS}-Zug erhält, bekommst du nicht. 

	\medskip

	Alle Karten, die der Spieler entsorgen müsste, werden separat neben den Müllstapel gelegt. Für weitere Anweisungen, die sich auf entsorgte Karten beziehen, gilt die Karte während des Extra-Zuges als entsorgt. Am Ende des Extra-Zuges legt der Mitspieler diese auf seinen eigenen Ablagestapel. 

	\medskip

	Alle Karten, die der Mitspieler während des Extra-Zuges in den Vorrat zurücklegen muss, werden tatsächlich in den Vorrat zurückgelegt.}}
\end{tikzpicture}
\hspace{-0.6cm}
\begin{tikzpicture}
	\card
	\cardstrip
	\cardbanner{banner/gold.png}
	\cardicon{icons/coin.png}
	\cardprice{4}
	\cardtitle{Trank}
\end{tikzpicture}
\hspace{-0.6cm}
\begin{tikzpicture}
	\card
	\cardstrip
	\cardbanner{banner/white.png}
	\cardtitle{\scriptsize{Empfohlene 10er Sätze\qquad}}
	\cardcontent{\emph{Verbotene Künste} (Alchemisten + \textit{Basisspiel}):\\
	Besessenheit, Lehrling, Universität, Vertrauter, \textit{Dieb}, \textit{Gärten}, \textit{Keller}, \textit{Laboratorium}, \textit{Ratsversammlung}, \textit{Thronsaal}

	\smallskip

	\emph{Quacksalber:} (Alchemisten + \textit{Basisspiel}):\\
	Alchemist, Apotheker, Golem, Kräuterkundiger, Verwandlung, \textit{Jahrmarkt}, \textit{Kanzler}, \textit{Keller}, \textit{Miliz}, \textit{Schmiede}

	\smallskip

	\emph{Chemiestunde:} (Alchemisten + \textit{Basisspiel}):\\
	Alchemist, Golem, Stein der Weisen, Universität, \textit{Burggraben}, \textit{Bürokrat}, \textit{Hexe}, \textit{Holzfäller}, \textit{Markt}, \textit{Umbau}

	\smallskip

	\emph{Diener:} (Alchemisten + \textit{Die Intrige}):\\
	Besessenheit, Golem, Verwandlung, Vision, Weinberg, \textit{Große Halle}, \textit{Handlanger}, \textit{Lakai}, \textit{Verschwörer}, \textit{Verwalter}

	\smallskip

	\emph{Geheime Forschungen:} (Alchemisten + \textit{Die Intrige}):\\
	Kräuterkundiger, Stein der Weisen, Universität, Vertrauter, \textit{Adlige}, \textit{Armenviertel}, \textit{Brücke}, \textit{Kerkermeister}, \textit{Lakai}, \textit{Maskerade}

	\smallskip

	\emph{Tröpfe, Tränke, Trottel:} (Alchemisten + \textit{Die Intrige}):\\
	Apotheker, Golem, Lehrling, Vision, \textit{Adlige}, \textit{Baron}, \textit{Eisenhütte}, \textit{Handelsposten}, \textit{Kupferschmied}, \textit{Wunschbrunnen}}
\end{tikzpicture}
\hspace{0.6cm}

	    % Basic settings for this card set
\renewcommand{\cardcolor}{prosperity}
\renewcommand{\cardextension}{Erweiterung III}
\renewcommand{\cardextensiontitle}{Blütezeit}

\clearpage
\newpage
\section{\cardextension \ - \cardextensiontitle}

\begin{tikzpicture}
	\card
	\cardstrip
	\cardbanner{banner/white.png}
	\cardicon{banner/coin.png}
	\cardprice{}
	\cardtitle{}
	\cardcontent{}
\end{tikzpicture}

\hspace{-1cm}
\begin{tikzpicture}
\card
\cardstrip
\cardbanner{banner/white.png}
\cardicon{}
\cardprice{}
\cardtitle{\scriptsize{Empfohlene 10er Sätze\qquad}}
\cardcontent{\emph{Name:}
	\\
	Karten ...
	\\
	\smallskip
	\\
	\emph{Name:}
	\\
	Karten ...
	\\
	\smallskip
	\\
	\emph{Name:}
	\\
	Karten ...
	\\
	\smallskip
	\\
	\emph{Name:}
	\\
	Karten ...
	\\
	\smallskip
	\\
	\emph{Name:}
	\\
	Karten ...
	\\
	\smallskip
	\\
	\emph{Name:}
	\\
	Karten ...
	\\
}
\end{tikzpicture}
\hspace{1cm}
	    % Basic settings for this card set
\renewcommand{\cardcolor}{prosperity}
\renewcommand{\cardextension}{Erweiterung III}
\renewcommand{\cardextensiontitle}{Blütezeit}
\renewcommand{\seticon}{prosperity.png}

\clearpage
\newpage
\section{\cardextension \ - \cardextensiontitle \ (Rio Grande Games 2016)}

\begin{tikzpicture}
	\card
	\cardstrip
	\cardbanner{banner/white.png}
	\cardicon{icons/coin.png}
	\cardprice{3}
	\cardtitle{\footnotesize{Handelsroute}}
	\cardcontent{Du erhältst + 1 Kauf und +\coin[1] pro Geldmarker, der sich zum Zeitpunkt des Ausspielens auf dem Handelsrouten-Tableau befindet. Du musst eine Handkarte entsorgen, wenn du mindestens 1 Karte auf der Hand hast.

	\medskip

	In allen Spielen mit der \emph{HANDELSROUTE} (auch als Teil des \emph{SCHWARZMARKTES}) wird zu Beginn des Spiels das Handelsrouten-Tableau neben dem Vorrat bereit gelegt. Außerdem wird auf jeden Punkte-Vorratsstapel (\emph{ANWESEN}, \emph{HERZOGTUM}, \emph{PROVINZ}, ggf. \emph{KOLONIE} sowie alle kombinierten oder reinen Punktekarten unter den Königreichkarten (z.B. \emph{GÄRTEN} aus Basisspiel oder \emph{INSEL} aus Seaside)) ein Geldmarker gelegt. Sobald die erste Karte eines Stapels genommen wird (egal von welchem Spieler und egal auf welche Weise), legt ihr den entsprechenden Geldmarker auf das Handelsrouten-Tableau. Es wird kein neuer Marker auf den Vorratsstapel gelegt. Auch wird unter keinen Umständen ein Marker vom Handelsrouten-Tableau entfernt.}
\end{tikzpicture}
\hspace{-0.6cm}
\begin{tikzpicture}
	\card
	\cardstrip
	\cardbanner{banner/gold.png}
	\cardicon{icons/coin.png}
	\cardprice{3}
	\cardtitle{Lohn}
	\cardcontent{Diese Karte ist eine Geldkarte mit zusätzlichen Anweisungen. Sie hat den Wert \coin[1].

	\medskip

	Decke solange Karten von deinem Nachziehstapel auf, bis du die erste Geld- oder kombinierte Geldkarte aufdeckst. Entsorge die aufgedeckte Geldkarte oder lege sie ab. Alle anderen aufgedeckten Karten legst du ab.}
\end{tikzpicture}
\hspace{-0.6cm}
\begin{tikzpicture}
	\card
	\cardstrip
	\cardbanner{banner/blue.png}
	\cardicon{icons/coin.png}
	\cardprice{3}
	\cardtitle{Wachturm}
	\cardcontent{Diese Karte ist eine kombinierte Aktions- und Reaktionskarte. Sie kann in der Aktionsphase ausgespielt werden (Anweisung über der Trennlinie) oder als Reaktion auf die unter der Trennlinie angegebene Situation.

	\medskip

	Spielst du den \emph{WACHTURM} in deiner Aktionsphase aus, ziehst du solange Karten nach, bis du 6 Karten auf der Hand hast. Hast du bereits 6 oder mehr Handkarten, ziehst du keine Karten nach. 

	\medskip

	Wenn du den \emph{WACHTURM} auf der Hand hast und du eine Karte nimmst (in deinem eigenen Zug oder während des Zuges eines Mitspielers), darfst du ihn als Reaktion aufdecken und die genommene Karte entweder entsorgen oder auf deinen Nachziehstapel legen. Anschließend nimmst du den \emph{WACHTURM} wieder auf die Hand. Nimmst du anschließend eine oder mehrere weitere Karten (durch die gleiche oder eine andere Anweisung bzw. durch einen Kauf), kannst du den \emph{WACHTURM} erneut aufdecken – solange du ihn auf der Hand hast. Hast du den \emph{WACHTURM} in deiner nächsten Aktionsphase noch immer auf der Hand, darfst du ihn ausspielen. 

	\medskip

	Wenn ein Mitspieler gegen dich die \emph{BESESSENHEIT} (aus \emph{Alchemie}) gespielt hat und du den Extrazug ausführst, darfst du den \emph{WACHTURM} nicht aufdecken, da der Mitspieler die Karten nimmt und nicht du.}
\end{tikzpicture}
\hspace{-0.6cm}
\begin{tikzpicture}
	\card
	\cardstrip
	\cardbanner{banner/white.png}
	\cardicon{icons/coin.png}
	\cardprice{4}
	\cardtitle{Arbeiterdorf}
	\cardcontent{Du \emph{musst} eine Karte ziehen, \emph{darfst} 2 weitere Aktionen ausführen und in der Kaufphase einen zusätzlichen Kauf durchführen.}
\end{tikzpicture}
\hspace{-0.6cm}
\begin{tikzpicture}
	\card
	\cardstrip
	\cardbanner{banner/white.png}
	\cardicon{icons/coin.png}
	\cardprice{4}
	\cardtitle{Bischof}
	\cardcontent{Du erhältst +\coin[1] und legst einen \victorypointtoken-Marker auf dein Spieler-Tableau. Dann musst du eine Handkarte entsorgen, wenn du mindestens 1 Karte auf der Hand hast. Lege halb so viele \victorypointtoken-Marker auf dein Spieler-Tableau, wie die entsorgte Karte gekostet hat. Ungerade Kosten werden abgerundet. Die \potion-Kosten (z.B. aus \emph{Alchemie}) spielen keine Rolle. So erhältst du je 2 \victorypointtoken-Marker für ein entsorgtes \emph{ARBEITERDORF} (Kosten: \coin[4]), ebenso wie für ein \emph{GESINDEL} (Kosten: \coin[5]) oder einen \emph{GOLEM} (Kosten: \coin[4] und \potion). 
	\
	\medskip

	Jeder Mitspieler darf eine Handkarte entsorgen, erhält dafür aber keine Siegpunktmarker.}
\end{tikzpicture}
\hspace{-0.6cm}
\begin{tikzpicture}
	\card
	\cardstrip
	\cardbanner{banner/white.png}
	\cardicon{icons/coin.png}
	\cardprice{4}
	\cardtitle{Denkmal}
	\cardcontent{Du erhältst +\coin[2] und legst einen \victorypointtoken-Marker aus dem Vorrat auf dein Spieler-Tableau.}
\end{tikzpicture}
\hspace{-0.6cm}
\begin{tikzpicture}
	\card
	\cardstrip
	\cardbanner{banner/gold.png}
	\cardicon{icons/coin.png}
	\cardprice{4}
	\cardtitle{Steinbruch}
	\cardcontent{Diese Karte ist eine Geldkarte mit zusätzlichen Anweisungen. Sie hat den Wert \coin[1].

	\medskip

	Solange diese Karte im Spiel ist, kosten alle Aktionskarten (auch kombinierte) \coin[2] weniger. Dies betrifft alle Aktionskarten, d.h. auch Handkarten, Karten in den Ablage- und Nachziehstapeln etc. Der Effekt ist kumulativ, d.h. mit einem zweiten \emph{STEINBRUCH} oder anderen Aktionskarten, die die Kosten von Karten reduzieren, können die Kosten weiter gesenkt werden.}
\end{tikzpicture}
\hspace{-0.6cm}
\begin{tikzpicture}
	\card
	\cardstrip
	\cardbanner{banner/gold.png}
	\cardicon{icons/coin.png}
	\cardprice{4}
	\cardtitle{Talisman}
	\cardcontent{Diese Karte ist eine Geldkarte mit zusätzlichen Anweisungen. Sie hat den  Wert \coin[1].

	\medskip

	Solange diese Karte im Spiel ist und du eine Nicht-Punktekarte kaufst (nicht wenn du sie auf andere Weise nimmst), die zu diesem Zeitpunkt maximal \coin[4] kostet, nimmst du dir eine weitere gleiche Karte vom Vorrat. Ist keine gleiche Karte im Vorrat, nimmst du dir keine weitere Karte. Kaufst du in einem Zug mehrere Karten, die maximal \coin[4] kosten, wendest du den Effekt des \emph{TALISMANS} auf alle diese Karten an.}
\end{tikzpicture}
\hspace{-0.6cm}
\begin{tikzpicture}
	\card
	\cardstrip
	\cardbanner{banner/gold.png}
	\cardicon{icons/coin.png}
	\cardprice{5}
	\cardtitle{Abenteuer}
	\cardcontent{Diese Karte ist eine Geldkarte mit zusätzlichen Anweisungen. Sie hat den Wert \coin[1]. 

	\medskip

	Sobald du diese Karte ausspielst (normalerweise in der Kaufphase), deckst du solange Karten vom Nachziehstapel auf, bis du die erste Geldkarte (auch kombinierte) aufdeckst. Ist der Nachziehstapel aufgebraucht, ohne eine Geldkarte zu finden, mischst du deinen Ablagestapel. Findest du auch dort keine Geldkarte, legst du alle aufgedeckten Karten ab. Spiele die erste aufgedeckte Geldkarte sofort aus und führe ggf. zusätzliche Anweisungen auf dieser Karte aus. Lege alle anderen aufgedeckten Karten ab.}
\end{tikzpicture}
\hspace{-0.6cm}
\begin{tikzpicture}
	\card
	\cardstrip
	\cardbanner{banner/white.png}
	\cardicon{icons/coin.png}
	\cardprice{5}
	\cardtitle{Gesindel}
	\cardcontent{Ziehe drei Karten nach. Anschließend muss jeder Mitspieler (beginnend bei deinem linken Nachbarn) die obersten drei Karten seines Nachziehstapels aufdecken und alle aufgedeckten Geldkarten sowie Aktionskarten (auch kombinierte), ablegen. Alle anderen aufgedeckten Karten legt er in beliebiger Reihenfolge zurück auf den Nachziehstapel.}
\end{tikzpicture}
\hspace{-0.6cm}
\begin{tikzpicture}
	\card
	\cardstrip
	\cardbanner{banner/white.png}
	\cardicon{icons/coin.png}
	\cardprice{5}
	\cardtitle{Gewölbe}
	\cardcontent{Ziehe 2 Karten nach. Lege anschließend beliebig viele Handkarten (auch 0) ab. Du darfst auch Karten ablegen, die du gerade erst nachgezogen hast. Für jede abgelegte Karte erhältst du +\coin[1].

	\medskip

	Jeder Mitspieler darf 2 Handkarten ablegen und eine Karte nachziehen. Falls ein Mitspieler nur 1 Handkarte hat, darf er diese zwar ablegen, jedoch keine Karte nachziehen.}
\end{tikzpicture}
\hspace{-0.6cm}
\begin{tikzpicture}
	\card
	\cardstrip
	\cardbanner{banner/gold.png}
	\cardicon{icons/coin.png}
	\cardprice{6}
	\cardtitle{Hort}
	\cardcontent{Diese Karte ist eine Geldkarte mit zusätzlichen Anweisungen. Sie hat den Wert \coin[2].

	\medskip

	Solange diese Karte im Spiel ist und du eine Punktekarte (auch kombinierte) kaufst, nimmst du ein \emph{GOLD} vom Vorrat. Wenn kein \emph{GOLD} mehr im Vorrat ist, erhältst du nichts. Nimmst du eine Punktekarte auf andere Weise (d.h. nicht durch einen Kauf), nimmst du dir kein \emph{GOLD}. Hast du zwei \emph{HORTE} im Spiel, nimmst du dir pro gekaufter Punktekarte zwei \emph{GOLD} usw. Kaufst du in einem Spielzug zwei oder mehr Punktekarten, nimmst du dir für jede Punktekarte entsprechend viele \emph{GOLD} vom Vorrat. Du erhältst auch \emph{GOLD}, wenn du die gekaufte Punktekarte im gleichen Spielzug wieder entsorgst.}
\end{tikzpicture}
\hspace{-0.6cm}
\begin{tikzpicture}
	\card
	\cardstrip
	\cardbanner{banner/gold.png}
	\cardicon{icons/coin.png}
	\cardprice{5}
	\cardtitle{\tiny{Königliches Siegel}}
	\cardcontent{Diese Karte ist eine Geldkarte mit zusätzlichen Anweisungen. Sie hat den Wert \coin[2].

	\medskip

	Solange diese Karte im Spiel ist, entscheidest du für jede Karte, die du kaufst oder auf andere Weise nimmst, ob du sie ablegen oder oben auf deinen Nachziehstapel legen möchtest.}
\end{tikzpicture}
\hspace{-0.6cm}
\begin{tikzpicture}
	\card
	\cardstrip
	\cardbanner{banner/white.png}
	\cardicon{icons/coin.png}
	\cardprice{5}
	\cardtitle{Leihhaus}
	\cardcontent{Sieh dir deinen kompletten Ablagestapel an und nimm beliebig viele \emph{KUPFER} auf die Hand. Zeige diese vorher deinen Mitspielern. Die restlichen Karten legst du in beliebiger Reihenfolge wieder auf den Ablagestapel.}
\end{tikzpicture}
\hspace{-0.6cm}
\begin{tikzpicture}
	\card
	\cardstrip
	\cardbanner{banner/white.png}
	\cardicon{icons/coin.png}
	\cardprice{5}
	\cardtitle{Münzer}
	\cardcontent{Du darfst eine Geldkarte aus deiner Hand aufdecken. Wenn du das tust, nimm dir eine Karte mit gleichem Namen vom Vorrat. Ist keine entsprechende Karte im Vorrat vorhanden, erhältst du nichts. Nimm die aufgedeckte Geldkarte zurück auf die Hand.

	\medskip

	Wenn du den \emph{MÜNZER} kaufst, musst du alle Geldkarten, die du im Spiel hast, sofort entsorgen. Sofern du nach dem Kauf des \emph{MÜNZERS} noch \coin und einen weiteren Kauf übrig hast, darfst du die Geldwerte der entsorgten Karten in diesem Zug noch verwenden. Wenn du den \emph{MÜNZER} kaufst und eine Geldkarte im Spiel ist, deren zusätzliche Anweisung in Kraft tritt, sobald du eine Karte kaufst (z.B. \emph{KÖNIGLICHES SIEGEL}), wird diese Geldkarte entsorgt, bevor deren Effekt eintreten kann.

	\medskip

	Beachte, dass du alle Geldkarten, die du in deiner Kaufphase verwenden möchtest, vor deinem ersten Kauf ausspielen musst. Du kannst nicht 1 \emph{GOLD} und 1 \emph{SILBER} ausspielen, den \emph{MÜNZER} kaufen, diese Geldkarten entsorgen und dann weiteres Geld auslegen. Sobald eine Karte gekauft wurde, dürfen keinen weiteren Geldkarten mehr ausgespielt werden.}
\end{tikzpicture}
\hspace{-0.6cm}
\begin{tikzpicture}
	\card
	\cardstrip
	\cardbanner{banner/white.png}
	\cardicon{icons/coin.png}
	\cardprice{5}
	\cardtitle{Quacksalber}
	\cardcontent{Du erhältst +\coin[2].

	\medskip

	Jeder Mitspieler (beginnend bei deinem linken Nachbarn) legt entweder einen \emph{FLUCH} aus seiner Hand ab oder er nimmt einen \emph{FLUCH} und ein \emph{KUPFER} vom Vorrat und legt diese ab. Dies dürfen die Spieler auch wählen, wenn der \emph{KUPFER}- oder \emph{FLUCH}-Stapel leer sind. Nimmt ein Spieler den \emph{FLUCH} und das \emph{KUPFER} und hat einen \emph{WACHTURM} auf der Hand, darf er diesen nach jeder genommenen Karte aufdecken (oder wahlweise nur nach einer) und z.B. den \emph{FLUCH} entsorgen und das \emph{KUPFER} auf den Nachziehstapel legen (oder ebenfalls entsorgen).}
\end{tikzpicture}
\hspace{-0.6cm}
\begin{tikzpicture}
	\card
	\cardstrip
	\cardbanner{banner/gold.png}
	\cardicon{icons/coin.png}
	\cardprice{5}
	\cardtitle{\scriptsize{Schmuggelware}}
	\cardcontent{Diese Karte ist eine Geldkarte mit zusätzlichen Anweisungen. Sie hat den Wert \coin[3]. Du erhältst + 1 Kauf.

	\medskip

	Dein linker Mitspieler nennt den Namen einer beliebigen Karte (z.B. \enquote{Provinz}). Diese muss nicht Teil des Vorrats sein. Du darfst diese Karte in diesem Zug nicht kaufen. Wenn du die Karte auf andere Weise nehmen kannst, darfst du das tun. Spielst du mehrere \emph{SCHMUGGELWAREN} aus, nennt dein linker Mitspieler entsprechend viele Karten.}
\end{tikzpicture}
\hspace{-0.6cm}
\begin{tikzpicture}
	\card
	\cardstrip
	\cardbanner{banner/white.png}
	\cardicon{icons/coin.png}
	\cardprice{5}
	\cardtitle{Stadt}
	\cardcontent{Du erhältst + 1 Karte sowie + 2 Aktionen. Wenn noch kein Vorratsstapel leer ist, passiert nichts weiter. Wenn genau 1 Vorratsstapel leer ist, erhältst du nochmal + 1 Karte. Wenn 2 oder mehr Vorratsstapel leer sind, erhältst du stattdessen + 1 Karte, +\coin[1] und + 1 Kauf. }
\end{tikzpicture}
\hspace{-0.6cm}
\begin{tikzpicture}
	\card
	\cardstrip
	\cardbanner{banner/white.png}
	\cardicon{icons/coin.png}
	\cardprice{6}
	\cardtitle{\scriptsize{Großer Markt}}
	\cardcontent{Du erhältst + 1 Karte, + 1 Aktion, + 1 Kauf sowie +\coin[2].

	\medskip

	Wenn du diese Karte kaufen möchtest, darfst du zu diesem Zeitpunkt kein \emph{KUPFER} im Spiel haben. Wenn du zu einem früheren Zeitpunkt in deinem Zug \emph{KUPFER} im Spiel hattest, dieses aber entsorgt hast, darfst du den \emph{GROSSEN MARKT} kaufen. Kannst du den \emph{GROSSEN MARKT} auf andere Art nehmen, darfst du das jederzeit tun, auch wenn du \emph{KUPFER} im Spiel hast.}
\end{tikzpicture}
\hspace{-0.6cm}
\begin{tikzpicture}
	\card
	\cardstrip
	\cardbanner{banner/white.png}
	\cardicon{icons/coin.png}
	\cardprice{7}
	\cardtitle{Ausbau}
	\cardcontent{Entsorge eine beliebige Handkarte und nimm eine Karte vom Vorrat, die bis zu \coin[3] mehr kostet als die entsorgte Karte. Du darfst den Betrag nicht mit zusätzlichem \coin erhöhen. Hast du keine Karte auf der Hand, die du entsorgen kannst, darfst du dir keine Karte vom Vorrat nehmen. Den ausgespielten \emph{AUSBAU} selbst darfst du nicht entsorgen, da er sich nicht mehr auf deiner Hand befindet.}
\end{tikzpicture}
\hspace{-0.6cm}
\begin{tikzpicture}
	\card
	\cardstrip
	\cardbanner{banner/white.png}
	\cardicon{icons/coin.png}
	\cardprice{6}
	\cardtitle{\scriptsize{Halsabschneider}}
	\cardcontent{Du erhältst + 1 Kauf sowie +\coin[2]. Alle Mitspieler müssen Handkarten ablegen, bis sie nur noch 3 Karten auf der Hand haben. Hat ein Mitspieler bereits 3 oder weniger Karten auf der Hand, muss er keine Karten ablegen.

	\medskip

	Solange diese Karte im Spiel ist, legst du immer, wenn du eine Karte kaufst (nicht wenn du sie auf andere Weise nimmst), einen \victorypointtoken-Marker auf dein Spieler-Tableau. Hast du zwei \emph{HALSABSCHNEIDER} im Spiel, legst du zwei \victorypointtoken-Marker pro gekaufter Karte auf dein Tableau usw.}
\end{tikzpicture}
\hspace{-0.6cm}
\begin{tikzpicture}
	\card
	\cardstrip
	\cardbanner{banner/gold.png}
	\cardicon{icons/coin.png}
	\cardprice{7}
	\cardtitle{Bank}
	\cardcontent{Diese Karte ist eine Geldkarte mit einem variablen Wert: Pro Geldkarte (inklusive dieser / auch kombinierte Geldkarten), die du im Spiel hast, ist sie \coin[1] wert. Spielst du die \emph{BANK} als erste Geldkarte in deinem Zug aus, ist sie genau \coin[1] wert. Spielst du dagegen zuerst ein \emph{GOLD}, ein \emph{SILBER} und zwei \emph{KUPFER} und dann die \emph{BANK}, ist die \emph{BANK} \coin[5] wert. Spielst du im Anschluss noch eine \emph{BANK} aus, bleibt die erste \emph{BANK} \coin[5] wert, die zweite \coin[6].}
\end{tikzpicture}
\hspace{-0.6cm}
\begin{tikzpicture}
	\card
	\cardstrip
	\cardbanner{banner/white.png}
	\cardicon{icons/coin.png}
	\cardprice{7}
	\cardtitle{Königshof}
	\cardcontent{Diese Karte ist ähnlich dem \emph{THRONSAAL} aus dem Basisspiel – mit dem Unterschied, dass du die Aktionskarte, die du aus deiner Hand wählst, dreimal (statt zweimal) ausspielst. Lege die gewählte Aktionskarte aus, führe die Anweisungen darauf komplett aus, nimm sie zurück auf die Hand, spiele sie erneut aus, führe die Anweisungen darauf komplett aus, nimm sie zurück auf die Hand und spiele sie ein drittes Mal aus und führe die Anweisungen darauf komplett aus. Für das dreimalige Ausspielen der Aktionskarte benötigst du keine Aktionen. Du darfst zwischen dem dreimaligen Ausspielen der Aktionskarte keine andere Aktion ausspielen, außer die Aktionskarte selbst gibt dazu die Anweisung.  }
\end{tikzpicture}
\hspace{-0.6cm}
\begin{tikzpicture}
	\card
	\cardstrip
	\cardbanner{banner/white.png}
	\cardicon{icons/coin.png}
	\cardprice{7}
	\cardtitle{\footnotesize{Kunstschmiede}}
	\cardcontent{Egal ob du keine Karte entsorgst (\coin[0] insgesamt) oder z.B. drei Karten, die jeweils \coin[2] kosten (\coin[6] insgesamt) – du \emph{musst} eine Karte vom Vorrat nehmen, die genau so viel kostet wie die entsorgten Karten zusammen gekostet haben, außer es ist keine entsprechende Karte im Vorrat vorhanden. Entsorgst du keine Karten und ist beispielsweise der \emph{KUPFER}-Stapel leer, musst du dir u.U. (wenn keine anderen Karten mit \coin[0]-Kosten vorhanden sind) einen \emph{FLUCH} vom Vorrat nehmen, der ebenfalls \coin[0] kostet. \potion-Kosten für Karten aus Alchemie haben für die \emph{KUNSTSCHMIEDE} keine Auswirkung. Es darf auch keine Karte, die \potion-Kosten enthält, genommen werden.}
\end{tikzpicture}
\hspace{-0.6cm}
\begin{tikzpicture}
	\card
	\cardstrip
	\cardbanner{banner/white.png}
	\cardicon{icons/coin.png}
	\cardprice{8*}
	\cardtitle{Hausierer}
	\cardcontent{Diese Karte ist eine Karte mit variablen Kosten (vgl. NEUE REGELN; S. 6). Du erhältst + 1 Karte, + 1 Aktion sowie +\coin[1]. 

	\medskip

	Wenn du diese Karte in deiner Kaufphase kaufst, kostet sie für jede Aktionskarte, die du im Spiel hast, \coin[2] weniger, niemals allerdings weniger als \coin[0]. Kaufst du eine Karte außerhalb der Kaufphase (z.B. durch den \emph{SCHWARZMARKT}), kostet der HAUSIERER \coin[8], egal ob du weitere Aktionskarten im Spiel hast oder nicht. Aktionskarten, die durch den \emph{THRONSAAL} (aus Basisspiel) oder den \emph{KÖNIGSHOF} mehrfach ausgespielt wurden, sind trotzdem jeweils nur einmal im Spiel und reduzieren die Kosten eines \emph{HAUSIERERS} um \coin[2].}
\end{tikzpicture}
\hspace{-0.6cm}
\begin{tikzpicture}
	\card
	\cardstrip
	\cardbanner{banner/gold.png}
	\cardicon{icons/coin.png}
	\cardprice{9}
	\cardtitle{Platin}
	\cardcontent{Diese Karte ist eine Basiskarte und keine Königreichkarte. Spielt ihr ausschließlich mit Königreichkarten aus Blütezeit, wird diese Karte zusätzlich zu den Basis-Geldkarten \emph{KUPFER}, \emph{SILBER} und \emph{GOLD} in der Spielvorbereitung in den Vorrat gelegt. Bei Spielen mit Königreichkarten aus verschiedenen Editionen oder Erweiterungen entscheidet vor Spielbeginn, ob ihr \emph{PLATIN} in den Vorrat legen wollt oder nicht (vgl. SPIELVORBEREITUNG, S. 4).}
\end{tikzpicture}
\hspace{-0.6cm}
\begin{tikzpicture}
	\card
	\cardstrip
	\cardbanner{banner/green.png}
	\cardicon{icons/coin.png}
	\cardprice{11}
	\cardtitle{Kolonien}
	\cardcontent{Diese Karte ist eine Basiskarte und keine Königreichkarte. Spielt ihr ausschließlich mit Königreichkarten aus Blütezeit, wird diese Karte zusätzlich zu den Basis-Punktekarten \emph{ANWESEN}, \emph{HERZOGTUM} und \emph{PROVINZ} in der Spielvorbereitung in den Vorrat gelegt. Bei Spielen mit Königreichkarten aus verschiedenen Editionen oder Erweiterungen entscheidet vor Spielbeginn, ob ihr die \emph{KOLONIE} in den Vorrat legen wollt oder nicht. Achtet darauf, dass in diesem Fall das Spiel auch endet, wenn der Vorratsstapel \emph{KOLONIE} leer ist (vgl. SPIELVORBEREITUNG, S. 4 sowie ALTERNATIVES SPIELENDE, S. 6).}
\end{tikzpicture}
\hspace{-0.6cm}
\begin{tikzpicture}
	\card
	\cardstrip
	\cardbanner{banner/white.png}
	\cardtitle{\scriptsize{Empfohlene 10er Sätze\qquad}}
	\cardcontent{\emph{Anfänger:}\\
	Abenteuer, Arbeiterdorf, Ausbau, Bank, Denkmal, Gesindel, Halsabschneider, Königliches Siegel, Leihhaus, Wachturm

	\smallskip

	\emph{Freundliche Interaktion:}\\
	Arbeiterdorf, Bischof, Gewölbe, Handelsroute, Hausierer, Hort, Königliches Siegel, Kunstschmiede, Schmuggelware, Stadt 

	\smallskip

	\emph{Große Aktionen:}\\
	Ausbau, Gesindel, Gewölbe, Großer Markt, Königshof, Lohn, Münzer, Stadt, Steinbruch, Talisman

	\smallskip

	\emph{Haufenweise Geld} (Blütezeit + \textit{Basisspiel}):\\
	Abenteuer, Bank, Großer Markt, Königliches Siegel, Münzer, \textit{Abenteurer}, \textit{Geldverleiher}, \textit{Laboratorium}, \textit{Mine}, \textit{Spion}

	\smallskip

	\emph{Die Armee des Königs} (Blütezeit + \textit{Basisspiel}):\\
	Ausbau, Gesindel, Gewölbe, Handlanger, Königshof, \textit{Bürokrat}, \textit{Burggraben}, \textit{Dorf}, \textit{Ratsversammlung}, \textit{Spion}

	\smallskip

	\emph{Ein gutes Leben:} (Blütezeit + \textit{Basisspiel}):\\
	Denkmal, Hort, Leihhaus, Quacksalber, Schmuggelware, \textit{Bürokrat}, \textit{Dorf}, \textit{Gärten}, \textit{Kanzler}, \textit{Keller}}
\end{tikzpicture}
\hspace{-0.6cm}
\begin{tikzpicture}
	\card
	\cardstrip
	\cardbanner{banner/white.png}
	\cardtitle{\scriptsize{Empfohlene 10er Sätze\qquad}}
	\cardcontent{\emph{Pfade zum Sieg} (Blütezeit + \textit{Die Intrige}):\\
	Bischof, Denkmal, Halsabschneider, Hausierer, Leihhaus, \textit{Anbau}, \textit{Armenviertel}, \textit{Baron}, \textit{Handlanger}, \textit{Harem}

	\smallskip

	\emph{All along the watchtower} (Blütezeit + \textit{Die Intrige}):\\
	Gewölbe, Handelsroute, Hort, Talisman, Wachturm, \textit{Bergwerk}, \textit{Brücke}, \textit{Große Halle}, \textit{Handlanger}, \textit{Kerkermeister}

	\smallskip

	\emph{Glücksritter} (Blütezeit + \textit{Die Intrige}):\\
	Ausbau, Bank, Gewölbe, Königshof, Kunstschmiede, \textit{Brücke}, \textit{Kupferschmied}, \textit{Tribut}, \textit{Trickser}, \textit{Wunschbrunnen}}
\end{tikzpicture}
\hspace{0.6cm}

	    \input{sets/de/cornucopia.tex}
	    % Basic settings for this card set
\renewcommand{\cardcolor}{cornucopia}
\renewcommand{\cardextension}{Erweiterung IV}
\renewcommand{\cardextensiontitle}{Reiche Ernte}
\renewcommand{\seticon}{cornucopia.png}

\clearpage
\newpage
\section{\cardextension \ - \cardextensiontitle \ (Rio Grande Games 2014)}

\begin{tikzpicture}
	\card
	\cardstrip
	\cardbanner{banner/white.png}
	\cardicon{icons/coin.png}
	\cardprice{2}
	\cardtitle{Weiler}
	\cardcontent{Du \emph{darfst} 1 oder 2 Karten ablegen. Wenn du 1 Karte ablegst, erhältst du  + 1 Aktion. Wenn du 2 Karten ablegst, erhältst du + 1 Aktion und + 1 Kauf. Wenn du keine Karte ablegst, erhältst du nichts.}
\end{tikzpicture}
\hspace{-0.6cm}
\begin{tikzpicture}
	\card
	\cardstrip
	\cardbanner{banner/white.png}
	\cardicon{icons/coin.png}
	\cardprice{3}
	\cardtitle{Menagerie}
	\cardcontent{Zeige deine Handkarten vor. Hast du nur Karten mit unterschiedlichen  Namen (z. B. eine \emph{MENAGERIE}, ein \emph{SILBER} und ein \emph{KUPFER}), ziehst du 3 Karten.  Hast du mindestens eine Karte doppelt auf der Hand (z. B. 2 \emph{KUPFER}), ziehst du eine Karte.}
\end{tikzpicture}
\hspace{-0.6cm}
\begin{tikzpicture}
	\card
	\cardstrip
	\cardbanner{banner/white.png}
	\cardicon{icons/coin.png}
	\cardprice{3}
	\cardtitle{Wahrsagerin}
	\cardcontent{Jeder Mitspieler, beginnend bei deinem linken Nachbarn, deckt solange Karten von seinem Nachziehstapel auf, bis er entweder eine Punkte- oder eine Fluchkarte aufgedeckt hat. Diese Karte muss er oben auf seinen Nachziehstapel legen. Ist der Nachziehstapel aufgebraucht, ohne dass eine entsprechende Karte aufgedeckt wurde, muss der Ablagestapel gemischt und weitere Karten aufgedeckt werden. Wird trotzdem keine Punkte- oder Fluchkarte aufgedeckt, legt der Spieler keine Karte auf den Nachziehstapel. Alle anderen aufgedeckten Karten werden abgelegt.}
\end{tikzpicture}
\hspace{-0.6cm}
\begin{tikzpicture}
	\card
	\cardstrip
	\cardbanner{banner/white.png}
	\cardicon{icons/coin.png}
	\cardprice{4}
	\cardtitle{Bauerndorf}
	\cardcontent{Decke solange Karten von deinem Nachziehstapel auf, bis du entweder eine Geldkarte oder eine Aktionskarte aufgedeckt hast. Nimm diese Karte auf die Hand und lege die anderen aufgedeckten Karten ab. Hast du in deinem Nachziehstapel (auch nach dem Mischen des Ablagestapels) keine Geld- oder Aktionskarte (oder entsprechende kombinierte Karte), nimmst du keine Karte auf die Hand.}
\end{tikzpicture}
\hspace{-0.6cm}
\begin{tikzpicture}
	\card
	\cardstrip
	\cardbanner{banner/white.png}
	\cardicon{icons/coin.png}
	\cardprice{4}
	\cardtitle{Junge Hexe}
	\cardcontent{Wenn die \emph{JUNGE HEXE} als Königreichkarte für das Spiel ausgewählt wurde, benötigt ihr als 11. Stapel im Vorrat einen Bannstapel (Spielvorbereitung, S. 3).

	\medskip

	Wenn du die \emph{JUNGE HEXE} ausspielst, ziehst du zuerst 2 Karten und legst dann 2 Handkarten ab. Jeder Mitspieler, der keine Bannkarte von seiner Hand aufdeckt, muss sich, beginnend bei deinem linken Nachbarn, einen \emph{FLUCH} vom Vorrat nehmen. Wird der Vorrat an \emph{FLÜCHEN} dabei aufgebraucht, müssen die Spieler, für die kein \emph{FLUCH} mehr vorhanden ist, keinen \emph{FLUCH} nehmen. 

	\medskip

	Die Spieler dürfen auch Reaktionskarten ausspielen, bevor sie eine Bannkarte ausspielen. Sind die Bannkarten gleichzeitig Reaktionskarten, dürfen sie zuerst als Reaktionskarten und dann als Bannkarten aufgedeckt werden.}
\end{tikzpicture}
\hspace{-0.6cm}
\begin{tikzpicture}
	\card
	\cardstrip
	\cardbanner{banner/white.png}
	\cardicon{icons/coin.png}
	\cardprice{4}
	\cardtitle{Nachbau}
	\cardcontent{Entsorge eine Handkarte. Nimm dafür eine Karte vom Vorrat, die genau \coin[1] mehr kostet als die entsorgte Karte. Lege die neue Karte ab. Entsorge dann eine weitere Handkarte und nimm eine Karte, die genau \coin[1] mehr kostet. Lege auch diese Karte ab. Ist im Vorrat keine Karte, die genau \coin[1] mehr kostet, musst du die Handkarte trotzdem entsorgen, erhältst dafür aber nichts. Du kannst keine Karte vom Vorrat nehmen und diese gleich wieder entsorgen, da du sie ablegen musst.}
\end{tikzpicture}
\hspace{-0.6cm}
\begin{tikzpicture}
	\card
	\cardstrip
	\cardbanner{banner/blue.png}
	\cardicon{icons/coin.png}
	\cardprice{4}
	\cardtitle{\footnotesize{Pferdehändler}}
	\cardcontent{Wenn du diese Karte ausspielst, erhältst du + 1 Kauf und +\coin[3]. Dann legst du 2 Handkarten ab. Wenn du weniger als 2 Handkarten hast, legst du so viele Karten ab wie möglich.

	\medskip

	Wenn ein Mitspieler eine Angriffskarte ausspielt, darfst du diese Karte aus deiner Hand aufdecken. Dann legst du den \emph{PFERDEHÄNDLER} zur Seite und der Angriff wird ausgeführt. Zu Beginn deines nächsten Zuges ziehst du eine Karte nach und nimmst den (oder die) zur Seite gelegten \emph{PFERDEHÄNDLER} wieder auf die Hand.}
\end{tikzpicture}
\hspace{-0.6cm}
\begin{tikzpicture}
	\card
	\cardstrip
	\cardbanner{banner/white.png}
	\cardicon{icons/coin.png}
	\cardprice{4}
	\cardtitle{Turnier}
	\cardcontent{Wenn du eine \emph{PROVINZ} aus deiner Hand ablegst, nimmst du dir entweder ein \emph{HERZOGTUM} vom Vorrat oder eine beliebige Karte vom Preisstapel. Ist der Stapel, für den du dich entscheidest, leer, nimmst du keine Karte. Lege die Karte, die du nimmst, oben auf deinen Nachziehstapel.

	\medskip

	Dann dürfen alle Mitspieler eine \emph{PROVINZ} aus ihrer Hand aufdecken. Wenn keiner eine \emph{PROVINZ} aufdeckt, erhältst du + 1 Karte und +\coin[1].}
\end{tikzpicture}
\hspace{-0.6cm}
\begin{tikzpicture}
	\card
	\cardstrip
	\cardbanner{banner/white.png}
	\cardicon{icons/coin.png}
	\cardprice{5}
	\cardtitle{Ernte}
	\cardcontent{Decke die obersten 4 Karten von deinem Nachziehstapel auf. Für jede aufgedeckte Karte mit unterschiedlichem Namen, erhältst du +\coin[1]. Kannst du (auch nach dem  Mischen des Ablagestapels) weniger als 4 Karten aufdecken, deckst du nur so viele Karten auf, wie möglich.}
\end{tikzpicture}
\hspace{-0.6cm}
\begin{tikzpicture}
	\card
	\cardstrip
	\cardbanner{banner/gold.png}
	\cardicon{icons/coin.png}
	\cardprice{5}
	\cardtitle{Füllhorn}
	\cardcontent{Diese Karte ist eine Geldkarte mit dem Basiswert 0. Wenn du das \emph{FÜLLHORN} ausspielst, zählst du zunächst, wie viele Karten \emph{mit unterschiedlichen Namen} im Spiel sind. Im Spiel sind: Geld- und Aktionskarten, die du in diesem Zug ausgespielt hast sowie evtl. bei dir ausliegende Dauerkarten (aus \emph{Seaside}). Nicht im Spiel dagegen sind Karten, die du in diesem Zug entsorgt hast. Nimm dir eine Karte aus dem Vorrat, die maximal soviel kostet, wie du Karten mit unterschiedlichen Namen im Spiel hast.

	\medskip

	Nimmst du eine Punktekarte, entsorgst du das \emph{FÜLLHORN}. Nimmst oder erhältst du eine Punktekarte auf andere Weise, entsorgst du das \emph{FÜLLHORN} nicht.}
\end{tikzpicture}
\hspace{-0.6cm}
\begin{tikzpicture}
	\card
	\cardstrip
	\cardbanner{banner/white.png}
	\cardicon{icons/coin.png}
	\cardprice{5}
	\cardtitle{Harlekin}
	\cardcontent{Jeder Mitspieler, beginnend mit deinem linken Nachbarn, deckt die oberste Karte seines Nachziehstapels auf und legt sie ab.

	\medskip

	Ist es eine Punktekarte (auch eine kombinierte), muss sich der Spieler einen \emph{FLUCH} nehmen. Wird der Vorrat an \emph{FLÜCHEN} dabei aufgebraucht, erhalten die Spieler, für die kein \emph{FLUCH} mehr vorhanden ist, keinen \emph{FLUCH}.

	\medskip

	Ist es keine Punktekarte, darfst du wählen: Entweder muss sich der Spieler eine Karte mit gleichem Namen (sofern vorhanden) aus dem Vorrat nehmen und ablegen oder du nimmst eine Karte mit gleichem Namen (sofern vorhanden) aus dem Vorrat und legst sie ab.}
\end{tikzpicture}
\hspace{-0.6cm}
\begin{tikzpicture}
	\card
	\cardstrip
	\cardbanner{banner/white.png}
	\cardicon{icons/coin.png}
	\cardprice{5}
	\cardtitle{Treibjagd}
	\cardcontent{Decke alle deine Handkarten auf. Decke dann so lange Karten vom Nachziehstapel auf, bis du die erste Karte aufdeckst, die einen Namen hat, der nicht bei deinen Handkarten dabei ist. Nimm diese Karte und die aufgedeckten Handkarten auf die Hand und lege die anderen aufgedeckten Karten ab. Kannst du (auch nach dem Mischen des Ablagestapels) eine solche Karte nicht aufdecken, nimmst du nur deine aufgedeckten Handkarten wieder auf die Hand.}
\end{tikzpicture}
\hspace{-0.6cm}
\begin{tikzpicture}
	\card
	\cardstrip
	\cardbanner{banner/green.png}
	\cardicon{icons/coin.png}
	\cardprice{6}
	\cardtitle{Festplatz}
	\cardcontent{Diese Karte ist eine Punktekarte und hat bis zum Ende des Spiels keine Funktion. Bei der Wertung des Spiels erhältst du pro 5 Karten mit unterschiedlichen Namen in deinem Kartensatz (Handkarten, Nachzieh- und Ablagestapel) 2 Siegpunkte. Ein Kartensatz mit bis zu 4 unterschiedlichen Kartennamen bringt dir zum Beispiel keine Punkte, Kartensätze mit 10 bis 14 unterschiedlichen Namen dagegen 4 \victorypoint.}
\end{tikzpicture}
\hspace{-0.6cm}
\begin{tikzpicture}
	\card
	\cardstrip
	\cardbanner{banner/white.png}
	\cardicon{icons/coin.png}
	\cardprice{0*}
	\cardtitle{Preiskarten}
	\cardcontent{\tiny{\begin{Spacing}{1}
	\vspace{1em}
	\emph{Diadem:} Das \emph{DIADEM} ist eine Geldkarte. Für jede in deinem Zug nicht verbrauchte Aktion erhältst du +\coin[1] für die Kaufphase. Hast du zum Beispiel in der Aktionsphase gar keine  Aktionskarte ausgespielt, erhältst du +\coin[1] für deine freie Aktion. Hast du zum Beispiel nur das \emph{BAUERNDORF} ausgespielt, erhältst du +\coin[2] für die beiden Aktionen des \emph{BAUERNDORFS}, da du deine freie Aktion verbraucht hast.

	\smallskip

	\emph{Ein Sack voll Gold:} Ist der Nachziehstapel leer, wenn du dir ein \emph{GOLD} nimmst, legst du das \emph{GOLD} auf die leere Stelle. Es ist dann die einzige Karte in deinem Nachziehstapel.

	\smallskip

	\emph{Gefolge:} Ist kein \emph{ANWESEN} mehr im Vorrat, erhältst du keins. Beginnend mit deinem linken Nachbarn nimmt sich jeder Mitspieler einen \emph{FLUCH}. Mitspieler, die mehr als  3 Karten auf der Hand haben, legen außerdem Karten ab, bis sie nur noch 3 Karten auf der Hand haben. Ist für einen Spieler kein \emph{FLUCH} mehr im Vorrat, erhält der Spieler keinen. Handkarten muss er dennoch ablegen, wenn er mehr als 3 Karten hat.

	\smallskip

	\emph{Prinzessin:} Die Anweisung, dass jede Karte \coin[2] weniger kostet, wenn die \emph{PRINZESSIN} im Spiel ist, betrifft die Karten auf der Hand und in den Ablage- und Nachziehstapeln aller Spieler sowie alle Karten im Vorrat. Wird die \emph{PRINZESSIN} auf einen \emph{THRONSAAL} folgend ausgespielt, kosten alle Karten trotzdem nur \coin[2] weniger, da die \emph{PRINZESSIN} nur einmal im Spiel ist.

	\smallskip

	\emph{Streitross:} Diese Karte beinhaltet 4 Anweisungen, von denen du 2 unterschiedliche wählst. Dann führst du die Anweisungen der Reihenfolge auf der Karte nach aus. Du darfst auch einen Anweisung wählen, die du nicht (oder nur teilweise) erfüllen kannst, z. B. wenn nur noch 3 \emph{SILBER} im Vorrat sind. Du darfst deinen Nachziehstapel nicht durchsehen, bevor du ihn ablegst.
	\end{Spacing}}}
\end{tikzpicture}
\hspace{-0.6cm}
\begin{tikzpicture}
	\card
	\cardstrip
	\cardbanner{banner/white.png}
	\cardtitle{\scriptsize{Empfohlene 10er Sätze\qquad}}
	\cardcontent{\emph{Kopfgeld} (Reiche Ernte + \textit{Basisspiel}):\\
	Ernte, Füllhorn, Menagerie, Treibjagd, Turnier (+ Preiskarten), \textit{Geldverleiher}, \textit{Jahrmarkt}, \textit{Keller}, \textit{Miliz}, \textit{Schmiede}

	\smallskip

	\emph{Böses Omen} (Reiche Ernte + \textit{Basisspiel}):\\
	Füllhorn, Harlekin, Nachbau, Wahrsagerin, Weiler, \textit{Abenteurer}, \textit{Bürokrat}, \textit{Laboratorium}, \textit{Spion}, \textit{Thronsaal}

	\smallskip

	\emph{Wanderzirkus} (Reiche Ernte + \textit{Basisspiel}):\\
	Bauerndorf, Festplatz, Harlekin, Junge Hexe (+ Kanzler als Bannstapel), Pferdehändler, \textit{Festmahl}, \textit{Laboratorium}, \textit{Markt}, \textit{Umbau}, \textit{Werkstatt}

	\emph{Wer zuletzt lacht} (Reiche Ernte + \textit{Die Intrige}):\\
	Bauerndorf, Ernte, Harlekin, Pferdehändler, Treibjagd, \textit{Adlige}, \textit{Handlanger}, \textit{Lakai}, \textit{Trickser}, \textit{Verwwalter}

	\smallskip

	\emph{Würze des Lebens}	(Reiche Ernte + \textit{Die Intrige}):\\
	Festplatz, Füllhorn, Junge Hexe (+ Wunschbrunnen als Bannstapel), Nachbau, Turnier (+ Preiskarten), \textit{Bergwerk}, \textit{Burghof}, \textit{Große Halle}, \textit{Kupferschmied}, \textit{Tribut} 

	\smallskip

	\emph{Kleine  Siege} (Reiche Ernte + \textit{Die Intrige}):\\
	Nachbau, Treibjagd, Turnier (+ Preiskarten), Wahrsagerin, Weiler, \textit{Große Halle}, \textit{Handlanger}, \textit{Harem}, \textit{Herzog}, \textit{Verschwörer}}
\end{tikzpicture}
\hspace{0.6cm}

	    \input{sets/de/hinterlands.tex}
	    \input{sets/de/darkages.tex}
	    % Basic settings for this card set
\renewcommand{\cardcolor}{guilds}
\renewcommand{\cardextension}{Erweiterung VII}
\renewcommand{\cardextensiontitle}{Die Gilden}

\clearpage
\newpage
\section{\cardextension \ - \cardextensiontitle}

\begin{tikzpicture}
	\card
	\cardstrip
	\cardbanner{banner/white.png}
	\cardicon{banner/coin.png}
	\cardprice{3+}
	\cardtitle{Arzt}
	\cardcontent{Wenn du diese Karte kaufst, darfst du mehr dafür zahlen. Pro \coin{1}, das du zusätzlich zahlst (d. h. „überzahlst“), darfst du dir die oberste Karte deines Nachziehstapels ansehen und sie entsorgen, ablegen oder auf den Nachziehstapel zurücklegen. Wenn dein Nachziehstapel aufgebraucht ist, mischst du deinen Ablagestapel und machst ihn zum neuen Nachziehstapel. Wenn dann noch immer nicht genügend Karten im Nachziehstapel sind, siehst du dir keine Karten an. Auch wenn du mehr als \coin{1} überzahlst, wird jede oberste Karte vom Nachziehstapel einzeln komplett abgehandelt, bevor du die nächste Karte ziehst.\\ \smallskip
	Wenn du diese Karte in der Aktionsphase spielst, nennst du den Namen einer beliebigen Karte, deckst die obersten drei Karten deines Nachziehstapels auf und entsorgst jede Karte mit dem von dir genannten Namen. Die übrigen Karten legst du in beliebiger Reihenfolge wieder oben auf deinen Nachziehstapel. Du musst keine Karte nennen, die in diesem Spiel verwendet wird. Wenn dein Nachziehstapel keine drei Karten mehr umfasst, deckst du die darin noch enthaltenen Karten auf, mischst dann deinen Ablagestapel (der die bereits aufgedeckten Karten nicht umfasst) und machst ihn zum neuen Nachziehstapel, von dem du die noch fehlenden Karten aufdeckst. Sind dann noch immer nicht genügend Karten aufgedeckt, belässt du es bei den bereits aufgedeckten Karten.}
\end{tikzpicture}
\hspace{-1cm}
\begin{tikzpicture}
	\card
	\cardstrip
	\cardbanner{banner/white.png}
	\cardicon{banner/coin.png}
	\cardprice{5}
	\cardtitle{Bäcker}
	\cardcontent{Wenn du diese Karte spielst, ziehst du eine Karte, darfst eine weitere Aktionskarte ausspielen und nimmst dir eine Münze.\\ \smallskip
	Wird ein Spiel mit dieser Karte gespielt, erhält jeder Spieler zu Beginn des Spiels eine Münze. Das gilt auch für Partien mit dem Schwarzmarkt , bei denen der Bäcker sich im Schwarzmarkt-Stapel befindet.}
\end{tikzpicture}
\hspace{-1cm}
\begin{tikzpicture}
	\card
	\cardstrip
	\cardbanner{banner/white.png}
	\cardicon{banner/coin.png}
	\cardprice{4}
	\cardtitle{Berater}
	\cardcontent{Wenn dein Nachziehstapel keine drei Karten mehr umfasst, deckst du die darin noch enthaltenen Karten auf, mischst dann deinen Ablagestapel und machst ihn zum neuen Nachziehstapel, von dem du die noch fehlenden Karten aufdeckst. Sind dann noch immer nicht genügend Karten aufgedeckt, belässt du es bei den bereits aufgedeckten Karten. Unabhängig davon, wie viele Karten du aufgedeckt hast, wählt dein linker Nachbar eine davon aus, die du ablegst. Die verbleibenden Karten nimmst du auf die Hand.}
\end{tikzpicture}
\hspace{-1cm}
\begin{tikzpicture}
	\card
	\cardstrip
	\cardbanner{banner/white.png}
	\cardicon{banner/coin.png}
	\cardprice{4+}
	\cardtitle{Herold}
	\cardcontent{\tiny{Wenn du diese Karte kaufst, darfst du mehr dafür zahlen. Wenn du das tust legst du pro 1 Geld, das du überzahlst, eine beliebige Karte deines Ablagestapels oben auf deinen Nachziehstapel. Du darfst dir dafür die Karten deines Ablagestapels ansehen, was normalerweise nicht möglich ist. Allerdings darfst du nicht zuerst deinen Ablagestapel durchsehen, um zu entscheiden, wie viel du überzahlen willst. Sobald du überzahlt hast, musst du die entsprechende Anzahl an Karten in beliebiger Reihenfolge oben auf deinen Nachziehstapel legen, sofern möglich. Wenn du so viel überzahlst, dass du mehr Karten auf deinen Nachziehstapel legen müsstest, als sich in deinem Ablagestapel befinden, legst du einfach alle Karten deines Ablagestapels in beliebiger Reihenfolge auf deinen Nachziehstapel. Falls du den Herold kaufst, ohne zu überzahlen, darfst du deinen Ablagestapel nicht durchsehen.\\ \smallskip
	Wenn du diese Karte in der Aktionsphase spielst, ziehst du zuerst eine Karte. Du erhältst dann eine weitere Aktion, die du spielen darfst, nachdem du alle Anweisungen dieser Karte (soweit möglich) erfüllt hast. Dann deckst du die oberste Karte deines Nachziehstapels auf. Wenn es sich um eine Aktionskarte handelt, musst du sie spielen. Die Karte zu spielen, verbraucht keine Aktion. Karten, die mehreren Kartentypen angehören, von denen einer Aktion ist (z. B. Große Halle aus Dominion – Die Intrige), sind Aktionskarten. Alle anderen Kartentypen werden auf den Nachziehstapel zurückgelegt, ohne sie zu spielen.\\ \smallskip
	\emph{Hinweis:} Wenn durch den Herold eine Dauer-Karte (aus Dominion – Seaside) ausgespielt wird, wird der Herold dennoch am Ende der Runde wie gewohnt abgelegt, da er nicht gebraucht wird, um an etwas zu erinnern.\\}}
\end{tikzpicture}
\hspace{-1cm}
\begin{tikzpicture}
	\card
	\cardstrip
	\cardbanner{banner/white.png}
	\cardicon{banner/coin.png}
	\cardprice{5}
	\cardtitle{\footnotesize{Kaufmannsgilde}}
	\cardcontent{Wenn diese Karte im Spiel ist (d. h. du hast sie in der Aktionsphase gespielt), darfst du eine weitere Karte kaufen und hast \coin{1} zusätzlich zur Verfügung. Jedes Mal, wenn du eine Karte kaufst, nimmst du dir eine Münze (1 Münze, wenn du eine Karte kaufst; 2 Münzen, wenn du zwei Karten kaufst etc.). Denke daran, dass du Münzen nur einsetzen kannst, bevor du Karten kaufst: Du darfst also diese Münze nicht sofort aufwenden. Diese Anweisung ist kumulativ: Wenn du zwei Kaufmannsgilden ausgespielt hast, bringt dir jede Karte, die du kaufst, zwei Münzen ein. Wenn du jedoch eine Kaufmannsgilde mehrfach spielst, aber nur eine im Spiel hast – wie beim Thronsaal (Dominion – Basisspiel) oder Königshof (Dominion – Blütezeit) –, bekommst du beim Kauf einer Karte nur eine Münze.}
\end{tikzpicture}
\hspace{-1cm}
\begin{tikzpicture}
	\card
	\cardstrip
	\cardbanner{banner/white.png}
	\cardicon{banner/coin.png}
	\cardprice{2}
	\cardtitle{\footnotesize{Leuchtenmacher}}
	\cardcontent{Du darfst in der Aktionsphase eine weitere Aktionskarte ausspielen. Du darfst in der Kaufphase einen zusätzlichen Kauf tätigen. Nimm dir eine Münze.}
\end{tikzpicture}
\hspace{-1cm}
\begin{tikzpicture}
	\card
	\cardstrip
	\cardbanner{banner/gold.png}
	\cardicon{banner/coin.png}
	\cardprice{3+}
	\cardtitle{Meisterstück}
	\cardcontent{Dies ist eine Geldkarte mit dem Wert \coin{1}, wie Kupfer. Wenn du sie kaufst, nimmst du dir ein Silber pro \coin{1}, das du überzahlst. Falls du zum Beispiel \coin{6} für das Meisterstück zahlst, erhältst du drei Silber. Das Meisterstück ist eine Geldkarte und wird regeltechnisch auch so behandelt.}
\end{tikzpicture}
\hspace{-1cm}
\begin{tikzpicture}
	\card
	\cardstrip
	\cardbanner{banner/white.png}
	\cardicon{banner/coin.png}
	\cardprice{5}
	\cardtitle{Metzger}
	\cardcontent{Zuerst nimmst du dir 2 Münzen. Dann darfst du eine Karte aus deiner Hand entsorgen und eine beliebige Anzahl an Münzen einsetzen (auch 0 Münzen). Da du den Metzger nicht mehr auf der Hand hast, kannst du diese Karte nicht entsorgen. Allerdings kannst du eine andere Metzger-Karte entsorgen. Wenn du eine Karte entsorgt hast, nimmst du dir eine Karte, deren Kosten höchstens der Summe aus den Kosten der entsorgten Karte und der Anzahl der eingesetzten Münzen entsprechen darf. So könntest du beispielsweise ein Anwesen entsorgen und sechs Münzen zahlen, um dir eine Provinz zu nehmen; oder du könntest einen weiteren Metzger entsorgen und null Münzen zahlen, um dir ein Herzogtum zu nehmen. Die Münzen, die du einsetzt, werden in den Vorrat zurückgelegt und dürfen in der Kaufphase nicht mehr benutzt werden, um andere Karten zu kaufen.}
\end{tikzpicture}
\hspace{-1cm}
\begin{tikzpicture}
	\card
	\cardstrip
	\cardbanner{banner/white.png}
	\cardicon{banner/coin.png}
	\cardprice{4}
	\cardtitle{Platz}
	\cardcontent{Zuerst ziehst du eine Karte. Du erhältst zwei weitere Aktionen, die du spielen darfst, nachdem du alle Anweisungen dieser Karte (soweit möglich) erfüllt hast. Dann darfst du eine Geldkarte (auch ein Meisterstück) ablegen. Du kannst die Karte ablegen, die du eben gezogen hast, falls es sich um eine Geldkarte handelt. Wenn du eine Geldkarte abgelegt hast, nimmst du dir eine Münze. Karten, die mehreren Kartentypen angehören, von denen einer Geld ist (wie der Harem aus Dominion – Die Intrige), sind Geldkarten.}
\end{tikzpicture}
\hspace{-1cm}
\begin{tikzpicture}
	\card
	\cardstrip
	\cardbanner{banner/white.png}
	\cardicon{banner/coin.png}
	\cardprice{2+}
	\cardtitle{Steinmetz}
	\cardcontent{\tiny{Wenn du die Karte Steinmetz kaufst, darfst du mehr dafür bezahlen. Wenn du das tust, nimmst du dir zwei Aktionskarten, die beide jeweils genau so viel kosten, wie du überzahlt hast. Du kannst dir zweimal die gleiche oder zwei verschiedene Karten nehmen. Wenn du zum Beispiel den Steinmetz für \coin{6} kaufst, könntest du dir zwei Herolde nehmen. Die Aktionskarten stammen aus dem Vorrat und werden auf deinen Ablagestapel gelegt. Sollten keine Karten mit den entsprechenden Kosten im Vorrat sein, bekommst du keine. Wenn du mit einem Trank (Dominion – Die Alchemisten) überzahlst, bekommst du Karten im Wert des Tranks . Karten, die mehreren Kartentypen angehören, von denen einer Aktion ist (wie die Große Halle aus Dominion – Die Intrige), sind Aktionskarten. Wenn du beschließt, nicht mehr als die normalen Kosten zu zahlen, bekommst du keine Karten. Es ist nicht möglich, dann Aktionskarten zu nehmen, die 0 Geld kosten. \\ \smallskip
	Wenn du diese Karte spielst, entsorgst du eine Karte aus deiner Hand und nimmst dir zwei Karten, die jeweils weniger kosten als die Karte, die du entsorgt hast. Wenn du keine Karte zum Entsorgen mehr auf der Hand hast, bekommst du keine Karten. Du kannst dir zweimal die gleiche oder zwei verschiedene Karten aus dem Vorrat nehmen. Diese werden auf deinen Ablagestapel gelegt. Falls es keine preiswerteren Karten im Vorrat gibt (weil du beispielsweise ein Kupfer entsorgst), bekommst du keine Karten. Wenn es im Vorrat nur noch eine einzige Karte gibt, die preiswerter ist als die entsorgte Karte, nimmst du dir diese. Du nimmst dir die Karten einzeln nacheinander; das kann bei Karten eine Rolle spielen, die beim Nehmen eine Auswirkung haben (wie das Gasthaus aus Dominion – Hinterland).\\}}
\end{tikzpicture}
\hspace{-1cm}
\begin{tikzpicture}
	\card
	\cardstrip
	\cardbanner{banner/white.png}
	\cardicon{banner/coin.png}
	\cardprice{4}
	\cardtitle{\scriptsize{Steuereintreiber}}
	\cardcontent{Du darfst eine Geldkarte aus deiner Hand entsorgen. Karten, die mehreren Kartentypen angehören, von denen einer Geld ist (wie der Harem aus Dominion – Die Intrige), sind Geldkarten. Wenn du eine Geldkarte entsorgst, muss jeder Mitspieler, der mindestens fünf Karten in der Hand hat, die gleiche Karte aus seiner Hand auf seinen Ablagestapel legen oder seine Handkarten vorzeigen, damit die anderen Spieler sehen, dass er diese Karte nicht auf der Hand hat. Nimm dir für die entsorgte Karte eine Geldkarte, die bis zu \coin{3} mehr kostet als die entsorgte Geldkarte, und lege sie oben auf deinen Nachziehstapel. Wenn dein Nachziehstapel aufgebraucht ist, wird dies die einzige Karte deines Nachziehstapels. Du musst dir keine teurere Geldkarte nehmen, sondern darfst dir auch eine gleich teure oder preiswertere Geldkarte nehmen.}
\end{tikzpicture}
\hspace{-1cm}
\begin{tikzpicture}
	\card
	\cardstrip
	\cardbanner{banner/white.png}
	\cardicon{banner/coin.png}
	\cardprice{5}
	\cardtitle{Wahrsager}
	\cardcontent{Das Gold und die Fluchkarten stammen aus dem Vorrat und werden auf den Ablagestapel gelegt. Wenn kein Gold mehr vorhanden ist, erhältst du keins. Sind nicht mehr genügend Fluchkarten da, teilst du sie reihum aus, beginnend mit deinem linken Nachbarn. Jeder Spieler, der eine Fluchkarte erhalten hat, muss eine Karte vom Nachziehstapel ziehen. Falls ein Spieler keine Fluchkarte bekommen hat – weil nicht genügend Fluchkarten da waren oder aus einem anderen Grund –, zieht er keine Karte. Nutzt ein Spieler den Wachturm (Dominion – Blütezeit), um den Fluch zu entsorgen, hat er dennoch eine Fluchkarte erhalten und zieht deshalb eine Karte. Nutzt ein Spieler den Fahrenden Händler (Dominion – Hinterland), um stattdessen ein Silber zu nehmen, hat er keine Fluchkarte erhalten und zieht infolgedessen auch keine Karte.}
\end{tikzpicture}
\hspace{-1cm}
\begin{tikzpicture}
	\card
	\cardstrip
	\cardbanner{banner/white.png}
	\cardicon{banner/coin.png}
	\cardprice{5}
	\cardtitle{\footnotesize{Wandergeselle}}
	\cardcontent{Zunächst nennst du eine Karte. Dabei muss es sich nicht um den Namen einer Karte handeln, die in diesem Spiel verwendet wird. Dann deckst du solange Karten vom Nachziehstapel auf, bis du drei Karten aufgedeckt hast, die nicht dem von dir genannten Namen entsprechen. Nimm diese Karten auf die Hand und lege die anderen ab. Sollte dabei der Nachziehstapel aufgebraucht werden, bevor du auf drei entsprechende Karten gestoßen bist, mischst du deinen Ablagestapel und nutzt ihn als neuen Nachziehstapel. Wenn du beim weiteren Aufdecken noch immer nicht auf drei Karten mit einem anderen als dem genannten Namen kommst, beendest du das Aufdecken. Nimm die gefundenen Karten anderen Namens auf die Hand und lege den Rest auf deinen Ablagestapel.}
\end{tikzpicture}
\hspace{-1cm}
\begin{tikzpicture}
	\card
	\cardstrip
	\cardbanner{banner/white.png}
	\cardicon{}
	\cardprice{}
	\cardtitle{\scriptsize{Empfohlene 10er Sätze\qquad}}
	\cardcontent{\emph{Die Gilden und Basisspiel:}\\ \smallskip
	\emph{Kunsthandwerk:} \\ 
	Steinmetz, Berater, Bäcker, Wandergeselle, Kaufmannsgilde / Laboratorium, Keller, Werkstatt, Jahrmarkt, Geldverleiher \\
	\smallskip \emph{Rechtschaffen und anständig:} \\ 
	Metzger, Bäcker, Leuchtenmacher, Arzt, Wahrsager / Miliz, Dieb, Geldverleiher, Gärten, Dorf \\
	\smallskip \emph{Des Guten zuviel:} \\ 
	Platz, Meisterstück, Leuchtenmacher, Steuereintreiber, Herold / Bibliothek, Umbau, Abenteurer, Markt, Kanzler 
}
\end{tikzpicture}
\hspace{-1cm}
\begin{tikzpicture}
	\card
	\cardstrip
	\cardbanner{banner/white.png}
	\cardicon{}
	\cardprice{}
	\cardtitle{\scriptsize{Empfohlene 10er Sätze\qquad}}
	\cardcontent{\emph{Die Gilden und Die Intrigen:}\\ \smallskip
	\emph{Nenne diese Karte:} \\ 
	Bäcker, Arzt, Platz, Berater, Meisterstück / Burghof, Wunschbrunnen, Harem, Tribut, Adelige \\
	\smallskip \emph{Geschäftstricks:} \\ 
	Steinmetz, Herold, Wahrsager, Wandergeselle, Metzger / Große Halle, Adelige, Verschwörer, Maskerade, Kupferschmied \\
	\smallskip \emph{Entscheidungen, Entscheidungen:} \\ 
	Kaufmannsgilde, Leuchtenmacher, Meisterstück, Steuereintreiber, Metzger / Brücke, Handlanger, Bergwerk, Anbau, Herzog
}
\end{tikzpicture}
\hspace{-1cm}
\begin{tikzpicture}
	\card
	\cardstrip
	\cardbanner{banner/white.png}
	\cardtitle{\scriptsize{Neue Regeln (1/2)}\qquad}
	\cardcontent{\tiny{\emph{Die Münzen:}\\
	Einige Karten in Die Gilden erlauben den Spielern, sich Münzen zu nehmen. Münzen werden immer vom Vorrat genommen, nicht von anderen Spielern. Diese Münzen können – anders als die Geldkarten – über den Zug, in dem ein Spieler sie erhält, hinaus aufbewahrt werden. In der Kaufphase eines Spielers kann dieser, bevor er Karten kauft, eine beliebige Anzahl an Münzen einsetzen; jede eingesetzte Münze bringt dem Spieler +1 Geld. Eingesetzte Münzen werden in den Vorrat zurückgelegt. Die Anzahl der Münzen im Vorrat ist nicht begrenzt: Sollten einmal nicht ausreichend Münzen vorhanden sein, verwenden die Spieler einen beliebigen Ersatz. Zwar handelt es sich um die gleichen Marker / Münzen wie in Seaside und Blütezeit, allerdings können die durch Anweisungen auf den Karten aus Die Gilden erworbenen Münzen ausschließlich in der Kaufphase ausgegeben werden. Sie dürfen weder zum Kauf einer Karte mithilfe der Karte Schwarzmarkt benutzt werden, noch dürfen sie auf ein Piratenschiff-Tableau (Dominion – Seaside) oder die Handelsroute (Dominion – Blütezeit) gelegt werden.\\}}
\end{tikzpicture}
\hspace{-1cm}
\begin{tikzpicture}
	\card
	\cardstrip
	\cardbanner{banner/white.png}
	\cardtitle{\scriptsize{Neue Regeln (2/2)}\qquad}
	\cardcontent{\tiny{\emph{Überzahlen:}\\
	Für einige Karten in Die Gilden darf man mehr zahlen, als sie eigentlich kosten. Bei diesen Karten steht hinter den Kosten ein „+“, z. B. 2+ Geld. Wenn ein Spieler vor dem Kauf einen beliebigen zusätzlichen Betrag für eine solche Karte zahlt, tritt ein auf der jeweiligen Karte genannter Effekt ein, der von der Höhe der Überzahlung abhängt. Überzahlt werden kann mit allen Geldwerten, mit denen Karten gekauft werden: Geldkarten, Münzen und Geldwerte auf Aktionskarten. Die Karten Trank (Dominion – Die Alchemisten) können ebenfalls zum Überzahlen verwendet werden (allerdings ist das nicht immer sinnvoll). Das zum Überzahlen verwendete Geld ist ausgegeben und kann dann nicht mehr eingesetzt werden. Geldkarten werden abgelegt und Münzen kommen zurück in den Vorrat. Spieler können eine Karte ausschließlich beim Kauf überzahlen, nicht wenn sie diese auf eine andere Weise erhalten. Das „+“ ist lediglich eine Erinnerung: Für eine Karte, die hinter den Kosten das Symbol „+“ aufweist, gelten nach wie vor für alle Zwecke die normalen Kosten. Grundsätzlich gibt es keine Wechselwirkungen zwischen Karten, welche die Kartenkosten reduzieren – wie die Brücke (Dominion – Die Intrige) oder die Fernstraße (Dominion – Hinterland) – und dem „Überzahlen“. \\ \smallskip
	\emph{Zwei Dinge passieren zur gleichen Zeit:} \\
	Wenn einem Spieler zwei Dinge gleichzeitig passieren, entscheidet er selbst, in welcher Reihenfolge sie eintreten. Wenn unterschiedlichen Spielern zwei Dinge gleichzeitig passieren, werden diese reihum in Spielreihenfolge abgehandelt, beginnend mit dem Spieler, der gerade an der Reihe ist.\\}}
\end{tikzpicture}
\hspace{1cm}
	    % Basic settings for this card set
\renewcommand{\cardcolor}{guilds}
\renewcommand{\cardextension}{Erweiterung VII}
\renewcommand{\cardextensiontitle}{Die Gilden}
\renewcommand{\seticon}{guilds.png}

\clearpage
\newpage
\section{\cardextension \ - \cardextensiontitle \ (Rio Grande Games 2013)}

\begin{tikzpicture}
	\card
	\cardstrip
	\cardbanner{banner/white.png}
	\cardicon{icons/coin.png}
	\cardprice{2}
	\cardtitle{\scriptsize{Leuchtenmacher}}
	\cardcontent{Du darfst in der Aktionsphase eine weitere Aktionskarte ausspielen. Du darfst in der Kaufphase einen zusätzlichen Kauf tätigen. Nimm dir eine Münze.}
\end{tikzpicture}
\hspace{-0.6cm}
\begin{tikzpicture}
	\card
	\cardstrip
	\cardbanner{banner/white.png}
	\cardicon{icons/coin.png}
	\cardprice{2+}
	\cardtitle{Steinmetz}
	\cardcontent{\tiny{\begin{Spacing}{1}
	\vspace{1em}
	Wenn du die Karte \emph{STEINMETZ} kaufst, darfst du mehr dafür bezahlen. Wenn du das tust, nimmst du dir zwei Aktionskarten, die beide jeweils genau so viel kosten, wie du überzahlt hast. Du kannst dir zweimal die gleiche oder zwei verschiedene Karten nehmen. Wenn du zum Beispiel den \emph{STEINMETZ} für \coin[6] kaufst, könntest du dir zwei \emph{HEROLDE} nehmen. Die Aktionskarten stammen aus dem Vorrat und werden auf deinen Ablagestapel gelegt. Sollten keine Karten mit den entsprechenden Kosten im Vorrat sein, bekommst du keine. Wenn du mit einem \emph{TRANK} (\emph{Die Alchemisten}) überzahlst, bekommst du Karten im Wert des \emph{TRANKS}. Karten, die mehreren Kartentypen angehören, von denen einer AKTION ist (wie die \emph{GROSSE HALLE} aus \emph{Die Intrige}), sind Aktionskarten. Wenn du beschließt, nicht mehr als die normalen Kosten zu zahlen, bekommst du keine Karten. Es ist nicht möglich, dann Aktionskarten zu nehmen, die \coin[0]	kosten.

	\medskip

	Wenn du diese Karte spielst, entsorgst du eine Karte aus deiner Hand und nimmst dir zwei Karten, die jeweils weniger kosten als die Karte, die du entsorgt hast. Wenn du keine Karte zum Entsorgen mehr auf der Hand hast, bekommst du keine Karten. Du kannst dir zweimal die gleiche oder zwei verschiedene Karten aus dem Vorrat nehmen. Diese werden auf deinen Ablagestapel gelegt. Falls es keine preiswerteren Karten im Vorrat gibt (weil du beispielsweise ein Kupfer entsorgst), bekommst du keine Karten. Wenn es im Vorrat nur noch eine einzige Karte gibt, die preiswerter ist als die entsorgte Karte, nimmst du dir diese. Du nimmst dir die Karten einzeln nacheinander; das kann bei Karten eine Rolle spielen, die beim Nehmen eine Auswirkung haben (wie das \emph{GASTHAUS} aus \emph{Hinterland}).
	\end{Spacing}}}
\end{tikzpicture}
\hspace{-0.6cm}
\begin{tikzpicture}
	\card
	\cardstrip
	\cardbanner{banner/white.png}
	\cardicon{icons/coin.png}
	\cardprice{3+}
	\cardtitle{Arzt}
	\cardcontent{\tiny{\begin{Spacing}{1}
	\vspace{1em}
	Wenn du diese Karte kaufst, darfst du mehr dafür zahlen. Pro \coin[1], das du zusätzlich zahlst (d.h. \enquote{überzahlst}), darfst du dir die oberste Karte deines Nachziehstapels ansehen und sie entsorgen, ablegen oder auf den Nachziehstapel zurücklegen. Wenn dein Nachziehstapel aufgebraucht ist, mischst du deinen Ablagestapel und machst ihn zum neuen Nachziehstapel. Wenn dann noch immer nicht genügend Karten im Nach- ziehstapel sind, siehst du dir keine Karten an. Auch wenn du mehr als \coin[1] überzahlst, wird jede oberste Karte vom Nachziehstapel einzeln komplett abgehandelt, bevor du die nächste Karte ziehst.

	\medskip

	Wenn du diese Karte in der Aktionsphase spielst, nennst du den Namen einer beliebigen Karte, deckst die obersten drei Karten deines Nachziehstapels auf und entsorgst jede Karte mit dem von dir genannten Namen. Die übrigen Karten legst du in beliebiger Reihenfolge wieder oben auf deinen Nachziehstapel. Du musst keine Karte nennen, die in diesem Spiel verwendet wird. Wenn dein Nachziehstapel keine drei Karten mehr umfasst, deckst du die darin noch enthaltenen Karten auf, mischst dann deinen Ablagestapel (der die bereits aufgedeckten Karten nicht umfasst) und machst ihn zum neuen Nachziehstapel, von dem du die noch fehlenden Karten aufdeckst. Sind dann noch immer nicht genügend Karten aufgedeckt, belässt du es bei den bereits aufgedeckten Karten.
	\end{Spacing}}}
\end{tikzpicture}
\hspace{-0.6cm}
\begin{tikzpicture}
	\card
	\cardstrip
	\cardbanner{banner/white.png}
	\cardicon{icons/coin.png}
	\cardprice{4}
	\cardtitle{Berater}
	\cardcontent{Wenn dein Nachziehstapel keine drei Karten mehr umfasst, deckst du die darin noch enthaltenen Karten auf, mischst dann deinen Ablagestapel und machst ihn zum neuen Nachziehstapel, von dem du die noch fehlenden Karten aufdeckst. Sind dann noch immer nicht genügend Karten aufgedeckt, belässt du es bei den bereits aufgedeckten Karten. Unabhängig davon, wie viele Karten du aufgedeckt hast, wählt dein linker Nachbar eine davon aus, die du ablegst. Die verbleibenden Karten nimmst du auf die Hand.}
\end{tikzpicture}
\hspace{-0.6cm}
\begin{tikzpicture}
	\card
	\cardstrip
	\cardbanner{banner/white.png}
	\cardicon{icons/coin.png}
	\cardprice{4}
	\cardtitle{Platz}
	\cardcontent{Zuerst ziehst du eine Karte. Du erhältst zwei weitere Aktionen, die du spielen darfst, nachdem du alle Anweisungen dieser Karte (soweit möglich) erfüllt hast. Dann darfst du eine Geldkarte (auch ein \emph{MEISTERSTÜCK}) ablegen. Du kannst die Karte ablegen, die du eben gezogen hast, falls es sich um eine Geldkarte handelt. Wenn du eine Geldkarte abgelegt hast, nimmst du dir eine Münze. Karten, die mehreren Kartentypen angehören, von denen einer \emph{GELD} ist (wie der \emph{HAREM} aus Die Intrige), sind Geldkarten.}
\end{tikzpicture}
\hspace{-0.6cm}
\begin{tikzpicture}
	\card
	\cardstrip
	\cardbanner{banner/white.png}
	\cardicon{icons/coin.png}
	\cardprice{4}
	\cardtitle{\scriptsize{Steuereintreiber}}
	\cardcontent{Du darfst eine Geldkarte aus deiner Hand entsorgen. Karten, die mehreren Kartentypen angehören, von denen einer GELD ist (wie der \emph{HAREM} aus \emph{Die Intrige}), sind Geldkarten. Wenn du eine Geldkarte entsorgst, muss jeder Mitspieler, der mindestens fünf Karten in der Hand hat, die gleiche Karte aus seiner Hand auf seinen Ablagestapel legen oder seine Handkarten vorzeigen, damit die anderen Spieler sehen, dass er diese Karte nicht auf der Hand hat. Nimm dir für die entsorgte Karte eine Geldkarte, die bis zu \coin[3] mehr kostet als die entsorgte Geldkarte, und lege sie oben auf deinen Nachziehstapel. Wenn dein Nachziehstapel aufgebraucht ist, wird dies die einzige Karte deines Nachziehstapels. Du musst dir keine teurere Geldkarte nehmen, sondern darfst dir auch eine gleich teure oder preiswertere Geldkarte nehmen.}
\end{tikzpicture}
\hspace{-0.6cm}
\begin{tikzpicture}
	\card
	\cardstrip
	\cardbanner{banner/white.png}
	\cardicon{icons/coin.png}
	\cardprice{4+}
	\cardtitle{Herold}
	\cardcontent{\tiny{\begin{Spacing}{1}
	\vspace{1em}
	Wenn du diese Karte kaufst, darfst du mehr dafür zahlen. Wenn du das tust legst du pro \coin[1], das du überzahlst, eine beliebige Karte deines Ablagestapels oben auf deinen Nachziehstapel. Du darfst dir dafür die Karten deines Ablagestapels ansehen, was normalerweise nicht möglich ist. Allerdings darfst du nicht zuerst deinen Ablagestapel durchsehen, um zu entscheiden, wie viel du überzahlen willst. Sobald du überzahlt hast, musst du die entsprechende Anzahl an Karten in beliebiger Reihenfolge oben auf deinen Nachziehstapel legen, sofern möglich. Wenn du so viel überzahlst, dass du mehr Karten auf deinen Nachziehstapel legen müsstest, als sich in deinem Ablagestapel be nden, legst du einfach alle Karten deines Ablagestapels in beliebiger Reihenfolge auf deinen Nachziehstapel. Falls du den \emph{HEROLD} kaufst, ohne zu überzahlen, darfst du deinen Ablagestapel nicht durchsehen.

	\medskip

	Wenn du diese Karte in der Aktionsphase spielst, ziehst du zuerst eine Karte. Du erhältst dann eine weitere Aktion, die du spielen darfst, nachdem du alle Anweisungen dieser Karte (soweit möglich) erfüllt hast. Dann deckst du die oberste Karte deines Nachziehstapels auf. Wenn es sich um eine Aktionskarte handelt, \emph{musst} du sie spielen. Die Karte zu spielen, verbraucht keine Aktion. Karten, die mehreren Kartentypen angehören, von denen einer AKTION ist (z. B. \emph{GROSSE HALLE} aus \emph{Die Intrige}), sind Aktionskarten. Alle anderen Kartentypen werden auf den Nachziehstapel zurückgelegt, ohne sie zu spielen.

	\medskip

	\emph{Hinweis:} Wenn durch den \emph{HEROLD} eine Dauer-Karte (aus \emph{Seaside}) ausgespielt wird, wird der \emph{HEROLD} dennoch am Ende der Runde wie gewohnt abgelegt, da er nicht gebraucht wird, um an etwas zu erinnern.
	\end{Spacing}}}
\end{tikzpicture}
\hspace{-0.6cm}
\begin{tikzpicture}
	\card
	\cardstrip
	\cardbanner{banner/gold.png}
	\cardicon{icons/coin.png}
	\cardprice{3+}
	\cardtitle{\footnotesize{Meisterstück}}
	\cardcontent{Dies ist eine Geldkarte mit dem Wert \coin[1], wie Kupfer. Wenn du sie kaufst, nimmst du dir ein Silber pro \coin[1], das du überzahlst. Falls du zum Beispiel \coin[6] für das \emph{MEISTERSTÜCK} zahlst, erhältst du drei Silber. Das \emph{MEISTERSTÜCK} ist eine Geldkarte und wird regeltechnisch auch so behandelt.}
\end{tikzpicture}
\hspace{-0.6cm}
\begin{tikzpicture}
	\card
	\cardstrip
	\cardbanner{banner/white.png}
	\cardicon{icons/coin.png}
	\cardprice{5}
	\cardtitle{Bäcker}
	\cardcontent{Wenn du diese Karte spielst, ziehst du eine Karte, darfst eine weitere Aktionskarte ausspielen und nimmst dir eine Münze.

	\medskip

	Wird ein Spiel mit dieser Karte gespielt, erhält jeder Spieler zu Beginn des Spiels eine Münze. Das gilt auch für Partien mit dem \emph{SCHWARZMARKT}, bei denen der \emph{BÄCKER} sich im \emph{SCHWARZMARKT}-Stapel befindet.}
\end{tikzpicture}
\hspace{-0.6cm}
\begin{tikzpicture}
	\card
	\cardstrip
	\cardbanner{banner/white.png}
	\cardicon{icons/coin.png}
	\cardprice{5}
	\cardtitle{\scriptsize{Kaufmannsgilde}}
	\cardcontent{Wenn diese Karte im Spiel ist (d.h. du hast sie in der Aktionsphase gespielt), darfst du eine weitere Karte kaufen und hast zusätzlich \coin[1] zur Verfügung. Jedes Mal, wenn du eine Karte kaufst, nimmst du dir eine Münze (1 Münze, wenn du eine Karte kaufst; 2 Münzen, wenn du zwei Karten kaufst etc.). Denke daran, dass du Münzen nur einsetzen kannst, bevor du Karten kaufst: Du darfst also diese Münze nicht sofort aufwenden. Diese Anweisung ist kumulativ: Wenn du zwei \emph{KAUFMANNSGILDEN} ausgespielt hast, bringt dir jede Karte, die du kaufst, zwei Münzen ein. Wenn du jedoch eine \emph{KAUFMANNSGILDE} mehrfach spielst, aber nur eine im Spiel hast – wie beim \emph{THRONSAAL} (\emph{DOMINION®}) oder \emph{KÖNIGSHOF} (\emph{Blütezeit}) –, bekommst du beim Kauf einer Karte nur eine Münze.}
\end{tikzpicture}
\hspace{-0.6cm}
\begin{tikzpicture}
	\card
	\cardstrip
	\cardbanner{banner/white.png}
	\cardicon{icons/coin.png}
	\cardprice{5}
	\cardtitle{Metzger}
	\cardcontent{Zuerst nimmst du dir 2 Münzen. Dann darfst du eine Karte aus deiner Hand entsorgen und eine beliebige Anzahl an Münzen einsetzen (auch 0 Münzen). Da du den \emph{METZGER} nicht mehr auf der Hand hast, kannst du diese Karte nicht entsorgen. Allerdings kannst du eine andere \emph{METZGER}-Karte entsorgen. Wenn du eine Karte entsorgt hast, nimmst du dir eine Karte, deren Kosten höchstens der Summe aus den Kosten der entsorgten Karte und der Anzahl der eingesetzten Münzen entsprechen darf. So könntest du beispielsweise ein \emph{ANWESEN} entsorgen und sechs Münzen zahlen, um dir eine \emph{PROVINZ} zu nehmen; oder du könntest einen weiteren \emph{METZGER} entsorgen und null Münzen zahlen, um dir ein \emph{HERZOGTUM} zu nehmen. Die Münzen, die du einsetzt, werden in den Vorrat zurückgelegt und dürfen in der Kaufphase nicht mehr benutzt werden, um andere Karten zu kaufen.}
\end{tikzpicture}
\hspace{-0.6cm}
\begin{tikzpicture}
	\card
	\cardstrip
	\cardbanner{banner/white.png}
	\cardicon{icons/coin.png}
	\cardprice{5}
	\cardtitle{Hellseherin}
	\cardcontent{Das Gold und die Fluchkarten stammen aus dem Vorrat und werden auf den Ablagestapel gelegt. Wenn kein Gold mehr vorhanden ist, erhältst du keins. Sind nicht mehr genügend Fluchkarten da, teilst du sie reihum aus, beginnend mit deinem linken Nachbarn. Jeder Spieler, der eine Fluchkarte erhalten hat, muss eine Karte vom Nach- ziehstapel ziehen. Falls ein Spieler keine Fluchkarte bekommen hat – weil nicht genügend Fluchkarten da waren oder aus einem anderen Grund –, zieht er keine Karte. Nutzt ein Spieler den \emph{WACHTURM} (\emph{Blütezeit}), um den Fluch zu entsorgen, hat er dennoch eine Fluchkarte erhalten und zieht deshalb eine Karte. Nutzt ein Spieler den \emph{FAHRENDEN HÄNDLER} (\emph{Hinterland}), um stattdessen ein Silber zu nehmen, hat er keine Fluchkarte erhalten und zieht infolgedessen auch keine Karte.}
\end{tikzpicture}
\hspace{-0.6cm}
\begin{tikzpicture}
	\card
	\cardstrip
	\cardbanner{banner/white.png}
	\cardicon{icons/coin.png}
	\cardprice{5}
	\cardtitle{\footnotesize{Wandergeselle}}
	\cardcontent{Zunächst nennst du eine Karte. Dabei muss es sich nicht um den Namen einer Karte handeln, die in diesem Spiel verwendet wird. Dann deckst du solange Karten vom Nachziehstapel auf, bis du drei Karten aufgedeckt hast, die \emph{nicht} dem von dir genannten Namen entsprechen. Nimm diese Karten auf die Hand und lege die anderen ab. Sollte dabei der Nachziehstapel aufgebraucht werden, bevor du auf drei entsprechende Karten gestoßen bist, mischst du deinen Ablagestapel und nutzt ihn als neuen Nachziehstapel. Wenn du beim weiteren Aufdecken noch immer nicht auf drei Karten mit einem anderen als dem genannten Namen kommst, beendest du das Aufdecken. Nimm die gefundenen Karten anderen Namens auf die Hand und lege den Rest auf deinen Ablagestapel.}
\end{tikzpicture}
\hspace{-0.6cm}
\begin{tikzpicture}
	\card
	\cardstrip
	\cardbanner{banner/white.png}
	\cardtitle{\scriptsize{Empfohlene 10er Sätze\qquad}}
	\cardcontent{\emph{Kunsthandwerk} (Die Gilden + \textit{Basisspiel}):\\
	Steinmetz, Berater, Bäcker, Wandergeselle, Kaufmannsgilde, \textit{Laboratorium}, \textit{Keller}, \textit{Werkstatt}, \textit{Jahrmarkt}, \textit{Geldverleiher}

	\smallskip

	\emph{Rechtschaffen und anständig} (Die Gilden + \textit{Basisspiel}):\\
	Metzger, Bäcker, Leuchtenmacher, Arzt, Wahrsager, \textit{Miliz}, \textit{Dieb}, \textit{Geldverleiher}, \textit{Gärten}, \textit{Dorf}

	\smallskip

	\emph{Des Guten zuviel} (Die Gilden + \textit{Basisspiel}):\\
	Platz, Meisterstück, Leuchtenmacher, Steuereintreiber, Herold, \textit{Bibliothek}, \textit{Umbau}, \textit{Abenteurer}, \textit{Markt}, \textit{Kanzler}

	\smallskip

	\emph{Nenne diese Karte} (Die Gilden + \textit{Die Intrige}):\\
	Bäcker, Arzt, Platz, Berater, Meisterstück, \textit{Burghof}, \textit{Wunschbrunnen}, \textit{Harem}, \textit{Tribut}, \textit{Adelige}

	\smallskip

	\emph{Geschäftstricks} (Die Gilden + \textit{Die Intrige}):\\
	Steinmetz, Herold, Wahrsager, Wandergeselle, Metzger, \textit{Große Halle}, \textit{Adelige}, \textit{Verschwörer}, \textit{Maskerade}, \textit{Kupferschmied}

	\smallskip

	\emph{Entscheidungen, Entscheidungen} (Die Gilden + \textit{Die Intrige}):\\
	Kaufmannsgilde, Leuchtenmacher, Meisterstück, Steuereintreiber, Metzger, \textit{Brücke}, \textit{Handlanger}, \textit{Bergwerk}, \textit{Anbau}, \textit{Herzog}}
\end{tikzpicture}
\hspace{-0.6cm}
\begin{tikzpicture}
	\card
	\cardstrip
	\cardbanner{banner/white.png}
	\cardtitle{\footnotesize{Neue Regeln (1/2)}\qquad}
	\cardcontent{\tiny{\begin{Spacing}{1}
	\vspace{1em}
	\emph{Die Münzen:} Einige Karten in \emph{Die Gilden} erlauben den Spielern, sich Münzen zu nehmen. Münzen werden immer vom Vorrat genommen, nicht von anderen Spielern. Diese Münzen können – anders als die Geldkarten – über den Zug, in dem ein Spieler sie erhält, hinaus aufbewahrt werden. In der Kaufphase eines Spielers kann dieser, \emph{bevor} er Karten kauft, eine beliebige Anzahl an Münzen einsetzen; jede eingesetzte Münze bringt dem Spieler +\coin[1]. Eingesetzte Münzen werden in den Vorrat zurückgelegt. Die Anzahl der Münzen im Vorrat ist nicht begrenzt: Sollten einmal nicht ausreichend Münzen vorhanden sein, verwenden die Spieler einen beliebigen Ersatz. Zwar handelt es sich um die gleichen Marker/Münzen wie in \emph{Seaside} und \emph{Blütezeit}, allerdings können die durch Anweisungen auf den Karten aus \emph{Die Gilden} erworbenen Münzen ausschließlich in der Kaufphase ausgegeben werden. Sie dürfen weder zum Kauf einer Karte mithilfe der Karte \emph{SCHWARZMARKT} benutzt werden, noch dürfen sie auf ein Piratenschiff-Tableau (\emph{Seaside}) oder die \emph{HANDELSROUTE} (\emph{Blütezeit}) gelegt werden.
	\end{Spacing}}}
\end{tikzpicture}
\hspace{-0.6cm}
\begin{tikzpicture}
	\card
	\cardstrip
	\cardbanner{banner/white.png}
	\cardtitle{\footnotesize{Neue Regeln (2/2)}\qquad}
	\cardcontent{\tiny{\begin{Spacing}{1}
	\vspace{1em}
	\emph{Überzahlen:} Für einige Karten in \emph{Die Gilden} darf man mehr zahlen, als sie eigentlich kosten. Bei diesen Karten steht hinter den Kosten ein \enquote{+}, z.B. \coin[2\textsuperscript{+}]. Wenn ein Spieler vor dem Kauf einen \emph{beliebigen zusätzlichen Betrag} für eine solche Karte zahlt, tritt ein auf der jeweiligen Karte genannter Effekt ein, der von der Höhe der Überzahlung abhängt. Überzahlt werden kann mit allen Geldwerten, mit denen Karten gekauft werden: Geldkarten, Münzen und Geldwerte auf Aktionskarten. Die Karten \emph{TRANK} (\emph{Die Alchemisten}) können ebenfalls zum Überzahlen verwendet werden (allerdings ist das nicht immer sinnvoll). Das zum Überzahlen verwendete Geld ist ausgegeben und kann dann nicht mehr eingesetzt werden. Geldkarten werden abgelegt und Münzen kommen zurück in den Vorrat. Spieler können eine Karte ausschließlich beim Kauf überzahlen, nicht wenn sie diese auf eine andere Weise erhalten. Das \enquote{+} ist lediglich eine Erinnerung: Für eine Karte, die hinter den Kosten das Symbol \enquote{+} aufweist, gelten nach wie vor für alle Zwecke die normalen Kosten.
	Grundsätzlich gibt es keine Wechselwirkungen zwischen Karten, welche die Kartenkosten reduzieren – wie die \emph{BRÜCKE} (\emph{Die Intrige}) oder die \emph{FERNSTRASSE} (\emph{Hinterland}) – und dem \enquote{Überzahle}.

	\emph{Zwei Dinge passieren zur gleichen Zeit:} Wenn einem Spieler zwei Dinge gleichzeitig passieren, entscheidet er selbst, in welcher Reihenfolge sie eintreten. Wenn unterschiedlichen Spielern zwei Dinge gleichzeitig passieren, werden diese reihum in Spielreihenfolge abgehandelt, beginnend mit dem Spieler, der gerade an der Reihe ist.
	\end{Spacing}}}
\end{tikzpicture}
\hspace{0.6cm}

	    % Basic settings for this card set
\renewcommand{\cardcolor}{adventures}
\renewcommand{\cardextension}{Erweiterung VIII}
\renewcommand{\cardextensiontitle}{Abenteuer}

\clearpage
\newpage
\section{\cardextension \ - \cardextensiontitle}

\begin{tikzpicture}
	\card
	\cardstrip
	\cardbanner{banner/goldlightbrown.png}
	\cardicon{banner/coin.png}
	\cardprice{2}
	\cardtitle{\scriptsize{Königliche Münzen}}
	\cardcontent{Diese Karte ist eine kombinierte Geld- und Reservekarte. Sie hat den Wert 1 Geld. Spiele sie in der Kaufphase aus und lege sie anschließend auf das Wirtshaustableau. Sobald du eine Aktionskarte ausspielst und die Anweisungen darauf ausgeführt hast, darfst du die Königlichen Münzen von deinem Wirtshaustableau aufrufen. Lege die Karte in deinen Spielbereich und du erhältst +2 Aktionen, nicht aber +1 Geld, da diese Anweisung nur beim Ausspielen der Karte ausgeführt wird. Bis zum Ende der Aufräumphase befinden sich die Königlichen Münzen im Spiel und werden dann abgelegt.}
\end{tikzpicture}
\hspace{-1cm}
\begin{tikzpicture}
	\card
	\cardstrip
	\cardbanner{banner/lightbrown.png}
	\cardicon{banner/coin.png}
	\cardprice{2}
	\cardtitle{Rattenfänger}
	\cardcontent{Du erhältst +1 Karte und +1 Aktion. Lege diese Karte auf dein Wirtshaustableau. Du darfst diese Karte zu Beginn deines Zuges vom Tableau aufrufen. Wenn du das tust, lege die Karte in deinen Spielbereich. Dann musst du eine beliebige Karte aus deiner Hand entsorgen. Lege den Rattenfänger in der Aufräumphase ab.}
\end{tikzpicture}
\hspace{-1cm}
\begin{tikzpicture}
	\card
	\cardstrip
	\cardbanner{banner/white.png}
	\cardicon{banner/coin.png}
	\cardprice{2}
	\cardtitle{Zerstörung}
	\cardcontent{Du erhältst +1 Aktion. Dann musst du entweder diese Zerstörung oder eine Handkarte entsorgen. Je nach Höhe der Kosten der entsorgten Karte siehst du dir die entsprechende Anzahl Karten oben von deinem Nachziehstapel an. Nimm eine davon auf die Hand und lege den Rest ab. Liegt der -1 Karte Marker auf dem Nachziehstapel, wird dieser kurz zur Seite gelegt und nach dem Ziehen der Karten wieder oben auf den Stapel zurückgelegt.}
\end{tikzpicture}
\hspace{-1cm}
\begin{tikzpicture}
	\card
	\cardstrip
	\cardbanner{banner/orange.png}
	\cardicon{banner/coin.png}
	\cardprice{3}
	\cardtitle{Amulett}
	\cardcontent{Diese Karte ist eine Dauerkarte. Wähle sowohl in diesem als auch zu Beginn deines nächsten Zuges jeweils eine der folgenden Optionen: +1 Geld oder entsorge eine Handkarte oder nimm ein Silber vom Vorrat. Du darfst beim 2. Mal etwas anderes wählen. Lege die Karte in der Aufräumphase des nächsten Zuges ab.}
\end{tikzpicture}
\hspace{-1cm}
\begin{tikzpicture}
	\card
	\cardstrip
	\cardbanner{banner/orange.png}
	\cardicon{banner/coin.png}
	\cardprice{3}
	\cardtitle{Ausrüstung}
	\cardcontent{Diese Karte ist eine Dauerkarte. Du erhältst +2 Karten und legst dann bis zu 2 Karten (inkl. eventuell gerade gezogener Karten) verdeckt zur Seite. Entscheidest du dich dafür, keine Karten zur Seite zu legen, legst du die Ausrüstung am Ende des Zuges ab.
	\\
	\medskip
	\\
	Legst du 1 oder 2 Karten zur Seite, nimmst du zu Beginn deines nächsten Zuges die zur Seite gelegten Karten auf die Hand. Lege die Ausrüstung am Ende dieses Zuges ab.}
\end{tikzpicture}
\hspace{-1cm}
\begin{tikzpicture}
	\card
	\cardstrip
	\cardbanner{banner/orangeblue.png}
	\cardicon{banner/coin.png}
	\cardprice{3}
	\cardtitle{\scriptsize{Karawanenwächter}}
	\cardcontent{Diese Karte ist eine kombinierte Dauer- und Reaktionskarte. Wenn du sie ausspielst, erhältst du +1 Karte und +1 Aktion.
	\\
	\medskip
	\\
	Zu Beginn deines nächsten Zuges erhältst du +1 Geld. Lege die Karte am Ende dieses Zuges ab.
	\\
	\medskip
	\\
	Wenn ein Mitspieler eine Angriffskarte ausspielt, darfst du diese Karte aus deiner Hand ausspielen. Du erhältst +1 Karte und +1 Aktion. Die Anweisung +1 Aktion kann in diesem Fall – da gerade ein anderer Spieler am Zug ist – nicht ausgeführt werden. Zu Beginn deines nächsten Zuges erhältst du +1 Geld. Die Karte wird erst in der Aufräumphase deines nächsten Zuges abgelegt.}
\end{tikzpicture}
\hspace{-1cm}
\begin{tikzpicture}
	\card
	\cardstrip
	\cardbanner{banner/lightbrown.png}
	\cardicon{banner/coin.png}
	\cardprice{3}
	\cardtitle{Kundschafter}
	\cardcontent{Diese Karte ist eine Reservekarte. Wenn du sie ausspielst erhältst du +1 Karte und +1 Aktion. Dann legst du die Karte auf dein Wirtshaustableau.
	\\
	\medskip
	\\
	Zu Beginn deines Zuges darfst du diese Karte aufrufen. Wenn du das tust, legst du alle Karten, die du auf der Hand hast, ab und ziehst 5 Karten nach. Lege die Karte am Ende der Aufräumphase ab.}
\end{tikzpicture}
\hspace{-1cm}
\begin{tikzpicture}
	\card
	\cardstrip
	\cardbanner{banner/orange.png}
	\cardicon{banner/coin.png}
	\cardprice{3}
	\cardtitle{Verlies}
	\cardcontent{Diese Karte ist eine Dauerkarte. Du erhältst +2 Karten, +1 Aktion und legst 2 Karten ab.
	\\
	\medskip
	\\
	Zu Beginn deines nächsten Zuges erhältst du +2 Karten und legst 2 Karten ab. Lege das Verlies am Ende dieses Zuges ab.}
\end{tikzpicture}
\hspace{-1cm}
\begin{tikzpicture}
	\card
	\cardstrip
	\cardbanner{banner/lightbrown.png}
	\cardicon{banner/coin.png}
	\cardprice{4}
	\cardtitle{Duplikat}
	\cardcontent{Diese Karte ist eine Reservekarte. Wenn du sie ausspielst, legst du sie auf dein Wirtshaustableau.
	\\
	\medskip
	\\
	Sobald du (auch außerhalb deines eigenen Zuges) eine Karte nimmst, die bis zu 6 Geld kostet, darfst du diese Karte aufrufen. Wenn du das tust, nimmst du dir eine Karte mit gleichem Namen vom Vorrat. Lege diese Karte ab. Wenn keine solche Karte mehr im Vorrat ist, erhältst du nichts. Wenn du eine Karte von einem Stapel nimmst, der nicht zum Vorrat gehört, darfst du dir keine weitere Karte mit gleichem Namen nehmen. Das Duplikat wird in der Aufräumphase des Zuges, in dem es aufgerufen wird (auch außerhalb deines Zuges) abgelegt.}
\end{tikzpicture}
\hspace{-1cm}
\begin{tikzpicture}
	\card
	\cardstrip
	\cardbanner{banner/white.png}
	\cardicon{banner/coin.png}
	\cardprice{4}
	\cardtitle{Elster}
	\cardcontent{Du erhältst +1 Karte und +1 Aktion. Decke die oberste Karte deines Nachziehstapels auf. Ist dies ein Geldkarte (auch eine kombinierte oder Königreichkarte), nimm sie auf die Hand. Ist es eine Aktions- oder Punktekarte, nimmst du eine Elster vom Vorrat. Wenn keine Elster mehr im Vorrat ist, erhältst du nichts. Lege die aufgedeckte Karte zurück auf den Nachziehstapel. Ist es eine kombinierte Punkte- und Geldkarte, nimmst du die Karte auf die Hand und nimmst eine Elster vom Vorrat.}
\end{tikzpicture}
\hspace{-1cm}
\begin{tikzpicture}
	\card
	\cardstrip
	\cardbanner{banner/white.png}
	\cardicon{banner/coin.png}
	\cardprice{4}
	\cardtitle{Geizhals}
	\cardcontent{Du darfst entweder ein Kupfer aus deiner Hand auf dein Wirtshaustableau legen oder du erhältst pro Kupfer auf deinem Tableau +1 Geld. Kupfer auf deinem Wirtshaustableau be nden sich nicht im Spiel, zählen aber bei Spielende zu deinen Karten.}
\end{tikzpicture}
\hspace{-1cm}
\begin{tikzpicture}
	\card
	\cardstrip
	\cardbanner{banner/white.png}
	\cardicon{banner/coin.png}
	\cardprice{4}
	\cardtitle{Hafenstadt}
	\cardcontent{Wenn du diese Karte ausspielst, erhältst du +1 Karte und +2 Aktionen. Wenn du diese Karte kaufst (nicht wenn du sie auf andere Art und Weise erhältst), nimm dir eine Hafenstadt vom Vorrat.}
\end{tikzpicture}
\hspace{-1cm}
\begin{tikzpicture}
	\card
	\cardstrip
	\cardbanner{banner/white.png}
	\cardicon{banner/coin.png}
	\cardprice{4}
	\cardtitle{Kurier}
	\cardcontent{Wenn du diese Karte ausspielst, erhältst du +1 Kauf sowie +2 Geld. Außerdem darfst du deinen Nachziehstapel sofort komplett auf deinen Ablagestapel legen. Karten, die auf diese Weise abgelegt werden, gelten nicht als \enquote{abgelegt}.
	\\
	\medskip
	\\
	Wenn du diese Karte kaufst und dies der erste Kauf (inklusive eventuell erworbener Ereignisse) in deinem Zug ist, nimmst du eine Karte vom Vorrat, die bis zu 4 Geld kostet. Jeder Mitspieler nimmt sich – beginnend bei deinem linken Mitspieler – eine Karte mit gleichem Namen vom Vorrat. Karten, die nicht zum Vorrat gehören, dürfen nicht genommen werden.}
\end{tikzpicture}
\hspace{-1cm}
\begin{tikzpicture}
	\card
	\cardstrip
	\cardbanner{banner/lightbrown.png}
	\cardicon{banner/coin.png}
	\cardprice{4}
	\cardtitle{\footnotesize{Transformation}}
	\cardcontent{Diese Karte ist eine Reservekarte. Wenn du sie ausspielst, erhältst du +1 Aktion. Lege sie dann auf dein Wirtshaustableau.
	\\
	\medskip
	\\
	Zu Beginn deines Zuges darfst du diese Karte aufrufen. Wenn du das tust, entsorgst du eine Handkarte und nimmst eine Karte vom Vorrat, die bis zu 1 Geld mehr kostet als die entsorgte Karte. Nimm die Karte auf die Hand. Du darfst nur Karten aus dem Vorrat nehmen. Lege die Transformation in der Aufräumphase ab.}
\end{tikzpicture}
\hspace{-1cm}
\begin{tikzpicture}
	\card
	\cardstrip
	\cardbanner{banner/white.png}
	\cardicon{banner/coin.png}
	\cardprice{4}
	\cardtitle{Wildhüter}
	\cardcontent{Du erhältst +1 Kauf. Drehe deinen Reise-Marker um (zu Beginn des Spiels liegt der Marker mit der Vorderseite nach oben). Liegt jetzt die Vorderseite oben, ziehst du 5 Karten nach. Liegt die Rückseite oben, passiert nichts. Du darfst in einem Zug mehrere Wildhüter ausspielen und drehst bei jedem Wildhüter den Marker um.}
\end{tikzpicture}
\hspace{-1cm}
\begin{tikzpicture}
	\card
	\cardstrip
	\cardbanner{banner/orange.png}
	\cardicon{banner/coin.png}
	\cardprice{5}
	\cardtitle{Brückentroll}
	\cardcontent{Diese Karte ist eine Dauerkarte. Alle Mitspieler müssen ihren -1 Geld Marker vor sich ablegen. Sie erhalten beim nächsten Mal, wenn sie in irgendeiner Art und Weise mindestens 1 Geld erhalten würden, 1 Geld weniger. Danach wird der Marker wieder neben den Vorrat gelegt. Du erhältst jetzt und zu Beginn deines nächsten Zuges +1 Kauf.
	\\
	\medskip
	\\
	So lange diese Karte im Spiel ist, kosten alle Karten \emph{in deinem Zug} 1 Geld weniger, allerdings nie weniger als 0 Geld. Dies betrifft nicht nur den Kauf, sondern alle Aktionen, bei denen die Kosten einer Karte eine Rolle spielen. Der Brückentroll beeinflusst nicht die Kosten von Ereignissen. Die Wirkung des Brückentrolls ist kumulativ, d.h. wenn du zwei oder mehr Brückentrolle gleichzeitig im Spiel hast, kannst du die Kosten um 2 Geld oder mehr verringern.}
\end{tikzpicture}
\hspace{-1cm}
\begin{tikzpicture}
	\card
	\cardstrip
	\cardbanner{banner/lightbrowngreen.png}
	\cardicon{banner/coin.png}
	\cardprice{5}
	\cardtitle{Ferne Lande}
	\cardcontent{Diese Karte ist eine kombinierte Aktions-, Reserve- und Punktekarte. Wenn du sie ausspielst, lege sie auf dein Wirtshaustableau. Dort bleibt sie bis zum Spielende. Pro Ferne Lande, die du bei Spielende auf deinem Tableau liegen hast, erhältst du 4 Siegpunkte. Alle anderen Ferne Lande zählen 0 Siegpunkte.}
\end{tikzpicture}
\hspace{-1cm}
\begin{tikzpicture}
	\card
	\cardstrip
	\cardbanner{banner/orange.png}
	\cardicon{banner/coin.png}
	\cardprice{5}
	\cardtitle{Geisterwald}
	\cardcontent{Diese Karte ist eine Dauerkarte. Bis zu Beginn deines nächsten Zuges muss jeder Mitspieler, der eine Karte kauft, sofort alle restlichen Handkarten in beliebiger Reihenfolge auf den Nachziehstapel legen. Das Erwerben eines Ereignisses ist nicht mit dem Kauf einer Karte gleichzusetzen. Spieler, die mit einer Reaktionskarte reagieren wollen, müssen dies sofort beim Ausspielen des Geisterwalds tun.
	\\
	\medskip
	\\
	Ziehe zu Beginn deines nächsten Zuges 3 Karten nach. Lege die Karte in der Aufräumphase des nächsten Zuges ab.}
\end{tikzpicture}
\hspace{-1cm}
\begin{tikzpicture}
	\card
	\cardstrip
	\cardbanner{banner/white.png}
	\cardicon{banner/coin.png}
	\cardprice{5}
	\cardtitle{\tiny{Geschichtenerzähler}}
	\cardcontent{Du erhältst +1 Aktion sowie +1 Geld. Spiele bis zu 3 Geldkarten aus deiner Hand aus und zahle alle Geld, die du bisher in diesem Zug ausgespielt hast. Das beinhaltet alle Geldwerte von ausgespielten Geldkarten sowie alle, z. B. durch ausgespielte Aktionskarten, erhaltene zusätzliche Geldwerte (+1 Geld) inkl. dem +1 Geld dieses Geschichtenerzählers. Du hast danach 0 Geld. Für jedes gezahlte 1 Geld erhältst du +1 Karte. Trank-Kosten (aus Dominion – Die Alchemisten) sind davon nicht betroffen.}
\end{tikzpicture}
\hspace{-1cm}
\begin{tikzpicture}
	\card
	\cardstrip
	\cardbanner{banner/white.png}
	\cardicon{banner/coin.png}
	\cardprice{5}
	\cardtitle{\scriptsize{Kunsthandwerker}}
	\cardcontent{Du erhältst +1 Karte, +1 Aktion und +1 Geld. Lege beliebig viele Handkarten ab. Du darfst eine Karte vom Vorrat nehmen, die genau so viel kostet, wie du Karten abgelegt hast. Du kannst dich auch entscheiden, keine Karte abzulegen. Dann darfst du eine Karte nehmen, die genau 0 Geld kostet. Lege die genommene Karte verdeckt auf deinen Nachziehstapel.}
\end{tikzpicture}
\hspace{-1cm}
\begin{tikzpicture}
	\card
	\cardstrip
	\cardbanner{banner/lightbrown.png}
	\cardicon{banner/coin.png}
	\cardprice{5}
	\cardtitle{\scriptsize{Königliche Kutsche}}
	\cardcontent{Diese Karte ist eine Reservekarte. Wenn du sie ausspielst, erhältst du +1 Aktion und legst sie dann auf dein Wirtshaustableau.
	\\
	\medskip
	\\
	Sobald du eine Aktionskarte ausspielst und die Anweisungen darauf ausgeführt hast, darfst du die Königliche Kutsche sofort danach aufrufen, falls die Aktionskarte noch im Spiel ist. Wenn du das tust, führst du die Aktion sofort noch einmal aus. Du kannst die Königliche Kutsche nicht aufrufen, wenn die Aktion bereits entsorgt wurde oder auf Reservekarten, die bereits auf das Tableau gelegt wurden. Lege die Königliche Kutsche in der Aufräumphase ab. Wird die Königliche Kutsche auf eine Dauerkarte gespielt, bleibt sie solange im Spiel, bis die Dauerkarte abgelegt wird.}
\end{tikzpicture}
\hspace{-1cm}
\begin{tikzpicture}
	\card
	\cardstrip
	\cardbanner{banner/gold.png}
	\cardicon{banner/coin.png}
	\cardprice{5}
	\cardtitle{Relikt}
	\cardcontent{Diese Karte ist eine Geldkarte mit zusätzlichen Anweisungen. Sie hat den Wert 2 Geld. Jeder Mitspieler muss seinen -1 Karte Marker auf seinen Nachziehstapel legen. Sobald ein Mitspieler Karten nachziehen muss, zieht er 1 Karte weniger. Dann legt er den Marker neben den Vorrat zurück. Bei Anweisungen, durch die man seine Kartenhand auf X Karten auffüllen darf, wird zunächst der Marker zurückgelegt und dann die Kartenhand auf X Karten aufgefüllt.
	\\
	\medskip
	\\
	Das Relikt ist eine Angriffskarte. Entsprechend kann darauf mit Reaktionskarten reagiert werden. Reagiert ein Spieler mit dem Karawanenwächter, führt er zunächst die Anweisungen aus (inkl. +1 Karte) und legt dann den Marker auf seinen Nachziehstapel.}
\end{tikzpicture}
\hspace{-1cm}
\begin{tikzpicture}
	\card
	\cardstrip
	\cardbanner{banner/white.png}
	\cardicon{banner/coin.png}
	\cardprice{5}
	\cardtitle{Riese}
	\cardcontent{Drehe deinen Reise-Marker um (zu Beginn des Spiels liegt der Marker mit der Vorderseite nach oben). Liegt jetzt die Rückseite oben, erhältst du +1 Geld. Liegt die Vorderseite oben, erhältst du +5 Geld und alle Mitspieler – beginnend bei deinem linken Mitspieler – müssen die oberste Karte ihres Nachziehstapels aufdecken. Aufgedeckte Karten, die 3 bis 6 Geld kosten, müssen entsorgt werden. Karten mit Trank-Kosten müssen nicht entsorgt werden, Karten mit + Geld (z. B. aus Dominion – Die Gilden) oder * Geld müssen entsorgt werden.
	\\
	\medskip
	\\
	Ansonsten legt der Spieler die aufgedeckte Karte ab und nimmt sich einen Fluch vom Vorrat. Die Mitspieler können auf das Ausspielen eines Riesen mit einer Reaktionskarte reagieren, auch wenn die Rückseite des Reise-Markers oben liegt und der Angriff gar nicht ausgeführt wird.}
\end{tikzpicture}
\hspace{-1cm}
\begin{tikzpicture}
	\card
	\cardstrip
	\cardbanner{banner/gold.png}
	\cardicon{banner/coin.png}
	\cardprice{5}
	\cardtitle{Schatz}
	\cardcontent{Diese Karte ist eine Geldkarte mit zusätzlichen Anweisungen. Sie hat den Wert 2 Geld. Nimm ein Gold und ein Kupfer (in beliebiger Reihenfolge) vom Vorrat und lege sie ab. Ist ein Stapel leer, erhältst du nur die andere Karte.}
\end{tikzpicture}
\hspace{-1cm}
\begin{tikzpicture}
	\card
	\cardstrip
	\cardbanner{banner/orange.png}
	\cardicon{banner/coin.png}
	\cardprice{5}
	\cardtitle{Sumpfhexe}
	\cardcontent{Diese Karte ist eine Dauerkarte. Bis zu Beginn deines nächsten Zuges muss jeder Mitspieler für jede Karte, die er kauft, jeweils einen Fluch vom Vorrat nehmen. Das Erwerben eines Ereignisses ist nicht mit dem Kauf einer Karte gleichzusetzen. Spieler, die mit einer Reaktionskarte reagieren wollen, müssen dies sofort beim Ausspielen der Sumpfhexe tun.
	\\
	\medskip
	\\
	Zu Beginn deines nächsten Zuges erhältst du +3 Geld. Lege die Karte in der Aufräumphase des nächsten Zuges ab.}
\end{tikzpicture}
\hspace{-1cm}
\begin{tikzpicture}
	\card
	\cardstrip
	\cardbanner{banner/white.png}
	\cardicon{banner/coin.png}
	\cardprice{5}
	\cardtitle{\scriptsize{Verlorene Stadt}}
	\cardcontent{Wenn du diese Karte ausspielst, erhältst du +2 Karten und +2 Aktionen. Wenn du diese Karte nimmst, zieht jeder Mitspieler 1 Karte.}
\end{tikzpicture}
\hspace{-1cm}
\begin{tikzpicture}
	\card
	\cardstrip
	\cardbanner{banner/lightbrown.png}
	\cardicon{banner/coin.png}
	\cardprice{5}
	\cardtitle{Weinhändler}
	\cardcontent{Diese Karte ist eine Reservekarte. Wenn du sie ausspielst, erhältst du +1 Kauf sowie +4 Geld und legst die Karte dann auf dein Wirtshaustableau.
	\\
	\medskip
	\\
	Wenn du am Ende deiner Kaufphase mindestens 2 Geld nicht ausgegeben hast, darfst du diese Karte von deinem Wirtshaustableau ablegen. Hast du mehrere Weinhändler auf deinem Tableau liegen, kannst du alle ablegen. Du brauchst nicht für jeden Weinhändler 2 Geld übrig zu haben.}
\end{tikzpicture}
\hspace{-1cm}
\begin{tikzpicture}
	\card
	\cardstrip
	\cardbanner{banner/orange.png}
	\cardicon{banner/coin.png}
	\cardprice{6}
	\cardtitle{Gefolgsmann}
	\cardcontent{Diese Karte ist eine Dauerkarte. Wenn du sie ausspielst, lege sie neben deinen Spielbereich. Sie bleibt bis zum Spielende im Spiel und wird nicht abgelegt.
	\\
	\medskip
	\\
	Für den Rest des Spiels erhältst du zu Beginn deines Zuges +1 Karte. Wenn du einen Gefolgsmann zweimal ausspielen darfst (z. B. durch den Thronsaal aus der Basis), lege beide Karten zur Seite und du erhältst bis zum Spielende zu Beginn jedes Zuges +2 Karten.}
\end{tikzpicture}
\hspace{-1cm}
\begin{tikzpicture}
	\card
	\cardstrip
	\cardbanner{banner/white.png}
	\cardtitle{Ereignisse (1/8)\quad}
	\cardcontent{Ereignisse können nur in der Kaufphase erworben werden. Dies benötigt 1 Kauf sowie genügend (vorher ausgespielte) Geldwerte. Der benötigte Geldwert ist auf jedem Ereignis oben links zu finden. Sobald du ein Ereignis erwirbst, führst du die darauf beschriebene Anweisung aus. Du nimmst das Ereignis aber nicht an dich.
	\\
	\medskip
	\\
	\emph{Almosen (0):} Dieses Ereignis darf nur 1x pro Zug erworben werden. Falls du keine Geldkarten zu diesem Zeitpunkt im Spiel hast, nimm dir eine Karte vom Vorrat, die bis zu 4 Geld kostet. Eintausch-Karten oder Preiskarten (aus Dominion – Reiche Ernte) mit * Geld gehören nicht zum Vorrat und dürfen nicht genommen werden.
	\\
	\medskip
	\\
	\emph{Leihgabe (0):} Dieses Ereignis darf nur 1x pro Zug erworben werden. Du erhältst +1 Kauf. Wenn dein -1 Karte Marker nicht auf deinem Nachziehstapel liegt, lege ihn dorthin und erhalte +1 Geld. Das nächste Mal, wenn du Karten nachziehen musst, ziehst du 1 Karte weniger.}
\end{tikzpicture}
\hspace{-1cm}
\begin{tikzpicture}
	\card
	\cardstrip
	\cardbanner{banner/white.png}
	\cardtitle{Ereignisse (2/8)\quad}}
	\cardcontent{\emph{Quest (0):} Wähle eine der Optionen, um ein Gold zu nehmen und abzulegen: Entweder legst du 1 Angri skarte aus deiner Hand oder 2 Flüche oder 6 beliebige Karten ab. Du kannst dich entscheiden, nur 1 Fluch oder weniger als 6 Karten abzulegen, wenn du nicht genügend entsprechende Karten auf der Hand hast – dann erhältst du kein Gold.
	\\
	\medskip
	\\
	\emph{Zuflucht (1):} Dieses Ereignis darf nur 1x pro Zug erworben werden. Du erhältst +1 Kauf und legst eine beliebige Handkarte verdeckt zur Seite. Nachdem du in der Aufräumphase Karten nachgezogen hast, nimmst du die zur Seite gelegte Karte wieder auf die Hand.
	\\
	\medskip
	\\
	\emph{Spähtrupp (2):} Du erhältst +1 Kauf. Schau dir die obersten 5 Karten deines Nachziehstapels an. Lege 3 davon ab und den Rest in beliebiger Reihenfolge zurück auf den Nachziehstapel. Hast du nach dem Mischen des Ablagestapels weniger als 5 Karten zur Verfügung, legst du zuerst – wenn möglich – 3 Karten ab. Nur den Rest (0, 1 oder 2 Karten) legst du zurück auf den Nachziehstapel.
	\\
	\medskip
	\\
	Der \emph{Spähtrupp} ist nicht betroffen vom -1 Karte Marker, der eventuell auf deinem Nachziehstapel liegt, da du die Karten nicht nimmst sondern nur ansiehst. In diesem Fall legst du den Marker nach dem Ausführen des SPÄHTRUPPS auf den Nachziehstapel zurück.}
\end{tikzpicture}
\hspace{-1cm}
\begin{tikzpicture}
	\card
	\cardstrip
	\cardbanner{banner/white.png}
	\cardtitle{Ereignisse (3/8)\quad}}
	\cardcontent{\emph{Wanderzirkus (2):} Du erhältst +2 Käufe. Wenn du in dem Zug, in dem du den Wanderzirkus erwirbst, eine Karte nimmst, darfst du diese oben auf deinen Nachziehstapel legen. Wenn du dies nicht möchtest, lege die genommene Karte ab. Der Wanderzirkus funktioniert nicht bei den Eintausch-Karten mit einem * Geld, da diese nicht genommen, sondern eingetauscht werden.
	\\
	\medskip
	\\
	\emph{Expedition (3):} In der Aufräumphase des Zuges, in dem du diese Karte erwirbst, ziehst du 2 Karten zusätzlich. Normalerweise ziehst du 5 Karten nach, mit einer Expedition 7 Karten, mit zwei Expeditionen 9 Karten usw.
	\\
	\medskip
	\\
	\emph{Freudenfeuer (3):} Entsorge bis zu 2 Karten, die du gerade im Spiel hast. Du darfst keine Handkarten entsorgen. Entsorgst du eine oder zwei Geldkarten, kannst du trotzdem den durch diese Karten produzierten Geldwert nutzen.
	\\
	\medskip
	\\
	\emph{Planung (3):} Lege deinen Entsorgungs-Marker auf einen beliebigen Aktions-Vorratsstapel. Solange dieser Marker auf dem Stapel liegt, darfst du, immer wenn du eine Karte von diesem Stapel kaufst, eine beliebige Handkarte entsorgen.
	\\
	\medskip
	\\
	Wenn du eine weitere Planung erwirbst, legst du den Marker auf einen anderen Aktions-Vorratsstapel.}
\end{tikzpicture}
\hspace{-1cm}
\begin{tikzpicture}
	\card
	\cardstrip
	\cardbanner{banner/white.png}
	\cardtitle{Ereignisse (4/8)\quad}}
	\cardcontent{\emph{Überfahrt (3):} Lege deinen -2 Geld Kosten-Marker auf einen beliebigen Aktions-Vorratsstapel. Solange dieser Marker auf dem Stapel liegt, kosten Karten, die von diesem Stapel stammen, für dich für alle Belange 2 Geld weniger, niemals aber weniger als 0 Geld.
	\\
	\medskip
	\\
	Wenn du eine weitere Überfahrt erwirbst, legst du den Marker auf einen anderen Aktions-Vorratsstapel.
	\\
	\medskip
	\\
	\emph{Mission (4):} Dieses Ereignis darf nur 1x pro Zug erworben werden. Du spielst nach diesem Zug einen weiteren Zug, falls der vorangegangene Zug von einem Mitspieler gespielt wurde und nicht von dir (z. B. durch den Außenposten aus Dominion – Seaside). In diesem Zug darfst du keine Karten kaufen. Du darfst allerdings Ereignisse erwerben, Karten auf andere Weise nehmen oder Reisende eintauschen. Erwirbst du in deinem Extrazug eine weitere Mission, darfst du keinen weiteren Zug ausführen, da der vorherige Zug dein eigener war.}
\end{tikzpicture}
\hspace{-1cm}
\begin{tikzpicture}
	\card
	\cardstrip
	\cardbanner{banner/white.png}
	\cardtitle{Ereignisse (5/8)\quad}}
	\cardcontent{\emph{Pilgerfahrt (4):} Dieses Ereignis darf nur 1x pro Zug erworben werden. Drehe deinen Reise-Marker um (zu Spielbeginn liegt dieser mit der Vorderseite nach oben). Liegt dann die Rückseite oben, passiert nichts. Liegt die Vorderseite oben, wählst du bis zu 3 Karten mit unterschiedlichem Namen, die du gerade im Spiel hast, nimmst von jeder dieser Karten eine weitere Karte vom Vorrat und legst sie ab. Eintausch-Karten oder Preiskarten (aus Dominion – Reiche Ernte) mit * Geld gehören nicht zum Vorrat und dürfen nicht genommen werden.
	\\
	\medskip
	\\
	\emph{Ball (5):} Nimm deinen -1 Geld Marker und lege ihn vor dir ab. Das nächste Mal, wenn du mindestens 1 Geld erhältst, erhältst du stattdessen 1 Geld weniger. Danach legst du den Marker wieder neben den Vorrat.
	\\
	\medskip
	\\
	Nimm 2 Karten vom Vorrat, die jede bis zu 4 Geld kosten. Das können gleiche oder verschiedene Karten sein. Lege beide Karten ab. Eintausch-Karten oder Preiskarten (aus Dominion – Reiche Ernte) gehören nicht zum Vorrat und dürfen nicht genommen werden.
	\\
	\medskip
	\\
	\emph{Handel (5):} Entsorge bis zu 2 Handkarten. Du darfst auch 1 oder gar keine Karte entsorgen. Für jede so entsorgte Karte nimmst du ein Silber vom Vorrat und legst es ab.}
\end{tikzpicture}
\hspace{-1cm}
\begin{tikzpicture}
	\card
	\cardstrip
	\cardbanner{banner/white.png}
	\cardtitle{Ereignisse (6/8)\quad}}
	\cardcontent{\emph{Seeweg (5):} Nimm eine Aktionskarte vom Vorrat, die bis zu 4 Geld kostet, und lege diese ab. Lege deinen +1 Kauf-Marker auf den Vorratsstapel, von dem du die Aktionskarte genommen hast (dieser kann dann auch leer sein). Du musst eine Karte vom Vorrat nehmen, um den Marker auf den entsprechenden Stapel legen zu dürfen. Solange dieser Marker auf dem Stapel liegt, erhältst du +1 Kauf, sobald du eine Karte, die von diesem Stapel stammt, ausspielst. Kostet die Aktionskarte, die du durch den Seeweg nimmst, nur in diesem Zug bis zu 4 Geld, z. B. wenn du vorher einen Brückentroll gespielt hast und dadurch alle Karte in diesem Zug 1 Geld weniger kosten, bleibt der Marker später liegen, auch wenn die entsprechende Karte nach deinem Zug wieder 5 Geld kostet. Auf Stapel, die nicht zum Vorrat gehören, darf der Marker nicht gelegt werden.
	\\
	\medskip
	\\
	Durch den Erwerb eines weiteren Seeweges oder das Aufrufen eines Lehrers darfst du den Marker auf einen anderen Aktions-Vorratsstapel legen.}
\end{tikzpicture}
\hspace{-1cm}
\begin{tikzpicture}
	\card
	\cardstrip
	\cardbanner{banner/white.png}
	\cardtitle{Ereignisse (7/8)\quad}}
	\cardcontent{\emph{Überfall (5):} Für jedes Silber, das du gerade im Spiel hast, wenn du diese Karte erwirbst, nimm ein weiteres Silber vom Vorrat und lege es ab. Außerdem legen alle Mitspieler ihren -1 Karte Marker auf ihren Nachziehstapel. Das nächste Mal, wenn sie Karten nachziehen müssen, ziehen sie stattdessen 1 Karte weniger. Auch wenn diese Anweisung wie eine Angriff fungiert, ist der Überfall keine Angriffskarte. Entsprechend können die Mitspieler nicht mit einer Reaktionskarte darauf reagieren.
	\\
	\medskip
	\\
	\emph{Training (6):} Lege deinen +1 Geld Marker auf einen beliebigen Aktions-Vorratsstapel. Solange dieser Marker auf dem Stapel liegt, erhältst du +1 Geld, sobald du eine Karte, die von diesem Stapel stammt, ausspielst.
	\\
	\medskip
	\\
	Durch den Erwerb eines weiteren Trainings oder das Aufrufen eines Lehrers darfst du den Marker auf einen anderen Aktions-Vorratsstapel legen.
	\\
	\medskip
	\\
	\emph{Verlorene Kunst (6):} Lege deinen +1 Aktion-Marker auf einen beliebigen Aktions-Vorratsstapel. Solange dieser Marker auf dem Stapel liegt, erhältst du +1 Aktion, sobald du eine Karte, die von diesem Stapel stammt, ausspielst.
	\\
	\medskip
	\\
	Durch den Erwerb einer weiteren Verlorenen Kunst oder das Aufrufen eines Lehrers darfst du den Marker auf einen anderen Aktions-Vorratsstapel legen.}
\end{tikzpicture}
\hspace{-1cm}
\begin{tikzpicture}
	\card
	\cardstrip
	\cardbanner{banner/white.png}
	\cardtitle{Ereignisse (8/8)\quad}}
	\cardcontent{\emph{Erbschaft (7):} Dieses Ereignis darf nur 1x pro Spiel und Spieler erworben werden. Lege eine beliebige Nicht-Punktekarte vom Vorrat, die eine Aktionskarte ist und bis zu 4 Geld kostet, zur Seite. Lege deinen Anwesen-Marker darauf. Ab jetzt fungieren alle deine Anwesen wie die markierte Karte. Du kannst Dauer- oder Reservekarten, Geldkarten, Angriffskarten usw. markieren. Wenn du ein Anwesen auf der Hand hast, darfst du es in der Phase, in der die markierte Karte ausgespielt werden würde, auch ausspielen. Führe alle Anweisungen der markierten Karte aus.
	\\
	\medskip
	\\
	Eigenschaften, Anweisungen und Typ (z. B. Aktion, Angriff etc.) der markierten Karte werden durch die Erbschaft kopiert, nicht aber Name oder Wert bzw. von welchem Stapel die markierte Karte stammt. So kann ein Spieler, der mit Hilfe der Erbschaft einen Wildhüter markiert und gleichzeitig auf diesem Stapel seinen +1 Aktion-Marker liegen hat, durch das Ausspielen eines Anwesens zwar die Anweisungen des Wildhüters nutzen, er erhält aber nicht zusätzlich +1 Aktion.
	\\
	\medskip
	\\
	Bei Spielende zählen deine Anwesen weiterhin jeweils 1 Siegpunkt.
	\\
	\medskip
	\\
	\emph{Wegsuche (8):} Lege deinen +1 Karte-Marker auf einen beliebigen Aktions-Vorratsstapel. Solange dieser Marker auf dem Stapel liegt, erhältst du +1 Karte, sobald du eine Karte, die von diesem Stapel stammt, ausspielst.}
\end{tikzpicture}
\hspace{-1cm}
\begin{tikzpicture}
	\card
	\cardstrip
	\cardbanner{banner/white.png}
	\cardicon{banner/coin.png}
	\cardprice{2}
	\cardtitle{Kleinbauer}
	\cardcontent{Der Kleinbauer ist eine Königreichkarte und kann wie jede andere Königreichkarte gekauft oder genommen werden. Der Kleinbauer ist außerdem ein \emph{Reisender}, der im Spielverlauf in einen Soldaten eingetauscht werden kann.
	\\
	\medskip
	\\
	Wenn du den Kleinbauern ausspielst, erhältst du +1 Kauf sowie +1 Geld. In der Aufräumphase darfst du entscheiden, ob du den Kleinbauern ablegst oder zurück auf den Vorratsstapel legst. Wenn du das tust, erhältst du einen Soldaten und legst ihn ab. Die angegebenen Kosten werden \emph{nicht} bezahlt.}
\end{tikzpicture}
\hspace{-1cm}
\begin{tikzpicture}
	\card
	\cardstrip
	\cardbanner{banner/white.png}
	\cardicon{banner/coin.png}
	\cardprice{3\textsuperscript{*}}
	\cardtitle{Soldat}
	\cardcontent{Einen Soldaten erhältst du nur, wenn du einen Kleinbauern eintauschst. Der Soldat ist ein \emph{Reisender}, der im Spielverlauf in einen Flüchtling eingetauscht werden kann.
	\\
	\medskip
	\\
	Wenn du den Soldaten ausspielst, erhältst du +2 Geld sowie für jede andere Angriffskarte (außer diesem Soldaten), die du zu diesem Zeitpunkt im Spiel hast, +1 Geld. Außerdem muss jeder Mitspieler, der 4 oder mehr Karten auf der Hand hat, 1 Karte ablegen.
	\\
	\medskip
	\\
	In der Aufräumphase darfst du entscheiden, ob du den Soldaten ablegst oder zurück auf den entsprechenden Stapel legst. Wenn du das tust, erhältst du einen Flüchtling und legst ihn ab. Die angegebenen Kosten werden \emph{nicht} bezahlt.}
\end{tikzpicture}
\hspace{-1cm}
\begin{tikzpicture}
	\card
	\cardstrip
	\cardbanner{banner/white.png}
	\cardicon{banner/coin.png}
	\cardprice{4\textsuperscript{*}}
	\cardtitle{Flüchtling}
	\cardcontent{Einen Flüchtling erhältst du nur, wenn du einen Soldaten eintauschst. Der Flüchtling ist ein \emph{Reisender}, der im Spielverlauf in einen Schüler eingetauscht werden kann.
	\\
	\medskip
	\\
	Wenn du den Flüchtling ausspielst, erhältst du + 2 Karten sowie + 1 Aktion und du musst eine Handkarte ablegen.
	\\
	\medskip
	\\
	In der Aufräumphase darfst du entscheiden, ob du den Soldaten ablegst oder zurück auf den entsprechenden Stapel legst. Wenn du das tust, erhältst du einen Schüler und legst ihn ab. Die angegebenen Kosten werden \emph{nicht} bezahlt.}
\end{tikzpicture}
\hspace{-1cm}
\begin{tikzpicture}
	\card
	\cardstrip
	\cardbanner{banner/white.png}
	\cardicon{banner/coin.png}
	\cardprice{5\textsuperscript{*}}
	\cardtitle{Schüler}
	\cardcontent{Einen Schüler erhältst du nur, wenn du einen Flüchtling eintauschst. Der Schüler ist ein \emph{Reisender}, der im Spielverlauf in einen Lehrer eingetauscht werden kann.
	\\
	\medskip
	\\
	Wenn du den Schüler ausspielst, darfst du eine beliebige Aktionskarte aus deiner Hand zweimal direkt hintereinander ausspielen. Ist die ausgespielte Aktionskarte eine Karte vom Vorrat, nimm dir eine Karte mit gleichem Namen vom Vorrat und lege sie ab. Gehört der Stapel der ausgespielten Karte \emph{nicht} zum Vorrat (z. B. Eintausch-Karten oder Preiskarten aus Dominion – Reiche Ernte), darfst du dir keine weitere Karte nehmen.
	\\
	\medskip
	\\
	In der Aufräumphase darfst du entscheiden, ob du den Schüler ablegst oder zurück auf den entsprechenden Stapel legst. Wenn du das tust, erhältst du einen Lehrer und legst ihn ab. Die angegebenen Kosten werden \emph{nicht} bezahlt.}
\end{tikzpicture}
\hspace{-1cm}
\begin{tikzpicture}
	\card
	\cardstrip
	\cardbanner{banner/lightbrown.png}
	\cardicon{banner/coin.png}
	\cardprice{6\textsuperscript{*}}
	\cardtitle{Lehrer}
	\cardcontent{Einen Lehrer erhältst du nur, wenn du einen Schüler eintauschst. Er kann in keine andere Karte eingetauscht werden.
	\\
	\medskip
	\\
	Wenn du den Lehrer ausspielst, lege ihn auf dein Wirtshaustableau.
	\\
	\medskip
	\\
	Zu Beginn deines Zuges darfst du den Lehrer von deinem Tableau aufrufen und in den Spielbereich legen. Wenn du das tust, legst du entweder den +1 Karte-, +1 Aktion-, +1 Kauf- oder +1 Geld-Marker auf einen beliebigen Aktions-Vorratsstapel, auf dem du zu diesem Zeitpunkt keine weiteren eigenen Marker liegen hast. Immer wenn du eine Karte, die von diesem Stapel stammt, spielst, erhältst du zuerst (bevor die Anweisungen der Karte ausgeführt werden) den entsprechenden Bonus des Markers. Lege den Lehrer in der Aufräumphase ab.}
\end{tikzpicture}
\hspace{-1cm}
\begin{tikzpicture}
	\card
	\cardstrip
	\cardbanner{banner/white.png}
	\cardicon{banner/coin.png}
	\cardprice{2}
	\cardtitle{Page}
	\cardcontent{Der Page ist eine Königreichkarte und kann wie jede andere Königreichkarte gekauft oder genommen werden. Der Page ist außerdem ein \emph{Reisender}, der im Spielverlauf in einen Schatzsucher eingetauscht werden kann.
	\\
	\medskip
	\\
	Wenn du den Pagen ausspielst, erhältst du +1 Karte sowie +1 Aktion.
	\\
	\medskip
	\\
	In der Aufräumphase darfst du entscheiden, ob du den Pagen ablegst oder zurück auf den Vorratsstapel legst. Wenn du das tust, erhältst du einen Schatzsucher und legst ihn ab. Die angegebenen Kosten werden \emph{nicht} bezahlt.}
\end{tikzpicture}
\hspace{-1cm}
\begin{tikzpicture}
	\card
	\cardstrip
	\cardbanner{banner/white.png}
	\cardicon{banner/coin.png}
	\cardprice{3\textsuperscript{*}}
	\cardtitle{Schatzsucher}
	\cardcontent{Einen Schatzsucher erhältst du nur, wenn du einen Pagen eintauschst. Der Schatzsucher ist ein \emph{Reisender}, der im Spielverlauf in einen Krieger eingetauscht werden kann.
	\\
	\medskip
	\\
	Wenn du den Schatzsucher ausspielst, erhältst du +1 Aktion sowie +1 Geld. Für jede Karte, die dein rechter Mitspieler in seinem letzten Zug gekauft oder genommen hat, nimmst du ein Silber vom Vorrat und legst es ab.
	\\
	\medskip
	\\
	In der Aufräumphase darfst du entscheiden, ob du den Schatzsucher ablegst oder zurück auf den entsprechenden Stapel legst. Wenn du das tust, erhältst du einen Krieger und legst ihn ab. Die angegebenen Kosten werden \emph{nicht} bezahlt.}
\end{tikzpicture}
\hspace{-1cm}
\begin{tikzpicture}
	\card
	\cardstrip
	\cardbanner{banner/white.png}
	\cardicon{banner/coin.png}
	\cardprice{4\textsuperscript{*}}
	\cardtitle{Krieger}
	\cardcontent{Einen Krieger erhältst du nur, wenn du einen Schatzsucher eintauschst. Der Krieger ist ein \emph{Reisender}, der im Spielverlauf in einen Helden eingetauscht werden kann.
	\\
	\medskip
	\\
	Wenn du den Krieger ausspielst, erhältst du +2 Karten. Für jeden Reisenden, den du (inkl. dieses Kriegers) zu diesem Zeitpunkt im Spiel hast, müssen alle Mitspieler – beginnend bei deinem linken Nachbarn – die oberste Karte ihres Nachziehstapels aufdecken und ablegen. Abgelegte Karten, die genau 3 Geld oder 4 Geld kosten, müssen entsorgt werden. Karten mit Trank-Kosten müssen nicht entsorgt werden. Karten mit * Geld oder + Geld müssen entsorgt werden, wenn sie 3 Geld oder 4 Geld kosten. Alle aufgedeckten Karten, die nicht genau 3 Geld oder 4 Geld kosten, müssen abgelegt werden.
	\\
	\medskip
	\\
	In der Aufräumphase darfst du entscheiden, ob du den Krieger ablegst oder zurück auf den entsprechenden Stapel legst. Wenn du das tust, erhältst du einen Helden und legst ihn ab. Die angegebenen Kosten werden \emph{nicht} bezahlt.}
\end{tikzpicture}
\hspace{-1cm}
\begin{tikzpicture}
	\card
	\cardstrip
	\cardbanner{banner/white.png}
	\cardicon{banner/coin.png}
	\cardprice{5\textsuperscript{*}}
	\cardtitle{Held}
	\cardcontent{Einen Helden erhältst du nur, wenn du einen Krieger eintauschst. Der Held ist ein \emph{Reisender}, der im Spielverlauf in einen Champion eingetauscht werden kann.
	\\
	\medskip
	\\
	Wenn du den Helden ausspielst, erhältst du +2 Geld und darfst dir eine beliebige Geldkarte nehmen, die in diesem Spiel verwendet wird. Dazu gehören auch Geldkarten, die zu den Königreichkarten gehören. Lege die Geldkarte ab.
	\\
	\medskip
	\\
	In der Aufräumphase darfst du entscheiden, ob du den Helden ablegst oder zurück auf den entsprechenden Stapel legst. Wenn du das tust, erhältst du einen Champion und legst ihn ab. Die angegebenen Kosten werden \emph{nicht} bezahlt.}
\end{tikzpicture}
\hspace{-1cm}
\begin{tikzpicture}
	\card
	\cardstrip
	\cardbanner{banner/orange.png}
	\cardicon{banner/coin.png}
	\cardprice{6\textsuperscript{*}}
	\cardtitle{Champion}
	\cardcontent{Einen Champion erhältst du nur, wenn du einen Helden eintauschst. Er kann in keine andere Karte eingetauscht werden.
	\\
	\medskip
	\\
	Wenn du den Champion ausspielst, erhältst du +1 Aktion. Der Champion ist eine Dauerkarte. Er bleibt bis zum Spielende im Spiel.
	\\
	\medskip
	\\
	Immer wenn ein Mitspieler ab jetzt eine Angriffskarte ausspielt, bist du davon nicht betroffen (auch wenn du das möchtest). Für den Rest des Spiels erhältst du jedes Mal, wenn du eine Aktionskarte ausspielst, +1 Aktion. Die Anweisung über der Trennlinie wird nur beim Ausspielen des Champion ausgeführt.}
\end{tikzpicture}
\hspace{-1cm}
\begin{tikzpicture}
	\card
	\cardstrip
	\cardbanner{banner/white.png}
	\cardicon{}
	\cardprice{}
	\cardtitle{\scriptsize{Empfohlene 10er Sätze\qquad}}
	\cardcontent{\tiny{\emph{Sanfte Einführung}
	\\
	\smallskip
	\\
	Spähtrupp (Ereignis), Amulett, Duplikat, Ferne Lande, Gefolgsmann, Hafenstadt, Rattenfänger, Riese, Schatz, Verlies, Wildhüter
	\\
	\smallskip
	\\
	\emph{Profi-Einführung}
	\\
	\smallskip
	\\
	Mission (Ereignis), Planung (Ereignis), Elster, Geisterwald, Karawanenwächter, Kleinbauer (+ Eintauschkarten), Königliche Münzen, Sumpfhexe, Verlorene Stadt, Weinhändler, Zerstörung, Transformation
	\\
	\smallskip
	\\
	\emph{Auf höchster Ebene (mit Basisspiel)}
	\\
	\smallskip
	\\
	Training (Ereignis), Ausrüstung, Geizhals, Kundschafter, Verlies, Verlorene Stadt, Markt, Miliz, Spion, Thronsaal, Werkstatt
	\\
	\smallskip
	\\
	\emph{Verzerrte Größen (mit Basisspiel)}
	\\
	\smallskip
	\\
	Freudenfeuer (Ereignis), Überfall (Ereignis), Amulett, Duplikat, Kurier, Riese, Schatz, Bürokrat, Dieb, Gärten, Geldverleiher, Hexe
	\\
	\smallskip
	\\
	\emph{Lizenz Fabrik (mit Die Intrige)}
	\\
	\smallskip
	\\
	Pilgerfahrt (Ereignis), Brückentroll, Duplikat, Königliche Kutsche, Page (+ Eintauschkarten), Zerstörung, Adelige, Geheimkammer, Harem, Trickser, Verschwörer
	\\
	\smallskip
	\\
	\emph{Meister der Finanzen (mit Die Intrige)}
	\\
	\smallskip
	\\
	Ball (Ereignis), Leihgabe (Ereignis), Ausrüstung, Ferne Lande, Kunsthandwerker, Transformation, Weinhändler, Anbau, Armenviertel, Brücke, Handlanger, Verwalter
	\\}}
\end{tikzpicture}
\hspace{-1cm}
\begin{tikzpicture}
	\card
	\cardstrip
	\cardbanner{banner/white.png}
	\cardicon{}
	\cardprice{}
	\cardtitle{\scriptsize{Empfohlene 10er Sätze\qquad}}
	\cardcontent{\tiny{\emph{Prinz von Oranien (mit Seaside)}
	\\
	\smallskip
	\\
	Mission (Ereignis), Amulett, Geisterwald, Page (+ Eintauschkarten), Sumpfhexe, Verlies, Fischerdorf, Handelsschiff, Karawane, Schatzkarte, Taktiker
	\\
	\smallskip
	\\
	\emph{Geschenke und triviale Präsente (mit Seaside)}
	\\
	\smallskip
	\\
	Expedition (Ereignis), Quest (Ereignis), Brückentroll, Gefolgsmann, Karawanenwächter, Kurier, Verlorene Stadt, Botschafter, Embargo, Hafen, Müllverwerter, Schmuggler
	\\
	\smallskip
	\\
	\emph{Letzter Wille und das Denkmal (mit Blütezeit)}
	\\
	\smallskip
	\\
	Erbschaft (Ereignis), Königliche Münzen, Kurier, Relikt, Schatz, Verlies, Bischof, Denkmal, Gesindel, Gewölbe, Leihhaus, Platin- und Kolonie-Karten
	\\
	\smallskip
	\\
	\emph{Gedanken im großen Stil (mit Blütezeit)}
	\\
	\smallskip
	\\
	Ball (Ereignis), Überfahrt (Ereignis), Ferne Lande, Geizhals, Gefolgsmann, Geschichtenerzähler, Riese, Ausbau, Hausierer, Hort, Königshof, Schmuggelware, Platin- und Kolonie-Karten
	\\
	\smallskip
	\\
	\emph{Rückkehr des Helden (mit Reiche Ernte)}
	\\
	\smallskip
	\\
	Wanderzirkus (Ereignis), Geizhals, Kunsthandwerker, Page (+ Eintauschkarten), Relikt, Wildhüter, Bauerndorf, Festplatz, Harlekin, Menagerie, Pferdehändler
	\\
	\smallskip
	\\
	\emph{Seefahrerei und Hexerei (mit Reiche Ernte)}
	\\
	\smallskip
	\\
	Seeweg (Ereignis), Überfahrt (Ereignis). Geschichtenerzähler, Kleinbauer (+ Eintauschkarten), Kundschafter (als Bannkarte für die Junge Hexe), Sumpfhexe, Transformation, Weinhändler, Füllhorn, Junge Hexe, Turnier, Wahrsagerin, Weiler
	\\}}
\end{tikzpicture}
\hspace{-1cm}
\begin{tikzpicture}
	\card
	\cardstrip
	\cardbanner{banner/white.png}
	\cardicon{}
	\cardprice{}
	\cardtitle{\scriptsize{Empfohlene 10er Sätze\qquad}}
	\cardcontent{\tiny{\emph{Händler und Räuber (mit Hinterland)}
	\\
	\smallskip
	\\
	Überfall (Ereignis). Geisterwald, Hafenstadt, Page (+ Eintauschkarten), Verlorene Stadt, Weinhändler, Aufbau, Fahrender Händler, Feilscher, Fruchtbares Land, Gewürzhändler
	\\
	\smallskip
	\\
	\emph{Reisen (mit Hinterland)}
	\\
	\smallskip
	\\
	Erbschaft (Ereignis), Expedition (Ereignis). Brückentroll, Ferne Lande, Kundschafter, Riese, Wildhüter, Fernstraße, Gasthaus, Kartograph, Seidenstraße, Wegkreuzung
	\\
	\smallskip
	\\
	\emph{Friedhofspolka (mit Dark Ages)}
	\\
	\smallskip
	\\
	Almosen (Ereignis). Amulett, Gefolgsmann, Karawanenwächter, Kleinbauer (+ Eintauschkarten), Relikt, Barde, Grabräuber, Marodeur, Prozession, Schurke, Unterschlupf-Karten
	\\
	\smallskip
	\\
	\emph{Starker Verfall (mit Dark Ages)}
	\\
	\smallskip
	\\
	Verlorene Kunst (Ereignis), Wegsuche (Ereignis). Geisterwald, Rattenfänger, Transformation, Verlies, Zerstörung, Festung, Kultist, Leichenkarren, Ratten, Ritter, Unterschlupf-Karten
	\\
	\smallskip
	\\
	\emph{Verschwender (mit Die Gilden)}
	\\
	\smallskip
	\\
	Verlorene Kunst (Ereignis). Ausrüstung, Elster, Geizhals, Geschichtenerzähler, Kunsthandwerker, Arzt, Kaufmannsgilde, Meisterstück, Steinmetz, Wahrsager
	\\
	\smallskip
	\\
	\emph{Königin der Sonnenbräune (mit Die Gilden)}
	\\
	Wegsuche (Ereignis), Zuflucht (Ereignis). Duplikat, Königliche Kutsche, Königliche Münzen, Kundschafter, Rattenfänger, Berater, Herold, Leuchtenmacher, Metzger, Wandergeselle
	\\}}
\end{tikzpicture}
\hspace{1cm}
	    % Basic settings for this card set
\renewcommand{\cardcolor}{adventures}
\renewcommand{\cardextension}{Erweiterung VIII}
\renewcommand{\cardextensiontitle}{Abenteuer}

\clearpage
\newpage
\section{\cardextension \ - \cardextensiontitle \ (Rio Grande Games 2015)}

\begin{tikzpicture}
	\card
	\cardstrip
	\cardbanner{banner/goldlightbrown.png}
	\cardicon{banner/coin.png}
	\cardprice{2}
	\cardtitle{\scriptsize{Königliche Münzen}}
	\cardcontent{}
\end{tikzpicture}
\hspace{-1cm}
\begin{tikzpicture}
	\card
	\cardstrip
	\cardbanner{banner/lightbrown.png}
	\cardicon{banner/coin.png}
	\cardprice{2}
	\cardtitle{Rattenfänger}
	\cardcontent{}
\end{tikzpicture}
\hspace{-1cm}
\begin{tikzpicture}
	\card
	\cardstrip
	\cardbanner{banner/white.png}
	\cardicon{banner/coin.png}
	\cardprice{2}
	\cardtitle{Zerstörung}
	\cardcontent{}
\end{tikzpicture}
\hspace{-1cm}
\begin{tikzpicture}
	\card
	\cardstrip
	\cardbanner{banner/orange.png}
	\cardicon{banner/coin.png}
	\cardprice{3}
	\cardtitle{Amulett}
	\cardcontent{}
\end{tikzpicture}
\hspace{-1cm}
\begin{tikzpicture}
	\card
	\cardstrip
	\cardbanner{banner/orange.png}
	\cardicon{banner/coin.png}
	\cardprice{3}
	\cardtitle{Ausrüstung}
	\cardcontent{}
\end{tikzpicture}
\hspace{-1cm}
\begin{tikzpicture}
	\card
	\cardstrip
	\cardbanner{banner/orangeblue.png}
	\cardicon{banner/coin.png}
	\cardprice{3}
	\cardtitle{\scriptsize{Karawanenwächter}}
	\cardcontent{}
\end{tikzpicture}
\hspace{-1cm}
\begin{tikzpicture}
	\card
	\cardstrip
	\cardbanner{banner/lightbrown.png}
	\cardicon{banner/coin.png}
	\cardprice{3}
	\cardtitle{Kundschafter}
	\cardcontent{}
\end{tikzpicture}
\hspace{-1cm}
\begin{tikzpicture}
	\card
	\cardstrip
	\cardbanner{banner/orange.png}
	\cardicon{banner/coin.png}
	\cardprice{3}
	\cardtitle{Verlies}
	\cardcontent{}
\end{tikzpicture}
\hspace{-1cm}
\begin{tikzpicture}
	\card
	\cardstrip
	\cardbanner{banner/lightbrown.png}
	\cardicon{banner/coin.png}
	\cardprice{4}
	\cardtitle{Duplikat}
	\cardcontent{}
\end{tikzpicture}
\hspace{-1cm}
\begin{tikzpicture}
	\card
	\cardstrip
	\cardbanner{banner/white.png}
	\cardicon{banner/coin.png}
	\cardprice{4}
	\cardtitle{Elster}
	\cardcontent{}
\end{tikzpicture}
\hspace{-1cm}
\begin{tikzpicture}
	\card
	\cardstrip
	\cardbanner{banner/white.png}
	\cardicon{banner/coin.png}
	\cardprice{4}
	\cardtitle{Geizhals}
	\cardcontent{}
\end{tikzpicture}
\hspace{-1cm}
\begin{tikzpicture}
	\card
	\cardstrip
	\cardbanner{banner/white.png}
	\cardicon{banner/coin.png}
	\cardprice{4}
	\cardtitle{Hafenstadt}
	\cardcontent{}
\end{tikzpicture}
\hspace{-1cm}
\begin{tikzpicture}
	\card
	\cardstrip
	\cardbanner{banner/white.png}
	\cardicon{banner/coin.png}
	\cardprice{4}
	\cardtitle{Kurier}
	\cardcontent{}
\end{tikzpicture}
\hspace{-1cm}
\begin{tikzpicture}
	\card
	\cardstrip
	\cardbanner{banner/lightbrown.png}
	\cardicon{banner/coin.png}
	\cardprice{4}
	\cardtitle{\footnotesize{Transformation}}
	\cardcontent{}
\end{tikzpicture}
\hspace{-1cm}
\begin{tikzpicture}
	\card
	\cardstrip
	\cardbanner{banner/white.png}
	\cardicon{banner/coin.png}
	\cardprice{4}
	\cardtitle{Wildhüter}
	\cardcontent{}
\end{tikzpicture}
\hspace{-1cm}
\begin{tikzpicture}
	\card
	\cardstrip
	\cardbanner{banner/orange.png}
	\cardicon{banner/coin.png}
	\cardprice{5}
	\cardtitle{Brückentroll}
	\cardcontent{}
\end{tikzpicture}
\hspace{-1cm}
\begin{tikzpicture}
	\card
	\cardstrip
	\cardbanner{banner/lightbrowngreen.png}
	\cardicon{banner/coin.png}
	\cardprice{5}
	\cardtitle{Ferne Lande}
	\cardcontent{}
\end{tikzpicture}
\hspace{-1cm}
\begin{tikzpicture}
	\card
	\cardstrip
	\cardbanner{banner/orange.png}
	\cardicon{banner/coin.png}
	\cardprice{5}
	\cardtitle{Geisterwald}
	\cardcontent{}
\end{tikzpicture}
\hspace{-1cm}
\begin{tikzpicture}
	\card
	\cardstrip
	\cardbanner{banner/white.png}
	\cardicon{banner/coin.png}
	\cardprice{5}
	\cardtitle{\tiny{Geschichtenerzähler}}
	\cardcontent{}
\end{tikzpicture}
\hspace{-1cm}
\begin{tikzpicture}
	\card
	\cardstrip
	\cardbanner{banner/white.png}
	\cardicon{banner/coin.png}
	\cardprice{5}
	\cardtitle{\scriptsize{Kunsthandwerker}}
	\cardcontent{}
\end{tikzpicture}
\hspace{-1cm}
\begin{tikzpicture}
	\card
	\cardstrip
	\cardbanner{banner/lightbrown.png}
	\cardicon{banner/coin.png}
	\cardprice{5}
	\cardtitle{\scriptsize{Königliche Kutsche}}
	\cardcontent{}
\end{tikzpicture}
\hspace{-1cm}
\begin{tikzpicture}
	\card
	\cardstrip
	\cardbanner{banner/gold.png}
	\cardicon{banner/coin.png}
	\cardprice{5}
	\cardtitle{Relikt}
	\cardcontent{}
\end{tikzpicture}
\hspace{-1cm}
\begin{tikzpicture}
	\card
	\cardstrip
	\cardbanner{banner/white.png}
	\cardicon{banner/coin.png}
	\cardprice{5}
	\cardtitle{Riese}
	\cardcontent{}
\end{tikzpicture}
\hspace{-1cm}
\begin{tikzpicture}
	\card
	\cardstrip
	\cardbanner{banner/gold.png}
	\cardicon{banner/coin.png}
	\cardprice{5}
	\cardtitle{Schatz}
	\cardcontent{}
\end{tikzpicture}
\hspace{-1cm}
\begin{tikzpicture}
	\card
	\cardstrip
	\cardbanner{banner/orange.png}
	\cardicon{banner/coin.png}
	\cardprice{5}
	\cardtitle{Sumpfhexe}
	\cardcontent{}
\end{tikzpicture}
\hspace{-1cm}
\begin{tikzpicture}
	\card
	\cardstrip
	\cardbanner{banner/white.png}
	\cardicon{banner/coin.png}
	\cardprice{5}
	\cardtitle{\scriptsize{Verlorene Stadt}}
	\cardcontent{}
\end{tikzpicture}
\hspace{-1cm}
\begin{tikzpicture}
	\card
	\cardstrip
	\cardbanner{banner/lightbrown.png}
	\cardicon{banner/coin.png}
	\cardprice{5}
	\cardtitle{Weinhändler}
	\cardcontent{}
\end{tikzpicture}
\hspace{-1cm}
\begin{tikzpicture}
	\card
	\cardstrip
	\cardbanner{banner/orange.png}
	\cardicon{banner/coin.png}
	\cardprice{6}
	\cardtitle{Gefolgsmann}
	\cardcontent{}
\end{tikzpicture}
\hspace{-1cm}
\begin{tikzpicture}
	\card
	\cardstrip
	\cardbanner{banner/white.png}
	\cardtitle{Ereignisse (1/7)\quad}
	\cardcontent{}
\end{tikzpicture}
\hspace{-1cm}
\begin{tikzpicture}
	\card
	\cardstrip
	\cardbanner{banner/white.png}
	\cardtitle{Ereignisse (2/7)\quad}}
	\cardcontent{}
\end{tikzpicture}
\hspace{-1cm}
\begin{tikzpicture}
	\card
	\cardstrip
	\cardbanner{banner/white.png}
	\cardtitle{Ereignisse (3/7)\quad}}
	\cardcontent{}
\end{tikzpicture}
\hspace{-1cm}
\begin{tikzpicture}
	\card
	\cardstrip
	\cardbanner{banner/white.png}
	\cardtitle{Ereignisse (4/7)\quad}}
	\cardcontent{}
\end{tikzpicture}
\hspace{-1cm}
\begin{tikzpicture}
	\card
	\cardstrip
	\cardbanner{banner/white.png}
	\cardtitle{Ereignisse (5/7)\quad}}
	\cardcontent{}
\end{tikzpicture}
\hspace{-1cm}
\begin{tikzpicture}
	\card
	\cardstrip
	\cardbanner{banner/white.png}
	\cardtitle{Ereignisse (6/7)\quad}}
	\cardcontent{}
\end{tikzpicture}
\hspace{-1cm}
\begin{tikzpicture}
	\card
	\cardstrip
	\cardbanner{banner/white.png}
	\cardtitle{Ereignisse (7/7)\quad}}
	\cardcontent{}
\end{tikzpicture}
\hspace{-1cm}
\begin{tikzpicture}
	\card
	\cardstrip
	\cardbanner{banner/white.png}
	\cardicon{banner/coin.png}
	\cardprice{2}
	\cardtitle{Kleinbauer}
	\cardcontent{}
\end{tikzpicture}
\hspace{-1cm}
\begin{tikzpicture}
	\card
	\cardstrip
	\cardbanner{banner/white.png}
	\cardicon{banner/coin.png}
	\cardprice{3\textsuperscript{*}}
	\cardtitle{Soldat}
	\cardcontent{}
\end{tikzpicture}
\hspace{-1cm}
\begin{tikzpicture}
	\card
	\cardstrip
	\cardbanner{banner/white.png}
	\cardicon{banner/coin.png}
	\cardprice{4\textsuperscript{*}}
	\cardtitle{Flüchtling}
	\cardcontent{}
\end{tikzpicture}
\hspace{-1cm}
\begin{tikzpicture}
	\card
	\cardstrip
	\cardbanner{banner/white.png}
	\cardicon{banner/coin.png}
	\cardprice{5\textsuperscript{*}}
	\cardtitle{Schüler}
	\cardcontent{}
\end{tikzpicture}
\hspace{-1cm}
\begin{tikzpicture}
	\card
	\cardstrip
	\cardbanner{banner/lightbrown.png}
	\cardicon{banner/coin.png}
	\cardprice{6\textsuperscript{*}}
	\cardtitle{Lehrer}
	\cardcontent{}
\end{tikzpicture}
\hspace{-1cm}
\begin{tikzpicture}
	\card
	\cardstrip
	\cardbanner{banner/white.png}
	\cardicon{banner/coin.png}
	\cardprice{2}
	\cardtitle{Page}
	\cardcontent{}
\end{tikzpicture}
\hspace{-1cm}
\begin{tikzpicture}
	\card
	\cardstrip
	\cardbanner{banner/white.png}
	\cardicon{banner/coin.png}
	\cardprice{3\textsuperscript{*}}
	\cardtitle{Schatzsucher}
	\cardcontent{}
\end{tikzpicture}
\hspace{-1cm}
\begin{tikzpicture}
	\card
	\cardstrip
	\cardbanner{banner/white.png}
	\cardicon{banner/coin.png}
	\cardprice{4\textsuperscript{*}}
	\cardtitle{Krieger}
	\cardcontent{}
\end{tikzpicture}
\hspace{-1cm}
\begin{tikzpicture}
	\card
	\cardstrip
	\cardbanner{banner/white.png}
	\cardicon{banner/coin.png}
	\cardprice{5\textsuperscript{*}}
	\cardtitle{Held}
	\cardcontent{}
\end{tikzpicture}
\hspace{-1cm}
\begin{tikzpicture}
	\card
	\cardstrip
	\cardbanner{banner/orange.png}
	\cardicon{banner/coin.png}
	\cardprice{6\textsuperscript{*}}
	\cardtitle{Champion}
	\cardcontent{}
\end{tikzpicture}
\hspace{-1cm}
\begin{tikzpicture}
	\card
	\cardstrip
	\cardbanner{banner/white.png}
	\cardicon{}
	\cardprice{}
	\cardtitle{\scriptsize{Empfohlene 10er Sätze\qquad}}
	\cardcontent{\emph{Sanfte Einführung} (+ \underline{Ereignisse}): 
	\\
	\smallskip
	\\
	\underline{Spähtrupp}, Amulett, Duplikat, Ferne Lande, Gefolgsmann, Hafenstadt, Rattenfänger, Riese, Schatz, Verlies, Wildhüter
	\\
	\smallskip
	\\
	\emph{Profi-Einführung} (+ \underline{Ereignisse}): 
	\\
	\smallskip
	\\
	\underline{Mission}, \underline{Planung}, Elster, Geisterwald, Karawanenwächter, Kleinbauer (+ Eintausch-Karten), Königliche Münzen, Sumpfhexe, Verlorene Stadt, Weinhändler, Zerstörung, Transformation
	\\
	\smallskip
	\\
	\emph{Auf höchster Ebene} (Abenteuer + \underline{Ereignisse} + \textit{Basisspiel}):
	\\
	\smallskip
	\\
	\underline{Training}, Ausrüstung, Geizhals, Kundschafter, Verlies, Verlorene Stadt, \textit{Markt}, \textit{Miliz}, \textit{Spion}, \textit{Thronsaal}, \textit{Werkstatt}
	\\
	\smallskip
	\\
	\emph{Verzerrte Größen} (Abenteuer + \underline{Ereignisse} + \textit{Basisspiel}):
	\\
	\smallskip
	\\
	\underline{Freudenfeuer}, \underline{Überfall}, Amulett, Duplikat, Kurier, Riese, Schatz, \textit{Bürokrat}, \textit{Dieb}, \textit{Gärten}, \textit{Geldverleiher}, \textit{Hexe}
	\\}
\end{tikzpicture}
\hspace{1cm}
	    % Basic settings for this card set
\renewcommand{\cardcolor}{empires}
\renewcommand{\cardextension}{Erweiterung IX}
\renewcommand{\cardextensiontitle}{Empires}

\clearpage
\newpage
\section{\cardextension \ - \cardextensiontitle}

\begin{tikzpicture}
	\card
	\cardstrip
	\cardbanner{banner/white.png}
	\cardicon{banner/coin.png}
	\cardprice{2}
	\cardtitle{Feldlager}
	\cardcontent{Du darfst ein Gold oder ein Diebesgut aus der Hand aufdecken. Wenn du das nicht kannst oder möchtest, legst du diese Karte zur Seite und legst sie zu Beginn deiner Aufräumphase zurück in den Vorrat. Sollte dort zu diesem Zeitpunkt bereits ein Diebesgut offen liegen, muss nun erst wieder das zurückgelegte Feldlager genommen werden, bevor das Diebesgut genommen werden darf.}
\end{tikzpicture}
\hspace{-0.6cm}
\begin{tikzpicture}
	\card
	\cardstrip
	\cardbanner{banner/white.png}
	\cardicon{banner/coin.png}
	\cardprice{2}
	\cardtitle{Patrizier}
	\cardcontent{Du musst die oberste Karte deines Nachziehstapels aufdecken. Wenn der Stapel aufgebraucht ist, mischst du deinen Ablagestapel und legst ihn als Nachziehstapel bereit. Wenn auch dort keine Karten liegen, erhältst du nichts.

	\medskip

	Nur eine Karte, die mehr als 5 Geld kostet, darfst du auf die Hand nehmen. Ob die Karte außerdem noch Kosten in Form von Schulden aufweist, ist dabei unerheblich (z. B. Reichtum darf auf die Hand genommen werden, Stadtviertel nicht).}
\end{tikzpicture}
\hspace{-0.6cm}
\begin{tikzpicture}
	\card
	\cardstrip
	\cardbanner{banner/white.png}
	\cardicon{banner/coin.png}
	\cardprice{2}
	\cardtitle{Siedler}
	\cardcontent{Auch wenn du weißt, dass sich kein Kupfer in deinem Ablagestapel befindet, darfst du ihn ansehen. Du musst kein Kupfer auf die Hand nehmen, wenn du das nicht möchtest.}
\end{tikzpicture}
\hspace{-0.6cm}
\begin{tikzpicture}
	\card
	\cardstrip
	\cardbanner{banner/white.png}
	\cardicon{banner/coin.png}
	\cardprice{3}
	\cardtitle{\footnotesize{Bauernmarkt}}
	\cardcontent{Diese Karte beinhaltet den neuen Typ Sammlung, d. h. hier kommen die Siegpunkt-Marker zum Einsatz.

	\medskip

	Wenn diese Karte das erste Mal ausgespielt wird, legt der Spieler einen Siegpunkt-Marker auf den Bauernmarkt-Vorratsstapel und erhält dann +1 Geld für den gerade gelegten Marker. Wird die Karte zum zweiten, dritten und vierten Mal ausgespielt, legt der Spieler jeweils einen weiten Marker auf den Stapel und erhält +2 Geld, +3 Geld bzw. +4 Geld, egal welcher Spieler die vorherigen Marker auf den Stapel gelegt hat. Wird die Karte danach erneut ausgespielt, nimmt der Spieler die 4 Siegpunkt-Marker (dafür aber kein Geld) und muss den ausgespielten Bauernmarkt entsorgen.

	\medskip

	Danach beginnt der Vorgang wieder von vorn und wird fortgesetzt, auch wenn der Vorratsstapel leer ist.}
\end{tikzpicture}
\hspace{-0.6cm}
\begin{tikzpicture}
	\card
	\cardstrip
	\cardbanner{banner/white.png}
	\cardicon{banner/coin.png}
	\cardprice{3}
	\cardtitle{Gladiator}
	\cardcontent{Wenn du mindestens 1 Handkarte hast, musst du diese aufdecken. Wenn dein linker Mitspieler keine Karte mit gleichem Namen aufdecken kann oder will (z. B. auch, wenn du keine Handkarte aufdecken konntest, weil du keine hast), erhältst du zusätzlich +1 Geld. Sind noch Karten auf dem Gladiator-Vorratsstapel vorhanden, musst du eine entsorgen. Deckt der Mitspieler eine Karte mit gleichem Namen auf, erhältst du nur +2 Geld und darfst keinen Gladiator entsorgen.}
\end{tikzpicture}
\hspace{-0.6cm}
\begin{tikzpicture}
	\card
	\cardstrip
	\cardbanner{banner/white.png}
	\cardicon{banner/coin.png}
	\cardprice{3}
	\cardtitle{Katapult}
	\cardcontent{Wenn du mindestens 1 Handkarte hast, musst du auch eine entsorgen. Kostet die entsorgte Karte 3 Geld oder mehr, nimmt sich jeder Mitspieler (beginnend bei deinem linken Nachbarn) einen Fluch. Karten mit Schulden kosten nur dann 3 Geld oder mehr, wenn sie zusätzlich zu etwaigen Schulden-Kosten mindestens 3 Geld kosten. Ist die entsorgte Karte eine Geldkarte muss jeder Mitspieler - unabhängig von den Kosten der Karte - seine Handkarten auf 3 Geld reduzieren.}
\end{tikzpicture}
\hspace{-0.6cm}
\begin{tikzpicture}
	\card
	\cardstrip
	\cardbanner{banner/white.png}
	\cardicon{banner/coin.png}
	\cardprice{3}
	\cardtitle{\footnotesize{Wagenrennen}}
	\cardcontent{Nimm deine aufgedeckte Karte nach dem Vergleich der Kosten mit der aufgedeckten Karte deines linken Mitspielers auf die Hand. Der Mitspieler legt seine aufgedeckte Karte zurück auf den Nachziehstapel.

	\medskip

	Kosten beide Karten gleich viel oder kostet die Karte des Mitspielers mehr, erhältst du nichts. Kostet deine Karte mehr erhältst du +1 Geld und +1 Siegpunkt-Marker. Hast entweder du oder dein linker Mitspieler (auch nach dem eventuellen Mischen des Ablagestapels) keine Karte zum Aufdecken, erhältst du nichts.}
\end{tikzpicture}
\hspace{-0.6cm}
\begin{tikzpicture}
	\card
	\cardstrip
	\cardbanner{banner/orange.png}
	\cardicon{banner/coin.png}
	\cardprice{3}
	\cardtitle{Zauberin}
	\cardcontent{Spieler, die mit einer Reaktionskarte wie dem Burggraben (aus dem Basisspiel) reagieren möchten, müssen dies tun, sobald die Zauberin ausgespielt wurde, auch wenn der Angriff sie erst in ihrem nächsten Zug betrifft.

	\medskip

	Jeder Mitspieler erhält in seinem nächsten Zug für die erste gespielte Aktionskarte +1 Karte sowie +1 Aktion, darf aber den eigentlichen Effekt der Karte beim Ausspielen nicht durchführen. Anweisungen, die sich auf einen anderen Zeitpunkt im Spiel beziehen (z. B. die beim Kauf der Karte zum Tragen kommen), werden nicht beeinflusst.

	\medskip

	Um anzuzeigen, dass die erste ausgespielte Aktionskarte von der Zauberin beeinflusst wird, empfehlen wir, diese beim Ausspielen quer auszulegen. Karten, die bereits ausgespielt wurden (z. B. Dauerkarten wie das Archiv), werden zu Beginn des Zuges normal abgehandelt und nicht von der Zauberin beeinflusst. Spielt ein Spieler in seiner Aktionsphase keine Aktionskarte aus, dafür aber in seiner Kaufphase eine Krone (kombinierte Aktions- und Geldkarte), kommt der Effekt der Zauberin zum Tragen, da es sich um eine Aktionskarte handelt, auch wenn diese in der Kaufphase ausgespielt wurde. Normalerweise kann der Spieler die +1 Aktion zu diesem Zeitpunkt nicht nutzen, es sei denn, er kauft zum Beispiel eine Villa.}
\end{tikzpicture}
\hspace{-0.6cm}
\begin{tikzpicture}
	\card
	\cardstrip
	\cardbanner{banner/gold.png}
	\cardicon{banner/coin.png}
	\cardprice{4}
	\cardtitle{Felsen}
	\cardcontent{Wenn du diese Karte in deiner Kaufphase nimmst oder entsorgst, nimm ein Silber und lege es auf deinen Nachziehstapel. Wenn du diese Karte zu einem anderen Zeitpunkt (auch während des Zuges eines anderen Spielers) nimmst oder entsorgst, nimm ein Silber auf die Hand.}
\end{tikzpicture}
\hspace{-0.6cm}
\begin{tikzpicture}
	\card
	\cardstrip
	\cardbanner{banner/white.png}
	\cardicon{banner/coin.png}
	\cardprice{4}
	\cardtitle{Opfer}
	\cardcontent{Wenn die entsorgte Karte eine kombinierte Karte ist, erhältst du die Boni aller entsprechenden Typen dieser Karte. Entsorgst du eine Karte, die keinem der angegebenen Typen entspricht (z. B. einen Fluch), erhältst du nichts.}
\end{tikzpicture}
\hspace{-0.6cm}
\begin{tikzpicture}
	\card
	\cardstrip
	\cardbanner{banner/white.png}
	\cardicon{banner/coin.png}
	\cardprice{4}
	\cardtitle{Tempel}
	\cardcontent{Es dürfen nur Karten mit unterschiedlichem Namen entsorgt werden, z. B. ein Kupfer und ein Anwesen.

	\medskip

	Auch wenn der Tempel-Vorratsstapel leer ist, legst du einen Siegpunkt-Marker auf den leeren Platz. Das kann relevant werden, wenn durch Anweisungen auf anderen Karten ein Tempel in den Vorrat zurückgelegt wird (z. B. durch den Botschafter aus Seaside).

	\medskip

	Wenn du einen Tempel nimmst, nimmst du auch alle Siegpunkt-Marker, die zu diesem Zeitpunkt auf dem Vorratsstapel liegen.}
\end{tikzpicture}
\hspace{-0.6cm}
\begin{tikzpicture}
	\card
	\cardstrip
	\cardbanner{banner/white.png}
	\cardicon{banner/coin.png}
	\cardprice{4}
	\cardtitle{Villa}
	\cardcontent{Wenn du diese Karte in deiner Aktionsphase nimmst (z. B. durch die Ingenieurin), nimm sie sofort auf die Hand und erhalte +1 Aktion. Dadurch kannst du z. B. die gerade genommene Villa sofort ausspielen. Wenn du diese Karte in deiner Kaufphase nimmst (z. B. indem du sie kaufst), nimm sie auf die Hand und kehre sofort in die Aktionsphase zurück, wo du +1 Aktion hast. Hast du die Aktionsphase erneut komplett abgeschlossen, kehrst du wieder zur Kaufphase zurück. Hier kannst du weitere Geldkarten ausspielen (und z. B. die Arena kommt wieder zum Tragen). Wenn du diese Karte während des Zuges eines Mitspielers nimmst, nimmst du die Karte auf die Hand und erhältst zwar +1 Aktion, kannst diese aber nicht nutzen, da es nicht dein Zug ist. Es ist möglich, mehrmals pro Zug (z. B. durch das Nehmen mehrerer Villen) in die Aktionsphase zurückzukehren. Dies bedeutet aber nicht, dass du an den \enquote{Beginn deines Zuges} zurückkehrst. Anweisungen, die sich darauf beziehen, haben keine Auswirkung.}
\end{tikzpicture}
\hspace{-0.6cm}
\begin{tikzpicture}
	\card
	\cardstrip
	\cardbanner{banner/orange.png}
	\cardicon{banner/coin.png}
	\cardprice{5}
	\cardtitle{Archiv}
	\cardcontent{Lege die obersten drei Karten deines Nachziehstapels zur Seite und schau sie dir an. Nimm eine der Karten sofort auf die Hand und lege die anderen Karten unter dieses Archiv. Spielst du zwei Archive, lege die Karten für die nächsten Züge unter das jeweils ausgespielte Archiv. Hast du nicht genügend Karten, um drei Karten zur Seite zu legen, legst du nur so viele wie möglich zur Seite. Das Archiv wird in dem Spielzug abgelegt, in dem die letzte zur Seite gelegte Karte des jeweiligen Archivs auf die Hand genommen wurde.}
\end{tikzpicture}
\hspace{-0.6cm}
\begin{tikzpicture}
	\card
	\cardstrip
	\cardbanner{banner/gold.png}
	\cardicon{banner/coin.png}
	\cardprice{5}
	\cardtitle{Diebesgut}
	\cardcontent{Nimm dir jedes Mal, wenn du diese Karte spielst, einen Siegpunkt-Marker und lege ihn bei dir ab.}
\end{tikzpicture}
\hspace{-0.6cm}
\begin{tikzpicture}
	\card
	\cardstrip
	\cardbanner{banner/white.png}
	\cardicon{banner/coin.png}
	\cardprice{5}
	\cardtitle{Emsiges Dorf}
	\cardcontent{Du darfst deinen Ablagestapel auch dann durchsehen, wenn du weißt, dass du keine Siedler darin hast. Du darfst die Reihenfolge der Karten in deinem Ablagestapel nicht verändern.}
\end{tikzpicture}
\hspace{-0.6cm}
\begin{tikzpicture}
	\card
	\cardstrip
	\cardbanner{banner/white.png}
	\cardicon{banner/coin.png}
	\cardprice{5}
	\cardtitle{Forum}
	\cardcontent{Wenn du diese Karte kaufst, erhältst du +1 Kauf. Du kannst beispielsweise mit 13 Geld und nur einem freien Kauf, zuerst diese Karte kaufen und dann mit dem zusätzlichen Kauf noch eine Provinz.}
\end{tikzpicture}
\hspace{-0.6cm}
\begin{tikzpicture}
	\card
	\cardstrip
	\cardbanner{banner/white.png}
	\cardicon{banner/coin.png}
	\cardprice{5}
	\cardtitle{Gärtnerin}
	\cardcontent{Ist diese Karte im Spiel und du nimmst eine Punktekarte – egal in welcher Spielphase – nimmst du dir einen Siegpunkt-Marker und legst ihn bei dir ab. Wenn du mehrere Punktekarten nimmst, nimmst du dir für jede genommene Punktekarte einen Siegpunkt-Marker. Hast du mehrere Gärtnerinnen im Spiel, nimmst du dir für jede Gärtnerin pro genommener Punktekarte einen Siegpunkt-Marker.

	\medskip

	Wenn du z. B. eine Gärtnerin auf eine Krone spielst, befindet sich die Gärtnerin trotzdem nur einmal im Spiel und du darfst dir pro genommener Punktekarte nur einen Siegpunkt-Marker nehmen.}
\end{tikzpicture}
\hspace{-0.6cm}
\begin{tikzpicture}
	\card
	\cardstrip
	\cardbanner{banner/white.png}
	\cardicon{banner/coin.png}
	\cardprice{5}
	\cardtitle{\footnotesize{Handelsplatz}}
	\cardcontent{Zu den Aktionskarten, die du zu diesem Zeitpunkt im Spiel hast zählen alle Aktionskarten, die du ausgespielt hast, Dauerkarten, die sich aus vergangenen Zügen im Spiel befinden und Reservekarten (aus Abenteuer), die du in diesem Zug bereits aufgerufen hast. Wenn du diese Karte außerhalb deines Zuges nimmst, hast du keine Aktionskarten im Spiel und du darfst dir keine Siegpunkt-Marker nehmen.}
\end{tikzpicture}
\hspace{-0.6cm}
\begin{tikzpicture}
	\card
	\cardstrip
	\cardbanner{banner/whitegold.png}
	\cardicon{banner/coin.png}
	\cardprice{5}
	\cardtitle{Krone}
	\cardcontent{Diese Karte ist eine kombinierte Aktions- und Geldkarte. Wenn du sie in deiner Aktionsphase ausspielst, darfst du eine Aktionskarte von deiner Hand wählen und ausspielen. Du nimmst die gewählte Karte nicht wieder auf die Hand, sondern spielst die Aktion ein zweites Mal. Dafür benötigst du keine weiteren Aktionen. Wählst du eine Krone, musst du diese auch als Aktionskarte ausspielen (und dann darfst du bis zu zwei weitere Aktionskarten jeweils zweimal spielen).

	\medskip

	Spielst du diese Karte in deiner Aktionsphase als Geldkarte aus (z. B. durch den Geschichtenerzähler aus Abenteuer), darfst du trotzdem eine Aktionskarte zweimal ausspielen.

	\medskip

	Spielst du diese Karte in deiner Kaufphase, darfst du eine beliebige Geldkarte von deiner Hand wählen, sie ausspielen und zweimal ausführen. Wählst du eine Krone, spielst du diese aus und dann eine weitere Geldkarte von der Hand zweimal und dann noch eine Geldkarte zweimal.}
\end{tikzpicture}
\hspace{-0.6cm}
\begin{tikzpicture}
	\card
	\cardstrip
	\cardbanner{banner/white.png}
	\cardicon{banner/coin.png}
	\cardprice{5}
	\cardtitle{Legionär}
	\cardcontent{Mitspieler, die auf das Ausspielen dieser Karte mit einer Reaktionskarte reagieren möchten, müssen dies tun, bevor du dich entscheidest, ob du ein Gold aufdeckst oder nicht.

	\medskip

	Mitspieler, die bereits zwei oder weniger Karten auf der Hand haben, müssen keine Karte ablegen, müssen gleichwohl aber eine Karte ziehen.}
\end{tikzpicture}
\hspace{-0.6cm}
\begin{tikzpicture}
	\card
	\cardstrip
	\cardbanner{banner/gold.png}
	\cardicon{banner/coin.png}
	\cardprice{5}
	\cardtitle{Vermögen}
	\cardcontent{\emph{Errata:} Der Geldwert oben links und rechts auf der Karte muss 6 Geld lauten, nicht 5 Geld.

	\medskip

	Diese Karte ist eine Geldkarte mit zusätzlichen Anweisungen. Sie hat den Wert 6 Geld. Außerdem erhältst du +1 Kauf.

	\medskip

	Wenn du diese Karte ablegst (in der Regel in deiner Aufräumphase), nimm 6 Schulden-Marker vom Vorrat. Dann kannst du sofort beliebig viele Schulden-Marker (auch mehr als die 6 Schulden-Marker, die du durch das Ablegen dieser Karte erhalten hast) zurückzahlen. Wenn du diese Karte nicht ablegst (z. B. wenn du sie stattdessen entsorgst), erhältst du keine Schulden-Marker. Wenn du diese Karte zweimal ausgespielt hast (z. B. durch eine Krone), erhältst du trotzdem nur 6 Schulden-Marker, da du nur eine Karte ablegst.}
\end{tikzpicture}
\hspace{-0.6cm}
\begin{tikzpicture}
	\card
	\cardstrip
	\cardbanner{banner/white.png}
	\cardicon{banner/coin.png}
	\cardprice{5}
	\cardtitle{Wilde Jagd}
	\cardcontent{Wählst du die erste Option, lege einen Siegpunkt-Marker vom Vorrat auf den Wilde-Jagd-Vorratsstapel.

	\medskip

	Wählst du die zweite Option und der Anwesen-Vorratsstapel ist leer (d. h. du kannst dir kein Anwesen nehmen), darfst du dir die Siegpunkt-Marker vom Wilde-Jagd-Vorratsstapel nicht nehmen. Du darfst aber diese Option trotzdem wählen.

	\medskip

	Ist der Wilde-Jagd-Vorratsstapel leer, funktioniert das Ausspielen dieser Karte trotzdem in der beschriebenen Weise weiter. Nutzt die Platzhalterkarte, um den Vorratsstapel zu markieren.}
\end{tikzpicture}
\hspace{-0.6cm}
\begin{tikzpicture}
	\card
	\cardstrip
	\cardbanner{banner/gold.png}
	\cardicon{banner/coin.png}
	\cardprice{5}
	\cardtitle{Zauber}
	\cardcontent{Wenn du diese Karte ausspielst und dich für die zweite Option entscheidest, darfst du (musst aber nicht) sofort, wenn du die \emph{nächste} Karte in deinem Zug kaufst, eine Karte mit anderem Namen nehmen, die \emph{exakt so viel} kostet, wie die gekaufte Karte. Dann erst nimmst du die gekaufte Karte. Das kann wichtig bei Karten sein, die Anweisungen beim Nehmen einer Karte beinhalten.

	\medskip

	Spielst du mehrere Zauber in einem Zug, darfst du dir für die nächste gekaufte Karte mehrere Karten mit anderem Namen als die gekaufte Karte aber mit den gleichen Kosten nehmen. Die Karten, die du nimmst müssen zwar einen anderen Namen
	als die gekaufte Karte haben, dürfen aber untereinander alle den gleichen Namen haben.}
\end{tikzpicture}
\hspace{-0.6cm}
\begin{tikzpicture}
	\card
	\cardstrip
	\cardbanner{banner/white.png}
	\cardicon{banner/hex.png}
	\cardprice{\textcolor{white}{4}}
	\cardtitle{Ingenieurin}
	\cardcontent{Du darfst dir keine Karte nehmen, die mehr als 4 Geld kostet oder Schulden in den Kosten hat. Nimm die gewählte Karte.

	\medskip

	Dann darfst du diese Ingenieurin entsorgen. Wenn du das tust, nimm eine weitere Karte, die bis zu 4 Geld kostet. Dies kann die gleiche Karte wie die erste sein oder eine andere.}
\end{tikzpicture}
\hspace{-0.6cm}
\begin{tikzpicture}
	\card
	\cardstrip
	\cardbanner{banner/white.png}
	\cardicon{banner/hex.png}
	\cardprice{\textcolor{white}{8}}
	\cardtitle{\miniscule{Königlicher Schmied}}
	\cardcontent{Du musst, nachdem du 5 Karten nachgezogen hast, alle deine Handkarten vorzeigen und jedes Kupfer, das du zu diesem Zeitpunkt auf der Hand hast, ablegen.}
\end{tikzpicture}
\hspace{-0.6cm}
\begin{tikzpicture}
	\card
	\cardstrip
	\cardbanner{banner/white.png}
	\cardicon{banner/hex.png}
	\cardprice{\textcolor{white}{8}}
	\cardtitle{Lehnsherr}
	\cardcontent{Wähle eine Karte vom Vorrat, die zu diesem Zeitpunkt bis zu 5 Geld kostet, d. h. du darfst keine Karte eines leeren Stapels, eine nicht sichtbare Karte eines gemischten Stapels oder eine Karte eines Nicht-Vorratsstapels wählen.

	\medskip

	Behandle nun den ausgespielten Lehnsherr, wie die gewählte Karte (und nicht mehr als Lehnsherr) – bis sie nicht mehr im Spiel ist. Das heißt du befolgst alle Anweisungen der anderen Karte. Auch nimmt der Lehnsherr den Namen, die Kosten und den Typ der gewählten Karte an, bis er nicht mehr im Spiel ist. Als Dauerkarte bleibt dieser Lehnsherr ebenso im Spiel, wie er als Reservekarte (aus Abenteuer) zur Seite gelegt wird. Spielst du diesen Lehnsherr auf einen Thronsaal (aus dem Basisspiel), wählst du beim ersten Ausspielen die Karte, die dieser Lehnsherr ab sofort ist – beim zweiten Ausspielen ist er damit wieder genau diese Karte – du darfst keine andere Karte wählen. Erst mit dem Ausspielen des Lehnsherrn nimmt er Typ und Namen der gewählten Karte an – d. h. du darfst ihn nicht als Krone in deiner Kaufphase spielen, da er selbst keine Geldkarte ist und nicht in der Kaufphase ausgespielt werden darf.}
\end{tikzpicture}
\hspace{-0.6cm}
\begin{tikzpicture}
	\card
	\cardstrip
	\cardbanner{banner/white.png}
	\cardicon{banner/hex.png}
	\cardprice{\textcolor{white}{8}}
	\cardtitle{Stadtviertel}
	\cardcontent{\emph{Errata:} Der Kartentext sollte lauten: \enquote{+2 Aktionen}

	\medskip

	Du erhältst 2 zusätzliche Aktionen. Außerdem deckst du deine Handkarten auf und ziehst anschließend eine Karte von deinem Nachziehstapel für jede Aktionskarte, die du aufgedeckt hast.

	\medskip

	Kombinierte Karten, die den Kartentyp Aktion besitzen, wie z. B. die Krone, sind auch Aktionskarten.}
\end{tikzpicture}
\hspace{-0.6cm}
\begin{tikzpicture}
	\card
	\cardstrip
	\cardbanner{banner/white.png}
	\cardicon{banner/coin.png}
	\cardprice{8}
	\cardiconaddition{banner/hex.png}
	\cardpriceaddition{\textcolor{white}{8}}
	\cardtitle{\quad Reichtum}
	\cardcontent{Es werden nur alle Geld verdoppelt, die du vor dem Ausspielen dieser Karte ausgespielt hast und nur, wenn du in diesem Zug noch keinen Reichtum ausgespielt hast. Für jedes weitere Ausspielen eines Reichtums erhältst du nur +1 Kauf.}
\end{tikzpicture}
\hspace{-0.6cm}
\begin{tikzpicture}
	\card
	\cardstrip
	\cardbanner{banner/green.png}
	\cardtitle{Schlösser (1/2)\quad}
	\cardcontent{Der Schloss-Stapel ist ein gemischter Vorratsstapel. Alle Schlösser werden nach Kosten sortiert auf dem Vorratsstapel bereitgelegt (die teuerste zuunterst). Es darf immer nur die oberste Karte gekauft oder genommen werden.

	\medskip

	\emph{Bescheidenes Schloss:} Spielst du sie in deiner Kaufphase aus, ist sie 1 Geld wert. Bei Spielende erhältst du pro Karte, die den Typ Schloss beinhaltet (inklusive dieser Karte), einen Siegpunkt-Marker.

	\medskip

	\emph{Verfallendes Schloss:} Wenn du diese Karte während des Spiels nimmst, nimm dir einen Siegpunkt-Marker sowie ein Silber vom Vorrat. Wenn du diese Karte während des Spiels entsorgst, nimm dir einen weiteren Siegpunkt-Marker sowie ein Silber vom Vorrat. Bei Spielende ist diese Karte 1 Siegpunkt wert.

	\medskip

	\emph{Kleines Schloss:} Spielst du sie in deiner Aktionsphase aus, entsorge dieses Kleine Schloss oder eine andere Schloss-Karte aus deiner Hand. Wenn du das tust, nimm dir die Schloss-Karte vom Vorratsstapel, die zu diesem Zeitpunkt oben liegt. Dies kann eine teurere sein, als die, die du entsorgst. Du musst die Kosten nicht bezahlen. Bei Spielende ist diese Karte 2 Siegpunkte wert.}
\end{tikzpicture}
\hspace{-0.6cm}
\begin{tikzpicture}
	\card
	\cardstrip
	\cardbanner{banner/green.png}
	\cardtitle{Schlösser (2/2)\quad}
	\cardcontent{\emph{Spukschloss:} Wenn du diese Karte während deines Zuges nimmst (kaufst oder auf andere Art und Weise nimmst), nimm dir ein Gold vom Vorrat. Ist kein Gold mehr im Vorrat, erhältst du nichts. Außerdem (egal ob du ein Gold nehmen kannst oder nicht) müssen alle Mitspieler mit 5 oder mehr Handkarten 2 Handkarten auf ihren Nachziehstapel zurücklegen. Da diese Karte keine Angriffskarte ist, dürfen die Mitspieler keine Reaktionskarte spielen. Bei Spielende ist diese Karte 2 Siegpunkte wert.

	\smallskip

	\emph{Reiches Schloss:} Spielst du sie in deiner Aktionsphase aus, lege beliebig viele Punktekarten (auch ggf. kombinierte) aus deiner Hand ab. Pro abgelegter Karte erhältst du +2 Geld. Bei Spielende ist diese Karte 3 Siegpunkte wert.

	\smallskip

	\emph{Ausgedehntes Schloss:} Wenn du diese Karte kaufst oder auf andere Art und Weise nimmst, nimm ein Herzogtum oder drei Anwesen. Bei Spielende ist diese Karte 4 Siegpunkte wert.

	\smallskip

	\emph{Prunkschloss:} Wenn du diese Karte kaufst oder auf andere Art und Weise nimmst, zeige deine Handkarten vor. Nimm einen Siegpunkt-Marker vom Vorrat für jede Punktekarte (auch ggf. kombinierte), die du zu diesem Zeitpunkt auf der Hand oder im Spiel hast. Bei Spielende ist diese Karte 5 Siegpunkte wert.

	\smallskip

	\emph{Königsschloss:} Bei Spielende erhältst du pro Karte, die den Typ Schloss beinhaltet (inklusive dieser Karte) 2 Siegpunkte.}
\end{tikzpicture}
\hspace{-0.6cm}
\begin{tikzpicture}
	\card
	\cardstrip
	\cardbanner{banner/white.png}
	\cardtitle{Katapult/Felsen\qquad}
	\cardcontent{Spielvorbereitung: Legt auf diese Karte 5 Felsen und oben darauf 5 Katapulte.

	\bigskip

	Es darf immer nur die oberste Karte des Stapels genommen oder gekauft werden.}
\end{tikzpicture}
\hspace{-0.6cm}
\begin{tikzpicture}
	\card
	\cardstrip
	\cardbanner{banner/white.png}
	\cardtitle{\scriptsize{Gladiator/Reichtum}\qquad}
	\cardcontent{Spielvorbereitung: Legt auf diese Karte 5 Reichtum und oben darauf 5 Gladiatoren. 

	\bigskip

	Es darf immer nur die oberste Karte des Stapels genommen oder gekauft werden.}
\end{tikzpicture}
\hspace{-0.6cm}
\begin{tikzpicture}
	\card
	\cardstrip
	\cardbanner{banner/white.png}
	\cardtitle{\scriptsize{Siedler/Emsiges Dorf}\qquad}
	\cardcontent{Spielvorbereitung: Legt auf diese Karte 5 Emsige Dörfer und oben darauf 5 Siedler. 

	\bigskip

	Es darf immer nur die oberste Karte des Stapels genommen oder gekauft werden.}
\end{tikzpicture}
\hspace{-0.6cm}
\begin{tikzpicture}
	\card
	\cardstrip
	\cardbanner{banner/white.png}
	\cardtitle{\tiny{Patrizier/Handelsplatz}\qquad}
	\cardcontent{Spielvorbereitung: Legt auf diese Karte 5 Handelsplätze und oben darauf 5 Patrizier. 

	\bigskip

	Es darf immer nur die oberste Karte des Stapels genommen oder gekauft werden.}
\end{tikzpicture}
\hspace{-0.6cm}
\begin{tikzpicture}
	\card
	\cardstrip
	\cardbanner{banner/white.png}
	\cardtitle{\scriptsize{Feldlager/Diebesgut}\qquad}
	\cardcontent{Spielvorbereitung: Legt auf diese Karte 5 Diebesgut und oben darauf 5 Feldlager. 

	\bigskip

	Es darf immer nur die oberste Karte des Stapels genommen oder gekauft werden.}
\end{tikzpicture}
\hspace{-0.6cm}
\begin{tikzpicture}
	\card
	\cardstrip
	\cardbanner{banner/white.png}
	\cardtitle{Ereignisse (1/3)\quad}
	\cardcontent{\emph{Aufstieg:} Wenn du keine Aktionskarte entsorgst passiert nichts weiter.

	\medskip

	\emph{Erforschen:} Jeder Erwerb eines Erforschen gibt dir den Kauf zurück, den du für den Erwerb benötigt hast. Mit 7 Geld und 1 Kauf kannst du zum Beispiel 2 Erforschen erwerben und dann eine Karte kaufen oder ein Ereignis für 3 Geld erwerben.

	\medskip

	\emph{Steuer:} Auf jeden Vorratsstapel (d.h. alle Königreichkarten, Fluchkarten und Basiskarten, nicht Ereignisse und Landmarken) wird in der Spielvorbereitung 1 Schulden-Marker gelegt. Spieler, die eine Karte von einem Stapel kaufen, auf dem Schulden-Marker liegen, müssen alle Marker des Stapels nehmen. Nimmt ein Spieler eine Karte auf andere Art und Weise (d.h. er kauft sie nicht), werden eventuelle Schulden-Marker auf die nächste Karte des Vorratsstapels gelegt. Wenn du dieses Ereignis erwirbst, legst du 2 Schulden-Marker auf einen beliebigen Vorratsstapel – egal ob dort zu diesem Zeitpunkt bereits Schulden-Marker liegen oder nicht.

	\medskip

	\emph{Bankett:} Du kannst dieses Ereignis auch kaufen, wenn der Kupfer-Vorratsstapel aufgebraucht ist.}
\end{tikzpicture}
\hspace{-0.6cm}
\begin{tikzpicture}
	\card
	\cardstrip
	\cardbanner{banner/white.png}
	\cardtitle{Ereignisse (2/3)\quad}
	\cardcontent{\emph{Versalztes Land:} Wenn die entsorgte Karte eine Anweisung beinhaltet, die eintritt wenn diese Karte entsorgt wird, musst du diese Anweisung ausführen.

	\medskip

	\emph{Ritual:} Wenn du keinen Fluch nehmen kannst (z.B. weil der Vorratsstapel leer ist), passiert nichts. Es werden nur die Geld-Kosten gezählt – für Schulden-Kosten oder Trank-Kosten (aus Alchemisten) erhältst du nichts.

	\medskip

	\emph{Glücksfall:} Wenn weniger als 3 Gold im Vorrat sind, nimm dir die restlichen Gold.

	\medskip

	\emph{Eroberung:} Pro Silber, das du in diesem Zug genommen hast (inklusive der 2 Silber durch diese Karte), nimm dir einen Siegpunkt-Marker vom Vorrat. Dies ist kumulativ. Erwirbst du z.B. eine Eroberung und erhältst dafür 2 Siegpunkt-Marker (für die beiden Silber durch diese Karte) und dann noch eine Eroberung, für die du 2 Silber nehmen kannst, erhältst du für die zweite Eroberung schon 4 Siegpunkt-Marker. Sind nicht genügend Silber im Vorrat, nimmst du dir so viele wie möglich. Dann erhältst du aber auch entsprechend weniger Siegpunkt-Marker.

	\medskip

	\emph{Beherrschen:} Ist der Provinz-Vorratsstapel leer oder du kannst aus einem anderen Grund keine Provinz nehmen, hat dieses Ereignis keine Auswirkung.}
\end{tikzpicture}
\hspace{-0.6cm}
\begin{tikzpicture}
	\card
	\cardstrip
	\cardbanner{banner/white.png}
	\cardtitle{Ereignisse (3/3)\quad}
	\cardcontent{\emph{Hochzeit:} Den Siegpunkt-Marker nimmst du in jedem Fall – auch wenn der Gold-Vorratsstapel leer ist.

	\medskip

	\emph{Siegeszug:} Wenn du ein Anwesen nimmst, nimmst du für jede Karte, die du in diesem Zug bereits genommen hast (inklusive dem Anwesen jedoch nicht für Ereignisse), einen Siegpunkt-Marker. Wenn du kein Anwesen nehmen kannst (z.B. weil der Vorratsstapel leer ist), passiert nichts.

	\medskip

	\emph{Schlacht:} Du kannst dieses Ereignis auch erwerben, wenn der Herzogtum-Vorratsstapel leer ist. Die bis zu 5 ausgewählten Karten verbleiben in deinem Ablagestapel. Die restlichen Karten mischst du in deinen Nachziehstapel.

	\medskip

	\emph{Spende:} Befinden sich unter den entsorgten Karten welche, die Anweisungen beinhalten, die beim Entsorgen ausgeführt werden, musst du diese ausführen, bevor du die restlichen Karten mischst. Die Spende wird erst nach dem Zug, in dem sie erworben wird, ausgeführt (d.h. zwischen zwei Zügen). Damit hat zum Beispiel die Besessenheit (aus Alchemisten) auf diese Anweisung keine Auswirkung.}
\end{tikzpicture}
\hspace{-0.6cm}
\begin{tikzpicture}
	\card
	\cardstrip
	\cardbanner{banner/green.png}
	\cardtitle{\footnotesize{Landmarken (1/8)}\quad}
	\cardcontent{\emph{Aquädukt:} Wenn du eine Geldkarte von einem Vorratsstapel nimmst, auf dem ein oder mehrere Siegpunkt-Marker liegen (auch ggf. kombinierte Karten oder Kupfer, wenn dort durch Anweisungen auf Karten oder Ereignissen Siegpunkt-Marker platziert wurden), nimm einen Siegpunkt-Marker und lege ihn hierher auf das Aquädukt. Wenn du eine Punktekarte (auch ggf. kombinierte) nimmst, nimm dir alle Siegpunkt-Marker, die zu diesem Zeitpunkt hier auf dem Aquädukt liegen. Wenn du eine kombinierte Geld- und Punktekarte nimmst, kannst du dich entscheiden, in welcher Reihenfolge du die Anweisungen ausführst. 

	\emph{Errata:} Der Kartentext sollte lauten: \enquote{Wenn du ein\emph{e} Geld\emph{karte} nimmst, ...}

	\medskip

	\emph{Arena:} Beginnst du (z.B. durch die Villa) in deinem Zug mehrfach mit deiner Kaufphase, kannst du die Arena mehrfach nutzen.

	\medskip

	\emph{Badehaus:} Egal ob du eine Karte kaufst oder auf andere Art und Weise nimmst (bzw. nehmen musst), erhältst du in diesem Fall keine Siegpunkt-Marker vom Badehaus. Wer ein Ereignis erwirbt, nimmt damit keine Karte und kann, insofern keine andere Karte genommen wurde, 2 Siegpunkt- Marker von hier nehmen.}
\end{tikzpicture}
\hspace{-0.6cm}
\begin{tikzpicture}
	\card
	\cardstrip
	\cardbanner{banner/green.png}
	\cardtitle{\footnotesize{Landmarken (2/8)}\quad}
	\cardcontent{\emph{Basilika:} Für jede Karte die du kaufst, nimmst du 2 Siegpunkt-Marker von der Basilika, falls du zu diesem Zeitpunkt mindestens 2 Geld ausgespielt aber noch nicht verbraucht hast. Hast du beispielsweise 4 Geld und 3 Käufe, kannst du ein Kupfer kaufen (4 Geld übrig), dir 2 Siegpunkt-Marker nehmen, ein Anwesen kaufen (2 Geld übrig), dir 2 Siegpunkt-Marker nehmen und ein weiteres Anwesen kaufen (0 Geld übrig) – für den letzten Kauf erhältst du keine Siegpunkt-Marker.

	\medskip

	\emph{Bollwerk:} Hier werden alle Geldkarten (auch ggf. kombinierte) ausgewertet, die im Spiel benutzt wurden (auch ggf. Geldkarten, die im Schwarzmarkt (aus Basisspiel Special Edition bzw. Promokarte) enthalten waren). Haben zwei oder mehrere Spieler die gleiche höchste Anzahl einer Geldkarte, erhalten alle diese Spieler 5 Siegpunkte.

	\medskip

	\emph{Brunnen:} Du erhältst entweder 15 Siegpunkte oder 0 Siegpunkte. Es gibt keinen Extra-Bonus, wenn du mehr als 10 Kupfer besitzt.}
\end{tikzpicture}
\hspace{-0.6cm}
\begin{tikzpicture}
	\card
	\cardstrip
	\cardbanner{banner/green.png}
	\cardtitle{\footnotesize{Landmarken (3/8)}\quad}
	\cardcontent{\emph{Entweihter Schrein:} Immer wenn du eine beliebige Aktionskarte nimmst und auf dem entsprechenden Vorratsstapel ein oder mehrere Siegpunkt-Marker liegen (egal ob sie dort auf Grund der Anweisung auf dieser Landmarken-Karte oder einer anderen Karte, Ereignis oder Landmarken-Karte liegen), nimm einen Siegpunkt-Marker von dort und lege ihn hierher auf den Entweihten Schrein. Nur wenn du einen Fluch kaufst (nicht, wenn du ihn auf andere Art und Weise nimmst), nimmst du alle Siegpunkt-Marker, die zu diesem Zeitpunkt hier liegen. In der Spielvorbereitung legt ihr auf jeden Vorratsstapel, der den Typ Aktion, nicht aber den Typ Sammlung (also nicht auf die Karten Bauernmarkt, Tempel und Wilde Jagd) beinhaltet, 2 Siegpunkt-Marker.}
\end{tikzpicture}
\hspace{-0.6cm}
\begin{tikzpicture}
	\card
	\cardstrip
	\cardbanner{banner/green.png}
	\cardtitle{\footnotesize{Landmarken (4/8)}\quad}
	\cardcontent{\emph{Gebirgspass:} Diese Landmarken-Karte wird genau einmal pro Spiel ausgeführt – nämlich nach Beendigung des Zuges, in dem ein Spieler die erste Provinz aus dem Vorrat nimmt. Entsorgt vorher ein Spieler bereits eine Provinz (z.B. durch das Ereignis Versalztes Land), hat jener Spieler diese Provinz aber nicht vorher genommen und erfüllt deshalb diese Bedingung auch noch nicht. In einem Spiel, indem keine Provinz genommen wird, findet diese Landmarken-Karte keine Anwendung.

	\medskip

	Der Gebirgspass wird zwischen zwei Zügen ausgeführt und kann damit z.B. von der Besessenheit (aus Alchemisten) nicht beeinflusst werden. Der Mitspieler links von dem Spieler, der die erste Provinz genommen hat, beginnt mit einem Gebot oder passt. Ein Gebot besteht aus einer Anzahl Schulden zwischen 1 und 40. Der nächste Spieler muss mindestens 1 Schulden mehr bieten als der vorherige oder passen. Ein Gebot von 40 Schulden kann nicht überboten werden. Haben alle Spieler ein Gebot abgegeben oder gepasst, bzw. wurde bereits das Höchstgebot von 40 Schulden erreicht, erhält der Spieler mit dem höchsten Gebot die entsprechende Anzahl Schulden-Marker sowie 8 Siegpunkt-Marker. Passen alle Spieler, erhält keiner etwas.}
\end{tikzpicture}
\hspace{-0.6cm}
\begin{tikzpicture}
	\card
	\cardstrip
	\cardbanner{banner/green.png}
	\cardtitle{\footnotesize{Landmarken (5/8)}\quad}
	\cardcontent{\emph{Grabmal:} Dies funktioniert auch außerhalb deines Zuges (z.B. mit dem Trickser aus Intrige) oder wenn du eine Karte entsorgst, die nicht deine eigene ist (z.B. durch das Ereignis Versalztes Land).

	\medskip

	\emph{Kolonnaden:} Wenn du eine Aktionskarte kaufst (nicht, wenn du sie auf andere Art und Weise nimmst), musst du eine Karte mit dem gleichen Namen bereits im Spiel haben, um 2 Siegpunkt-Marker von hier zu erhalten. Karten eines Stapels haben nicht unbedingt alle den gleichen Namen (z.B. bei gemischten Stapeln).

	\medskip

	\emph{Labyrinth:} Dies kann nur einmal pro Zug eines Spielers eintreten, nämlich genau in dem Moment, in dem er die zweite Karte in seinem Zug nimmt. Nimmt er außerhalb seines Zuges zwei Karten, erhält er nichts.

	\medskip

	\emph{Mauer:} Hast du mehr als 15 Karten in deinem Kartensatz, bekommst für jede Karte darüber hinaus 1 Siegpunkt. Spieler, die zum Beispiel 27 Karten im Kartensatz haben, erhalten -12 Siegpunkte, Spieler mit 14 Karten im Kartensatz erhalten keinen Siegpunkt Abzug. Die Gesamtpunktzahl kann damit auch negativ sein.}
\end{tikzpicture}
\hspace{-0.6cm}
\begin{tikzpicture}
	\card
	\cardstrip
	\cardbanner{banner/green.png}
	\cardtitle{\footnotesize{Landmarken (6/8)}\quad}
	\cardcontent{\emph{Museum:} Auch Karten, die vom gleichen Stapel stammen, aber unterschiedliche Namen haben (z.B. gemischte Stapel), werden mit jeweils 2 Siegpunkten abgerechnet.

	\medskip

	\emph{Obelisk:} Es zählen alle Karten des gewählten Stapels, auch wenn sie unterschiedliche Namen haben (z.B. bei gemischten Stapeln). Zu Spielbeginn ermittelt ihr einen zufälligen Stapel, der den Typ Aktion beinhaltet (auch ggf. kombinierte Karten) und zum Vorrat gehört. Ruinen (aus Dark Ages) können bestimmt werden, ebenfalls der Stapel, der auch als Bannstapel für die Junge Hexe (aus Reiche Ernte) genutzt wird. Dazu zählen jedoch nicht die Eintauschkarten (aus Abenteuer) und Preiskarten (aus Reiche Ernte), da diese nicht zum Vorrat gehören.

	\medskip

	\emph{Obstgarten:} Du erhältst keinen zusätzlichen Bonus, wenn du zum Beispiel von einer Aktionskarte 6 Exemplare besitzt, d.h. du erhältst für eine Aktionskarte, von der du 3 Exemplare besitzt genauso 4 Siegpunkte wie für eine, von der du 7 Exemplare besitzt.}
\end{tikzpicture}
\hspace{-0.6cm}
\begin{tikzpicture}
	\card
	\cardstrip
	\cardbanner{banner/green.png}
	\cardtitle{\footnotesize{Landmarken (7/8)}\quad}
	\cardcontent{\emph{Palast:} Wenn du bei Spielende beispielsweise 7 Kupfer, 5 Silber und 2 Gold in deinem Kartensatz hast, erhältst du 6 Siegpunkte, da du zwei komplette Sätze aus je 1 Kupfer, Silber und Gold besitzt. Hättest du noch ein drittes Gold, würdest du 9 Siegpunkte erhalten.

	\medskip

	\emph{Räuberfestung:} Hast du bei Spielende zum Beispiel 3 Silber und 1 Gold in deinem Kartensatz, werden dir 8 Siegpunkte abgezogen. Die Gesamtpunktzahl kann damit auch negativ sein.

	\medskip

	\emph{Schlachtfeld:} Du erhältst 2 Siegpunkt-Marker von hier, egal ob du die Punktekarte (auch ggf. kombinierte) kaufst oder auf andere Art und Weise nimmst. Dies funktioniert auch außerhalb deines Zuges. Falls mehrere Spieler eine Punktekarte nehmen, wird dies in Spielerreihenfolge (beginnend bei dem Spieler links des aktuellen Spielers) getan.}
\end{tikzpicture}
\hspace{-0.6cm}
\begin{tikzpicture}
	\card
	\cardstrip
	\cardbanner{banner/green.png}
	\cardtitle{\footnotesize{Landmarken (8/8)}\quad}
	\cardcontent{\emph{Triumphbogen:} Wenn du bei Spielende beispielsweise 7 Villen und 4 Wilde Jagden (und keine andere (auch ggf. kombinierte) Aktionskarte häufiger) in deinem Kartensatz hast, erhältst du 12 Siegpunkte (d.h. 3 Siegpunkte für jede der 4 Wilde Jagden). Hast du neben 7 Villen auch 7 Wilde Jagden, erhältst du für beide zusammen 21 Siegpunkte. 
	\emph{Errata:} Es sollte lauten \enquote{bei Gleichstand zählt nur eine} statt \enquote{bei Gleichstand zählen beide}.

	\medskip

	\emph{Turm:} Der Vorratsstapel muss leer sein. Ein gemischter Stapel, bei dem nur eine Sorte Karten fehlt, zählt nicht. Die Vorratsstapel mit Punktekarten zählen ebenfalls nicht, ein leerer Fluch-Stapel aber schon.

	\medskip

	\emph{Wolfsbau:} Du bekommst keine Minuspunkte durch den Wolfsbau, wenn du von einer Karte gar keine bzw. zwei oder mehr Stück in deinem kompletten Kartensatz besitzt. Hast du zum Beispiel einen Fluch in deinem Nachziehstapel und einen in deinem Ablagestapel, hast du insgesamt zwei Flüche und erhältst keine Minuspunkte durch den Wolfsbau. Die Gesamtpunktzahl kann negativ sein.}
\end{tikzpicture}
\hspace{-0.6cm}
\begin{tikzpicture}
	\card
	\cardstrip
	\cardbanner{banner/white.png}
	\cardtitle{\scriptsize{Empfohlene 10er Sätze\qquad}}
	\cardcontent{\emph{Basis Einführung}\\
	Hochzeit (Ereignis), Turm (Landmarke), Bauernmarkt, Forum, Ingenieurin, Legionär, Patrizier/Handelsplatz, Opfer, Schlösser, Stadtviertel, Villa, Wagenrennen

	\smallskip

	\emph{Fortgeschrittene Einführung}\\
	Arena (Landmarke), Triumphbogen (Landmarke), Archiv, Gärtnerin, Gladiator/Reichtum, Katapult/Felsen, Königlicher Schmied, Krone, Siedler/Emsiges Dorf, Tempel, Vermögen, Zauberin

	\smallskip

	\emph{Alles in Maßen (mit Basisspiel)}\\
	Glücksfall (Ereignis), Obstgarten (Landmarke), Forum, Legionär, Lehnsherr, Tempel, Zauberin, Bibliothek, Dorf, Keller, Umbau, Werkstatt

	\smallskip

	\emph{Silberne Kugeln (mit Basisspiel)}\\
	Eroberung (Ereignis), Aquädukt (Landmarke), Bauernmarkt, Gärtnerin, Katapult/Felsen, Patrizier/Handelsplatz, Zauber, Bürokrat, Gärtner, Geldverleiher, Laboratorium, Markt

	\smallskip

	\emph{Köstliche Folter (mit Die Intrige)}\\
	Bankett (Ereignis), Arena (Landmarke), Gärtnerin, Krone, Opfer, Schlösser, Siedler/Emsiges Dorf, Baron, Brücke, Eisenhütte, Harem, Kerkermeister}
\end{tikzpicture}
\hspace{-0.6cm}
\begin{tikzpicture}
	\card
	\cardstrip
	\cardbanner{banner/white.png}
	\cardtitle{\scriptsize{Empfohlene 10er Sätze\qquad}}
	\cardcontent{\emph{Buddy-Prinzip (mit Die Intrige)}\\
	Versalztes Land (Ereignis), Wolfsbau (Landmarke), Archiv, Forum, Ingenieurin, Katapult/Felsen, Vermögen, Adlige, Bergwerk, Handelsposten, Handlanger, Maskerade

	\emph{Eingeengt (mit Seaside)}\\
	Steuer (Ereignis), Mauer (Landmarke), Feldlager/Diebesgut, Gladiator/Reichtum, Schlösser, Wagenrennen, Zauberin, Lagerhaus, Müllverwerter, Schmuggler, Taktiker, Werft

	\emph{König der Meere (mit Seaside)}\\
	Erforschen (Ereignis), Brunnen (Landmarke), Archiv, Bauernmarkt, Lehnsherr, Tempel, Wilde Jagd, Eingeborenendorf, Entdecker, Hafen, Piratenschi , Seehexe

	\emph{Sammler (mit Die Alchemisten)}\\
	Kolonnaden (Landmarke), Museum (Landmarke), Bauernmarkt, Feldlager/Diebesgut, Krone, Stadtviertel, Zauberin, Apotheker, Kräuterkundiger, Lehrling, Universität, Verwandlung

	\emph{Gewaltig (mit Blütezeit)}\\
	Beherrschen (Ereignis), Obelisk (Landmarke), Gladiator/Reichtum, Königlicher Schmied, Patrizier/Handelsplatz, Vermögen, Villa, Bank, Großer Markt, Königliches Siegel, Kunstschmiede, Lohn, Platin- und Kolonie-Karten}
\end{tikzpicture}
\hspace{-0.6cm}
\begin{tikzpicture}
	\card
	\cardstrip
	\cardbanner{banner/white.png}
	\cardtitle{\scriptsize{Empfohlene 10er Sätze\qquad}}
	\cardcontent{\emph{Vergoldete Pforten (mit Blütezeit)}\\
	Basilika (Landmarke), Palast (Landmarke), Feldlager/Diebesgut, Gärtnerin, Stadtviertel, Wagenrennen, Wilde Jagd, Bischof, Denkmal, Hausierer, Münzer, Talisman, Platin- und Kolonie-Karten

	\emph{Zoowärter (mit Reiche Ernte)}\\
	Schlacht (Ereignis), Kolonnaden (Landmarke), Lehnsherr, Opfer, Siedler/Emsiges Dorf, Villa, Wilde Jagd, Festplatz, Harlekin, Menagerie, Pferdehändler, Turnier
	
	\emph{Einfache Pläne (mit Hinterland)}\\
	Spende (Ereignis), Labyrinth (Landmarke), Forum, Katapult/Felsen, Patrizier/Handelsplatz, Tempel, Villa, Aufbau, Blutzoll, Feilscher, Grenzdorf, Stallungen
	
	\emph{Ausbreitung (mit Hinterland)}\\
	Brunnen (Landmarke), Schlachtfeld (Landmarke), Feldlager/Diebesgut, Ingenieurin, Legionär, Schlösser, Zauber, Fernstraße, Fruchtbares Land, Gewürzhändler, Schatztruhe, Tunnel
	
	\emph{Das Grabmal des Rattenkönigs (mit Dark Ages)}\\
	Aufstieg (Ereignis), Grabmal (Landmarke), Katapult/Felsen, Legionär, Schlösser, Stadtviertel, Wagenrennen, Festung, Lagerraum, Leichenkarren, Ratten, Raubzug, Unterschlupf-Karten}
\end{tikzpicture}
\hspace{-0.6cm}
\begin{tikzpicture}
	\card
	\cardstrip
	\cardbanner{banner/white.png}
	\cardtitle{\scriptsize{Empfohlene 10er Sätze\qquad}}
	\cardcontent{\emph{Der Triumph des Banditenkönigs (mit Dark Ages)}\\
	Siegeszug (Ereignis), Entweihter Schrein (Landmarke), Gärtnerin, Ingenieurin, Legionär, Vermögen, Zauber, Banditenlager, Jagdgründe, Katakomben, Marktplatz, Prozession
	
	\emph{Das Ritual des Knappen (mit Dark Ages)}\\
	Ritual (Ereignis), Museum (Landmarke), Archiv, Katapult/Felsen, Krone, Patrizier/Handelsplatz, Siedler/Emsiges Dorf, Eisenhändler, Eremit, Knappe, Lehen, Schurke
	
	\emph{Geldfluss (mit Die Gilden)}\\
	Badehaus (Landmarke), Gebirgspass (Landmarke), Gladiator/Reichtum, Ingenieurin, Königlicher Schmied, Schlösser, Stadtviertel, Arzt, Bäcker, Herold, Metzger, Wahrsager
	
	\emph{Kontrollbereich (mit Abenteuer)}\\
	Bankett (Ereignis), Bollwerk (Landmarke), Bauernmarkt, Katapult/Felsen, Krone, Vermögen, Zauber, Königliche Münzen, Page (+ Eintauschkarten), Relikt, Schatz, Weinhändler
	
	\emph{Kein Geld, keine Probleme (mit Abenteuer)}\\
	Mission (Ereignis), Räuberfestung (Landmarke), Archiv, Feldlager/Diebesgut, Königlicher Schmied, Tempel, Villa, Duplikat, Gefolgsmann, Kleinbauer (+ Eintauschkarten), Transformation, Verlies}
\end{tikzpicture}
\hspace{0.6cm}

	    % Basic settings for this card set
\renewcommand{\cardcolor}{empires}
\renewcommand{\cardextension}{Erweiterung IX}
\renewcommand{\cardextensiontitle}{Empires}
\renewcommand{\seticon}{empires.png}

\clearpage
\newpage
\section{\cardextension \ - \cardextensiontitle \ (Rio Grande Games 2016)}

\begin{tikzpicture}
	\card
	\cardstrip
	\cardbanner{banner/white.png}
	\cardicon{icons/coin.png}
	\cardprice{2}
	\cardtitle{Feldlager}
	\cardcontent{Du darfst ein \emph{GOLD} oder ein \emph{DIEBESGUT} aus der Hand aufdecken. Wenn du das nicht kannst oder möchtest, legst du diese Karte zur Seite und legst sie zu Beginn deiner Aufräumphase zurück in den Vorrat. Sollte dort zu diesem Zeitpunkt bereits ein \emph{DIEBESGUT} offen liegen, muss nun erst wieder das zurückgelegte \emph{FELDLAGER} genommen werden, bevor das \emph{DIEBESGUT} genommen werden darf.}
\end{tikzpicture}
\hspace{-0.6cm}
\begin{tikzpicture}
	\card
	\cardstrip
	\cardbanner{banner/white.png}
	\cardicon{icons/coin.png}
	\cardprice{2}
	\cardtitle{Patrizier}
	\cardcontent{Du musst die oberste Karte deines Nachziehstapels aufdecken. Wenn der Stapel aufgebraucht ist, mischst du deinen Ablagestapel und legst ihn als Nachziehstapel bereit. Wenn auch dort keine Karten liegen, erhältst du nichts.

	\medskip

	Nur eine Karte, die mehr als \coin[5] kostet, darfst du auf die Hand nehmen. Ob die Karte außerdem noch Kosten in Form von Schulden \hex aufweist, ist dabei unerheblich (z.B. \emph{REICHTUM} darf auf die Hand genommen werden, \emph{STADTVIERTEL} nicht).}
\end{tikzpicture}
\hspace{-0.6cm}
\begin{tikzpicture}
	\card
	\cardstrip
	\cardbanner{banner/white.png}
	\cardicon{icons/coin.png}
	\cardprice{2}
	\cardtitle{Siedler}
	\cardcontent{Auch wenn du weißt, dass sich kein \emph{KUPFER} in deinem Ablagestapel befindet, darfst du ihn ansehen. Du musst kein \emph{KUPFER} auf die Hand nehmen, wenn du das nicht möchtest.}
\end{tikzpicture}
\hspace{-0.6cm}
\begin{tikzpicture}
	\card
	\cardstrip
	\cardbanner{banner/white.png}
	\cardicon{icons/coin.png}
	\cardprice{3}
	\cardtitle{\footnotesize{Bauernmarkt}}
	\cardcontent{Diese Karte beinhaltet den neuen Typ SAMMLUNG, d.h. hier kommen die Siegpunktmarker zum Einsatz.

	\medskip

	Wenn diese Karte das erste Mal ausgespielt wird, legt der Spieler einen \victorypointtoken-Marker auf den BAUERNMARKT-Vorratsstapel und erhält dann +\coin[1] für den gerade gelegten Marker. Wird die Karte zum zweiten, dritten und vierten Mal ausgespielt, legt der Spieler jeweils einen weiten Marker auf den Stapel und erhält +\coin[2], +\coin[3] bzw. +\coin[4], egal welcher Spieler die vorherigen Marker auf den Stapel gelegt hat. Wird die Karte danach erneut ausgespielt, nimmt der Spieler die 4 \victorypointtoken-Marker (dafür aber kein \coin) und muss den ausgespielten \emph{BAUERNMARKT} entsorgen.

	\medskip

	Danach beginnt der Vorgang wieder von vorn und wird fortgesetzt, falls der Vorratsstapel leer ist.}
\end{tikzpicture}
\hspace{-0.6cm}
\begin{tikzpicture}
	\card
	\cardstrip
	\cardbanner{banner/white.png}
	\cardicon{icons/coin.png}
	\cardprice{3}
	\cardtitle{Gladiator}
	\cardcontent{Wenn du mindestens 1 Handkarte hast, musst du diese aufdecken. Wenn dein linker Mitspieler keine Karte mit gleichem Namen aufdecken kann oder will (z.B. auch, wenn du keine Handkarte aufdecken konntest, weil du keine hast), erhältst du zusätzlich +\coin[1] . Sind noch Karten auf dem \emph{GLADIATOR}-Vorratsstapel vorhanden, musst du eine entsorgen. Deckt der Mitspieler eine Karte mit gleichem Namen auf, erhältst du nur +\coin[2] und darfst keinen \emph{GLADIATOR} entsorgen.}
\end{tikzpicture}
\hspace{-0.6cm}
\begin{tikzpicture}
	\card
	\cardstrip
	\cardbanner{banner/white.png}
	\cardicon{icons/coin.png}
	\cardprice{3}
	\cardtitle{Katapult}
	\cardcontent{Wenn du mindestens 1 Handkarte hast, musst du auch eine entsorgen. Kostet die entsorgte Karte \coin[3] oder mehr, nimmt sich jeder Mitspieler (beginnend bei deinem linken Nachbarn) einen \emph{FLUCH}. Karten mit Schulden kosten nur dann \coin[3] oder mehr, wenn sie zusätzlich zu etwaigen Schulden-Kosten mindestens \coin[3] kosten. Ist die entsorgte Karte eine Geldkarte muss jeder Mitspieler - unabhängig von den Kosten der Karte - seine Handkarten auf 3 reduzieren.}
\end{tikzpicture}
\hspace{-0.6cm}
\begin{tikzpicture}
	\card
	\cardstrip
	\cardbanner{banner/white.png}
	\cardicon{icons/coin.png}
	\cardprice{3}
	\cardtitle{\footnotesize{Wagenrennen}}
	\cardcontent{Nimm deine aufgedeckte Karte nach dem Vergleich der Kosten mit der aufgedeckten Karte deines linken Mitspielers auf die Hand. Der Mitspieler legt seine aufgedeckte Karte zurück auf den Nachziehstapel.

	Kosten beide Karten gleich viel oder kostet die Karte des Mitspielers mehr, erhältst du nichts. Kostet deine Karte mehr erhältst du +1 \coin und +1 \victorypointtoken-Marker. Hast entweder du oder dein linker Mitspieler (auch nach dem eventuellen Mischen des Ablagestapels) keine Karte zum Aufdecken, erhältst du nichts.}
\end{tikzpicture}
\hspace{-0.6cm}
\begin{tikzpicture}
	\card
	\cardstrip
	\cardbanner{banner/orange.png}
	\cardicon{icons/coin.png}
	\cardprice{3}
	\cardtitle{Zauberin}
	\cardcontent{Spieler, die mit einer Reaktionskarte wie dem \emph{BURGGRABEN} (aus dem Basisspiel) reagieren möchten, müssen dies tun, sobald die \emph{ZAUBERIN} ausgespielt wurde, auch wenn der Angriff sie erst in ihrem nächsten Zug betrifft.

	\medskip

	Jeder Mitspieler erhält in seinem nächsten Zug für die erste gespielte Aktionskarte + 1 Karte sowie + 1 Aktion, darf aber den eigentlichen Effekt der Karte beim Ausspielen nicht durchführen. Anweisungen, die sich auf einen anderen Zeitpunkt im Spiel beziehen (z.B. die beim Kauf der Karte zum Tragen kommen), werden nicht beeinflusst.

	Um anzuzeigen, dass die erste ausgespielte Aktionskarte von der \emph{ZAUBERIN} beeinflusst wird, empfehlen wir, diese beim Ausspielen quer auszulegen. Karten, die bereits ausgespielt wurden (z.B. Dauerkarten wie das \emph{ARCHIV}), werden zu Beginn des Zuges normal abgehandelt und nicht von der \emph{ZAUBERIN} beeinflusst. Spielt ein Spieler in seiner Aktionsphase keine Aktionskarte aus, dafür aber in seiner Kaufphase eine \emph{KRONE} (kombinierte Aktions- und Geldkarte), kommt der Effekt der \emph{ZAUBERIN} zum Tragen, da es sich um eine Aktionskarte handelt, auch wenn diese in der Kaufphase ausgespielt wurde. Normalerweise kann der Spieler die + 1 Aktion zu diesem Zeitpunkt nicht nutzen, es sei denn, er kauft zum Beispiel eine \emph{VILLA}.}
\end{tikzpicture}
\hspace{-0.6cm}
\begin{tikzpicture}
	\card
	\cardstrip
	\cardbanner{banner/gold.png}
	\cardicon{icons/coin.png}
	\cardprice{4}
	\cardtitle{Felsen}
	\cardcontent{Wenn du diese Karte in deiner Kaufphase nimmst oder entsorgst, nimm ein \emph{SILBER} und lege es auf deinen Nachziehstapel. Wenn du diese Karte zu einem anderen Zeitpunkt (auch während des Zuges eines anderen Spielers) nimmst oder entsorgst, nimm ein \emph{SILBER} auf die Hand.}
\end{tikzpicture}
\hspace{-0.6cm}
\begin{tikzpicture}
	\card
	\cardstrip
	\cardbanner{banner/white.png}
	\cardicon{icons/coin.png}
	\cardprice{4}
	\cardtitle{Opfer}
	\cardcontent{Wenn die entsorgte Karte eine kombinierte Karte ist, erhältst du die Boni aller entsprechenden Typen dieser Karte. Entsorgst du eine Karte, die keinem der angegebenen Typen entspricht (z.B. einen \emph{FLUCH}), erhältst du nichts.}
\end{tikzpicture}
\hspace{-0.6cm}
\begin{tikzpicture}
	\card
	\cardstrip
	\cardbanner{banner/white.png}
	\cardicon{icons/coin.png}
	\cardprice{4}
	\cardtitle{Tempel}
	\cardcontent{Es dürfen nur Karten mit unterschiedlichem Namen entsorgt werden, z.B. ein \emph{KUPFER} und ein \emph{ANWESEN}.

	Auch wenn der \emph{TEMPEL}-Vorratsstapel leer ist, legst du einen \victorypointtoken-Marker auf den leeren Platz. Das kann relevant werden, wenn durch Anweisungen auf anderen Karten ein \emph{TEMPEL} in den Vorrat zurückgelegt wird (z.B. durch den \emph{BOTSCHAFTER} aus \emph{Seaside}) nehmen darf.

	\medskip

	Wenn du einen \emph{TEMPEL} nimmst, nimmst du auch alle \victorypointtoken-Marker, die zu diesem Zeitpunkt auf dem Vorratsstapel liegen.}
\end{tikzpicture}
\hspace{-0.6cm}
\begin{tikzpicture}
	\card
	\cardstrip
	\cardbanner{banner/white.png}
	\cardicon{icons/coin.png}
	\cardprice{4}
	\cardtitle{Villa}
	\cardcontent{Wenn du diese Karte in deiner Aktionsphase nimmst (z.B. durch die \emph{INGENIEURIN}), nimm sie sofort auf die Hand und erhalte + 1 Aktion. Dadurch kannst du z.B. die gerade genommene \emph{VILLA} sofort ausspielen. Wenn du diese Karte in deiner Kaufphase nimmst (z.B. indem du sie kaufst), nimm sie auf die Hand und kehre sofort in die Aktionsphase zurück, wo du + 1 Aktion hast. Hast du die Aktionsphase erneut komplett abgeschlossen, kehrst du wieder zur Kaufphase zurück. Hier kannst du weitere Geldkarten ausspielen (und z.B. die \emph{ARENA} kommt wieder zum Tragen). Wenn du diese Karte während des Zuges eines Mitspielers nimmst, nimmst du die Karte auf die Hand und erhältst zwar + 1 Aktion, kannst diese aber nicht nutzen, da es nicht dein Zug ist. Es ist möglich, mehrmals pro Zug (z.B. durch das Nehmen mehrerer \emph{VILLEN}) in die Aktionsphase zurückzukehren. Dies bedeutet aber nicht, dass du an den \enquote{Beginn deines Zuges} zurückkehrst. Anweisungen, die sich darauf beziehen, haben keine Auswirkung.}
\end{tikzpicture}
\hspace{-0.6cm}
\begin{tikzpicture}
	\card
	\cardstrip
	\cardbanner{banner/orange.png}
	\cardicon{icons/coin.png}
	\cardprice{5}
	\cardtitle{Archiv}
	\cardcontent{Lege die obersten drei Karten deines Nachziehstapels zur Seite und schau sie dir an. Nimm eine der Karten sofort auf die Hand und lege die anderen Karten unter dieses \emph{ARCHIV}. Spielst du zwei \emph{ARCHIVE}, lege die Karten für die nächsten Züge unter das jeweils ausgespielte \emph{ARCHIV}. Hast du nicht genügend Karten, um drei Karten zur Seite zu legen, legst du nur so viele wie möglich zur Seite. Das \emph{ARCHIV} wird in dem Spielzug abgelegt, in dem die letzte zur Seite gelegte Karte des jeweiligen \emph{ARCHIVS} auf die Hand genommen wurde.}
\end{tikzpicture}
\hspace{-0.6cm}
\begin{tikzpicture}
	\card
	\cardstrip
	\cardbanner{banner/gold.png}
	\cardicon{icons/coin.png}
	\cardprice{5}
	\cardtitle{Diebesgut}
	\cardcontent{Nimm dir jedes Mal, wenn du diese Karte spielst, einen \victorypointtoken-Marker und lege ihn bei dir ab.}
\end{tikzpicture}
\hspace{-0.6cm}
\begin{tikzpicture}
	\card
	\cardstrip
	\cardbanner{banner/white.png}
	\cardicon{icons/coin.png}
	\cardprice{5}
	\cardtitle{Emsiges Dorf}
	\cardcontent{Du darfst deinen Ablagestapel auch dann durchsehen, wenn du weißt, dass du keine \emph{SIEDLER} darin hast. Du darfst die Reihenfolge der Karten in deinem Ablagestapel nicht verändern.}
\end{tikzpicture}
\hspace{-0.6cm}
\begin{tikzpicture}
	\card
	\cardstrip
	\cardbanner{banner/white.png}
	\cardicon{icons/coin.png}
	\cardprice{5}
	\cardtitle{Forum}
	\cardcontent{Wenn du diese Karte kaufst, erhältst du + 1 Kauf. Du kannst beispielsweise mit \coin[\hspace{-0.3em}13] und nur einem freien Kauf, zuerst diese Karte kaufen und dann mit dem zusätzlichen Kauf noch eine \emph{PROVINZ}.}
\end{tikzpicture}
\hspace{-0.6cm}
\begin{tikzpicture}
	\card
	\cardstrip
	\cardbanner{banner/white.png}
	\cardicon{icons/coin.png}
	\cardprice{5}
	\cardtitle{Gärtnerin}
	\cardcontent{Ist diese Karte im Spiel und du nimmst eine Punktekarte – egal in welcher Spielphase – nimmst du dir einen \victorypointtoken-Marker und legst ihn bei dir ab. Wenn du mehrere Karten nimmst, nimmst du dir für jede genommene Karten einen \victorypointtoken-Marker. Hast du mehrere \emph{GÄRTNERINNEN} im Spiel, nimmst du dir für jede \emph{GÄRTNERIN} pro genommener Karte einen \victorypointtoken-Marker.

	Wenn du z.B. eine \emph{GÄRTNERIN} auf eine \emph{KRONE} spielst, befindet sich die \emph{GÄRTNERIN} trotzdem nur einmal im Spiel und du darfst dir pro genommener Karte nur einen \victorypointtoken-Marker nehmen.}
\end{tikzpicture}
\hspace{-0.6cm}
\begin{tikzpicture}
	\card
	\cardstrip
	\cardbanner{banner/white.png}
	\cardicon{icons/coin.png}
	\cardprice{5}
	\cardtitle{\footnotesize{Handelsplatz}}
	\cardcontent{Zu den Aktionskarten, die du zu diesem Zeitpunkt im Spiel hast zählen alle Aktionskarten, die du ausgespielt hast, Dauerkarten, die sich aus vergangenen Zügen im Spiel befinden und Reservekarten (aus \emph{Abenteuer}), die du in diesem Zug bereits aufgerufen hast. Wenn du diese Karte außerhalb deines Zuges nimmst, hast du keine Aktionskarten im Spiel und du darfst dir keine \victorypointtoken-Marker nehmen.}
\end{tikzpicture}
\hspace{-0.6cm}
\begin{tikzpicture}
	\card
	\cardstrip
	\cardbanner{banner/whitegold.png}
	\cardicon{icons/coin.png}
	\cardprice{5}
	\cardtitle{Krone}
	\cardcontent{Diese Karte ist eine kombinierte Aktions- und Geldkarte. Wenn du sie in deiner Aktionsphase ausspielst, darfst du eine Aktionskarte von deiner Hand wählen und ausspielen. Du nimmst die gewählte Karte nicht wieder auf die Hand, sondern spielst die Aktion ein zweites Mal. Dafür benötigst du keine weiteren Aktionen. Wählst du eine \emph{KRONE}, musst du diese auch als Aktionskarte ausspielen (und dann darfst du bis zu zwei weitere Aktionskarten jeweils zweimal spielen).

	Spielst du diese Karte in deiner Aktionsphase als Geldkarte aus (z.B. durch den \emph{GESCHICHTENERZÄHLER} aus \emph{Abenteuer}), darfst du trotzdem eine Aktionskarte zweimal ausspielen.

	\medskip

	Spielst du diese Karte in deiner Kaufphase, darfst du eine beliebige Geldkarte von deiner Hand wählen, sie ausspielen und zweimal ausführen. Wählst du eine \emph{KRONE}, spielst du diese aus und dann eine weitere Geldkarte von der Hand zweimal und dann noch eine Geldkarte zweimal.}
\end{tikzpicture}
\hspace{-0.6cm}
\begin{tikzpicture}
	\card
	\cardstrip
	\cardbanner{banner/white.png}
	\cardicon{icons/coin.png}
	\cardprice{5}
	\cardtitle{Legionär}
	\cardcontent{Mitspieler, die auf das Ausspielen dieser Karte mit einer Reaktionskarte reagieren möchten, müssen dies tun, bevor du dich entscheidest, ob du ein \emph{GOLD} aufdeckst oder nicht.

	\medskip

	Mitspieler, die bereits zwei oder weniger Karten auf der Hand haben, müssen keine Karte ablegen, müssen gleichwohl aber eine Karte ziehen.}
\end{tikzpicture}
\hspace{-0.6cm}
\begin{tikzpicture}
	\card
	\cardstrip
	\cardbanner{banner/gold.png}
	\cardicon{icons/coin.png}
	\cardprice{5}
	\cardtitle{Vermögen}
	\cardcontent{Diese Karte ist eine Geldkarte mit zusätzlichen Anweisungen. Sie hat den Wert \coin[6]. Außerdem erhältst du + 1 Kauf.

	\medskip

	Wenn du diese Karte ablegst (in der Regel in deiner Aufräumphase), nimm \hex[6] vom Vorrat. Dann kannst du sofort beliebig viele \hex (auch mehr als die \hex[6], die du durch das Ablegen dieser Karte erhalten hast) zurückzahlen.

	\medskip

	Wenn du diese Karte nicht ablegst (z.B. wenn du sie stattdessen entsorgst), erhältst du keine \hex. Wenn du diese Karte zweimal ausgespielt hast (z.B. durch eine \emph{KRONE}), erhältst du trotzdem nur \hex[6], da du nur eine Karte ablegst.}
\end{tikzpicture}
\hspace{-0.6cm}
\begin{tikzpicture}
	\card
	\cardstrip
	\cardbanner{banner/white.png}
	\cardicon{icons/coin.png}
	\cardprice{5}
	\cardtitle{Wilde Jagd}
	\cardcontent{Wählst du die erste Option, lege einen \victorypointtoken-Marker vom Vorrat auf den \emph{WILDE-JAGD}-Vorratsstapel.

	\medskip

	Wählst du die zweite Option und der \emph{ANWESEN}-Vorratsstapel ist leer (d.h. du kannst dir kein \emph{ANWESEN} nehmen), darfst du dir die \victorypointtoken-Marker vom \emph{WILDE-JAGD}-Vorratsstapel nicht nehmen. Du darfst aber diese Option trotzdem wählen.

	\medskip

	Ist der \emph{WILDE-JAGD}-Vorratsstapel leer, funktioniert das Ausspielen dieser Karte trotzdem in der beschriebenen Weise weiter. Nutzt die Platzhalterkarte, um den Vorratsstapel zu markieren.}
\end{tikzpicture}
\hspace{-0.6cm}
\begin{tikzpicture}
	\card
	\cardstrip
	\cardbanner{banner/gold.png}
	\cardicon{icons/coin.png}
	\cardprice{5}
	\cardtitle{Zauber}
	\cardcontent{Wenn du diese Karte ausspielst und dich für die zweite Option entscheidest, darfst du (musst aber nicht) sofort, wenn du die \emph{nächste} Karte in deinem Zug kaufst, eine Karte mit anderem Namen nehmen, die \emph{exakt so viel} kostet, wie die gekaufte Karte. Dann erst nimmst du die gekaufte Karte. Das kann wichtig bei Karten sein, die Anweisungen beim Nehmen einer Karte beinhalten.

	Spielst du mehrere \emph{ZAUBER} in einem Zug, darfst du dir für die nächste gekaufte Karte mehrere Karten mit anderem Namen als die gekaufte aber gleichen Kosten nehmen. Die Karten, die du nimmst müssen zwar einen anderen Namen als die Gekaufte haben, dürfen aber untereinander alle den gleichen Namen haben.}
\end{tikzpicture}
\hspace{-0.6cm}
\begin{tikzpicture}
	\card
	\cardstrip
	\cardbanner{banner/white.png}
	\cardicon{icons/hex.png}
	\cardprice{\textcolor{white}{4}}
	\cardtitle{Ingenieurin}
	\cardcontent{Du darfst dir keine Karte nehmen, die mehr als \coin[4] kostet oder mit \hex in den Kosten hat. Nimm die gewählte Karte.

	Dann darfst du diese \emph{INGENIEURIN} entsorgen. Wenn du das tust, nimm eine weitere Karte, die bis zu \coin[4] kostet. Dies kann die gleiche Karte wie die erste sein oder eine andere.}
\end{tikzpicture}
\hspace{-0.6cm}
\begin{tikzpicture}
	\card
	\cardstrip
	\cardbanner{banner/white.png}
	\cardicon{icons/hex.png}
	\cardprice{\textcolor{white}{8}}
	\cardtitle{\miniscule{KöniglicherSchmied}}
	\cardcontent{Du musst, nachdem du 5 Karten nachgezogen hast, alle deine Handkarten vorzeigen und jedes \emph{KUPFER}, das du zu diesem Zeitpunkt auf der Hand hast, ablegen.}
\end{tikzpicture}
\hspace{-0.6cm}
\begin{tikzpicture}
	\card
	\cardstrip
	\cardbanner{banner/white.png}
	\cardicon{icons/hex.png}
	\cardprice{\textcolor{white}{8}}
	\cardtitle{Lehnsherr}
	\cardcontent{Wähle eine Karte vom Vorrat, die zu diesem Zeitpunkt bis zu \coin[5] kostet, d.h. du darfst keine Karte eines leeren Stapels, eine nicht sichtbare Karte eines gemischten Stapels oder eine Karte eines Nicht-Vorratsstapels wählen.

	\medskip

	Behandle nun den ausgespielten \emph{LEHNSHERR}, wie die gewählte Karte (und nicht mehr als \emph{LEHNSHERR}) – bis sie nicht mehr im Spiel ist. Das heißt du befolgst alle Anweisungen der anderen Karte. Auch nimmt der \emph{LEHNSHERR} den Namen, die Kosten und den Typ der gewählten Karte an, bis er nicht mehr im Spiel ist. Als Dauerkarte bleibt dieser \emph{LEHNSHERR} ebenso im Spiel, wie er als Reservekarte (aus \emph{Abenteuer}) zur Seite gelegt wird. Spielst du diesen \emph{LEHNSHERR} auf einen \emph{THRONSAAL} (aus dem \emph{Basisspiel}), wählst du beim ersten Ausspielen die Karte, die dieser \emph{LEHNSHERR} ab sofort ist – beim zweiten Ausspielen ist er damit wieder genau diese Karte – du darfst keine andere Karte wählen. Erst mit dem Ausspielen des \emph{LEHNSHERRN} nimmt er Typ und Namen der gewählten Karte an – d.h. du darfst ihn nicht als \emph{KRONE} in deiner Kaufphase spielen, da er selbst keine Geldkarte ist und nicht in der Kaufphase ausgespielt werden darf.}
\end{tikzpicture}
\hspace{-0.6cm}
\begin{tikzpicture}
	\card
	\cardstrip
	\cardbanner{banner/white.png}
	\cardicon{icons/hex.png}
	\cardprice{\textcolor{white}{8}}
	\cardtitle{Stadtviertel}
	\cardcontent{Du musst deine Handkarten aufdecken. Für jede Aktionskarte (auch ggf. kombinierte), die du aufdeckst, ziehst du eine Karte nach.}
\end{tikzpicture}
\hspace{-0.6cm}
\begin{tikzpicture}
	\card
	\cardstrip
	\cardbanner{banner/white.png}
	\cardicon{icons/coin.png}
	\cardprice{8}
	\cardiconaddition{icons/hex.png}
	\cardpriceaddition{\textcolor{white}{8}}
	\cardtitle{\quad Reichtum}
	\cardcontent{Es werden nur alle \coin verdoppelt, die du vor dem Ausspielen dieser Karte ausgespielt hast und nur, wenn du in diesem Zug noch keinen \emph{REICHTUM} ausgespielt hast. Für jedes weitere Ausspielen eines \emph{REICHTUMS} erhältst du nur + 1 Kauf.}
\end{tikzpicture}
\hspace{-0.6cm}
\begin{tikzpicture}
	\card
	\cardstrip
	\cardbanner{banner/green.png}
	\cardtitle{Schlösser (1/2)\quad}
	\cardcontent{\emph{Schloss-Karten:} Der Schloss-Stapel ist ein gemischter Vorratsstapel. Alle Schlösser werden nach Kosten sortiert auf dem Vorratsstapel bereitgelegt (die teuerste zuunterst).

	\bigskip

	\emph{Bescheidenes Schloss:} Spielst du sie in deiner Kaufphase aus, ist sie \coin[1] wert. Bei Spielende erhältst du pro Karte, die den Typ \emph{SCHLOSS} beinhaltet, einen \victorypointtoken-Marker.

	\medskip

	\emph{Verfallendes Schloss:} Diese Karte ist zu Spielende 1 \victorypoint wert – wie ein \emph{ANWESEN}. Wenn du diese Karte während des Spiels nimmst, nimm dir einen \victorypointtoken-Marker sowie ein \emph{SILBER} vom Vorrat. Wenn du diese Karte während des Spiels entsorgst, nimm dir einen weiteren \victorypointtoken-Marker sowie ein \emph{SILBER} vom Vorrat.

	\medskip

	\emph{Kleines Schloss:} Spielst du sie in deiner Aktionsphase aus, entsorge dieses \emph{KLEINE SCHLOSS} oder eine andere \emph{SCHLOSS}-Karte aus deiner Hand. Wenn du das tust, nimm dir die \emph{SCHLOSS}-Karte vom Vorratsstapel, die zu diesem Zeitpunkt oben liegt. Dies kann eine teurere sein, als die, die du entsorgst. Du musst die Kosten nicht bezahlen. Bei Spielende ist diese Karte 2 \victorypoint wert.}
\end{tikzpicture}
\hspace{-0.6cm}
\begin{tikzpicture}
	\card
	\cardstrip
	\cardbanner{banner/green.png}
	\cardtitle{Schlösser (2/2)\quad}
	\cardcontent{\emph{Spukschloss:} Diese Karte ist zu Spielende 2 \victorypoint wert. Wenn du diese Karte während deines Zuges nimmst (kaufst oder auf andere Art und Weise nimmst), nimm dir ein \emph{GOLD} vom Vorrat. Ist kein \emph{GOLD} mehr im Vorrat, erhältst du nichts. Außerdem (egal ob du ein \emph{GOLD} nehmen kannst oder nicht) müssen alle Mitspieler mit 5 oder mehr Handkarten 2 Handkarten auf ihren Nachziehstapel zurücklegen. Da diese Karte keine Angriffskarte ist, dürfen die Mitspieler keine Reaktionskarte spielen.

	\smallskip

	\emph{Reiches Schloss:} Spielst du sie in deiner Aktionsphase aus, lege beliebig viele Punktekarten (auch ggf. kombinierte) aus deiner Hand ab. Pro abgelegter Karte erhältst du +\coin[2]. Bei Spielende ist diese Karte 3 \victorypoint wert.

	\smallskip

	\emph{Ausgedehntes Schloss:} Wenn du diese Karte kaufst oder auf andere Art und Weise nimmst, nimm ein \emph{HERZOGTUM} oder drei \emph{ANWESEN}. Bei Spielende ist diese Karte 4 \victorypoint wert.

	\smallskip

	\emph{Prunkschloss:} Wenn du diese Karte kaufst oder auf andere Art und Weise nimmst, zeige deine Handkarten vor. Nimm einen \victorypointtoken-Marker vom Vorrat für jede Punktekarte (auch ggf. kombinierte), die du zu diesem Zeitpunkt auf der Hand oder im Spiel hast.

	\smallskip
 
	\emph{Königsschloss:} Bei Spielende erhältst du pro Karte, die den Typ \emph{SCHLOSS} beinhaltet (inklusive dieser Karte) 2 \victorypoint.}
\end{tikzpicture}
\hspace{-0.6cm}
\begin{tikzpicture}
	\card
	\cardstrip
	\cardbanner{banner/white.png}
	\cardtitle{Katapult/Felsen\qquad}
	\cardcontent{Spielvorbereitung: Legt auf diese Karte 5 Felsen und oben darauf 5 Katapulte.

	\bigskip

	Es darf immer nur die oberste Karte des Stapels genommen oder gekauft werden.}
\end{tikzpicture}
\hspace{-0.6cm}
\begin{tikzpicture}
	\card
	\cardstrip
	\cardbanner{banner/white.png}
	\cardtitle{\scriptsize{Gladiator/Reichtum}\qquad}
	\cardcontent{Spielvorbereitung: Legt auf diese Karte 5 Reichtum und oben darauf 5 Gladiatoren. 

	\bigskip

	Es darf immer nur die oberste Karte des Stapels genommen oder gekauft werden.}
\end{tikzpicture}
\hspace{-0.6cm}
\begin{tikzpicture}
	\card
	\cardstrip
	\cardbanner{banner/white.png}
	\cardtitle{\scriptsize{Siedler/Emsiges Dorf}\qquad}
	\cardcontent{Spielvorbereitung: Legt auf diese Karte 5 Emsige Dörfer und oben darauf 5 Siedler. 

	\bigskip

	Es darf immer nur die oberste Karte des Stapels genommen oder gekauft werden.}
\end{tikzpicture}
\hspace{-0.6cm}
\begin{tikzpicture}
	\card
	\cardstrip
	\cardbanner{banner/white.png}
	\cardtitle{\scriptsize{Patrizier/Handelsplatz}\qquad}
	\cardcontent{Spielvorbereitung: Legt auf diese Karte 5 Handelsplätze und oben darauf 5 Patrizier. 

	\bigskip

	Es darf immer nur die oberste Karte des Stapels genommen oder gekauft werden.}
\end{tikzpicture}
\hspace{-0.6cm}
\begin{tikzpicture}
	\card
	\cardstrip
	\cardbanner{banner/white.png}
	\cardtitle{\scriptsize{Feldlager/Diebesgut}\qquad}
	\cardcontent{Spielvorbereitung: Legt auf diese Karte 5 Diebesgut und oben darauf 5 Feldlager. 

	\bigskip

	Es darf immer nur die oberste Karte des Stapels genommen oder gekauft werden.}
\end{tikzpicture}
\hspace{-0.6cm}
\begin{tikzpicture}
	\card
	\cardstrip
	\cardbanner{banner/white.png}
	\cardtitle{Ereignisse (1/4)\quad}
	\cardcontent{Ereignisse können nur in der Kaufphase erworben werden. Dies benötigt 1 Kauf sowie genügend (vorher ausgespielte) Geldwerte. Erwirbst du ein Ereignis mit Schulden \hex, nimmst du die entsprechende Anzahl \hex-Marker an dich. Die Kosten (\hex \coin) sind auf jedem Ereignis oben links zu finden. Sobald du ein Ereignis erwirbst, führst du die darauf beschriebene Anweisung aus. Du nimmst das Ereignis aber \emph{nicht} an dich.

	\bigskip
 
	\emph{Aufstieg:} Wenn du keine Aktionskarte entsorgst passiert nichts weiter.

	\medskip
 
	\emph{Erforschen:} Jeder Erwerb eines \emph{ERFORSCHEN} gibt dir den Kauf zurück, den du für den Erwerb benötigt hast. Mit \coin[7] und 1 Kauf kannst du zum Beispiel 2 \emph{ERFORSCHEN} erwerben und dann eine Karte kaufen oder ein Ereignis für \coin[3] erwerben.}
\end{tikzpicture}
\hspace{-0.6cm}
\begin{tikzpicture}
	\card
	\cardstrip
	\cardbanner{banner/white.png}
	\cardtitle{Ereignisse (2/4)\quad}
	\cardcontent{\emph{Steuer:} Auf jeden Vorratsstapel (d.h. alle Königreichkarten, Fluchkarten und Basiskarten, nicht Ereignisse und Landmarken) wird in der Spielvorbereitung 1 \hex-Marker gelegt. Spieler, die eine Karte von einem Stapel kaufen, auf dem \hex-Marker liegen, müssen alle Marker des Stapels nehmen. Nimmt ein Spieler eine Karte auf andere Art und Weise (d.h. er kauft sie nicht), werden eventuelle \hex-Marker auf die nächste Karte des Vorratsstapels gelegt. Wenn du dieses Ereignis erwirbst, legst du 2 \hex-Marker auf einen beliebigen Vorratsstapel – egal ob dort zu diesem Zeitpunkt bereits \hex-Marker liegen oder nicht.

	\medskip
 
	\emph{Bankett:} Du kannst dieses Ereignis auch kaufen, wenn der \emph{KUPFER}-Vorratsstapel aufgebraucht ist.

	\medskip
 
	\emph{Versalztes Land:} Wenn die entsorgte Karte eine Anweisung beinhaltet, die eintritt, wenn diese Karte entsorgt wird, musst du diese Anweisung ausführen.

	\medskip
 
	\emph{Ritual:} Wenn du keinen \emph{FLUCH} nehmen kannst (z.B. weil der Vorratsstapel leer ist), passiert nichts. Es werden nur die \coin-Kosten gezählt – für \hex-Kosten oder \potion-Kosten (aus \emph{Alchemisten}) erhältst du nichts.}
\end{tikzpicture}
\hspace{-0.6cm}
\begin{tikzpicture}
	\card
	\cardstrip
	\cardbanner{banner/white.png}
	\cardtitle{Ereignisse (3/4)\quad}
	\cardcontent{\emph{Glücksfall:} Wenn weniger als 3 \emph{GOLD} im Vorrat sind, nimm dir die restlichen \emph{GOLD}.

	\medskip
 
	\emph{Eroberung:} Pro \emph{SILBER}, das du in diesem Zug genommen hast (inklusive der 2 \emph{SILBER} durch diese Karte), nimm dir einen \victorypointtoken-Marker vom Vorrat. Dies ist kumulativ. Erwirbst du z.B. eine \emph{EROBERUNG} und erhältst dafür 2 \victorypointtoken-Marker (für die beiden SILBER durch diese Karte) und dann noch eine \emph{EROBERUNG}, für die du 2 \emph{SILBER} nehmen kannst, erhältst du für die zweite \emph{EROBERUNG} schon 4 \victorypointtoken-Marker. Sind nicht genügend \emph{SILBER} im Vorrat, nimmst du dir so viele wie möglich. Dann erhältst du aber auch entsprechend weniger \victorypointtoken-Marker.

	\medskip
 
	\emph{Beherrschen:} Ist der \emph{PROVINZ}-Vorratsstapel leer oder du kannst aus einem anderen Grund keine \emph{PROVINZ} nehmen, hat dieses Ereignis keine Auswirkung.

	\medskip
 
	\emph{Hochzeit:} Den \victorypointtoken-Marker nimmst du in jedem Fall – auch wenn der \emph{GOLD}-Vorratsstapel leer ist.}
\end{tikzpicture}
\hspace{-0.6cm}
\begin{tikzpicture}
	\card
	\cardstrip
	\cardbanner{banner/white.png}
	\cardtitle{Ereignisse (4/4)\quad}
	\cardcontent{\emph{Siegeszug:} Wenn du ein \emph{ANWESEN} nimmst, nimmst du für jede Karte, die du in diesem Zug bereits genommen hast (inklusive dem \emph{ANWESEN} jedoch nicht für Ereignisse), einen \victorypointtoken-Marker. Wenn du kein \emph{ANWESEN} nehmen kannst (z.B. weil der Vorratsstapel leer ist), passiert nichts.

	\medskip
 
	\emph{Schlacht:} Du kannst dieses Ereignis auch erwerben wenn der \emph{HERZOGTUM}-Vorratsstapel leer ist. Die bis zu ausgewählten 5 Karten verbleiben in deinem Ablagestapel. Die restlichen Karten mischst du in deinen Nachziehstapel.

	\medskip
 
	\emph{Spende:} Befinden sich unter den entsorgten Karten welche, die Anweisungen beinhalten, die beim Entsorgen ausgeführt werden, musst du diese ausführen, bevor du die restlichen Karten mischst. Die \emph{SPENDE} wird erst nach dem Zug, in dem sie erworben wird, ausgeführt (d.h. zwischen zwei Zügen). Damit hat zum Beispiel die \emph{BESESSENHEIT} (aus \emph{Alchemisten}) auf diese Anweisung keine Auswirkung.}
\end{tikzpicture}
\hspace{-0.6cm}
\begin{tikzpicture}
	\card
	\cardstrip
	\cardbanner{banner/green.png}
	\cardtitle{\footnotesize{Landmarken (1/8)}\quad}
	\cardcontent{Einige Landmarken enthalten Anweisungen für die Spielvorbereitung (unterhalb der Trennlinie). Spielt ihr mit einer dieser Karten, beachtet dies in der Spielvorbereitung.

	\medskip
 
	Darfst du dir auf Grund einer Anweisung \victorypointtoken-Marker von einer Landmarkenkarte oder einem Vorratsstapel nehmen und dort sind zu diesem Zeitpunkt keine \victorypointtoken-Marker vorhanden, erhältst du nichts. Sind die zu Spielbeginn platzierten \victorypointtoken-Marker aufgebraucht, werden keine neuen \victorypointtoken-Marker platziert.

	\bigskip
 
	\emph{Aquädukt:} Wenn du eine Geldkarte von einem Vorratsstapel nimmst, auf dem ein oder mehrere \victorypointtoken-Marker liegen (auch ggf. kombinierte Karten oder \emph{KUPFER}, wenn dort durch Anweisungen auf Karten oder Ereignissen \victorypointtoken-Marker platziert wurden), nimm einen \victorypointtoken-Marker und lege ihn hierher auf das \emph{AQUÄDUKT}.

	\smallskip
 
	Wenn du eine Punktekarte (auch ggf. kombinierte) nimmst, nimm dir alle \victorypointtoken{\ }-Marken, die zu diesem Zeitpunkt hier auf dem \emph{AQUÄDUKT} liegen.

	\smallskip
 
	Wenn du eine kombinierte Geld- und Punktekarte nimmst, kannst du dich entscheiden, in welcher Reihenfolge du die Anweisungen ausführst.}
\end{tikzpicture}
\hspace{-0.6cm}
\begin{tikzpicture}
	\card
	\cardstrip
	\cardbanner{banner/green.png}
	\cardtitle{\footnotesize{Landmarken (2/8)}\quad}
	\cardcontent{\emph{Arena:} Beginnst du (z.B. durch die \emph{VILLA}) in deinem Zug mehrfach mit deiner Kaufphase, kannst du die \emph{ARENA} mehrfach nutzen.

	\medskip
 
	\emph{Badehaus:} Egal ob du eine Karte kaufst oder auf andere Art und Weise nimmst (bzw. nehmen musst) – erhältst du in diesem Fall keine \victorypointtoken{\ }-Marker vom \emph{BADEHAUS}. Wer ein Ereignis erwirbt, nimmt damit keine Karte und kann – insofern keine andere Karte genommen wurde - 2 \victorypointtoken-Marker von hier nehmen.

	\medskip
 
	\emph{Basilika:} Für jede Karte die du kaufst, nimmst du 2 \victorypointtoken-Marker von der \emph{BASILIKA}, falls du zu diesem Zeitpunkt mindestens \coin[2] ausgespielt aber noch nicht verbraucht hast. Hast du beispielsweise \coin[4] und 3 Käufe, kannst du ein \emph{KUPFER} kaufen (\coin[4] übrig), dir 2 \victorypointtoken-Marker nehmen, ein \emph{ANWESEN} kaufen (\coin[2] übrig), dir 2 \victorypointtoken-Marker nehmen und ein weiteres \emph{ANWESEN} kaufen (\emph{0} übrig) – für den letzten Kauf erhältst du keine \victorypointtoken-Marker.}
\end{tikzpicture}
\hspace{-0.6cm}
\begin{tikzpicture}
	\card
	\cardstrip
	\cardbanner{banner/green.png}
	\cardtitle{\footnotesize{Landmarken (3/8)}\quad}
	\cardcontent{\emph{Bollwerk:} Hier werden alle Geldkarten (auch ggf. kombinierte) ausgewertet, die im Spiel benutzt wurden (auch ggf. Geldkarten, die im \emph{SCHWARZMARKT} (aus Basisspiel \emph{Special Edition} bzw. \emph{Promokarte}) enthalten waren). Haben zwei oder mehrere Spieler die gleiche höchste Anzahl einer Geldkarte, erhalten alle diese Spieler 5 \victorypointtoken-Marker.

	\medskip
 
	\emph{Brunnen:} Du erhältst entweder 15 \victorypoint oder 0 \victorypoint. Es gibt keinen Extra-Bonus, wenn du mehr als 10 \emph{KUPFER} besitzt.

	\medskip
 
	\emph{Entweihter Schrein:} Immer wenn du eine beliebige Aktionskarte nimmst und auf dem entsprechenden Vorratsstapel ein oder mehrere \victorypointtoken-Marker liegen (egal ob sie dort auf Grund der Anweisung auf dieser Landmarken-Karte oder einer anderen Karte, Ereignis oder Landmarken-Karte liegen), nimm einen \victorypointtoken-Marker von dort und lege ihn hierher auf den \emph{ENTWEIHTEN SCHREIN}.

	\smallskip
 
	Nur wenn du einen \emph{FLUCH} kaufst (nicht, wenn du ihn auf andere Art und Weise nimmst), nimmst du alle \victorypointtoken-Marker, die zu diesem Zeitpunkt hier liegen.

	In der Spielvorbereitung legt ihr auf jeden Vorratsstapel, der den Typ AKTION, nicht aber den Typ SAMMLUNG (also nicht auf die Karten \emph{BAUERNMARKT}, \emph{TEMPEL} und \emph{WILDE JAGD}) beinhaltet, 2 \victorypointtoken-Marker.}
\end{tikzpicture}
\hspace{-0.6cm}
\begin{tikzpicture}
	\card
	\cardstrip
	\cardbanner{banner/green.png}
	\cardtitle{\footnotesize{Landmarken (4/8)}\quad}
	\cardcontent{\emph{Gebirgspass:} Diese Landmarken-Karte wird genau einmal pro Spiel ausgeführt – nämlich nach Beendigung des Zuges, in dem ein Spieler die erste \emph{PROVINZ} aus dem Vorrat nimmt. Entsorgt vorher ein Spieler bereits eine \emph{PROVINZ} (z.B. durch das Ereignis \emph{VERSALZTES LAND}), hat jener Spieler diese \emph{PROVINZ} aber nicht vorher genommen und erfüllt deshalb diese Bedingung auch noch nicht. In einem Spiel, indem keine \emph{PROVINZ} genommen wird, findet diese Landmarken-Karte keine Anwendung.

	\smallskip
 
	Der \emph{GEBIRGSPASS} wird zwischen zwei Zügen ausgeführt und kann damit z.B. von der \emph{BESESSENHEIT} (aus \emph{Alchemisten}) nicht beeinflusst werden. Der Mitspieler links von dem Spieler, der die erste \emph{PROVINZ} genommen hat, beginnt mit einem Gebot oder passt. Ein Gebot besteht aus einer Anzahl \hex zwischen \hex[1] und \hex[\hspace{-0.25em}40]. Der nächste Spieler muss mindestens \hex[1] mehr bieten als der vorherige oder passen. Ein Gebot von \hex[\hspace{-0.25em}40] kann nicht überboten werden. Haben alle Spieler ein Gebot abgegeben oder gepasst, bzw. wurde bereits das Höchstgebot von \hex[\hspace{-0.25em}40] erreicht, erhält der Spieler mit dem höchsten Gebot die entsprechende Anzahl \hex-Marker sowie 8 \victorypointtoken-Marker. Passen alle Spieler, erhält keiner etwas.}
\end{tikzpicture}
\hspace{-0.6cm}
\begin{tikzpicture}
	\card
	\cardstrip
	\cardbanner{banner/green.png}
	\cardtitle{\footnotesize{Landmarken (5/8)}\quad}
	\cardcontent{\emph{Grabmal:} Dies funktioniert auch außerhalb deines Zuges (z.B. mit dem \emph{TRICKSER} aus \emph{Intrige}) oder wenn du eine Karte entsorgst, die nicht deine eigene ist (z.B. durch das Ereignis \emph{VERSALZTES LAND}).

	\medskip
 
	\emph{Kolonnaden:} Wenn du eine Aktionskarte kaufst (nicht, wenn du sie auf andere Art und Weise nimmst), musst du eine Karte mit dem gleichen Namen bereits im Spiel haben, um 2 \victorypointtoken-Marker von hier zu erhalten. Karten eines Stapels haben nicht unbedingt alle den gleichen Namen (z.B. bei gemischten Stapeln).

	\medskip
 
	\emph{Labyrinth:} Dies kann nur einmal pro Zug eines Spielers eintreten, nämlich genau in dem Moment, in dem er die zweite Karte in seinem Zug nimmt. Nimmt er außerhalb seines Zuges zwei Karten, erhält er nichts.

	\medskip
 
	\emph{Mauer:} Hast du mehr als 15 Karten in deinem Kartensatz, bekommst für jede Karte darüber hinaus 1 \victorypoint. Spieler, die zum Beispiel 27 Karten im Kartensatz haben, erhalten -12 \victorypoint, Spieler mit 14 Karten im Kartensatz erhalten keinen \victorypoint Abzug. Die Gesamtpunktzahl kann damit auch negativ sein.}
\end{tikzpicture}
\hspace{-0.6cm}
\begin{tikzpicture}
	\card
	\cardstrip
	\cardbanner{banner/green.png}
	\cardtitle{\footnotesize{Landmarken (6/8)}\quad}
	\cardcontent{\emph{Museum:} Auch Karten, die vom gleichen Stapel stammen, aber unterschiedliche Namen haben (z.B. gemischte Stapel), werden mit jeweils 2 \victorypoint abgerechnet.

	\medskip
 
	\emph{Obelisk:} Es zählen alle Karten des gewählten Stapels, auch wenn sie unterschiedliche Namen haben (z.B. bei gemischten Stapeln).

	\smallskip
 
	Zu Spielbeginn ermittelt ihr einen zufälligen Stapel, der den Typ AKTION beinhaltet (auch ggf. kombinierte Karten) und zum Vorrat gehört. \emph{RUINEN} (aus \emph{Dark Ages}) können bestimmt werden, ebenfalls der Stapel, der auch als Bannstapel für die \emph{JUNGE HEXE} (aus \emph{Reiche Ernte}) genutzt wird. Dazu zählen jedoch nicht die Eintausch- und Preiskarten (aus \emph{Reiche Ernte}), da diese nicht zum Vorrat gehören.

	\medskip
 
	\emph{Obstgarten:} Du erhältst keinen zusätzlichen Bonus, wenn du zum Beispiel von einer Aktionskarte 6 Exemplare besitzt, d.h. du erhältst für eine Aktionskarte, von der du 3 Exemplare besitzt genauso 4 \victorypoint wie für eine, von der du 7 Exemplare besitzt.}
\end{tikzpicture}
\hspace{-0.6cm}
\begin{tikzpicture}
	\card
	\cardstrip
	\cardbanner{banner/green.png}
	\cardtitle{\footnotesize{Landmarken (7/8)}\quad}
	\cardcontent{\emph{Palast:} Wenn du bei Spielende beispielsweise 7 \emph{KUPFER}, 5 \emph{SILBER} und 2 \emph{GOLD} in deinem Kartensatz hast, erhältst du 6 \victorypoint, da du zwei komplette Sätze aus je 1 \emph{KUPFER}, \emph{SILBER} und \emph{GOLD} besitzt. Hättest du noch ein drittes \emph{GOLD}, würdest du 9\victorypoint  erhalten.

	\medskip
 
	\emph{Räuberfestung:} Hast du bei Spielende zum Beispiel 3 \emph{SILBER} und 1 \emph{GOLD} in deinem Kartensatz, werden dir 8 \victorypoint abgezogen. Die Gesamtpunktzahl kann damit auch negativ sein.

	\medskip
 
	\emph{Schlachtfeld:} Du erhältst 2 \victorypointtoken-Marker von hier, egal ob du die Punktekarte (auch ggf. kombinierte) kaufst oder auf andere Art und Weise nimmst. Dies funktioniert auch außerhalb deines Zuges. Falls mehrere Spieler eine Punktekarte nehmen, wird dies in Spielerreihenfolge (beginnend bei dem Spieler links des aktuellen Spielers) getan.

	\medskip
 
	\emph{Triumphbogen:} Wenn du bei Spielende beispielsweise 7 \emph{VILLEN} und 4 \emph{WILDE JAGDEN} (und keine andere (auch ggf. kombinierte) Aktionskarte häufiger) in deinem Kartensatz hast, erhältst du 12 \victorypoint (d.h. 3 \victorypoint für jede der 4 \emph{WILDE JAGDEN}). Hast du neben 7 \emph{VILLEN} auch 7 \emph{WILDE JAGDEN}, erhältst du für beide zusammen 21 \victorypoint.}
\end{tikzpicture}
\hspace{-0.6cm}
\begin{tikzpicture}
	\card
	\cardstrip
	\cardbanner{banner/green.png}
	\cardtitle{\footnotesize{Landmarken (8/8)}\quad}
	\cardcontent{\emph{Turm:} Der Vorratsstapel muss leer sein. Ein gemischter Stapel, bei dem nur eine Sorte Karten fehlt, zählt nicht. Die Vorratsstapel mit Punktekarten zählen ebenfalls nicht, ein leerer Fluch-Stapel aber schon.

	\medskip
 
	\emph{Wolfsbau:} Du bekommst keine Minuspunkte durch den \emph{WOLFSBAU}, wenn du von einer Karte gar keine bzw. zwei oder mehr Stück in deinem kompletten Kartensatz besitzt. Hast du zum Beispiel einen \emph{FLUCH} in deinem Nachziehstapel und einen in deinem Ablagestapel, hast du insgesamt zwei \emph{FLÜCHE} und erhältst keine Minuspunkte durch den \emph{WOLFSBAU}. Die Gesamtpunktzahl kann negativ sein.}
\end{tikzpicture}
\hspace{-0.6cm}
\begin{tikzpicture}
	\card
	\cardstrip
	\cardbanner{banner/white.png}
	\cardtitle{\scriptsize{Empfohlene 10er Sätze\qquad}}
	\cardcontent{\emph{Basis Einführung} (\underline{Ereignisse und Landmarken-Karten}):\\
	\underline{Turm}, \underline{Hochzeit}, Schlösser (alle 8 bzw. 12 Schlosskarten), Wagenrennen, Stadtviertel, Ingenieurin, Bauernmarkt, Forum, Legionär, Patrizier/Handelsplatz, Opfer, Villa

	\smallskip

	\emph{Fortgeschrittene Einführung} (\underline{Ereignisse und Landmarken-Karten}):\\
	\underline{Arena}, \underline{Triumphbogen}, \underline{Hochzeit}, \underline{Spende}, Archiv, Vermögen, Katapult/Felsen, Krone, Zauberin, Gladiator/Reichtum, Gärtnerin, Königlicher Schmied, Siedler/Emsiges Dorf, Tempel

	\smallskip

	\emph{Alles in Maßen} (Empires + \underline{Ereignisse und Landmarken-Karten} + \textit{Basisspiel}):\\
	\underline{Obstgarten}, \underline{Glücksfall}, Zauberin, Forum, Legionär, Lehnsherr, Tempel, \textit{Keller}, \textit{Bibliothek}, \textit{Umbau}, \textit{Dorf}, \textit{Werkstatt}

	\smallskip

	\emph{Silberne Kugeln} (Empires + \underline{Ereignisse und Landmarken-Karten} + \textit{Basisspiel}):\\
	\underline{Aquädukt}, \underline{Eroberung}, Katapult/Felsen, Zauber, Bauernmarkt, Gärtnerin, Patrizier/Handelsplatz, \textit{Bürokrat}, \textit{Gärtner}, \textit{Laboratorium}, \textit{Markt}, \textit{Geldverleiher}}
\end{tikzpicture}
\hspace{-0.6cm}
\begin{tikzpicture}
	\card
	\cardstrip
	\cardbanner{banner/white.png}
	\cardtitle{\scriptsize{Empfohlene 10er Sätze\qquad}}
	\cardcontent{\emph{Köstliche Folter} (Empires + \underline{Ereignisse und Landmarken-Karten} + \textit{Intrige}):\\
	\underline{Arena}, \underline{Bankett}, Schlösser (alle 8 bzw. 12 Schlosskarten), Krone, Gärtnerin, Opfer, Siedler/Emsiges Dorf, \textit{Baron}, \textit{Brücke}, \textit{Harem}, \textit{Eisenhütte}, \textit{Kerkermeister}

	\smallskip

	\emph{Buddy-Prinzip} (Empires + \underline{Ereignisse und Landmarken-Karten} + \textit{Intrige}):\\
	\underline{Aussaat}, \underline{Wolfsbau}, Archiv, Vermögen, Katapult/Felsen, Ingenieurin, Forum, \textit{Maskerade}, \textit{Bergwerk}, \textit{Adlige}, \textit{Handlanger}, \textit{Handelsposten}

	\smallskip

	\emph{Kontrollbereich} (Empires + \underline{Ereignisse und Landmarken-Karten} + \textit{Abenteuer}):\\
	\underline{Bankett}, \underline{Bollwerk}, Vermögen, Katapult/Felsen, Zauber, Krone, Bauernmarkt, \textit{Königliche Münzen}, \textit{Page}, \textit{Relikt}, \textit{Schatz}, \textit{Weinhändler}

	\smallskip

	\emph{Kein Geld, keine Probleme} (Empires + \underline{Ereignisse und Landmarken-Karten} + \textit{Abenteuer}):\\
	\underline{Räuberfestung}, Archiv, Feldlager/Diebesgut, Königlicher Schmied, Tempel, Villa, \textit{Mission}, \textit{Verlies}, \textit{Duplikat}, \textit{Gefolgsmann}, \textit{Kleinbauer}, \textit{Transformation}}
\end{tikzpicture}
\hspace{0.6cm}

	    % Basic settings for this card set
\renewcommand{\cardcolor}{nocturne}
\renewcommand{\cardextension}{Erweiterung X}
\renewcommand{\cardextensiontitle}{Nocturne}

\clearpage
\newpage
\section{\cardextension \ - \cardextensiontitle}

\begin{tikzpicture}
	\card
	\cardstrip
	\cardbanner{banner/white.png}
	\cardicon{banner/coin.png}
	\cardprice{}
	\cardtitle{}
	\cardcontent{}
\end{tikzpicture}

\hspace{-0.6cm}
\begin{tikzpicture}
	\card
	\cardstrip
	\cardbanner{banner/white.png}
	\cardtitle{\scriptsize{Empfohlene 10er Sätze\qquad}}
	\cardcontent{\emph{Name:}\\
	Karten ...

	\smallskip

	\emph{Name:}\\
	Karten ...

	\smallskip

	\emph{Name:}\\
	Karten ...

	\smallskip

	\emph{Name:}\\
	Karten ...

	\smallskip

	\emph{Name:}\\
	Karten ...

	\smallskip

	\emph{Name:}\\
	Karten ...}
\end{tikzpicture}
\hspace{0.6cm}

	    % Basic settings for this card set
\renewcommand{\cardcolor}{promo}
\renewcommand{\cardextension}{Promokarte}
\renewcommand{\cardextensiontitle}{}
\renewcommand{\seticon}{empty.png}

\clearpage
\newpage
\section{\cardextension}

\begin{tikzpicture}
	\card
	\cardstrip
	\cardbanner{banner/white.png}
	\cardtitle{Platzhalter\quad}
\end{tikzpicture}
\hspace{-0.6cm}
\begin{tikzpicture}
	\card
	\cardstrip
	\cardbanner{banner/white.png}
	\cardicon{icons/coin.png}
	\cardprice{4}
	\cardtitle{Gesandter}
	\cardcontent{Die aufgedeckten Karten legst du zunächst offen vor dir aus. Kannst du auch nach dem Mischen deines Ablagestapels nur weniger als 5 Karten aufdecken, deckst du nur so viele auf, wie möglich. Dann wählt der Spieler links von dir eine dieser offenen Karten. Die gewählte Karte legst du auf deinen Ablagestapel, die restlichen Karten nimmst du auf die Hand.}
\end{tikzpicture}
\hspace{-0.6cm}
\begin{tikzpicture}
	\card
	\cardstrip
	\cardbanner{banner/white.png}
	\cardicon{icons/coin.png}
	\cardprice{3}
	\cardtitle{\footnotesize{Schwarzmarkt}}
	\cardcontent{\miniscule{Vor dem Spiel mit der Aktionskarte Schwarzmarkt muss der dazugehörige Schwarzmarkt-Stapel zusammengestellt werden. Die Spieler einigen sich welche Karten verwendet werden sollen.

	\medskip

	Für die Zusammenstellung ist Folgendes zu beachten:
	\begin{itemize}
	\item Der Schwarzmarkt-Stapel muss aus mindestens 15 Karten bestehen.
	\item Es dürfen nur Königreichkarten verwendet werden, die nicht im Vorrat sind.
	\item Jede Karte darf nur einmal verwendet werden.
	\end{itemize}
	Die Spieler dürfen sich die Karten vor dem Spiel ansehen. Dann werden die verwendeten Karten gemischt und verdeckt neben dem Vorrat bereit gelegt. Der Schwarzmarkt-Stapel ist nicht Teil des Vorrats. Er wird weder für die Spielende-Bedingung beachtet, noch für andere Zwecke, die auf den Vorrat Bezug nehmen, z.B. Aktionskarten, die erlauben eine Karte zu nehmen (Werkstatt).

	\medskip

	Spielst du den Schwarzmarkt aus, musst du zunächst die obersten 3 Karten vom Schwarzmarkt-Stapel aufdecken. Nun darfst du eine dieser Karten kaufen. Es handelt sich hierbei um einen Kauf in der Aktionsphase, d. h. du darfst sowohl Geldkarten als auch virtuelles Geld verwenden. Der Kauf läuft also in gleicher Weise ab, wie in der Kaufphase. (Der einzige Unterschied ist, dass du nicht wie üblich eine Karte aus dem Vorrat kaufst, sondern eine der 3 aufgedeckten Karten vom Schwarzmarktstapel.) Die nicht gekauften Karten legst du in beliebiger Reihenfolge verdeckt unter den Schwarzmarkt-Stapel zurück. Du musst deinen Mitspielern nicht zeigen in welcher Reihenfolge du die Karten zurücklegst. Die ausgelspielten Geldkarten lässt du bis zur Aufräumphase vor dir liegen. Dieser Kauf verbraucht nicht deinen freien Kauf, du darfst also in der Kaufphase (mindestens) eine weitere Karte kaufen. Nicht verwendetes virtuelles Geld und auch überzählige Münzen ausgespielter Geldkarten stehen dir in der Kaufphase zur Verfügung.

	\medskip

	Wenn du den Schwarzmarkt ausspielst, aber keine Karte kaufen willst oder kannst (z.B. weil der Schwarzmarktstapel leer ist oder du nicht genügend Geld hast), erhältst du trotzdem +\coin[2] für die Kaufphase.}}
\end{tikzpicture}
\hspace{-0.6cm}
\begin{tikzpicture}
	\card
	\cardstrip
	\cardbanner{banner/gold.png}
	\cardicon{icons/coin.png}
	\cardprice{5}
	\cardtitle{Geldversteck}
	\cardcontent{Das Geldversteck ist eine Geldkarte mit dem Wert \coin[2], wie ein Silber. In der Aufräumphase wird das Geldversteck wie üblich abgelegt. Immer wenn ein Spieler seinen Ablagestapel mischt, führt er folgende Schritte aus:
	\begin{noindlist}
	\item Er sucht zunächst alle Geldversteck-Karten aus seinem Ablagestapel und legt sie beiseite.
	\item Dann mischt er die verbliebenen Karten des Ablagestapels.
	\item Nun sortiert er die Karten Geldversteck wieder an beliebigen Stellen seiner Wahl (auch ganz oben oder unten) in den Stapel ein. Dabei darf er Karten des Stapels abzählen, aber nicht ansehen.
	\item Zuletzt legt er den Stapel als neuen Nachziehstapel bereit.
	\end{noindlist}}
\end{tikzpicture}
\hspace{-0.6cm}
\begin{tikzpicture}
	\card
	\cardstrip
	\cardbanner{banner/white.png}
	\cardicon{icons/coin.png}
	\cardprice{4}
	\cardtitle{Carcassonne}
	\cardcontent{\emph{Errata:} Der Kartentext ist falsch, es sollte \enquote{Wenn du zu Beginn deiner Aufräumphase nicht mehr als eine weitere Aktionskarte im Spiel hast\dots} statt \enquote{Wenn du zu Beginn deiner Aufräumphase nur noch eine weitere Aktionskarte im Spiel hast\dots} heißen.

	\medskip

	Zuerst ziehst du immer eine Karte nach und erhältst +2 Aktionen. Wenn du zu Beginn deiner Aufräumphase die Karte Carcassonne und nicht mehr als eine weitere Aktionskarte im Spiel hast, darfst du dich entscheiden, Carcassonne oben auf deinen Nachziehstapel zu legen oder wie üblich auf den Ablagestapel zu legen. Hast du die Karte Carcassonne genau zweimal im Spiel und ansonsten keine weitere Aktionskarte im Spiel, darfst du eine oder beide Karten Carcassonne auf deinen Nachziehstapel legen.}
\end{tikzpicture}
\hspace{-0.6cm}
\begin{tikzpicture}
	\card
	\cardstrip
	\cardbanner{banner/white.png}
	\cardicon{icons/coin.png}
	\cardprice{4}
	\cardtitle{\scriptsize{Befestigtes Dorf}}
	\cardcontent{Zuerst ziehst du immer eine Karte nach und erhältst +2 Aktionen. Wenn du zu Beginn deiner Aufräumphase die Karte Befestigtes Dorf und nicht mehr als eine weitere Aktionskarte im Spiel hast, darfst du dich entscheiden, Befestigtes Dorf oben auf deinen Nachziehstapel zu legen oder wie üblich auf den Ablagestapel zu legen. Hast du die Karte Befestigtes Dorf genau zweimal im Spiel und ansonsten keine weitere Aktionskarte im Spiel, darfst du eine oder beide Karten Befestigtes Dorf auf deinen Nachziehstapel legen.}
\end{tikzpicture}
\hspace{-0.6cm}
\begin{tikzpicture}
	\card
	\cardstrip
	\cardbanner{banner/white.png}
	\cardicon{icons/coin.png}
	\cardprice{5}
	\cardtitle{Gouverneur}
	\cardcontent{Zuerst erhältst du +1 Aktion. Dann wählst du eine der folgenden Optionen:
	\begin{noindlist}
	\item Du ziehst 3 Karten und jeder andere Spieler zieht 1 Karte.
	\item Du nimmst dir 1 Gold und jeder andere Spieler nimmt sich 1 Silber.
	\item Du darfst 1 Karte aus deiner Hand entsorgen und nimmst dir 1 Karte, die genau
	\coin[2] mehr kostet als die entsorgte Karte und jeder andere Spieler darf 1 Karte entsorgen, die genau \coin[1] mehr kostet als die entsorgte Karte.
	\end{noindlist}
	Geh nach der Reihenfolge, beginnend mit dir selbst. Die Karten werden vom Vorrat genommen und auf den Ablagestapel gelegt. Sind im Vorrat keine Karten mehr übrig, können keine entsprechenden Karten genommen werden. Wählst du z.B. die zweite Option und es ist nur noch 1 Silber im Vorrat, bekommt es der Spieler links von dir und die anderen Spieler erhalten nichts. Bei der dritten Option nimmst du dir nur dann 1 Karte, wenn du vorher 1 Karte entsorgt hast und wenn 1 Karte mit den genauen geforderten Kosten im Vorrat verfügbar ist. Wenn du 1 Karte entsorgst, musst du dir 1 Karte nehmen, sofern möglich. Du kannst den ausgespielten Gouverneur nicht entsorgen, da er sich nicht mehr auf deiner Hand befindet. Du kannst aber einen anderen Gouverneur aus deiner Hand entsorgen.}
\end{tikzpicture}
\hspace{-0.6cm}
\begin{tikzpicture}
	\card
	\cardstrip
	\cardbanner{banner/white.png}
	\cardicon{icons/coin.png}
	\cardprice{8}
	\cardtitle{Prinz}
	\cardcontent{\tiny{\begin{Spacing}{1}
	\vspace{1em}
	Wenn du dich entscheidest, einen Prinzen zu spielen, legst du ihn sofort zur Seite. Er befindet sich damit allerdings nicht im Spiel. Anschließend wählst du eine Aktionskarte von deiner Hand, die zu diesem Zeitpunkt maximal \coin[4] kostet und legst diese ebenfalls zur Seite.

	\medskip

	Auch Karten mit Kosten von \coin[0], wie die Preiskarten aus Reiche Ernte sowie Karten mit Kosten von \coin[X] + aus Die Gilden dürfen mit Hilfe des Prinzen zur Seite gelegt werden. Karten, deren Kosten einen Trank enthalten, dürfen dagegen nicht zur Seite gelegt werden.

	\medskip

	Zu Beginn eines Zuges musst du die zur Seite gelegte Aktionskarte spielen. Sie verbraucht dabei nicht deine freie Aktion. Sobald du die Aktionskarte ablegen musst, legst du sie stattdessen wieder zur Seite. Wenn du die Aktionskarte während deines Spielzugs aus dem Spielbereich entfernst (z.B. sie entsorgen musst) und sie dementsprechend in der Aufräumphase nicht mehr im Spiel ist, darfst du die Karte nicht wieder zur Seite legen. Die Wirkung des Prinzen wird sofort aufgehoben.

	\medskip

	Wenn du zu Beginn deines Zuges mehrere Karten spielen musst (z.B. Aktionskarten durch mehrere Prinzen oder Dauerkarten aus Seaside), darfst du selbst entscheiden, in welcher Reihenfolge du sie ausspielst. Die Karte \emph{PRINZ} muss zur Seite gelegt werden, damit sie einen Effekt hat. Den Prinzen zum Beispiel auf einen Thronsaal zu spielen erlaubt dir nicht, zwei Karten zur Seite zu legen, da du den Prinzen nur einmal zur Seite legen kannst. Alle zur Seite gelegten Prinzen und Aktionskarten gehören zum Kartensatz eines Spielers.
	\end{Spacing}}}
\end{tikzpicture}
\hspace{-0.6cm}
\begin{tikzpicture}
	\card
	\cardstrip
	\cardbanner{banner/white.png}
	\cardtitle{Sauna/Eisloch\qquad}
	\cardcontent{Spielvorbereitung: Legt auf diese Karte 5 Eisloch und oben darauf 5 Sauna.

	\medskip

	Es darf immer nur die oberste Karte des Stapels genommen oder gekauft werden.}
\end{tikzpicture}
\hspace{-0.6cm}
\begin{tikzpicture}
	\card
	\cardstrip
	\cardbanner{banner/white.png}
	\cardicon{icons/coin.png}
	\cardprice{4}
	\cardtitle{Sauna}
	\cardcontent{\tiny{\begin{Spacing}{1}
	\vspace{1em}
	\emph{Siehe auch die Hinweise zur Karte Eisloch!}

	\medskip

	Wenn du die Sauna ausspielst, ziehst du zuerst eine Karte und bekommst +1 Aktion. Du kannst dann sofort ein Eisloch aus deiner Hand ausspielen. Das verbraucht keine deiner Aktionen, einschließlich der Aktion, die die Sauna gewährt. Du darfst ein Eisloch auf diese Weise nur direkt nach dem Ausspielen der Sauna spielen, nicht, wenn du zwischendurch eine andere Aktionskarte ausgespielt hast, selbst wenn du eine Sauna im Spiel hast.

	\medskip

	Solange die Sauna im Spiel ist, darfst du jedes Mal, wenn du ein Silber ausspielst, eine Karte aus deiner Hand entsorgen. Wenn du das gleiche Silber mehrmals spielst, wie z.B. mit dem Falschgeld (Dominion - Dark Ages) oder der Krone (Dominion - Empires), darfst du jedes Mal eine Karte entsorgen, wenn du das Silber spielst.

	\medskip

	Wenn du ein Silber ausspielst, kannst du dir jedes Mal überlegen, ob du eine Karte entsorgen möchtest, du musst diese Entscheidung nicht einmal für den gesamten Zug treffen. Wenn du mehrere Saunen im Spiel hast und ein Silber ausspielst, kannst du für jede Sauna, die du im Spiel hast, eine Karte aus deiner Hand entsorgen. Du kannst dich jedes Mal immer noch dazu entschließen, keine Karte zu entsorgen.

	\medskip

	Wenn die Sauna das Spiel verlässt, weil sie zum Beispiel mit der Prozession (Dominion - Dark Ages) entsorgt wurde, kann ihr Effekt nicht mehr genutzt werden.
	\end{Spacing}}}
\end{tikzpicture}
\hspace{-0.6cm}
\begin{tikzpicture}
\card
	\cardstrip
	\cardbanner{banner/white.png}
	\cardicon{icons/coin.png}
	\cardprice{5}
	\cardtitle{Eisloch}
	\cardcontent{Wenn du das Eisloch ausspielst, ziehst du zuerst 3 Karten. Du kannst dann sofort eine Sauna aus deiner Hand ausspielen. Das verbraucht keine deiner Aktionen, und du erhältst trotzdem die +1 Aktion der Sauna, wenn du sie auf diese Weise spielst.

	\medskip

	Du darfst eine Sauna auf diese Weise nur direkt nach dem Ausspielen des Eislochs spielen, nicht, wenn du zwischendurch eine andere Aktionskarte ausgespielt hast, selbst wenn du ein Eisloch im Spiel hast.

	\medskip

	\emph{Folgendes gilt sowohl für das Eisloch als auch für die Sauna:}

	\medskip

	Du kannst die Sauna und das Eisloch durch die Effekte der jeweils anderen Karte abwechselnd spielen, wobei du nur die Aktion für die erste ausgespielte Karte verbrauchst. Du kannst damit fortfahren, bis du nach dem Ausspielen der einen Karte die entsprechende andere Karte nicht mehr auf der Hand hast.

	\medskip

	Wenn du eine Sauna ausspielst, kannst du nicht sofort eine weitere Sauna aus deiner Hand ausspielen, ohne eine Aktion zu verbrauchen. Das gleiche gilt für das Ausspielen eines Eislochs nach einem anderen Eisloch.}
\end{tikzpicture}
\hspace{-0.6cm}
\begin{tikzpicture}
	\card
	\cardstrip
	\cardbanner{banner/white.png}
	\cardtitle{Ereignisse\qquad}
	\cardcontent{\tiny{\begin{Spacing}{1}
	\vspace{1em}
	\emph{Einladung:} Wenn du das Ereignis kaufst, nimmst du dir vom Vorrat eine Aktionskarte, die bis zu \coin[4] kostet und legst sie offen zur Seite. Wenn du sie beiseite gelegt hast, dann spielst du die Aktionskarte zu Beginn des nächsten Zuges aus. Das Ausspielen verbraucht nicht deine Standardaktion für den Zug. Um dich daran zu erinnern, dass du die Karte in deinem nächsten Zug ausspielst, kannst du sie seitwärts oder diagonal drehen, und sie dann richtig herum drehen, sobald du sie ausspielst.

	\medskip

	Wenn du die Aktionskarte bewegst, nachdem du sie genommen, aber bevor du sie zur Seite gelegt hast (z.B. indem du sie mit dem Wachturm (Dominion – Blütezeit) auf den Nachziehstapel legst), dann wird die Einladung zu der Aktionskarte den \enquote{Anschluss verlieren} und nicht in der Lage sein, sie zur Seite zu legen; in diesem Fall wirst du sie zu Beginn deines nächsten Zuges nicht ausspielen.

	\medskip

	Wenn du die Einladung nutzt, um ein Nomadencamp (Dominion – Hinterland) zu nehmen, wird die Einladung wissen, dass das Nomadencamp auf deinem Nachziehstapel zu finden ist, so dass du es in diesem Fall zur Seite legst (sofern du es nicht über einen anderen Effekt an eine andere Stelle verschoben hast).

	\medskip

	\emph{Errata:} Der letzte Satz auf der Karte müsste heißen: \enquote{Wenn du das tust, spiele sie zu Beginn deines nächsten Zuges.}
	\end{Spacing}}}
\end{tikzpicture}
\hspace{-0.6cm}
\begin{tikzpicture}
	\card
	\cardstrip
	\cardbanner{banner/white.png}
	\cardicon{icons/coin.png}
	\cardprice{5}
	\cardtitle{Höflinge}
	\cardcontent{
	Decke eine Karte aus deiner Hand auf. Zähle dann die Typen, denen diese  Karte  angehört  –  also  Aktion,  Geld,  Reaktion,  Angriff,  Punkte,  Fluch  etc.  Pro  Typ,  dem  die  Karte  angehört,  entscheidest  du  dich  für  eine  der vier angegebenen Optionen. Dabei darfst du keine der Optionen doppelt auswählen. 
	
		\medskip

	Wenn du zum Beispiel eine \emph{PATROUILLE} aus \emph{Ergänzung - Die Intrige} (Aktion) aufdeckst, darfst du eine Option auswählen, deckst du einen \emph{KARAWANENWÄCHTER} aus \emph{Abenteuer} (Aktion – Dauer – Reaktion) auf, darfst du 3 unterschiedliche Optionen wählen. Entscheidest du dich für das \emph{Gold}, legst du dieses auf den Ablagestapel. Kannst du keine Handkarte aufdecken, erhältst du nichts.}
\end{tikzpicture}
\hspace{-0.6cm}
\begin{tikzpicture}
	\card
	\cardstrip
	\cardbanner{banner/white.png}
	\cardicon{icons/coin.png}
	\cardprice{4}
	\cardtitle{Abbruch}
	\cardcontent{Entsorgen ist nicht optional.
	
			\medskip
			
	Entsorgst du eine Karte, die \coin[0] kostet, oder hast du keine Karte mehr auf der Hand, die du entsorgen könntest, passiert sonst nichts. 

			\medskip
			
	Entsorgst du eine Karte, die \coin[1] oder mehr kostet, nimmst du dir zuerst eine billigere Karte, anschließend ein \emph{GOLD}. Die Karten müssen aus dem Vorrat genommen und in der Reihenfolge, in der sie genommen wurden, auf den Ablagestapel gelegt werden, d.h. das \emph{GOLD} zuletzt. Zwar wird fast immer eine billigere Karte im Vorrat vorhanden sein, da \emph{KUPFER} und \emph{FLUCH} \coin[0] kosten, sollte dies aber einmal nicht der Fall sein, darfst du dir dennoch ein \emph{GOLD} nehmen. Sollte kein \emph{GOLD} mehr im Vorrat vorhanden sein, darfst du dir dennoch die billigere Karte nehmen. 

			\medskip
	
	Karten, die nur Kosten in Form von \potion\ (wie die \emph{VERWANDLUNG} aus \emph{Die Alchemisten}) oder \hex\ aufweisen (wie die \emph{INGENIEURIN} aus \emph{Empires}), kosten nicht \coin[1] oder mehr.}
\end{tikzpicture}
\hspace{-0.6cm}
\begin{tikzpicture}
	\card
	\cardstrip
	\cardbanner{banner/orange.png}
	\cardicon{icons/coin.png}
	\cardprice{3}
	\cardtitle{\scriptsize{Schweriner Dom}}
	\cardcontent{Du kannst keine, eine, zwei oder drei Karten aus deiner Hand verdeckt zur Seite legen, darfst sie aber ansehen.
	
			\medskip
			
	Unabhängig davon, wie viele Karten du zur Seite gelegt hast, kannst du zu Beginn deines nächsten Zuges eine Karte entsorgen.
	
			\medskip
			
	Die Karte, die du entsorgst, kann eine Karte sein, die du zur Seite gelegt hast, oder eine, die du bereits auf der Hand hattest.
	
			\medskip
			
	Spielst du mehrere Male den \emph{SCHWERINER DOM} (oder einen \emph{SCHWERINER DOM} mehrfach, wie z.B. mittels \emph{THRONSAAL} aus dem \emph{Basisspiel}), darfst du entsprechend viele Sätze von jeweils bis zu drei Karten verdeckt zur Seite legen. Zu Beginn deines nächsten Zuges verfährst du folgendermaßen: Nimm dir einen Satz der zur Seite gelegten Karten auf die Hand, anschließend darfst du eine Karte entsorgen, dann wiederhole diese Schritte, bis du alle Sätze auf die Hand genommen hast. Die Reihenfolge, in der du die einzelnen Sätze auf die Hand nimmst, ist frei wählbar.}
\end{tikzpicture}
\hspace{-0.6cm}
\begin{tikzpicture}
	\card
	\cardstrip
	\cardbanner{banner/orange.png}
	\cardicon{icons/coin.png}
	\cardprice{6}
	\cardtitle{\scriptsize{Kapitän Tobias}}
	\cardcontent{\miniscule{\begin{Spacing}{1}
	\vspace{1em}
	Du wählst eine Aktionskarte aus dem Vorrat, die keine Dauerkarte und keine Befehlskarte* ist, und bis zu \coin[\miniscule{4}] kostet, spielst sie und lässt sie im Vorrat liegen. Zu Beginn deines nächsten Zuges wiederholst Du diesen Vorgang; dabei darfst du eine andere Karte auswählen, aber auch dieselbe, sofern diese noch im Vorrat vorhanden ist.
	
	Es kann nur eine Karte aus dem Vorrat gespielt werden, die sichtbar ist und oben auf einem Stapel liegt; weder kann eine Karte von einem leeren Stapel gespielt werden, noch eine Karte von einem gemischten Vorratsstapel, die noch nicht aufgedeckt wurde, oder bereits vergriffen ist, und auch keine Karte, die nicht zum Vorrat gehört (wie bspw.\ der \emph{SÖLDNER} aus \emph{Dark Ages}).
	
	Wenn es in dem Zug, in dem du \emph{KAPITÄN TOBIAS} spielst, im Vorrat keine Aktionskarte gibt, die keine Dauerkarte und keine Befehlskarte ist und bis zu \coin[\miniscule{4}] kostet, bleibt \emph{KAPITÄN TOBIAS} trotzdem im Spiel und du versuchst, zu Beginn deines nächsten Zuges eine solche Karte zu spielen.
	
		Wenn \emph{KAPITÄN TOBIAS} eine Karte spielt, die eine Dauerkarte spielt, beeinflusst das nicht, in welcher Runde \emph{KAPITÄN TOBIAS} aus dem Spiel genommen wird.
	
	\medskip
	
	Die gespielte Aktionskarte bleibt im Vorrat; versucht irgendein Effekt, diese Karte zu bewegen (bspw.\ die \emph{INSEL} aus \emph{Seaside}, die beim Spielen auf dein Insel-Tableau gelegt wird), wird das Bewegen nicht ausgeführt.
	
	\emph{KAPITÄN TOBIAS} kann eine Karte spielen, die sich selbst entsorgt, wenn sie gespielt wird; immer wenn diese Karte überprüft, ob sie entsorgt wurde (wie das \emph{BERGWERK} aus \emph{Die Intrige}), so gilt sie als nicht entsorgt; wenn sie nicht überprüft, ob sie entsorgt wurde (wie die \emph{SCHAUSPIELTRUPPE} aus \emph{Renaissance}), funktioniert sie wie gewohnt.
		
	Karten, die normalerweise andere Karten aus dem Vorrat bewegen, können sich selbst bewegen, wenn sie mittels \emph{KAPITÄN TOBIAS} gespielt werden; bspw.\ kann die \emph{WERKSTATT} aus dem \emph{Basisspiel} sich selbst nehmen und die \emph{Herumtreiberin} aus \emph{Ergänzung - Die Intrige} kann sich selbst entsorgen.
	
	Da die gespielte Karte nicht im Spiel ist, haben Fähigkeiten, die an die Bedingung "`Solange diese Karte im Spiel ist"' geknüpft sind (wie bspw.\ beim \emph{HALSABSCHNEIDER} aus \emph{Blütezeit}), keine Auswirkungen.
	
	\medskip
	
	*Befehlskarten sind ein neuer Kartentyp, der in den Errata 2019 eingeführt wurde, um zu verhindern, dass manche Fähigkeiten in Endlosschleife gespielt werden können. Befehlskarten sind Emulatoren, die Karten aus dem Vorrat spielen, sie aber dort belassen (bspw.\ \emph{VOGELFREIE} aus \emph{Dark Ages} und der \emph{LEHNSHERR} aus \emph{Empires}).
	\end{Spacing}}}
\end{tikzpicture}
\hspace{-0.6cm}
\begin{tikzpicture}
	\card
	\cardstrip
	\cardbanner{banner/white.png}
	\cardtitle{\scriptsize{Spielvorbereitung}\qquad}
	\cardcontent{Promo-Karten nach Belieben zum Aufbau des Königreiches verwenden.}
\end{tikzpicture}
\hspace{0.6cm}

	    % Basic settings for this card set
\renewcommand{\cardcolor}{basicgame}
\renewcommand{\cardextension}{Template}
\renewcommand{\cardextensiontitle}{Template}

\clearpage
\newpage
\section{\cardextension \ - \cardextensiontitle}

\begin{tikzpicture}
	\card
	\cardstrip
	\cardbanner{banner/white.png}
	\cardicon{banner/coin.png}
	\cardprice{}
	\cardtitle{}
	\cardcontent{}
\end{tikzpicture}
\hspace{-1cm}
\begin{tikzpicture}
	\card
	\cardstrip
	\cardbanner{banner/gold.png}
	\cardicon{banner/coin.png}
	\cardprice{}
	\cardtitle{}
	\cardcontent{}
\end{tikzpicture}
\hspace{-1cm}
\begin{tikzpicture}
	\card
	\cardstrip
	\cardbanner{banner/purple.png}
	\cardicon{banner/coin.png}
	\cardprice{}
	\cardtitle{}
	\cardcontent{}
\end{tikzpicture}
\hspace{-1cm}
\begin{tikzpicture}
	\card
	\cardstrip
	\cardbanner{banner/blue.png}
	\cardicon{banner/coin.png}
	\cardprice{}
	\cardiconaddition{banner/hex.png}
	\cardpriceaddition{}
	\cardtitle{}
	\cardcontent{}
\end{tikzpicture}
\hspace{-1cm}
\begin{tikzpicture}
	\card
	\cardstrip
	\cardbanner{banner/green.png}
	\cardicon{banner/coin.png}
	\cardprice{}
	\cardiconaddition{banner/potion.png}
	\cardpriceaddition{}
	\cardtitle{}
	\cardcontent{}
\end{tikzpicture}
\hspace{-1cm}
\begin{tikzpicture}
	\card
	\cardstrip
	\cardbanner{banner/white.png}
	\cardtitle{\scriptsize{Empfohlene 10er Sätze\qquad}}
	\cardcontent{\emph{Name:}
	\\
	Karten ...
	\\
	\smallskip
	\\
	\emph{Name:}
	\\
	Karten ...
	\\
	\smallskip
	\\
	\emph{Name:}
	\\
	Karten ...
	\\
	\smallskip
	\\
	\emph{Name:}
	\\
	Karten ...
	\\
	\smallskip
	\\
	\emph{Name:}
	\\
	Karten ...
	\\
	\smallskip
	\\
	\emph{Name:}
	\\
	Karten ...
	\\}
\end{tikzpicture}
\hspace{1cm}
	\end{center}
\end{document}

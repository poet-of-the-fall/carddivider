%!TEX program = lualatex

\documentclass[a4paper,oneside]{memoir}
\usepackage[margin=0cm,top=1.75cm,bottom=1.75cm]{geometry}
\usepackage[utf8]{inputenc}
\usepackage[german]{babel}
\usepackage{microtype}
\usepackage{graphicx}
\usepackage{color}
\usepackage{xcolor}
\usepackage{tikz}
\usepackage{rotating}
\usepackage[babel,german=quotes]{csquotes}
\usepackage{xifthen}

% Colors for basic elements
\definecolor{framebg}{RGB}{0,0,0}
\definecolor{contentbg}{RGB}{255,255,255}

% Colors to distinguish the extensions
\definecolor{basicgame}{RGB}{152,170,149}
\definecolor{intrigue}{RGB}{225,106,85}
\definecolor{seaside}{RGB}{100,145,189}
\definecolor{alchemy}{RGB}{255,245,158}
\definecolor{prosperity}{RGB}{125,195,170}
\definecolor{cornucopia}{RGB}{154,199,119}
\definecolor{hinterlands}{RGB}{175,159,104}
\definecolor{darkages}{RGB}{198,58,69}
\definecolor{guilds}{RGB}{225,111,159}
\definecolor{adventures}{RGB}{199,186,152}
\definecolor{empires}{RGB}{105,181,103}
\definecolor{nocturne}{RGB}{59,75,189}
\definecolor{promo}{RGB}{241,177,59}

% Use Trajan as cardtitle font and TexGrey Termes as normal font
\usepackage{fontspec}
\setmainfont{TexGyre Termes}
\newfontfamily{\trajan}[BoldFont=Trajan Pro Bold]{Trajan Pro}

% Change emph command to writing text bold
\DeclareTextFontCommand{\emph}{\boldmath\bfseries}

% Change section numbering to Alph, center section and reduce space after
\renewcommand\thesection{\Alph{section}}
\renewcommand\secheadstyle{\centering\Large\normalfont\noindent}
\setaftersecskip{1sp}

% Define Header / Footer
\makepagestyle{mystyle}
\setheaderspaces{*}{3\onelineskip}{*}
\makeevenhead{mystyle}{}{Dominion Card Dividers - Created by Danilo Gasdzik based on the works of Eiko Wagenknecht. \\ Dominion Kartentrenner - Erstellt von Danilo Gasdzik basierend auf der Arbeit von Eiko Wagenknecht. \\ \rightmark}{}
\makeoddhead{mystyle}{}{Dominion Card Dividers - Created by Danilo Gasdzik based on the works of Eiko Wagenknecht. \\ Dominion Kartentrenner - Erstellt von Danilo Gasdzik basierend auf der Arbeit von Eiko Wagenknecht. \\ \leftmark}{}
\setheadfoot{1em}{\onelineskip}
\makeevenfoot{mystyle}{\tiny{\hspace{1cm}Entlang der schwarzen Linien ausschneiden. Je nach gewünschter\\\vspace{-0.5em}\hspace{1cm}Größe der Kartentrenner die obere oder untere Linie wählen.}}{\thepage}{\tiny{Cut along the black edge. For your desired size choose the upper or lower line.\hspace{1cm}}}
\makeoddfoot{mystyle}{\tiny{\hspace{1cm}Entlang der schwarzen Linien ausschneiden. Je nach gewünschter\\\vspace{-0.5em}\hspace{1cm}Größe der Kartentrenner die obere oder untere Linie wählen.}}{\thepage}{\tiny{Cut along the black edge. For your desired size choose the upper or lower line.\hspace{1cm}}}
\renewcommand*{\sectionmark}[1]{%
  \markboth%
  {\thesection.\ \MakeTextUppercase{#1}}%
  {\thesection.\ \MakeTextUppercase{#1}}}
\pagestyle{mystyle}

% Enum without indent
\newcounter{mycounter}  
\newenvironment{noindlist}
{\begin{list}{\arabic{mycounter}.~~}{\usecounter{mycounter} \labelsep=0em \labelwidth=2em \leftmargin=2em \itemindent=0em \itemsep=-0.5em \topsep=1em}}
{\end{list}}

% Itemize without spacing
\usepackage{enumitem}
\setlist[itemize]{leftmargin=2em, itemsep=-1em, topsep=1em}

% Shortcuts for often used icons
\newcommand{\coin}[1][]{\raisebox{-0.25em}{\includegraphics[height=1em]{icons/coin.png}}%
	\ifthenelse{\isempty{#1}}%
	{\hspace{0.1em}}%
	{\hspace{-0.75em}\raisebox{0}{\tiny{#1}}\hspace{0.3em}}}
\newcommand{\hex}[1][]{\raisebox{-0.25em}{\includegraphics[height=1em]{icons/hex.png}}%
	\ifthenelse{\isempty{#1}}%
	{\hspace{0.2em}}%
	{\hspace{-0.8em}\raisebox{0}{\tiny{\textcolor{white}{#1}}}\hspace{0.3em}}}
\newcommand{\victorypointtoken}{\includegraphics[height=1em]{icons/victorypointtoken.png}}
\newcommand{\victorypoint}{\includegraphics[height=1em]{icons/victorypoint.png}\hspace{0.1em}}
\newcommand{\potion}{\includegraphics[height=1em]{icons/potion.png}\hspace{0.1em}}
\newcommand{\negativecardmarker}{\includegraphics[height=1em]{icons/negativecardmarker.png}}
\newcommand{\negativecoinmarker}{\includegraphics[height=1em]{icons/negativecoinmarker.png}}

% Draw cards in portrait format
%   Card in portrait
%   ---------------------------------------

%   TikZ/PGF size settings for cards
\pgfmathsetmacro{\cardwidth}{6.3cm}
\pgfmathsetmacro{\cardheight}{11.1cm}
\pgfmathsetmacro{\bannerwidth}{3.2cm}
\pgfmathsetmacro{\iconwidth}{0.5cm}
\pgfmathsetmacro{\iconoffset}{0.175cm}
\pgfmathsetmacro{\stripwidth}{0.2cm}
\pgfmathsetmacro{\stripheight}{0.7cm}
\pgfmathsetmacro{\contentoffset}{0.25cm}
\pgfmathsetmacro{\contentoffsettop}{1.6cm}
\pgfmathsetmacro{\contentwidth}{\cardwidth-\contentoffset-\contentoffset}
\pgfmathsetmacro{\contentheight}{\cardheight-\contentoffset-\contentoffsettop}
\pgfmathsetmacro{\cardoffset}{0.5cm}

%   Stylings for elements
\tikzset{
    %   runde Ecken für die Karten
    cardcorners/.style={
        rounded corners=0.4cm
    }
}


%   Create card with black background, white box and extension on top right
\newcommand{\card}{
	\draw ([xshift=0.1cm,yshift=-\cardoffset]0,0) -- ([xshift=0.1cm,yshift=\cardheight+\cardoffset]0,0);
	\draw ([xshift=\cardwidth-0.1cm,yshift=-\cardoffset]0,0) -- ([xshift=\cardwidth-0.1cm,yshift=\cardheight+\cardoffset]0,0);
	\draw ([xshift=-\cardoffset,yshift=0.1cm]0,0) -- ([xshift=\cardwidth+\cardoffset,yshift=0.1cm]0,0);
	\draw ([xshift=-\cardoffset,yshift=\cardheight-0.1cm]0,0) -- ([xshift=\cardwidth+\cardoffset,yshift=\cardheight-0.1cm]0,0);
	\draw ([xshift=-\cardoffset,yshift=\cardheight-0.3cm]0,0) -- ([xshift=\cardwidth+\cardoffset,yshift=\cardheight-0.3cm]0,0);
	\node [anchor=south west,fill=framebg,minimum width=\cardwidth,minimum height=\cardheight] (main) at (0,0) {};
	\node [anchor=south west,fill=contentbg,minimum width=\contentwidth,minimum height=\contentheight,cardcorners] (content) at ([yshift=\contentoffset,xshift=\contentoffset]0,0) {};
	\node [anchor=south east,\cardcolor,minimum width=\stripwidth,minimum height=\stripheight] at ([xshift=-0.6*\stripwidth,yshift=\contentoffset]content.north east) {\scriptsize{\textsc{\cardextension}}};
	\node [anchor=south east,\cardcolor,minimum width=\stripwidth,minimum height=\stripheight] at ([xshift=-0.6*\stripwidth]content.north east) {\scriptsize{\textsc{\cardextensiontitle}}};
}

%   Add banner on top left
\newcommand{\cardbanner}[1]{
	\node [anchor=south west] (cardbanner) at ([yshift=0.2*\contentoffset]content.north west) {
		\includegraphics[width=\bannerwidth]{#1}
	};
}

%   Add icon to banner
\newcommand{\cardicon}[1]{
	\node [anchor=south west] (cardicon) at ([xshift=\iconoffset,yshift=\iconoffset]cardbanner.south west) {
		\includegraphics[width=\iconwidth]{#1}
	};
}

%   Add price to icon
\newcommand{\cardprice}[1]{
	\node [] at (cardicon) {
		\small{
			\textsc{#1}
		}
	};
}


%   Add second icon
\newcommand{\cardiconaddition}[1]{
	\node [anchor=south west] (cardiconaddition) at ([xshift=\iconwidth]cardicon.south west) {
		\includegraphics[width=\iconwidth]{#1}
	};
}

%   Add price to second icon
\newcommand{\cardpriceaddition}[1]{
	\node [] at (cardiconaddition) {
		\small{
			\textsc{#1}
		}
	};
}

%   Add title of the card to banner
\newcommand{\cardtitle}[1]{
	\node [] at ([xshift=0.55*\iconwidth,yshift=1.1*\iconwidth]cardbanner.south) {
		\small{
			\textsc{#1}
		}
	};
}

%   Add colored strip on top right
\newcommand{\cardstrip}{
	\node [anchor=north east,fill=\cardcolor,minimum width=\stripwidth,minimum height=\stripheight] at ([xshift=-\contentoffset]main.north east) {};
}

%   Write content into white area
\newcommand{\cardcontent}[1]{
	\node [] at (content) {
		\rotatebox{270}{
			\parbox[t][\contentwidth][c]{\contentheight}{
				\begin{minipage}{\contentheight}
					\centering
					\begin{minipage}{0.9\textwidth}
						\raggedright
						\scriptsize{
							\textsf{#1}
						}
					\end{minipage}
				\end{minipage}
			}
		}
	};
}
% Draw cards in landscape format
%\input{format/tikzcardslandscape.tex}

% Basic settings for this card set will be overwritten on each card set
\newcommand{\cardcolor}{}
\newcommand{\cardextension}{}
\newcommand{\cardextensiontitle}{}

\begin{document}
	\begin{center}
		\begin{minipage}{0.75\textwidth}
			\Huge 
			Dominion Kartentrenner
			\normalsize
			\tableofcontents
		\end{minipage}
	    	
	    % Basic settings for this card set
\renewcommand{\cardcolor}{}
\renewcommand{\cardextension}{}
\renewcommand{\cardextensiontitle}{}
\renewcommand{\seticon}{empty.png}

\clearpage
\newpage
\section{Anleitung und Grundausstattung}

\begin{tikzpicture}
	\card
	\cardbanner{banner/white.png}
	\cardtitle{Anleitung (1)\quad}
	\cardcontent{\tiny{\emph{Spielablauf:} Dominion wird zugweise gespielt. Der Spieler an der Reihe hat am Beginn seines Zuges normalerweise 5 Karten auf der Hand. Er führt nun seinen Zug aus, der aus den 3 folgenden Phasen besteht, die immer in dieser Reihenfolge gespielt werden müssen.

	\medskip

	\emph{1. Phase:} Aktion - Der Spieler \emph{darf} Aktionskarten ausspielen.

	\emph{2. Phase:} Kauf - Der Spieler \emph{darf} Karten kaufen.

	\emph{3. Phase:} Aufräumen - Der Spieler \emph{muss} alle  ausgespielten \emph{und} alle Handkarten offen auf seinen Ablagestapel legen und \emph{sofort} 5 Karten für den nächsten Zug nachziehen.

	\medskip

	Die 1. und die 2. Phase darf, die 3. Phase muss gespielt werden. Wenn der Spieler seinen Zug beendet hat, ist der nächste Spieler an der Reihe. Das Spiel verläuft in dieser Weise bis Spielende.
	
	\medskip

	\begin{tabbing}
		Stapel der anderen Spieler: \= Links \kill
		Eigener Ablagestapel: \> Darf weder durchgezählt noch durchgesehen werden. \\
		Eigener Nachziehstapel: \> Darf durchgezählt, nicht aber durchgesehen werden. \\
		Stapel der anderen Spieler: \> Dürfen weder durchgezählt noch durchgesehen werden. \\
		Stapel im Vorrat: \> Dürfen jederzeit durchgezählt und durchgesehen werden. \\
		Müllstapel: \> Darf jederzeit durchgezählt und durchgesehen werden. \\
	\end{tabbing}}}
\end{tikzpicture}
\hspace{-0.6cm}
\begin{tikzpicture}
	\card
	\cardbanner{banner/white.png}
	\cardtitle{Anleitung (2)\quad}
	\cardcontent{\tiny{\begin{tabbing}
	Spielende:xxx \=  18x Provinzen,xxx \= 12 andere Punktekarten,xxx \= 50 Flüche,xxx \= Geld: \kill
	2 Spieler: \> 8x Provinzen, \> 8 andere Punktekarten, \> 10 Flüche, \> Geld: 46 K, 40 S, 30 G\\
	3 Spieler: \> 12x Provinzen, \> 12 andere Punktekarten, \> 20 Flüche, \> Geld: 39 K, 40 S, 30 G\\
	4 Spieler: \> 12x Provinzen, \> 12 andere Punktekarten, \> 30 Flüche, \> Geld: 32 K, 40 S, 30 G\\
	5 Spieler: \> 15x Provinzen, \> 12 andere Punktekarten, \> 40 Flüche, \> Geld: 85 K, 80 S, 60 G\\
	6 Spieler: \> 18x Provinzen, \> 12 andere Punktekarten, \> 50 Flüche, \> Geld: 78 K, 80 S, 60 G\\
	Spielende: \> Provinzstapel, Kolonienstapel (Dominion – Blütezeit) \emph{oder} \\
					\>3 Stapel (1 - 4 Spieler) bzw. 4 Stapel (5 - 6 Spieler) aus dem Vorrat leer \\
	\end{tabbing}
	Punktekarten aus Erweiterungen werden in Anzahl der \enquote{anderen Punktekarten} ausgelegt. Zur Spielende-Bedingung zählen alle Karten im Vorrat, also auch Fluch-, Geld- und Punktekarten, nicht jedoch z. B. der Müllstapel.

	\medskip

	\emph{Platin und Kolonie (Dominion – Blütezeit):} Die im Spiel befindlichen Königreich-Platzhalterkarten werden gemischt. Ist die erste gezogene Königreichskarte eine Karte aus der Blütezeit-Erweiterung, so wird mit Platin und Kolonie gespielt, ansonsten ohne.

	\medskip

	\emph{Unterschlupf-Karten (Dominion – Dark Ages):} Die im Spiel befindlichen Königreich-Platzhalterkarten werden gemischt. Ist die erste gezogene Königreichskarte eine Karte aus der Dark Ages-Erweiterung, so startet jeder Spieler mit 7 Kupfer, 1 Hütte, 1 Totenstadt und 1 Verfallenes Anwesen, andernfalls erhält jeder Spieler 7 Kupfer und 3 Anwesen.}}
\end{tikzpicture}
\hspace{-0.6cm}
\begin{tikzpicture}
	\card
	\cardbanner{banner/white.png}
	\cardtitle{Platzhalter\quad}
\end{tikzpicture}
\hspace{-0.6cm}
\begin{tikzpicture}
	\card
	\cardbanner{banner/gold.png}
	\cardicon{icons/coin.png}
	\cardprice{0}
	\cardtitle{Kupfer}
\end{tikzpicture}
\hspace{-0.6cm}
\begin{tikzpicture}
	\card
	\cardbanner{banner/gold.png}
	\cardicon{icons/coin.png}
	\cardprice{3}
	\cardtitle{Silber}
\end{tikzpicture}
\hspace{-0.6cm}
\begin{tikzpicture}
	\card
	\cardbanner{banner/gold.png}
	\cardicon{icons/coin.png}
	\cardprice{6}
	\cardtitle{Gold}
\end{tikzpicture}
\hspace{-0.6cm}
\begin{tikzpicture}
	\card
	\cardbanner{banner/green.png}
	\cardicon{icons/coin.png}
	\cardprice{2}
	\cardtitle{Anwesen}
\end{tikzpicture}
\hspace{-0.6cm}
\begin{tikzpicture}
	\card
	\cardbanner{banner/green.png}
	\cardicon{icons/coin.png}
	\cardprice{5}
	\cardtitle{Herzogtum}
\end{tikzpicture}
\hspace{-0.6cm}
\begin{tikzpicture}
	\card
	\cardbanner{banner/green.png}
	\cardicon{icons/coin.png}
	\cardprice{8}
	\cardtitle{Provinz}
\end{tikzpicture}
\hspace{-0.6cm}
\begin{tikzpicture}
	\card
	\cardbanner{banner/purple.png}
	\cardicon{icons/coin.png}
	\cardprice{0}
	\cardtitle{Fluch}
\end{tikzpicture}	
\hspace{0.6cm}

	    % Basic settings for this card set
\renewcommand{\cardcolor}{basicgame}
\renewcommand{\cardextension}{Edition I}
\renewcommand{\cardextensiontitle}{Das Basisspiel}
\renewcommand{\seticon}{basic1.png}

\clearpage
\newpage
\section{\cardextension \ - \cardextensiontitle \ (Hans Im Glück 2008)}

\begin{tikzpicture}
	\card
	\cardstrip
	\cardbanner{banner/blue.png}
	\cardicon{icons/coin.png}
	\cardprice{2}
	\cardtitle{Burggraben}
	\cardcontent{Angriffskarten sind mit der Aufschrift \enquote{Angriff} (normalerweise \enquote{Aktion - Angriff}) gekennzeichnet. Spielt ein anderer Spieler eine Angriffskarte aus, kannst du die Karte Burggraben vorzeigen, falls du sie auf der Hand hast. Du nimmst den Burggraben zurück auf deine Hand, bevor der Angriff abgewickelt wird. Du bist vom Angriff nicht betroffen: Du musst dir keine Fluchkarte von der Hexe nehmen, musst beim Spion keine Karten aufdecken usw. Es ist so, als wärst du nicht im Spiel. Dein Burggraben hat keine Auswirkungen auf die übrigen Mitspieler, diese sind wie üblich vom Angriff betroffen. Für den \enquote{Angriffsspieler} selbst gilt: Auch wenn einer oder mehrere Spieler einen Burggraben vorzeigen, kann er die übrigen Anweisungen der Angriffskarte durchführen. Spielt er z. B. eine Hexe aus, so zieht er trotzdem 2 Karten nach.

	\medskip

	Spielst du den Burggraben in deinem Zug aus, musst du 2 Karten nachziehen.}
\end{tikzpicture}
\hspace{-0.6cm}
\begin{tikzpicture}
	\card
	\cardstrip
	\cardbanner{banner/white.png}
	\cardicon{icons/coin.png}
	\cardprice{2}
	\cardtitle{Keller}
	\cardcontent{Du kannst den ausgespielten Keller selbst nicht ablegen, da du die Karte nicht mehr auf der Hand hältst, wenn du die Anweisung ausführst. Du sagst zuerst an, wieviele Handkarten du ablegst, legst diese Karten auf deinen Ablagestapel und ziehst dann die gleiche Anzahl Karten vom Nachziehstapel. Geht der Nachziehstapel zu Ende, werden die gerade abgelegten Karten wieder eingemischt.}
\end{tikzpicture}
\hspace{-0.6cm}
\begin{tikzpicture}
	\card
	\cardstrip
	\cardbanner{banner/white.png}
	\cardicon{icons/coin.png}
	\cardprice{3}
	\cardtitle{Dorf}
	\cardcontent{Spielst du mehrere Dörfer aus, zählst du laut mit, wie viele Aktionskarten du insgesamt noch ausspielen darfst, z.B.: Ich spiele das Dorf und darf noch 2 Aktionskarten ausspielen. Ich spiele einen Markt und darf noch insgesamt 2 Aktionskarten ausspielen. Ich spiele ein Dorf und darf noch insgesamt 3 Aktionskarten ausspielen ...}
\end{tikzpicture}
\hspace{-0.6cm}
\begin{tikzpicture}
	\card
	\cardstrip
	\cardbanner{banner/white.png}
	\cardicon{icons/coin.png}
	\cardprice{3}
	\cardtitle{Werkstatt}
	\cardcontent{Du nimmst dir eine Karte aus dem Vorrat und legst diese sofort auf deinen Ablagestapel. Du kannst keine Geldkarten oder virtuelles Geld von anderen Aktionskarten verwenden, um den Betrag zu erhöhen.}
\end{tikzpicture}
\hspace{-0.6cm}
\begin{tikzpicture}
	\card
	\cardstrip
	\cardbanner{banner/white.png}
	\cardicon{icons/coin.png}
	\cardprice{3}
	\cardtitle{Holzfäller}
	\cardcontent{In der Kaufphase darfst du +2 Geld zu deinen ausgelegten Geldkarten hinzuzählen und du darfst eine weitere Karte kaufen.}
\end{tikzpicture}
\hspace{-0.6cm}
\begin{tikzpicture}
	\card
	\cardstrip
	\cardbanner{banner/white.png}
	\cardicon{icons/coin.png}
	\cardprice{4}
	\cardtitle{Schmiede}
	\cardcontent{Du musst 3 Karten von deinem Nachziehstapel nachziehen.}
\end{tikzpicture}
\hspace{-0.6cm}
\begin{tikzpicture}
	\card
	\cardstrip
	\cardbanner{banner/white.png}
	\cardicon{icons/coin.png}
	\cardprice{4}
	\cardtitle{Umbau}
	\cardcontent{Du kannst den ausgespielten Umbau selbst nicht entsorgen, da du die Karte nicht mehr auf der Hand hältst, wenn du die Anweisung ausführst. Hast du einen weiteren Umbau auf der Hand, kannst du diesen entsorgen. Hast du keine Karte auf der Hand, die du entsorgen kannst, nimmst du dir auch keine neue Karte. Die Karte, die du nimmst, kann bis zu 2 Geld mehr kosten als die entsorgte Karte. Du kannst keine Geldkarten oder virtuelles Geld von anderen Aktionskarten verwenden, um den Betrag zu erhöhen. Du kannst auch eine Karte entsorgen und dir eine identische Karte aus dem Vorrat nehmen.}
\end{tikzpicture}
\hspace{-0.6cm}
\begin{tikzpicture}
	\card
	\cardstrip
	\cardbanner{banner/white.png}
	\cardicon{icons/coin.png}
	\cardprice{4}
	\cardtitle{Miliz}
	\cardcontent{Deine Mitspieler müssen Handkarten ablegen, bis sie nur noch 3 Karten auf der Hand haben. Spieler, die bereits 3 oder weniger Karten auf der Hand haben, wenn die Miliz ausgespielt wird, müssen keine Karten ablegen.}
\end{tikzpicture}
\hspace{-0.6cm}
\begin{tikzpicture}
	\card
	\cardstrip
	\cardbanner{banner/white.png}
	\cardicon{icons/coin.png}
	\cardprice{5}
	\cardtitle{Markt}
	\cardcontent{Du musst zuerst eine Karte nachziehen. Dann kannst du eine weitere Aktionskarte ausspielen. In der Kaufphase hast du einen weiteren Kauf und +1 Geld zur Verfügung.}
\end{tikzpicture}
\hspace{-0.6cm}
\begin{tikzpicture}
	\card
	\cardstrip
	\cardbanner{banner/white.png}
	\cardicon{icons/coin.png}
	\cardprice{5}
	\cardtitle{Mine}
	\cardcontent{Normalerweise entsorgst du eine Kupferkarte und nimmst dir dafür eine Silberkarte aus dem Vorrat auf die Hand oder du entsorgst eine Silberkarte und nimmst dir dafür eine Goldkarte auf die Hand. Du kannst jedoch auch eine identische oder eine kleinere Geldkarte nehmen. Die aufgenommene Karte kannst du noch in diesem Zug einsetzen. Hast du keine Geldkarte, die du entsorgen kannst, erhältst du nichts.

	\medskip

	Du darfst mit der Mine (neben Kupfer, Silber und Gold) auch Königreich-Geldkarten (z. B. Füllhorn) entsorgen und nehmen. \emph{Achtung:} Du darfst das Diadem (Dominion – Reiche Ernte) nicht nehmen, da dies eine Preiskarte ist und nicht zum Vorrat gehört.}
\end{tikzpicture}
\hspace{-0.6cm}
\begin{tikzpicture}
	\card
	\cardstrip
	\cardbanner{banner/white.png}
	\cardicon{icons/coin.png}
	\cardprice{6}
	\cardtitle{Abenteurer}
	\cardcontent{Die aufgedeckten Karten legst du zunächst offen vor dir aus. Sollte dein Nachziehstapel beim Aufdecken zu Ende gehen, mischt du deinen Ablagestapel. Die bereits aufgedeckten Karten mischst du jedoch nicht mit ein, da diese Karten erst am Ende der Aktion auf den Ablagestapel gelegt werden. Hast du keine Karten mehr zum Aufdecken und noch immer nur eine Geldkarte, bekommst du nur diese eine.}
\end{tikzpicture}
\hspace{-0.6cm}
\begin{tikzpicture}
	\card
	\cardstrip
	\cardbanner{banner/white.png}
	\cardicon{icons/coin.png}
	\cardprice{5}
	\cardtitle{Bibliothek}
	\cardcontent{Du kannst Aktionskarten zur Seite legen, musst dies jedoch nicht. Hast du bereits 7 oder mehr Karten auf der Hand, wenn du die Bibliothek ausspielst, ziehst du keine Karten nach. Solltest du den Ablagestapel mischen müssen, mischst du die zur Seite gelegten Karten jedoch nicht mit ein. Diese Karten legst du erst auf den Ablagestapel, sobald du 7 Karten auf der Hand hast. Sollten die Karten nicht ausreichen, ziehst du nur so viele Karten nach wie möglich.}
\end{tikzpicture}
\hspace{-0.6cm}
\begin{tikzpicture}
	\card
	\cardstrip
	\cardbanner{banner/white.png}
	\cardicon{icons/coin.png}
	\cardprice{4}
	\cardtitle{Bürokrat}
	\cardcontent{Ist dein Nachziehstapel aufgebraucht, wenn du diese Karte ausspielst, legst du die Silberkarte verdeckt ab. Dein eigener Nachziehstapel besteht dann nur aus dieser Karte. Das Gleiche gilt für die übrigen Spieler, die eine Karte auf ihren eigenen Nachziehstapel legen müssen.}
\end{tikzpicture}
\hspace{-0.6cm}
\begin{tikzpicture}
	\card
	\cardstrip
	\cardbanner{banner/white.png}
	\cardicon{icons/coin.png}
	\cardprice{4}
	\cardtitle{Dieb}
	\cardcontent{Die beiden aufgedeckten Karten legen deine Mitspieler zunächst o en vor sich aus. Hat ein Spieler nur noch eine Karte im eigenen Nachziehstapel, deckt er diese auf, mischt seinen Ablagestapel (ohne die gerade aufgedeckte Karte) und deckt dann die zweite Karte auf. Hat ein Spieler keine Karten mehr im eigenen Nachziehstapel, mischt er sofort und deckt dann 2 Karten auf. Hat ein Spieler nach dem Mischen immer noch nicht genug Karten, deckt er nur so viele auf wie möglich. Jeder Spieler entsorgt höchstens eine Geldkarte, nach deiner Wahl. Dann nimmst du beliebig viele der gerade (nicht in früheren Zügen) entsorgten Karten und legst sie auf deinen Ablagestapel. Die übrigen aufgedeckten Karten legen die Spieler auf ihren Ablagestapel.}
\end{tikzpicture}
\hspace{-0.6cm}
\begin{tikzpicture}
	\card
	\cardstrip
	\cardbanner{banner/white.png}
	\cardicon{icons/coin.png}
	\cardprice{4}
	\cardtitle{Festmahl}
	\cardcontent{Du nimmst dir 1 Karte, die bis zu 5 Geld kostet, aus dem Vorrat und legst diese sofort auf deinen Ablagestapel. Du kannst keine Geldkarten oder virtuelles Geld von anderen Aktionskarten verwenden, um den Betrag zu erhöhen. Spielst du das Festmahl nach dem Thronsaal, erhältst du 2 Karten, obwohl du das Festmahl nur einmal entsorgen kannst.}
\end{tikzpicture}
\hspace{-0.6cm}
\begin{tikzpicture}
	\card
	\cardstrip
	\cardbanner{banner/green.png}
	\cardicon{icons/coin.png}
	\cardprice{4}
	\cardtitle{Gärten}
	\cardcontent{Diese Königreichkarte ist eine Punktekarte, keine Aktionskarte. Sie hat bis zum Ende des Spiels keine Funktion. Bei der Wertung zählt sie 1 Punkt pro volle 10 Karten im gesamten Kartensatz des Spielers. Du zählst alle deine Karten bei Spielende (auch die Gärten), teilst die Anzahl durch 10 und rundest ab. Für 39 Karten erhältst du beispielsweise 3 Punkte.}
\end{tikzpicture}
\hspace{-0.6cm}
\begin{tikzpicture}
	\card
	\cardstrip
	\cardbanner{banner/white.png}
	\cardicon{icons/coin.png}
	\cardprice{4}
	\cardtitle{\footnotesize{Geldverleiher}}
	\cardcontent{Hast du kein Kupfer auf der Hand, das du entsorgen könntest, so erhältst du auch kein virtuelles Geld für die Kaufphase.}
\end{tikzpicture}
\hspace{-0.6cm}
\begin{tikzpicture}
	\card
	\cardstrip
	\cardbanner{banner/white.png}
	\cardicon{icons/coin.png}
	\cardprice{5}
	\cardtitle{Hexe}
	\cardcontent{Sind nicht mehr genügend Fluchkarten im Vorrat, wenn du die Hexe ausspielst, werden die restlichen Fluchkarten, beginnend beim Spieler links von dir, in Spielreihenfolge verteilt. Du ziehst immer 2 Karten nach, auch wenn keine Fluchkarten mehr im allgemeinen Vorrat sind. Die Fluchkarten legen die Spieler sofort auf ihren Ablagestapel.}
\end{tikzpicture}
\hspace{-0.6cm}
\begin{tikzpicture}
	\card
	\cardstrip
	\cardbanner{banner/white.png}
	\cardicon{icons/coin.png}
	\cardprice{5}
	\cardtitle{Jahrmarkt}
	\cardcontent{Spielst du mehrere Jahrmärkte aus, zählst du laut mit, wie viele Aktionskarten du noch ausspielen darfst.
	}
\end{tikzpicture}
\hspace{-0.6cm}
\begin{tikzpicture}
	\card
	\cardstrip
	\cardbanner{banner/white.png}
	\cardicon{icons/coin.png}
	\cardprice{3}
	\cardtitle{Kanzler}
	\cardcontent{Wenn du deinen Nachziehstapel auf den Ablagestapel legst, musst du das tun, bevor du weitere Aktionskarten ausspielst oder die Aktionsphase beendest. Du darfst deinen Nachziehstapel nicht durchsehen, bevor du ihn ablegst.

	\medskip

	\emph{Anmerkung:} Durch das direkte Ablegen wird die Karte Tunnel (Dominion – Hinterland) nicht ausgelöst.\\}
\end{tikzpicture}
\hspace{-0.6cm}
\begin{tikzpicture}
	\card
	\cardstrip
	\cardbanner{banner/white.png}
	\cardicon{icons/coin.png}
	\cardprice{2}
	\cardtitle{Kapelle}
	\cardcontent{Du kannst die ausgespielte Kapelle selbst nicht entsorgen, da du die Karte nicht mehr auf der Hand hältst, wenn du die Anweisung ausführst. Hast du eine weitere Kapelle auf der Hand, kannst du diese entsorgen.}
\end{tikzpicture}
\hspace{-0.6cm}
\begin{tikzpicture}
	\card
	\cardstrip
	\cardbanner{banner/white.png}
	\cardicon{icons/coin.png}
	\cardprice{5}
	\cardtitle{\footnotesize{Laboratorium}}
	\cardcontent{Du musst zuerst 2 Karten nachziehen und kannst dann eine weitere Aktionskarte ausspielen.}
\end{tikzpicture}
\hspace{-0.6cm}
\begin{tikzpicture}
	\card
	\cardstrip
	\cardbanner{banner/white.png}
	\cardicon{icons/coin.png}
	\cardprice{5}
	\cardtitle{\scriptsize{Ratsversammlung}}
	\cardcontent{Die Mitspieler müssen eine Karte nachziehen, auch wenn sie nicht wollen.}
\end{tikzpicture}
\hspace{-0.6cm}
\begin{tikzpicture}
	\card
	\cardstrip
	\cardbanner{banner/white.png}
	\cardicon{icons/coin.png}
	\cardprice{4}
	\cardtitle{Spion}
	\cardcontent{Du musst zuerst 1 Karte nachziehen. Erst danach decken alle Spieler (auch du selbst) die oberste Karte von ihrem Nachziehstapel auf. Du entscheidest dann bei jedem Spieler (auch bei dir selbst), ob er die aufgedeckte Karte auf seinen Ablagestapel oder zurück auf seinen Nachziehstapel legt.

	\medskip

	Hat ein Spieler keine Karte mehr im Nachziehstapel, so mischt er seinen Ablagestapel und deckt dann die oberste Karte auf. Spieler, die dann immer noch keine Karten haben, decken keine Karte auf. Wenn den Spielern die Reihenfolge wichtig ist, deckst du zuerst eine Karte auf, danach folgen die Mitspieler in Spielreihenfolge.}
\end{tikzpicture}
\hspace{-0.6cm}
\begin{tikzpicture}
	\card
	\cardstrip
	\cardbanner{banner/white.png}
	\cardicon{icons/coin.png}
	\cardprice{4}
	\cardtitle{Thronsaal}
	\cardcontent{Wenn du den Thronsaal ausspielst, wählst du 1 weitere Aktionskarte aus deiner Hand, legst diese o en aus und führst die angegebene(n) Anweisung(en) aus. Dann nimmst du diese Karte zurück auf die Hand, legst sie ein weiteres Mal aus und führst die angegebene(n) Anweisung(en) nochmals aus. Dieses Auslegen der Aktionskarte kostet keine Aktionen. Du musst die Anweisung(en) der Karte soweit möglich vollständig ausführen, bevor du sie zurück auf die Hand nimmst und ein weiteres Mal auslegst. Legst du einen weiteren Thronsaal nach einem Thronsaal aus, legst du eine Aktionskarte zweimal aus und danach eine weitere ebenfalls zweimal. Du kannst nicht ein und dieselbe Karte 4 mal auslegen. Wenn die nach dem Thronsaal ausgelegte Karte \enquote{+1 Aktion} erlaubt (wie z. B. der Markt), darfst du anschließend noch 2 weitere Aktionen ausführen. Würdest du 2 Märkte regulär ausspielen, dürftest du nur noch eine zusätzliche Aktion ausführen, da das Ausspielen des zweiten Markts eine Aktion aufbraucht. Hier ist es besonders wichtig, die verbleibenden Aktionen laut mitzuzählen. Du kannst keine andere Aktionskarte ausspielen, bevor der Thronsaal vollständig abgehandelt ist.}
\end{tikzpicture}
\hspace{-0.6cm}
\begin{tikzpicture}
	\card
	\cardstrip
	\cardbanner{banner/white.png}
	\cardtitle{\scriptsize{Empfohlene 10er Sätze\qquad}}
	\cardcontent{\emph{Erstes Spiel:}\\
	Burggraben, Dorf, Holzfäller, Keller, Markt, Miliz, Mine, Schmiede, Umbau, Werkstatt

	\smallskip

	\emph{Großes Geld:}\\
	Abenteurer, Bürokrat, Festmahl, Geldverleiher, Kanzler, Kapelle, Laboratorium, Mark, Mine, Thronsaal

	\smallskip

	\emph{Interaktion:}\\
	Bibliothek, Burggraben, Bürokrat, Dieb, Dorf, Jahrmarkt, Kanzler, Miliz, Ratsversammlung, Spion

	\smallskip

	\emph{Im Wandel:}\\
	Dieb, Dorf, Festmahl, Gärten, Hexe, Holzfäller, Kapelle, Keller, Laboratorium, Werkstatt

	\smallskip

	\emph{Dorfplatz:}\\
	Bibliothek, Bürokrat, Dorf, Holzfäller, Jahrmarkt, Keller, Markt, Schmiede, Thronsaal, Umbau}
\end{tikzpicture}
\hspace{0.6cm}

		% Basic settings for this card set
\renewcommand{\cardcolor}{basicgame}
\renewcommand{\cardextension}{Special Edition}
\renewcommand{\cardextensiontitle}{Das Basisspiel}
\renewcommand{\seticon}{basic1.png}

\clearpage
\newpage
\section{\cardextension \ - \cardextensiontitle \ (Rio Grande Games 2013)}

\begin{tikzpicture}
	\card
	\cardstrip
	\cardbanner{banner/blue.png}
	\cardicon{icons/coin.png}
	\cardprice{2}
	\cardtitle{Burggraben}
	\cardcontent{Spielt ein anderer Spieler eine Angriffskarte (mit der Aufschrift AKTION -- ANGRIFF), kannst du die Karte \emph{BURGGRABEN} vorzeigen, falls du sie in diesem Moment auf der Hand hast. In diesem Fall bist du von den Auswirkungen des Angriffs nicht betroffen, d.h. du musst bei der \emph{HEXE} keine Fluchkarte nehmen usw. Haben mehrere Spieler einen \emph{BURGGRABEN} auf der Hand, dürfen diese auch eingesetzt und vorgezeigt werden. Danach nehmen die Spieler ihre Karte zurück auf die Hand. Der Spieler, der den Angriff gespielt hat, darf unabhängig davon, ob ein oder mehrere \emph{BURGGRÄBEN} gespielt werden, die weiteren Anweisungen seiner Aktionskarte ausführen. Der \emph{BURGGRABEN} darf auch in der eigenen Aktionsphase gespielt werden – dann ziehst du 2 Karten nach.}
\end{tikzpicture}
\hspace{-0.6cm}
\begin{tikzpicture}
	\card
	\cardstrip
	\cardbanner{banner/white.png}
	\cardicon{icons/coin.png}
	\cardprice{2}
	\cardtitle{Keller}
	\cardcontent{Der ausgespielte \emph{KELLER} selbst darf nicht abgelegt werden, da er sich nicht mehr in deiner Hand befindet. Sage an, wie viele Karten du ablegst und lege diese auf deinen Ablagestapel. Danach ziehst du die gleiche Anzahl Karten vom Nachziehstapel. Sollte während dieses Vorgangs der Nachziehstapel aufgebraucht werden, wird dein Ablagestapel zusammen mit den soeben abgelegten Karten gemischt und als neuer Nachziehstapel bereitgelegt.}
\end{tikzpicture}
\hspace{-0.6cm}
\begin{tikzpicture}
	\card
	\cardstrip
	\cardbanner{banner/white.png}
	\cardicon{icons/coin.png}
	\cardprice{3}
	\cardtitle{Dorf}
	\cardcontent{Spielst du mehrere \emph{DÖRFER} hintereinander, zählst du am besten laut mit, wie viele Aktionen du noch ausspielen darfst, damit du den Überblick behältst.}
\end{tikzpicture}
\hspace{-0.6cm}
\begin{tikzpicture}
	\card
	\cardstrip
	\cardbanner{banner/white.png}
	\cardicon{icons/coin.png}
	\cardprice{3}
	\cardtitle{Werkstatt}
	\cardcontent{Nimm dir eine Karte aus dem Vorrat und lege diese sofort auf deinen Ablagestapel. Du kannst weder Geldkarten noch zusätzlich über Aktionskarten erhaltenes Geld oder Münzen (bei Erweiterungen mit Münzen) einsetzen, um den angegebenen Betrag auf der Karte zu erhöhen.}
\end{tikzpicture}
\hspace{-0.6cm}
\begin{tikzpicture}
	\card
	\cardstrip
	\cardbanner{banner/white.png}
	\cardicon{icons/coin.png}
	\cardprice{3}
	\cardtitle{Holzfäller}
	\cardcontent{In der Kaufphase darfst du 1 zusätzliche Karte kaufen, also insgesamt 2 Käufe tätigen. Für deine Käufe stehen dir in diesem Zug insgesamt 2 Geld zusätzlich zur Verfügung.}
\end{tikzpicture}
\hspace{-0.6cm}
\begin{tikzpicture}
	\card
	\cardstrip
	\cardbanner{banner/white.png}
	\cardicon{icons/coin.png}
	\cardprice{4}
	\cardtitle{Schmiede}
	\cardcontent{Du musst 3 Karten von deinem Nachziehstapel ziehen und auf die Hand nehmen.}
\end{tikzpicture}
\hspace{-0.6cm}
\begin{tikzpicture}
	\card
	\cardstrip
	\cardbanner{banner/white.png}
	\cardicon{icons/coin.png}
	\cardprice{4}
	\cardtitle{Umbau}
	\cardcontent{Der ausgespielte \emph{UMBAU} selbst darf nicht entsorgt werden, da er sich nicht mehr in deiner Hand befindet. Weitere \emph{UMBAU}-Karten in deiner Hand dürfen entsorgt werden. Wenn du keine Karte zum Entsorgen auf der Hand hast, darfst du dir auch keine neue Karte nehmen. Die neue Karte, die du dir nimmst, darf maximal bis zu \coin[2] mehr als die entsorgte Karte kosten. Der Betrag darf weder durch weitere Geldkarten, Münzen oder zusätzliches Geld von anderen Aktionskarten erhöht werden. Die neue Karte kann die gleiche Karte sein wie die, die du entsorgt hast. Lege die neue Karte auf deinen Ablagestapel.}
\end{tikzpicture}
\hspace{-0.6cm}
\begin{tikzpicture}
	\card
	\cardstrip
	\cardbanner{banner/white.png}
	\cardicon{icons/coin.png}
	\cardprice{4}
	\cardtitle{Miliz}
	\cardcontent{Deine Mitspieler müssen Karten aus ihrer Hand ablegen, bis sie nur noch 3 Karten auf der Hand haben. Spieler, die zum Zeitpunkt des Angriffs bereits 3 oder weniger Karten auf der Hand haben, müssen keine weiteren Karten ablegen.}
\end{tikzpicture}
\hspace{-0.6cm}
\begin{tikzpicture}
	\card
	\cardstrip
	\cardbanner{banner/white.png}
	\cardicon{icons/coin.png}
	\cardprice{5}
	\cardtitle{Markt}
	\cardcontent{Du musst eine Karte vom Nachziehstapel auf die Hand nehmen. Du \emph{darfst} in der Aktionsphase eine weitere Aktionskarte ausspielen. Du \emph{darfst} in der Kaufphase einen zusätzlichen Kauf tätigen und hast dafür ein zusätzliches Geld zur Verfügung.}
\end{tikzpicture}
\hspace{-0.6cm}
\begin{tikzpicture}
	\card
	\cardstrip
	\cardbanner{banner/white.png}
	\cardicon{icons/coin.png}
	\cardprice{5}
	\cardtitle{Mine}
	\cardcontent{\emph{Errata:} Der Kartentext ist falsch, es sollte \enquote{Du \emph{darfst} eine beliebige Geldkarte aus der Hand entsorgen. Nimm eine Geldkarte vom Vorrat auf die Hand, die bis zu \coin[3] mehr kostet.} heißen.

	\medskip

	Normalerweise entsorgst du ein Kupfer und nimmst dir dafür ein Silber, oder du entsorgst ein Silber und nimmst dir ein Gold. Du kannst dir aber auch eine gleichwertige oder billigere Karte nehmen. Die neue Karte nimmst du sofort auf die Hand und darfst sie noch während deines Zuges einsetzen. Wer keine Geldkarte zum Entsorgen hat, erhält keine neue Karte.}
\end{tikzpicture}
\hspace{-0.6cm}
\begin{tikzpicture}
	\card
	\cardstrip
	\cardbanner{banner/white.png}
	\cardicon{icons/coin.png}
	\cardprice{6}
	\cardtitle{Abenteurer}
	\cardcontent{Sollte dein Nachziehstapel während des Aufdeckens aufgebraucht werden, mische deinen Ablagestapel. Die bereits aufgedeckten Karten werden nicht mit gemischt, sondern bleiben zunächst offen liegen. Solltest du auch mit Hilfe des neuen Nachziehstapels nicht genügend Geldkarten aufdecken, bekommst du nur diejenigen Geldkarten, die du aufgedeckt hast.}
\end{tikzpicture}
\hspace{-0.6cm}
\begin{tikzpicture}
	\card
	\cardstrip
	\cardbanner{banner/white.png}
	\cardicon{icons/coin.png}
	\cardprice{5}
	\cardtitle{Bibliothek}
	\cardcontent{Aktionskarten darfst du zur Seite legen, sobald du sie ziehst, musst dies aber nicht tun. Hast du bereits 7 oder mehr Karten auf der Hand, wenn du die \emph{BIBLIOTHEK} ausspielst, ziehst du keine Karten nach. Wenn dein Nachziehstapel während des Ziehens aufgebraucht ist, mischst du den Ablagestapel, mischst aber die zur Seite gelegten Aktionskarten nicht mit ein. Diese werden erst auf den Ablagestapel gelegt, sobald du 7 Karten auf der Hand hast. Sollten die Karten nicht reichen, ziehst du nur so viele Karten wie möglich.}
\end{tikzpicture}
\hspace{-0.6cm}
\begin{tikzpicture}
	\card
	\cardstrip
	\cardbanner{banner/white.png}
	\cardicon{icons/coin.png}
	\cardprice{4}
	\cardtitle{Bürokrat}
	\cardcontent{Ist dein Nachziehstapel aufgebraucht, wenn du diese Karte spielst, legst du die Silberkarte verdeckt ab. Sie bildet dann deinen Nachziehstapel. Das Gleiche gilt für alle Mitspieler, die eine Punktekarte verdeckt auf den eigenen Nachziehstapel legen müssen.}
\end{tikzpicture}
\hspace{-0.6cm}
\begin{tikzpicture}
	\card
	\cardstrip
	\cardbanner{banner/white.png}
	\cardicon{icons/coin.png}
	\cardprice{4}
	\cardtitle{Dieb}
	\cardcontent{Jeder Mitspieler legt die beiden aufgedeckten Karten zunächst offen vor sich ab. Wer nur noch 1 Karte im Nachziehstapel hat, legt diese vor sich ab und mischt erst dann seinen Ablagestapel. Hat ein Spieler nach dem Mischen noch immer nicht genug Karten, deckt er nur so viele auf wie möglich. Hat ein Spieler 2 Geldkarten offen liegen, wählst du eine davon aus, die der Spieler entsorgen muss. Die andere Karte legt er auf seinen Ablagestapel. Hat ein Spieler 1 Geldkarte offen liegen, muss er diese entsorgen. Hat ein Spieler keine Geldkarte aufgedeckt, muss er keine Karte entsorgen. Von den auf diese Weise entsorgten Karten darfst du eine beliebige Anzahl nehmen.}
\end{tikzpicture}
\hspace{-0.6cm}
\begin{tikzpicture}
	\card
	\cardstrip
	\cardbanner{banner/white.png}
	\cardicon{icons/coin.png}
	\cardprice{4}
	\cardtitle{Festmahl}
	\cardcontent{Du nimmst dir eine beliebige Karte aus dem Vorrat, die höchstens \coin[5] kostet und legst sie sofort auf deinen Ablagestapel. Du darfst den Betrag weder durch weitere Geldkarten, Münzen oder zusätzliches Geld von anderen Aktionskarten erhöhen. Spielst du das \emph{FESTMAHL} direkt nach dem \emph{THRONSAAL}, erhältst du 2 Karten, obwohl du das \emph{FESTMAHL} nur einmal entsorgen kannst.}
\end{tikzpicture}
\hspace{-0.6cm}
\begin{tikzpicture}
	\card
	\cardstrip
	\cardbanner{banner/green.png}
	\cardicon{icons/coin.png}
	\cardprice{4}
	\cardtitle{Gärten}
	\cardcontent{Diese Karte ist die einzige Punktekarte unter den Königreichkarten. Sie hat bis zum Ende des Spiels keine Funktion. Bei der Wertung des Spiels erhält der Spieler, der diese Karte in seinem Kartensatz (Nachziehstapel, Handkarten und Ablagestapel) hat, für jeweils 10 Karten einen Siegpunkt. Es wird immer abgerundet, d.h. 39 Karten ergeben 3 Siegpunkte, ebenso wie 31 Karten 3 Siegpunkte ergeben. Wer mehrere \emph{GÄRTEN} besitzt, erhält jeden \emph{GARTEN} die entsprechende Anzahl an Siegpunkten.}
\end{tikzpicture}
\hspace{-0.6cm}
\begin{tikzpicture}
	\card
	\cardstrip
	\cardbanner{banner/white.png}
	\cardicon{icons/coin.png}
	\cardprice{4}
	\cardtitle{\footnotesize{Geldverleiher}}
	\cardcontent{\emph{Errata:} Der Kartentext ist falsch, es sollte \enquote{Du \emph{darfst} ein Kupfer aus der Hand entsorgen. Wenn du das tust: + \coin[3].} heißen.

	\medskip

	Wenn du kein Kupfer zum Entsorgen auf der Hand hast, erhältst du kein zusätzliches Geld für die Kaufphase.}
\end{tikzpicture}
\hspace{-0.6cm}
\begin{tikzpicture}
	\card
	\cardstrip
	\cardbanner{banner/white.png}
	\cardicon{icons/coin.png}
	\cardprice{5}
	\cardtitle{Hexe}
	\cardcontent{Wenn du die \emph{HEXE} spielst und nicht mehr genügend Fluchkarten vorrätig sind, werden diese im Uhrzeigersinn (beginnend mit deinem linken Nachbarn) verteilt. Die Mitspieler legen die Fluchkarten sofort auf ihren Ablagestapel. Du ziehst immer 2 Karten von deinem Nachziehstapel, auch wenn keine Fluchkarten mehr im Vorrat sind.}
\end{tikzpicture}
\hspace{-0.6cm}
\begin{tikzpicture}
	\card
	\cardstrip
	\cardbanner{banner/white.png}
	\cardicon{icons/coin.png}
	\cardprice{5}
	\cardtitle{Jahrmarkt}
	\cardcontent{Spielst du mehrere \emph{JAHRMÄRKTE} hintereinander, zählst du am besten laut mit, wie viele Aktionen du noch ausspielen darfst, damit du den Überblick behältst.}
\end{tikzpicture}
\hspace{-0.6cm}
\begin{tikzpicture}
	\card
	\cardstrip
	\cardbanner{banner/white.png}
	\cardicon{icons/coin.png}
	\cardprice{3}
	\cardtitle{Kanzler}
	\cardcontent{Legst du deinen Nachziehstapel auf deinen Ablagestapel, musst du dies tun, bevor du eine weitere Aktion ausspielst oder zur Kaufphase übergehst. Du darfst deinen Nachziehstapel nicht einsehen, bevor du ihn ablegst.}
\end{tikzpicture}
\hspace{-0.6cm}
\begin{tikzpicture}
	\card
	\cardstrip
	\cardbanner{banner/white.png}
	\cardicon{icons/coin.png}
	\cardprice{2}
	\cardtitle{Kapelle}
	\cardcontent{Die ausgespielte \emph{KAPELLE} selbst darf nicht entsorgt werden, da sie sich nicht mehr auf der Hand befindet. Weitere \emph{KAPELLEN} auf der Hand dürfen entsorgt werden.}
\end{tikzpicture}
\hspace{-0.6cm}
\begin{tikzpicture}
	\card
	\cardstrip
	\cardbanner{banner/white.png}
	\cardicon{icons/coin.png}
	\cardprice{5}
	\cardtitle{\footnotesize{Laboratorium}}
	\cardcontent{Du \emph{musst} zuerst zwei Karten vom Nachziehstapel auf die Hand nehmen. Dann \emph{darfst} du eine weitere Aktionskarte ausspielen.}
\end{tikzpicture}
\hspace{-0.6cm}
\begin{tikzpicture}
	\card
	\cardstrip
	\cardbanner{banner/white.png}
	\cardicon{icons/coin.png}
	\cardprice{5}
	\cardtitle{\scriptsize{Ratsversammlung}}
	\cardcontent{Jeder Mitspieler \emph{muss} eine Karte von seinem Nachziehstapel auf die Hand nehmen.}
\end{tikzpicture}
\hspace{-0.6cm}
\begin{tikzpicture}
	\card
	\cardstrip
	\cardbanner{banner/white.png}
	\cardicon{icons/coin.png}
	\cardprice{4}
	\cardtitle{Spion}
	\cardcontent{\emph{Zuerst} musst du eine Karte vom Nachziehstapel auf die Hand nehmen. Dann \emph{muss} jeder Spieler (auch du) die oberste Karte seines Nachziehstapels aufdecken. Du entscheidest für jeden Spieler, ob die aufgedeckte Karte auf den Nachziehstapel zurück- oder auf den Ablagestapel abgelegt werden soll. Spieler, deren Nachziehstapel aufgebraucht ist, mischen ihren Ablagestapel und legen ihn als neuen Nachziehstapel bereit. Nur wer weder einen Nachzieh- noch einen Ablagestapel hat, braucht keine Karte aufzudecken. Ist den Mitspielern die Reihenfolge des Aufdeckens wichtig, deckst du zuerst auf – die anderen Spieler folgen im Uhrzeigersinn.}
\end{tikzpicture}
\hspace{-0.6cm}
\begin{tikzpicture}
	\card
	\cardstrip
	\cardbanner{banner/white.png}
	\cardicon{icons/coin.png}
	\cardprice{4}
	\cardtitle{Thronsaal}
	\cardcontent{\emph{Errata:} Der Kartentext ist falsch, es sollte \enquote{Du \emph{darfst} eine beliebige Aktionskarte aus der Hand zweimal ausspielen.} heißen.

	\medskip

	Wähle eine Aktionskarte aus deiner Hand und spiele sie zweimal aus, d. h. du legst die Aktionskarte aus, führst die Anweisungen der Karte komplett aus, nimmst die Karte zurück auf die Hand, legst sie noch einmal aus und führst die Anweisungen erneut aus. Für das doppelte Ausspielen dieser Aktionskarte muss der Spieler keine zusätzlichen Aktionen (+1 Aktion) zur Verfügung haben – sie ist sozusagen \enquote{kostenlos}. Legst du zwei \emph{THRONSAAL}-Karten aus, darfst du zuerst eine Aktion doppelt ausführen und dann eine andere Aktion ebenfalls doppelt ausführen. Du darfst aber nicht ein und dieselbe Aktion viermal ausführen.

	\medskip

	Erlaubt die doppelt ausgespielte Karte +1 Aktion (z.B. der \emph{MARKT}), hast du nach der vollständigen Ausführung des \emph{THRONSAALS} zwei weitere Aktionen zur Verfügung. Hättest du zwei \emph{MARKT}-Karten regulär hintereinander ausgespielt, bliebe dir nur noch eine zusätzliche Aktion zur Verfügung, da das Ausspielen der zweiten Marktkarte schon die zusätzliche Aktion der ersten Karte aufgebraucht hätte. Beim \emph{THRONSAAL} ist es besonders wichtig, laut die verbleibende Anzahl an Aktionen mitzuzählen. Du darfst keine weitere Aktion ausspielen, bevor der \emph{THRONSAAL} komplett abgearbeitet ist.}
\end{tikzpicture}
\hspace{-0.6cm}
\begin{tikzpicture}
	\card
	\cardstrip
	\cardbanner{banner/white.png}
	\cardicon{icons/coin.png}
	\cardprice{3}
	\cardtitle{\footnotesize{Schwarzmarkt}}
	\cardcontent{\tiny{Habt ihr den \emph{SCHWARZMARKT} als eine der 10 Königreichkarten-Sätze ausgewählt, müsst ihr vor Spielbeginn zusätzlich einen verdeckten \emph{SCHWARZMARKT}-Stapel bilden.

	\medskip

	\emph{Der SCHWARZMARKT-Stapel ...}
	... darf nur aus Königreichkarten bestehen, die nicht im Spiel sind, sich also nicht im Vorrat be nden. Er muss mindestens 15 Karten umfassen.

	\medskip

	Ihr einigt euch darauf, welche Königreichkarten ihr im \emph{SCHWARZMARKT}-Stapel haben wollt. Das können auch mehr als 15 Karten sein, ja sogar alle (nicht verwendeten) König- reichkarten dürfen im \emph{SCHWARZMARKT}-Stapel vorkommen.

	\medskip

	Dann kommt von jeder ausgewählten Königreichkarte eine Karte in den \emph{SCHWARZMARKT}-Stapel. Alle Spieler dürfen sich noch mal die Karten im Stapel anschauen. Danach wird der \emph{SCHWARZMARKT}-Stapel gemischt und als verdeckter Stapel neben dem Vorrat bereitgelegt. Dieser Stapel gehört nicht zum Vorrat.

	\medskip

	\emph{Der SCHWARZMARKT-Kauf}
	Er  findet in der Aktionsphase statt, d.h. schon vor der eigentlichen Kaufphase. Zunächst deckt der Spieler die obersten drei Karten des \emph{SCHWARZMARKT}-Stapels auf. Dann darf er beliebig viele Geldkarten ausspielen und eine der drei aufgedeckten Karten kaufen, wenn das Geld (ausgespielte Geldkarten und Geldwerte auf Aktionskarten) dazu reicht. Er darf auch darauf verzichten, eine der aufgedeckten Karten zu kaufen. Nicht gekaufte Karten werden in beliebiger Reihenfolge zurück unter den \emph{SCHWARZMARKT}-Stapel gelegt. Die ausgespielten Geldkarten bleiben zunächst offen liegen. Nicht verwendetes Geld, Geldwerte und Münzen darf der Spieler noch in der Kaufphase seines Zuges nutzen.

	\smallskip}}
\end{tikzpicture}
\hspace{-0.6cm}
\begin{tikzpicture}
	\card
	\cardstrip
	\cardbanner{banner/white.png}
	\cardtitle{\scriptsize{Empfohlene 10er Sätze\qquad}}
	\cardcontent{\emph{Erstes Spiel:}\\
	Burggraben, Dorf, Holzfäller, Keller, Markt, Miliz, Mine, Schmiede, Umbau, Werkstatt

	\smallskip

	\emph{Großes Geld:}\\
	Abenteurer, Bürokrat, Festmahl, Geldverleiher, Kanzler, Kapelle, Laboratorium, Mark, Mine, Thronsaal

	\smallskip

	\emph{Interaktion:}\\
	Bibliothek, Burggraben, Bürokrat, Dieb, Dorf, Jahrmarkt, Kanzler, Miliz, Ratsversammlung, Spion

	\smallskip

	\emph{Im Wandel:}\\
	Dieb, Dorf, Festmahl, Gärten, Hexe, Holzfäller, Kapelle, Keller, Laboratorium, Werkstatt

	\smallskip

	\emph{Dorfplatz:}\\
	Bibliothek, Bürokrat, Dorf, Holzfäller, Jahrmarkt, Keller, Markt, Schmiede, Thronsaal, Umbau}
\end{tikzpicture}
\hspace{0.6cm}

		% Basic settings for this card set
\renewcommand{\cardcolor}{}
\renewcommand{\cardextension}{}
\renewcommand{\cardextensiontitle}{}
\renewcommand{\seticon}{empty.png}

\clearpage
\newpage
\section{Anleitung und Grundausstattung}

\begin{tikzpicture}
	\card
	\cardbanner{banner/white.png}
	\cardtitle{Anleitung (1)\quad}
	\cardcontent{\tiny{\emph{Spielablauf:} Dominion wird zugweise gespielt. Der Spieler an der Reihe hat am Beginn seines Zuges normalerweise 5 Karten auf der Hand. Er führt nun seinen Zug aus, der aus den 3 folgenden Phasen besteht, die immer in dieser Reihenfolge gespielt werden müssen.

	\medskip

	\emph{1. Phase:} Aktion - Der Spieler \emph{darf} Aktionskarten ausspielen.

	\emph{2. Phase:} Kauf - Der Spieler \emph{darf} Karten kaufen.

	\emph{3. Phase:} Aufräumen - Der Spieler \emph{muss} alle  ausgespielten \emph{und} alle Handkarten offen auf seinen Ablagestapel legen und \emph{sofort} 5 Karten für den nächsten Zug nachziehen.

	\medskip

	Die 1. und die 2. Phase darf, die 3. Phase muss gespielt werden. Wenn der Spieler seinen Zug beendet hat, ist der nächste Spieler an der Reihe. Das Spiel verläuft in dieser Weise bis Spielende.
	
	\medskip

	\begin{tabbing}
		Stapel der anderen Spieler: \= Links \kill
		Eigener Ablagestapel: \> Darf weder durchgezählt noch durchgesehen werden. \\
		Eigener Nachziehstapel: \> Darf durchgezählt, nicht aber durchgesehen werden. \\
		Stapel der anderen Spieler: \> Dürfen weder durchgezählt noch durchgesehen werden. \\
		Stapel im Vorrat: \> Dürfen jederzeit durchgezählt und durchgesehen werden. \\
		Müllstapel: \> Darf jederzeit durchgezählt und durchgesehen werden. \\
	\end{tabbing}}}
\end{tikzpicture}
\hspace{-0.6cm}
\begin{tikzpicture}
	\card
	\cardbanner{banner/white.png}
	\cardtitle{Anleitung (2)\quad}
	\cardcontent{\tiny{\begin{tabbing}
	Spielende:xxx \=  18x Provinzen,xxx \= 12 andere Punktekarten,xxx \= 50 Flüche,xxx \= Geld: \kill
	2 Spieler: \> 8x Provinzen, \> 8 andere Punktekarten, \> 10 Flüche, \> Geld: 46 K, 40 S, 30 G\\
	3 Spieler: \> 12x Provinzen, \> 12 andere Punktekarten, \> 20 Flüche, \> Geld: 39 K, 40 S, 30 G\\
	4 Spieler: \> 12x Provinzen, \> 12 andere Punktekarten, \> 30 Flüche, \> Geld: 32 K, 40 S, 30 G\\
	5 Spieler: \> 15x Provinzen, \> 12 andere Punktekarten, \> 40 Flüche, \> Geld: 85 K, 80 S, 60 G\\
	6 Spieler: \> 18x Provinzen, \> 12 andere Punktekarten, \> 50 Flüche, \> Geld: 78 K, 80 S, 60 G\\
	Spielende: \> Provinzstapel, Kolonienstapel (Dominion – Blütezeit) \emph{oder} \\
					\>3 Stapel (1 - 4 Spieler) bzw. 4 Stapel (5 - 6 Spieler) aus dem Vorrat leer \\
	\end{tabbing}
	Punktekarten aus Erweiterungen werden in Anzahl der \enquote{anderen Punktekarten} ausgelegt. Zur Spielende-Bedingung zählen alle Karten im Vorrat, also auch Fluch-, Geld- und Punktekarten, nicht jedoch z. B. der Müllstapel.

	\medskip

	\emph{Platin und Kolonie (Dominion – Blütezeit):} Die im Spiel befindlichen Königreich-Platzhalterkarten werden gemischt. Ist die erste gezogene Königreichskarte eine Karte aus der Blütezeit-Erweiterung, so wird mit Platin und Kolonie gespielt, ansonsten ohne.

	\medskip

	\emph{Unterschlupf-Karten (Dominion – Dark Ages):} Die im Spiel befindlichen Königreich-Platzhalterkarten werden gemischt. Ist die erste gezogene Königreichskarte eine Karte aus der Dark Ages-Erweiterung, so startet jeder Spieler mit 7 Kupfer, 1 Hütte, 1 Totenstadt und 1 Verfallenes Anwesen, andernfalls erhält jeder Spieler 7 Kupfer und 3 Anwesen.}}
\end{tikzpicture}
\hspace{-0.6cm}
\begin{tikzpicture}
	\card
	\cardbanner{banner/white.png}
	\cardtitle{Platzhalter\quad}
\end{tikzpicture}
\hspace{-0.6cm}
\begin{tikzpicture}
	\card
	\cardbanner{banner/gold.png}
	\cardicon{icons/coin.png}
	\cardprice{0}
	\cardtitle{Kupfer}
\end{tikzpicture}
\hspace{-0.6cm}
\begin{tikzpicture}
	\card
	\cardbanner{banner/gold.png}
	\cardicon{icons/coin.png}
	\cardprice{3}
	\cardtitle{Silber}
\end{tikzpicture}
\hspace{-0.6cm}
\begin{tikzpicture}
	\card
	\cardbanner{banner/gold.png}
	\cardicon{icons/coin.png}
	\cardprice{6}
	\cardtitle{Gold}
\end{tikzpicture}
\hspace{-0.6cm}
\begin{tikzpicture}
	\card
	\cardbanner{banner/green.png}
	\cardicon{icons/coin.png}
	\cardprice{2}
	\cardtitle{Anwesen}
\end{tikzpicture}
\hspace{-0.6cm}
\begin{tikzpicture}
	\card
	\cardbanner{banner/green.png}
	\cardicon{icons/coin.png}
	\cardprice{5}
	\cardtitle{Herzogtum}
\end{tikzpicture}
\hspace{-0.6cm}
\begin{tikzpicture}
	\card
	\cardbanner{banner/green.png}
	\cardicon{icons/coin.png}
	\cardprice{8}
	\cardtitle{Provinz}
\end{tikzpicture}
\hspace{-0.6cm}
\begin{tikzpicture}
	\card
	\cardbanner{banner/purple.png}
	\cardicon{icons/coin.png}
	\cardprice{0}
	\cardtitle{Fluch}
\end{tikzpicture}	
\hspace{0.6cm}

		% Basic settings for this card set
\renewcommand{\cardcolor}{basicgame}
\renewcommand{\cardextension}{Second Edition}
\renewcommand{\cardextensiontitle}{Das Basisspiel}
\renewcommand{\seticon}{basic2.png}

\clearpage
\newpage
\section{\cardextension \ - \cardextensiontitle \ (Rio Grande Games 2017)}

\begin{tikzpicture}
	\card
	\cardstrip
	\cardbanner{banner/blue.png}
	\cardicon{icons/coin.png}
	\cardprice{2}
	\cardtitle{Burggraben}
	\cardcontent{Spielt ein anderer Spieler eine Angriffskarte (mit der Aufschrift AKTION -- ANGRIFF), darfst du den \emph{BURGGRABEN} aus deiner Hand vorzeigen, \emph{bevor} der Spieler die Anweisung(en) der Angriffskarte ausführt. In diesem Fall bist \emph{du} davon nicht betroffen, d.h. du musst bei der \emph{HEXE} keine Fluchkarte nehmen usw. Haben mehrere Spieler einen \emph{BURGGRABEN} auf der Hand, dürfen diese auch eingesetzt und vorgezeigt werden. Danach nehmen die Spieler ihre Karte zurück auf die Hand. Der Spieler, der den Angriff gespielt hat, muss unabhängig davon, ob ein oder mehrere \emph{BURGGRÄBEN} gezeigt wurden, die weiteren Anweisungen seiner Aktionskarte ausführen. 

	\medskip

	Der \emph{BURGGRABEN} darf auch in der eigenen Aktionsphase gespielt werden – dann ziehst du 2 Karten nach.}
\end{tikzpicture}
\hspace{-0.6cm}
\begin{tikzpicture}
	\card
	\cardstrip
	\cardbanner{banner/white.png}
	\cardicon{icons/coin.png}
	\cardprice{2}
	\cardtitle{Kapelle}
	\cardcontent{Die ausgespielte \emph{KAPELLE} selbst darf nicht entsorgt werden, da sie sich nicht mehr auf der Hand befindet. Weitere \emph{KAPELLEN} auf der Hand dürfen entsorgt werden.}
\end{tikzpicture}
\hspace{-0.6cm}
\begin{tikzpicture}
	\card
	\cardstrip
	\cardbanner{banner/white.png}
	\cardicon{icons/coin.png}
	\cardprice{2}
	\cardtitle{Keller}
	\cardcontent{Der ausgespielte \emph{KELLER} selbst darf nicht abgelegt werden, da er sich nicht mehr in deiner Hand befindet. Sage an, wie viele Karten du ablegst und lege diese auf deinen Ablagestapel. Danach ziehst du die gleiche Anzahl Karten vom Nachziehstapel. Sollte während dieses Vorgangs der Nachziehstapel aufgebraucht werden, wird dein Ablagestapel zusammen mit den soeben abgelegten Karten gemischt und als neuer Nachziehstapel bereitgelegt.}
\end{tikzpicture}
\hspace{-0.6cm}
\begin{tikzpicture}
	\card
	\cardstrip
	\cardbanner{banner/white.png}
	\cardicon{icons/coin.png}
	\cardprice{3}
	\cardtitle{Dorf}
	\cardcontent{Spielst du mehrere \emph{DÖRFER} hintereinander, zählst du am besten laut mit, wie viele Aktionen du noch ausspielen darfst, damit du den Überblick behältst.}
\end{tikzpicture}
\hspace{-0.6cm}
\begin{tikzpicture}
	\card
	\cardstrip
	\cardbanner{banner/white.png}
	\cardicon{icons/coin.png}
	\cardprice{3}
	\cardtitle{Händlerin}
	\cardcontent{Du ziehst 1 Karte und erhältst + 1 Aktion. Wenn du in diesem Zug vor dem Ausspielen dieser \emph{HÄNDLERIN} noch kein Silber ausgespielt hast, erhältst du für das erste danach ausgespielte Silber +\coin[1]. Für jedes weitere ausgespielte Silber erhältst du keinen zusätzlichen Bonus. Hast du mehrere \emph{HÄNDLERINNEN} ausgespielt, erhältst du pro \emph{HÄNDLERIN} +\coin[1].}
\end{tikzpicture}
\hspace{-0.6cm}
\begin{tikzpicture}
	\card
	\cardstrip
	\cardbanner{banner/white.png}
	\cardicon{icons/coin.png}
	\cardprice{3}
	\cardtitle{Vasall}
	\cardcontent{Ist die aufgedeckte Karte eine Aktionskarte (auch ggf. kombinierte), \emph{darfst} du sie sofort ausspielen. Wenn du sie ausspielst, legst du sie in deinen Spielbereich und führst sofort die Anweisungen darauf aus. Dafür benötigst du keine zusätzliche Aktion. Das Ausspielen der Aktionskarte verbraucht auch keine freie oder zusätzliche Aktion, die du durch das Ausspielen anderer Karten bereits gesammelt hast.}
\end{tikzpicture}
\hspace{-0.6cm}
\begin{tikzpicture}
	\card
	\cardstrip
	\cardbanner{banner/white.png}
	\cardicon{icons/coin.png}
	\cardprice{3}
	\cardtitle{Vorbotin}
	\cardcontent{Du ziehst 1 Karte und erhältst + 1 Aktion. Schau dir deinen Ablagestapel an. Du \emph{darfst} eine Karte daraus auswählen und oben auf deinen Nachziehstapel legen. Die restlichen Karten (oder alle) legst du in beliebiger Reihenfolge zurück auf den Ablagestapel. Ist dein Ablagestapel leer, passiert nichts.}
\end{tikzpicture}
\hspace{-0.6cm}
\begin{tikzpicture}
	\card
	\cardstrip
	\cardbanner{banner/white.png}
	\cardicon{icons/coin.png}
	\cardprice{3}
	\cardtitle{Werkstatt}
	\cardcontent{Nimm dir eine Karte aus dem Vorrat und lege diese sofort auf deinen Ablagestapel. Du kannst weder Geldkarten noch zusätzlich über Aktionskarten erhaltenes Geld oder Münzen (bei Erweiterungen mit Münzen) einsetzen, um den angegebenen Betrag auf der Karte zu erhöhen.}
\end{tikzpicture}
\hspace{-0.6cm}
\begin{tikzpicture}
	\card
	\cardstrip
	\cardbanner{banner/white.png}
	\cardicon{icons/coin.png}
	\cardprice{4}
	\cardtitle{Bürokrat}
	\cardcontent{Ist dein Nachziehstapel aufgebraucht, wenn du diese Karte spielst, legst du die Silberkarte verdeckt ab. Sie bildet dann deinen Nachziehstapel. Das Gleiche gilt für alle Mitspieler, die eine Punktekarte verdeckt auf den eigenen Nachziehstapel legen müssen.}
\end{tikzpicture}
\hspace{-0.6cm}
\begin{tikzpicture}
	\card
	\cardstrip
	\cardbanner{banner/green.png}
	\cardicon{icons/coin.png}
	\cardprice{4}
	\cardtitle{Gärten}
	\cardcontent{Diese Karte ist die einzige Punktekarte unter den Königreichkarten. Sie hat bis zum Ende des Spiels keine Funktion. Bei der Wertung des Spiels erhält der Spieler, der diese Karte in seinem Kartensatz (Nachziehstapel, Handkarten und Ablagestapel) hat, für jeweils 10 Karten einen Siegpunkt. Es wird immer abgerundet, d. h. 39 Karten ergeben  3 Siegpunkte, ebenso wie 31 Karten 3 Siegpunkte ergeben. Wer mehrere \emph{GÄRTEN} besitzt, erhält für jeden \emph{GARTEN} die entsprechende Anzahl an Siegpunkten. }
\end{tikzpicture}
\hspace{-0.6cm}
\begin{tikzpicture}
	\card
	\cardstrip
	\cardbanner{banner/white.png}
	\cardicon{icons/coin.png}
	\cardprice{4}
	\cardtitle{\footnotesize{Geldverleiher}}
	\cardcontent{\emph{Errata:} Der Kartentext ist falsch, es sollte \enquote{Du \emph{darfst} ein Kupfer aus der Hand entsorgen. Wenn du das tust: + \coin[3].} heißen.

	\medskip

	Wer kein Kupfer auf der Hand hat oder keines entsorgen will, erhält kein zusätzliches Geld für die Kaufphase.}
\end{tikzpicture}
\hspace{-0.6cm}
\begin{tikzpicture}
	\card
	\cardstrip
	\cardbanner{banner/white.png}
	\cardicon{icons/coin.png}
	\cardprice{4}
	\cardtitle{Miliz}
	\cardcontent{Deine Mitspieler müssen Karten aus ihrer Hand ablegen, bis sie nur noch 3 Karten auf der Hand haben. Spieler, die zum Zeitpunkt des Angriffs bereits 3 oder weniger Karten auf der Hand haben, müssen keine weiteren Karten ablegen.}
\end{tikzpicture}
\hspace{-0.6cm}
\begin{tikzpicture}
	\card
	\cardstrip
	\cardbanner{banner/white.png}
	\cardicon{icons/coin.png}
	\cardprice{4}
	\cardtitle{Schmiede}
	\cardcontent{Du musst 3 Karten vom Nachziehstapel ziehen und auf die Hand nehmen.}
\end{tikzpicture}
\hspace{-0.6cm}
\begin{tikzpicture}
	\card
	\cardstrip
	\cardbanner{banner/white.png}
	\cardicon{icons/coin.png}
	\cardprice{4}
	\cardtitle{Thronsaal}
	\cardcontent{\emph{Errata:} Der Kartentext ist falsch, es sollte \enquote{Du \emph{darfst} eine beliebige Aktionskarte aus der Hand zweimal ausspielen.} heißen.

	\smallskip

	Spiele den \emph{THRONSAAL} aus, spiele die ausgewählte Aktionskarte aus, führe die Anweisungen vollständig aus und führe sie dann noch einmal vollständig aus \emph{(nimm sie dazwischen NICHT wieder auf die Hand)}. Die Aktionskarte muss beim zweiten Mal nicht mehr im Spiel sein (z.B. weil sie sich selbst entsorgt hat); du führst ihre Abweisungen dann trotzdem so vollständig wie möglich aus. Für das doppelte Ausspielen dieser Aktionskarte muss der Spieler keine zusätzlichen Aktionen (+1 Aktion) zur Verfügung haben – sie ist sozusagen \enquote{kostenlos}. Legst du zwei \emph{THRONSAAL}-Karten aus, darfst du zuerst eine Aktion doppelt ausführen und dann eine andere Aktion ebenfalls doppelt ausführen. Du darfst aber nicht ein und dieselbe Aktion viermal ausführen. 

	Erlaubt die doppelt ausgespielte Karte +1 Aktion (z. B. der \emph{MARKT}), hast du nach der vollständigen Ausführung des \emph{THRONSAALS} zwei weitere Aktionen zur Verfügung. Hättest du zwei \emph{MARKT}-Karten regulär hintereinander ausgespielt, bliebe dir nur noch eine zusätzliche Aktion zur Verfügung, da das Ausspielen der zweiten Marktkarte schon die zusätzliche Aktion der ersten Karte aufgebraucht hätte. Beim \emph{THRONSAAL} ist es besonders wichtig, laut die verbleibende Anzahl an Aktionen mitzuzählen. Du darfst keine weitere Aktion ausspielen, bevor der \emph{THRONSAAL} komplett abgearbeitet ist.}
\end{tikzpicture}
\hspace{-0.6cm}
\begin{tikzpicture}
	\card
	\cardstrip
	\cardbanner{banner/white.png}
	\cardicon{icons/coin.png}
	\cardprice{4}
	\cardtitle{Umbau}
	\cardcontent{Der ausgespielte \emph{UMBAU} selbst darf nicht entsorgt werden, da er sich nicht mehr in deiner Hand befindet. Weitere \emph{UMBAU}-Karten in deiner Hand dürfen entsorgt werden. Wenn du keine Karte zum Entsorgen auf der Hand hast, darfst du dir auch keine neue Karte nehmen. Die neue Karte darf maximal bis zu \coin[2] mehr als die entsorgte Karte kosten. Der Betrag darf weder durch weitere Geldkarten, Münzen oder zusätzliches Geld von anderen Aktionskarten erhöht werden. Die neue Karte kann die gleiche Karte sein wie die, die du entsorgt hast. Lege die neue Karte auf deinen Ablagestapel.}
\end{tikzpicture}
\hspace{-0.6cm}
\begin{tikzpicture}
	\card
	\cardstrip
	\cardbanner{banner/white.png}
	\cardicon{icons/coin.png}
	\cardprice{4}
	\cardtitle{Wilddiebin}
	\cardcontent{Du ziehst 1 Karte, erhältst + 1 Aktion und +\coin[1]. Dann schaust du, wie viele Vorratsstapel (Fluch-, Geld-, Punkte- und Aktionskarten) bereits leer sind. Ist kein Stapel leer, musst du keine Handkarten ablegen. Ist ein Stapel leer, legst du 1 Handkarte ab usw. Wenn du nicht so viele Karten auf der Hand hast, wie Vorratsstapel leer sind, legst du so viele Karten ab, wie du kannst.}
\end{tikzpicture}
\hspace{-0.6cm}
\begin{tikzpicture}
	\card
	\cardstrip
	\cardbanner{banner/white.png}
	\cardicon{icons/coin.png}
	\cardprice{5}
	\cardtitle{Banditin}
	\cardcontent{Zuerst nimmst du ein Gold vom Vorrat und legst es auf deinen Ablagestapel. Dann deckt jeder Mitspieler – beginnend bei deinem linken Mitspieler – die obersten zwei Karten seines Nachziehstapels auf. Deckt ein Spieler zwei Geldkarten (auch ggf. kombinierte) außer Kupfer auf, muss er \emph{eine} davon entsorgen. Dabei darf er selbst entscheiden, welche Geldkarte er entsorgt. Die andere Geldkarte wird – genauso wie alle anderen Karten – abgelegt. Deckt ein Spieler eine Geldkarte außer Kupfer sowie eine andere Karte (z. B. ein Kupfer oder eine beliebige Aktionskarte) auf, wird diese Geldkarte entsorgt. Die andere aufgedeckte Karte wird abgelegt.}
\end{tikzpicture}
\hspace{-0.6cm}
\begin{tikzpicture}
	\card
	\cardstrip
	\cardbanner{banner/white.png}
	\cardicon{icons/coin.png}
	\cardprice{5}
	\cardtitle{Bibliothek}
	\cardcontent{Aktionskarten darfst du zur Seite legen, sobald du sie ziehst, musst dies aber nicht tun. Hast du bereits 7 oder mehr Karten auf der Hand, wenn du die \emph{BIBLIOTHEK} ausspielst, ziehst du keine Karten nach. Wenn dein Nachziehstapel während des Ziehens aufgebraucht ist, mischst du den Ablagestapel, mischst aber die zur Seite gelegten Aktionskarten nicht mit ein. Diese werden erst auf den Ablagestapel gelegt, sobald du  das Ziehen beendet hast. Sollten die Karten nicht reichen, ziehst du nur so viele Karten wie möglich.}
\end{tikzpicture}
\hspace{-0.6cm}
\begin{tikzpicture}
	\card
	\cardstrip
	\cardbanner{banner/white.png}
	\cardicon{icons/coin.png}
	\cardprice{5}
	\cardtitle{Hexe}
	\cardcontent{Wenn du die \emph{HEXE} spielst und nicht mehr genügend Fluchkarten vorrätig sind, werden diese im Uhrzeigersinn (beginnend mit deinem linken Nachbarn) verteilt. Die Mitspieler legen die Fluchkarten sofort auf ihren Ablagestapel. Du ziehst immer 2 Karten von deinem Nachziehstapel, auch wenn keine Fluchkarten mehr im Vorrat sind.}
\end{tikzpicture}
\hspace{-0.6cm}
\begin{tikzpicture}
	\card
	\cardstrip
	\cardbanner{banner/white.png}
	\cardicon{icons/coin.png}
	\cardprice{5}
	\cardtitle{Jahrmarkt}
	\cardcontent{Spielst du mehrere \emph{JAHRMÄRKTE} hintereinander, zählst du am besten laut mit, wie viele Aktionen du noch ausspielen darfst, damit du den Überblick behältst.}
\end{tikzpicture}
\hspace{-0.6cm}
\begin{tikzpicture}
	\card
	\cardstrip
	\cardbanner{banner/white.png}
	\cardicon{icons/coin.png}
	\cardprice{5}
	\cardtitle{\footnotesize{Laboratorium}}
	\cardcontent{Du \emph{musst} zuerst zwei Karten vom Nachziehstapel auf die Hand nehmen. Dann \emph{darfst} du eine weitere Aktionskarte ausspielen.}
\end{tikzpicture}
\hspace{-0.6cm}
\begin{tikzpicture}
	\card
	\cardstrip
	\cardbanner{banner/white.png}
	\cardicon{icons/coin.png}
	\cardprice{5}
	\cardtitle{Markt}
	\cardcontent{Du musst eine Karte vom Nachziehstapel auf die Hand nehmen. Du \emph{darfst} in der Aktionsphase eine weitere Aktionskarte ausspielen. Du \emph{darfst} in der Kaufphase einen zusätzlichen Kauf tätigen und hast dafür ein zusätzliches Geld zur Verfügung.}
\end{tikzpicture}
\hspace{-0.6cm}
\begin{tikzpicture}
	\card
	\cardstrip
	\cardbanner{banner/white.png}
	\cardicon{icons/coin.png}
	\cardprice{5}
	\cardtitle{Mine}
	\cardcontent{\emph{Errata:} Der Kartentext ist falsch, es sollte \enquote{Du \emph{darfst} eine beliebige Geldkarte aus der Hand entsorgen. Nimm eine Geldkarte vom Vorrat auf die Hand, die bis zu \coin[3] mehr kostet.} heißen.

	\medskip

	Normalerweise entsorgst du ein Kupfer und nimmst dir dafür ein Silber, oder du entsorgst ein Silber und nimmst dir ein Gold. Du kannst dir aber auch eine gleichwertige oder billigere Karte nehmen. Die neue Karte nimmst du sofort auf die Hand und darfst sie noch während deines Zuges einsetzen. Wer keine Geldkarte zum Entsorgen hat oder keine entsorgen will, darf keine Geldkarte nehmen.}
\end{tikzpicture}
\hspace{-0.6cm}
\begin{tikzpicture}
	\card
	\cardstrip
	\cardbanner{banner/white.png}
	\cardicon{icons/coin.png}
	\cardprice{5}
	\cardtitle{\scriptsize{Ratsversammlung}}
	\cardcontent{Jeder Mitspieler \emph{muss} eine Karte von seinem Nachziehstapel auf die Hand nehmen.}
\end{tikzpicture}
\hspace{-0.6cm}
\begin{tikzpicture}
	\card
	\cardstrip
	\cardbanner{banner/white.png}
	\cardicon{icons/coin.png}
	\cardprice{5}
	\cardtitle{\footnotesize{Torwächterin}}
	\cardcontent{Du ziehst 1 Karte und erhältst + 1 Aktion. Dann siehst du dir die obersten 2 Karten deines Nachziehstapels an. Du kannst beide Karten entsorgen, beide Karten ablegen oder sie in beliebiger Reihenfolge zurück auf den Nachziehstapel legen. Du kannst aber auch eine entsorgen und eine ablegen, oder eine entsorgen und die andere zurück auf den Nachziehstapel legen, oder eine ablegen und die andere zurücklegen.}
\end{tikzpicture}
\hspace{-0.6cm}
\begin{tikzpicture}
	\card
	\cardstrip
	\cardbanner{banner/white.png}
	\cardicon{icons/coin.png}
	\cardprice{6}
	\cardtitle{Töpferei}
	\cardcontent{Nimm eine Karte vom Vorrat, die zu diesem Zeitpunkt maximal \coin[5] kostet. Du darfst kein zusätzliches \coin einsetzen, um dir eine teurere Karte zu nehmen. Außer \coin darf die Karte keine zusätzlichen Kosten enthalten. 

	\medskip

	Du darfst dir zum Beispiel keine Karte mit \potion (aus Alchemie) oder \hex (aus Empires) in den Kosten nehmen. Die genommene Karte nimmst du direkt auf die Hand. Anschließend legst du eine beliebige Handkarte (das kann die gerade genommene oder eine andere sein) oben auf deinen Nachziehstapel.}
\end{tikzpicture}
\hspace{-0.6cm}
\begin{tikzpicture}
	\card
	\cardstrip
	\cardbanner{banner/white.png}
	\cardtitle{\scriptsize{Empfohlene 10er Sätze\qquad}}
	\cardcontent{\emph{Erstes Spiel:}\\
		Burggraben, Dorf, Händlerin, Keller, Markt, Miliz, Mine, Schmiede, Umbau, Werkstatt
	
		\smallskip
	
		\emph{Verzerrte Größen:}\\
		Banditin, Bürokrat, Gärtner, Hexe, Jahrmarkt, Kapelle, Thronsaal, Töpferei, Torwächterin, Werkstatt
	
		\smallskip
	
		\emph{Schleichweg:}\\
		Bürokrat, Dorf, Geldverleiher, Laboratorium, Jahrmarkt, Ratsversammlung, Töpferei, Torwächterin, Vasall, Vorbotin
	
		\smallskip
	
		\emph{Kunststück:}\\
		Bibliothek, Gärten, Jahrmarkt, Keller, Miliz, Ratsversammlung, Schmiede, Thronsaal, Vorbotin, Wilddiebin
	
		\smallskip
	
		\emph{Verbesserungen:}\\
		Burggraben, Geldverleiher, Händlerin, Hexe, Keller, Markt, Mine, Töpferei, Umbau, Wilddiebin
	
		\smallskip
	
		\emph{Silber \& Gold:}\\
		Bandit, Bürokrat, Geldverleiher, Händlerin, Kapelle, Laboratorium, Mine, Thronsaal, Vasall, Vorbotin}
\end{tikzpicture}
\hspace{0.6cm}

		% Basic settings for this card set
\renewcommand{\cardcolor}{intrigue}
\renewcommand{\cardextension}{Edition II}
\renewcommand{\cardextensiontitle}{Die Intrige}
\renewcommand{\seticon}{intrigue1.png}

\clearpage
\newpage
\section{\cardextension \ - \cardextensiontitle \ (Hans im Glück 2009)}


\begin{tikzpicture}
	\card
	\cardstrip
	\cardbanner{banner/white.png}
	\cardicon{icons/coin.png}
	\cardprice{2}
	\cardtitle{Burghof}
	\cardcontent{Du ziehst 3 Karten und nimmst diese auf die Hand bevor du eine Karte auf den Nachziehstapel legst. Die Karte, die du auf den Nachziehstapel legst, muss keine der 3 gerade gezogenen Karten sein.}
\end{tikzpicture}
\hspace{-0.6cm}
\begin{tikzpicture}
	\card
	\cardstrip
	\cardbanner{banner/blue.png}
	\cardicon{icons/coin.png}
	\cardprice{2}
	\cardtitle{\footnotesize{Geheimkammer}}
	\cardcontent{Wenn du die Geheimkammer in deinem Zug ausspielst, legst du zuerst eine beliebige Anzahl Handkarten ab. Du erhältst dann +1 virtuelles Geld für jede abgelegte Karte. Du darfst auch 0 Karten ablegen. Die andere Anweisung auf der Karte tritt nur in Kraft, wenn du sie als Reaktion auf einen Angriff aus deiner Hand aufdeckst. Wenn du das machst, ziehst du zuerst 2 Karten nach und legst dann 2 Karten aus deiner Hand auf den Nachziehstapel. Du kannst 2 beliebige deiner Handkarten auf den Nachziehstapel legen, nicht notwendigerweise die gerade gezogenen. Du kannst auch die Geheimkammer selbst auf den Nachziehstapel legen. Du kannst die Geheimkammer immer aufdecken, wenn ein anderer Spieler einen Angriff ausspielt, auch wenn dich der Angriff nicht betrifft. Du kannst auch mehrere Reaktionskarten bei einem Angriff aufdecken. Du kannst z.B. zuerst die Geheimkammer aufdecken und abwickeln und danach den Angriff mit dem Burggraben abwehren. Die Geheimkammer selbst wehrt einen Angriff nicht ab.}
\end{tikzpicture}
\hspace{-0.6cm}
\begin{tikzpicture}
	\card
	\cardstrip
	\cardbanner{banner/white.png}
	\cardicon{icons/coin.png}
	\cardprice{2}
	\cardtitle{Handlanger}
	\cardcontent{Wähle 2 verschiedene Anweisungen. Du darfst nicht eine Anweisung zweimal wählen. Du musst zuerst beide Anweisungen auswählen und sie dann erst (in jeder möglichen Reihenfolge) ausführen. Du kannst nicht eine Karte nachziehen und dann erst die zweite Anweisung wählen.}
\end{tikzpicture}
\hspace{-0.6cm}
\begin{tikzpicture}
	\card
	\cardstrip
	\cardbanner{banner/white.png}
	\cardicon{icons/coin.png}
	\cardprice{3}
	\cardtitle{Armenviertel}
	\cardcontent{Du erhältst 2 zusätzliche Aktionen. Dann \emph{musst} du deine Kartenhand vorzeigen. Wenn du keine Aktionskarten auf der Hand hast (auch kombinierte Aktions-/Punktekarten sind Aktionskarten), musst du 2 Karten nachziehen. Sollte die erste der gezogenen Karten eine Aktionskarte sein, ziehst du trotzdem eine zweite Karte.}
\end{tikzpicture}
\hspace{-0.6cm}
\begin{tikzpicture}
	\card
	\cardstrip
	\cardbanner{banner/whitegreen.png}
	\cardicon{icons/coin.png}
	\cardprice{3}
	\cardtitle{Grosse Halle}
	\cardcontent{Diese Karte ist /emph{zugleich} eine Aktions- und eine Punktekarte. Wenn du sie ausspielst, ziehst du sofort eine Karte nach und darfst dann eine weitere Aktionskarte ausspielen. Bei Spielende ist die Karte 1 Punkt wert, wie ein Anwesen. Im Spiel zu 3. und zu 4. werden 12 Karten verwendet, im Spiel zu 2. werden 8 Karten verwendet.}
\end{tikzpicture}
\hspace{-0.6cm}
\begin{tikzpicture}
	\card
	\cardstrip
	\cardbanner{banner/white.png}
	\cardicon{icons/coin.png}
	\cardprice{3}
	\cardtitle{Maskerade}
	\cardcontent{Du ziehst zuerst 2 Karten. Dann wählen alle Spieler gleichzeitig eine Karte aus ihrer Hand und legen diese verdeckt zwischen sich und den Spieler zu ihrer Linken. Erst dann nehmen alle Spieler die Karten, die sie vom Spieler rechts bekommen haben auf. Die Spieler wählen also zuerst, welche Karte sie weiter geben, bevor sie sehen, welche Karte sie bekommen. Am Ende darfst nur du eine Karte aus deiner Hand entsorgen. Die Maskerade ist kein Angriff. Die übrigen Spieler dürfen also keine Reaktionskarten aus ihrer Hand vorzeigen um sich zu schützen.}
\end{tikzpicture}
\hspace{-0.6cm}
\begin{tikzpicture}
	\card
	\cardstrip
	\cardbanner{banner/white.png}
	\cardicon{icons/coin.png}
	\cardprice{3}
	\cardtitle{Trickser}
	\cardcontent{Jeder Mitspieler, beginnend mit dem Spieler links vom Angreifer, muss die oberste Karte von seinem Nachziehstapel aufdecken. Er muss diese Karte entsorgen und du wählst eine Karte aus dem Vorrat, die das gleiche kostet. Diese Karte nimmt sich der Spieler und legt sie bei sich ab. Ist im Vorrat keine Karte, mit den gleichen Kosten, erhält der Spieler nichts, muss jedoch die Karte trotzdem entsorgen. Entsorgt er z. B. ein Kupfer, kannst du einen Fluch auswählen, den er nehmen muss. Du kannst auch die selbe Karte wählen, die er entsorgt hat. Die gewählte Karte muss im Vorrat zur Verfügung stehen. Du kannst also keine Karte aus einem leeren Stapel oder vom Müll wählen. Sind keine Karten mehr im Vorrat, die genau so viel kosten, wie die entsorgte Karte, erhält der Spieler nichts. Deckt ein Spieler den Burggraben aus seiner Hand auf, muss er keine Karte vom Nachziehstapel aufdecken und entsorgen und erhält auch keine Karte.}
\end{tikzpicture}
\hspace{-0.6cm}
\begin{tikzpicture}
	\card
	\cardstrip
	\cardbanner{banner/white.png}
	\cardicon{icons/coin.png}
	\cardprice{3}
	\cardtitle{Verwalter}
	\cardcontent{Wenn du dich entscheidest, 2 Karten zu entsorgen und 2 oder mehr Karten auf der Hand hast, musst du genau 2 Karten entsorgen. Wenn du dich entscheidest, 2 Karten zu entsorgen, aber aber nur 1 Karte auf der Hand hast musst du diese Karte entsorgen. Du kannst die verschiedenen Anweisungen nicht mischen, du musst wählen: \emph{entweder} +2 Karten \emph{oder} +2 Geld \emph{oder} 2 Karten entsorgen.}
\end{tikzpicture}
\hspace{-0.6cm}
\begin{tikzpicture}
	\card
	\cardstrip
	\cardbanner{banner/white.png}
	\cardicon{icons/coin.png}
	\cardprice{3}
	\cardtitle{\scriptsize{Wunschbrunnen}}
	\cardcontent{Du ziehst zuerst eine Karte nach. Dann benennst du eine Karte (z. B. \enquote{Kupfer}, nicht \enquote{Geld}) und deckst die oberste Karte von deinem Nachziehstapel auf. Wenn es sich um die benannte Karte handelt, nimmst du sie auf die Hand. Wenn nicht, legst du sie zurück auf den Nachziehstapel.}
\end{tikzpicture}
\hspace{-0.6cm}
\begin{tikzpicture}
	\card
	\cardstrip
	\cardbanner{banner/white.png}
	\cardicon{icons/coin.png}
	\cardprice{4}
	\cardtitle{Baron}
	\cardcontent{Du musst kein Anwesen ablegen, auch wenn du eines auf der Hand hast. Wenn du jedoch keines ablegst, musst du dir ein Anwesen nehmen, so lange noch welche im Vorrat sind. Du kannst nicht nur den +1 Kauf nutzen und die übrigen Anweisungen ignorieren.}
\end{tikzpicture}
\hspace{-0.6cm}
\begin{tikzpicture}
	\card
	\cardstrip
	\cardbanner{banner/white.png}
	\cardicon{icons/coin.png}
	\cardprice{4}
	\cardtitle{Bergwerk}
	\cardcontent{Du ziehst immer eine Karte nach und erhältst 2 zusätzliche Aktionen. Dann musst du entscheiden, ob du das Bergwerk entsorgst, bevor du weitere Aktionen ausspielst oder in die anderen Phasen übergehst. Wenn du das Bergwerk auf einen Thronsaal spielst, kannst du die Karte nur einmal entsorgen (d. h., du erhältst insgesamt +2 Karten, +4 Aktionen aber nur +2 Geld ).}
\end{tikzpicture}
\hspace{-0.6cm}
\begin{tikzpicture}
	\card
	\cardstrip
	\cardbanner{banner/white.png}
	\cardicon{icons/coin.png}
	\cardprice{4}
	\cardtitle{Brücke}
	\cardcontent{Die Kosten sind für alle Belange um 1 Geld reduziert. Wenn du z. B. ein Bergwerk ausspielst, danach eine Brücke, dann eine Eisenhütte, könntest du dir für die Eisen- hütte ein Herzogtum nehmen (kostet durch die Brücke nur noch 4 Geld). Die Karten der Spieler (Handkarten, Nachziehstapel und Ablagestapel) sind auch betroffen. Der Effekt ist kumulativ. Wenn du die Brücke auf einen Thronsaal spielst, sind die Kosten der Karten für diesen Zug um 2 Geld reduziert. Die Kosten sinken niemals unter 0 Geld. Wenn du eine Brücke und dann einen Anbau ausspielst, kannst du ein Kupfer entsorgen (das immer noch 0 Geld kostet) und dir einen Handlanger dafür nehmen (kostet durch die Brücke nur noch 1 Geld).}
\end{tikzpicture}
\hspace{-0.6cm}
\begin{tikzpicture}
	\card
	\cardstrip
	\cardbanner{banner/white.png}
	\cardicon{icons/coin.png}
	\cardprice{4}
	\cardtitle{Eisenhütte}
	\cardcontent{Du nimmst dir eine Karte vom Vorrat und legst sie auf deinen Ablagestapel. Je nach Kartentyp der genommenen Karte erhältst du einen Bonus. Nimmst du eine Karte mit kombiniertem Kartentyp, z. B. Große Halle erhältst du +1 Aktion (weil die Große Halle eine Aktionskarte ist) und +1 Karte (weil die Große Halle auch eine Punktekarte ist).}
\end{tikzpicture}
\hspace{-0.6cm}
\begin{tikzpicture}
	\card
	\cardstrip
	\cardbanner{banner/white.png}
	\cardicon{icons/coin.png}
	\cardprice{4}
	\cardtitle{\footnotesize{Kupferschmied}}
	\cardcontent{Diese Karte verändert, wieviel Geld ein Kupfer einbringt. Der Effekt ins kumulativ, wenn du den Kupferschmied auf einen Thronsaal spielst, bringt jedes Kupfer 3 Geld.}
\end{tikzpicture}
\hspace{-0.6cm}
\begin{tikzpicture}
	\card
	\cardstrip
	\cardbanner{banner/white.png}
	\cardicon{icons/coin.png}
	\cardprice{4}
	\cardtitle{Späher}
	\cardcontent{Wenn der Nachziehstapel leer ist und du deinen Nachziehstapel mischt, werden
	die bereits aufgedeckten Karten nicht mit eingemischt. Du musst alle Punktekarten auf die Hand nehmen. Kombinierte Aktions-/Punktekarten sind auch Punktekarten. Fluchkarten sind keine Punktekarten. Du musst die Reihenfolge, in der du die Karten auf den Nachziehstapel legst nicht zeigen.}
\end{tikzpicture}
\hspace{-0.6cm}
\begin{tikzpicture}
	\card
	\cardstrip
	\cardbanner{banner/white.png}
	\cardicon{icons/coin.png}
	\cardprice{4}
	\cardtitle{Verschwörer}
	\cardcontent{Du überprüfst die Bedingung ob du +1 Karte und +1 Aktion erhältst wenn du den Verschwörer ausgespielt hast. Wenn die Bedingung später im Zug erfüllt wird, überprüfst du die Bedingung nicht rückwirkend. Wird eine Karte auf den Thronsaal gespielt, zählt der Thronsaal selbst als gespielte Aktionskarte und die darauf gespielte Aktionskarte zusätzlich zweimal als gespielte Aktionskarte. Wenn du z. B. den Verschwörer auf den Thronsaal spielst, ist der Thronsaal die erste Aktionskarte, der zuerst ausgespielte Verschwörer ist die zweite Aktionskarte (du erhältst also keine +1 Karte und keine +1 Aktion). Wenn du den Verschwörer zum zweiten mal ausspielst, hast du 3 Aktionskarten ausgespielt und erhältst +1 Karte und +1 Aktion.}
\end{tikzpicture}
\hspace{-0.6cm}
\begin{tikzpicture}
	\card
	\cardstrip
	\cardbanner{banner/white.png}
	\cardicon{icons/coin.png}
	\cardprice{5}
	\cardtitle{Anbau}
	\cardcontent{Du ziehst zuerst eine Karte. Danach \emph{musst} du eine Karte aus deiner Hand entsorgen und dann eine Karte nehmen, die genau 1 Geld mehr kostet als die entsorgte Karte. Ist keine solche Karte im Vorrat, erhältst du keine Karte, musst jedoch trotzdem eine entsorgen. Wenn du keine Karte zum Entsorgen hast, entsorgst du keine und nimmst dir keine Karte.}
\end{tikzpicture}
\hspace{-0.6cm}
\begin{tikzpicture}
	\card
	\cardstrip
	\cardbanner{banner/white.png}
	\cardicon{icons/coin.png}
	\cardprice{5}
	\cardtitle{\footnotesize{Handelsposten}}
	\cardcontent{Wenn du 2 oder mehr Karten auf der Hand hast, \emph{musst} du 2 Karten entsorgen und dir dafür ein Silber nehmen. Du nimmst das Silber direkt auf die Hand und kannst es auch in der Kaufphase verwenden. Wenn kein Silber mehr im Vorrat ist, erhältst du kein Silber, musst jedoch trotzdem 2 Karten entsorgen. Wenn du nur 1 Karte auf der Hand hast, musst du diese entsorgen, erhältst jedoch kein Silber. Wenn du keine Karte mehr auf der Hand hast, kannst du nichts entsorgen und erhältst auch kein Silber.}
\end{tikzpicture}
\hspace{-0.6cm}
\begin{tikzpicture}
	\card
	\cardstrip
	\cardbanner{banner/green.png}
	\cardicon{icons/coin.png}
	\cardprice{5}
	\cardtitle{Herzog}
	\cardcontent{Diese Karte hat bis zum Ende des Spiels keine Funktion. Bei Spielende ist der Herzog 1 Punkt pro Herzogtum in Handkarten, Nachziehstapel und Ablagestapel wert. Im Spiel zu 3. und zu 4. werden 12 Karten verwendet, im Spiel zu 2. werden 8 Karten verwendet.}
\end{tikzpicture}
\hspace{-0.6cm}
\begin{tikzpicture}
	\card
	\cardstrip
	\cardbanner{banner/white.png}
	\cardicon{icons/coin.png}
	\cardprice{5}
	\cardtitle{\footnotesize{Kerkermeister}}
	\cardcontent{Jeder Mitspieler, beginnend mit dem Spieler links vom Angreifer, muss sich eine der beiden Anweisungen wählen und diese dann ausführen. Ein Spieler kann wählen, 2 Karten abzulegen, auch wenn er weniger als 2 Karten auf der Hand hat. Hat er nur eine Karte auf der Hand legt er diese ab. Hat er keine Karte mehr auf der Hand, muss er auch keine ablegen. Ein Spieler kann wählen einen Fluch zu nehmen, auch wenn keine Fluchkarten mehr im Vorrat sind. In diesem Fall nimmt er keinen Fluch. Fluchkarten nehmen die Spieler direkt auf die Hand.}
\end{tikzpicture}
\hspace{-0.6cm}
\begin{tikzpicture}
	\card
	\cardstrip
	\cardbanner{banner/white.png}
	\cardicon{icons/coin.png}
	\cardprice{5}
	\cardtitle{Lakai}
	\cardcontent{Zunächst entscheidest du dich für eine der der beiden Anweisungen. Entweder erhältst du +2 virtuelles Geld \emph{oder} du wählst die zweite Anweisung für den Angriff. In diesem Fall sind nur Spieler mit 5 oder mehr Karten auf der Hand betroffen. Wehrt ein Spieler den Angriff mit einem Burggraben ab, darf er weder Karten nachziehen, noch muss er Karten ablegen. Ein Spieler kann auf den Angriff mit der Geheimkammer reagieren, auch wenn er weniger als 5 Karten auf der Hand hat. Anschließend hast du +1 Aktion, unabhängig davon, welche der beiden Anweisungen du gewählt hast.}
\end{tikzpicture}
\hspace{-0.6cm}
\begin{tikzpicture}
	\card
	\cardstrip
	\cardbanner{banner/white.png}
	\cardicon{icons/coin.png}
	\cardprice{5}
	\cardtitle{Saboteur}
	\cardcontent{Jeder Mitspieler, beginnend mit dem Spieler links vom Angreifer, muss solange Karten von seinem Nachziehstapel aufdecken, bis er eine Karte aufdeckt, die 3 Geld oder mehr kostet. Wenn der Nachziehstapel leer ist, mischt er die bereits aufgedeckten Karten nicht mit ein. Wenn er eine Karte aufdeckt, die 3 Geld oder mehr kostet, muss er diese entsorgen und darf sich dann eine Karte aus dem Vorrat nehmen, die 2 Geld weniger kostet oder billiger ist. Die übrigen aufgedeckten Karten legt er ab. Findet er keine Karte, die 3 Geld oder mehr kostet, legt er seine aufgedeckten Karten ab und nichts weiter passiert.}
\end{tikzpicture}
\hspace{-0.6cm}
\begin{tikzpicture}
	\card
	\cardstrip
	\cardbanner{banner/white.png}
	\cardicon{icons/coin.png}
	\cardprice{5}
	\cardtitle{Tribut}
	\cardcontent{Wenn sein Nachziehstapel leer ist, mischt der Spieler links von dir die bereits aufgedeckte Karte nicht mit ein. Dann legt er die Karten ab. Hat er weniger als 2 Karten in Nachzieh- und Ablagestapel, deckt er nur so viele auf, wie er hat. Du erhältst Boni für die Kartentypen und nur für unterschiedliche Karten. Wenn der Spieler z.B. ein Kupfer und einen Harem aufdeckt, bekommst du +4 Geld und +2 Karten. Wenn er 2 Silber aufdeckt bekommst du +2 Geld.}
\end{tikzpicture}
\hspace{-0.6cm}
\begin{tikzpicture}
	\card
	\cardstrip
	\cardbanner{banner/whitegreen.png}
	\cardicon{icons/coin.png}
	\cardprice{6}
	\cardtitle{Adelige}
	\cardcontent{Diese Karte ist \emph{zugleich} eine Aktions- und eine Punktekarte. Wenn du sie ausspielst, kannst du wählen, 3 Karten nachzuziehen \emph{oder} 2 zusätzliche Aktionen zu erhalten. Die beiden Anweisungen können jedoch nicht geteilt und gemischt werden. Bei Spielende sind die Adeligen 2 Punkte wert. Im Spiel zu 3. und zu 4. werden 12 Karten verwendet, im Spiel zu 2. werden 8 Karten verwendet.}
\end{tikzpicture}
\hspace{-0.6cm}
\begin{tikzpicture}
	\card
	\cardstrip
	\cardbanner{banner/goldgreen.png}
	\cardicon{icons/coin.png}
	\cardprice{6}
	\cardtitle{Harem}
	\cardcontent{Diese Karte ist zugleich eine Geld- und eine Punktekarte. Du kannst sie in der Kaufphase spielen, genau wie ein Silber. Bei Spielende ist der Harem 2 Punkte wert. Im Spiel zu 3. und zu 4. werden 12 Karten verwendet, im Spiel zu 2. werden 8 Karten verwendet.}
\end{tikzpicture}
\hspace{-0.6cm}
\begin{tikzpicture}
	\card
	\cardstrip
	\cardbanner{banner/white.png}
	\cardtitle{\scriptsize{Empfohlene 10er Sätze\qquad}}
	\cardcontent{\emph{Siegestanz:}\\
	Handlanger, Große Halle, Maskerade, Brücke, Eisenhütte, Späher, Anbau, Herzog, Adlige, Harem

	\smallskip

	\emph{Geheime Pläne:}\\
	Handlanger, Armenviertel, Trickser, Verwalter, Eisenhütte, Verschwörer, Handelsposten, Saboteur, Tribut, Harem

	\smallskip

	\emph{Beste Wünsche:}\\
	Burghof, Armenviertel, Maskerade, Verwalter, Wunschbrunnen, Kupferschmied, Späher, Anbau, Handelsposten, Kerkermeister

	\smallskip

	\emph{Demontage} (Intrige + \textit{Basisspiel}):\\
	Geheimkammer, Trickser, Bergwerk, Brücke, Kerkermeister, Saboteur, \textit{Dieb}, \textit{Spion}, \textit{Thronsaal}, \textit{Umbau}

	\smallskip

	\emph{Eine Hand voll} (Intrige + \textit{Basisspiel}):\\
	Burghof, Verwalter, Kerkermeister, Lakai, Adlige, \textit{Bürokrat}, \textit{Kanzler}, \textit{Miliz}, \textit{Mine}, \textit{Ratsversammlung}

	\smallskip

	\emph{Untergebene} (Intrige + \textit{Basisspiel}):\\
	Handlanger, Maskerade, Verwalter, Baron, Lakai, Adlige, \textit{Bibliothek}, \textit{Hexe}, \textit{Jahrmarkt}, \textit{Keller}}
\end{tikzpicture}
\hspace{-0.6cm}
	    % Basic settings for this card set
\renewcommand{\cardcolor}{intrigue}
\renewcommand{\cardextension}{Erweiterung}
\renewcommand{\cardextensiontitle}{Die Intrige}
\renewcommand{\seticon}{intrigue1.png}

\clearpage
\newpage
\section{\cardextension \ - \cardextensiontitle \ (Rio Grande Games 2014)}

\begin{tikzpicture}
	\card
	\cardstrip
	\cardbanner{banner/white.png}
	\cardicon{icons/coin.png}
	\cardprice{2}
	\cardtitle{Burghof}
	\cardcontent{Du ziehst 3 Karten von deinem Nachziehstapel und nimmst sie auf die Hand. Dann wählst du eine beliebige Karte aus deiner Hand und legst sie verdeckt auf den Nachziehstapel.}
\end{tikzpicture}
\hspace{-0.6cm}
\begin{tikzpicture}
	\card
	\cardstrip
	\cardbanner{banner/blue.png}
	\cardicon{icons/coin.png}
	\cardprice{2}
	\cardtitle{\footnotesize{Geheimkammer}}
	\cardcontent{Wenn du diese Karte in deiner eigenen Aktionsphase ausspielst, legst du eine beliebige Anzahl Karten (auch 0 Karten) aus deiner Hand ab. \emph{Danach} erhältst du +\coin[1] pro abgelegte Karte. 

	\medskip

	Spielt ein anderer Spieler eine Angriffskarte aus, darfst du diese Karte vorzeigen, wenn du sie in diesem Moment auf der Hand hast, auch wenn dich der Angriff nicht betrifft. Wenn du das tust, ziehst du zuerst 2 Karten nach und legst dann 2 beliebige Karten aus deiner Hand verdeckt zurück auf den Nachziehstapel. Du darfst auch die \emph{GEHEIMKAMMER} selbst auf den Nachziehstapel legen, da sich die Karte auch nach dem Vorzeigen auf deiner Hand befindet. Du darfst pro Angriff so viele Reaktionskarten vorzeigen, wie du möchtest. So kannst du zum Beispiel zuerst die \emph{GEHEIMKAMMER} abwickeln und danach einen \emph{BURGGRABEN} vorzeigen, um den Angriff abzuwehren. Die \emph{GEHEIMKAMMER} selbst wehrt einen Angriff nicht ab.}
\end{tikzpicture}
\hspace{-0.6cm}
\begin{tikzpicture}
	\card
	\cardstrip
	\cardbanner{banner/white.png}
	\cardicon{icons/coin.png}
	\cardprice{2}
	\cardtitle{Handlanger}
	\cardcontent{Du darfst von den vier Anweisungen der Karte \emph{genau 2} auswählen und diese für deinen Zug nutzen. Du musst zwei verschiedene Anweisungen wählen und darfst nicht z. B. zwei Karten ziehen oder 2 zusätzliche Käufe tätigen. Du musst dich sofort entscheiden, welche zwei Anweisungen du nutzen möchtest. Du darfst nicht eine Karte nachziehen und dich erst dann entscheiden, welche zweite Anweisung du ausführen möchtest.}
\end{tikzpicture}
\hspace{-0.6cm}
\begin{tikzpicture}
	\card
	\cardstrip
	\cardbanner{banner/white.png}
	\cardicon{icons/coin.png}
	\cardprice{3}
	\cardtitle{Armenviertel}
	\cardcontent{Du darfst in deiner Aktionsphase zwei zusätzliche Aktionen ausführen. Zeige deine Kartenhand vor. Wenn du keine Aktionskarte (oder Aktions-/Punktekarte) auf der Hand hast, ziehst du zwei Karten nach. Sollten sich darunter Aktionskarten befinden, darfst du diese auch gleich nutzen.}
\end{tikzpicture}
\hspace{-0.6cm}
\begin{tikzpicture}
	\card
	\cardstrip
	\cardbanner{banner/whitegreen.png}
	\cardicon{icons/coin.png}
	\cardprice{3}
	\cardtitle{Große Halle}
	\cardcontent{Diese Karte ist eine kombinierte Aktions- und Punktekarte. Sie kann in der Aktionsphase eingesetzt werden und bringt \emph{zusätzlich} bei Spielende 1 Siegpunkt. Wenn du diese Karte ausspielst, ziehst du eine Karte und darfst eine zusätzliche Aktion ausspielen.}
\end{tikzpicture}
\hspace{-0.6cm}
\begin{tikzpicture}
	\card
	\cardstrip
	\cardbanner{banner/white.png}
	\cardicon{icons/coin.png}
	\cardprice{3}
	\cardtitle{Maskerade}
	\cardcontent{Ziehe zwei Karten. Anschließend wählen alle Spieler (einschließlich dir selbst) eine beliebige Karte aus ihrer Hand und legen sie verdeckt neben ihrem linken Nachbarn ab. Wenn alle Karten verteilt sind, nimmt jeder Spieler die erhaltene Karte auf die Hand. Da die \emph{MASKERADE} keine Angriffskarte ist, dürfen die anderen Spieler keine Reaktionskarten vorzeigen. Danach darfst du eine Karte aus deiner Hand entsorgen.}
\end{tikzpicture}
\hspace{-0.6cm}
\begin{tikzpicture}
	\card
	\cardstrip
	\cardbanner{banner/white.png}
	\cardicon{icons/coin.png}
	\cardprice{3}
	\cardtitle{Trickser}
	\cardcontent{Alle Mitspieler müssen die oberste Karte ihres Nachziehstapels aufdecken und diese entsorgen. Du wählst für jeden Mitspieler jeweils eine Karte aus dem Vorrat, die genauso viel kostet, wie die entsorgte Karte und gibst sie dem Mitspieler. Dieser legt die neue Karte auf seinen Ablagestapel. Befindet sich keine Karte mit den gleichen Kosten im Vorrat, muss der Mitspieler seine Karte trotzdem entsorgen, erhält dafür aber keine neue Karte. Du darfst dem Mitspieler auch eine gleiche Karte wie die, die er entsorgt hat, zurückgeben. }
\end{tikzpicture}
\hspace{-0.6cm}
\begin{tikzpicture}
	\card
	\cardstrip
	\cardbanner{banner/white.png}
	\cardicon{icons/coin.png}
	\cardprice{3}
	\cardtitle{Verwalter}
	\cardcontent{Du wählst von den drei Anweisungen der Karte \emph{genau 1} aus und führst diese dann komplett aus. Wenn du dich entscheidest, zwei Karten zu entsorgen, du aber nur eine Karte auf der Hand hast, musst du diese entsorgen.}
\end{tikzpicture}
\hspace{-0.6cm}
\begin{tikzpicture}
	\card
	\cardstrip
	\cardbanner{banner/white.png}
	\cardicon{icons/coin.png}
	\cardprice{3}
	\cardtitle{\scriptsize{Wunschbrunnen}}
	\cardcontent{Zuerst ziehst du eine Karte. Du darfst in der Aktionsphase eine zusätzliche Aktion ausführen. Nenne eine Karte (z.B. \emph{KUPFER}) und decke die oberste Karte deines Nachziehstapels auf. Handelt es sich dabei um die von dir genannte Karte, nimmst du sie auf die Hand. Ansonsten legst du sie zurück auf den Nachziehstapel.}
\end{tikzpicture}
\hspace{-0.6cm}
\begin{tikzpicture}
	\card
	\cardstrip
	\cardbanner{banner/white.png}
	\cardicon{icons/coin.png}
	\cardprice{4}
	\cardtitle{Baron}
	\cardcontent{Du \emph{darfst} ein Anwesen (sofern du gerade eins auf der Hand hast) ablegen und erhältst dafür +\coin[4] für die Kaufphase. Wenn du das nicht tun kannst (weil du kein Anwesen auf der Hand hast) oder willst, musst du dir ein Anwesen nehmen, solange noch welche im Vorrat sind.}
\end{tikzpicture}
\hspace{-0.6cm}
\begin{tikzpicture}
	\card
	\cardstrip
	\cardbanner{banner/white.png}
	\cardicon{icons/coin.png}
	\cardprice{4}
	\cardtitle{Bergwerk}
	\cardcontent{Du ziehst zuerst eine Karte nach und \emph{darfst} dann diese Karte entsorgen, bevor du ggf. weitere Aktionen ausspielst. Du erhältst dafür +\coin[2]. Wenn du das \emph{BERGWERK} auf einen \emph{THRONSAAL} spielst, erhältst du den Bonus für das Entsorgen nur einmal, da du die Karte nur einmal entsorgen kannst. Die anderen Anweisungen (+ 1 Karte sowie + 2 Aktionen) werden durch den \emph{THRONSAAL} dagegen verdoppelt.}
\end{tikzpicture}
\hspace{-0.6cm}
\begin{tikzpicture}
	\card
	\cardstrip
	\cardbanner{banner/white.png}
	\cardicon{icons/coin.png}
	\cardprice{4}
	\cardtitle{Brücke}
	\cardcontent{Die Kosten aller Karten (auch Handkarten, Karten aus den Nachzieh- und Ablagestapeln) werden in diesem Spielzug für alle Belange um \coin[1] reduziert (nicht aber unter \coin[0]). Dieser Effekt ist kumulativ, d. h. die Kosten pro Karte können durch das Ausspielen bestimmter Karten (z. B. den \emph{THRONSAAL}) auch um \coin[2] oder mehr reduziert werden.}
\end{tikzpicture}
\hspace{-0.6cm}
\begin{tikzpicture}
	\card
	\cardstrip
	\cardbanner{banner/white.png}
	\cardicon{icons/coin.png}
	\cardprice{4}
	\cardtitle{Eisenhütte}
	\cardcontent{Du nimmst dir eine beliebige Karte vom Vorrat, die maximal \coin[4] kostet. Durch das Ausspielen bestimmter Aktionskarten (z. B. die \emph{BRÜCKE}) können die Kosten der Karten reduziert werden.

	\medskip

	Je nachdem, ob du dich für eine Aktions-, Geld- oder Punktekarte entschieden hast, erhältst du einen anderen Bonus. Solltest du dich für eine kombinierte Karte entscheiden, erhältst du die Boni beider Kartentypen.}
\end{tikzpicture}
\hspace{-0.6cm}
\begin{tikzpicture}
	\card
	\cardstrip
	\cardbanner{banner/white.png}
	\cardicon{icons/coin.png}
	\cardprice{4}
	\cardtitle{\footnotesize{Kupferschmied}}
	\cardcontent{Mit dieser Karte erhöhst du den Wert aller in diesem Zug gespielten \emph{KUPFER} um +\coin[1]. Der Effekt ist kumulativ, d. h. durch das Ausspielen anderer Aktionskarten (z. B. den \emph{THRONSAAL} oder einen weiteren \emph{KUPFERSCHMIED}) kann der Wert weiter erhöht werden.}
\end{tikzpicture}
\hspace{-0.6cm}
\begin{tikzpicture}
	\card
	\cardstrip
	\cardbanner{banner/white.png}
	\cardicon{icons/coin.png}
	\cardprice{4}
	\cardtitle{Späher}
	\cardcontent{Sollten für das Aufdecken der vier Karten nicht genügend Karten im Nachziehstapel sein, ziehst du zunächst die restlichen Karten und mischst dann deinen Ablagestapel neu, ohne die bereits aufgedeckten Karten mit einzumischen. Sollten auch dann nicht genügend Karten zur Verfügung stehen, ziehst du nur so viele Karten wie möglich. Du musst alle Punktekarten auf die Hand nehmen, die restlichen Karten legst du in beliebiger Reihenfolge auf den Nachziehstapel. Diese musst du deinen Mitspielern nicht zeigen. Kombinierte Aktions-/Punktekarten sind auch Punktekarten. }
\end{tikzpicture}
\hspace{-0.6cm}
\begin{tikzpicture}
	\card
	\cardstrip
	\cardbanner{banner/white.png}
	\cardicon{icons/coin.png}
	\cardprice{4}
	\cardtitle{Verschwörer}
	\cardcontent{Wenn du zu dem Zeitpunkt an dem du den \emph{VERSCHWÖRER} spielst, bereits mindestens 3 Aktionskarten (inklusive diesem \emph{VERSCHWÖRER}) ausgespielt hast, erhältst du den Bonus. Wenn du erst im weiteren Verlauf deiner Aktionsphase diese Bedingung erfüllst, erhältst du den Bonus nicht. Aktionskarten, die z. B. durch den \emph{THRONSAAL} doppelt ausgespielt werden dürfen, gelten als 2 ausgespielte Karten.}
\end{tikzpicture}
\hspace{-0.6cm}
\begin{tikzpicture}
	\card
	\cardstrip
	\cardbanner{banner/white.png}
	\cardicon{icons/coin.png}
	\cardprice{5}
	\cardtitle{Anbau}
	\cardcontent{Nachdem du dir eine Karte genommen und eine zusätzliche Aktion erhalten hast, musst du eine Karte aus deiner Hand entsorgen sofern du noch Handkarten hast. Du nimmst dir dafür eine Karte vom Vorrat, die \emph{genau} \coin[1] mehr kostet als die entsorgte Karte. Wenn keine solche Karte vorhanden ist, musst du zwar eine Karte entsorgen, erhältst aber keine Karte vom Vorrat. }
\end{tikzpicture}
\hspace{-0.6cm}
\begin{tikzpicture}
	\card
	\cardstrip
	\cardbanner{banner/white.png}
	\cardicon{icons/coin.png}
	\cardprice{5}
	\cardtitle{\footnotesize{Handelsposten}}
	\cardcontent{Wenn du nur eine Karte auf der Hand hast, musst du sie entsorgen, erhältst dafür aber kein Silber. Wenn du zwei oder mehr Karten auf der Hand hast, musst du genau zwei Karten entsorgen und nimmst dir dafür ein Silber direkt auf die Hand. Sollte kein Silber mehr im Vorrat sein, musst du die Karten trotzdem entsorgen, erhältst aber kein Silber. }
\end{tikzpicture}
\hspace{-0.6cm}
\begin{tikzpicture}
	\card
	\cardstrip
	\cardbanner{banner/green.png}
	\cardicon{icons/coin.png}
	\cardprice{5}
	\cardtitle{Herzog}
	\cardcontent{Diese Karte ist die einzige reine Punktekarte unter den Königreichkarten. Sie hat bis zum Ende des Spiels keine Funktion. Bei Spielende erhält der Spieler, der diese Karte in seinem Kartensatz (Nachziehstapel, Handkarten und Ablagestapel) hat, für jedes \emph{HERZOGTUM} im Kartensatz 1 Siegpunkt. Wer mehrere \emph{HERZÖGE} besitzt, erhält für jeden \emph{HERZOG} die entsprechende Anzahl Siegpunkte.} 
\end{tikzpicture}
\hspace{-0.6cm}
\begin{tikzpicture}
	\card
	\cardstrip
	\cardbanner{banner/white.png}
	\cardicon{icons/coin.png}
	\cardprice{5}
	\cardtitle{\footnotesize{Kerkermeister}}
	\cardcontent{Jeder Mitspieler (beginnend bei deinem linken Nachbarn) muss entweder zwei Karten ablegen oder einen \emph{FLUCH} vom Stapel nehmen. Ein Spieler kann sich entscheiden, die Karten abzulegen, auch wenn er nur eine oder gar keine Karte auf der Hand hat. Er legt dann nur so viele Karten ab, wie er kann. Er kann sich auch entscheiden, einen \emph{FLUCH} zu nehmen, wenn es keine \emph{FLÜCHE} mehr im Vorrat gibt.}
\end{tikzpicture}
\hspace{-0.6cm}
\begin{tikzpicture}
	\card
	\cardstrip
	\cardbanner{banner/white.png}
	\cardicon{icons/coin.png}
	\cardprice{5}
	\cardtitle{Lakai}
	\cardcontent{Du entscheidest dich für eine der beiden Optionen: Entweder erhältst du +\coin[2] in diesem Zug oder du legst alle deine Handkarten ab und ziehst vier neue Karten nach. Wenn du die zweite Option wählst, müssen außerdem alle Mitspieler, die fünf oder mehr Karten auf der Hand haben (alle anderen sind nicht betroffen), diese ablegen und ebenfalls vier Karten nachziehen. Jeder Spieler (auch wenn er von dem Angriff nicht betroffen ist) kann eine oder mehrere Reaktionskarten vorzeigen, wenn du den \emph{LAKAIEN} spielst.}
\end{tikzpicture}
\hspace{-0.6cm}
\begin{tikzpicture}
	\card
	\cardstrip
	\cardbanner{banner/white.png}
	\cardicon{icons/coin.png}
	\cardprice{5}
	\cardtitle{Saboteur}
	\cardcontent{Jeder Mitspieler (beginnend bei deinem linken Nachbarn) muss solange Karten von seinem Nachziehstapel aufdecken, bis er eine Karte aufdeckt, die mindestens \emph{3} kostet (\emph{Achtung:} Die Kosten einer Karte können durch die \emph{BRÜCKE} verringert werden). Er muss diese Karte sofort entsorgen und darf sich dafür eine Karte aus dem Vorrat nehmen, die mindestens \emph{2} weniger kostet. Die restlichen aufgedeckten Karten werden abgelegt. Sollte im gesamten restlichen Nachziehstapel keine Karte mit passendem Wert vorhanden sein, wird der Ablagestapel ohne die bereits aufgedeckten Karten gemischt und zum neuen Nachziehstapel. Sollte dann immer noch keine passende Karte zu finden sein, legt der Spieler alle Karten ab und nichts passiert.}
\end{tikzpicture}
\hspace{-0.6cm}
\begin{tikzpicture}
	\card
	\cardstrip
	\cardbanner{banner/white.png}
	\cardicon{icons/coin.png}
	\cardprice{5}
	\cardtitle{Tribut}
	\cardcontent{Dein linker Mitspieler muss die obersten beiden Karten seines Nachziehstapels aufdecken und ablegen. Da der \emph{TRIBUT} keine Angriffskarte ist, kann sich der Mitspieler nicht gegen diese Anweisung wehren. Der Spieler, der ein \emph{TRIBUT} ausgespielt hat, erhält für die erste Karte den genannten Bonus. Für die zweite Karte erhält der Spieler nur dann ein weiteres Mal den Bonus, wenn nicht die gleiche Karte wie zuvor aufgedeckt wurde. Kombinierte Karten bringen dem Spieler auch doppelte Boni.}
\end{tikzpicture}
\hspace{-0.6cm}
\begin{tikzpicture}
	\card
	\cardstrip
	\cardbanner{banner/whitegreen.png}
	\cardicon{icons/coin.png}
	\cardprice{6}
	\cardtitle{Adelige}
	\cardcontent{Diese Karte ist eine kombinierte Aktions- und Punktekarte. Sie kann in der Aktionsphase eingesetzt werden und bringt außerdem bei Spielende 2 Siegpunkte. Wenn du diese Karte ausspielst, musst du dich entscheiden, ob du entweder 3 Karten ziehst oder  2 weitere Aktionen ausspielen willst. Du darfst die Anweisungen aber nicht mischen oder aufteilen. Wenn du die Karte das nächste Mal auf der Hand hast und ausspielst, darfst du natürlich eine andere Wahl treffen.}
\end{tikzpicture}
\hspace{-0.6cm}
\begin{tikzpicture}
	\card
	\cardstrip
	\cardbanner{banner/goldgreen.png}
	\cardicon{icons/coin.png}
	\cardprice{6}
	\cardtitle{Harem}
	\cardcontent{Diese Karte ist eine kombinierte Geld- und Punktekarte. Sie wird während des Zugs wie eine normale Geldkarte eingesetzt und bringt außerdem bei Spielende 2 Siegpunkte.}
\end{tikzpicture}
\hspace{-0.6cm}
\begin{tikzpicture}
	\card
	\cardstrip
	\cardbanner{banner/white.png}
	\cardtitle{\scriptsize{Empfohlene 10er Sätze\qquad}}
	\cardcontent{\emph{Siegestanz:}\\
	Adlige, Anbau, Brücke, Eisenhütte, Große Halle, Handlanger, Harem, Herzog, Maskerade, Späher 

	\smallskip

	\emph{Geheime Pläne:}\\
	Armenviertel, Eisenhütte, Handelsposten, Handlanger, Harem, Saboteur, Tribut, Trickser, Verschwörer, Verwalter

	\smallskip

	\emph{Beste Wünsche:}\\
	Anbau, Armenviertel, Burghof, Handelsposten, Kerkermeister, Kupferschmied, Maskerade, Späher, Verwalter, Wunschbrunnen

	\smallskip

	\emph{Demontage} (Intrige + \textit{Basisspiel}):\\
	Bergwerk, Brücke, Geheimkammer, Kerkermeister, Trickser, Saboteur, \textit{Dieb}, \textit{Spion}, \textit{Thronsaal}, \textit{Umbau}

	\smallskip

	\emph{Eine Hand voll} (Intrige + \textit{Basisspiel}):\\
	Adlige, Burghof, Kerkermeister, Lakai, Verwalter, \textit{Bürokrat}, \textit{Kanzler}, \textit{Miliz}, \textit{Mine}, \textit{Ratsversammlung}

	\smallskip

	\emph{Untergebene} (Intrige + \textit{Basisspiel}):\\
	Adlige, Baron, Lakai, Maskerade, Verwalter, Handlanger, \textit{Bibliothek}, \textit{Hexe}, \textit{Jahrmarkt}, \textit{Keller}}
\end{tikzpicture}
\hspace{-0.6cm}

	    % Basic settings for this card set
\renewcommand{\cardcolor}{seaside}
\renewcommand{\cardextension}{Erweiterung I}
\renewcommand{\cardextensiontitle}{Seaside}
\renewcommand{\seticon}{seaside.png}

\clearpage
\newpage
\section{\cardextension \ - \cardextensiontitle \ (Hans Im Glück 2009)}

\begin{tikzpicture}
	\card
	\cardstrip
	\cardbanner{banner/white.png}
	\cardicon{icons/coin.png}
	\cardprice{2}
	\cardtitle{\tiny{Eingeborenendorf}}
	\cardcontent{Sobald du dein erstes Eingeborenendorf nimmst oder kaufst, nimmst du auch ein Tableau Eingeborenendorf zu dir. Wenn du das Eingeborenendorf ausspielst, musst du eine der beiden folgenden Möglichkeiten wählen: Entweder du siehst die oberste Karte von deinem Nachziehstapel an und legst sie dann verdeckt zur Seite auf das Tableau Eingeborenendorf. Oder du nimmst alle zur Seite gelegten Karten vom Tableau Eingeborenendorf auf die Hand. Du kannst jede der beiden Möglichkeiten wählen, auch wenn du sie nicht ausführen kannst. Du darfst die zur Seite gelegten Karten jederzeit ansehen. Das ausgespielte Eingeborenendorf selbst legst du nicht zur Seite, sondern legst es in der Aufräumphase ab. Bei Spielende nimmst du alle Karten, die noch auf dem Tableau Eingeborenendorf liegen, zu deinen übrigen Karten.}
\end{tikzpicture}
\hspace{-0.6cm}
\begin{tikzpicture}
	\card
	\cardstrip
	\cardbanner{banner/white.png}
	\cardicon{icons/coin.png}
	\cardprice{2}
	\cardtitle{Embargo}
	\cardcontent{Du kannst jeden Stapel des Vorrats wählen. Wenn mehrere Embargo-Marker auf einem Stapel liegen, musst du dir für jeden Marker eine Fluchkarte nehmen, wenn du eine Karte von diesem Stapel kaufst. Du nimmst dir nur Flüche, wenn du dir eine Karte von diesem Stapel kaufst. Wenn du dir auf eine andere Weise eine Karte von diesem Stapel nimmst (z. B. durch die Schmuggler), nimmst du dir keinen Fluch. Wenn du das Embargo auf einen Thronsaal spielst, legst du 2 Embargo-Marker. Die beiden Embargo-Marker darfst du auch zu unterschiedlichen Stapeln legen. Sollten keine Embargo-Marker mehr vorhanden sein, benutze bitte einen adäquaten Ersatz. Wenn keine Fluchkarten mehr im Vorrat sind, haben die Embargo-Marker keinen Effekt.}
\end{tikzpicture}
\hspace{-0.6cm}
\begin{tikzpicture}
	\card
	\cardstrip
	\cardbanner{banner/orange.png}
	\cardicon{icons/coin.png}
	\cardprice{2}
	\cardtitle{Hafen}
	\cardcontent{ Du ziehst zuerst eine Karte nach, dann wählst du eine Karte aus deiner Hand und legst diese verdeckt zur Seite auf den ausgespielten Hafen. Du musst deinen Mitspielern die zur Seite gelegte Karte nicht zeigen. Du musst eine Karte aus deiner Hand zur Seite legen, wenn du den Hafen ausspielst. Der Hafen und die zur Seite gelegte Karte bleiben bis zum Beginn deines nächsten Zuges liegen. Bei Beginn deines nächsten Zuges nimmst du die zur Seite gelegte Karte auf die Hand. Den Hafen legst du in der Aufräumphase dieses nächsten Zuges ab.}
\end{tikzpicture}
\hspace{-0.6cm}
\begin{tikzpicture}
	\card
	\cardstrip
	\cardbanner{banner/orange.png}
	\cardicon{icons/coin.png}
	\cardprice{2}
	\cardtitle{Leuchtturm}
	\cardcontent{Du erhältst +1 Aktion und +1 Geld in diesem Zug. Bei Beginn deines nächsten Zuges erhältst du nochmals +1 Geld. Solange der Leuchtturm o en vor dir ausliegt, bist du von Angriffskarten deiner Mitspieler nicht betroffen, selbst wenn du das möchtest. Du kannst trotz ausliegendem Leuchtturm auch mit der Geheimkammer oder anderen Reaktionskarten auf Angriffe reagieren, um deren Vorteil zu erhalten. Angriffskarten, die du selbst ausspielst, werden durch den Leuchtturm nicht abgewehrt. Der Leuchtturm bleibt bis zur Aufräumphase deines nächsten Zuges im Spiel.}
\end{tikzpicture}
\hspace{-0.6cm}
\begin{tikzpicture}
	\card
	\cardstrip
	\cardbanner{banner/white.png}
	\cardicon{icons/coin.png}
	\cardprice{2}
	\cardtitle{\footnotesize{Perlentaucher}}
	\cardcontent{Zu ziehst zuerst eine Karte nach, bevor du dir die unterste Karte deines Nachziehstapels ansiehst. Nimm die unterste Karte so vom Stapel, dass du nicht auch die nächste Karte sehen kannst.}
\end{tikzpicture}
\hspace{-0.6cm}
\begin{tikzpicture}
	\card
	\cardstrip
	\cardbanner{banner/white.png}
	\cardicon{icons/coin.png}
	\cardprice{3}
	\cardtitle{Ausguck}
	\cardcontent{Sieh dir alle 3 Karten an, bevor du entscheidest, welche Karte du entsorgst, welche du ablegst und welche du zurück auf den Nachziehstapel legst. Wenn du weniger als 3 Karten im Nachziehstapel hast (auch nach dem eventuell nötigen Mischen deines Ablagestapels), führst du die Anweisungen in der Reihenfolge auf der Karte aus, d. h. zuerst entsorgst du eine Karte, dann legst du eine Karte ab.}
\end{tikzpicture}
\hspace{-0.6cm}
\begin{tikzpicture}
	\card
	\cardstrip
	\cardbanner{banner/white.png}
	\cardicon{icons/coin.png}
	\cardprice{3}
	\cardtitle{Botschafter}
	\cardcontent{Du wählst 1 Karte aus deiner Hand, zeigst sie deinen Mitspielern und nimmst sie auf die Hand zurück. Dann darfst du bis zu 2 dieser Karten aus deiner Hand zurück in den Vorrat legen. Du darfst jedoch auch entscheiden, keine Karte zurück in den Vorrat zu legen. Danach muss sich jeder Mitspieler, beginnend mit dem Spieler
	zu deiner Linken, eine solche Karte aus dem Vorrat nehmen. Wenn der Stapel leer ist, werden keine Karten mehr genommen. Wenn du nach dem Ausspielen des Botschafters keine Karte mehr auf der Hand hast, legst du nichts zurück in den Vorrat und die Mitspieler nehmen auch keine Karten. Wenn zu irgendeinem Zeitpunkt während des Zuges entweder der dritte Stapel oder der Provinzstapel leer wird, endet das Spiel nach diesem Zug, auch wenn der Stapel später wieder aufgefüllt wird. 

	\smallskip

	Der Botschafter darf keine Unterschlupf-Karten (Dominion – Dark Ages) zurücklegen.}
\end{tikzpicture}
\hspace{-0.6cm}
\begin{tikzpicture}
	\card
	\cardstrip
	\cardbanner{banner/orange.png}
	\cardicon{icons/coin.png}
	\cardprice{3}
	\cardtitle{Fischerdorf}
	\cardcontent{Du erhältst +2 Aktionen und +1 Geld in diesem Zug. Bei Beginn deines nächsten Zuges erhältst du nochmals +1 Aktion und +1 Geld. Das Fischerdorf bleibt bis zur Aufräumphase deines nächsten Zuges im Spiel.}
\end{tikzpicture}
\hspace{-0.6cm}
\begin{tikzpicture}
	\card
	\cardstrip
	\cardbanner{banner/white.png}
	\cardicon{icons/coin.png}
	\cardprice{3}
	\cardtitle{Lagerhaus}
	\cardcontent{Wenn du (auch nach dem eventuell nötigen Mischen deines Ablagestapels) nicht mehr genügend Karten nachziehen kannst, ziehst du so viele Karten wie möglich. Du musst dann 3 Karten ablegen, auch wenn du weniger Karten nachgezogen hast. Wenn du weniger als 3 Karten auf der Hand hast, legst du alle Handkarten ab.}
\end{tikzpicture}
\hspace{-0.6cm}
\begin{tikzpicture}
	\card
	\cardstrip
	\cardbanner{banner/white.png}
	\cardicon{icons/coin.png}
	\cardprice{3}
	\cardtitle{Schmuggler}
	\cardcontent{Es ist egal, auf welche Weise der Spieler rechts von dir die Karte genommen oder gekauft hat (z. B. ebenfalls durch die Schmuggler). Wenn der Spieler rechts von dir mehrere Karten genommen bzw. gekauft hat, darfst du wählen, welche Karte du dir nimmst. Gibt es keine entsprechende Karte mehr im Vorrat, erhältst du nichts. Die Schmuggler sind kein Angriff, sie können nicht durch den Leuchtturm oder den Burggraben gestoppt werden. Diese Karte bezieht sich nie auf deinen eigenen vorherigen Zug. Wenn der Spieler rechts von dir im vorherigen Zug keine Karte genommen oder gekauft hat, die 6 Geld oder weniger gekostet hat, haben die Schmuggler keine Wirkung.}
\end{tikzpicture}
\hspace{-0.6cm}
\begin{tikzpicture}
	\card
	\cardstrip
	\cardbanner{banner/white.png}
	\cardicon{icons/coin.png}
	\cardprice{4}
	\cardtitle{\scriptsize{Beutelschneider}}
	\cardcontent{Deine Mitspieler müssen 1 Kupferkarte aus ihrer Hand ablegen. Hat ein Mitspieler kein Kupfer auf der Hand, muss er seine Kartenhand vorzeigen.}
\end{tikzpicture}
\hspace{-0.6cm}
\begin{tikzpicture}
	\card
	\cardstrip
	\cardbanner{banner/whitegreen.png}
	\cardicon{icons/coin.png}
	\cardprice{4}
	\cardtitle{Insel}
	\cardcontent{Die Insel ist zugleich eine Aktions- und eine Punktekarte. Sobald du deine erste Insel nimmst oder kaufst, nimmst du auch ein Tableau Insel zu dir. Wenn du die Karte Insel ausspielst, legst du die ausgespielte Insel und eine weitere Karte aus deiner Hand o en zur Seite auf das Tableau Insel, wo sie bis zum Spielende verbleiben. Wenn du keine weitere Karte auf der Hand hast, nachdem du die Insel ausgespielt hast, legst du nur die Insel zur Seite. Wenn du die Insel auf einen Thronsaal ausspielst, legst du zunächst die Insel und eine weitere Karte zur Seite, dann legst du noch eine Karte aus deiner Hand zur Seite auf das Tableau Insel. Du und auch deine Mitspieler dürfen die zur Seite gelegten Karten auf dem Tableau Insel jederzeit ansehen. Bei Spielende nimmst du alle Karten, die auf dem Tableau Insel liegen, zu deinen Karten. Bei Spielende ist die Insel 2 Punkte wert. Im Spiel mit 3 oder 4 Spielern werden 12 Inseln verwendet, im Spiel mit 2 Spielern nur 8.}
\end{tikzpicture}
\hspace{-0.6cm}
\begin{tikzpicture}
	\card
	\cardstrip
	\cardbanner{banner/orange.png}
	\cardicon{icons/coin.png}
	\cardprice{4}
	\cardtitle{Karawane}
	\cardcontent{Ziehe eine Karte bei Beginn deines nächsten Zuges. Die Karawane bleibt bis zur Aufräumphase dieses nächsten Zuges im Spiel.}
\end{tikzpicture}
\hspace{-0.6cm}
\begin{tikzpicture}
	\card
	\cardstrip
	\cardbanner{banner/white.png}
	\cardicon{icons/coin.png}
	\cardprice{4}
	\cardtitle{\footnotesize{Müllverwerter}}
	\cardcontent{Du musst eine Karte entsorgen, wenn du noch eine auf der Hand hast. Wenn du keine Karte auf der Hand hast, erhältst du nur den +1 Kauf, jedoch kein virtuelles Geld.}
\end{tikzpicture}
\hspace{-0.6cm}
\begin{tikzpicture}
	\card
	\cardstrip
	\cardbanner{banner/white.png}
	\cardicon{icons/coin.png}
	\cardprice{4}
	\cardtitle{Navigator}
	\cardcontent{Du legst entweder alle 5 Karten ab oder keine davon. Wenn du die Karten nicht ablegst, darfst du sie in beliebiger Reihenfolge zurück auf deinen Nachziehstapel legen. Wenn du (auch nach dem eventuell nötigen Mischen deines Ablagestapels) nicht mehr genügend Karten ansehen kannst, siehst du so viele Karten wie möglich an und entscheidest dann, ob du sie ablegst oder zurück auf deinen Nachziehstapel legst.}
\end{tikzpicture}
\hspace{-0.6cm}
\begin{tikzpicture}
	\card
	\cardstrip
	\cardbanner{banner/white.png}
	\cardicon{icons/coin.png}
	\cardprice{4}
	\cardtitle{\footnotesize{Piratenschiff}}
	\cardcontent{Sobald du dein erstes Piratenschiff nimmst oder kaufst, nimmst du auch ein Tableau Piratenschiff zu dir. Wenn du das Piratenschiff ausspielst, entscheidest du dich zunächst für eine der beiden Anweisungen. Entweder wählst du die Möglichkeit, Geldkarten deiner Mitspieler zu entsorgen oder +X Geld . Wählst du die erste Möglichkeit, muss jeder Mitspieler höchstens eine Geldkarte entsorgen. Hat ein Mit- spieler mehrere Geldkarten aufgedeckt, wählst du aus, welche davon er entsorgen muss. Hat mindestens ein Mitspieler eine Geldkarte entsorgt, erhältst du einen Geld-Marker, den du auf dein Tableau Piratenschiff legst. Du erhältst höchstens einen Geld-Marker, auch wenn mehrere Mitspieler Geldkarten entsorgen mussten. Wählst du die zweite Möglichkeit, erhältst du +1 Geld für jeden Geld-Marker auf deinem Tableau Piratenschiff. Die Geld-Marker bleiben bis zum Spielende auf dem Tableau. Wenn du das Piratenschiff z. B. 3mal (erfolgreich) verwendet hast, um Geldkarten deiner Mitspieler zu entsorgen, hast du 3 Geld-Marker auf deinem Tableau Piratenschiff. Damit kannst du jedesmal, wenn du das Piratenschiff ausspielst +3 Geld erhalten. Das Piratenschiff ist eine Angriffskarte, deine Mitspieler können also z. B. mit der Geheimkammer reagieren, auch wenn du +X Geld wählst.}
\end{tikzpicture}
\hspace{-0.6cm}
\begin{tikzpicture}
	\card
	\cardstrip
	\cardbanner{banner/white.png}
	\cardicon{icons/coin.png}
	\cardprice{4}
	\cardtitle{Schatzkarte}
	\cardcontent{Du kannst diese Karte ausspielen, auch wenn du keine weitere Schatzkarte auf der Hand hast. In diesem Fall entsorgst du die ausgespielte Schatzkarte, erhältst jedoch nichts dafür. Du musst wirklich 2 Schatzkarten entsorgen, um dir die 4 Gold zu nehmen. Wenn du z. B. eine Schatzkarte auf einen Thronsaal ausspielst und noch 2 weitere Schatzkarten auf der Hand hast, entsorgst du zuerst die ausgespielte Schatzkarte und eine weitere Schatzkarte aus deiner Hand. Beim zweiten Ausspielen entsorgst du die andere Schatzkarte aus deiner Hand, erhältst jedoch nichts, da du nur eine Schatzkarte entsorgt hast. Die ausgespielte Schatzkarte hast du bereits beim ersten Ausspielen entsorgt. Wenn weniger als 4 Gold im Vorrat sind, nimmst du dir die übrigen Gold. Die Karten legst du verdeckt auf deinen Nachziehstapel.}
\end{tikzpicture}
\hspace{-0.6cm}
\begin{tikzpicture}
	\card
	\cardstrip
	\cardbanner{banner/white.png}
	\cardicon{icons/coin.png}
	\cardprice{4}
	\cardtitle{Seehexe}
	\cardcontent{Ein Spieler, dessen Nachziehstapel leer ist, mischt zuerst seinen Ablagestapel und legt dann die oberste Karte von seinem neuen Nachziehstapel ab. Danach müssen alle Mitspieler, beginnend bei dem Spieler links vom Angreifer, einen Fluch aus dem Vorrat nehmen und diesen verdeckt auf ihren Nachziehstapel legen. Gibt es nicht mehr genügend Fluchkarten im Vorrat, werden die übrigen in Spielerreihenfolge verteilt.}
\end{tikzpicture}
\hspace{-0.6cm}
\begin{tikzpicture}
	\card
	\cardstrip
	\cardbanner{banner/orange.png}
	\cardicon{icons/coin.png}
	\cardprice{5}
	\cardtitle{\footnotesize{Aussenposten}}
	\cardcontent{Der Extrazug ist wie ein normaler Zug, mit dem Unterschied, dass du diesen Zug
	nur mit 3 Karten ausführst. Du ziehst also in der Aufräumphase des Zuges, in
	dem du den Aussenposten ausspielst nur 3, statt den üblichen 5 Karten. Lass den Aussenposten offen vor dir liegen, bis zum Ende des Extrazuges. Wenn du den Aussenposten zusammen mit einer \enquote{zu Beginn deines nächsten Zuges}-Karte ausspielst, wie z. B. das Handelsschiff, ist der Extrazug vom Handelsschiff dieser nächste Zug, in dem du +2 Geld bekommst. Wenn du den Aussenposten mehr als einmal ausspielst, bekommst du trotzdem nur einen Extrazug. Wenn du während des Extrazuges einen Aussenposten ausspielst, bekommst du keinen weiteren Extrazug. Am Ende deines Extrazuges ziehst du wieder 5 Karten nach.}
\end{tikzpicture}
\hspace{-0.6cm}
\begin{tikzpicture}
	\card
	\cardstrip
	\cardbanner{banner/white.png}
	\cardicon{icons/coin.png}
	\cardprice{5}
	\cardtitle{Bazar}
	\cardcontent{Du ziehst eine Karte nach, darfst 2 weitere Aktionen ausführen und erhältst für die Kaufphase + 1 Geld.}
\end{tikzpicture}
\hspace{-0.6cm}
\begin{tikzpicture}
	\card
	\cardstrip
	\cardbanner{banner/white.png}
	\cardicon{icons/coin.png}
	\cardprice{5}
	\cardtitle{Entdecker}
	\cardcontent{Wenn du eine Provinz aus deiner Hand aufdeckst, nimmst du eine Goldkarte direkt auf die Hand. Wenn du keine Provinz auf der Hand hast oder die Provinz nicht aufdecken möchtest, nimmst du dir ein Silber direkt auf die Hand.}
\end{tikzpicture}
\hspace{-0.6cm}
\begin{tikzpicture}
	\card
	\cardstrip
	\cardbanner{banner/white.png}
	\cardicon{icons/coin.png}
	\cardprice{5}
	\cardtitle{\footnotesize{Geisterschiff}}
	\cardcontent{Deine Mitspieler entscheiden, welche Karten sie aus ihrer Hand zurück auf den Nachziehstapel legen und in welcher Reihenfolge sie die Karten zurücklegen. Spieler, die nur 3 oder weniger Karten auf der Hand halten, sind nicht betroffen.}
\end{tikzpicture}
\hspace{-0.6cm}
\begin{tikzpicture}
	\card
	\cardstrip
	\cardbanner{banner/orange.png}
	\cardicon{icons/coin.png}
	\cardprice{5}
	\cardtitle{\footnotesize{Handelsschiff}}
	\cardcontent{Du erhältst +2 Geld für die Kaufphase in diesem Zug und nochmals +2 Geld für die Kaufphase in deinem nächsten Zug. Das Handelsschiff bleibt bis zur Aufräumphase dieses nächsten Zuges im Spiel.}
\end{tikzpicture}
\hspace{-0.6cm}
\begin{tikzpicture}
	\card
	\cardstrip
	\cardbanner{banner/white.png}
	\cardicon{icons/coin.png}
	\cardprice{5}
	\cardtitle{\footnotesize{Schatzkammer}}
	\cardcontent{Wenn du eine oder mehrere Karten kaufst und mindestens eine davon eine Punktekarte ist, darfst du die Schatzkammer nicht zurück auf den Nachziehstapel legen. Hast du mehrere Schatzkammern ausgespielt und in diesem Zug keine Punktekarte gekauft, darfst du in der Aufräumphase beliebig viele deiner ausgespielten Schatzkammern zurück auf deinen Nachziehstapel legen. Wenn du vergisst, die Schatzkammer auf deinen Nachziehstapel zu legen, hast du dich dafür entschieden, die Karte abzulegen. Du darfst die Schatzkammer nicht nachträglich aus dem Ablagestapel heraussuchen, um sie noch zurück auf den Nachziehstapel zu legen. Wenn du eine Punktekarte nimmst, sie aber nicht gekauft hast (z. B. durch die Schmuggler), darfst du die Schatzkammer zurück auf deinen Nachziehstapel legen.}
\end{tikzpicture}
\hspace{-0.6cm}
\begin{tikzpicture}
	\card
	\cardstrip
	\cardbanner{banner/orange.png}
	\cardicon{icons/coin.png}
	\cardprice{5}
	\cardtitle{Taktiker}
	\cardcontent{Du ziehst die 5 Karten erst bei Beginn deines nächsten Zuges. Du ziehst die Karten nicht in der Aufräumphase des Zuges, in dem du den Taktiker ausgespielt hast. Der Taktiker bleibt bis zur Aufräumphase deines nächsten Zuges o en vor dir ausliegen. Wenn du den Taktiker nach einem Thronsaal ausspielst, erhältst du den Bonus nur einmal, da du auch beim 2. Mal mindestens eine Karte ablegen müsstest. Du kannst jedoch nach dem 1. Ausspielen keine Karten mehr auf der Hand haben.}
\end{tikzpicture}
\hspace{-0.6cm}
\begin{tikzpicture}
	\card
	\cardstrip
	\cardbanner{banner/orange.png}
	\cardicon{icons/coin.png}
	\cardprice{5}
	\cardtitle{Werft}
	\cardcontent{Du ziehst zuerst 2 Karten nach und erhältst einen weiteren Kauf für diesen Zug. Bei Beginn deines nächsten Zuges ziehst du 2 weitere Karten und erhältst nochmals einen zusätzlichen Kauf. Du ziehst die beiden Karten nicht, bevor dein nächster Zug begonnen hat. Die Werft bleibt bis zur Aufräumphase deines nächsten Zuges im Spiel.}
\end{tikzpicture}
\hspace{-0.6cm}
\begin{tikzpicture}
	\card
	\cardstrip
	\cardbanner{banner/white.png}
	\cardtitle{\scriptsize{Empfohlene 10er Sätze\qquad}}
	\cardcontent{\emph{Blütezeit}

	\smallskip 
	
	\emph{Auf hoher See:} \\ 
	Embargo, Hafen, Ausguck, Schmuggler, Insel, Karawane, Piratenschiff, Bazar, Entdecker, Werft

	\smallskip 
	
	\emph{Geheime Pläne:} \\ 
	Leuchtturm, Perlentaucher, Botschafter, Fischerdorf, Lagerhaus, Beutelschneider, Außenposten, Schatzkarte, Taktiker, Werft

	\smallskip 
	
	\emph{Schiffswracks:} \\ 
	Eingeborenendorf, Perlentaucher, Lagerhaus, Schmuggler, Müllverwerter, Navigator, Seehexe, Geisterschiff, Handelsschiff, Schatzkammer}
\end{tikzpicture}
\hspace{-0.6cm}
\begin{tikzpicture}
	\card
	\cardstrip
	\cardbanner{banner/white.png}
	\cardtitle{\scriptsize{Empfohlene 10er Sätze\qquad}}
	\cardcontent{\emph{Blütezeit und Basisspiel:}

	\smallskip 
	
	\emph{Griff nach den Sternen:} \\ 
	Keller, Ausguck, Dorf, Beutelschneider, Seehexe, Spion, Geisterschiff, Ratsversammlung, Schatzkarte, Abenteurer

	\smallskip 
	
	\emph{Wiederholungen:} \\ 
	Perlentaucher, Kanzler, Werkstatt, Karawane, Miliz, Piratenschiff, Außenposten, Entdecker, Jahrmarkt, Schatzkammer

	\smallskip 
	
	\emph{Geben und Nehmen:} \\ 
	Hafen, Botschafter, Fischerdorf, Schmuggler, Geldverleiher, Insel, Müllverwerter, Bibliothek, Hexe, Markt}
\end{tikzpicture}
\hspace{0.6cm}

	    % Basic settings for this card set
\renewcommand{\cardcolor}{seaside}
\renewcommand{\cardextension}{Erweiterung I}
\renewcommand{\cardextensiontitle}{Seaside}
\renewcommand{\seticon}{seaside.png}

\clearpage
\newpage
\section{\cardextension \ - \cardextensiontitle \ (Rio Grande Games 2014)}

\begin{tikzpicture}
	\card
	\cardstrip
	\cardbanner{banner/white.png}
	\cardicon{icons/coin.png}
	\cardprice{2}
	\cardtitle{\tiny{Eingeborenendorf}}
	\cardcontent{Wenn du dein erstes \emph{EINGEBORENENDORF} nimmst oder kaufst, erhältst du ein Eingeborenen-Tableau und legst es vor dir ab. 

	\medskip

	Immer wenn du ein \emph{EINGEBORENENDORF} ausspielst, wählst du genau eine der beiden Anweisungen und führst sie wenn möglich aus. Du darfst eine Anweisung auch wählen, wenn du sie nicht ausführen kannst. Karten, die du auf das Tableau legst, werden immer verdeckt abgelegt. Du darfst dir jederzeit die Karten auf deinem Tableau ansehen. 

	\medskip

	Die ausgespielte Aktionskarte \emph{EINGEBORENENDORF} legst du in der Aufräumphase ab. Alle Karten auf dem Tableau gehören auch zum Kartensatz eines Spielers. Alle Karten auf den Tableaus werden bei Spielende mit berücksichtigt.}
\end{tikzpicture}
\hspace{-0.6cm}
\begin{tikzpicture}
	\card
	\cardstrip
	\cardbanner{banner/white.png}
	\cardicon{icons/coin.png}
	\cardprice{2}
	\cardtitle{Embargo}
	\cardcontent{Wenn du das \emph{EMBARGO} in deiner Aktionsphase ausspielst, musst du es entsorgen und einen Embargomarker auf einen beliebigen Vorratsstapel (Königreichkarten \emph{und} Geldkarten sind erlaubt) legen. Um die +\coin[2] in der Kaufphase nicht zu \enquote{vergessen}, empfehlen wir, das entsorgte \emph{EMBARGO} zunächst separat neben den Müllstapel zu legen und erst in der Aufräumphase endgültig zu entsorgen.

	\medskip
Wenn du das \emph{EMBARGO} auf einen \emph{THRONSAAL} folgend ausspielst, legst du 2 Embargomarker. Du kannst sie auf denselben oder unterschiedliche Stapel legen. Wenn keine Embargomarker mehr vorhanden sind, benutze einen geeigneten Ersatz (z.B. echte Geldmünzen). Auf jeden Vorratsstapel dürfen beliebig viele Embargomarker gelegt werden. 

	\medskip

	Spieler, die Karten von einem Vorratsstapel mit einem oder mehreren Embargomarkern kaufen, müssen pro Marker eine Fluchkarte nehmen. Wer Karten von einem Stapel mit Embargomarker(n) auf eine andere Weise nimmt (z. B. durch den \emph{SCHMUGGLER}), nimmt \emph{keine} Fluchkarte. Wenn keine Fluchkarten mehr vorrätig sind, haben die Marker keinen Effekt.}
\end{tikzpicture}
\hspace{-0.6cm}
\begin{tikzpicture}
	\card
	\cardstrip
	\cardbanner{banner/orange.png}
	\cardicon{icons/coin.png}
	\cardprice{2}
	\cardtitle{Hafen}
	\cardcontent{Der \emph{HAFEN} ist eine Dauerkarte. Lege eine Handkarte verdeckt auf den \emph{HAFEN}. Diese und der \emph{HAFEN} werden in der Aufräumphase nicht abgelegt. Nimm zu Beginn deines nächsten Zuges die zur Seite gelegte Karte auf die Hand. Lege den \emph{HAFEN} in der Aufräumphase ab.}
\end{tikzpicture}
\hspace{-0.6cm}
\begin{tikzpicture}
	\card
	\cardstrip
	\cardbanner{banner/orange.png}
	\cardicon{icons/coin.png}
	\cardprice{2}
	\cardtitle{Leuchtturm}
	\cardcontent{Der \emph{LEUCHTTURM} ist eine Dauerkarte. Solange der \emph{LEUCHTTURM} offen vor dir liegt (im Spielbereich oder darüber), bist du grundsätzlich nicht betroffen, wenn Mitspieler Angriffskarten ausspielen (sogar wenn du das möchtest). Selber ausgespielte Angriffskarten werden vom \emph{LEUCHTTURM} nicht abgewehrt. Auf Angriffe von Mitspielern darfst du weiterhin zusätzlich Reaktionskarten ausspielen. Lege den \emph{LEUCHTTURM} in der Aufräumphase des nächsten Zuges ab.}
\end{tikzpicture}
\hspace{-0.6cm}
\begin{tikzpicture}
	\card
	\cardstrip
	\cardbanner{banner/white.png}
	\cardicon{icons/coin.png}
	\cardprice{2}
	\cardtitle{\footnotesize{Perlentaucher}}
	\cardcontent{Zieh die unterste Karte des Nachziehstapels so hervor, dass du die benachbarte Karte nicht sehen kannst. Schaue sie dir an und lege sie dann verdeckt oben auf den Nachziehstapel \emph{oder} zurück unter den Nachziehstapel.}
\end{tikzpicture}
\hspace{-0.6cm}
\begin{tikzpicture}
	\card
	\cardstrip
	\cardbanner{banner/white.png}
	\cardicon{icons/coin.png}
	\cardprice{3}
	\cardtitle{Ausguck}
	\cardcontent{Sieh dir erst alle 3 Karten an, bevor du die Anweisungen ausführst. Solltest du weniger als 3 Karten im Nachziehstapel haben, auch nachdem du ggf. den Ablagestapel gemischt hast, führst du die Anweisungen der Reihenfolge nach aus. Anweisungen, für die es keine Karten mehr im Stapel gibt, entfallen.}
\end{tikzpicture}
\hspace{-0.6cm}
\begin{tikzpicture}
	\card
	\cardstrip
	\cardbanner{banner/white.png}
	\cardicon{icons/coin.png}
	\cardprice{3}
	\cardtitle{Botschafter}
	\cardcontent{Wähle eine beliebige Handkarte und zeige sie deinen Mitspielern. Du darfst dann bis zu 2 dieser Karten von deiner Hand zurück in den Vorrat legen. Jeder Mitspieler nimmt sich anschließend eine solche Karte aus dem Vorrat (reihum im Uhrzeigersinn, beginnend beim linken Mitspieler). Der ausgespielte Botschafter kann nicht in den Vorrat zurückgelegt werden.}
\end{tikzpicture}
\hspace{-0.6cm}
\begin{tikzpicture}
	\card
	\cardstrip
	\cardbanner{banner/orange.png}
	\cardicon{icons/coin.png}
	\cardprice{3}
	\cardtitle{Fischerdorf}
	\cardcontent{Das \emph{FISCHERDORF} ist eine Dauerkarte. Du \emph{darfst} 2 weitere Aktionen ausführen und erhältst für die Kaufphase +\coin[1]. 

	\medskip

	In deinem nächsten Zug \emph{darfst} du eine weitere Aktion ausführen und erhältst für die Kaufphase +\coin[1].}
\end{tikzpicture}
\hspace{-0.6cm}
\begin{tikzpicture}
	\card
	\cardstrip
	\cardbanner{banner/white.png}
	\cardicon{icons/coin.png}
	\cardprice{3}
	\cardtitle{Lagerhaus}
	\cardcontent{Ziehe 3 Karten und spiele dann eine Aktionskarte. Danach legst du 3 Handkarten ab. Wenn du weniger als 3 Karten auf der Hand hast, legst du alle Handkarten ab.}
\end{tikzpicture}
\hspace{-0.6cm}
\begin{tikzpicture}
	\card
	\cardstrip
	\cardbanner{banner/white.png}
	\cardicon{icons/coin.png}
	\cardprice{3}
	\cardtitle{Schmuggler}
	\cardcontent{Hat der rechts von dir sitzende Mitspieler in seinem letzten Zug eine Karte mit Kosten von \coin[6] oder weniger genommen, gekauft oder auf andere Art erhalten, nimmst du dir eine gleiche Karte vom Vorrat. Hat der Spieler mehrere Karten genommen, darfst du wählen, welche du nimmst. Da der \emph{SCHMUGGLER} keine Angriffskarte ist, dürfen keine Reaktionskarten ausgespielt werden.}
\end{tikzpicture}
\hspace{-0.6cm}
\begin{tikzpicture}
	\card
	\cardstrip
	\cardbanner{banner/white.png}
	\cardicon{icons/coin.png}
	\cardprice{4}
	\cardtitle{\scriptsize{Beutelschneider}}
	\cardcontent{Alle Mitspieler müssen eine Kupferkarte aus der Hand ablegen. Da der Beutelschneider eine Angriffskarte ist, dürfen die Mitspieler mit einer Reaktionskarte auf diesen Angriff reagieren.}
\end{tikzpicture}
\hspace{-0.6cm}
\begin{tikzpicture}
	\card
	\cardstrip
	\cardbanner{banner/whitegreen.png}
	\cardicon{icons/coin.png}
	\cardprice{4}
	\cardtitle{Insel}
	\cardcontent{Die \emph{INSEL} ist eine kombinierte Aktions- und Punktekarte. Sie kann in der Aktionsphase eingesetzt werden und bringt zusätzlich bei Spielende 2 Punkte. Wenn du deine erste \emph{INSEL} nimmst oder kaufst, erhältst du ein Insel-Tableau und legst es vor dir ab. 

	\medskip

	Immer wenn du eine \emph{INSEL} ausspielst, legst du die ausgespielte \emph{INSEL} und eine beliebige Handkarte offen auf dein Insel-Tableau. Dort verbleiben sie bis zum Spielende. Wenn du mindestens eine Karte auf der Hand hast, musst du eine Handkarte auf dein Insel-Tableau legen. Wenn du keine Karte auf der Hand hast, nachdem du die \emph{INSEL} ausgespielt hast, legst du nur die \emph{INSEL} auf das Tableau. 

	\medskip

	Bei Spielende nimmst du alle Karten vom Insel-Tableau zu deinen Karten.}
\end{tikzpicture}
\hspace{-0.6cm}
\begin{tikzpicture}
	\card
	\cardstrip
	\cardbanner{banner/orange.png}
	\cardicon{icons/coin.png}
	\cardprice{4}
	\cardtitle{Karawane}
	\cardcontent{Die \emph{KARAWANE} ist eine Dauerkarte. Sie wird in der Aufräumphase nicht abgelegt. Ziehe zu Beginn des nächsten Zuges eine Karte und lege die \emph{KARAWANE} in der Aufräumphase dieses Zuges ab.}
\end{tikzpicture}
\hspace{-0.6cm}
\begin{tikzpicture}
	\card
	\cardstrip
	\cardbanner{banner/white.png}
	\cardicon{icons/coin.png}
	\cardprice{4}
	\cardtitle{\footnotesize{Müllverwerter}}
	\cardcontent{Du musst eine Karte entsorgen, sofern du eine auf der Hand hast. Entsprechend der Kosten der entsorgten Karte erhältst du für die Kaufphase +\emph{X}. Wenn du keine Karte entsorgen kannst, erhältst du kein zusätzliches Geld.}
\end{tikzpicture}
\hspace{-0.6cm}
\begin{tikzpicture}
	\card
	\cardstrip
	\cardbanner{banner/white.png}
	\cardicon{icons/coin.png}
	\cardprice{4}
	\cardtitle{Navigator}
	\cardcontent{Schau dir die obersten 5 Karten deines Nachziehstapels an. Sind nicht genügend Karten im Stapel, mischst du deinen Ablagestapel und legst ihn verdeckt unter deinen Nachziehstapel. Sind es nun immer noch weniger als 5 Karten, schaust du dir alle an und legst sie dann entweder ab oder in einer beliebigen Reihenfolge zurück auf den Nachziehstapel.}
\end{tikzpicture}
\hspace{-0.6cm}
\begin{tikzpicture}
	\card
	\cardstrip
	\cardbanner{banner/white.png}
	\cardicon{icons/coin.png}
	\cardprice{4}
	\cardtitle{\footnotesize{Piratenschiff}}
	\cardcontent{Wenn du dein erstes \emph{PIRATENSCHIFF} nimmst oder kaufst, erhältst du ein Piratenschiff-Tableau und legst es vor dir ab. 

	\medskip

	Immer wenn du ein \emph{PIRATENSCHIFF} ausspielst, wählst du \emph{eine} der beiden Anweisungen: 

	\smallskip

	Entweder die \emph{erste Anweisung}: Alle Mitspieler decken die beiden obersten Karten ihres Nachziehstapels auf. Dann entsorgen sie jeweils eine Geldkarte nach deiner Wahl. Hat ein Mitspieler keine Geldkarte aufgedeckt, entsorgt er keine Karte. Die restlichen aufgedeckten Karten werden abgelegt. Wird mindestens eine Karte entsorgt, erhältst du einen \emph{Geldmarker} und legst ihn auf dein Tableau; 

	\smallskip

	oder die \emph{zweite Anweisung}: Du erhältst pro \emph{Geldmarker} auf deinem Piratenschiff-Tableau in der Kaufphase +\coin[1].

	\medskip

	Nach der Nutzung in der Kaufphase verbleiben die Geldmarker auf dem Tableau und können beim erneuten Ausspielen eines \emph{PIRATENSCHIFFES} wieder eingesetzt werden. Mitspieler können auf das Ausspielen eines \emph{PIRATENSCHIFFES} mit Reaktionskarten reagieren, auch wenn du die zweite Anweisung wählst, die deine Mitspieler nicht direkt betrifft.}
\end{tikzpicture}
\hspace{-0.6cm}
\begin{tikzpicture}
	\card
	\cardstrip
	\cardbanner{banner/white.png}
	\cardicon{icons/coin.png}
	\cardprice{4}
	\cardtitle{Schatzkarte}
	\cardcontent{Nur wenn du zusätzlich zu der ausgespielten \emph{SCHATZKARTE} noch eine weitere auf der Hand hast und beide entsorgst, erhältst du 4 Gold. Sollten weniger als  4 Gold im Vorrat sein, nimmst du dir so viele Goldkarten wie vorhanden sind. Lege alle auf diese Weise erhaltenen Goldkarten verdeckt auf den Nachziehstapel. Solltest du nur eine \emph{SCHATZKARTE} auf der Hand haben und diese ausspielen, musst du diese Karte entsorgen, erhältst aber nichts dafür.}
\end{tikzpicture}
\hspace{-0.6cm}
\begin{tikzpicture}
	\card
	\cardstrip
	\cardbanner{banner/white.png}
	\cardicon{icons/coin.png}
	\cardprice{4}
	\cardtitle{Seehexe}
	\cardcontent{Sollte der Nachziehstapel eines Mitspielers leer sein, mischt er seinen Ablagestapel und legt die oberste Karte des neuen Nachziehstapels ab. Hat ein Spieler keine Karten mehr in seinem Nachziehstapel, kann er zwar keine Karte ablegen, nimmt sich aber trotzdem eine Fluchkarte. Beginnend mit dem Mitspieler links von dem Spieler, der die \emph{SEEHEXE} ausgespielt hat, nimmt sich jeder Mitspieler einen \emph{FLUCH} vom Vorrat. Sollten nicht mehr genügend Fluchkarten für alle Spieler vorhanden sein, werden die restlichen in o. g. Reihenfolge verteilt. }
\end{tikzpicture}
\hspace{-0.6cm}
\begin{tikzpicture}
	\card
	\cardstrip
	\cardbanner{banner/orange.png}
	\cardicon{icons/coin.png}
	\cardprice{5}
	\cardtitle{\footnotesize{Aussenposten}}
	\cardcontent{Der \emph{AUSSENPOSTEN} ist eine Dauerkarte, die bis zum Ende des nächsten Zuges (Extrazug) im Spiel bleibt und erst in der Aufräumphase des nächsten Zuges (Extrazug) abgelegt wird. Der \emph{AUSSENPOSTEN} kommt erst in der Aufräumphase des Zuges, in dem er ausgespielt wird, zum Einsatz. Du ziehst in diesem Fall nur 3 statt 5 Karten nach und führst den Extrazug \emph{sofort} aus. 

	\medskip

	Wenn du den \emph{AUSSENPOSTEN} zusammen mit weiteren Dauerkarten ausgespielt hast, kommen die \enquote{Zu Beginn deines nächsten Zuges}-Anweisungen der Dauerkarten in deinem Extrazug zum Einsatz. Spielst du in deinem Extrazug einen weiteren \emph{AUSSENPOSTEN}, erhältst du keinen weiteren Extrazug. Am Ende deines Extrazuges legst du den \emph{AUSSENPOSTEN} ab und ziehst 5 Karten nach.}
\end{tikzpicture}
\hspace{-0.6cm}
\begin{tikzpicture}
	\card
	\cardstrip
	\cardbanner{banner/white.png}
	\cardicon{icons/coin.png}
	\cardprice{5}
	\cardtitle{Bazar}
	\cardcontent{Du \emph{musst} eine Karte nachziehen, \emph{darfst} 2 weitere Aktionen ausführen und erhältst für die Kaufphase +\coin[2].}
\end{tikzpicture}
\hspace{-0.6cm}
\begin{tikzpicture}
	\card
	\cardstrip
	\cardbanner{banner/white.png}
	\cardicon{icons/coin.png}
	\cardprice{5}
	\cardtitle{Entdecker}
	\cardcontent{Wenn du eine Provinz aus der Hand aufdeckst, erhältst du ein Gold. Wenn du das nicht tun kannst (weil du keine Provinz auf der Hand hast) oder willst (weil du deine Provinz nicht zeigen möchtest), erhältst du ein Silber. Nimm das Gold oder Silber auf die Hand.}
\end{tikzpicture}
\hspace{-0.6cm}
\begin{tikzpicture}
	\card
	\cardstrip
	\cardbanner{banner/white.png}
	\cardicon{icons/coin.png}
	\cardprice{5}
	\cardtitle{\footnotesize{Geisterschiff}}
	\cardcontent{Deine Mitspieler müssen Karten aus ihrer Hand verdeckt auf den Nachziehstapel legen, bis sie nur noch 3 Karten auf der Hand haben. Welche Karten sie auf den Nachziehstapel legen, entscheiden die Mitspieler selbst. Spieler, die zum Zeitpunkt des Angriffs bereits 3 oder weniger Karten auf der Hand haben, müssen keine Karten auf den Nachziehstapel legen. }
\end{tikzpicture}
\hspace{-0.6cm}
\begin{tikzpicture}
	\card
	\cardstrip
	\cardbanner{banner/orange.png}
	\cardicon{icons/coin.png}
	\cardprice{5}
	\cardtitle{\footnotesize{Handelsschiff}}
	\cardcontent{Das \emph{HANDELSSCHIFF} ist eine Dauerkarte. Du erhältst für deine Kaufphase +\coin[2]. 

	\medskip

	Zu Beginn deines nächsten Zuges erhältst du +\coin[2] für die Kaufphase. Lege das \emph{HANDELSSCHIFF} in der Aufräumphase dieses Zuges ab.}
\end{tikzpicture}
\hspace{-0.6cm}
\begin{tikzpicture}
	\card
	\cardstrip
	\cardbanner{banner/white.png}
	\cardicon{icons/coin.png}
	\cardprice{5}
	\cardtitle{\footnotesize{Schatzkammer}}
	\cardcontent{Wenn du eine \emph{SCHATZKAMMER} spielst und in diesem Zug \emph{keine} Punktekarte gekauft hast, \emph{darfst} du die ausgespielte \emph{SCHATZKAMMER} in der Aufräumphase zurück auf den Nachziehstapel legen. 

	\medskip

	Wenn du mehrere \emph{SCHATZKAMMERN} ausgespielt hast, darfst du auch diese \emph{SCHATZKAMMERN} auf den Nachziehstapel zurücklegen. Wenn du eine Punktekarte auf andere Art nimmst bzw. erhältst (d. h. \emph{nicht} kaufst), darfst du \emph{SCHATZKAMMERN} zurück auf den Nachziehstapel legen. 

	\medskip

	Wenn du deine ausgespielte \emph{SCHATZKAMMER} gern zurücklegen möchtest, das aber in der Aufräumphase vergisst und die Karte bereits auf den Ablagestapel gelegt hast, darfst du dies nachträglich \emph{nicht} rückgängig machen.}
\end{tikzpicture}
\hspace{-0.6cm}
\begin{tikzpicture}
	\card
	\cardstrip
	\cardbanner{banner/orange.png}
	\cardicon{icons/coin.png}
	\cardprice{5}
	\cardtitle{Taktiker}
	\cardcontent{Der \emph{TAKTIKER} ist eine Dauerkarte. Sobald du diese Karte ausspielst, legst du alle Handkarten ab. Nur wenn du auf diese Weise mindestens eine Handkarte abgelegt hast, ziehst du zu Beginn deines nächsten Zuges 5 Karten. Außerdem erhältst du dann im nächsten Zug eine zusätzliche Aktion und einen zusätzlichen Kauf. 

	\medskip

	\emph{Grundsätzlich gilt: Nur wenn du mindestens eine Handkarte ablegen kannst, erhältst du den Bonus im nächsten Zug.}

	\medskip

	Wenn du den \emph{TAKTIKER} auf einen \emph{THRONSAAL} spielst, erhältst du den Bonus im nächsten Zug nur einmal, da du beim zweiten Ausspielen des \emph{TAKTIKERS} keine Handkarte mehr auf der Hand hast und damit die Bedingung nicht erfüllst. }
\end{tikzpicture}
\hspace{-0.6cm}
\begin{tikzpicture}
	\card
	\cardstrip
	\cardbanner{banner/orange.png}
	\cardicon{icons/coin.png}
	\cardprice{5}
	\cardtitle{Werft}
	\cardcontent{ Die \emph{WERFT} ist eine Dauerkarte. Du \emph{musst} sofort 2 Karten nachziehen und \emph{darfst} einen weiteren Kauf tätigen. 

	\medskip

	Zu Beginn deines nächsten Zuges (nicht vorher) \emph{musst} du wieder 2 Karten ziehen und \emph{darfst} einen weiteren Kauf tätigen.}
\end{tikzpicture}
\hspace{-0.6cm}
\begin{tikzpicture}
	\card
	\cardstrip
	\cardbanner{banner/white.png}
	\cardtitle{\scriptsize{Empfohlene 10er Sätze\qquad}}
	\cardcontent{\emph{Auf hoher See:}\\
	Ausguck, Bazar, Embargo, Entdecker, Hafen, Insel, Karawane, Piratenschiff, Schmuggler, Werft

	\smallskip

	\emph{Vergrabene Schätze:}\\
	Außenposten, Beutelschneider, Botschafter, Fischerdorf, Lagerhaus, Leuchtturm, Perlentaucher, Schatzkarte, Taktiker, Werft

	\smallskip

	\emph{Schiffswracks:}\\
	Eingeborenen, Geisterschiff, Handelsschiff, Lagerhaus, Leuchtturm, Perlentaucher, Schatzkammer, Schmuggler, Seehexe

	\smallskip

	\emph{Griff nach den Sternen} (Seaside + \textit{Basisspiel}):\\
	Ausguck, Beutelschneider, Geisterschiff, Schatzkarte, Seehexe, \textit{Abenteurer}, \textit{Dorf}, \textit{Keller}, \textit{Ratsversammlung}, \textit{Spion}

	\smallskip

	\emph{Wiederholungen} (Seaside + \textit{Basisspiel}):\\
	Außenposten, Entdecker, Karawane, Perlentaucher, Piratenschiff, Schatzkammer, \textit{Jahrmarkt}, \textit{Kanzler}, \textit{Miliz}, \textit{Werkstatt}

	\smallskip

	\emph{Geben und Nehmen} (Seaside + \textit{Basisspiel}):\\
	Botschafter, Fischerdorf, Hafen, Insel, Müllverwerter, Schmuggler, \textit{Bibliothek}, \textit{Geldverleiher}, \textit{Hexe}, \textit{Markt}}
\end{tikzpicture}
\hspace{0.6cm}

	    % Basic settings for this card set
\renewcommand{\cardcolor}{alchemy}
\renewcommand{\cardextension}{Erweiterung II}
\renewcommand{\cardextensiontitle}{Die Alchemisten}
\renewcommand{\seticon}{alchemy.png}

\clearpage
\newpage
\section{\cardextension \ - \cardextensiontitle \ (Hans Im Glück 2010)}

\begin{tikzpicture}
	\card
	\cardstrip
	\cardbanner{banner/green.png}
	\cardicon{icons/potion.png}
	\cardtitle{Weinberg}
	\cardcontent{Diese Königreichkarte ist eine Punktekarte, keine Aktionskarte. Sie hat bis zum Ende des Spiels keine Funktion. Bei der Wertung zählt sie 1 Punkt pro volle 3 Aktionskarten im gesamten Kartensatz (Nachziehstapel, Ablagestapel und Handkarten) des Spielers. Du zählst alle deine Aktionskarten bei Spielende, teilst die Anzahl durch 3 und rundest ab. Kombinierte Aktionskarten sind auch Aktionskarten. Für 11 Aktionskarten erhältst du beispielsweise für jeden deiner Weinberge 3 Punkte. Im Spiel zu 3. und 4. werden 12 Karten verwendet, im Spiel zu 2. werden 8 Karten verwendet.}
\end{tikzpicture}
\hspace{-0.6cm}
\begin{tikzpicture}
	\card
	\cardstrip
	\cardbanner{banner/white.png}
	\cardicon{icons/potion.png}
	\cardtitle{Verwandlung}
	\cardcontent{Hast du keine Karte mehr auf der Hand, die du entsorgen könntest, erhältst du nichts. Wenn du einen Fluch entsorgst erhältst du nichts. Der Fluch ist keine Punkte-, keine Aktions- und keine Geldkarte. Entsorgst du eine Karte mit kombiniertem Kartentyp, erhältst du den Bonus für beide Kartentypen. Für die Adeligen (Dominion – Die Intrige) nimmst du dir z. B. ein Herzogtum und ein Gold. Die Karten nimmst du dir vom Vorrat. Ist keine entsprechende Karte mehr im Vorrat, erhältst du nichts.
	}
\end{tikzpicture}
\hspace{-0.6cm}
\begin{tikzpicture}
	\card
	\cardstrip
	\cardbanner{banner/white.png}
	\cardicon{icons/coin.png}
	\cardprice{2}
	\cardtitle{\scriptsize{Kräuterkundiger}}
	\cardcontent{Wenn du diese Karte ausspielst erhältst du +1 Geld für die Kaufphase und du darfst eine weitere Karte kaufen. Wenn du den Kräuterkundigen ablegst (normalerweise in der Aufräumphase), darfst du eine Geldkarte, die vor dir ausliegt, zurück auf deinen Nachziehstapel legen, statt sie abzulegen. Wenn dein Nachziehstapel leer ist, legst du nur die Geldkarte als neuen Nachziehstapel bereit. Du entscheidest, in welcher Reihenfolge du die vor dir ausliegenden Karten ablegst. Hast du z. B. einen Kräuterkundigen, einen Alchemisten und einen Trank ausliegen, darfst du zuerst den Alchemisten zurück auf deinen Nachziehstapel legen, dann den Kräuterkundigen ablegen und dafür den Trank zurück auf deinen Nachziehstapel legen. Wenn du mehrere Kräuterkundige im Spiel hast, darfst du für jeden davon eine Geldkarte zurück auf deinen Nachziehstapel legen.
	}
\end{tikzpicture}
\hspace{-0.6cm}
\begin{tikzpicture}
	\card
	\cardstrip
	\cardbanner{banner/white.png}
	\cardicon{icons/coin.png}
	\cardprice{2}
	\cardiconaddition{icons/potion.png}
	\cardtitle{\quad \footnotesize{Apotheker}}
	\cardcontent{Wenn du diese Karte ausspielst musst du sofort eine Karte nachziehen. Dann deckst du die obersten 4 Karten von deinem Nachziehstapel auf. Sollte dein Nachziehstapel beim Aufdecken zu Ende gehen, mischt du deinen Ablagestapel. Hast du nach dem Mischen immernoch weniger als 4 Karten zum Aufdecken, deckst du nur so viele Karten auf wie möglich. Danach musst du alle aufgedeckten Kupfer- und Trankkarten auf die Hand nehmen. Die übrigen aufgedeckten Karten legst du in beliebiger Reihenfolge zurück auf deinen Nachziehstapel. Wenn dein Nachziehstapel nach dem aufdecken der Karten leer ist, werden die zurück gelegten Karten dein neuer Nachziehstapel.
	}
\end{tikzpicture}
\hspace{-0.6cm}
\begin{tikzpicture}
	\card
	\cardstrip
	\cardbanner{banner/white.png}
	\cardicon{icons/coin.png}
	\cardprice{2}
	\cardiconaddition{icons/potion.png}
	\cardtitle{\quad \footnotesize{Universität}}
	\cardcontent{Du darfst die eine Aktionskarte, die bis zu 5 Geld kostet, vom Vorrat nehmen. Du musst jedoch keine Karte nehmen. Kombinierte Aktionskarten sind auch Aktionskarten. Du darfst keine Karte mit Trank in den Kosten nehmen.
	}
\end{tikzpicture}
\hspace{-0.6cm}
\begin{tikzpicture}
	\card
	\cardstrip
	\cardbanner{banner/white.png}
	\cardicon{icons/coin.png}
	\cardprice{2}
	\cardiconaddition{icons/potion.png}
	\cardtitle{\quad Vision}
	\cardcontent{Zuerst decken alle Spieler die oberste Karte ihres Nachziehstapels auf. Du entscheidest dann Spieler für Spieler extra (auch bei dir selbst), ob er die aufgedeckte Karte auf seinen Ablagestapel oder zurück auf seinen Nachziehstapel legt. Danach deckst du solange Karten von deinem Nachziehstapel auf, bis du eine Karte aufgedeckt hast, die keine Aktionskarte ist. Nun nimmst du alle gerade aufgedeckten Karten auf die Hand (auch die Nicht-Aktionskarte). Ist bereits die erste Karte, die du dabei aufgedeckt hast, keine Aktionskarte, so nimmst du nur diese eine Karte auf die Hand. Wenn du nach dem Mischen deines Ablagestapels alle Karten aufgedeckt hast und nur Aktionskarten o en liegen, so nimmst du alle aufgedeckten Karten auf die Hand. Kombinierte Aktionskarten sind auch Aktionskarten. Die durch die erste Anweisung aufgedeckten Karten aller Spieler werden hierbei nicht mehr beachtet. Du kannst auch keine dieser Karten auf die Hand nehmen.
	}
\end{tikzpicture}
\hspace{-0.6cm}
\begin{tikzpicture}
	\card
	\cardstrip
	\cardbanner{banner/white.png}
	\cardicon{icons/coin.png}
	\cardprice{3}
	\cardiconaddition{icons/potion.png}
	\cardtitle{\quad Alchemist}
	\cardcontent{Wenn du diese Karte ausspielst musst du sofort 2 Karten nachziehen und darfst dann eine weitere Aktionskarte ausspielen. Wenn mindestens ein Trank im Spiel ist, darfst du in der Aufräumphase alle vor dir ausliegenden Alchemisten zurück auf deinen Nachziehstapel legen, statt sie abzulegen. Hast du mehrere Alchemisten im Spiel, darfst du für jeden einzeln entscheiden, ob du ihn normal ablegst oder zurück auf deinen Nachziehstapel legst. Ist dein Nachziehstapel leer, legst du den Alchemisten alleine als Nachziehstapel bereit. Du legst den oder die Alchemisten ab oder zurück auf deinen Nachziehstapel, bevor du die Karten am Ende deines Zuges nachziehst. Du darfst in der Kaufphase einen Trank ausspielen, auch wenn du keine Karte damit kaufst.
	}
\end{tikzpicture}
\hspace{-0.6cm}
\begin{tikzpicture}
	\card
	\cardstrip
	\cardbanner{banner/gold.png}
	\cardicon{icons/coin.png}
	\cardprice{3}
	\cardiconaddition{icons/potion.png}
	\cardtitle{\quad \tiny{Stein der Weisen}}
	\cardcontent{Diese Karte ist eine Geldkarte und eine Königreichkarte. Sie ist nur im Spiel, wenn sie als eine der 10 ausliegenden Königreichkarten für dieses Spiel ausgewühlt wurde. Der Stein der Weisen darf, wie andere Geldkarten, in der Kaufphase ausgespielt werden. Wenn du den Stein der Weisen ausspielst zählst du zunächst die momentane Anzahl der Karten in deinem Nachziehstapel und in deinem Ablagestapel. Dann zählst du die Anzahl beider Stapel zusammen, teilst die Summe durch 5 und rundest ab. Das Ergebnis gibt den Wert der Karte für diese Kaufphase an. Der Wert gilt für diese gesamte Kaufphase, auch wenn sich die Anzahl der Karten noch verändert. Spielt ein Spieler mehrere dieser Karten, so hat jede den ermittelten Wert. Wird die Karte in einem späteren Zug erneut ausgespielt, wird auch der Wert neu ermittelt. Du darfst beim Durchzählen deines Nachziehstapels weder die Karten ansehen, noch deren Reihenfolge verändern. Beim Durchzählen deines Ablagestapels darfst du die Karten ansehen und deren Reihenfolge verändern. Du zählst nur die Karten der beiden Stapel, ausgespielte, beiseite gelegte und Handkarten zählst du nicht mit. Du darfst keine Geldkarten mehr ausspielen, nachdem du eine Karte gekauft hast. Du darfst also weder den Stein der Weisen noch andere Geldkarten ausspielen, wenn du in dieser Kaufphase bereits eine Karte gekauft hast.
	}
\end{tikzpicture}
\hspace{-0.6cm}
\begin{tikzpicture}
	\card
	\cardstrip
	\cardbanner{banner/white.png}
	\cardicon{icons/coin.png}
	\cardprice{3}
	\cardiconaddition{icons/potion.png}
	\cardtitle{\quad \footnotesize{Vertrauter}}
	\cardcontent{Sind nicht mehr genügend Fluchkarten im Vorrat, wenn du den Vertrauten aus- spielst, werden die restlichen Fluchkarten, beginnend beim Spieler links von dir, in Spielerreihenfolge verteilt. Du ziehst immer eine Karte nach und darfst eine weitere Aktionskarte ausspielen, auch wenn keine Fluchkarten mehr im allgemeinen Vorrat sind. Die Fluchkarten legen die Spieler sofort auf ihren Ablagestapel.
	}
\end{tikzpicture}
\hspace{-0.6cm}
\begin{tikzpicture}
	\card
	\cardstrip
	\cardbanner{banner/white.png}
	\cardicon{icons/coin.png}
	\cardprice{4}
	\cardiconaddition{icons/potion.png}
	\cardtitle{\quad Golem}
	\cardcontent{Du deckst solange Karten von deinem Nachziehstapel auf, bis 2 Aktionskarten offen liegen, die keine Golemkarten sind. Dann legst du alle aufgedeckten Golemkarten und alle Karten, die keine Aktionskarten sind, ab. Wenn du auch nach dem Mischen deines Ablagestapels keine oder nur 1 Aktionskarte (außer dem Golem) aufdecken kannst, so führst du die Anweisung mit weniger als 2 Karten aus. Hast du auf diese Weise 1 oder 2 Aktionskarten aufgedeckt, musst du diese in beliebiger Reihenfolge ausspielen. Du darfst auf diese Weise aufgedeckte Aktionskarten nicht auf die Hand nehmen. Alle Anweisungen, die sich auf deine Handkarten beziehen, haben keinen Effekt auf die beiden aufgedeckten Karten. Ist z. B. eine der aufgedeckten Karten ein Thronsaal, so kannst du die andere der beiden Karten nicht für diesen Thronsaal auswühlen.
	}
\end{tikzpicture}
\hspace{-0.6cm}
\begin{tikzpicture}
	\card
	\cardstrip
	\cardbanner{banner/white.png}
	\cardicon{icons/coin.png}
	\cardprice{5}
	\cardtitle{Lehrling}
	\cardcontent{Wenn du eine Karte auf deiner Hand hast, musst du eine Karte entsorgen. Wenn du eine Karte, die 0 Geld kostet (z. B. Kupfer oder Fluch) entsorgst oder wenn du keine Karte mehr auf der Hand hast, ziehst du keine Karte nach. Ansonsten ziehst du für eine Karte, die x Geld kostet, X Karten nach. Zusätzlich dazu ziehst du 2 Karten nach, wenn die entsorgte Karte kostet. Wenn du z. B. die Karte Golem (4 Geld, Trank) entsorgst, ziehst du 6 Karten nach.
	}
\end{tikzpicture}
\hspace{-0.6cm}
\begin{tikzpicture}
	\card
	\cardstrip
	\cardbanner{banner/white.png}
	\cardicon{icons/coin.png}
	\cardprice{6}
	\cardiconaddition{icons/potion.png}
	\cardtitle{\quad \scriptsize{Besessenheit}}
	\cardcontent{\tiny{\begin{Spacing}{1}
	\vspace{1em}
	(Teil 1)

	Der Spieler links von dir ist der aktive Spieler bei dem Extra-Zug durch die Besessenheit. Kartenanweisungen, die sich auf den aktiven Spieler beziehen, betreffen also den Spieler links von dir und seinen Kartensatz. In den Kartenanweisungen wird der aktive Spieler meist mit \enquote{du} angesprochen. Du darfst alle Karten sehen, die der Spieler links von dir in seinem Extra-Zug sieht. Das betrifft auch die Karten, die er in der Aufräumphase für seinen nächsten Zug nachzieht. Du darfst auch alle Karten ansehen, die er ansehen darf, z. B. zur Seite gelegte Karten auf dem Eingeborenendorf (Dominion – Seaside). Und du darfst alle Kartenstapel durchzählen, die er durchzählen darf. Du füllst alle Entscheidungen für den Spieler links von dir, welche Karten er ausspielt und in welcher Reihenfolge, alle Entscheidungen, die durch Kartenanweisungen erlaubt werden und welche Karten er kauft. Alle Karten, die der Spieler links von dir in seinem Extra-Zug nimmt oder kauft, legst du auf deinen Ablagestapel. Dies gilt auch, wenn er die Karte aufgrund einer bestimmten Anweisung auf die Hand oder nehmen oder anderswo ablegen müsste. Du erhältst nur Karten, die er nimmt oder kauft, keine Marker, z. B. Piratenschiff (Dominion – Seaside). Wenn der Spieler links von dir in seinem Extra-Zug Karten entsorgt, werden diese zunächst zur Seite gelegt und am Ende des Zuges (nach der Aufräumphase) auf seinen Ablagestapel gelegt. Für die Kartenanweisung, die das Entsorgen der Karte fordert, gilt diese als entsorgt, z. B. könntest du das Bergwerk (Dominion – Die Intrige) entsorgen und + erhalten. Das Bergwerk wird dabei nicht auf den Müllstapel gelegt, der Spieler verliert die Karte also nicht. Karten anderer Spieler, die während dieses Zuges entsorgt werden (z. B. durch Angriffskarten, wie Trickser oder Saboteur, Dominion – Die Intrige), werden dauerhaft entsorgt. Karten, die weitergegeben werden (z. B. durch die Maskerade, Dominion – Die Intrige), erhält der Spieler am Ende des Zuges nicht zurück.
	\end{Spacing}}}
\end{tikzpicture}
\hspace{-0.6cm}
\begin{tikzpicture}
	\card
	\cardstrip
	\cardbanner{banner/white.png}
	\cardicon{icons/coin.png}
	\cardprice{6}
	\cardiconaddition{icons/potion.png}
	\cardtitle{\quad \scriptsize{Besessenheit}}
	\cardcontent{\tiny{\begin{Spacing}{1}
	\vspace{1em}
	(Teil 2)
	
	Karten, die in den Vorrat zurückgelegt werden (z. B. Botschafter, Dominion – Seaside), erhält der Spieler auch nicht zurück. Spielt der Spieler links von dir (auf deinen Wunsch hin) eine Angriffskarte, so bist du, wie alle übrigen Spieler, normal betroffen. Du kannst, wie üblich, mit Reaktionskarten aus deiner eigenen Hand auf den Angriff reagieren. Du kannst keine Reaktionskarten des Spielers links von dir verwenden um auf den Angriff zu reagieren. Besessenheit bewirkt einen Extra-Zug, wie z. B. der Außenposten (Dominion – Seaside). Dieser Extra-Zug  findet erst nach deinem Zug statt. Du hast also alle Karten abgelegt und die Handkarten für deinen nächsten Zug nachgezogen. Der Außenposten verhindert nur einen weiteren Extra-Zug durch einen weiteren Außenposten. Du kannst weitere Extrazüge durch andere Karten, wie z. B. die Besessenheit erhalten. Wenn du in deinem Zug den Außenposten (Dominion – Seaside) und die Besessenheit spielst, führst du zuerst den Extra-Zug für den Außenposten aus, danach den Extra-Zug für die Besessenheit. Wenn der Spieler links von dir (auf deinen Wunsch hin) einen Außenposten spielt, erhält er dadurch einen Extra-Zug. In diesem Extra-Zug füllt er selbst seine Entscheidungen und erhält auch Karten, die er nimmt oder kauft selbst. Wenn der Spieler links von dir (auf deinen Wunsch hin) eine weitere Besessenheit ausspielt, so wird ein weiterer Extra-Zug gespielt, in dem der Spieler links von dir (nicht du) die Entscheidungen für den Spieler links von ihm füllt. Extra-Züge (z. B. durch Besessenheit oder Außenposten) werden für die Siegbedingung nicht beachtet. Im Gegensatz zum Außenposten ist die Besessenheit keine Dauer-Karte und wird in der Aufräumphase abgelegt. Die Wirkung der Karte Besessenheit ist kumulativ, spielst du z. B. die Karte in deinem Zug zweimal, so werden auch 2 Extra-Züge durchgeführt. Wichtig: Der Extrazug durch die Besessenheit ist nicht dein Extrazug, sondern der Extrazug des Spielers links von dir. Nach dem Extrazug (oder den Extrazügen) führt der Spieler seinen normalen Zug aus.
	\end{Spacing}}}
\end{tikzpicture}
\hspace{-0.6cm}
\begin{tikzpicture}
	\card
	\cardstrip
	\cardbanner{banner/gold.png}
	\cardicon{icons/coin.png}
	\cardprice{4}
	\cardtitle{Trank}
\end{tikzpicture}
\hspace{-0.6cm}
\begin{tikzpicture}
	\card
	\cardstrip
	\cardbanner{banner/white.png}
	\cardtitle{\scriptsize{Empfohlene 10er Sätze\qquad}}
	\cardcontent{\emph{Verbotene Künste} (Alchemisten + \textit{Basisspiel}\\
	Besessenheit, Lehrling, Universität, Vertrauter, \textit{Dieb}, \textit{Gärten}, \textit{Keller}, \textit{Laboratorium}, \textit{Ratsversammlung}, \textit{Thronsaal}

	\smallskip

	\emph{Quacksalber:} (Alchemisten + \textit{Basisspiel}):\\
	Alchemist, Apotheker, Golem, Kräuterkundiger, Verwandlung, \textit{Jahrmarkt}, \textit{Kanzler}, \textit{Keller}, \textit{Miliz}, \textit{Schmiede}

	\smallskip

	\emph{Chemiestunde:} (Alchemisten + \textit{Basisspiel}):\\
	Alchemist, Golem, Stein der Weisen, Universität, \textit{Burggraben}, \textit{Bürokrat}, \textit{Hexe}, \textit{Holzfäller}, \textit{Markt}, \textit{Umbau}

	\smallskip

	\emph{Diener:} (Alchemisten + \textit{Die Intrige}):\\
	Besessenheit, Golem, Verwandlung, Vision, Weinberg, \textit{Große Halle}, \textit{Handlanger}, \textit{Lakai}, \textit{Verschwörer}, \textit{Verwalter}

	\smallskip

	\emph{Geheime Forschungen:} (Alchemisten + \textit{Die Intrige}):\\
	Kräuterkundiger, Stein der Weisen, Universität, Vertrauter, \textit{Adlige}, \textit{Armenviertel}, \textit{Brücke}, \textit{Kerkermeister}, \textit{Lakai}, \textit{Maskerade}

	\smallskip

	\emph{Tröpfe, Tränke, Trottel:} (Alchemisten + \textit{Die Intrige}):\\
	Apotheker, Golem, Lehrling, Vision, \textit{Adlige}, \textit{Baron}, \textit{Eisenhütte}, \textit{Handelsposten}, \textit{Kupferschmied}, \textit{Wunschbrunnen}}
\end{tikzpicture}
\hspace{0.6cm}


		% Basic settings for this card set
\renewcommand{\cardcolor}{alchemy}
\renewcommand{\cardextension}{Erweiterung II}
\renewcommand{\cardextensiontitle}{Die Alchemisten}
\renewcommand{\seticon}{alchemy.png}

\clearpage
\newpage
\section{\cardextension \ - \cardextensiontitle \ (Rio Grande Games 2015)}

\begin{tikzpicture}
	\card
	\cardstrip
	\cardbanner{banner/green.png}
	\cardicon{icons/potion.png}
	\cardtitle{Weinberg}
	\cardcontent{Diese Karte ist eine Punktekarte und hat bis zum Ende des Spiels keine Funktion. Bei Spielende erhält der Spieler, der diese Karte in seinem Kartensatz (Nachziehstapel, Ablagestapel und Handkarten) hat, für jeweils 3 Aktionskarten (auch kombinierte Aktionskarten) 1 Siegpunkt. Es wird immer abgerundet, d.h. wer z.B. 12, 13 oder 14 Aktionskarten besitzt, erhält 4 Siegpunkte. Wer mehrere \emph{WEINBERGE} besitzt, erhält für jeden \emph{WEINBERG} die entsprechende Anzahl Siegpunkte.}
\end{tikzpicture}
\hspace{-0.6cm}
\begin{tikzpicture}
	\card
	\cardstrip
	\cardbanner{banner/white.png}
	\cardicon{icons/potion.png}
	\cardtitle{Verwandlung}
	\cardcontent{Wenn du diese Karte ausspielst und noch mindestens eine Karte auf der Hand hast, musst du eine Handkarte entsorgen. Wenn du keine Karte oder einen \emph{FLUCH} entsorgst, erhältst du nichts. Entsorgst du eine Aktions-, Punkte- oder Geldkarte, erhältst du den jeweiligen Bonus. Entsorgst du eine kombinierte Karte, erhältst du den Bonus beider Kartentypen. Sollte keine entsprechende Karte mehr im Vorrat sein, erhältst du nichts.}
\end{tikzpicture}
\hspace{-0.6cm}
\begin{tikzpicture}
	\card
	\cardstrip
	\cardbanner{banner/white.png}
	\cardicon{icons/coin.png}
	\cardprice{2}
	\cardtitle{\scriptsize{Kräuterkundiger}}
	\cardcontent{Wenn du den \emph{KRÄUTERKUNDIGEN} in der Aufräumphase ablegst, darfst du eine Geldkarte, die vor dir ausliegt, oben auf den Nachziehstapel legen. Ist der Nachziehstapel leer, legst du die Geldkarte auf den leeren Platz; sie ist dann die einzige Karte im Nachziehstapel. Wenn du mehrere \emph{KRÄUTERKUNDIGE} im Spiel hast und ablegst, darfst du für jeden eine ausliegende Geldkarte auf den Nachziehstapel legen.}
\end{tikzpicture}
\hspace{-0.6cm}
\begin{tikzpicture}
	\card
	\cardstrip
	\cardbanner{banner/white.png}
	\cardicon{icons/coin.png}
	\cardprice{2}
	\cardiconaddition{icons/potion.png}
	\cardtitle{\quad \footnotesize{Apotheker}}
	\cardcontent{Sollte der Nachziehstapel während des Aufdeckens aufgebraucht werden, mischst du deinen Ablagestapel und legst ihn als neuen Nachziehstapel bereit. Hast du dann trotzdem nicht genug Karten im Nachziehstapel, um 4 Karten aufzudecken, deckst du nur so viele Karten auf, wie möglich. Alle aufgedeckten \emph{KUPFER} und \emph{TRÄNKE} nimmst du auf die Hand und legst die anderen aufgedeckten Karten zurück auf den Nachziehstapel.}
\end{tikzpicture}
\hspace{-0.6cm}
\begin{tikzpicture}
	\card
	\cardstrip
	\cardbanner{banner/white.png}
	\cardicon{icons/coin.png}
	\cardprice{2}
	\cardiconaddition{icons/potion.png}
	\cardtitle{\quad \footnotesize{Universität}}
	\cardcontent{Du darfst eine Aktionskarte vom Vorrat nehmen, die bis zu \coin[5] kostet. Du darfst allerdings keine Karte nehmen, deren Kosten einen \potion beinhalten.}
\end{tikzpicture}
\hspace{-0.6cm}
\begin{tikzpicture}
	\card
	\cardstrip
	\cardbanner{banner/white.png}
	\cardicon{icons/coin.png}
	\cardprice{2}
	\cardiconaddition{icons/potion.png}
	\cardtitle{\quad Vision}
	\cardcontent{Alle Spieler, auch du selbst, decken die oberste Karte ihres Nachziehstapels auf. Für jeden Spieler entscheidest du separat, ob er die Karte ablegt oder zurück auf seinen Nachziehstapel legt. 

	\medskip

	Danach deckst du so lange Karten von deinem Nachziehstapel auf, bis du eine Karte, die \emph{keine} Aktionskarte ist, aufgedeckt hast (kombinierte Aktionskarten sind auch Aktionskarten). Nimm alle gerade aufgedeckten Karten auf die Hand. Hast du nur Aktionskarten aufgedeckt und dein Nachziehstapel ist aufgebraucht, mischst du deinen Ablagestapel und ziehst weiter, bis du eine Karte aufdeckst, die keine Aktionskarte ist. Findest du auch in diesem Stapel nur Aktionskarten, nimmst du alle Karten auf die Hand.}
\end{tikzpicture}
\hspace{-0.6cm}
\begin{tikzpicture}
	\card
	\cardstrip
	\cardbanner{banner/white.png}
	\cardicon{icons/coin.png}
	\cardprice{3}
	\cardiconaddition{icons/potion.png}
	\cardtitle{\quad Alchemist}
	\cardcontent{Wenn du zusätzlich zu dieser Karte einen \emph{TRANK} im Spiel hast, darfst du diese Karte in der Aufräumphase zurück auf den Nachziehstapel legen, statt sie abzulegen. Wenn du mindestens 1 \emph{TRANK} im Spiel hast, darfst du beliebig viele ausgespielte \emph{ALCHEMISTEN} zurück auf den Nachziehstapel legen.}
\end{tikzpicture}
\hspace{-0.6cm}
\begin{tikzpicture}
	\card
	\cardstrip
	\cardbanner{banner/gold.png}
	\cardicon{icons/coin.png}
	\cardprice{3}
	\cardiconaddition{icons/potion.png}
	\cardtitle{\quad \tiny{Stein der Weisen}}
	\cardcontent{Diese Karte ist eine Geldkarte mit einem variablen Wert und gehört nicht zu den Basiskarten (wie \emph{KUPFER} oder \emph{TRANK}), sondern zu den Königreichkarten. Ausgespielt wird sie aber – wie andere Geldkarten auch – in der Kaufphase. 

	\medskip

	\emph{Wichtig:} Geldkarten dürfen in der Kaufphase nur ausgespielt werden, bevor du die erste Karte kaufst (neue Regeln, S. 4).

	\medskip

	Zähle die Karten, die du in diesem Moment im Nachzieh- und Ablagestapel hast (Summe). Pro volle 5 Karten erhöht sich der Geldwert des \emph{STEIN DER WEISEN} für die Kaufphase in diesem Zug um \emph{1}. Spielst du mehrere \emph{STEIN DER WEISEN} aus, hat jede Karte den entsprechenden Geldwert. Du darfst beim Durchzählen der Karten deines Nachzieh- und Ablagestapels diese weder anschauen, noch ihre Reihenfolge verändern.}
\end{tikzpicture}
\hspace{-0.6cm}
\begin{tikzpicture}
	\card
	\cardstrip
	\cardbanner{banner/white.png}
	\cardicon{icons/coin.png}
	\cardprice{3}
	\cardiconaddition{icons/potion.png}
	\cardtitle{\quad \footnotesize{Vertrauter}}
	\cardcontent{Jeder Mitspieler, beginnend bei deinem linken Nachbarn, muss einen \emph{FLUCH} vom Vorrat nehmen und ihn ablegen. Wird der Vorrat an \emph{FLÜCHEN} dabei aufgebraucht, erhalten die Spieler, für die kein \emph{FLUCH} mehr vorhanden ist, nichts. }
\end{tikzpicture}
\hspace{-0.6cm}
\begin{tikzpicture}
	\card
	\cardstrip
	\cardbanner{banner/white.png}
	\cardicon{icons/coin.png}
	\cardprice{4}
	\cardiconaddition{icons/potion.png}
	\cardtitle{\quad Golem}
	\cardcontent{Decke solange Karten von deinem Nachziehstapel auf, bis du 2 Aktionskarten aufgedeckt hast, die keine \emph{GOLEMS} sind. Alle aufgedeckten \emph{GOLEMS} und Karten, die keine Aktionskarten sind, legst du ab. Hast du, auch nach dem Mischen des Ablagestapels, nur 1 oder gar keine Aktionskarte aufgedeckt, spielt du entsprechend weniger Aktionskarten aus. Du musst die aufgedeckten Aktionskarten ausspielen, darfst allerdings die Reihenfolge selbst bestimmen. Du darfst die Aktionskarten nicht auf die Hand nehmen. Anweisungen, die sich auf Handkarten beziehen, haben keine Auswirkungen auf die aufgedeckten Aktionskarten. Ist eine der aufgedeckten Karten z.B. ein \emph{THRONSAAL}, darfst du eine Karte aus der Hand auswählen und diese zwei Mal ausspielen, du darfst aber nicht die anderen aufgedeckten Karten dafür auswählen, da sie sich nicht auf deiner Hand befinden.}
\end{tikzpicture}
\hspace{-0.6cm}
\begin{tikzpicture}
	\card
	\cardstrip
	\cardbanner{banner/white.png}
	\cardicon{icons/coin.png}
	\cardprice{5}
	\cardtitle{Lehrling}
	\cardcontent{Entsorge eine Handkarte. Wenn du mindestens eine Handkarte hast, musst du eine Karte entsorgen. Pro \coin[X], das die entsorgte Karte kostet, ziehst du eine Karte nach. Wenn die entsorgte Karte außerdem \potion kostet, ziehst du weitere 2 Karten nach.}
\end{tikzpicture}
\hspace{-0.6cm}
\begin{tikzpicture}
	\card
	\cardstrip
	\cardbanner{banner/white.png}
	\cardicon{icons/coin.png}
	\cardprice{6}
	\cardiconaddition{icons/potion.png}
	\cardtitle{\quad \scriptsize{Besessenheit}}
	\cardcontent{\tiny{Zuerst spielst du deinen aktuellen Zug regulär zu Ende, bevor dein linker Nachbar einen Extra-Zug ausführen muss. Da die \emph{BESESSENHEIT} keine Angriffskarte ist, kann sich der Mitspieler nicht gegen den Extra-Zug \enquote{wehren}. 

	\medskip

	Zu Beginn des Extra-Zuges zeigt dein linker Nachbar dir seine Handkarten. Du entscheidest in diesem Zug alles für den Mitspieler – welche Aktionskarten und Geldkarten er spielt und welche Karten er kauft, nimmt, entsorgt etc. Du darfst alle Karten sehen, die auch der Mitspieler sieht – d. h. Handkarten, nachgezogene und angesehene Karten sowie die Karten, die der Mitspieler in der Aufräumphase des Extra-Zuges nachzieht.

	\medskip

	Alle Karten, die er nehmen, kaufen oder auf andere Art erhalten würde, erhältst stattdessen du und legst sie auf deinen Ablagestapel. Das betrifft auch Karten, die er auf die Hand nehmen oder anderweitig ablegen müsste. Alle Münzen (z. B. aus \emph{Die Gilden}) und Geldmarker (z. B. aus \emph{Seaside}), die der Spieler im \emph{BESESSENHEITS}-Zug erhält, bekommst du nicht. 

	\medskip

	Alle Karten, die der Spieler entsorgen müsste, werden separat neben den Müllstapel gelegt. Für weitere Anweisungen, die sich auf entsorgte Karten beziehen, gilt die Karte während des Extra-Zuges als entsorgt. Am Ende des Extra-Zuges legt der Mitspieler diese auf seinen eigenen Ablagestapel. 

	\medskip

	Alle Karten, die der Mitspieler während des Extra-Zuges in den Vorrat zurücklegen muss, werden tatsächlich in den Vorrat zurückgelegt.}}
\end{tikzpicture}
\hspace{-0.6cm}
\begin{tikzpicture}
	\card
	\cardstrip
	\cardbanner{banner/gold.png}
	\cardicon{icons/coin.png}
	\cardprice{4}
	\cardtitle{Trank}
\end{tikzpicture}
\hspace{-0.6cm}
\begin{tikzpicture}
	\card
	\cardstrip
	\cardbanner{banner/white.png}
	\cardtitle{\scriptsize{Empfohlene 10er Sätze\qquad}}
	\cardcontent{\emph{Verbotene Künste} (Alchemisten + \textit{Basisspiel}):\\
	Besessenheit, Lehrling, Universität, Vertrauter, \textit{Dieb}, \textit{Gärten}, \textit{Keller}, \textit{Laboratorium}, \textit{Ratsversammlung}, \textit{Thronsaal}

	\smallskip

	\emph{Quacksalber:} (Alchemisten + \textit{Basisspiel}):\\
	Alchemist, Apotheker, Golem, Kräuterkundiger, Verwandlung, \textit{Jahrmarkt}, \textit{Kanzler}, \textit{Keller}, \textit{Miliz}, \textit{Schmiede}

	\smallskip

	\emph{Chemiestunde:} (Alchemisten + \textit{Basisspiel}):\\
	Alchemist, Golem, Stein der Weisen, Universität, \textit{Burggraben}, \textit{Bürokrat}, \textit{Hexe}, \textit{Holzfäller}, \textit{Markt}, \textit{Umbau}

	\smallskip

	\emph{Diener:} (Alchemisten + \textit{Die Intrige}):\\
	Besessenheit, Golem, Verwandlung, Vision, Weinberg, \textit{Große Halle}, \textit{Handlanger}, \textit{Lakai}, \textit{Verschwörer}, \textit{Verwalter}

	\smallskip

	\emph{Geheime Forschungen:} (Alchemisten + \textit{Die Intrige}):\\
	Kräuterkundiger, Stein der Weisen, Universität, Vertrauter, \textit{Adlige}, \textit{Armenviertel}, \textit{Brücke}, \textit{Kerkermeister}, \textit{Lakai}, \textit{Maskerade}

	\smallskip

	\emph{Tröpfe, Tränke, Trottel:} (Alchemisten + \textit{Die Intrige}):\\
	Apotheker, Golem, Lehrling, Vision, \textit{Adlige}, \textit{Baron}, \textit{Eisenhütte}, \textit{Handelsposten}, \textit{Kupferschmied}, \textit{Wunschbrunnen}}
\end{tikzpicture}
\hspace{0.6cm}

	    % Basic settings for this card set
\renewcommand{\cardcolor}{prosperity}
\renewcommand{\cardextension}{Erweiterung III}
\renewcommand{\cardextensiontitle}{Blütezeit}
\renewcommand{\seticon}{prosperity.png}

\clearpage
\newpage
\section{\cardextension \ - \cardextensiontitle \ (Hans Im Glück 2010)}

\begin{tikzpicture}
	\card
	\cardstrip
	\cardbanner{banner/gold.png}
	\cardicon{icons/coin.png}
	\cardprice{5}
	\cardtitle{Abenteuer}
	\cardcontent{\emph{Errata:} Die Reihenfolge der beiden letzten Anweisungen sollte lauten: \enquote{Lege die übrigen aufgedeckten Karten ab. Lege diese Geldkarte aus.} 

	\smallskip

	Diese Geldkarte hat den Wert \coin[1], wie ein Kupfer. Wenn du diese Karte auslegst, deckst du solange Karten von deinem Nachziehstapel auf, bis du eine Geldkarte aufgedeckt hast. Die auf diese Weise aufgedeckte Geldkarte legst du sofort aus. Die übrigen gerade aufgedeckten Karten legst du auf deinen Ablagestapel. Wenn du auch nach dem Mischen keine Geldkarte aufdecken kannst, legst du alle aufgedeckten Karten ab. Enthält die gerade ausgelegte Geldkarte zusätzliche Anweisungen, so führst du diese Anweisungen nun aus. Wenn du z. B. durch das Abenteuer ein weiteres Abenteuer aufdeckst, deckst du erneut Karten auf, bis du eine weitere Geldkarte \enquote{findest}. Denke daran, dass du die Geldkarten aus deiner Hand in beliebiger Reihenfolge auslegen kannst. Hast du z. B. das Abenteuer und den Lohn auf der Hand, darfst du wählen, welche der Karten du zuerst auslegst.}
\end{tikzpicture}
\hspace{-0.6cm}
\begin{tikzpicture}
	\card
	\cardstrip
	\cardbanner{banner/white.png}
	\cardicon{icons/coin.png}
	\cardprice{4}
	\cardtitle{Arbeiterdorf}
	\cardcontent{Du ziehst zuerst eine Karte nach, dann darfst du 2 weitere Aktionskarten auslegen und in der Kaufphase eine weitere Karte kaufen.}
\end{tikzpicture}
\hspace{-0.6cm}
\begin{tikzpicture}
	\card
	\cardstrip
	\cardbanner{banner/white.png}
	\cardicon{icons/coin.png}
	\cardprice{7}
	\cardtitle{Ausbau}
	\cardcontent{Du kannst den ausgelegten Ausbau selbst nicht entsorgen, da du die Karte nicht mehr auf der Hand hast, wenn du die Anweisung ausführst. Hast du einen weiteren Ausbau auf der Hand, kannst du diesen jedoch entsorgen. Hast du keine Karte auf der Hand, die du entsorgen könntest, darfst du dir auch keine neue Karte nehmen. Die neue Karte darf bis zu \coin[3] mehr kosten, als die entsorgte Karte. Du darfst keine Geldkarten oder virtuelles Geld verwenden, um den Betrag zu erhöhen. Du darfst auch eine Karte nehmen, die weniger kostet oder eine der entsorgten identische Karte nehmen. Die Karte, die du nimmst, legst du auf deinen Ablagestapel.}
\end{tikzpicture}
\hspace{-0.6cm}
\begin{tikzpicture}
	\card
	\cardstrip
	\cardbanner{banner/gold.png}
	\cardicon{icons/coin.png}
	\cardprice{7}
	\cardtitle{Bank}
	\cardcontent{Diese Geldkarte hat einen variablen Wert. Die Bank hat einen Wert von \coin[1] für jede Geldkarte, die zu diesem Zeitpunkt im Spiel ist. Die Bank selbst wird dabei mitgezählt. Hierfür ist es wichtig, dass du Geldkarten in beliebiger Reihenfolge auslegen kannst. Wenn du die Bank alleine ausliegen hast, hat sie einen Wert von \coin[1]. Wenn du 2 Banken hintereinander auslegst, hat die erste einen Wert von \coin[1], die zweite Bank einen Wert von \coin[2]. Der Wert jeder Bank wird sofort nach dem Auslegen bestimmt und ändert sich nicht mehr z. B. durch das Auslegen weiterer Geldkarten.}
\end{tikzpicture}
\hspace{-0.6cm}
\begin{tikzpicture}
	\card
	\cardstrip
	\cardbanner{banner/white.png}
	\cardicon{icons/coin.png}
	\cardprice{4}
	\cardtitle{Bischof}
	\cardcontent{Für diese Karte werden die Punkte-Marker benötigt (siehe neue Regeln). Zuerst erhältst du +\coin[1] und 1 Punkte-Marker auf dein Tableau. Dann musst du eine Karte entsorgen, wenn du noch eine Karte auf der Hand hast. Du darfst dir eine Anzahl Punkte-Marker nehmen, die der Hälfte der Geld-Kosten der entsorgten Karte entspricht. Es werden dabei nur die Geld-Kosten beachtet. Trank-Kosten werden ignoriert. Wenn du z. B. den Golem (\coin[4], \potion) (Dominion – Die Alchemisten) entsorgst, nimmst du dir 2 zusätzliche Punkte-Marker und legst diese auf dein Tableau. Danach dürfen deine Mitspieler reihum jeweils eine Karte entsorgen. Sie erhalten jedoch keine Punkte-Marker.}
\end{tikzpicture}
\hspace{-0.6cm}
\begin{tikzpicture}
	\card
	\cardstrip
	\cardbanner{banner/white.png}
	\cardicon{icons/coin.png}
	\cardprice{4}
	\cardtitle{Denkmal}
	\cardcontent{Für diese Karte werden die Punkte-Marker benötigt (siehe neue Regeln).}
\end{tikzpicture}
\hspace{-0.6cm}
\begin{tikzpicture}
	\card
	\cardstrip
	\cardbanner{banner/white.png}
	\cardicon{icons/coin.png}
	\cardprice{5}
	\cardtitle{Gesindel}
	\cardcontent{Du ziehst zunächst 3 Karten nach. Dann muss, beginnend mit dem Spieler links von dir, jeder Mitspieler 3 Karten von seinem Nachziehstapel aufdecken. Wenn er (auch nach dem eventuell nötigen Mischen seines Ablagestapels) nur weniger als 3 Karten aufdecken kann, deckt er nur so viele auf, wie möglich. Dann legt er alle auf diese Weise aufgedeckten Aktions- und Geldkarten auf seinen Ablagestapel. Karten mit kombinierten Kartentypen (z. B. Harem, Dominion – Die Intrige) sind davon auch betroffen. Die übrigen der gerade aufgedeckten Karten legt er in beliebiger Reihenfolge zurück auf seinen Nachziehstapel.}
\end{tikzpicture}
\hspace{-0.6cm}
\begin{tikzpicture}
	\card
	\cardstrip
	\cardbanner{banner/white.png}
	\cardicon{icons/coin.png}
	\cardprice{5}
	\cardtitle{Gewölbe}
	\cardcontent{Zuerst ziehst du 2 Karten nach. Dann legst du \enquote{eine beliebige Anzahl} Handkarten ab. Du kannst auch 0 Karten ablegen. Du kannst auch Karten ablegen, die du gerade nachgezogen hast. Beginnend mit dem Spieler links von dir, entscheidet sich dann jeder Mitspieler, ob er Handkarten ablegen möchte oder nicht. Wenn er sich dafür entscheidet Karten abzulegen, muss er genau 2 Karten ablegen und zieht dann eine Karte nach. Der Spieler kann sich auch für das Ablegen entscheiden, wenn er nur 1 Karte auf der Hand hat. In diesem Fall legt er diese ab, zieht dann allerdings keine Karte nach. Der Spieler darf sich nicht dafür entscheiden, nur 1 Karte abzulegen, wenn er 2 oder mehr Karten auf der Hand hat.}
\end{tikzpicture}
\hspace{-0.6cm}
\begin{tikzpicture}
	\card
	\cardstrip
	\cardbanner{banner/white.png}
	\cardicon{icons/coin.png}
	\cardprice{6}
	\cardtitle{\scriptsize{Großer Markt}}
	\cardcontent{Du musst nicht alle Geldkarten aus deiner Hand auslegen. Kupfer, die du auf der Hand hast, verbieten nicht, den Großen Markt zu kaufen. Kupfer, die in diesem Zug im Spiel waren, es nun aber nicht mehr sind, verbieten dir auch nicht, den Großen Markt zu kaufen. Hast zu z. B. 2 Käufe und 11 Kupfer auslegen und kaufst zuerst den Münzer, so entsorgst du nach diesem Kauf alle Geldkarten im Spiel. \emph{Nun hast du keine Kupfer mehr im Spiel und kannst den Großen Markt kaufen} (siehe auch: Münzer). Wenn du den Großen Markt auf eine andere Art nimmst, z. B. durch den Ausbau, so verhindern Kupfer, die im Spiel sind, dies auch nicht. Andere Geldkarten als Kupfer verbieten dir nicht, den Großen Markt zu kaufen, selbst wenn diese auch \coin[1] wert sind, wie z. B. Lohn.}
\end{tikzpicture}
\hspace{-0.6cm}
\begin{tikzpicture}
	\card
	\cardstrip
	\cardbanner{banner/white.png}
	\cardicon{icons/coin.png}
	\cardprice{6}
	\cardtitle{\scriptsize{Halsabschneider}}
	\cardcontent{Für diese Karte werden die Punkte-Marker benötigt (siehe neue Regeln). Zunächst erhältst du +1 Kauf und +\coin[2]. Dann müssen deine Mitspieler reihum ihre Handkarten auf 3 reduzieren. Für jede Karte, die du in dieser Runde kaufst, nimmst du dir einen Punkte-Marker und legst diesen auf dein Tableau. Hast du mehrere Handlanger im Spiel, erhältst du für jeden Handlanger einen Punkte-Marker pro gekaufter Karte. Hast du jedoch den Handlanger auf den Königshof gespielt, erhältst du dafür nur einen Punkte-Marker, weil die Karte nur einmal im Spiel ist.}
\end{tikzpicture}
\hspace{-0.6cm}
\begin{tikzpicture}
	\card
	\cardstrip
	\cardbanner{banner/white.png}
	\cardicon{icons/coin.png}
	\cardprice{3}
	\cardtitle{\footnotesize{Handelsroute}}
	\cardcontent{\tiny{\begin{Spacing}{1}
	\vspace{1em}
	Du erhältst einen zusätzlichen Kauf in deiner Kaufphase und +\coin[1] für jeden Geld-Marker, der auf dem Tableau Handelsroute liegt, wenn du die Karte auslegst. Dann musst du eine Karte aus deiner Hand entsorgen (wenn du noch eine Karte auf der Hand hast). 
 
	\emph{Spielvorbereitung:} Wenn die Handelsroute im Spiel verwendet wird (entweder als eine der 10 Königreichkarten oder im Schwarzmarktstapel), wird bei der Spielvorbereitung das Tableau Handelsroute bereit gelegt. Zusätzlich wird auf jeden Punktekartenstapel im gesamten Vorrat ein Geld-Marker gelegt. Punktekarten sind: die Basiskarten: Anwesen, Herzogtum, Provinz und Kolonie, die Königreichkarten: z. B. Gärten (Dominion – Basisspiel) und Herzog (Dominion – Die Intrige), kombinierte Karten: z. B. Harem und Adelige (Dominion – Die Intrige). Punktekarten, die nicht im Vorrat sind, werden nicht beachtet, z. B. aus dem Schwarzmarktstapel (Promokarte Schwarzmarkt). Auf den Ritter-Stapel (Dominion – Dark Ages) wird kein Marker gelegt, auch wenn die oberste Karte eine Punktekarte ist.
 
	\emph{Tableau Handelsroute:} Wird im Spielverlauf jeweils die erste Karte eines Punktekartenstapels genommen oder gekauft, legt der Spieler den Geld-Marker von diesem Stapel auf das Tableau Handelsroute. Legt ein Spieler die Handelsroute aus, so erhält er +1 virtuelles Geld für jeden Geld-Marker, der zu diesem Zeitpunkt auf dem Tableau liegt. Dabei ist egal, welcher Spieler den Geld-Marker auf das Tableau gelegt hat. Es wird nur beim Kaufen oder Nehmen der ersten Karte jedes Punktekartenstapels ein Geld-Marker auf das Tableau Handelsroute gelegt, also nur wenn der Geld-Marker, der bei Spielaufbau auf den Stapel gelegt wurde, noch dort liegt. Die Geld-Marker bleiben für den Rest des Spiels auf dem Tableau. Geld-Marker werden nicht vom Tableau entfernt und während des Spiels werden keine neuen Geld-Marker auf die Punktekartenstapel gelegt. Z. B. wenn ein Punktekartenstapel im Vorrat wieder durch den Botschafter (Dominion – Seaside) aufgefüllt wird.
	\end{Spacing}}}
\end{tikzpicture}
\hspace{-0.6cm}
\begin{tikzpicture}
	\card
	\cardstrip
	\cardbanner{banner/white.png}
	\cardicon{icons/coin.png}
	\cardprice{8*}
	\cardtitle{Hausierer}
	\cardcontent{Normalerweise kostet diese Karte \coin[8]. In der Kaufphase kostet der Hausierer pro Aktionskarte, die du selbst im Spiel hast, \coin[2] weniger. Das betrifft auch Hausierer auf der Hand oder in den Stapeln aller Spieler. Die Kosten sinken niemals unter \coin[0]. Wenn du z. B. das Arbeiterdorf auf den Königshof spielst, hast du 2 Aktionskarten im Spiel, obwohl du das Arbeiterdorf dreimal ausgespielt hast. Wenn du den Hausierer ausserhalb der üblichen Kaufphase (z. B. durch den Schwarzmarkt) kaufst oder nimmst, kostet der Hausierer \coin[8].}
\end{tikzpicture}
\hspace{-0.6cm}
\begin{tikzpicture}
	\card
	\cardstrip
	\cardbanner{banner/gold.png}
	\cardicon{icons/coin.png}
	\cardprice{6}
	\cardtitle{Hort}
	\cardcontent{Diese Geldkarte hat einen Wert von \coin[2], wie ein Silber. Wenn diese Karte im Spiel ist, und du eine Punktekarte kaufst, nimmst du dir zusätzlich ein Gold und legst es auf deinen Ablagestapel. Wenn kein Gold mehr im Vorrat ist, nimmst du nichts. Wenn du mehrere Karten Hort im Spiel hast, erhältst du mehrere Gold für den Kauf einer Punktekarte. Wenn du z. B. +1 Kauf hast und 2 Karten Hort auslegst, könntest du 2 Anwesen kaufen und dazu 4 Gold bekommen. Auch kombinierte Karten, wie z. B. Adlige und Harem (Dominion – Die Intrige) sind Punktekarten. Du nimmst dir ein Gold, auch wenn du z. B. den Wachturm vorzeigst und die gekaufte Punktekarte sofort entsorgst. Du nimmst dir nur ein Gold, wenn du eine Punktekarte tatsächlich kaufst, wenn du sie auf eine andere Weise nimmst, bekommst du kein Gold.}
\end{tikzpicture}
\hspace{-0.6cm}
\begin{tikzpicture}
	\card
	\cardstrip
	\cardbanner{banner/gold.png}
	\cardicon{icons/coin.png}
	\cardprice{5}
	\cardtitle{\tiny{Königliches Siegel}}
	\cardcontent{Diese Geldkarte hat den Wert \coin[2], wie ein Silber. Wenn du mehrere Karten nimmst oder kaufst, kannst du für jede Karte getrennt entscheiden, ob du sie auf den Ablagestapel oder auf den Nachziehstapel legst. Ist dein Nachziehstapel leer und du entscheidest dich, eine Karte auf den Nachziehstapel zu legen, so wird dies die einzige Karte deines Nachziehstapels. Karten, die du durch die Besessenheit (Dominion – Die Alchemisten) im Zug eines anderen Spielers erhältst, darfst du nicht auf den Nachziehstapel legen, da das Königliche Siegel nur für den besessenen Spieler gilt.}
\end{tikzpicture}
\hspace{-0.6cm}
\begin{tikzpicture}
	\card
	\cardstrip
	\cardbanner{banner/white.png}
	\cardicon{icons/coin.png}
	\cardprice{7}
	\cardtitle{Königshof}
	\cardcontent{\emph{Errata:} Der Text auf der Karte sollte heißen: \enquote{Du darfst eine Aktionskarte aus deiner Hand wählen. Spiele diese Aktionskarte dreimal aus.} 

	\smallskip
	
	Diese Karte funktioniert ähnlich wie der Thronsaal (Dominion – Basisspiel), mit dem Unterschied, dass du die gewählte Karte 3mal spielst. Du wählst also eine Aktionskarte aus deiner Hand, legst sie aus, führst sie komplett aus, nimmst sie zurück auf die Hand, legst sie erneut aus, führst die Anweisungen nochmals aus, nimmst sie wieder zurück auf die Hand und legst sie dann ein drittes Mal aus. Dieses dreimalige Auslegen verbraucht keine Aktionen. (Das Auslegen des Königshofs verbraucht eine Aktion.) Du darfst keine anderen Aktionskarten auslegen, bis du die gewählte Karte 3mal ausgelegt und ausgeführt hast, ausser die Anweisung auf der gewählte Karte erlaubt es explizit, wie es z. B. der Königshof selbst tut. Wenn du eine Karte wählst, die +1 Aktion gibt, hast du am Ende +3 Aktionen. Legst du einen Königshof auf einen anderen Königshof aus, so wählst du nacheinander 3 Karten aus und legst jede gewählte Karte 3mal aus. Du legst also nicht eine Karte 9mal aus.}
\end{tikzpicture}
\hspace{-0.6cm}
\begin{tikzpicture}
	\card
	\cardstrip
	\cardbanner{banner/white.png}
	\cardicon{icons/coin.png}
	\cardprice{7}
	\cardtitle{\footnotesize{Kunstschmiede}}
	\cardcontent{Du darfst eine beliebige Anzahl Karten aus deiner Hand entsorgen. Das bedeutet, du darfst auch 0 Karten entsorgen. Dafür musst du dir eine Karte nehmen, die 0 Geld kostet. Dies Anweisung unterscheidet sich von Karten wie z. B. Ausbau, weil hier die Summe der Kosten aller entsorgten Karten betrachtet wird, nicht die Anzahl oder der Wert der einzelnen Karten. Ist im Vorrat keine Karte, die exakt soviel kostet, wie die entsorgten Karten, darfst du dir keine Karte nehmen. Es werden dabei nur die Geld-Kosten beachtet. Trank-Kosten (Dominion – Die Alchemisten) werden ignoriert. Du zählst weder Trank zur Summe, noch darfst du eine Karte mit Trank-Kosten nehmen.}
\end{tikzpicture}
\hspace{-0.6cm}
\begin{tikzpicture}
	\card
	\cardstrip
	\cardbanner{banner/white.png}
	\cardicon{icons/coin.png}
	\cardprice{5}
	\cardtitle{Leihhaus}
	\cardcontent{Diese Karte erlaubt dir, deinen Ablagestapel durchzusehen (was normalerweise nicht erlaubt ist). Du darfst deinen Ablagestapel nur durchsehen, wenn du das Leihhaus gerade ausgelegt hast. Du musst deinen Mitspielern nicht die gesamten Karten deines Ablagestapels zeigen, nur die Kupfer, die du auf die Hand nimmst. Nachdem du die Kupfer auf die Hand genommen hast, legst du die übrigen Karten in beliebiger Reihenfolge auf den Ablagestapel zurück.}
\end{tikzpicture}
\hspace{-0.6cm}
\begin{tikzpicture}
	\card
	\cardstrip
	\cardbanner{banner/gold.png}
	\cardicon{icons/coin.png}
	\cardprice{3}
	\cardtitle{Lohn}
	\cardcontent{(Wir haben bei der Übersetzung von \enquote{Loan} aus dem englischen Original bewusst \enquote{Lohn} gewählt, da uns dieser Name als passender erscheint, als eine sprachlich korrekte Übersetzung.) Diese Geldkarte hat einen Wert von \coin[1], wie Kupfer. Wenn du sie auslegst, deckst du solange Karten von deinem Nachziehstapel auf, bis du eine Geldkarte aufgedeckt hast. Dann entscheidest du dich, ob du die aufgedeckte Geldkarte entsorgst oder ablegst. Danach legst du alle übrigen aufgedeckten Karten ab. Wenn du (auch nach dem Mischen deines Ablagestapels) keine Karten mehr aufdecken kannst, legst du alle aufgedeckten Karten ab. Beachte, dass du deine Geldkarten in beliebiger Reihenfolge auslegen kannst und nicht alle Geldkarten aus deiner Hand auslegen musst.}
\end{tikzpicture}
\hspace{-0.6cm}
\begin{tikzpicture}
	\card
	\cardstrip
	\cardbanner{banner/white.png}
	\cardicon{icons/coin.png}
	\cardprice{5}
	\cardtitle{Münzer}
	\cardcontent{Wenn du diese Karte kaufst, entsorgst du alle Geldkarten, die zu diesem Zeitpunkt im Spiel sind und nur diese (nicht aus deiner Hand oder sonst woher). Bedenke, dass du nicht alle Geldkarten auslegen musst. Wenn du in dieser Runde mehrere Karten kaufst, entsorgst du die Geldkarten im Spiel, direkt nachdem du den Münzer gekauft hast. Geldkarten, die du in dieser Runde ausgelegt hast, haben aber bereits Geld \enquote{produziert}, auch wenn du sie entsorgst. Du kannst also für den gesamten Wert der ausgelegten Geldkarten neue Karten kaufen. Du kannst allerdings keine weiteren Geldkarten mehr auslegen, sobald du eine Karte gekauft hast. Wenn du den Münzer auslegst, darfst du eine Geldkarte aus deiner Hand aufdecken. Dann nimmst du dir sofort eine identische Karte aus dem Vorrat und legst diese auf deinen Ablagestapel. Die aufgedeckte Geldkarte nimmst du zurück auf die Hand. Die aufgedeckte Geldkarte kann auch eine kombinierte Karte sein, z. B. Harem (Dominion – Die Intrige). Wenn du den Münzer kaufst und den Wachturm aus deiner Hand aufdeckst, darfst du den Münzer sofort auf deinen Nachziehstapel legen. Die ausgelegten Geldkarten müssen jedoch entsorgt werden. Wenn du den Münzer kaufst, während das Königliche Siegel im Spiel ist, wird das Königliche Siegel entsorgt, bevor du den Münzer zurück auf deinen Nachziehstapel legen dürftest.}
\end{tikzpicture}
\hspace{-0.6cm}
\begin{tikzpicture}
	\card
	\cardstrip
	\cardbanner{banner/white.png}
	\cardicon{icons/coin.png}
	\cardprice{5}
	\cardtitle{Quacksalber}
	\cardcontent{Beginnend mit dem Spieler links von dir muss jeder Mitspieler entscheiden, ob er einen Fluch aus seiner Hand ablegt oder sich ein Kupfer und einen Fluch vom Vorrat nimmt und auf seinen Ablagestapel legt. Er darf sich auch für die zweite Möglichkeit entscheiden, wenn einer oder beide Stapel leer sind. In diesem Fall nimmt er sich nur eine der noch vorhandenen Karten oder, wenn beide Stapel leer sind, keine Karte. Deckt ein Spieler einen Burggraben (Dominion – Basisspiel) aus seiner Hand auf, darf er weder eine Karte nehmen noch eine Karte ablegen. Er kann nicht nur einen Teil des Angriffs abwehren. Entscheidet sich ein Spieler für die zweite Möglichkeit und deckt einen Wachturm aus seiner Hand auf, darf er sofort eine oder beide Karten entsorgen.}
\end{tikzpicture}
\hspace{-0.6cm}
\begin{tikzpicture}
	\card
	\cardstrip
	\cardbanner{banner/gold.png}
	\cardicon{icons/coin.png}
	\cardprice{5}
	\cardtitle{\scriptsize{Schmuggelware}}
	\cardcontent{Diese Geldkarte hat einen Wert von \coin[3], wie ein Gold. Wenn du sie ausspielst, erhältst du zunächst +1 Kauf. Dann benennt der Spieler links von dir eine Karte, die du in dieser Runde nicht kaufen darfst. Er kann auch eine Karte benennen, die nicht im Vorrat ist, aber z. B. im Schwarzmarkt-Stapel. Wenn du mehrere Karten Schmuggelware auslegst, benennt der Spieler links von dir jedesmal eine Karte. Du darfst in diesem Zug keine der benannten Karten kaufen. Hierfür ist es wichtig, dass du Geldkarten in beliebiger Reihenfolge auslegen kannst. Du kannst also z. B. zuerst eine Schmuggelware auslegen, dann benennt der Spieler eine Karte, danach kannst du weitere Geldkarten auslegen. Die Anzahl der Karten, die ein Spieler noch auf der Hand hält, ist für die übrigen Spieler sichtbar. Du darfst die benannte Karte nicht kaufen, du darfst sie aber nehmen, wenn dir dies eine Anweisung einer anderen Karte erlaubt, z. B. Hort. Beachte, dass du in dieser Runde keine weiteren Geldkarten mehr auslegen darfst, sobald du eine Karte gekauft hast.}
\end{tikzpicture}
\hspace{-0.6cm}
\begin{tikzpicture}
	\card
	\cardstrip
	\cardbanner{banner/white.png}
	\cardicon{icons/coin.png}
	\cardprice{5}
	\cardtitle{Stadt}
	\cardcontent{\emph{Errata:} Auf der Karte Stadt hat sich ein Tippfehler eingeschlichen. Statt \enquote{1+ Kauf} sollte es heißen \enquote{+1 Kauf}.

	\smallskip
	
	Du ziehst zuerst eine Karte nach und erhältst +2 Aktionen. Wenn mindestens ein Stapel im Vorrat leer ist, ziehst du eine weitere Karte nach. Wenn genau 1 Stapel im Vorrat leer ist, erhältst du nur +1 Karte. Oder wenn 2 oder mehr Stapel im Vorrat leer sind, erhältst du +1 Karte, +\coin[1] und +1 Kauf. Es gibt keine weiteren Boni, wenn 3 oder mehr Stapel im Vorrat leer sind. Die jeweiligen Bedingungen müssen beim Ausspielen der Karte erfüllt sein. 

	\smallskip
	 
	Der Effekt der Karte wird nicht rückwirkend verändert, wenn ein Stapel im Vorrat später leer wird (z. B. durch den Ausbau) oder auch wieder aufgefüllt wird (z. B. durch den Botschafter, Dominion-Seaside). 

	\smallskip
	
	Der Müllstapel ist nicht Teil des Vorrats und wird somit nicht beachtet. Sind beim Ausspielen der Karte also z. B. 2 Stapel leer, so erhältst du insgesamt +2 Karten, +2 Aktionen, +\coin[1] und +1 Kauf.}
\end{tikzpicture}
\hspace{-0.6cm}
\begin{tikzpicture}
	\card
	\cardstrip
	\cardbanner{banner/gold.png}
	\cardicon{icons/coin.png}
	\cardprice{4}
	\cardtitle{Steinbruch}
	\cardcontent{Diese Geldkarte hat der Wert 1 Geld, wie ein Kupfer. Wenn diese Karte im Spiel ist, kosten Aktionskarten 2 Geld weniger. Dieser Effekt ist kumulativ. Wenn du z. B. in der Kaufphase 2 Steinbrüche ausspielst, kosten Aktionskarten um 4 Geld weniger. Der Effekt kann auch durch andere Karten ergänzt werde. Spielst du z. B. in der Aktionsphase ein Arbeiterdorf und in der Kaufphase 2 Steinbrüche, kostet der Hausierer nur noch 2 Geld. Handkarten, Karten im Nachziehstapel und im Ablagestapel sind auch durch diesen Effekt betroffen. Kombinierte Karten, wie z. B. Adelige (Dominion – Die Intrige) sind auch Aktionskarten.}
\end{tikzpicture}
\hspace{-0.6cm}
\begin{tikzpicture}
	\card
	\cardstrip
	\cardbanner{banner/gold.png}
	\cardicon{icons/coin.png}
	\cardprice{4}
	\cardtitle{Talisman}
	\cardcontent{Diese Geldkarte hat den Wert 1 Geld, wie ein Kupfer. Wenn diese Karte im Spiel ist und du eine Karte kaufst, die 4 Geld oder weniger kostet und die keine Punktekarte ist, nimmst du dir zusätzlich eine weitere identische Karte. Du nimmst diese Karte vom Vorrat und legst sie auf deinen Ablagestapel. Gibt es keine weitere Karte mit diesem Namen im Vorrat, nimmst du dir nichts. Hast du mehrere Talismane im Spiel, nimmst du dir für jeden Talisman eine zusätzliche Karte. Wenn du mehrere Karten kaufst, auf diese Bedingungen zutreffen (4 Geld oder weniger, keine Punktekarte), nimmst du dir von jeder eine weitere Karte. Wenn du z. B. 2 Talismane, 4 Kupfer und 2 Käufe hast und dir ein Silber und eine Handelsroute kaufst, nimmst du dir zusätzlich 2 weitere Silber und 2 weitere Handelsrouten. Der Talisman wirkt nur für Karten, die du kaufst. Nimmst du dir eine Karte auf eine andere Weise, z. B. durch den Ausbau, erhältst du keine weitere. Kombinierte Karten, wie z. B. die Große Halle (Dominion – Die Intrige) sind auch Punktekarten. Bei den Kosten von 4 Geld oder weniger werden die aktuell geltenden Kosten betrachtet. Das sind nicht unbedingt die aufgedruckten Kosten. Wenn du z. B. in deiner Aktionsphase 2 Aktionskarten ausgelegt hast, kostet der Hausierer nur noch 4 Geld und du darfst dir somit einen weiteren Hausierer nehmen.}
\end{tikzpicture}
\hspace{-0.6cm}
\begin{tikzpicture}
	\card
	\cardstrip
	\cardbanner{banner/blue.png}
	\cardicon{icons/coin.png}
	\cardprice{3}
	\cardtitle{Wachturm}
	\cardcontent{\tiny{\begin{Spacing}{1}
	Wenn du diese Karte in deinem Zug auslegst, ziehst du solange Karten von deinem Nachziehstapel, bis du 6 Karten auf der Hand hast. Hast du nach dem Ausspielen des Wachturmes bereits 6 oder mehr Karten auf der Hand, ziehst du keine Karte nach. Immer wenn du eine Karte nimmst oder kaufst, egal ob in deinem eigenen Zug oder im Zug eines Mitspielers, darfst du den Wachturm aus deinem Hand aufdecken und dann entscheiden, ob du die neue Karte entsorgst oder oben auf deinen Nachziehstapel legst. Du darfst den Wachturm jedesmal, wenn du eine Karte nimmst oder kaufst aus deiner Hand aufdecken. Wie üblich bei Reaktionskarten, deckst du den Wachturm nur auf und nimmst ihn dann zurück auf deine Hand. Spielt ein Mitspieler z. B. den Quacksalber, kannst du den Wachturm nur bei einer oder bei beiden Karten (Kupfer und Fluch) aufdecken und für jede Karte getrennt entscheiden. Du kannst mit dem Wachturm auch nacheinander auf mehrere Angriffe unterschiedlicher Mitspieler reagieren und ihn dann in deinem eigenen Zug nochmals einsetzen. Auch wenn du dich dafür entscheidest eine Karte, die du gerade genommen hast oder nehmen musstest, zu entsorgen, musst du diese Karte zuerst nehmen. Die Karte ist also nicht mehr im Vorrat, und andere Karten, die auf genommene Karten Bezug nehmen, wie z. B. die Schmuggler (Dominion – Seaside), können auch darauf angewandt werden. Wenn während des Extrazuges durch die Besessenheit (Dominion – Die Alchemisten) Karten genommen oder gekauft werden, kannst du den Wachturm nicht aus deiner Hand aufdecken, da der besessene Spieler die Karte nimmt. Du kannst den Wachturm auch aus deiner Hand aufdecken, wenn du eine Karte nimmst, die du nicht wie üblich auf deinen Ablagestapel legst, wie z. B. die Mine (Dominion – Basisspiel).
	\end{Spacing}}}
\end{tikzpicture}
\hspace{-0.6cm}
\begin{tikzpicture}
	\card
	\cardstrip
	\cardbanner{banner/gold.png}
	\cardicon{icons/coin.png}
	\cardprice{9}
	\cardtitle{Platin}
	\cardcontent{}
\end{tikzpicture}
\hspace{-0.6cm}
	\begin{tikzpicture}
	\card
	\cardstrip
	\cardbanner{banner/green.png}
	\cardicon{icons/coin.png}
	\cardprice{11}
	\cardtitle{Kolonien}
	\cardcontent{}
\end{tikzpicture}
\hspace{-0.6cm}
\begin{tikzpicture}
	\card
	\cardstrip
	\cardbanner{banner/white.png}
	\cardtitle{\scriptsize{Empfohlene 10er Sätze\qquad}}
	\cardcontent{\emph{Blütezeit und Basisspiel:}

	\smallskip 
	
	\emph{Haufenweise Geld:} \\ 
	Abenteuer, Bank, Großer Markt, Königliches Siegel, Münzer, Abenteurer, Geldverleiher, Laboratorium, Mine, Spion 

	\smallskip 
	
	\emph{Die Armee des Königs:} \\ 
	Ausbau, Gesindel, Gewölbe, Handlanger, Königshof, Bürokrat, Burggraben, Dorf, Ratsversammlung, Spion 

	\smallskip 
	
	\emph{Ein gutes Leben:} \\ 
	Denkmal, Hort, Leihhaus, Quacksalber, Schmuggelware, Bürokrat, Dorf, Gärten, Kanzler, Keller}
\end{tikzpicture}
\hspace{-0.6cm}
\begin{tikzpicture}
	\card
	\cardstrip
	\cardbanner{banner/white.png}
	\cardtitle{\scriptsize{Empfohlene 10er Sätze\qquad}}
	\cardcontent{\emph{Blütezeit und Die Intrige:}

	\smallskip 
	
	\emph{Pfade zum Sieg:} \\ 
	Bischof, Denkmal, Halsabschneider, Hausierer, Leihhaus, Anbau, Armenviertel, Baron, Handlanger, Harem
	
	\smallskip 
	
	\emph{All along the watchtower:} \\ 
	Gewölbe, Handelsroute, Hort, Talisman, Wachturm, Bergwerk, Brücke, Große Halle, Handlanger, Kerkermeister

	\smallskip 
	
	\emph{Glücksritter:} \\ 
	Ausbau, Bank, Gewölbe, Königshof, Kunstschmiede, Brücke, Kupferschmied, Tribut, Trickser, Wunschbrunnen}
\end{tikzpicture}
\hspace{-0.6cm}
\begin{tikzpicture}
	\card
	\cardstrip
	\cardbanner{banner/white.png}
	\cardtitle{\scriptsize{Empfohlene 10er Sätze\qquad}}
	\cardcontent{\emph{Blütezeit:}

	\smallskip 
	
	\emph{Anfänger:} \\ 
	Abenteuer, Arbeiterdorf, Ausbau, Bank, Denkmal, Gesindel, Halsabschneider, Königliches Siegel, Leihhaus, Wachturm 

	\smallskip 
	
	\emph{Freundliche Interaktion:} \\ 
	Arbeiterdorf, Bischof, Gewölbe, Handelsroute, Hausierer, Hort, Königliches Siegel, Kunstschmiede, Schmuggelware, Stadt 

	\smallskip 
	
	\emph{Große Aktionen:} \\ 
	Ausbau, Gesindel, Gewölbe, Großer Markt, Königshof, Lohn, Münzer, Stadt, Steinbruch, Talisman}
\end{tikzpicture}
\hspace{0.6cm}

	    % Basic settings for this card set
\renewcommand{\cardcolor}{prosperity}
\renewcommand{\cardextension}{Erweiterung III}
\renewcommand{\cardextensiontitle}{Blütezeit}
\renewcommand{\seticon}{prosperity.png}

\clearpage
\newpage
\section{\cardextension \ - \cardextensiontitle \ (Rio Grande Games 2016)}

\begin{tikzpicture}
	\card
	\cardstrip
	\cardbanner{banner/white.png}
	\cardicon{icons/coin.png}
	\cardprice{3}
	\cardtitle{\footnotesize{Handelsroute}}
	\cardcontent{Du erhältst + 1 Kauf und +\coin[1] pro Geldmarker, der sich zum Zeitpunkt des Ausspielens auf dem Handelsrouten-Tableau befindet. Du musst eine Handkarte entsorgen, wenn du mindestens 1 Karte auf der Hand hast.

	\medskip

	In allen Spielen mit der \emph{HANDELSROUTE} (auch als Teil des \emph{SCHWARZMARKTES}) wird zu Beginn des Spiels das Handelsrouten-Tableau neben dem Vorrat bereit gelegt. Außerdem wird auf jeden Punkte-Vorratsstapel (\emph{ANWESEN}, \emph{HERZOGTUM}, \emph{PROVINZ}, ggf. \emph{KOLONIE} sowie alle kombinierten oder reinen Punktekarten unter den Königreichkarten (z.B. \emph{GÄRTEN} aus Basisspiel oder \emph{INSEL} aus Seaside)) ein Geldmarker gelegt. Sobald die erste Karte eines Stapels genommen wird (egal von welchem Spieler und egal auf welche Weise), legt ihr den entsprechenden Geldmarker auf das Handelsrouten-Tableau. Es wird kein neuer Marker auf den Vorratsstapel gelegt. Auch wird unter keinen Umständen ein Marker vom Handelsrouten-Tableau entfernt.}
\end{tikzpicture}
\hspace{-0.6cm}
\begin{tikzpicture}
	\card
	\cardstrip
	\cardbanner{banner/gold.png}
	\cardicon{icons/coin.png}
	\cardprice{3}
	\cardtitle{Lohn}
	\cardcontent{Diese Karte ist eine Geldkarte mit zusätzlichen Anweisungen. Sie hat den Wert \coin[1].

	\medskip

	Decke solange Karten von deinem Nachziehstapel auf, bis du die erste Geld- oder kombinierte Geldkarte aufdeckst. Entsorge die aufgedeckte Geldkarte oder lege sie ab. Alle anderen aufgedeckten Karten legst du ab.}
\end{tikzpicture}
\hspace{-0.6cm}
\begin{tikzpicture}
	\card
	\cardstrip
	\cardbanner{banner/blue.png}
	\cardicon{icons/coin.png}
	\cardprice{3}
	\cardtitle{Wachturm}
	\cardcontent{Diese Karte ist eine kombinierte Aktions- und Reaktionskarte. Sie kann in der Aktionsphase ausgespielt werden (Anweisung über der Trennlinie) oder als Reaktion auf die unter der Trennlinie angegebene Situation.

	\medskip

	Spielst du den \emph{WACHTURM} in deiner Aktionsphase aus, ziehst du solange Karten nach, bis du 6 Karten auf der Hand hast. Hast du bereits 6 oder mehr Handkarten, ziehst du keine Karten nach. 

	\medskip

	Wenn du den \emph{WACHTURM} auf der Hand hast und du eine Karte nimmst (in deinem eigenen Zug oder während des Zuges eines Mitspielers), darfst du ihn als Reaktion aufdecken und die genommene Karte entweder entsorgen oder auf deinen Nachziehstapel legen. Anschließend nimmst du den \emph{WACHTURM} wieder auf die Hand. Nimmst du anschließend eine oder mehrere weitere Karten (durch die gleiche oder eine andere Anweisung bzw. durch einen Kauf), kannst du den \emph{WACHTURM} erneut aufdecken – solange du ihn auf der Hand hast. Hast du den \emph{WACHTURM} in deiner nächsten Aktionsphase noch immer auf der Hand, darfst du ihn ausspielen. 

	\medskip

	Wenn ein Mitspieler gegen dich die \emph{BESESSENHEIT} (aus \emph{Alchemie}) gespielt hat und du den Extrazug ausführst, darfst du den \emph{WACHTURM} nicht aufdecken, da der Mitspieler die Karten nimmt und nicht du.}
\end{tikzpicture}
\hspace{-0.6cm}
\begin{tikzpicture}
	\card
	\cardstrip
	\cardbanner{banner/white.png}
	\cardicon{icons/coin.png}
	\cardprice{4}
	\cardtitle{Arbeiterdorf}
	\cardcontent{Du \emph{musst} eine Karte ziehen, \emph{darfst} 2 weitere Aktionen ausführen und in der Kaufphase einen zusätzlichen Kauf durchführen.}
\end{tikzpicture}
\hspace{-0.6cm}
\begin{tikzpicture}
	\card
	\cardstrip
	\cardbanner{banner/white.png}
	\cardicon{icons/coin.png}
	\cardprice{4}
	\cardtitle{Bischof}
	\cardcontent{Du erhältst +\coin[1] und legst einen \victorypointtoken-Marker auf dein Spieler-Tableau. Dann musst du eine Handkarte entsorgen, wenn du mindestens 1 Karte auf der Hand hast. Lege halb so viele \victorypointtoken-Marker auf dein Spieler-Tableau, wie die entsorgte Karte gekostet hat. Ungerade Kosten werden abgerundet. Die \potion-Kosten (z.B. aus \emph{Alchemie}) spielen keine Rolle. So erhältst du je 2 \victorypointtoken-Marker für ein entsorgtes \emph{ARBEITERDORF} (Kosten: \coin[4]), ebenso wie für ein \emph{GESINDEL} (Kosten: \coin[5]) oder einen \emph{GOLEM} (Kosten: \coin[4] und \potion). 
	\
	\medskip

	Jeder Mitspieler darf eine Handkarte entsorgen, erhält dafür aber keine Siegpunktmarker.}
\end{tikzpicture}
\hspace{-0.6cm}
\begin{tikzpicture}
	\card
	\cardstrip
	\cardbanner{banner/white.png}
	\cardicon{icons/coin.png}
	\cardprice{4}
	\cardtitle{Denkmal}
	\cardcontent{Du erhältst +\coin[2] und legst einen \victorypointtoken-Marker aus dem Vorrat auf dein Spieler-Tableau.}
\end{tikzpicture}
\hspace{-0.6cm}
\begin{tikzpicture}
	\card
	\cardstrip
	\cardbanner{banner/gold.png}
	\cardicon{icons/coin.png}
	\cardprice{4}
	\cardtitle{Steinbruch}
	\cardcontent{Diese Karte ist eine Geldkarte mit zusätzlichen Anweisungen. Sie hat den Wert \coin[1].

	\medskip

	Solange diese Karte im Spiel ist, kosten alle Aktionskarten (auch kombinierte) \coin[2] weniger. Dies betrifft alle Aktionskarten, d.h. auch Handkarten, Karten in den Ablage- und Nachziehstapeln etc. Der Effekt ist kumulativ, d.h. mit einem zweiten \emph{STEINBRUCH} oder anderen Aktionskarten, die die Kosten von Karten reduzieren, können die Kosten weiter gesenkt werden.}
\end{tikzpicture}
\hspace{-0.6cm}
\begin{tikzpicture}
	\card
	\cardstrip
	\cardbanner{banner/gold.png}
	\cardicon{icons/coin.png}
	\cardprice{4}
	\cardtitle{Talisman}
	\cardcontent{Diese Karte ist eine Geldkarte mit zusätzlichen Anweisungen. Sie hat den  Wert \coin[1].

	\medskip

	Solange diese Karte im Spiel ist und du eine Nicht-Punktekarte kaufst (nicht wenn du sie auf andere Weise nimmst), die zu diesem Zeitpunkt maximal \coin[4] kostet, nimmst du dir eine weitere gleiche Karte vom Vorrat. Ist keine gleiche Karte im Vorrat, nimmst du dir keine weitere Karte. Kaufst du in einem Zug mehrere Karten, die maximal \coin[4] kosten, wendest du den Effekt des \emph{TALISMANS} auf alle diese Karten an.}
\end{tikzpicture}
\hspace{-0.6cm}
\begin{tikzpicture}
	\card
	\cardstrip
	\cardbanner{banner/gold.png}
	\cardicon{icons/coin.png}
	\cardprice{5}
	\cardtitle{Abenteuer}
	\cardcontent{Diese Karte ist eine Geldkarte mit zusätzlichen Anweisungen. Sie hat den Wert \coin[1]. 

	\medskip

	Sobald du diese Karte ausspielst (normalerweise in der Kaufphase), deckst du solange Karten vom Nachziehstapel auf, bis du die erste Geldkarte (auch kombinierte) aufdeckst. Ist der Nachziehstapel aufgebraucht, ohne eine Geldkarte zu finden, mischst du deinen Ablagestapel. Findest du auch dort keine Geldkarte, legst du alle aufgedeckten Karten ab. Spiele die erste aufgedeckte Geldkarte sofort aus und führe ggf. zusätzliche Anweisungen auf dieser Karte aus. Lege alle anderen aufgedeckten Karten ab.}
\end{tikzpicture}
\hspace{-0.6cm}
\begin{tikzpicture}
	\card
	\cardstrip
	\cardbanner{banner/white.png}
	\cardicon{icons/coin.png}
	\cardprice{5}
	\cardtitle{Gesindel}
	\cardcontent{Ziehe drei Karten nach. Anschließend muss jeder Mitspieler (beginnend bei deinem linken Nachbarn) die obersten drei Karten seines Nachziehstapels aufdecken und alle aufgedeckten Geldkarten sowie Aktionskarten (auch kombinierte), ablegen. Alle anderen aufgedeckten Karten legt er in beliebiger Reihenfolge zurück auf den Nachziehstapel.}
\end{tikzpicture}
\hspace{-0.6cm}
\begin{tikzpicture}
	\card
	\cardstrip
	\cardbanner{banner/white.png}
	\cardicon{icons/coin.png}
	\cardprice{5}
	\cardtitle{Gewölbe}
	\cardcontent{Ziehe 2 Karten nach. Lege anschließend beliebig viele Handkarten (auch 0) ab. Du darfst auch Karten ablegen, die du gerade erst nachgezogen hast. Für jede abgelegte Karte erhältst du +\coin[1].

	\medskip

	Jeder Mitspieler darf 2 Handkarten ablegen und eine Karte nachziehen. Falls ein Mitspieler nur 1 Handkarte hat, darf er diese zwar ablegen, jedoch keine Karte nachziehen.}
\end{tikzpicture}
\hspace{-0.6cm}
\begin{tikzpicture}
	\card
	\cardstrip
	\cardbanner{banner/gold.png}
	\cardicon{icons/coin.png}
	\cardprice{6}
	\cardtitle{Hort}
	\cardcontent{Diese Karte ist eine Geldkarte mit zusätzlichen Anweisungen. Sie hat den Wert \coin[2].

	\medskip

	Solange diese Karte im Spiel ist und du eine Punktekarte (auch kombinierte) kaufst, nimmst du ein \emph{GOLD} vom Vorrat. Wenn kein \emph{GOLD} mehr im Vorrat ist, erhältst du nichts. Nimmst du eine Punktekarte auf andere Weise (d.h. nicht durch einen Kauf), nimmst du dir kein \emph{GOLD}. Hast du zwei \emph{HORTE} im Spiel, nimmst du dir pro gekaufter Punktekarte zwei \emph{GOLD} usw. Kaufst du in einem Spielzug zwei oder mehr Punktekarten, nimmst du dir für jede Punktekarte entsprechend viele \emph{GOLD} vom Vorrat. Du erhältst auch \emph{GOLD}, wenn du die gekaufte Punktekarte im gleichen Spielzug wieder entsorgst.}
\end{tikzpicture}
\hspace{-0.6cm}
\begin{tikzpicture}
	\card
	\cardstrip
	\cardbanner{banner/gold.png}
	\cardicon{icons/coin.png}
	\cardprice{5}
	\cardtitle{\tiny{Königliches Siegel}}
	\cardcontent{Diese Karte ist eine Geldkarte mit zusätzlichen Anweisungen. Sie hat den Wert \coin[2].

	\medskip

	Solange diese Karte im Spiel ist, entscheidest du für jede Karte, die du kaufst oder auf andere Weise nimmst, ob du sie ablegen oder oben auf deinen Nachziehstapel legen möchtest.}
\end{tikzpicture}
\hspace{-0.6cm}
\begin{tikzpicture}
	\card
	\cardstrip
	\cardbanner{banner/white.png}
	\cardicon{icons/coin.png}
	\cardprice{5}
	\cardtitle{Leihhaus}
	\cardcontent{Sieh dir deinen kompletten Ablagestapel an und nimm beliebig viele \emph{KUPFER} auf die Hand. Zeige diese vorher deinen Mitspielern. Die restlichen Karten legst du in beliebiger Reihenfolge wieder auf den Ablagestapel.}
\end{tikzpicture}
\hspace{-0.6cm}
\begin{tikzpicture}
	\card
	\cardstrip
	\cardbanner{banner/white.png}
	\cardicon{icons/coin.png}
	\cardprice{5}
	\cardtitle{Münzer}
	\cardcontent{Du darfst eine Geldkarte aus deiner Hand aufdecken. Wenn du das tust, nimm dir eine Karte mit gleichem Namen vom Vorrat. Ist keine entsprechende Karte im Vorrat vorhanden, erhältst du nichts. Nimm die aufgedeckte Geldkarte zurück auf die Hand.

	\medskip

	Wenn du den \emph{MÜNZER} kaufst, musst du alle Geldkarten, die du im Spiel hast, sofort entsorgen. Sofern du nach dem Kauf des \emph{MÜNZERS} noch \coin und einen weiteren Kauf übrig hast, darfst du die Geldwerte der entsorgten Karten in diesem Zug noch verwenden. Wenn du den \emph{MÜNZER} kaufst und eine Geldkarte im Spiel ist, deren zusätzliche Anweisung in Kraft tritt, sobald du eine Karte kaufst (z.B. \emph{KÖNIGLICHES SIEGEL}), wird diese Geldkarte entsorgt, bevor deren Effekt eintreten kann.

	\medskip

	Beachte, dass du alle Geldkarten, die du in deiner Kaufphase verwenden möchtest, vor deinem ersten Kauf ausspielen musst. Du kannst nicht 1 \emph{GOLD} und 1 \emph{SILBER} ausspielen, den \emph{MÜNZER} kaufen, diese Geldkarten entsorgen und dann weiteres Geld auslegen. Sobald eine Karte gekauft wurde, dürfen keinen weiteren Geldkarten mehr ausgespielt werden.}
\end{tikzpicture}
\hspace{-0.6cm}
\begin{tikzpicture}
	\card
	\cardstrip
	\cardbanner{banner/white.png}
	\cardicon{icons/coin.png}
	\cardprice{5}
	\cardtitle{Quacksalber}
	\cardcontent{Du erhältst +\coin[2].

	\medskip

	Jeder Mitspieler (beginnend bei deinem linken Nachbarn) legt entweder einen \emph{FLUCH} aus seiner Hand ab oder er nimmt einen \emph{FLUCH} und ein \emph{KUPFER} vom Vorrat und legt diese ab. Dies dürfen die Spieler auch wählen, wenn der \emph{KUPFER}- oder \emph{FLUCH}-Stapel leer sind. Nimmt ein Spieler den \emph{FLUCH} und das \emph{KUPFER} und hat einen \emph{WACHTURM} auf der Hand, darf er diesen nach jeder genommenen Karte aufdecken (oder wahlweise nur nach einer) und z.B. den \emph{FLUCH} entsorgen und das \emph{KUPFER} auf den Nachziehstapel legen (oder ebenfalls entsorgen).}
\end{tikzpicture}
\hspace{-0.6cm}
\begin{tikzpicture}
	\card
	\cardstrip
	\cardbanner{banner/gold.png}
	\cardicon{icons/coin.png}
	\cardprice{5}
	\cardtitle{\scriptsize{Schmuggelware}}
	\cardcontent{Diese Karte ist eine Geldkarte mit zusätzlichen Anweisungen. Sie hat den Wert \coin[3]. Du erhältst + 1 Kauf.

	\medskip

	Dein linker Mitspieler nennt den Namen einer beliebigen Karte (z.B. \enquote{Provinz}). Diese muss nicht Teil des Vorrats sein. Du darfst diese Karte in diesem Zug nicht kaufen. Wenn du die Karte auf andere Weise nehmen kannst, darfst du das tun. Spielst du mehrere \emph{SCHMUGGELWAREN} aus, nennt dein linker Mitspieler entsprechend viele Karten.}
\end{tikzpicture}
\hspace{-0.6cm}
\begin{tikzpicture}
	\card
	\cardstrip
	\cardbanner{banner/white.png}
	\cardicon{icons/coin.png}
	\cardprice{5}
	\cardtitle{Stadt}
	\cardcontent{Du erhältst + 1 Karte sowie + 2 Aktionen. Wenn noch kein Vorratsstapel leer ist, passiert nichts weiter. Wenn genau 1 Vorratsstapel leer ist, erhältst du nochmal + 1 Karte. Wenn 2 oder mehr Vorratsstapel leer sind, erhältst du stattdessen + 1 Karte, +\coin[1] und + 1 Kauf. }
\end{tikzpicture}
\hspace{-0.6cm}
\begin{tikzpicture}
	\card
	\cardstrip
	\cardbanner{banner/white.png}
	\cardicon{icons/coin.png}
	\cardprice{6}
	\cardtitle{\scriptsize{Großer Markt}}
	\cardcontent{Du erhältst + 1 Karte, + 1 Aktion, + 1 Kauf sowie +\coin[2].

	\medskip

	Wenn du diese Karte kaufen möchtest, darfst du zu diesem Zeitpunkt kein \emph{KUPFER} im Spiel haben. Wenn du zu einem früheren Zeitpunkt in deinem Zug \emph{KUPFER} im Spiel hattest, dieses aber entsorgt hast, darfst du den \emph{GROSSEN MARKT} kaufen. Kannst du den \emph{GROSSEN MARKT} auf andere Art nehmen, darfst du das jederzeit tun, auch wenn du \emph{KUPFER} im Spiel hast.}
\end{tikzpicture}
\hspace{-0.6cm}
\begin{tikzpicture}
	\card
	\cardstrip
	\cardbanner{banner/white.png}
	\cardicon{icons/coin.png}
	\cardprice{7}
	\cardtitle{Ausbau}
	\cardcontent{Entsorge eine beliebige Handkarte und nimm eine Karte vom Vorrat, die bis zu \coin[3] mehr kostet als die entsorgte Karte. Du darfst den Betrag nicht mit zusätzlichem \coin erhöhen. Hast du keine Karte auf der Hand, die du entsorgen kannst, darfst du dir keine Karte vom Vorrat nehmen. Den ausgespielten \emph{AUSBAU} selbst darfst du nicht entsorgen, da er sich nicht mehr auf deiner Hand befindet.}
\end{tikzpicture}
\hspace{-0.6cm}
\begin{tikzpicture}
	\card
	\cardstrip
	\cardbanner{banner/white.png}
	\cardicon{icons/coin.png}
	\cardprice{6}
	\cardtitle{\scriptsize{Halsabschneider}}
	\cardcontent{Du erhältst + 1 Kauf sowie +\coin[2]. Alle Mitspieler müssen Handkarten ablegen, bis sie nur noch 3 Karten auf der Hand haben. Hat ein Mitspieler bereits 3 oder weniger Karten auf der Hand, muss er keine Karten ablegen.

	\medskip

	Solange diese Karte im Spiel ist, legst du immer, wenn du eine Karte kaufst (nicht wenn du sie auf andere Weise nimmst), einen \victorypointtoken-Marker auf dein Spieler-Tableau. Hast du zwei \emph{HALSABSCHNEIDER} im Spiel, legst du zwei \victorypointtoken-Marker pro gekaufter Karte auf dein Tableau usw.}
\end{tikzpicture}
\hspace{-0.6cm}
\begin{tikzpicture}
	\card
	\cardstrip
	\cardbanner{banner/gold.png}
	\cardicon{icons/coin.png}
	\cardprice{7}
	\cardtitle{Bank}
	\cardcontent{Diese Karte ist eine Geldkarte mit einem variablen Wert: Pro Geldkarte (inklusive dieser / auch kombinierte Geldkarten), die du im Spiel hast, ist sie \coin[1] wert. Spielst du die \emph{BANK} als erste Geldkarte in deinem Zug aus, ist sie genau \coin[1] wert. Spielst du dagegen zuerst ein \emph{GOLD}, ein \emph{SILBER} und zwei \emph{KUPFER} und dann die \emph{BANK}, ist die \emph{BANK} \coin[5] wert. Spielst du im Anschluss noch eine \emph{BANK} aus, bleibt die erste \emph{BANK} \coin[5] wert, die zweite \coin[6].}
\end{tikzpicture}
\hspace{-0.6cm}
\begin{tikzpicture}
	\card
	\cardstrip
	\cardbanner{banner/white.png}
	\cardicon{icons/coin.png}
	\cardprice{7}
	\cardtitle{Königshof}
	\cardcontent{Diese Karte ist ähnlich dem \emph{THRONSAAL} aus dem Basisspiel – mit dem Unterschied, dass du die Aktionskarte, die du aus deiner Hand wählst, dreimal (statt zweimal) ausspielst. Lege die gewählte Aktionskarte aus, führe die Anweisungen darauf komplett aus, nimm sie zurück auf die Hand, spiele sie erneut aus, führe die Anweisungen darauf komplett aus, nimm sie zurück auf die Hand und spiele sie ein drittes Mal aus und führe die Anweisungen darauf komplett aus. Für das dreimalige Ausspielen der Aktionskarte benötigst du keine Aktionen. Du darfst zwischen dem dreimaligen Ausspielen der Aktionskarte keine andere Aktion ausspielen, außer die Aktionskarte selbst gibt dazu die Anweisung.  }
\end{tikzpicture}
\hspace{-0.6cm}
\begin{tikzpicture}
	\card
	\cardstrip
	\cardbanner{banner/white.png}
	\cardicon{icons/coin.png}
	\cardprice{7}
	\cardtitle{\footnotesize{Kunstschmiede}}
	\cardcontent{Egal ob du keine Karte entsorgst (\coin[0] insgesamt) oder z.B. drei Karten, die jeweils \coin[2] kosten (\coin[6] insgesamt) – du \emph{musst} eine Karte vom Vorrat nehmen, die genau so viel kostet wie die entsorgten Karten zusammen gekostet haben, außer es ist keine entsprechende Karte im Vorrat vorhanden. Entsorgst du keine Karten und ist beispielsweise der \emph{KUPFER}-Stapel leer, musst du dir u.U. (wenn keine anderen Karten mit \coin[0]-Kosten vorhanden sind) einen \emph{FLUCH} vom Vorrat nehmen, der ebenfalls \coin[0] kostet. \potion-Kosten für Karten aus Alchemie haben für die \emph{KUNSTSCHMIEDE} keine Auswirkung. Es darf auch keine Karte, die \potion-Kosten enthält, genommen werden.}
\end{tikzpicture}
\hspace{-0.6cm}
\begin{tikzpicture}
	\card
	\cardstrip
	\cardbanner{banner/white.png}
	\cardicon{icons/coin.png}
	\cardprice{8*}
	\cardtitle{Hausierer}
	\cardcontent{Diese Karte ist eine Karte mit variablen Kosten (vgl. NEUE REGELN; S. 6). Du erhältst + 1 Karte, + 1 Aktion sowie +\coin[1]. 

	\medskip

	Wenn du diese Karte in deiner Kaufphase kaufst, kostet sie für jede Aktionskarte, die du im Spiel hast, \coin[2] weniger, niemals allerdings weniger als \coin[0]. Kaufst du eine Karte außerhalb der Kaufphase (z.B. durch den \emph{SCHWARZMARKT}), kostet der HAUSIERER \coin[8], egal ob du weitere Aktionskarten im Spiel hast oder nicht. Aktionskarten, die durch den \emph{THRONSAAL} (aus Basisspiel) oder den \emph{KÖNIGSHOF} mehrfach ausgespielt wurden, sind trotzdem jeweils nur einmal im Spiel und reduzieren die Kosten eines \emph{HAUSIERERS} um \coin[2].}
\end{tikzpicture}
\hspace{-0.6cm}
\begin{tikzpicture}
	\card
	\cardstrip
	\cardbanner{banner/gold.png}
	\cardicon{icons/coin.png}
	\cardprice{9}
	\cardtitle{Platin}
	\cardcontent{Diese Karte ist eine Basiskarte und keine Königreichkarte. Spielt ihr ausschließlich mit Königreichkarten aus Blütezeit, wird diese Karte zusätzlich zu den Basis-Geldkarten \emph{KUPFER}, \emph{SILBER} und \emph{GOLD} in der Spielvorbereitung in den Vorrat gelegt. Bei Spielen mit Königreichkarten aus verschiedenen Editionen oder Erweiterungen entscheidet vor Spielbeginn, ob ihr \emph{PLATIN} in den Vorrat legen wollt oder nicht (vgl. SPIELVORBEREITUNG, S. 4).}
\end{tikzpicture}
\hspace{-0.6cm}
\begin{tikzpicture}
	\card
	\cardstrip
	\cardbanner{banner/green.png}
	\cardicon{icons/coin.png}
	\cardprice{11}
	\cardtitle{Kolonien}
	\cardcontent{Diese Karte ist eine Basiskarte und keine Königreichkarte. Spielt ihr ausschließlich mit Königreichkarten aus Blütezeit, wird diese Karte zusätzlich zu den Basis-Punktekarten \emph{ANWESEN}, \emph{HERZOGTUM} und \emph{PROVINZ} in der Spielvorbereitung in den Vorrat gelegt. Bei Spielen mit Königreichkarten aus verschiedenen Editionen oder Erweiterungen entscheidet vor Spielbeginn, ob ihr die \emph{KOLONIE} in den Vorrat legen wollt oder nicht. Achtet darauf, dass in diesem Fall das Spiel auch endet, wenn der Vorratsstapel \emph{KOLONIE} leer ist (vgl. SPIELVORBEREITUNG, S. 4 sowie ALTERNATIVES SPIELENDE, S. 6).}
\end{tikzpicture}
\hspace{-0.6cm}
\begin{tikzpicture}
	\card
	\cardstrip
	\cardbanner{banner/white.png}
	\cardtitle{\scriptsize{Empfohlene 10er Sätze\qquad}}
	\cardcontent{\emph{Anfänger:}\\
	Abenteuer, Arbeiterdorf, Ausbau, Bank, Denkmal, Gesindel, Halsabschneider, Königliches Siegel, Leihhaus, Wachturm

	\smallskip

	\emph{Freundliche Interaktion:}\\
	Arbeiterdorf, Bischof, Gewölbe, Handelsroute, Hausierer, Hort, Königliches Siegel, Kunstschmiede, Schmuggelware, Stadt 

	\smallskip

	\emph{Große Aktionen:}\\
	Ausbau, Gesindel, Gewölbe, Großer Markt, Königshof, Lohn, Münzer, Stadt, Steinbruch, Talisman

	\smallskip

	\emph{Haufenweise Geld} (Blütezeit + \textit{Basisspiel}):\\
	Abenteuer, Bank, Großer Markt, Königliches Siegel, Münzer, \textit{Abenteurer}, \textit{Geldverleiher}, \textit{Laboratorium}, \textit{Mine}, \textit{Spion}

	\smallskip

	\emph{Die Armee des Königs} (Blütezeit + \textit{Basisspiel}):\\
	Ausbau, Gesindel, Gewölbe, Handlanger, Königshof, \textit{Bürokrat}, \textit{Burggraben}, \textit{Dorf}, \textit{Ratsversammlung}, \textit{Spion}

	\smallskip

	\emph{Ein gutes Leben:} (Blütezeit + \textit{Basisspiel}):\\
	Denkmal, Hort, Leihhaus, Quacksalber, Schmuggelware, \textit{Bürokrat}, \textit{Dorf}, \textit{Gärten}, \textit{Kanzler}, \textit{Keller}}
\end{tikzpicture}
\hspace{-0.6cm}
\begin{tikzpicture}
	\card
	\cardstrip
	\cardbanner{banner/white.png}
	\cardtitle{\scriptsize{Empfohlene 10er Sätze\qquad}}
	\cardcontent{\emph{Pfade zum Sieg} (Blütezeit + \textit{Die Intrige}):\\
	Bischof, Denkmal, Halsabschneider, Hausierer, Leihhaus, \textit{Anbau}, \textit{Armenviertel}, \textit{Baron}, \textit{Handlanger}, \textit{Harem}

	\smallskip

	\emph{All along the watchtower} (Blütezeit + \textit{Die Intrige}):\\
	Gewölbe, Handelsroute, Hort, Talisman, Wachturm, \textit{Bergwerk}, \textit{Brücke}, \textit{Große Halle}, \textit{Handlanger}, \textit{Kerkermeister}

	\smallskip

	\emph{Glücksritter} (Blütezeit + \textit{Die Intrige}):\\
	Ausbau, Bank, Gewölbe, Königshof, Kunstschmiede, \textit{Brücke}, \textit{Kupferschmied}, \textit{Tribut}, \textit{Trickser}, \textit{Wunschbrunnen}}
\end{tikzpicture}
\hspace{0.6cm}

	    % Basic settings for this card set
\renewcommand{\cardcolor}{cornucopia}
\renewcommand{\cardextension}{Erweiterung IV}
\renewcommand{\cardextensiontitle}{Reiche Ernte}
\renewcommand{\seticon}{cornucopia.png}

\clearpage
\newpage
\section{\cardextension \ - \cardextensiontitle \ (Hans Im Glück 2011)}

\begin{tikzpicture}
	\card
	\cardstrip
	\cardbanner{banner/white.png}
	\cardicon{icons/coin.png}
	\cardprice{4}
	\cardtitle{Bauerndorf}
	\cardcontent{Du erhältst zunächst +2 Aktionen. Dann deckst du solange Karten von deinem Nachziehstapel auf, bis entweder eine Geldkarte oder eine Aktionskarte offen liegt. Nimm diese Geld- oder Aktionskarte auf die Hand. Lege die übrigen aufgedeckten Karten auf deinen Ablagestapel. Du darfst nicht wählen, ob du eine Geldkarte oder eine Aktionskarte auf die Hand nehmen möchtest. Du musst die erste aufgedeckte Geld- oder Aktionskarte auf die Hand nehmen. Kombinierte Kartentypen sind gleichzeitig alle angegebenen Kartentypen. Du darfst die aufgenommene Karte auch in diesem Zug (nach den üblichen Regeln) ausspielen. Wenn du auch nach dem Mischen deines Ablagestapels keine Geld- oder Aktionskarte aufdecken kannst, nimmst du keine Karte auf die Hand.}
\end{tikzpicture}
\hspace{-0.6cm}
\begin{tikzpicture}
	\card
	\cardstrip
	\cardbanner{banner/white.png}
	\cardicon{icons/coin.png}
	\cardprice{5}
	\cardtitle{Ernte}
	\cardcontent{Decke die obersten 4 Karten von deinem Nachziehstapel auf. Du erhältst +\coin[1] für jede aufgedeckte Karte mit unterschiedlichem Namen. Deckst du z. B. 2 Kupfer, 1 Silber und 1 Anwesen auf, erhältst du +\coin[3]. Kannst du (auch nach dem Mischen des Ablagestapels) nur weniger als 4 Karten aufdecken, deckst du nur so viele auf, wie möglich.}
\end{tikzpicture}
\hspace{-0.6cm}
\begin{tikzpicture}
	\card
	\cardstrip
	\cardbanner{banner/green.png}
	\cardicon{icons/coin.png}
	\cardprice{6}
	\cardtitle{Festplatz}
	\cardcontent{Diese Königreichkarte ist eine Punktekarte, keine Aktionskarte. Sie hat bis zum Ende des Spiels keine Funktion. Bei der Wertung zählt sie 2 Siegpunkte für je volle 5 Karten mit unterschiedlichem Namen im gesamten Kartensatz (Nachziehstapel, Ablagestapel und Handkarten) des Spielers. Bei Spielende suchst du aus deinem gesamten Kartensatz je eine Karte jedes Namens heraus. Diese Karten zählst du und teilst die Anzahl durch 5. Das Ergebnis (abgerundet) multipliziert mit 2 ergibt die Punkte. Hast du 0-4 Karten mit unterschiedlichem Namen erhältst du keine Punkte, für 5-9 Karten 2 Punkte , für 10-14 Karten 4 Punkte usw.
	
	\smallskip
	
	Im Spiel mit 2 Spielern werden 8 Karten Festplatz verwendet, bei 3 oder mehr Spielern 12 Karten.}
\end{tikzpicture}
\hspace{-0.6cm}
\begin{tikzpicture}
	\card
	\cardstrip
	\cardbanner{banner/gold.png}
	\cardicon{icons/coin.png}
	\cardprice{5}
	\cardtitle{Füllhorn}
	\cardcontent{Du spielst diese Karte, wie jede andere Geldkarte, in der Kaufphase aus. Das Füllhorn hat den Wert \coin[0]. Es bringt also kein Geld für den Kauf. Du darfst dir jedoch sofort, wenn du die Karte ausspielst, eine Karte aus dem Vorrat nehmen. Diese Karte darf bis zu \coin[1] pro Karte mit unterschiedlichem Namen, die du im Spiel hast, kosten. Karten, die du im Spiel hast sind: in diesem Zug ausgespielte Aktions- und Geldkarten (das Füllhorn selbst eingeschlossen) und bei dir ausliegende Dauerkarten (Dominion – Seaside). Nicht im Spiel sind Karten, die du nach dem ausspielen entsorgt hast (z. B. Festmahl, Dominion – Basisspiel). Nimmst du dir mit dem Füllhorn eine Punktekarte (auch eine kombinierte), entsorgst du das Füllhorn. Du entsorgst das Füllhorn nicht, wenn du dir eine Punktekarte auf eine andere Art nimmst oder kaufst.}
\end{tikzpicture}
\hspace{-0.6cm}
\begin{tikzpicture}
	\card
	\cardstrip
	\cardbanner{banner/white.png}
	\cardicon{icons/coin.png}
	\cardprice{5}
	\cardtitle{Harlekin}
	\cardcontent{Du erhältst zunächst +\coin[2]. Dann muss, beginnend mit dem Spieler links von dir, reihum jeder Mitspieler die oberste Karte von seinem Nachziehstapel ablegen. Ist es eine Punktekarte (auch eine kombinierte), so muss er sich einen Fluch nehmen. Ist kein Fluch mehr im Vorrat, nimmt der Spieler keinen. Ist es keine Punktekarte, kannst du wählen: Entweder der Spieler muss sich eine Karte mit gleichem Namen aus dem Vorrat nehmen und auf seinen Ablagestapel legen oder du nimmst dir eine Karte mit gleichem Namen und legst sie auf deinen Ablagestapel. Ist keine Karte mit diesem Namen mehr im Vorrat nimmt keiner von beiden eine solche Karte.}
\end{tikzpicture}
\hspace{-0.6cm}
\begin{tikzpicture}
	\card
	\cardstrip
	\cardbanner{banner/white.png}
	\cardicon{icons/coin.png}
	\cardprice{4}
	\cardtitle{Junge Hexe}
	\cardcontent{\tiny{\begin{Spacing}{1}
	\emph{Spielvorbereitung:} Wird die Junge Hexe im Spiel verwendet, so wird zusätzlich ein Bannstapel benötigt. (Dies gilt auch, wenn sich die Junge Hexe im Schwarzmarktstapel befindet – Promokarte: Schwarzmarkt.) Für den Bannstapel wählen die Spieler eine nicht verwendete Königreichkarte, die \coin[2] oder \coin[3] kostet, aus einer beliebigen Edition oder Erweiterung aus. Der komplette Stapel dieser Karte wird als zusätzlicher Stapel in den Vorrat gelegt. Die Platzhalterkarte Junge Hexe wird quer unter den Stapel geschoben, um diesen Stapel als Bannstapel zu kennzeichnen. In diesem Spiel werden also 11 unterschiedliche Königreichskarten verwendet. Zusätzlich zur üblichen (aufgedruckten) Funktion sind alle diese Karten Bannkarten. Der Bannstapel wird behandelt wie die anderen Königreichstapel. Er ist Teil des Vorrats, die Karten können wie üblich gekauft oder genommen werden und der Stapel wird für die Spielendebedingung berücksichtigt.
	
	\smallskip
	
	Wenn du die Junge Hexe ausspielst, ziehst du zuerst 2 Karten nach, dann musst du 2 Karten aus deiner Hand ablegen. Beginnend mit dem Spieler links von dir darf nun reihum jeder Mitspieler eine Bannkarte aus seiner Hand aufdecken. Wenn er das nicht macht, muss er sich einen Fluch vom Vorrat nehmen. Ist kein Fluch mehr im Vorrat, muss der Spieler keinen nehmen. Die Spieler dürfen wie üblich auch Reaktionskarten (z. B. Pferdehändler) aus ihrer Hand aufdecken. Dies tun die Spieler, bevor sie eine Bannkarte aufdecken. Die Anweisungen auf der Reaktionskarte werden also ausgeführt, bevor der Angriff durch die Bannkarte abgewehrt wird. Ist die Bannkarte eine Reaktionskarte, kann sie zuerst aufgedeckt werden um die übliche Reaktion auszuführen und dann (wenn der Spieler sie dann noch auf der Hand hat) nochmals in ihrer Funktion als Bannkarte.
	\end{Spacing}}}
\end{tikzpicture}
\hspace{-0.6cm}
\begin{tikzpicture}
	\card
	\cardstrip
	\cardbanner{banner/white.png}
	\cardicon{icons/coin.png}
	\cardprice{3}
	\cardtitle{Menagerie}
	\cardcontent{Zuerst erhältst du +1 Aktion. Dann deckst du deine Handkarten auf. Hast du nur Karten mit unterschiedlichem Namen auf der Hand, ziehst du 3 Karten nach. Hast du wenigstens eine Karte mit gleichem Namen auf der Hand (eine Karte doppelt), ziehst du 1 Karte nach. Hast du z. B. nur ein Kupfer und ein Silber auf der Hand, hast du nur Karten mit unterschiedlichem Namen (obwohl beide Geldkarten sind). Du darfst also 3 Karten nachziehen.}
\end{tikzpicture}
\hspace{-0.6cm}
\begin{tikzpicture}
	\card
	\cardstrip
	\cardbanner{banner/white.png}
	\cardicon{icons/coin.png}
	\cardprice{4}
	\cardtitle{Nachbau}
	\cardcontent{Zuerst entsorgst du eine Karte aus deiner Hand und nimmst dir dafür eine Karte, die genau \coin[1] mehr kostet. Dann entsorgst du eine andere Karte aus deiner Hand und nimmst dir wieder eine Karte, die genau \coin[1] mehr kostet. Du nimmst die Karten vom Vorrat und legst sie auf deinen Ablagestapel. Ist im Vorrat keine Karte, die genau 1 Geld mehr kostet als die entsorgte, nimmst du dir keine Karte.}
\end{tikzpicture}
\hspace{-0.6cm}
\begin{tikzpicture}
	\card
	\cardstrip
	\cardbanner{banner/blue.png}
	\cardicon{icons/coin.png}
	\cardprice{4}
	\cardtitle{\footnotesize{Pferdehändler}}
	\cardcontent{Wenn du diese Karte in deinem Zug ausspielst, erhältst du zuerst +1 Kauf und +\coin[3]. Dann musst du 2 Karten aus deiner Hand ablegen. Hast du weniger als 2 Karten auf der Hand, legst du nur so viele wie möglich ab.

	\smallskip
				
	Spielt ein Mitspieler eine Angriffskarte, darfst du diese Karte aus deiner Hand vorzeigen und dann zur Seite legen. Dann wird der Angriff normal ausgeführt. Du kannst jeden Pferdehändler nur gegen einen Angriff verwenden, da du ihn danach nicht mehr auf der Hand hast. Du darfst mehrere Pferdehändler gegen den selben Angriff zur Seite legen. Zu Beginn deines nächsten Zuges nimmst du alle zur Seite gelegten Pferdehändler wieder auf die Hand und ziehst zusätzlich für jeden dieser Pferdehändler 1 Karte nach.}
\end{tikzpicture}
\hspace{-0.6cm}
\begin{tikzpicture}
	\card
	\cardstrip
	\cardbanner{banner/white.png}
	\cardicon{icons/coin.png}
	\cardprice{5}
	\cardtitle{Treibjagd}
	\cardcontent{Zuerst ziehst du eine Karte nach und erhältst +1 Aktion. Dann deckst du deine gesamten Handkarten auf und legst diese offen aus. Danach deckst du solange Karten von deinem Nachziehstapel auf, bis du eine Karte aufdeckst, die du nicht auf der (momentan offen ausliegenden) Hand hast. Nimm diese gerade aufgedeckte Karte und deine offen ausliegenden Handkarten zurück auf die Hand. Lege die übrigen aufgedeckten Karten auf deinen Ablagestapel. Kannst du (auch nach dem Mischen deines Ablagestapels) keine Karte aufdecken, von der du nicht bereits eine auf der Hand hast, legst du die aufgedeckten Karten ab und nimmst nur deine Handkarten zurück auf die Hand.}
\end{tikzpicture}
\hspace{-0.6cm}
\begin{tikzpicture}
	\card
	\cardstrip
	\cardbanner{banner/white.png}
	\cardicon{icons/coin.png}
	\cardprice{4}
	\cardtitle{Turnier}
	\cardcontent{\miniscule{\begin{Spacing}{1}
	\emph{Spielvorbereitung:} Wird das Turnier für das Spiel verwendet, so wird zusätzlich der Preisstapel benötigt. (Dies gilt auch, wenn sich das Turnier im Schwarzmarktstapel befindet – Promokarte: Schwarzmarkt.) Für den Preisstapel werden die 5 einmaligen Preiskarten benötigt. Alle Preiskarten werden als ein Stapel offen bereit gelegt. Der Preisstapel besteht nur aus diesen 5 Karten und ist nicht Teil des Vorrats. 
	
	\smallskip

	\emph{Errata:} Der Kartentext is falsch, es sollte \enquote{Jeder Spieler darf eine Provinz aus seiner Hand aufdecken. Wenn du das machst, lege die Provinz ab und nimm dir einen Preis (vom Preisstapel) oder ein Herzogtum und lege die neue Karte auf deinen Nachziehstapel. Wenn es kein Mitspieler macht: +1 Karte, +\coin[1].} statt \enquote{Du darfst eine Provinz aus deiner Hand ablegen. Wenn du das machst [...] Jeder Mitspieler darf...} heißen. 
	
	\smallskip

	Zuerst erhältst du +1 Aktion. Dann darf jeder Spieler eine Provinz aus seiner Hand aufdecken. Wenn du das machst, lege die Provinz ab und nehme dir eine frei wählbare Preiskarte vom Preisstapel oder ein Herzogtum vom Vorrat und lege die neue Karte auf deinen Nachziehstapel. Du kannst dich auch für einen der beiden Stapel entscheiden, wenn dieser leer ist. Ist der Stapel für den du dich entscheidest leer, nimmst du dir keine Karte. Wenn keiner deiner Mitspieler eine Provinz aufgedeckt hat, erhältst du +1 Karte und +\coin[1]. Es gibt also 4 mögliche Ergebnisse:
	\begin{itemize}	
	\item Du legst keine Provinz ab und keiner deiner Mitspieler deckt eine Provinz auf: Du erhältst +1 Karte, +1 Aktion und +\coin[1].\\
	\item Du legst keine Provinz ab und mindestens einer deiner Mitspieler deckt eine Provinz auf: Du erhältst +1 Aktion.\\
	\item Du legst eine Provinz ab aber keiner deiner Mitspieler deckt eine Provinz auf: Du nimmst dir eine Preiskarte oder ein Herzogtum und erhältst +1 Karte, +1 Aktion und +\coin[1].\\
	\item Du legst eine Provinz ab und mindestens einer deiner Mitspieler deckt eine Provinz auf: Du nimmst dir eine Preiskarte oder ein Herzogtum und erhältst +1 Aktion.\\
	\end{itemize}
	Du darfst den Preisstapel jederzeit durchsehen.
	\end{Spacing}}}
\end{tikzpicture}
\hspace{-0.6cm}
\begin{tikzpicture}
	\card
	\cardstrip
	\cardbanner{banner/white.png}
	\cardicon{icons/coin.png}
	\cardprice{3}
	\cardtitle{Wahrsagerin}
	\cardcontent{Du erhältst zunächst +\coin[2]. Dann muss, beginnend mit dem Spieler links von dir, reihum jeder Mitspieler solange Karten von seinem Nachziehstapel aufdecken, bis entweder eine Punktekarte oder ein Fluch offen liegt. Er legt diese Punkte- oder Fluchkarte verdeckt auf seinen Nachziehstapel. Der Spieler darf nicht wählen, ob er eine Punktekarte oder einen Fluch auf den Nachziehstapel legt, er muss die erste aufgedeckte Punkte- oder Fluchkarte zurücklegen. Die übrigen aufgedeckten Karten legt er auf seinen Ablagestapel. Wenn der Spieler auch nach dem Mischen seines Ablagestapels keine Punkte- oder Fluchkarte aufdecken kann, legt er alle aufgedeckten Karten ab. Karten mit kombinierten Kartentypen sind gleichzeitig alle angegebenen Kartentypen.}
\end{tikzpicture}
\hspace{-0.6cm}
\begin{tikzpicture}
	\card
	\cardstrip
	\cardbanner{banner/white.png}
	\cardicon{icons/coin.png}
	\cardprice{2}
	\cardtitle{Weiler}
	\cardcontent{Zuerst ziehst du immer eine Karte nach und erhältst +1 Aktion. Nun darfst du eine Karte aus deiner Hand ablegen um entweder zusätzlich +1 Aktion oder +1 Kauf zu erhalten. Oder du legst 2 Karten aus deiner Hand ab und erhältst zusätzlich +1 Aktion und +1 Kauf. Du kannst jedoch nicht +2 Aktionen oder +2 Käufe zusätzlich erhalten. Du kannst auch darauf verzichten, Karten abzulegen und erhältst dann nichts weiter.}
\end{tikzpicture}
\hspace{-0.6cm}
\begin{tikzpicture}
	\card
	\cardstrip
	\cardbanner{banner/white.png}
	\cardicon{icons/coin.png}
	\cardprice{0*}
	\cardtitle{Preiskarten}
	\cardcontent{\tiny{\begin{Spacing}{1}	
	\emph{Diadem:} Diese Geldkarte hat den Wert 2, wie ein Silber. Wenn du sie in deiner Kaufphase auslegst, erhältst du zusätzlich +1 virtuelles Geld für jede unverbrauchte Aktion. Hast du z. B. in deiner Aktionsphase keine Aktionskarte ausgespielt, so erhältst du für deine freie Aktion +1 Geld. Hast du in deiner Aktionsphase das Bauerndorf ausgespielt, so ist dadurch deine freie Aktion verbraucht. Das Bauerndorf gibt dir jedoch +2 Aktionen. Spielst du keine weitere Aktionskarte mehr aus, bringt dir das Diadem in der Kaufphase +2 Geld.

	\medskip

	\emph{Ein Sack voll Gold:} Zuerst erhältst du +1 Aktion. Dann nimmst du dir ein Gold aus dem Vorrat und legst es sofort auf deinen Nachziehstapel. Falls der Nachziehstapel leer ist, legst du dieses Gold an die Stelle deines Nachziehstapels. Ist kein Gold mehr im Vorrat, nimmst du keines.

	\medskip

	\emph{Gefolge:} Du ziehst zunächst 2 Karten von deinem Nachziehstapel. Dann nimmst du ein Anwesen vom Vorrat und legst es auf deinen Ablagestapel. Ist kein Anwesen mehr im Vorrat, nimmst du dir keines. Nun muss sich, beginnend mit dem Spieler links von dir, reihum jeder Mitspieler einen Fluch vom Vorrat nehmen und auf seinen Ablagestapel legen und danach zusätzlich solange Karten aus seiner Hand ablegen, bis er nur noch 3 Karten auf der Hand hält. Hat ein Mitspieler 3 oder weniger Karten auf der Hand, muss er keine Karten ablegen (er muss sich jedoch trotzdem einen Fluch nehmen). Ist kein Fluch mehr im Vorrat, nimmt der Spieler sich keinen.
	\end{Spacing}}}
\end{tikzpicture}
\hspace{-0.6cm}
\begin{tikzpicture}
	\card
	\cardstrip
	\cardbanner{banner/white.png}
	\cardicon{icons/coin.png}
	\cardprice{0*}
	\cardtitle{Preiskarten}
	\cardcontent{\tiny{\begin{Spacing}{1}
	\emph{Prinzessin:} Zuerst erhältst du +1 Kauf. Wenn diese Karte im Spiel ist, kosten alle Karten um 2 Geld weniger, niemals jedoch weniger als 0 Geld. Dies betrifft alle Karten im Vorrat, in den Nachzieh- und den Ablagestapeln aller Spieler und auch die Handkarten der Spieler. Wenn du mit dem Nachbau ein Kupfer (kostet weiterhin 0 Geld) aus deiner Hand entsorgst, kannst du dir dafür z. B. ein Silber nehmen, das nur noch 1 Geld kostet. Spielst du die Prinzessin auf einen Thronsaal, sind die Karten trotzdem nur um 2 Geld billiger, da die Prinzessin nur einmal im Spiel ist.

	\medskip

	\emph{Errata:} Der Kartentext des Streitross ist falsch, es sollte \enquote{[...] und lege sofort deinen kompletten Nachziehstapel auf deinen Ablagestapel.} statt \enquote{[...] und lege sofort deinen kompletten Nachziehstapel ab.} heißen. Durch das direkte Ablegen wird die Karte Tunnel (Dominion – Hinterland) nicht ausgelöst.

	\medskip

	\emph{Streitross:} Zuerst wählst du 2 unterschiedliche der 4 Anweisungen auf der Karte. Dann führst du die beiden gewählten Anweisungen in der Reihenfolge auf der Karte aus. Wenn du z. B. +2 Karten und die 4 Silber wählst, ziehst du die 2 Karten nach, bevor du die 4 Silber nimmst und deinen Nachziehstapel auf den Ablagestapel legst. Sind im Vorrat weniger als 4 Silber nimmst du dir nur die übrigen Silber im Vorrat. Du darfst deinen Nachziehstapel nicht durchsehen, bevor du ihn auf den Ablagestapel legst.
	\end{Spacing}}}
\end{tikzpicture}
\hspace{-0.6cm}
\begin{tikzpicture}
	\card
	\cardstrip
	\cardbanner{banner/white.png}
	\cardtitle{\scriptsize{Empfohlene 10er Sätze\qquad}}
	\cardcontent{\emph{Reiche Ernte	und	Basisspiel:}

	\smallskip

	\emph{Kopfgeld:}\\
	Ernte, Füllhorn, Menagerie, Treibjagd, Turnier (+ Preiskarten) / Geldverleiher, Jahrmarkt, Keller, Miliz, Schmiede

	\smallskip

	\emph{Böses Omen:}\\
	Füllhorn, Harlekin, Nachbau, Wahrsagerin, Weiler / Abenteurer, Bürokrat, Laboratorium, Spion, Thronsaal

	\smallskip

	\emph{Wanderzirkus:}\\
	Bauerndorf, Festplatz, Harlekin, Junge Hexe (+ Kanzler als Bannstapel), Pferdehändler / Festmahl, Laboratorium, Markt, Umbau, Werkstatt}
\end{tikzpicture}
\hspace{-0.6cm}
\begin{tikzpicture}
	\card
	\cardstrip
	\cardbanner{banner/white.png}
	\cardtitle{\scriptsize{Empfohlene 10er Sätze\qquad}}
	\cardcontent{\emph{Reiche Ernte und	Die	Intrige}

	\smallskip

	\emph{Wer zuletzt lacht:}\\
	Bauerndorf, Ernte, Harlekin, Pferdehändler, Treibjagd / Adlige, Handlanger, Lakai, Trickser, Verwalter

	\smallskip

	\emph{Würze des Lebens:}\\
	Festplatz, Füllhorn, Junge Hexe (+ Wunschbrunnen als Bannstapel), Nachbau, Turnier (+ Preiskarten) / 	Bergwerk, Burghof, Große Halle, Kupferschmied, Tribut 

	\smallskip

	\emph{Kleine  Siege:}\\
	Nachbau, Treibjagd, Turnier (+ Preiskarten), Wahrsagerin, Weiler / Große Halle, Handlanger, Harem, Herzog, Verschwörer}
\end{tikzpicture}
\hspace{0.6cm}

	    % Basic settings for this card set
\renewcommand{\cardcolor}{cornucopia}
\renewcommand{\cardextension}{Erweiterung IV}
\renewcommand{\cardextensiontitle}{Reiche Ernte}
\renewcommand{\seticon}{cornucopia.png}

\clearpage
\newpage
\section{\cardextension \ - \cardextensiontitle \ (Rio Grande Games 2014)}

\begin{tikzpicture}
	\card
	\cardstrip
	\cardbanner{banner/white.png}
	\cardicon{icons/coin.png}
	\cardprice{2}
	\cardtitle{Weiler}
	\cardcontent{Du \emph{darfst} 1 oder 2 Karten ablegen. Wenn du 1 Karte ablegst, erhältst du  + 1 Aktion. Wenn du 2 Karten ablegst, erhältst du + 1 Aktion und + 1 Kauf. Wenn du keine Karte ablegst, erhältst du nichts.}
\end{tikzpicture}
\hspace{-0.6cm}
\begin{tikzpicture}
	\card
	\cardstrip
	\cardbanner{banner/white.png}
	\cardicon{icons/coin.png}
	\cardprice{3}
	\cardtitle{Menagerie}
	\cardcontent{Zeige deine Handkarten vor. Hast du nur Karten mit unterschiedlichen  Namen (z. B. eine \emph{MENAGERIE}, ein \emph{SILBER} und ein \emph{KUPFER}), ziehst du 3 Karten.  Hast du mindestens eine Karte doppelt auf der Hand (z. B. 2 \emph{KUPFER}), ziehst du eine Karte.}
\end{tikzpicture}
\hspace{-0.6cm}
\begin{tikzpicture}
	\card
	\cardstrip
	\cardbanner{banner/white.png}
	\cardicon{icons/coin.png}
	\cardprice{3}
	\cardtitle{Wahrsagerin}
	\cardcontent{Jeder Mitspieler, beginnend bei deinem linken Nachbarn, deckt solange Karten von seinem Nachziehstapel auf, bis er entweder eine Punkte- oder eine Fluchkarte aufgedeckt hat. Diese Karte muss er oben auf seinen Nachziehstapel legen. Ist der Nachziehstapel aufgebraucht, ohne dass eine entsprechende Karte aufgedeckt wurde, muss der Ablagestapel gemischt und weitere Karten aufgedeckt werden. Wird trotzdem keine Punkte- oder Fluchkarte aufgedeckt, legt der Spieler keine Karte auf den Nachziehstapel. Alle anderen aufgedeckten Karten werden abgelegt.}
\end{tikzpicture}
\hspace{-0.6cm}
\begin{tikzpicture}
	\card
	\cardstrip
	\cardbanner{banner/white.png}
	\cardicon{icons/coin.png}
	\cardprice{4}
	\cardtitle{Bauerndorf}
	\cardcontent{Decke solange Karten von deinem Nachziehstapel auf, bis du entweder eine Geldkarte oder eine Aktionskarte aufgedeckt hast. Nimm diese Karte auf die Hand und lege die anderen aufgedeckten Karten ab. Hast du in deinem Nachziehstapel (auch nach dem Mischen des Ablagestapels) keine Geld- oder Aktionskarte (oder entsprechende kombinierte Karte), nimmst du keine Karte auf die Hand.}
\end{tikzpicture}
\hspace{-0.6cm}
\begin{tikzpicture}
	\card
	\cardstrip
	\cardbanner{banner/white.png}
	\cardicon{icons/coin.png}
	\cardprice{4}
	\cardtitle{Junge Hexe}
	\cardcontent{Wenn die \emph{JUNGE HEXE} als Königreichkarte für das Spiel ausgewählt wurde, benötigt ihr als 11. Stapel im Vorrat einen Bannstapel (Spielvorbereitung, S. 3).

	\medskip

	Wenn du die \emph{JUNGE HEXE} ausspielst, ziehst du zuerst 2 Karten und legst dann 2 Handkarten ab. Jeder Mitspieler, der keine Bannkarte von seiner Hand aufdeckt, muss sich, beginnend bei deinem linken Nachbarn, einen \emph{FLUCH} vom Vorrat nehmen. Wird der Vorrat an \emph{FLÜCHEN} dabei aufgebraucht, müssen die Spieler, für die kein \emph{FLUCH} mehr vorhanden ist, keinen \emph{FLUCH} nehmen. 

	\medskip

	Die Spieler dürfen auch Reaktionskarten ausspielen, bevor sie eine Bannkarte ausspielen. Sind die Bannkarten gleichzeitig Reaktionskarten, dürfen sie zuerst als Reaktionskarten und dann als Bannkarten aufgedeckt werden.}
\end{tikzpicture}
\hspace{-0.6cm}
\begin{tikzpicture}
	\card
	\cardstrip
	\cardbanner{banner/white.png}
	\cardicon{icons/coin.png}
	\cardprice{4}
	\cardtitle{Nachbau}
	\cardcontent{Entsorge eine Handkarte. Nimm dafür eine Karte vom Vorrat, die genau \coin[1] mehr kostet als die entsorgte Karte. Lege die neue Karte ab. Entsorge dann eine weitere Handkarte und nimm eine Karte, die genau \coin[1] mehr kostet. Lege auch diese Karte ab. Ist im Vorrat keine Karte, die genau \coin[1] mehr kostet, musst du die Handkarte trotzdem entsorgen, erhältst dafür aber nichts. Du kannst keine Karte vom Vorrat nehmen und diese gleich wieder entsorgen, da du sie ablegen musst.}
\end{tikzpicture}
\hspace{-0.6cm}
\begin{tikzpicture}
	\card
	\cardstrip
	\cardbanner{banner/blue.png}
	\cardicon{icons/coin.png}
	\cardprice{4}
	\cardtitle{\footnotesize{Pferdehändler}}
	\cardcontent{Wenn du diese Karte ausspielst, erhältst du + 1 Kauf und +\coin[3]. Dann legst du 2 Handkarten ab. Wenn du weniger als 2 Handkarten hast, legst du so viele Karten ab wie möglich.

	\medskip

	Wenn ein Mitspieler eine Angriffskarte ausspielt, darfst du diese Karte aus deiner Hand aufdecken. Dann legst du den \emph{PFERDEHÄNDLER} zur Seite und der Angriff wird ausgeführt. Zu Beginn deines nächsten Zuges ziehst du eine Karte nach und nimmst den (oder die) zur Seite gelegten \emph{PFERDEHÄNDLER} wieder auf die Hand.}
\end{tikzpicture}
\hspace{-0.6cm}
\begin{tikzpicture}
	\card
	\cardstrip
	\cardbanner{banner/white.png}
	\cardicon{icons/coin.png}
	\cardprice{4}
	\cardtitle{Turnier}
	\cardcontent{Wenn du eine \emph{PROVINZ} aus deiner Hand ablegst, nimmst du dir entweder ein \emph{HERZOGTUM} vom Vorrat oder eine beliebige Karte vom Preisstapel. Ist der Stapel, für den du dich entscheidest, leer, nimmst du keine Karte. Lege die Karte, die du nimmst, oben auf deinen Nachziehstapel.

	\medskip

	Dann dürfen alle Mitspieler eine \emph{PROVINZ} aus ihrer Hand aufdecken. Wenn keiner eine \emph{PROVINZ} aufdeckt, erhältst du + 1 Karte und +\coin[1].}
\end{tikzpicture}
\hspace{-0.6cm}
\begin{tikzpicture}
	\card
	\cardstrip
	\cardbanner{banner/white.png}
	\cardicon{icons/coin.png}
	\cardprice{5}
	\cardtitle{Ernte}
	\cardcontent{Decke die obersten 4 Karten von deinem Nachziehstapel auf. Für jede aufgedeckte Karte mit unterschiedlichem Namen, erhältst du +\coin[1]. Kannst du (auch nach dem  Mischen des Ablagestapels) weniger als 4 Karten aufdecken, deckst du nur so viele Karten auf, wie möglich.}
\end{tikzpicture}
\hspace{-0.6cm}
\begin{tikzpicture}
	\card
	\cardstrip
	\cardbanner{banner/gold.png}
	\cardicon{icons/coin.png}
	\cardprice{5}
	\cardtitle{Füllhorn}
	\cardcontent{Diese Karte ist eine Geldkarte mit dem Basiswert 0. Wenn du das \emph{FÜLLHORN} ausspielst, zählst du zunächst, wie viele Karten \emph{mit unterschiedlichen Namen} im Spiel sind. Im Spiel sind: Geld- und Aktionskarten, die du in diesem Zug ausgespielt hast sowie evtl. bei dir ausliegende Dauerkarten (aus \emph{Seaside}). Nicht im Spiel dagegen sind Karten, die du in diesem Zug entsorgt hast. Nimm dir eine Karte aus dem Vorrat, die maximal soviel kostet, wie du Karten mit unterschiedlichen Namen im Spiel hast.

	\medskip

	Nimmst du eine Punktekarte, entsorgst du das \emph{FÜLLHORN}. Nimmst oder erhältst du eine Punktekarte auf andere Weise, entsorgst du das \emph{FÜLLHORN} nicht.}
\end{tikzpicture}
\hspace{-0.6cm}
\begin{tikzpicture}
	\card
	\cardstrip
	\cardbanner{banner/white.png}
	\cardicon{icons/coin.png}
	\cardprice{5}
	\cardtitle{Harlekin}
	\cardcontent{Jeder Mitspieler, beginnend mit deinem linken Nachbarn, deckt die oberste Karte seines Nachziehstapels auf und legt sie ab.

	\medskip

	Ist es eine Punktekarte (auch eine kombinierte), muss sich der Spieler einen \emph{FLUCH} nehmen. Wird der Vorrat an \emph{FLÜCHEN} dabei aufgebraucht, erhalten die Spieler, für die kein \emph{FLUCH} mehr vorhanden ist, keinen \emph{FLUCH}.

	\medskip

	Ist es keine Punktekarte, darfst du wählen: Entweder muss sich der Spieler eine Karte mit gleichem Namen (sofern vorhanden) aus dem Vorrat nehmen und ablegen oder du nimmst eine Karte mit gleichem Namen (sofern vorhanden) aus dem Vorrat und legst sie ab.}
\end{tikzpicture}
\hspace{-0.6cm}
\begin{tikzpicture}
	\card
	\cardstrip
	\cardbanner{banner/white.png}
	\cardicon{icons/coin.png}
	\cardprice{5}
	\cardtitle{Treibjagd}
	\cardcontent{Decke alle deine Handkarten auf. Decke dann so lange Karten vom Nachziehstapel auf, bis du die erste Karte aufdeckst, die einen Namen hat, der nicht bei deinen Handkarten dabei ist. Nimm diese Karte und die aufgedeckten Handkarten auf die Hand und lege die anderen aufgedeckten Karten ab. Kannst du (auch nach dem Mischen des Ablagestapels) eine solche Karte nicht aufdecken, nimmst du nur deine aufgedeckten Handkarten wieder auf die Hand.}
\end{tikzpicture}
\hspace{-0.6cm}
\begin{tikzpicture}
	\card
	\cardstrip
	\cardbanner{banner/green.png}
	\cardicon{icons/coin.png}
	\cardprice{6}
	\cardtitle{Festplatz}
	\cardcontent{Diese Karte ist eine Punktekarte und hat bis zum Ende des Spiels keine Funktion. Bei der Wertung des Spiels erhältst du pro 5 Karten mit unterschiedlichen Namen in deinem Kartensatz (Handkarten, Nachzieh- und Ablagestapel) 2 Siegpunkte. Ein Kartensatz mit bis zu 4 unterschiedlichen Kartennamen bringt dir zum Beispiel keine Punkte, Kartensätze mit 10 bis 14 unterschiedlichen Namen dagegen 4 \victorypoint.}
\end{tikzpicture}
\hspace{-0.6cm}
\begin{tikzpicture}
	\card
	\cardstrip
	\cardbanner{banner/white.png}
	\cardicon{icons/coin.png}
	\cardprice{0*}
	\cardtitle{Preiskarten}
	\cardcontent{\tiny{\begin{Spacing}{1}
	\vspace{1em}
	\emph{Diadem:} Das \emph{DIADEM} ist eine Geldkarte. Für jede in deinem Zug nicht verbrauchte Aktion erhältst du +\coin[1] für die Kaufphase. Hast du zum Beispiel in der Aktionsphase gar keine  Aktionskarte ausgespielt, erhältst du +\coin[1] für deine freie Aktion. Hast du zum Beispiel nur das \emph{BAUERNDORF} ausgespielt, erhältst du +\coin[2] für die beiden Aktionen des \emph{BAUERNDORFS}, da du deine freie Aktion verbraucht hast.

	\smallskip

	\emph{Ein Sack voll Gold:} Ist der Nachziehstapel leer, wenn du dir ein \emph{GOLD} nimmst, legst du das \emph{GOLD} auf die leere Stelle. Es ist dann die einzige Karte in deinem Nachziehstapel.

	\smallskip

	\emph{Gefolge:} Ist kein \emph{ANWESEN} mehr im Vorrat, erhältst du keins. Beginnend mit deinem linken Nachbarn nimmt sich jeder Mitspieler einen \emph{FLUCH}. Mitspieler, die mehr als  3 Karten auf der Hand haben, legen außerdem Karten ab, bis sie nur noch 3 Karten auf der Hand haben. Ist für einen Spieler kein \emph{FLUCH} mehr im Vorrat, erhält der Spieler keinen. Handkarten muss er dennoch ablegen, wenn er mehr als 3 Karten hat.

	\smallskip

	\emph{Prinzessin:} Die Anweisung, dass jede Karte \coin[2] weniger kostet, wenn die \emph{PRINZESSIN} im Spiel ist, betrifft die Karten auf der Hand und in den Ablage- und Nachziehstapeln aller Spieler sowie alle Karten im Vorrat. Wird die \emph{PRINZESSIN} auf einen \emph{THRONSAAL} folgend ausgespielt, kosten alle Karten trotzdem nur \coin[2] weniger, da die \emph{PRINZESSIN} nur einmal im Spiel ist.

	\smallskip

	\emph{Streitross:} Diese Karte beinhaltet 4 Anweisungen, von denen du 2 unterschiedliche wählst. Dann führst du die Anweisungen der Reihenfolge auf der Karte nach aus. Du darfst auch einen Anweisung wählen, die du nicht (oder nur teilweise) erfüllen kannst, z. B. wenn nur noch 3 \emph{SILBER} im Vorrat sind. Du darfst deinen Nachziehstapel nicht durchsehen, bevor du ihn ablegst.
	\end{Spacing}}}
\end{tikzpicture}
\hspace{-0.6cm}
\begin{tikzpicture}
	\card
	\cardstrip
	\cardbanner{banner/white.png}
	\cardtitle{\scriptsize{Empfohlene 10er Sätze\qquad}}
	\cardcontent{\emph{Kopfgeld} (Reiche Ernte + \textit{Basisspiel}):\\
	Ernte, Füllhorn, Menagerie, Treibjagd, Turnier (+ Preiskarten), \textit{Geldverleiher}, \textit{Jahrmarkt}, \textit{Keller}, \textit{Miliz}, \textit{Schmiede}

	\smallskip

	\emph{Böses Omen} (Reiche Ernte + \textit{Basisspiel}):\\
	Füllhorn, Harlekin, Nachbau, Wahrsagerin, Weiler, \textit{Abenteurer}, \textit{Bürokrat}, \textit{Laboratorium}, \textit{Spion}, \textit{Thronsaal}

	\smallskip

	\emph{Wanderzirkus} (Reiche Ernte + \textit{Basisspiel}):\\
	Bauerndorf, Festplatz, Harlekin, Junge Hexe (+ Kanzler als Bannstapel), Pferdehändler, \textit{Festmahl}, \textit{Laboratorium}, \textit{Markt}, \textit{Umbau}, \textit{Werkstatt}

	\emph{Wer zuletzt lacht} (Reiche Ernte + \textit{Die Intrige}):\\
	Bauerndorf, Ernte, Harlekin, Pferdehändler, Treibjagd, \textit{Adlige}, \textit{Handlanger}, \textit{Lakai}, \textit{Trickser}, \textit{Verwwalter}

	\smallskip

	\emph{Würze des Lebens}	(Reiche Ernte + \textit{Die Intrige}):\\
	Festplatz, Füllhorn, Junge Hexe (+ Wunschbrunnen als Bannstapel), Nachbau, Turnier (+ Preiskarten), \textit{Bergwerk}, \textit{Burghof}, \textit{Große Halle}, \textit{Kupferschmied}, \textit{Tribut} 

	\smallskip

	\emph{Kleine  Siege} (Reiche Ernte + \textit{Die Intrige}):\\
	Nachbau, Treibjagd, Turnier (+ Preiskarten), Wahrsagerin, Weiler, \textit{Große Halle}, \textit{Handlanger}, \textit{Harem}, \textit{Herzog}, \textit{Verschwörer}}
\end{tikzpicture}
\hspace{0.6cm}

	    % Basic settings for this card set
\renewcommand{\cardcolor}{hinterlands}
\renewcommand{\cardextension}{Erweiterung V}
\renewcommand{\cardextensiontitle}{Hinterland}
\renewcommand{\seticon}{hinterlands.png}

\clearpage
\newpage
\section{\cardextension \ - \cardextensiontitle \ (Hans Im Glück 2011)}

\begin{tikzpicture}
	\card
	\cardstrip
	\cardbanner{banner/white.png}
	\cardicon{icons/coin.png}
	\cardprice{3}
	\cardtitle{Aufbau}
	\cardcontent{Zuerst musst du eine Karte aus deiner Hand entsorgen. Den gerade ausgespielten Ausbau kannst du nicht entsorgen, da du ihn nicht mehr auf der Hand hast. Hast du keine Karte mehr auf der Hand, kannst du keine Karte entsorgen und darfst dir auch keine Karten nehmen. Wenn du eine Karte entsorgt hast nimmst du dir 2 Karten, eine Karte, die genau \coin[1] mehr kostet, als die entsorgte Karte und eine Karte, die genau \coin[1] weniger kostet, als die entsorgte Karte. Du nimmst beide Karten aus dem Vorrat. Kannst du eine der beiden Karten nicht nehmen, weil keine Karte mit genau den geforderten Kosten im Vorrat ist, nimmst du dir trotzdem die andere Karte (wenn eine im Vorrat ist). Du legst die beiden Karten sofort auf deinen Nachziehstapel, anstatt auf deinen Ablagestapel. Du darfst die Karten in beliebiger Reihenfolge nehmen. Entsorgst du z. B. eine Herzogin (\coin[2]), so nimmst du dir eine Karte, die genau \coin[3] kostet (z. B. die Oase) und eine Karte, die genau \coin[1] kostet. Da es (üblicherweise) keine Karte gibt, die \coin[1] kostet, nimmst du dir keine weitere Karte. Entsorgst du z. B. ein Kupfer, so müsstest du dir eine Karte mit den Kosten \coin[1] und eine Karte mit den Kosten \coin[-1] nehmen, da es (üblicherweise) beide Karten nicht gibt, nimmst du dir keine Karte.}
\end{tikzpicture}
\hspace{-0.6cm}
\begin{tikzpicture}
	\card
	\cardstrip
	\cardbanner{banner/gold.png}
	\cardicon{icons/coin.png}
	\cardprice{5}
	\cardtitle{Blutzoll}
	\cardcontent{Diese Karte ist eine Geldkarte mit dem Wert \coin[1], wie ein Kupfer. Wenn du den Blutzoll ausspielst, darfst du dir ein Kupfer direkt auf die Hand nehmen. Du kannst dieses Kupfer also noch in diesem Zug ausspielen. Ist kein Kupfer mehr im Vorrat, so nimmst du dir kein Kupfer. Wenn du den Blutzoll nimmst oder kaufst, muss sich, beginnend mit dem Spieler links von dir, reihum jeder Mitspieler einen Fluch nehmen. Sind nicht mehr genügend Flüche im Vorrat, werden nur so viele verteilt wie möglich. Blutzoll ist keine Angriffskarte und kann z. B. nicht mit dem Burggraben abgewehrt werden.}
\end{tikzpicture}
\hspace{-0.6cm}
\begin{tikzpicture}
	\card
	\cardstrip
	\cardbanner{banner/white.png}
	\cardicon{icons/coin.png}
	\cardprice{5}
	\cardtitle{Botschaft}
	\cardcontent{Wenn du die Botschaft ausspielst, ziehst du zunächst 5 Karten auf deine Hand nach. Wenn du, auch nach dem Mischen deines Ablagestapels, nur weniger als 5 Karten nachziehen kannst, ziehst du nur so viele Karten nach wie möglich. Dann legst du 3 Karten aus deiner Hand ab. Du kannst also auch Karten ablegen, die du gerade nachgezogen hast. Wenn du die Botschaft nimmst oder kaufst, muss sich, beginnend mit dem Spieler links von dir, reihum jeder Spieler ein Silber nehmen. Ist kein Silber mehr im Vorrat, nimmt der Spieler kein Silber.}
\end{tikzpicture}
\hspace{-0.6cm}
\begin{tikzpicture}
	\card
	\cardstrip
	\cardbanner{banner/white.png}
	\cardicon{icons/coin.png}
	\cardprice{4}
	\cardtitle{Edler Räuber}
	\cardcontent{Wenn du diese Karte ausspielst, erhältst du +\coin[1]. Wenn du diese Karte ausspielst \emph{und auch} wenn du diese Karte kaufst, muss beginnend bei dem Spieler links von dir, reihum jeder Mitspieler die obersten beiden Karten von seinem Nachziehstapel aufdecken. Hat der Spieler 1 Silber oder 1 Gold aufgedeckt, so muss er dieses entsorgen. Hat der Spieler ein Silber und ein Gold (oder auch 2 Silber bzw. 2 Gold) aufgedeckt, entscheidest du welches davon er entsorgen muss. Die andere aufgedeckte Karte legt er jeweils auf seinen Ablagestapel. Hat der Spieler weder Silber noch Gold aufgedeckt, so legt er beide Karten ab. Hat der Spieler überhaupt keine Geldkarte aufgedeckt, so legt er beide Karten ab \emph{und} nimmt sich ein Kupfer. Du musst alle entsorgten Silber und Gold nehmen. Wenn du den Edlen Räuber ausspielst, können deine Mitspieler mit Reaktionskarten, wie z. B. Burggraben (Dominion – Basisspiel), auf den Angriff reagieren. Wenn du den Edlen Räuber jedoch kaufst, dürfen deine Mitspieler keine Reaktionskarten aufdecken, da du den Edlen Räuber nicht ausgespielt hast.}
\end{tikzpicture}
\hspace{-0.6cm}
\begin{tikzpicture}
	\card
	\cardstrip
	\cardbanner{banner/blue.png}
	\cardicon{icons/coin.png}
	\cardprice{4}
	\cardtitle{\tiny{Fahrender Händler}}
	\cardcontent{\miniscule{\begin{Spacing}{1}
	\vspace{1em}
	Wenn du diese Karte ausspielst, musst du eine Karte aus deiner Hand entsorgen. Wenn du das machst, nimmst du dir so viele Silber vom Vorrat, wie die Geld-Kosten der entsorgten Karte. Entsorgst du z. B. eine Karte, die 3 Geld kostet, so nimmst du dir 3 Silber, entsorgst du z. B. ein Kupfer (0 Geld), nimmst du dir kein Silber. Sind im Vorrat nicht mehr genügend Silber, nimmst du dir nur so viele wie möglich. Kannst du keine Karte aus der Hand entsorgen, nimmst du dir auch kein Silber. Sind die Kosten der Karten verändert (z. B. durch die Fernstraße), nimmst du dir nur so viele Silber, wie die entsorgte Karte in diesem Moment kostet. Entsorgst du z. B. ein Anwesen, während die Fernstraße im Spiel ist, nimmst du dir nur 1 Silber. Andere Kosten außer Geld, z. B. Tränke aus Dominion – Die Alchemisten, haben keine Auswirkungen darauf, wieviele Silber du dir nimmst.

	Der Fahrende Händler ist auch eine Reaktionskarte. Immer wenn du eine Karte nimmst, kannst du die Karte aus deiner Hand aufdecken. Du kannst den Fahrenden Händler aufdecken, wenn du von einem Angriff betroffen bist oder auch, wenn du in deinem eigenen Zug freiwillig oder unfreiwillig eine Karte nimmst oder kaufst. Wenn du das machst, nimmst du dir anstatt der Karte, die du dir eigentlich nehmen müsstest, ein Silber. Die andere Karte nimmst du nicht. Spielt z. B. ein Mitspieler die Hexe (Dominion – Basisspiel) aus, so kannst du den Fahrenden Händler aus deiner Hand aufdecken. Wenn du das machst, nimmst du dir keinen Fluch, sondern stattdessen ein Silber. Wenn du dir auf diese Art ein Silber nimmst, legst du es immer auf deinen Ablagestapel, auch wenn du die andere Karte z. B. auf die Hand genommen oder auf deinen Nachziehstapel gelegt hättest. Wenn du eine Karte kaufst, dir jedoch stattdessen ein Silber nimmst, musst du die ursprünglich gekaufte Karte trotzdem bezahlen. Wenn die Karte, die du ursprünglich nehmen müsstest, beim Nehmen eine Anweisung auslöst (\enquote{Wenn du diese Karte nimmst ...}), wird diese nicht ausgelöst, weil du die Karte nicht nimmst. Wenn du z. B. einen Blutzoll kaufst, kannst du den Fahrenden Händler aus deiner Hand aufdecken und statt dem Blutzoll ein Silber nehmen. Kein Mitspieler muss sich einen Fluch nehmen, weil du den Blutzoll nicht genommen hast. Wenn die Karte, die du ursprünglich nehmen müsstest, beim Kaufen eine Anweisung auslöst (\enquote{Wenn du diese Karte kaufst ...}), wird diese ausgelöst, weil der Kauf durchgeführt ist, obwohl du die Karte nicht nimmst. Wenn du z. B. ein Fruchtbares Land kaufst, kannst du den Fahrenden Händler aus deiner Hand aufdecken und statt dem Fruchtbaren Land ein Silber nehmen. Die zusätzliche Anweisung auf dem Fruchtbaren Land wird ausgelöst, weil du die Karte gekauft hast (selbst wenn du sie nicht wie üblich genommen hast). Du entsorgst also eine Karte aus deiner Hand und nimmst dir eine andere Karte, die 2 Geld kostet, genauso als hättest du das Fruchtbare Land nach deinem Kauf auch wirklich genommen.\end{Spacing}}}
\end{tikzpicture}
\hspace{-0.6cm}
\begin{tikzpicture}
	\card
	\cardstrip
	\cardbanner{banner/white.png}
	\cardicon{icons/coin.png}
	\cardprice{5}
	\cardtitle{Feilscher}
	\cardcontent{Wenn du diese Karte ausspielst erhältst du +\coin[2]. Wenn diese Karte im Spiel ist und du eine Karte kaufst, musst du dir immer eine zusätzliche Karte nehmen, die billiger ist als die gerade gekaufte und die keine Punktekarte ist. Die zusätzliche Karte nimmst du aus dem Vorrat und legst sie auf deinen Ablagestapel. Hast du einen Feilscher im Spiel und kaufst eine Provinz, so kannst du dir z. B. ein Gold nehmen. Du nimmst dir nur eine zusätzliche Karte, wenn du eine Karte kaufst. Wenn du eine Karte auf eine andere Art nimmst, nimmst du dir keine weitere Karte. Ist im Vorrat keine billigere Karte, als die gerade gekaufte, so nimmst du dir keine zusätzliche Karte. Diese Anweisung gilt nur solange der Feilscher im Spiel ist. Spielst du den Feilscher z. B. auf einen Thronsaal, spielst du die Karte zwar zweimal aus, die Karte ist jedoch nur einmal im Spiel. Du darfst dir in diesem Fall also nicht 2 zusätzliche Karten nehmen. Hast du 2 Feilscher im Spiel, so darfst (und musst) du dir auch 2 zusätzliche Karten nehmen. Kombinierte Punktekarten sind auch Punktekarten und dürfen nicht mit dem Feilscher genommen werden.}
\end{tikzpicture}
\hspace{-0.6cm}
\begin{tikzpicture}
	\card
	\cardstrip
	\cardbanner{banner/white.png}
	\cardicon{icons/coin.png}
	\cardprice{5}
	\cardtitle{Fernstrasse}
	\cardcontent{Wenn du diese Karte ausspielst, ziehst du zuerst eine Karte nach und erhältst +1 Aktion. Solange diese Karte im Spiel ist, kosten alle Karten \coin[1] weniger (niemals jedoch weniger als \coin[0]). Das gilt für alle Karten im Vorrat, auf den Händen der Spieler und in deren Kartenstapeln. Spielst du z. B. die Fernstraße, danach den Aufbau und entsorgst ein Kupfer aus deiner Hand, so kannst du dir z. B. ein Anwesen nehmen. Das Anwesen kostet nur noch 1 Geld, während das Kupfer immer noch 0 Geld kostet. Diese Anweisung gilt nur, solange die Fernstraße im Spiel ist. Spielst du die Fernstraße z. B. auf einen Thronsaal, spielst du die Karte zwar zweimal aus, die Karte ist jedoch nur einmal im Spiel. In diesem Fall kosten also alle Karten immer noch nur \coin[1] weniger. Hast du 2 Fernstraßen im Spiel, so kosten alle Karten \coin[2] weniger (immer noch bis zu einem Minimum von \coin[0]).}
\end{tikzpicture}
\hspace{-0.6cm}
\begin{tikzpicture}
	\card
	\cardstrip
	\cardbanner{banner/green.png}
	\cardicon{icons/coin.png}
	\cardprice{6}
	\cardtitle{\tiny{Fruchtbares Land}}
	\cardcontent{Diese Königreichkarte ist eine Punktekarte, keine Aktionskarte. Sie hat bis zum Ende des Spiels keine Funktion. Bei der Wertung zählt sie 2 Siegpunkte. Wenn du das Fruchtbare Land kaufst (nicht, wenn du die Karte auf eine andere Art nimmst), musst du eine Karte aus deiner Hand entsorgen und dir eine Karte nehmen, die genau \coin[2] mehr kostet als die entsorgte Karte. Wenn im Vorrat keine Karte ist, die genau \coin {2} mehr kostet als die entsorgte Karte, oder du keine Karte entsorgen kannst, nimmst du dir keine Karte (außer dem gerade gekauften Fruchtbaren Land).
	
	\smallskip

	Im Spiel zu 3. und 4. werden 12 Karten Fruchtbares Land verwendet, im Spiel zu 2. werden 8 Karten Fruchtbares Land verwendet.}
\end{tikzpicture}
\hspace{-0.6cm}
\begin{tikzpicture}
	\card
	\cardstrip
	\cardbanner{banner/white.png}
	\cardicon{icons/coin.png}
	\cardprice{5}
	\cardtitle{Gasthaus}
	\cardcontent{Wenn du diese Karte ausspielst, ziehst du zuerst 2 Karten nach und erhältst +2 Aktionen. Dann legst du 2 Karten aus deiner Hand ab. Du kannst auch Karten ablegen, die du gerade nachgezogen hast.
	
	\smallskip 

	Wenn du diese Karte nimmst oder kaufst, darfst du dir sofort deinen Nachziehstapel durchsehen (was üblicherweise nicht erlaubt ist) und beliebig viele Aktionskarten daraus in deinen Nachziehstapel einmischen. Du darfst auch das gerade genommene Gasthaus selbst einmischen, da es, wenn die Anweisung ausgeführt wird, bereits auf dem Ablagestapel liegt. Kombinierte Aktionskarten sind auch Aktionskarten. Du musst die Aktionskarten, die du in deinen Nachziehstapel einmischst, deinen Mitspielern vorzeigen.}
\end{tikzpicture}
\hspace{-0.6cm}
\begin{tikzpicture}
	\card
	\cardstrip
	\cardbanner{banner/white.png}
	\cardicon{icons/coin.png}
	\cardprice{4}
	\cardtitle{\scriptsize{Gewürzhändler}}
	\cardcontent{Du \emph{darfst} eine Karte aus deiner Hand entsorgen. Wenn du das machst, wählst du eine der beiden angegebenen Kombinationen. \emph{Entweder} ziehst du 2 Karten nach und erhältst +1 Aktion oder du erhältst +\coin[2] und +1 Kauf. Wenn du keine Karte aus deiner Hand entsorgen kannst oder willst, darfst du auch keine der angegebenen Anweisungen ausführen.}
\end{tikzpicture}
\hspace{-0.6cm}
\begin{tikzpicture}
	\card
	\cardstrip
	\cardbanner{banner/white.png}
	\cardicon{icons/coin.png}
	\cardprice{6}
	\cardtitle{Grenzdorf}
	\cardcontent{Wenn du das Grenzdorf ausspielst ziehst du 1 Karte von deinem Nachziehstapel und erhältst +2 Aktionen.\\ \smallskip Wenn du das Grenzdorf nimmst oder kaufst, musst du dir zusätzlich eine weitere Karte nehmen, die weniger kostet als das Grenzdorf. Üblicherweise nimmst du dir eine Karte, die bis zu \coin[5] kostet. Sollte das Grenzdorf jedoch weniger als \coin[6] gekostet haben, z. B. weil die Karte Fernstraße im Spiel ist, darfst du dir nur eine Karte nehmen, die weniger kostet als das Grenzdorf momentan. Du nimmst dir die zusätzliche Karte nur, wenn du das Grenzdorf nimmst oder kaufst, nicht jedesmal, wenn du es ausspielst.}
\end{tikzpicture}
\hspace{-0.6cm}
\begin{tikzpicture}
	\card
	\cardstrip
	\cardbanner{banner/white.png}
	\cardicon{icons/coin.png}
	\cardprice{2}
	\cardtitle{Herzogin}
	\cardcontent{Wenn du die Herzogin ausspielst, erhältst du zunächst +\coin[2]. Dann sieht sich, beginnend mit dem Spieler links von dir, reihum jeder Spieler (auch du selbst) die oberste Karte von seinem Nachziehstapel an und entscheidet selbst, ob er sie ablegt oder zurück auf seinen Nachziehstapel legt. Kann ein Spieler, auch nach dem Mischen seines Ablagestapels, keine Karte ansehen, so sieht er keine Karte an. 

	\smallskip

	Wenn die Herzogin als eine der 10 Königreichkarten verwendet wird, darf sich jeder Spieler, immer wenn er ein Herzogtum nimmt oder kauft, zusätzlich eine Karte Herzogin nehmen.}
\end{tikzpicture}
\hspace{-0.6cm}
\begin{tikzpicture}
	\card
	\cardstrip
	\cardbanner{banner/white.png}
	\cardicon{icons/coin.png}
	\cardprice{5}
	\cardtitle{Kartograph}
	\cardcontent{Du ziehst zunächst 1 Karte von deinem Nachziehstapel und erhältst +1 Aktion. Dann siehst du dir die obersten 4 Karten von deinem Nachziehstapel an. (Wenn du auch nach dem Mischen deines Ablagestapels nur weniger als 4 Karten ansehen kannst, siehst du nur so viele Karten an wie möglich.) Lege beliebig viele (also 0-4) dieser Karten ab. Lege die übrigen dieser angesehenen Karten in beliebiger Reihenfolge zurück auf deinen Nachziehstapel. Du musst die Karten, die du zurück legst, deinen Mitspielern nicht zeigen.}
\end{tikzpicture}
\hspace{-0.6cm}
\begin{tikzpicture}
	\card
	\cardstrip
	\cardbanner{banner/goldblue.png}
	\cardicon{icons/coin.png}
	\cardprice{2}
	\cardtitle{Katzengold}
	\cardcontent{Diese Karte ist gleichzeitig eine Geld- und eine Reaktionskarte. Katzengold kann, wie jede andere Geldkarte, in der Kaufphase ausgespielt werden. Wenn du in diesem Zug zum ersten Mal ein Katzengold ausspielst, ist es \coin[1] wert, wie ein Kupfer. Jedes weitere Katzengold, das du in diesem Zug ausspielst, ist \coin[4] wert. Spielst du z. B. 3 Karten Katzengold in deiner Kaufphase aus, so hast du \coin[9] (\coin[1] + \coin[4] + \coin[4]) für deinen Kauf bzw. deine Käufe zur Verfügung. Katzengold ist auch eine Reaktionskarte. Immer wenn ein anderer Spieler eine Provinz nimmt oder kauft, darfst du Katzengold aus deiner Hand entsorgen. Wenn du das machst, nimmst du dir ein Gold und legst es sofort auf deinen Nachziehstapel. Ist kein Gold mehr im Vorrat, erhältst du kein Gold, du kannst das Katzengold jedoch trotzdem entsorgen.}
\end{tikzpicture}
\hspace{-0.6cm}
\begin{tikzpicture}
	\card
	\cardstrip
	\cardbanner{banner/white.png}
	\cardicon{icons/coin.png}
	\cardprice{3}
	\cardtitle{Komplott}
	\cardcontent{\emph{Errata:} Der Kartentext ist falsch, es sollte \enquote{Zu Beginn deiner Aufräumphase darfst du eine Aktionskarte wählen, die du im Spiel hast. Wenn du die gewählte Karte in diesem Zug ablegst, lege sie auf deinen Nachziehstapel.} statt \enquote{Zu Beginn deiner Aufräumphase darfst du eine deiner ausgespielten Aktionskarten wählen. Wenn du die gewählte Karte in dieser Aufräumphase ablegen würdest, darfst du sie stattdessen auf deinen Nachziehstapel legen.} heißen. 

	\smallskip

	Wenn du diese Karte ausspielst, ziehst du zuerst 1 Karte nach und erhältst +1 Aktion. Zu Beginn deiner Aufräumphase darfst du eine Aktionskarte wählen, die du im Spiel hast. Du kannst das auch gerade ausgespielte Komplott selbst wählen. Statt die gewählte Aktionskarte in dieser Aufräumphase auf deinen Ablagestapel zu legen, darfst du sie auf deinen Nachziehstapel legen. Das machst du, bevor du Karten für die nächste Runde nachziehst. Wenn du die gewählte Aktionskarte nicht in dieser Aufräumphase ablegst (z. B. eine Dauerkarte aus Dominion – Seaside), kannst du sie auch nicht auf deinen Nachziehstapel legen.

	\smallskip
	
	\emph{Anmerkung:} Das \enquote{Ablegen} wird dennoch ausgelöst, bevor die Karte auf den Nachziehstapel gelegt wird.}
\end{tikzpicture}
\hspace{-0.6cm}
\begin{tikzpicture}
	\card
	\cardstrip
	\cardbanner{banner/white.png}
	\cardicon{icons/coin.png}
	\cardprice{4}
	\cardtitle{\scriptsize{Lebenskünstler}}
	\cardcontent{Diese Karte hat 4 verschiedene Anweisungen, die du nacheinander ausführen musst (nur die letzte ist optional). Zuerst nimmst du dir ein Silber vom Vorrat und legst es auf deinen Ablagestapel. Ist kein Silber mehr im Vorrat, nimmst du dir kein Silber. Danach siehst du dir die oberste Karte deines Nachziehstapels an und entscheidest, ob du sie zurück auf deinen Nachziehstapel legst oder ablegst. Wenn dein Nachzieh- und dein Ablagestapel leer sind, siehst du dir keine Karte an. Dann ziehst du solange Karten nach, bis du 5 Karten auf der Hand hast. Hast du bereits 5 oder mehr Karten auf der Hand, ziehst du keine Karten nach. Kannst du, auch nach dem Mischen deines Ablagestapels, nicht genügend Karten nachziehen, so ziehst du nur so viele Karten nach wie möglich. Zuletzt darfst du eine Karte aus deiner Hand entsorgen, die keine Geldkarte ist. Kombinierte Geldkarten sind auch Geldkarten.}
\end{tikzpicture}
\hspace{-0.6cm}
\begin{tikzpicture}
	\card
	\cardstrip
	\cardbanner{banner/white.png}
	\cardicon{icons/coin.png}
	\cardprice{5}
	\cardtitle{Mandarin}
	\cardcontent{Wenn du diese Karte ausspielst erhältst du zuerst +\coin[3]. Dann musst du eine Karte aus deiner Hand auf deinen Nachziehstapel legen. Wenn du keine Karte mehr auf der Hand hast, legst du keine Karte auf deinen Nachziehstapel.
	
	\smallskip
	
	Wenn du diese Karte nimmst oder kaufst, musst du alle Geldkarten, die du im Spiel hast, in beliebiger Reihenfolge zurück auf deinen Nachziehstapel legen. Du musst deinen Mitspielern nicht zeigen, in welcher Reihenfolge du die Karten zurück legst. Du musst alle Geldkarten, die du im Spiel hast, zurück legen. Geldkarten auf deiner Hand sind nicht im Spiel. Du legst sie also nicht auf den Nachziehstapel. Du hast jedoch alle Münzen, der ausgespielten Geldkarten in dieser Kaufphase zur Verfügung. Hast du z. B. +1 Kauf und 4 Gold ausgespielt und kaufst den Mandarin, so legst du die 4 Gold zurück auf deinen Nachziehstapel und hast trotzdem noch \coin[7] für den zweiten Kauf zur Verfügung. Du legst deine ausgespielten Geldkarten auch auf den Nachziehstapel zurück, wenn du den Mandarin nimmst, wobei du üblicherweise nur in der Kaufphase Geldkarten im Spiel hast.}
\end{tikzpicture}
\hspace{-0.6cm}
\begin{tikzpicture}
	\card
	\cardstrip
	\cardbanner{banner/white.png}
	\cardicon{icons/coin.png}
	\cardprice{5}
	\cardtitle{Markgraf}
	\cardcontent{Zuerst ziehst du 3 Karten nach und erhältst +1 Kauf. Dann muss, beginnend mit dem Spieler links von dir, reihum jeder Mitspieler eine Karte nachziehen und dann solange Karten aus seiner Hand ablegen, bis er nur noch 3 Karten auf der Hand hat. Hat ein Spieler, auch nachdem er eine Karte nachgezogen hat, nur 3 oder weniger Karten auf der Hand, so legt er keine Karte ab.}
\end{tikzpicture}
\hspace{-0.6cm}
	\begin{tikzpicture}
	\card
	\cardstrip
	\cardbanner{banner/white.png}
	\cardicon{icons/coin.png}
	\cardprice{4}
	\cardtitle{\footnotesize{Nomadencamp}}
	\cardcontent{Wenn du diese Karte ausspielst, erhältst du +1 Kauf und +\coin[2]. Wenn du das Nomadencamp nimmst, musst du es sofort auf deinen Nachziehstapel legen, statt wie üblich auf den Ablagestapel.}
\end{tikzpicture}
\hspace{-0.6cm}
\begin{tikzpicture}
	\card
	\cardstrip
	\cardbanner{banner/white.png}
	\cardicon{icons/coin.png}
	\cardprice{3}
	\cardtitle{Oase}
	\cardcontent{Du ziehst zuerst 1 Karte nach und erhältst +1 Aktion und +\coin[1]. Dann legst du eine Karte aus deiner Hand ab. Du darfst auch die gerade nachgezogene Karte ablegen. Wenn du keine Karte nachziehen kannst, weil dein Nachzieh- und Ablagestapel leer sind, musst du trotzdem eine Karte aus deiner Hand ablegen.}
\end{tikzpicture}
\hspace{-0.6cm}
\begin{tikzpicture}
	\card
	\cardstrip
	\cardbanner{banner/white.png}
	\cardicon{icons/coin.png}
	\cardprice{3}
	\cardtitle{Orakel}
	\cardcontent{Reihum, beginnend mit dem Spieler links von dir, muss jeder Spieler (auch du selbst) die obersten beiden Karten von seinem Nachziehstapel aufdecken. Du entscheidest, ob er beide Karten ablegen oder zurück auf seinen Nachziehstapel legen muss. Der Spieler entscheidet selbst, in welcher Reihenfolge er die Karten auf seinen Nachziehstapel legt. Dann erst ziehst du 2 Karten nach. Hast du also deine eigenen Karten zurück auf deinen Nachziehstapel gelegt, so ziehst du diese Karten nach.}
\end{tikzpicture}
\hspace{-0.6cm}
\begin{tikzpicture}
	\card
	\cardstrip
	\cardbanner{banner/gold.png}
	\cardicon{icons/coin.png}
	\cardprice{5}
	\cardtitle{Schatztruhe}
	\cardcontent{Diese Geldkarte ist \coin[3] wert, wie ein Gold.
	
	\smallskip

	Wenn du die Schatztruhe nimmst oder kaufst, musst du dir zusätzlich 2 Kupfer nehmen. Sind nur weniger als 2 Kupfer im Vorrat, so nimmst du dir nur so viele wie möglich. Du nimmst dir die 2 Kupfer nur, wenn du die Schatztruhe nimmst oder kaufst, nicht jedesmal, wenn du sie ausspielst.}
\end{tikzpicture}
\hspace{-0.6cm}
\begin{tikzpicture}
	\card
	\cardstrip
	\cardbanner{banner/green.png}
	\cardicon{icons/coin.png}
	\cardprice{4}
	\cardtitle{\footnotesize{Seidenstrasse}}
	\cardcontent{Diese Königreichkarte ist eine Punktekarte, keine Aktionskarte. Sie hat bis zum Ende des Spiels keine Funktion. Bei der Wertung zählt sie 1 Siegpunkt für je volle 4 Punktekarten in deinem gesamten Kartensatz (Nachziehstapel, Ablagestapel und Handkarten). Bei Spielende suchst du aus deinem gesamten Kartensatz alle Punktekarten heraus und zählst sie. (Die Seidenstraße selbst ist auch eine Punktekarte und zählt zur Gesamtzahl der Punktekarten.) Die Ergebnis durch 4 geteilt und abgerundet ergibt die Punkte.
	
	\smallskip

	 Im Spiel zu 3. und 4. werden 12 Karten Seidenstraße verwendet, im Spiel zu 2. werden 8 Karten Seidenstraße verwendet.}
\end{tikzpicture}
\hspace{-0.6cm}
\begin{tikzpicture}
	\card
	\cardstrip
	\cardbanner{banner/white.png}
	\cardicon{icons/coin.png}
	\cardprice{5}
	\cardtitle{Stallungen}
	\cardcontent{Du darfst eine Karte aus deiner Hand ablegen. Wenn du das machst, ziehst du 3 Karten nach und erhältst +1 Aktion. Wenn du keine Karte aus deiner Hand ablegen kannst oder willst, darfst du auch keine der angegebenen Anweisungen ausführen.}
\end{tikzpicture}
\hspace{-0.6cm}
\begin{tikzpicture}
	\card
	\cardstrip
	\cardbanner{banner/greenblue.png}
	\cardicon{icons/coin.png}
	\cardprice{3}
	\cardtitle{Tunnel}
	\cardcontent{\tiny{
	Diese Karte ist gleichzeitig eine Punktekarte und eine Reaktionskarte. Bei der Wertung zählt sie 2 Siegpunkte. 

	\smallskip

	Die Reaktion kann eingesetzt werden, wenn du den Tunnel außerhalb einer Aufräumphase ablegen musst. Du darfst dann den Tunnel aufdecken bevor du ihn ablegst und nimmst dir ein Gold. Du darfst den Tunnel nicht \enquote{freiwillig} ablegen, eine Kartenanweisung muss dich dazu zwingen. Dies kann in deinem eigenen Zug (z. B. durch die Oase) geschehen, oder im Zug eines Mitspielers (z. B. durch den Markgrafen). Die Reaktion setzt auch ein, wenn du den Tunnel nicht wie üblich aus deiner Hand ablegst, sondern von deinem Nachziehstapel oder von aufgedeckten Karten ablegst (z. B. durch das Orakel). Wenn der Tunnel normalerweise nicht aufgedeckt würde (z. B. durch den Kartographen), musst du ihn aufdecken, um dir ein Gold zu nehmen. Du musst den Tunnel nicht aufdecken (und dementsprechend auch kein Gold nehmen), wenn du ihn ablegst. Die Reaktion wird nicht ausgelöst, wenn du die Karte auf deinen Ablagestapel legst, ohne sie aufzudecken. Du kannst die Karte nicht aufdecken, wenn du sie gerade genommen oder gekauft hast. Du kannst den Tunnel auch nicht aufdecken, wenn du z. B. durch den Kanzler (Dominion – Basisspiel) deinen Nachziehstapel ablegst oder wenn durch die Besessenheit (Dominion – Die Alchemisten) entsorgte Karten auf deinen Ablagestapel gelegt werden. Die Reaktion setzt auch nicht ein, wenn du den Tunnel regulär in der Aufräumphase ablegst. Es muss eine Anweisung geben, die dich zwingt, eine Karte abzulegen. 

	\smallskip
	
	Das Gold nimmst du dir aus dem Vorrat. Ist dort kein Gold mehr, nimmst du dir keines.}}
\end{tikzpicture}
\hspace{-0.6cm}
\begin{tikzpicture}
	\card
	\cardstrip
	\cardbanner{banner/white.png}
	\cardicon{icons/coin.png}
	\cardprice{2}
	\cardtitle{\footnotesize{Wegkreuzung}}
	\cardcontent{Decke deine gesamte Kartenhand auf. Dann ziehe so viele Karten nach, wie du Punktekarten auf deiner Hand hast. Kombinierte Punktekarten sind auch Punktekarten. Hast du keine Punktekarten auf der Hand, ziehst du auch keine Karten nach. Wenn du in diesem Zug zum ersten Mal eine Wegkreuzung ausspielst, erhältst du +3 Aktionen. Für alle Wegkreuzungen, die du danach in diesem Zug ausspielst, erhältst nur noch +1 Aktion. Spielst du die Karte z. B. durch den Thronsaal (Dominion – Basisspiel) zweimal aus (und hast in diesem Zug noch keine Wegkreuzung ausgespielt), so erhältst du beim ersten Mal +3 Aktionen, beim zweiten Ausspielen nochmals +1 Aktion, also insgesamt +4 Aktionen.}
\end{tikzpicture}
\hspace{-0.6cm}
\begin{tikzpicture}
	\card
	\cardstrip
	\cardbanner{banner/white.png}
	\cardtitle{\scriptsize{Empfohlene 10er Sätze\qquad}}
	\cardcontent{\emph{Hinterland:}
	
	\smallskip

	\emph{Einführung:} \\ 
	Aufbau, Feilscher, Gewürzhändler, Lebenskünstler, Markgraf, Nomadencamp, Oase, Schatztruhe, Stallungen, Wegkreuzung 

	\smallskip 
	
	\emph{Lauterer Wettbewerb:} \\ 
	Aufbau, Blutzoll, Edler Räuber, Fahrender Händler, Fruchtbares Land, Grenzdorf, Herzogin, Kartograph, Seidenstraße, Stallungen 

	\smallskip 
	
	\emph{Gelegenheiten:} \\ 
	Fahrender Händler, Feilscher, Fernstraße, Gewürzhändler, Grenzdorf, Herzogin, Katzengold, Komplott, Nomadencamp, Schatztruhe

	\smallskip 
	
	\emph{Eröffnungen:} \\ 
	Botschaft, Gasthaus, Kartograph, Lebenskünstler, Mandarin, Nomadencamp, Oase, Orakel, Tunnel, Wegkreuzung}
\end{tikzpicture}
\hspace{-0.6cm}
\begin{tikzpicture}
	\card
	\cardstrip
	\cardbanner{banner/white.png}
	\cardtitle{\scriptsize{Empfohlene 10er Sätze\qquad}}
	\cardcontent{\emph{Hinterland und Basisspiel:}
	
	\smallskip

	\emph{Straßenräuber:} \\ 
	Bibliothek, Geldverleiher, Keller, Thronsaal, Werkstatt / Edler Räuber, Fernstraße, Gasthaus, Markgraf, Oase

	\smallskip 
	
	\emph{Abenteuerfahrt:} \\ 
	Abenteurer, Jahrmarkt, Kanzler, Laboratorium, Umbau/ Fruchtbares Land, Gewürzhändler, Katzengold, Orakel, Wegkreuzung

	\smallskip 
	
	\emph{Hinterland und die Intrige:} 

	\smallskip 
	
	\emph{Geld aus Nichts:} \\ 
	Armenviertel, Große Halle, Handlanger, Kerkermeister, Kupferschmied / Kartograph, Lebenskünstler, Schatztruhe, Seidenstraße, Tunnel

	\smallskip 
	
	\emph{Am Hof des Herzogs:} \\ 
	Anbau, Harem, Herzog, Maskerade, Verschwörer / Edler Räuber, Feilscher, Gasthaus, Herzogin, Komplott}
\end{tikzpicture}
\hspace{-0.6cm}
\begin{tikzpicture}
	\card
	\cardstrip
	\cardbanner{banner/white.png}
	\cardtitle{\scriptsize{Empfohlene 10er Sätze\qquad}}
	\cardcontent{\emph{Hinterland und Seaside:}
	
	\smallskip

	\emph{Reisende:} \\ 
	Ausguck, Beutelschneider, Handelsschiff, Insel, Lagerhaus / Fruchtbares Land, Kartograph, Seidenstraße, Stallungen, Wegkreuzung

	\smallskip 
	
	\emph{Diplomatie:} \\ 
	Bazar, Botschafter, Embargo, Karawane, Schmuggler / Blutzoll, Botschaft, Edler Räuber, Fahrender Händler, Fruchtbares Land 
	
	\smallskip 
	
	\emph{Hinterland und die Alchemisten:} \\ 
	
	\smallskip 
	
	\emph{Träume sind Schäume:} \\ 
	Apotheker, Kräuterkundiger, Lehrling, Stein der Wei- sen, Verwandlung / Blutzoll, Herzogin, Katzengold, Komplott, Lebenskünstler 
	
	\smallskip 
	
	\emph{Weinviertel:} \\ 
	Golem, Lehrling, Universität, Vertrauter, Weinberg / Feilscher, Fernstraße, Fruchtbares Land, Nomadencamp, Wegkreuzung}
\end{tikzpicture}
\hspace{-0.6cm}
\begin{tikzpicture}
	\card
	\cardstrip
	\cardbanner{banner/white.png}
	\cardtitle{\scriptsize{Empfohlene 10er Sätze\qquad}}
	\cardcontent{\emph{Hinterland und Blütezeit:}
	
	\smallskip
	
	\emph{Karriereleiter:} \\ 
	Ausbau, Bischof, Hort, Münzer, Wachturm / Blutzoll, Edler Räuber, Fahrender Händler, Feilscher, Fruchtbares Land 
	
	\smallskip 
	
	\emph{Schatzfund:} \\ 
	Abenteuer, Bank, Denkmal, Handelsroute, Königliches Siegel/ Aufbau, Blutzoll, Katzengold, Mandarin, Schatztruhe 
	
	
	\smallskip 
	
	\emph{Hinterland und Reiche Ernte:} 
	
	\smallskip 
	
	\emph{Schmalhans:} \\ 
	Füllhorn, Harlekin, Pferdehändler, Turnier, Weiler / Edler Räuber, Fahrender Händler, Katzengold, Mandarin, Tunnel 
	
	\smallskip 
	
	\emph{Wanderzirkus:} \\ 
	Bauerndorf, Festplatz, Harlekin, Menagerie, Treibjagd / Botschaft, Grenzdorf, Katzengold, Nomadencamp, Oase}
\end{tikzpicture}
\hspace{0.6cm}

	    % Basic settings for this card set
\renewcommand{\cardcolor}{darkages}
\renewcommand{\cardextension}{Erweiterung VI}
\renewcommand{\cardextensiontitle}{Dark Ages}
\renewcommand{\seticon}{darkages.png}

\clearpage
\newpage
\section{\cardextension \ - \cardextensiontitle \ (Hans Im Glück 2012)}

\begin{tikzpicture}
	\card
	\cardstrip
	\cardbanner{banner/white.png}
	\cardicon{icons/coin.png}
	\cardprice{6}
	\cardtitle{Altar}
	\cardcontent{Du musst eine Karte aus deiner Hand entsorgen, falls möglich. Dann nimmst du dir eine Karte, die bis zu \coin[5] kostet, aus dem Vorrat (auch wenn du keine Karte entsorgen konntest). Die neue Karte legst du auf deinen Ablagestapel.}
\end{tikzpicture}
\hspace{-0.6cm}
\begin{tikzpicture}
	\card
	\cardstrip
	\cardbanner{banner/white.png}
	\cardicon{icons/coin.png}
	\cardprice{1}
	\cardtitle{Armenhaus}
	\cardcontent{Wenn du diese Karte ausspielst, erhältst du zunächst +\coin[4] für die folgende Kaufphase. Dann musst du alle deine Handkarten aufdecken. Für jede Geldkarte auf deiner Hand reduzieren sich die +\coin[4] um \coin[1] (niemals jedoch unter \coin[0]). Kombinierte Geldkarten sind auch Geldkarten. Hast du z. B. 2 Kupfer auf der Hand, so bleiben dir noch +\coin[2].}
\end{tikzpicture}
\hspace{-0.6cm}
\begin{tikzpicture}
	\card
	\cardstrip
	\cardbanner{banner/white.png}
	\cardicon{icons/coin.png}
	\cardprice{5}
	\cardtitle{\footnotesize{Banditenlager}}
	\cardcontent{Du ziehst zuerst eine Karte nach, dann nimmst du dir eine Beute-Karte vom Beute-Stapel neben dem Vorrat und legst die Karte auf deinen Ablagestapel. Ist der Beute-Stapel leer, nimmst du dir keine Beute-Karte. Danach darfst du bis zu 2 weitere Aktionen ausführen.}
\end{tikzpicture}
\hspace{-0.6cm}
\begin{tikzpicture}
	\card
	\cardstrip
	\cardbanner{banner/white.png}
	\cardicon{icons/coin.png}
	\cardprice{4}
	\cardtitle{Barde}
	\cardcontent{Du ziehst zuerst eine Karte nach. Dann deckst du die obersten 3 Karten von deinem Nachziehstapel auf. Kannst du (auch nach dem Mischen des Ablagestapels) keine 3 Karten aufdecken, deckst du nur so viele Karten auf, wie möglich. Lege alle aufgedeckten Aktionskarten in beliebiger Reihenfolge zurück auf deinen Nachziehstapel und lege die übrigen aufgedeckten Karten ab. Kombinierte Aktionskarten sind auch Aktionskarten. Hast du keine Aktionskarten aufgedeckt, legst du auch keine Karten zurück auf deinen Nachziehstapel. Danach darfst du bis zu 2 weitere Aktionen ausführen.}
\end{tikzpicture}
\hspace{-0.6cm}
\begin{tikzpicture}
	\card
	\cardstrip
	\cardbanner{banner/blue.png}
	\cardicon{icons/coin.png}
	\cardprice{2}
	\cardtitle{Bettler}
	\cardcontent{Wenn du den Bettler in deiner Aktionsphase ausspielst, nimmst du dir 3 Kupfer aus dem Vorrat direkt auf deine Hand. Sind nicht mehr genügend Kupfer im Vorrat, nimmst du dir so viele, wie möglich. Wenn ein Mitspieler eine Angriffskarte ausspielt, darfst du den Bettler aus deiner Hand ablegen. (\emph{Achtung:} Der Bettler wird nicht wie üblich nur vorgezeigt und kommt dann zurück auf die Hand, sondern er wird abgelegt.) Wenn du das machst, nimmst du dir 2 Silber aus dem Vorrat. Eines davon legst du auf deinen Nachziehstapel, das andere legst du auf deinen Ablagestapel. Ist nur noch ein Silber im Vorrat so nimmst du nur dieses und legst es auf deinen Nachziehstapel. Ist kein Silber mehr im Vorrat, nimmst du dir keines.}
\end{tikzpicture}
\hspace{-0.6cm}
\begin{tikzpicture}
	\card
	\cardstrip
	\cardbanner{banner/gold.png}
	\cardicon{icons/coin.png}
	\cardprice{0*}
	\cardtitle{Beute}
	\cardcontent{Wird im Spiel mindestens eine der Königreichkarten \emph{Banditenlager}, \emph{Marodeur} oder \emph{Raubzug} verwendet, so wird auch der Beute-Stapel benötigt. Der Beute-Stapel ist niemals im Vorrat und wird \emph{neben} diesen bereit gelegt. Die Karten vom Beute-Stapel können nur durch die Anweisungen auf den 3 oben genannten Karten genommen werden. Auf andere Weise können keine Karten vom Beute-Stapel gekauft oder genommen werden. Der Botschafter (Dominion – Seaside) darf keine Karten auf den Beute-Stapel zurück legen. Der Beute-Stapel wird für die Spielende-Bedingung \emph{nicht} beachtet.
	
	\smallskip

	Die Beute ist eine Geldkarte mit Wert \coin[3], wie Gold. Wenn du die Beute ausspielst, musst du sie sofort zurück auf den Beute-Stapel legen. Du musst Geldkarten, die du auf der Hand hast, nicht ausspielen.}
\end{tikzpicture}
\hspace{-0.6cm}
\begin{tikzpicture}
	\card
	\cardstrip
	\cardbanner{banner/white.png}
	\cardicon{icons/coin.png}
	\cardprice{4}
	\cardtitle{\footnotesize{Eisenhändler}}
	\cardcontent{\emph{Errata:} Die Reihenfolge der Anweisungen ist falsch, es sollte \enquote{[...] Decke die oberste Karte von deinem Nachziehstapel auf. Du darfst sie ablegen. Unabhängig davon: Ist es eine...} heißen.
	
	\smallskip

	Du ziehst zuerst eine Karte nach, dann deckst du die oberste Karte von deinem Nachziehstapel auf. Dann entscheidest du dich, ob du die aufgedeckte Karte zurück auf den Nachziehstapel oder auf deinen Ablagestapel legst. Du erhältst einen Bonus entsprechend dem Kartentyp der aufgedeckten Karte. Ist die aufgedeckte Karte eine Karte mit kombinierten Kartentypen, erhältst du für jeden Kartentyp den angegebenen Bonus. Deckst du z. B. den Harem (Dominion – Die Intrige) auf, erhältst du +\coin[1] und +1 Karte. Danach darfst du eine weitere Aktion ausführen.}
\end{tikzpicture}
\hspace{-0.6cm}
\begin{tikzpicture}
	\card
	\cardstrip
	\cardbanner{banner/white.png}
	\cardicon{icons/coin.png}
	\cardprice{3}
	\cardtitle{Eremit}
	\cardcontent{Wenn du den Eremiten ausspielst, siehst du dir zunächst deinen Ablagestapel durch. Dann darfst du eine Karte, die keine Geldkarte ist, aus dem Ablagestapel oder aus deiner Hand entsorgen. Du musst keine Karte entsorgen und du darfst keine Geldkarte entsorgen. Kombinierte Geldkarten, wie z. B. der Harem (Dominion – Die Intrige) sind Geldkarten. Egal ob du eine Karte entsorgt hast oder nicht, musst du dir eine Karte nehmen, die bis zu \coin[3] kostet. Du nimmst diese Karte aus dem Vorrat und legst sie auf deinen Ablagestapel. Du musst eine Karte nehmen, wenn möglich. Du darfst nicht darauf verzichten.
	
	\smallskip

	Wenn du den Eremiten aus dem Spiel ablegst (normalerweise in der Aufräumphase am Ende der Runde, in der du die Karte ausgespielt hast) und du in diesem Zug keine Karte gekauft hast, entsorge den Eremiten und nimm dir einen Verrückten. Du nimmst den Verrückten vom Stapel neben dem Vorrat und legst ihn auf deinen Ablagestapel. Karten, die du auf andere Weise genommen hast, als sie zu kaufen, haben keinen Einfluss darauf, ob du den Eremiten entsorgst. Ist der Verrückten-Stapel leer, so nimmst du dir keinen. Wird der Eremit in der Aufräumphase nicht regulär abgelegt, sondern z. B. durch das Komplott (Dominion – Hinterland) zurück auf den Nachziehstapel gelegt, entsorgst du den Eremiten nicht und nimmst dir auch keinen Verrückten.}
\end{tikzpicture}
\hspace{-0.6cm}
\begin{tikzpicture}
	\card
	\cardstrip
	\cardbanner{banner/gold.png}
	\cardicon{icons/coin.png}
	\cardprice{5}
	\cardtitle{Falschgeld}
	\cardcontent{Diese Geldkarte hat den Wert \coin[1]. Du legst das Falschgeld wie üblich in der Kaufphase aus. Du erhältst +1 Kauf, dann darfst du eine Geldkarte aus deiner Hand wählen und diese zweimal ausspielen. Du erhältst also zweimal den Wert der gewählten Geldkarte (zusätzlich zu den \coin[1] Geld vom Falschgeld selbst) und führst auch zusätzliche Anweisungen auf der Karte zweimal aus. Danach musst du die gewählte Karte entsorgen. Wenn du das Falschgeld nutzt, um die Beute zweimal auszuspielen, erhältst du also \coin[6] (zusätzlich zu den \coin[1] vom Falschgeld selbst) und legst die Beute zurück auf den Beute-Stapel. Du kannst die Beute nicht entsorgen, da sie durch die Anweisung bereits anderweitig entfernt wurde. Kombinierte Geldkarten sind auch Geldkarten und können mit Falschgeld gewählt werden.}
\end{tikzpicture}
\hspace{-0.6cm}
\begin{tikzpicture}
	\card
	\cardstrip
	\cardbanner{banner/white.png}
	\cardicon{icons/coin.png}
	\cardprice{4}
	\cardtitle{Festung}
	\cardcontent{Wenn du diese Karte ausspielst, ziehst du zunächst eine Karte nach. Dann darfst du bis zu 2 weitere Aktionen ausführen.
	
	\smallskip

	Wenn du die Festung entsorgst, nimmst du die Karte zurück auf die Hand. Es spielt dabei keine Rolle, ob du die Festung in deinem eigenen Zug oder im Zug eines Mitspielers entsorgst. Die Festung gilt als entsorgt, obwohl du sie sofort zurück auf die Hand nimmst. Spielst du z. B. den Leichenkarren aus und entscheidest dich dafür, die Festung zu entsorgen, gilt die Bedingung als erfüllt und du muss den Leichenkarren nicht entsorgen.}
\end{tikzpicture}
\hspace{-0.6cm}
\begin{tikzpicture}
	\card
	\cardstrip
	\cardbanner{banner/white.png}
	\cardicon{icons/coin.png}
	\cardprice{3}
	\cardtitle{Gassenjunge}
	\cardcontent{Wenn du den Gassenjungen ausspielst, ziehst du zunächst eine Karte nach, dann muss jeder deiner Mitspieler seine Kartenhand auf 4 reduzieren. Mitspieler, die bereits 4 oder weniger Karten auf der Hand haben, legen keine Karten ab.
	
	\smallskip

	Wenn der Gassenjunge im Spiel ist und du eine weitere Angriffskarte ausspielst, darfst du (bevor du die neue Karte ausführst) den Gassenjungen entsorgen und dir einen Söldner nehmen. Du nimmst den Söldner vom Stapel neben dem Vorrat und legst ihn auf deinen Ablagestapel. Ist der Söldner-Stapel leer, so nimmst du dir keinen. Wenn du den selben Gassenjungen zweimal ausspielst, z. B. durch die Prozession, darfst du ihn nicht entsorgen um dir einen Söldner zu nehmen. Wenn du zwei einzelne Gassenjungen spielst, darfst du den ersten entsorgen und dir einen Söldner nehmen.}
\end{tikzpicture}
\hspace{-0.6cm}
\begin{tikzpicture}
	\card
	\cardstrip
	\cardbanner{banner/white.png}
	\cardicon{icons/coin.png}
	\cardprice{5}
	\cardtitle{Grabräuber}
	\cardcontent{Du musst eine der beiden Optionen wählen und diese ausführen, soweit möglich. Du darfst auch eine Option wählen, die du nicht ausführen kannst. Du darfst den Müll-Stapel jederzeit durchsehen. Wenn du dir eine Karte vom Müll-Stapel nimmst, zeigst du diese deinen Mitspielern und legst sie dann sofort auf deinen Nachziehstapel. Wenn keine Karte im Müll-Stapel ist, die \coin[3] - \coin[6] kostet, nimmst du dir keine Karte. Du darfst keine Karten mit Trank in den Kosten (Dominion – Die Alchemisten) nehmen.
	
	\smallskip

	Wenn du dich dafür entscheidest, eine Aktionskarte zu entsorgen, nimmst du dir eine Karte aus dem Vorrat und legst sie auf deinen Ablagestapel.}
\end{tikzpicture}
\hspace{-0.6cm}
\begin{tikzpicture}
	\card
	\cardstrip
	\cardbanner{banner/white.png}
	\cardicon{icons/coin.png}
	\cardprice{5}
	\cardtitle{Graf}
	\cardcontent{Diese Karte gibt dir 2 voneinander unabhängige Wahlmöglichkeiten. Du entscheidest dich zuerst, ob du 2 Handkarten ablegst oder eine Handkarte auf deinen Nachziehstapel legst oder dir ein Kupfer vom Vorrat nimmst und auf deinen Ablagestapel legst. Danach entscheidest du dich, ob du +\coin[3] möchtest oder alle deine verbliebenen Handkarten entsorgst oder dir ein Herzogtum vom Vorrat nimmst und auf deinen Ablagestapel legst. Du musst beide Wahlmöglichkeiten in dieser Reihenfolge ausführen. Du kannst z. B. zuerst 2 Handkarten ablegen und dir danach ein Herzogtum nehmen. Du darfst auch eine Option wählen, die du nicht ausführen kannst. Wenn du mehrere Karten entsorgst, die eine \enquote{Wenn du diese Karte entsorgst ...}-Anweisung haben, entsorgst du zuerst die Karten und führst die \enquote{Wenn du diese Karte entsorgst ...}-Anweisungen dann in beliebiger Reihenfolge aus.}
\end{tikzpicture}
\hspace{-0.6cm}
\begin{tikzpicture}
	\card
	\cardstrip
	\cardbanner{banner/white.png}
	\cardicon{icons/coin.png}
	\cardprice{6}
	\cardtitle{Jagdgründe}
	\cardcontent{Wenn du diese Karte ausspielst, ziehst du 4 Karten nach. Wenn du die Jagdgründe entsorgst, musst du wählen und entweder 1 Herzogtum oder 3 Anwesen vom Vorrat nehmen und auf deinen Ablagestapel legen. Du darfst auch eine Option wählen, die du nicht oder nur teilweise ausführen kannst, z. B. weil nur noch 2 Anwesen im Vorrat sind.}
\end{tikzpicture}
\hspace{-0.6cm}
\begin{tikzpicture}
	\card
	\cardstrip
	\cardbanner{banner/white.png}
	\cardicon{icons/coin.png}
	\cardprice{5}
	\cardtitle{Katakomben}
	\cardcontent{Wenn du diese Karte ausspielst, siehst du dir zuerst die obersten 3 Karten von deinem Nachziehstapel an. Dann musst du dich entscheiden, ob du entweder alle 3 Karten auf die Hand nimmst oder alle 3 Karten auf deinen Ablagestapel legst. Wenn du die 3 Karten ablegst (und nur dann), ziehst du die nächsten 3 Karten von deinem Nachziehstapel. Wenn du die Katakomben entsorgst, nimmst du dir eine Karte, die weniger kostet als die Katakomben selbst, vom Vorrat und legst sie auf deinen Ablagestapel. Dabei ist egal, ob du die Katakomben in deinem eigenen Zug oder im Zug eines Mitspielers entsorgst.}
\end{tikzpicture}
\hspace{-0.6cm}
\begin{tikzpicture}
	\card
	\cardstrip
	\cardbanner{banner/white.png}
	\cardicon{icons/coin.png}
	\cardprice{2}
	\cardtitle{Knappe}
	\cardcontent{Wenn du diese Karte ausspielst erhältst du zuerst +\coin[1]. Dann wählst du eine der 3 Optionen: +2 Aktionen, +2 Kauf oder du nimmst dir ein Silber vom Vorrat und legst es auf deinen Ablagestapel. Wenn du den Knappen entsorgst, nimmst du dir eine beliebige Angriffskarte aus dem Vorrat und legst sie auf deinen Ablagestapel. Du darfst nur sichtbare Karten aus dem Vorrat nehmen, also z. B. nur den offen liegenden Ritter.}
\end{tikzpicture}
\hspace{-0.6cm}
\begin{tikzpicture}
	\card
	\cardstrip
	\cardbanner{banner/white.png}
	\cardicon{icons/coin.png}
	\cardprice{5}
	\cardtitle{Kultist}
	\cardcontent{Wenn du diese Karte ausspielst, ziehst du zuerst 2 Karten nach. Dann muss sich, beginnend mit dem Spieler zu deiner Linken, reihum jeder Mitspieler die oberste Ruinen-Karte vom Ruinen-Stapel nehmen. Ist oder wird der Ruinen-Stapel während des Nehmens leer, können keine weiteren Ruinen mehr genommen werden. Danach darfst du einen weiteren Kultisten aus deiner Hand ausspielen. Das Ausspielen des ursprünglichen Kultisten kostet wie üblich eine Aktion. Das Ausspielen weiterer Kultisten durch die Kartenanweisung benötigt und verbraucht keine Aktion(en).
	
	\smallskip

	Wenn du einen Kultisten entsorgst, ziehst du 3 Karten nach. Dabei ist egal, ob du den Kultisten in deinem eigenen Zug oder im Zug eines Mitspielers entsorgst. Wenn du den Kultisten entsorgst, während du Karten aufdeckst, wie z. B. bei einem Ritter-Angriff, handelst du zuerst diese Aktion ab und legst die übrigen aufgedeckten Karten ab. Du ziehst also nicht die aufgedeckten Karten nach, sondern die nachfolgenden.}
\end{tikzpicture}
\hspace{-0.6cm}
\begin{tikzpicture}
	\card
	\cardstrip
	\cardbanner{banner/white.png}
	\cardicon{icons/coin.png}
	\cardprice{3}
	\cardtitle{Lagerraum}
	\cardcontent{Wenn du diese Karte ausspielst, darfst du zuerst eine beliebige Anzahl Karten (auch 0) aus deiner Hand auf den Ablagestapel legen und ebensoviele Karten nachziehen. Unabhängig davon, ob und wie viele Karten du abgelegt hast, darfst du nochmals beliebig viele deiner verbleibenden Handkarten (auch 0) ablegen und erhältst für jede dabei abgelegten Karte +\coin[1]. Weiterhin erhältst du +1 Kauf für die folgende Kaufphase.}
\end{tikzpicture}
\hspace{-0.6cm}
\begin{tikzpicture}
	\card
	\cardstrip
	\cardbanner{banner/white.png}
	\cardicon{icons/coin.png}
	\cardprice{2}
	\cardtitle{\footnotesize{Landstreicher}}
	\cardcontent{Wenn du diese Karte ausspielst, ziehst du zuerst eine Karte nach. Danach deckst du die oberste Karte von deinem Nachziehstapel auf. Ist es ein Fluch, eine Ruinen-Karte, eine Unterschlupf-Karte oder eine Punktekarte, nimm diese Karte auf die Hand. Dies gilt jeweils auch für kombinierte Kartentypen, die mindestens einen der genannten Typen enthalten. Ansonsten lege die Karte zurück auf den Nachziehstapel.}
\end{tikzpicture}
\hspace{-0.6cm}
\begin{tikzpicture}
	\card
	\cardstrip
	\cardbanner{banner/green.png}
	\cardicon{icons/coin.png}
	\cardprice{4}
	\cardtitle{Lehen}
	\cardcontent{Diese Königreichkarte ist eine Punktekarte, keine Aktionskarte. Sie hat bis zum Ende des Spiels keine Funktion. Bei der Wertung zählt sie 1 Siegpunkt für je volle 3 Silber in deinem gesamten Kartensatz (Nachziehstapel, Ablagestapel und Handkarten). Bei Spielende suchst du aus deinem gesamten Kartensatz alle Silber heraus und zählst sie. Das Ergebnis durch 3 geteilt und abgerundet ergibt die Punkte. Im Spiel zu 3. und 4. werden 12 Karten Lehen verwendet, im Spiel zu 2. werden 8 Karten Lehen verwendet. Wenn du das Lehen entsorgst (egal ob in deinem eigenen Zug oder im Zug eines Mitspielers), nimmst du dir 3 Silber vom Vorrat und legst diese auf deinen Ablagestapel. Sind nicht mehr genug Silber im Vorrat, nimmst du dir nur so viele wie möglich.}
\end{tikzpicture}
\hspace{-0.6cm}
\begin{tikzpicture}
	\card
	\cardstrip
	\cardbanner{banner/white.png}
	\cardicon{icons/coin.png}
	\cardprice{4}
	\cardtitle{\footnotesize{Leichenkarren}}
	\cardcontent{Wenn du den Leichenkarren ausspielst, musst du dich entscheiden, entweder genau eine Aktionskarte aus deiner Hand zu entsorgen oder diesen ausgespielten Leichenkarren zu entsorgen. Kombinierte Aktionskarten sind auch Aktionskarten.
	
	\smallskip

	Wenn du den Leichenkarren nimmst (durch Kauf oder durch eine andere Aktion), nimmst du dir zusätzlich die beiden obersten Ruinen-Karten vom Ruinen-Stapel und legst diese auf deinen Ablagestapel. Sind weniger als 2 Ruinen-Karten übrig, nimmst du die restlichen. Der Spieler, der den Leichenkarren nimmt, nimmt auch die Ruinen-Karten. Spielt ein Spieler die Besessenheit (Dominion – Die Alchemisten) aus und lässt sein \enquote{Opfer} einen Leichenkarren kaufen, so nimmt der Spieler, der die Besessenheit gespielt hat, am Ende den Leichenkarren und damit auch die beiden Ruinen. Verwendet ein Spieler den Fahrenden Händler (Dominion – Hinterland) und nimmt ein Silber anstatt dem Leichenkarren, so nimmt er sich keine Ruinen. Spielst du einen Botschafter (Dominion – Seaside) um Leichenkarren an deine Mitspieler zu verteilen, so nimmt sich jeder Spieler, der einen Leichenkarren nimmt, auch 2 Ruinen (wenn möglich). Wird ein Leichenkarren durch die Maskerade (Dominion – Die Intrige) weitergegeben, nimmt sich der Spieler keine Ruinen, da dies nicht als \enquote{nehmen} gilt. Weiterhin erhältst du +\coin[5] für die folgende Kaufphase.}
\end{tikzpicture}
\hspace{-0.6cm}
\begin{tikzpicture}
	\card
	\cardstrip
	\cardbanner{banner/white.png}
	\cardicon{icons/coin.png}
	\cardprice{4}
	\cardtitle{\scriptsize{Lumpensammler}}
	\cardcontent{\emph{Errata:} Der Kartentext ist falsch, es sollte \enquote{[...] Du darfst sofort deinen kompletten Nachziehstapel auf den Ablagestapel legen.} statt \enquote{[...] Du darfst sofort deinen kompletten Nachziehstapel ablegen.} heißen.
	
	\smallskip

	Du darfst deinen kompletten Nachziehstapel auf den Ablagestapel ablegen, musst dies jedoch nicht. Du musst jedoch eine Karte aus deinem Ablagestapel wählen und auf deinen Nachziehstapel legen, ausser der Ablagestapel ist leer. Ist dein Nachziehstapel leer, z. B. weil du ihn gerade komplett abgelegt hast, legst du die Karte an die Stelle des Nachziehstapels. Weiterhin erhältst du +\coin[2] für die folgende Kaufphase.
	
	\smallskip

	\emph{Anmerkung:} Durch das direkte Ablegen wird die Karte Tunnel (Dominion – Hinterland) nicht ausgelöst.}
\end{tikzpicture}
\hspace{-0.6cm}
\begin{tikzpicture}
	\card
	\cardstrip
	\cardbanner{banner/blue.png}
	\cardicon{icons/coin.png}
	\cardprice{3}
	\cardtitle{Marktplatz}
	\cardcontent{Wenn du diese Karte ausspielst ziehst du eine Karte nach. Dann darfst du eine weitere Aktion ausführen und erhältst +1 Kauf für die folgende Kaufphase.
	
	\smallskip

	Wenn du eine Karte entsorgst, darfst du den Marktplatz aus deiner Hand auf deinen Ablagestapel legen. Wenn du das machst, nimmst du 1 Gold vom Vorrat und legst es auf deinen Ablagestapel. Ist kein Gold mehr im Vorrat, nimmst du dir keines. Du darfst mehrere Karten Marktplatz aus deiner Hand ablegen, auch wenn du nur eine Karte entsorgst.}
\end{tikzpicture}
\hspace{-0.6cm}
\begin{tikzpicture}
	\card
	\cardstrip
	\cardbanner{banner/white.png}
	\cardicon{icons/coin.png}
	\cardprice{4}
	\cardtitle{Marodeur}
	\cardcontent{Wenn du diese Karte spielst, nimmst du dir zuerst eine Karte vom Beute-Stapel und legst sie auf deinen Ablagestapel. Ist der Beute-Stapel aufgebraucht, nimmst du dir keine Beute. Dann muss sich, beginnend mit dem Spieler zu deiner Linken, reihum jeder Mitspieler die oberste Karte vom Ruinen-Stapel nehmen. Ist oder wird der Ruinen-Stapel während des Nehmens leer, können keine weiteren Ruinen mehr genommen werden.}
\end{tikzpicture}
\hspace{-0.6cm}
\begin{tikzpicture}
	\card
	\cardstrip
	\cardbanner{banner/white.png}
	\cardicon{icons/coin.png}
	\cardprice{5}
	\cardtitle{Medium}
	\cardcontent{Wenn du diese Karte spielst, benennst du eine Karte (z. B. \enquote{Kupfer}, \emph{nicht} \enquote{Geld}) und deckst die oberste Karte von deinem Nachziehstapel auf. Wenn es sich um die benannte Karte handelt, nimmst du sie auf die Hand. Wenn nicht, legst du sie zurück auf den Nachziehstapel. Die benannte Karte muss eindeutig erkennbar sein, Sir Destry ist z. B. \emph{nicht} identisch mit Sir Martin. Du darfst auch eine Karte benennen, die nicht im Spiel ist.
	
	\smallskip

	Danach darfst du eine weitere Aktion ausführen und erhältst +\coin[2] für die folgende Kaufphase.}
\end{tikzpicture}
\hspace{-0.6cm}
\begin{tikzpicture}
	\card
	\cardstrip
	\cardbanner{banner/white.png}
	\cardicon{icons/coin.png}
	\cardprice{3}
	\cardtitle{Mundraub}
	\cardcontent{Wenn du diese Karte ausspielst erhältst du +1 Aktion und +1 Kauf für die folgende Kaufphase. Dann musst du eine beliebige Karte aus deiner Hand entsorgen. Wenn du keine Handkarten hast, musst du nichts entsorgen. Danach siehst du dir den Müll-Stapel durch und zählst wie viele Geldkarten mit unterschiedlichem Namen sich darin befinden. Für jede davon erhältst du +\coin[1] für die folgende Kaufphase. Befinden sich im Müll-Stapel z. B. 4 Kupfer, 1 Falschgeld und 6 Anwesen, so erhältst du +\coin[2]. Kombinierte Geldkarten (z. B. Harem, Dominion – Die Intrige) sind auch Geldkarten.}
\end{tikzpicture}
\hspace{-0.6cm}
\begin{tikzpicture}
	\card
	\cardstrip
	\cardbanner{banner/white.png}
	\cardicon{icons/coin.png}
	\cardprice{5}
	\cardtitle{Neubau}
	\cardcontent{Du benennst zuerst eine beliebige Karte. Es muss keine Punktekarte sein. Du darfst sogar eine Karte benennen, die in diesem Spiel nicht verwendet wird. Dann deckst du solange Karten von deinem Nachziehstapel auf, bis eine Punktekarte offen liegt, die nicht die zuvor benannte Karte ist. (Das Benennen einer Karte dient dazu, diese Karte zu schützen.) Wenn du auch nach dem Mischen des Ablagestapels keine solche Karte aufdecken kannst, passiert nichts weiter. Wenn du eine Punktekarte aufdeckst, die nicht die zuvor benannte Karte ist, entsorgst du diese Karte und nimmst dir eine Punktekarte, die bis zu \coin[3] mehr kostet, als die entsorgte Karte aus dem Vorrat und legst diese auf deinen Ablagestapel.}
\end{tikzpicture}
\hspace{-0.6cm}
\begin{tikzpicture}
	\card
	\cardstrip
	\cardbanner{banner/white.png}
	\cardicon{icons/coin.png}
	\cardprice{4}
	\cardtitle{Prozession}
	\cardcontent{\tiny{\begin{Spacing}{1}
	\emph{Errata:} Auf der Karte steht, dass man sich eine Karte nehmen muss, die genau \coin[1] mehr kostet. Richtig wäre: ...nimm dir eine Aktionskarte, die genau \coin[1] mehr kostet.
	
	\smallskip

	Du darfst eine Aktionskarte, die du noch auf der Hand hast, zweimal ausspielen. Du spielst die Karte zuerst aus und führst die Anweisungen komplett aus. Dann nimmst du die Karte zurück auf die Hand, spielst sie ein zweites Mal aus und führst die Anweisungen nochmal komplett aus. Das zweimalige Ausspielen der Aktionskarte kostet keine weiteren Aktionen. Das ursprüngliche Ausspielen der Prozession kostet wie üblich eine Aktion. Nachdem du die Karte zum zweiten Mal ausgeführt hast musst du sie entsorgen und dir dafür eine Aktionskarte, die genau \coin[1] mehr kostet als die entsorgte Karte, aus dem Vorrat nehmen (wenn möglich) und auf deinen Ablagestapel legen. Du musst dies auch machen, wenn du im Verlauf der Aktion den \emph{\enquote{Anschluss verloren}} hast (siehe: Neue Regeln). Du darfst andere Karten erst ausspielen, wenn die Prozession vollständig abgehandelt ist. Spielst du z. B. eine Festung auf eine Prozession, so führst du die zusätzlichen insgesamt 4 Aktionen erst aus, nachdem du die Festung entsorgt und dir eine neue Karte vom Vorrat genommen hast. Wenn du eine Prozession auf eine andere Prozession spielst, spielst du 2 weitere Aktionskarten aus deiner Hand je 2mal aus, entsorgst diese beiden Aktionskarten und nimmst dir 2 Karten vom Vorrat. Dann entsorgst du die zweite Prozession und nimmst dir auch dafür eine Karte vom Vorrat. Wenn du eine Dauer-Karte auf eine Prozession spielst, bleibt die Prozession bis zum nächsten Zug liegen, dann führst du den Effekt der Dauerkarte zweimal aus (obwohl die Dauerkarte bereits entsorgt ist).
	\end{Spacing}}}
\end{tikzpicture}
\hspace{-0.6cm}
\begin{tikzpicture}
	\card
	\cardstrip
	\cardbanner{banner/white.png}
	\cardicon{icons/coin.png}
	\cardprice{4}
	\cardtitle{Ratten}
	\cardcontent{Wenn du diese Karte ausspielst, ziehst du zuerst eine Karte nach. Dann nimmst du dir eine weitere Karte Ratten aus dem Vorrat und legst diese auf deinen Ablagestapel. Sind keine Ratten mehr im Vorrat, nimmst du dir keine. Danach musst du eine Karte aus deiner Hand entsorgen, jedoch keine Ratten. Hast du nur Ratten oder keine Karten auf der Hand, entsorgst du keine Karte. Hast du nur Ratten auf der Hand, musst du deine Kartenhand den anderen Spielern vorzeigen.
	
	\smallskip

	Wenn du die Ratten entsorgst, ziehst du sofort eine Karte nach. Dabei ist egal, ob du die Ratten in deinem eigenen Zug oder im Zug eines Mitspielers entsorgst.}
\end{tikzpicture}
\hspace{-0.6cm}
\begin{tikzpicture}
	\card
	\cardstrip
	\cardbanner{banner/white.png}
	\cardicon{icons/coin.png}
	\cardprice{5}
	\cardtitle{Raubzug}
	\cardcontent{Du entsorgst diese Karte direkt nachdem du sie ausgespielt hast. Dann muss jeder Mitspieler, beginnend mit dem Spieler zu deiner Linken, seine komplette Kartenhand aufdecken. Du wählst bei jedem Mitspieler eine seiner Handkarten, die dieser ablegen muss. Danach nimmst du dir 2 Karten vom Beute-Stapel neben dem Vorrat und legst diese auf deinen Ablagestapel. Sind nicht mehr genügend Karten im Beute-Stapel, nimmst du nur so viele wie möglich.}
\end{tikzpicture}
\hspace{-0.6cm}
\begin{tikzpicture}
	\card
	\cardstrip
	\cardbanner{banner/white.png}
	\cardicon{icons/coin.png}
	\cardprice{5}
	\cardtitle{Ritter}
	\cardcontent{\miniscule{\begin{Spacing}{1}
	\vspace{1em}
	Im Ritter-Stapel befinden sich bei Spielbeginn 10 Karten mit jeweils individuellem Namen. Alle Ritter haben die selbe Basisfähigkeit, zusätzlich jedoch noch eine individuelle Anweisung. Wenn die Ritter als Königreichkarte für ein Spiel gewählt werden, werden die Karten mit dem Typ \enquote{Ritter} gemischt, als verdeckter Stapel in den Vorrat gelegt und die oberste Karte davon aufgedeckt. Siehe hierzu auch unter Spielvorbereitung \enquote{gemischte Stapel}. Auf den Ritter-Stapel wird kein Marker für die Handelsroute (Dominion – Blütezeit) gelegt, auch wenn die oberste Karte eine Punktekarte ist. Bis auf diese Ausnahmen wird der Ritter-Stapel behandelt, wie jeder andere der 10 Kartenstapel. Er ist Teil des Vorrats und wird bei der Spielendebedingung beachtet. Die Anweisungen auf den Ritterkarten werden wie üblich von oben nach unten ausgeführt. Sir Michael z. B. lässt die Mitspieler Karten ablegen, bevor seine Basisfähigkeit ausgeführt wird. Die Basisfähigkeit, die alle Ritter gemeinsam haben, wird wie folgt ausgeführt: Beginnend mit dem Spieler links von dir, muss jeder Mitspieler die obersten beiden Karten von seinem Nachziehstapel aufdecken. Hat der Mitspieler keine Karte aufgedeckt, die 3 - 6 Geld kostet, muss er keine Karte entsorgen. Er legt beide aufgedeckten Karten ab. Kostet genau eine der beiden aufgedeckten Karten 3 - 6 Geld, so muss er diese Karte entsorgen, die andere Karte legt er ab. Kosten beide aufgedeckten Karten 3 - 6 Geld, so darf der Mitspieler entscheiden, welche der beiden Karten er entsorgen muss. Die andere aufgedeckte Karte legt er ab. Entsorgt ein Mitspieler auf diese Weise eine andere Ritter-Karte, so musst auch du den Ritter entsorgen, der den Angriff ausgelöst hat. Die Mitspieler dürfen keine Karten mit Trank in den Kosten (Dominion – Die Alchemisten) entsorgen. Sollen die Ritter im Schwarzmarktstapel verwendet werden, so wird einer der 10 Ritter zufällig gezogen und in den Schwarzmarktstapel eingemischt.
	
	\smallskip

	Die individuellen Fähigkeiten der einzelnen Ritter sind selbsterklärend. Darum hier nur kurze Anmerkungen zu einzel- nen Rittern. \emph{Sir Martin:} Es ist beabsichtigt, dass Sir Martin nur 4 kostet. Sir Martin ist der einzige Ritter, den ein Spieler z. B. durch die Waffenkammer nehmen kann (natürlich nur, wenn er offen liegt). \emph{Sir Vander:} Das Gold nimmst du aus dem Vorrat und legst es auf deinen Ablagestapel. Es ist egal, ob du Sir Vander in deinem eigenen Zug oder im Zug eines Mitspielers entsorgst. \emph{Dame Natalie:} Die Karte darfst (emph{Errata:} auf der Karte ist die Anweisung fälschlicherweise nicht optional) du aus dem Vorrat nehmen und sie dann auf deinen Ablagestapel legen. \emph{Dame Anna:} Du musst keine Karten entsorgen. \emph{Dame Josephine} ist eine Punktekarte. Der Ritter-Stapel ist jedoch dadurch kein Punktestapel. Wird die Handelsroute (Dominion – Blütezeit) verwendet, wird kein Marker auf den Ritter-Stapel gelegt.
	\end{Spacing}}}
\end{tikzpicture}
\hspace{-0.6cm}
\begin{tikzpicture}
	\card
	\cardstrip
	\cardbanner{banner/brown.png}
	\cardicon{icons/coin.png}
	\cardprice{0}
	\cardtitle{Ruinen}
	\cardcontent{Der Ruinen-Stapel besteht aus bis zu 5 unterschiedlichen Karten. Siehe hierzu auch unter Spielvorbereitung \enquote{gemischte Stapel} und \enquote{Ruinen}. Der Ruinen-Stapel ist Teil des Vorrats. Die offenliegende Karte (und nur diese) kann bzw. muss durch eine entsprechende Kartenanweisung genommen oder auch gekauft werden.
	
	\smallskip

	Wenn du die Überlebenden ausspielst, siehst du dir die obersten beiden Karten von deinem Nachziehstapel an und entscheidest dich dann, ob du beide Karten auf deinen Ablagestapel legst oder ob du beide Karten in beliebiger Reihenfolge zurück auf deinen Nachziehstapel legst. Du darfst nicht eine Karte ablegen und die andere zurück auf den Nachziehstapel legen.
	
	\smallskip

	Die übrigen 4 Ruinen-Karten sind selbsterklärend.}
\end{tikzpicture}
\hspace{-0.6cm}
\begin{tikzpicture}
	\card
	\cardstrip
	\cardbanner{banner/white.png}
	\cardicon{icons/coin.png}
	\cardprice{5}
	\cardtitle{\scriptsize{Schrotthändler}}
	\cardcontent{Wenn du diese Karte ausspielst, ziehst du zuerst eine Karte nach. Dann musst du eine Karte entsorgen, wenn du mindestens eine Karte auf der Hand hast. Danach darfst du eine weitere Aktion ausführen und erhältst +\coin[1] für die folgende Kaufphase.}
\end{tikzpicture}
\hspace{-0.6cm}
\begin{tikzpicture}
	\card
	\cardstrip
	\cardbanner{banner/white.png}
	\cardicon{icons/coin.png}
	\cardprice{5}
	\cardtitle{Schurke}
	\cardcontent{Wenn im Müll-Stapel mindestens eine Karte ist, die \coin[3] - \coin[6] kostet, musst du eine davon nehmen und auf deinen Ablagestapel legen. Du darfst nicht darauf verzichten. Du darfst den Müllstapel durchsehen, bevor du den Schurken ausspielst. Du zeigst die Karte, die du nimmst deinen Mitspielern. Du darfst keine Karten mit Trank in den Kosten (Dominion – Die Alchemisten) nehmen.
	
	\smallskip

	Wenn keine Karte im Müll-Stapel ist, die \coin[3] - \coin[6] kostet, nimmst du dir keine Karte. Stattdessen muss jeder Mitspieler, beginnend mit dem Spieler links von dir, die obersten beiden Karten von seinem Nachziehstapel aufdecken und eine Karte davon entsorgen, die \coin[3] - \coin[6] kostet. Hat der Mitspieler keine Karte aufgedeckt, die 3 - 6 Geld kostet, muss er keine Karte entsorgen. Er legt beide aufgedeckten Karten ab. Kostet genau eine der beiden aufgedeckten Karten \coin[3] - \coin[6], so muss er diese Karte entsorgen, die andere Karte legt er ab. Kosten beide aufgedeckten Karten \coin[3] - \coin[6], so darf der Mitspieler entscheiden, welche der beiden Karten er entsorgen muss. Die andere aufgedeckte Karte legt er ab.}
\end{tikzpicture}
\hspace{-0.6cm}
\begin{tikzpicture}
	\card
	\cardstrip
	\cardbanner{banner/white.png}
	\cardicon{icons/coin.png}
	\cardprice{0*}
	\cardtitle{Söldner}
	\cardcontent{Wird der Gassenjunge im Spiel verwendet, so wird auch der Söldner-Stapel benötigt. Der Söldner-Stapel wird \emph{neben} dem Vorrat bereit gelegt. Die Karten vom Söldner-Stapel können nur durch die Anweisung auf dem Gassenjunge genommen werden. Auf andere Weise können keine Karten vom Söldner-Stapel gekauft oder genommen werden. Der Botschafter (Dominion – Seaside) darf keine Karten auf den Söldner-Stapel zurück legen. Der Söldner-Stapel wird für die Spielende-Bedingung nicht beachtet.
	
	Wenn du den Söldner ausspielst, darfst du 2 Karten aus deiner Hand entsorgen. Wenn du das machst, musst du zuerst 2 Karten nachziehen und erhältst +\coin[2] für die folgende Kaufphase. Dann müssen alle deine Mitspieler, beginnend mit dem Spieler links von dir, solange Handkarten ablegen, bis sie nur noch höchstens 3 Karten auf der Hand haben. Mitspieler, die bereits 3 oder weniger Karten auf der Hand haben, legen keine Karten mehr ab. Spieler, die mit dem Bettler auf diesen Angriff reagieren wollen, müssen diesen ablegen, bevor du dich entscheidest 2 Karten zu entsorgen oder nicht. Wenn du dich dafür entscheidest, 2 Karten zu entsorgen, jedoch nur noch 1 Karte auf der Hand hast. Entsorgst du nur diese 1 Karte. Da die nachfolgenden Anweisungen jedoch an das entsorgen von 2 Karten gebunden ist, ziehst du keine Karten nach und erhältst auch kein virtuelles Geld. Wenn die Karten, die du entsorgst, Entsorgungs-Anweisungen haben, entsorgst du zuerst beide und führst die Anweisungen dann in beliebiger Reihenfolge aus.}
\end{tikzpicture}
\hspace{-0.6cm}
\begin{tikzpicture}
	\card
	\cardstrip
	\cardbanner{banner/red.png}
	\cardicon{icons/coin.png}
	\cardprice{1}
	\cardtitle{\footnotesize{Unterschlupf}}
	\cardcontent{\tiny{\begin{Spacing}{1}
	\vspace{1em}
	Unterschlupfkarten haben keinen Stapel (weder im Vorrat noch ausserhalb des Vorrats), sie können niemals gekauft werden.
	
	\smallskip

	\emph{Hütte:} Diese Karte ist ein Unterschlupf, siehe \enquote{Besondere Karten} und eine Reaktionskarte. Die Hütte ist niemals im Vorrat. Wenn du eine Punktekarte kaufst und die Hütte auf der Hand hast, darfst du die Hütte entsorgen. Dies gilt auch, wenn du eine kombinierte Punktekarte kaufst. Du erhältst keine weiteren Vorteile durch das Entsorgen der Hütte. Du wirst nur eine Karte los.
	
	\smallskip

	\emph{Totenstadt:} Diese Karte ist ein Unterschlupf, siehe \enquote{Besondere Karten} und eine Aktionskarte. Die Totenstadt ist niemals im Vorrat. Wenn du diese Karte ausspielst, erhältst du +2 Aktionen.
	
	\smallskip

	\emph{Verfallenes Anwesen:} Diese Karte ist ein Unterschlupf, siehe \enquote{Besondere Karten} und eine Punktekarte mit dem Wert 0 Siegpunkte. Das Verfallene Anwesen ist niemals im Vorrat. Wenn du diese Karte entsorgst, ziehst du sofort eine Karte nach, auch wenn du gerade eine andere Aktionskarte ausführst. Wenn du das Verfallene Anwesen z. B. durch den Altar entsorgst, ziehst du zuerst eine Karte nach und nimmst dir danach eine Karte vom Vorrat. Du darfst das Verfallene Anwesen nicht \enquote{freiwillig} entsorgen, sondern benötigst wie üblich eine andere Karte mit der Anweisung, die dir erlaubt, eine Karte zu entsorgen.
	\end{Spacing}}}
\end{tikzpicture}
\hspace{-0.6cm}
\begin{tikzpicture}
	\card
	\cardstrip
	\cardbanner{banner/white.png}
	\cardicon{icons/coin.png}
	\cardprice{0*}
	\cardtitle{Verrückter}
	\cardcontent{Wird der Eremit im Spiel verwendet, so wird auch der Verrückten-Stapel benötigt. Der Verrückten-Stapel wird \emph{neben} dem Vorrat bereit gelegt. Die Karten vom Verrückten-Stapel können nur durch die Anweisung auf dem Eremiten genommen werden. Auf andere Weise können keine Karten vom Verrückten-Stapel gekauft oder genommen werden. Der Botschafter (Dominion – Seaside) darf keine Karten auf den Verrückten-Stapel zurück legen. Der Verrückten-Stapel wird für die Spielende-Bedingung \emph{nicht} beachtet.
	
	\smallskip

	Diese Karte ist nicht Teil des Vorrats. Sie kann nur durch die Anweisung auf dem Eremiten genommen werden. Auf eine andere Art kann der Verrückte nicht genommen oder gekauft werden. Wenn du den Verrückten ausspielst, erhältst du +2 Aktionen. Dann legst du den Verrückten normalerweise auf den Verrückten-Stapel zurück und ziehst pro Karte, die du noch auf der Hand hast, eine Karte nach (du verdoppelst also die Anzahl deiner Handkarten). Es kann jedoch vorkommen, dass du den Verrückten nicht zurück legen kannst, weil du den \enquote{Anschluss verloren} hast (siehe Neue Regeln), z. B. weil du den Verrückten auf eine Prozession oder einen Thronsaal (Dominion – Basisspiel) ausgespielt hast.}
\end{tikzpicture}
\hspace{-0.6cm}
\begin{tikzpicture}
	\card
	\cardstrip
	\cardbanner{banner/white.png}
	\cardicon{icons/coin.png}
	\cardprice{5}
	\cardtitle{Vogelfreie}
	\cardcontent{\miniscule{\begin{Spacing}{1}
	\vspace{1em}
	Wenn du diese Karte ausspielst, wählst du eine Karte aus dem Vorrat, die weniger kostet, als die Vogelfreien selbst (also normalerweise bis zu 4 Geld). Dann behandelst du die vor dir liegende Karte Vogelfreie genau so, als wäre es die gewählte Karte. Normalerweise führst du einfach nur die Anweisungen auf der gewählten Karte aus. Wählst du z. B. die Festung, so ziehst du eine Karte nach und führst danach bis zu 2 weitere Aktionen aus. Die Vogelfreien nehmen auch die Kosten, den Namen und den oder die Kartentypen der gewählten Karte an.
	
	\smallskip

	Wenn du mit den Vogelfreien eine Karte wählst, die sich selbst entsorgt, so entsorgst du die Vogelfreien. Wenn die Karte nicht mehr vor dir liegt, wird sie wieder zu der Karte Vogelfreie selbst. Wenn du die Vogelfreien als Dauer-Karte (Dominion – Seaside) verwendest, bleibt die Karte bis zu deinem nächsten Zug im Spiel. Wenn du die Vogelfreien als Thronsaal (Dominion – Basisspiel), Königshof (Dominion – Blütezeit) oder Prozession verwendest um eine Dauer-Karte mehrmals auszuspielen, bleiben die Vogelfreien ebenso im Spiel. Wenn du die Vogelfreien mehrmals ausspielst (z. B. durch den Thronsaal), wählst du nur beim ersten Ausspielen eine Karte. Die Vogelfreien bleiben diese Karte auch beim nachfolgenden Ausspielen. Nutzt du z. B. die Prozession, um die Vogelfreien zweimal auszuspielen und kopierst die Festung, so ziehst du insgesamt 2 Karten nach und erhältst +4 Aktionen. Danach entsorgst du die ausliegende Karte, nimmst sie aber sofort wieder auf die Hand zurück, da es noch immer eine Festung ist. Zurück auf deiner Hand ist die Karte aus dem Spiel und wird wieder zu den Vogelfreien. Nun nimmst du dir noch eine Karte aus dem Vorrat, die genau 1 Geld mehr kostet, als die entsorgte Karte. Die Karte, die du entsorgt hast, ist inzwischen wieder die Karte Vogelfreie 5 Geld. Du nimmst dir also eine Karte die 6 Geld kostet. Wenn du die Vogelfreien als eine Karte nutzt, die etwas in der Aufräumphase macht (z. B. der Eremit), so führst du diese Anweisung aus. Wenn du das Füllhorn (Dominion – Reiche Ernte) ausspielst, werden die Vogelfreien als die Karte, die sie kopieren angesehen. Wenn du z. B. mit einer Karte Vogelfreie die Festung wählst und mit einer weiteren Karte Vogelfreie den Lumpensammler, nimmst du dir eine Karte, die bis zu 3 Geld kostet. Du darfst mit den Vogelfreien nur eine Karte wählen, die sichtbar im Vorrat liegt. Du darfst keine Karte wählen, deren Stapel leer ist. Du darfst keine Karte wählen, die nicht im Vorrat ist, wie z. B. den Söldner. In gemischten Stapeln, wie z. B. Ruinen oder Ritter, darfst du nur die oben liegende wählen.
	\end{Spacing}}}
\end{tikzpicture}
\hspace{-0.6cm}
\begin{tikzpicture}
	\card
	\cardstrip
	\cardbanner{banner/white.png}
	\cardicon{icons/coin.png}
	\cardprice{4}
	\cardtitle{\footnotesize{Waffenkammer}}
	\cardcontent{Wenn du diese Karte ausspielst, nimmst du dir eine Karte, die bis zu \coin[4] kostet vom Vorrat. Du legst diese sofort verdeckt auf deinen Nachziehstapel, anstatt auf den Ablagestapel wie üblich.}
\end{tikzpicture}
\hspace{-0.6cm}
\begin{tikzpicture}
	\card
	\cardstrip
	\cardbanner{banner/white.png}
	\cardicon{icons/coin.png}
	\cardprice{3}
	\cardtitle{Weiser}
	\cardcontent{Wenn du diese Karte ausspielst, erhältst du immer +1 Aktion, dann deckst du solange Karten von deinem Nachziehstapel auf, bis du eine Karte aufdeckst, die mindestens \coin[3] kostet und nimmst diese Karte auf die Hand. Lege die übrigen durch diese Aktion aufgedeckten Karten ab. Wenn du auch nach dem Mischen deines Ablagestapels keine Karte, die mindestens \coin[3] kostet, aufdecken kannst, legst du alle aufgedeckten Karten ab und nimmst keine Karte auf die Hand. Deckst du z. B. ein Kupfer, dann einen Fluch und dann eine Provinz auf, nimmst du die Provinz auf die Hand und legst Kupfer und Fluch ab.}
\end{tikzpicture}
\hspace{-0.6cm}
\begin{tikzpicture}
	\card
	\cardstrip
	\cardbanner{banner/white.png}
	\cardtitle{\scriptsize{Empfohlene 10er Sätze\qquad}}
	\cardcontent{\emph{Dark Ages:}
	
	\smallskip

	\emph{Leichenzug:} \\ 
	Festung, Jagdgründe, Katakomben, Kultist, Marktplatz, Mundraub, Prozession, Ritter, Vogelfreie, Waffenkammer

	\smallskip 
	
	\emph{Spiel mit dem Teufel} \\ 
	Banditenlager, Grabräuber, Lagerraum, Landstreicher, Lumpensammler, Medium, Ratten, Raubzug, Schrotthändler, Weiser

	\smallskip 
	
	\emph{Dark Ages und Basisspiel:} 

	\smallskip 
	
	\emph{Auf und Ab:} \\ 
	Armenhaus, Barde, Eremit, Jagdgründe, Medium / Geldverleiher, Hexe, Keller, Thronsaal, Werkstatt

	\smallskip 
	
	\emph{Ritterspiele:} \\ 
	Altar, Knappe, Lumpensammler, Ratten, Ritter / Bibliothek, Gärten, Jahrmarkt, Laboratorium, Umbau}
\end{tikzpicture}
\hspace{-0.6cm}
\begin{tikzpicture}
	\card
	\cardstrip
	\cardbanner{banner/white.png}
	\cardtitle{\scriptsize{Empfohlene 10er Sätze\qquad}}
	\cardcontent{\emph{Dark Ages und Die Intrige:}
	
	\smallskip

	\emph{Prophezeiung:} \\ 
	Eisenhändler, Landstreicher, Medium, Neubau, Waffenkammer / Adelige, Baron, Große Halle, Verschwörer, Wunschbrunnen

	\smallskip 
	
	\emph{Invasion:} \\ 
	Bettler, Gassenjunge, Knappe, Marodeur, Schurke / Anbau, Bergwerksdorf, Harem, Kerkermeister, Trickser
	
	\smallskip 
	
	\emph{Dark Ages und Seaside:} 

	\smallskip 
	
	\emph{Nasses Grab:} \\ 
	Eremit, Gassenjunge, Grabräuber, Graf, Lumpensammler / Eingeborenendorf, Müllverwerter, Piratenschiff, Schatzkarte, Schatzkammer

	\smallskip 
	
	\emph{Einfaches Volk:} \\ 
	Armenhaus, Gassenjunge, Landstreicher, Lehen, Leichenkarren / Fischerdorf, Hafen, Insel, Ausguck, Lagerhaus}
\end{tikzpicture}
\hspace{-0.6cm}
\begin{tikzpicture}
	\card
	\cardstrip
	\cardbanner{banner/white.png}
	\cardtitle{\scriptsize{Empfohlene 10er Sätze\qquad}}
	\cardcontent{\emph{Dark Ages und die Alchemisten:}
	
	\smallskip

	\emph{Seuchenherd:} \\ 
	Barde, Kultist, Lehen, Marktplatz, Ratten, Waffenkammer / Lehrling, Vision, Verwandlung, Weinberg

	\smallskip 
	
	\emph{Klagelied:} \\ 
	Bettler, Eisenhändler, Falschgeld, Katakomben, Mundraub, Raubzug / Apotheker, Golem, Kräuterkundiger, Universität
	
	\smallskip 
	
	\emph{Dark Ages und Blütezeit:}

	\smallskip 

	\emph{Des einen Müll ...:} \\ 
	Falschgeld, Grabräuber, Marktplatz, Mundraub, Schurke / Abenteuer, Denkmal, Großer Markt, Stadt, Talisman
	
	\smallskip 
	
	\emph{Ehre unter Dieben:} \\ 
	Banditenlager, Knappe, Neubau, Prozession, Schurke / Kunstschmiede, Hort, Hausierer, Steinbruch, Wachturm}
\end{tikzpicture}
\hspace{-0.6cm}
\begin{tikzpicture}
	\card
	\cardstrip
	\cardbanner{banner/white.png}
	\cardtitle{\scriptsize{Empfohlene 10er Sätze\qquad}}
	\cardcontent{\emph{Dark Ages und Reiche Ernte:}
	
	\smallskip

	\emph{Dunkler Karneval:} \\ 
	Eremit, Festung, Kultist, Ritter, Schrotthändler, Vogelfreie / Festplatz, Füllhorn, Menagerie, Weiler
	
	\smallskip 
	
	\emph{Auf den Sieger:} \\ 
	Banditenlager, Barde, Falschgeld, Leichenkarren, Marodeur, Raubzug / Ernte, Treibjagd Nachbau, Turnier
	
	\smallskip 
	
	\emph{Dark Ages und Hinterland:} 
	
	\smallskip 
	
	\emph{Fern von Zuhause:} \\ 
	Barde, Bettler, Graf, Lehen, Marodeur / Botschaft, Feilscher, Aufbau, Kartograph, Katzengold
	
	\smallskip 
	
	\emph{Expeditionen:} \\ 
	Altar, Armenhaus, Eisenhändler, Katakomben, Lagerraum / Fruchtbares Land, Fernstraße, Gewürzhändler, Tunnel, Wegkreuzung}
\end{tikzpicture}
\hspace{-0.6cm}
\begin{tikzpicture}
	\card
	\cardstrip
	\cardbanner{banner/white.png}
	\cardtitle{\scriptsize{Neue Regeln (1/5)}\qquad}
	\cardcontent{\tiny{\begin{Spacing}{1}
	\vspace{1em}
	\emph{\enquote{ausspielen}:} Ein Spieler, der eine Karte ausspielt, legt diese offen vor sich aus und führt dann die Anweisungen von oben nach unten aus. Auch wenn die Karte nicht vor dem Spieler liegen bleibt (sondern z. B. wie die Beute oder der Verrückte sofort auf den jeweiligen Stapel zurück gelegt werden), werden die Anweisungen soweit möglich vollständig ausgeführt.
	
	\smallskip

	\emph{\enquote{Im Spiel} (\enquote{ausgespielt}):} Aktions- und Geldkarten, die ein Spieler offen vor sich ausgelegt hat, sind \enquote{im Spiel} bis sie aufgeräumt werden (üblicherweise in der Aufräumphase abgelegt werden). Beiseite gelegte und entsorgte Karten, sowie Karten, welche die Spieler auf der Hand haben und Karten im Vorrat und in den Nachzieh- und Ablagestapeln der Spieler sind nicht \enquote{im Spiel}. Das Aufdecken einer Reaktionskarte bringt diese nicht \enquote{ins Spiel}.
	
	\smallskip

	\emph{Kaufphase:} In der Kaufphase muss der Spieler seine Geldkarten einzeln auslegen. Er darf dabei jedoch die Reihenfolge selbst bestimmen. Legt ein Spieler eine Geldkarte mit zusätzlichen Anweisungen aus, führt er diese Anweisungen nach Möglichkeit aus, bevor er eine weitere Geldkarte auslegt. Der Spieler muss alle Geldkarten, die er in dieser Runde auslegen möchte auslegen, bevor er eine Karte kauft (auch wenn er mehrere Käufe zur Verfügung hat). Der Spieler darf in dieser Kaufphase keine Geldkarten mehr auslegen, nachdem er eine Karte gekauft hat.
	
	\smallskip

	\emph{Geldwert:} Geldkarten \enquote{produzieren} ihren Wert im Moment, in dem sie ausgelegt werden. Werden Geldkarten noch während der Kaufphase verändert oder abgelegt (z. B. Beute), bleibt der \enquote{produzierte} Geldwert für diese Kaufphase trotzdem erhalten.
	\end{Spacing}}}
\end{tikzpicture}
\hspace{-0.6cm}
\begin{tikzpicture}
	\card
	\cardstrip
	\cardbanner{banner/white.png}
	\cardtitle{\scriptsize{Neue Regeln (2/5)}\qquad}
	\cardcontent{\tiny{\begin{Spacing}{1}
	\vspace{1em}
	\emph{Geldkarten mit zusätzlichen Anweisungen:} Dark Ages beinhaltet 2 Geldkarten mit zusätzlichen Anweisungen, Falschgeld und Beute. Es handelt sich dabei nicht um Basiskarten, die in jedem Spiel verwendet werden (wie z. B. Gold). Diese Geldkarten können wie üblich in der Kaufphase ausgelegt werden. Sobald ein Spieler eine Geldkarte mit Anweisungen auslegt, führt er die Anweisung auf der Karte aus. Diese Geldkarten sind, wie jede andere Geldkarte, von Anweisungen betroffen, die sich auf Geldkarten beziehen, z. B. Dieb (Dominion-Basisspiel).
	
	\smallskip

	\emph{Wenn mehrere Anweisungen gleichzeitig ausgeführt werden können} darfst du entscheiden, in welcher Reihenfolge du diese Anweisungen ausführst. Entsorgst du z. B. eine Karte Ratten und legst den Marktplatz aus deiner Hand ab, so entscheidest du, ob du zuerst die Anweisung der Ratten oder die Anweisung auf dem Marktplatz ausführst. Wenn mehrere Spieler von einer Anweisung betroffen sind (meistens durch Angriffskarten), so wird die Anweisung normalerweise in Spielerreihenfolge ausgeführt.
	
	\smallskip

	\emph{Mehrfache Reaktionen:} Jeder Spieler darf auf ein einziges Ereignis (meist ein Angriff ) auch mit mehreren Reaktionskarten reagieren, wenn er diese auf der Hand hat. Wie üblich muss jedoch zuerst eine Karte vollständig ausgeführt werden, bevor der Spieler eine weitere Reaktionskarte aus seiner Hand aufdeckt bzw. ablegt. Ein Spieler darf z. B. als Reaktion auf einen Angriff die Geheimkammer (Dominion – Die Intrige) aufdecken. Nachdem er die Geheimkammer vollständig ausgeführt hat, darf er auch noch den Bettler aus seiner Hand ablegen (auch wenn er diesen gerade durch die Geheimkammer nachgezogen hat) und auch diesen ausführen.
	\end{Spacing}}}
\end{tikzpicture}
\hspace{-0.6cm}
\begin{tikzpicture}
	\card
	\cardstrip
	\cardbanner{banner/white.png}
	\cardtitle{\scriptsize{Neue Regeln (3/5)}\qquad}
	\cardcontent{\tiny{\begin{Spacing}{1}
	\vspace{1em}
	\emph{Anschluss verloren:} In einigen Fällen ist es möglich, dass eine Kartenanweisung verlangt, eine Karte zu bewegen (z. B. abzulegen), die Karte jedoch bereits durch eine andere Anweisung anderweitig bewegt (z. B. entsorgt) wurde. Das hat zur Folge, dass du die Karte nicht bewegen kannst. Du führst die Anweisungen auf der Karte trotzdem aus (soweit möglich). Wenn du z. B. auf eine Prozession einen Verrückten spielst, legst du den Verrückten zurück auf den Verrückten-Stapel. Du erhältst +2 Aktionen und ziehst Karten nach. Nun musst du den Verrückten zum zweiten Mal ausspielen. Die Karte liegt jedoch bereits wieder auf dem Verrückten-Stapel. Du erhältst trotzdem nochmals die +2 Aktionen, kannst die Karte jedoch nicht mehr zurück auf den Stapel legen (sie liegt bereits dort). Du ziehst also keine Karten nach, weil diese Anweisung nur aktiv wird, wenn du die Karte auf den Verrückten-Stapel zurück legst. Danach müsstest du den Verrückten entsorgen, da dieser bereits auf seinen Stapel zurück gelegt ist, kannst du ihn nicht auf den Müllstapel legen. Du nimmst dir jedoch trotzdem eine Karte die genau 1 Geld kostet, wenn möglich.
	\end{Spacing}}}
\end{tikzpicture}
\hspace{-0.6cm}
\begin{tikzpicture}
	\card
	\cardstrip
	\cardbanner{banner/white.png}
	\cardtitle{\scriptsize{Neue Regeln (4/5)}\qquad}
	\cardcontent{\tiny{\begin{Spacing}{1}
	\vspace{1em}
	\emph{\enquote{Wenn du diese Karte entsorgst: ...}:} Immer, wenn du eine deiner Karten entsorgst, die eine solche Anweisung haben, führst du diese Anweisung aus. Es ist egal, ob du die Karte in deinem Zug oder im Zug eines Mitspielers (z. B. durch eine Angriffskarte) entsorgst. Jeder Spieler entsorgt immer nur seine eigenen Karten. Im Falle eines Angriffs entsorgt also nicht der Spieler, der die Angriffskarte ausgespielt hat, die Karten seiner Mitspieler, sondern immer nur der durch den Angriff betroffene Spieler. Der Spieler, der seine Karte entsorgt, führt dann auch die Anweisung aus, direkt nachdem er die Karte auf den Müll-Stapel gelegt hat. Dies kann dazu führen, dass der Spieler die Anweisung ausführt, bevor eine andere Aktionskarte vollständig ausgeführt wurde. Spielt ein Spieler z. B. einen Grabräuber und nutzt diesen, um einen Kultisten zu entsorgen, so zieht er zuerst 3 Karten für den Kultisten nach und nimmt danach eine Karte durch den Grabräuber. In einigen Fällen werden Karten aus dem Stapel des Spielers entfernt, ohne entsorgt zu werden, z. B. durch den Botschafter (Dominion – Seaside) oder die Maskerade (Dominion – Die Intrige). Wenn durch eine Anweisung mehrere Karten gleichzeitig entsorgt werden, so entsorgst du zunächst die Karten und führst dann die Entsorgungs-Anweisungen in beliebiger Reihenfolge aus.
	\end{Spacing}}}
\end{tikzpicture}
\hspace{-0.6cm}
\begin{tikzpicture}
	\card
	\cardstrip
	\cardbanner{banner/white.png}
	\cardtitle{\scriptsize{Neue Regeln (5/5)}\qquad}
	\cardcontent{\tiny{\begin{Spacing}{1}
	\vspace{1em}
	\emph{Nur zur Erinnerung:} Immer wenn ein Spieler Karten nachziehen müsste, sein Nachziehstapel je- doch leer ist, mischt der Spieler seinen Ablagestapel und legt ihn als neuen Nachziehstapel bereit.
	
	\smallskip

	Müsste der Spieler mehr Karten nachziehen, als sein Nachziehstapel noch enthält, so zieht er zuerst die verbleibenden Karten seines Nachziehstapels, mischt dann seinen Ablagestapel und zieht die restlichen Karten nach. Kann er, auch nachdem er seinen Ablagestapel gemischt und als neuen Nachziehstapel bereit gelegt hat, nicht genügend Karten nachziehen, so zieht er nur so viele Karten, wie möglich.
	
	\smallskip

	Gleiches gilt, wenn der Spieler Karten von seinem Nachziehstapel aufdecken oder ansehen muss. Muss ein Spieler Karten auf seinen Nachziehstapel legen und ist sein Nachziehstapel in diesem Moment leer, so legt er diese Karten an die Stelle seines Nachziehstapels. Er mischt seinen Ablagestapel nicht.
	\end{Spacing}}}
\end{tikzpicture}
\hspace{0.6cm}

	    % Basic settings for this card set
\renewcommand{\cardcolor}{guilds}
\renewcommand{\cardextension}{Erweiterung VII}
\renewcommand{\cardextensiontitle}{Die Gilden}
\renewcommand{\seticon}{guilds.png}

\clearpage
\newpage
\section{\cardextension \ - \cardextensiontitle \ (Rio Grande Games 2013)}

\begin{tikzpicture}
	\card
	\cardstrip
	\cardbanner{banner/white.png}
	\cardicon{icons/coin.png}
	\cardprice{2}
	\cardtitle{\scriptsize{Leuchtenmacher}}
	\cardcontent{Du darfst in der Aktionsphase eine weitere Aktionskarte ausspielen. Du darfst in der Kaufphase einen zusätzlichen Kauf tätigen. Nimm dir eine Münze.}
\end{tikzpicture}
\hspace{-0.6cm}
\begin{tikzpicture}
	\card
	\cardstrip
	\cardbanner{banner/white.png}
	\cardicon{icons/coin.png}
	\cardprice{2+}
	\cardtitle{Steinmetz}
	\cardcontent{\tiny{\begin{Spacing}{1}
	\vspace{1em}
	Wenn du die Karte \emph{STEINMETZ} kaufst, darfst du mehr dafür bezahlen. Wenn du das tust, nimmst du dir zwei Aktionskarten, die beide jeweils genau so viel kosten, wie du überzahlt hast. Du kannst dir zweimal die gleiche oder zwei verschiedene Karten nehmen. Wenn du zum Beispiel den \emph{STEINMETZ} für \coin[6] kaufst, könntest du dir zwei \emph{HEROLDE} nehmen. Die Aktionskarten stammen aus dem Vorrat und werden auf deinen Ablagestapel gelegt. Sollten keine Karten mit den entsprechenden Kosten im Vorrat sein, bekommst du keine. Wenn du mit einem \emph{TRANK} (\emph{Die Alchemisten}) überzahlst, bekommst du Karten im Wert des \emph{TRANKS}. Karten, die mehreren Kartentypen angehören, von denen einer AKTION ist (wie die \emph{GROSSE HALLE} aus \emph{Die Intrige}), sind Aktionskarten. Wenn du beschließt, nicht mehr als die normalen Kosten zu zahlen, bekommst du keine Karten. Es ist nicht möglich, dann Aktionskarten zu nehmen, die \coin[0]	kosten.

	\medskip

	Wenn du diese Karte spielst, entsorgst du eine Karte aus deiner Hand und nimmst dir zwei Karten, die jeweils weniger kosten als die Karte, die du entsorgt hast. Wenn du keine Karte zum Entsorgen mehr auf der Hand hast, bekommst du keine Karten. Du kannst dir zweimal die gleiche oder zwei verschiedene Karten aus dem Vorrat nehmen. Diese werden auf deinen Ablagestapel gelegt. Falls es keine preiswerteren Karten im Vorrat gibt (weil du beispielsweise ein Kupfer entsorgst), bekommst du keine Karten. Wenn es im Vorrat nur noch eine einzige Karte gibt, die preiswerter ist als die entsorgte Karte, nimmst du dir diese. Du nimmst dir die Karten einzeln nacheinander; das kann bei Karten eine Rolle spielen, die beim Nehmen eine Auswirkung haben (wie das \emph{GASTHAUS} aus \emph{Hinterland}).
	\end{Spacing}}}
\end{tikzpicture}
\hspace{-0.6cm}
\begin{tikzpicture}
	\card
	\cardstrip
	\cardbanner{banner/white.png}
	\cardicon{icons/coin.png}
	\cardprice{3+}
	\cardtitle{Arzt}
	\cardcontent{\tiny{\begin{Spacing}{1}
	\vspace{1em}
	Wenn du diese Karte kaufst, darfst du mehr dafür zahlen. Pro \coin[1], das du zusätzlich zahlst (d.h. \enquote{überzahlst}), darfst du dir die oberste Karte deines Nachziehstapels ansehen und sie entsorgen, ablegen oder auf den Nachziehstapel zurücklegen. Wenn dein Nachziehstapel aufgebraucht ist, mischst du deinen Ablagestapel und machst ihn zum neuen Nachziehstapel. Wenn dann noch immer nicht genügend Karten im Nach- ziehstapel sind, siehst du dir keine Karten an. Auch wenn du mehr als \coin[1] überzahlst, wird jede oberste Karte vom Nachziehstapel einzeln komplett abgehandelt, bevor du die nächste Karte ziehst.

	\medskip

	Wenn du diese Karte in der Aktionsphase spielst, nennst du den Namen einer beliebigen Karte, deckst die obersten drei Karten deines Nachziehstapels auf und entsorgst jede Karte mit dem von dir genannten Namen. Die übrigen Karten legst du in beliebiger Reihenfolge wieder oben auf deinen Nachziehstapel. Du musst keine Karte nennen, die in diesem Spiel verwendet wird. Wenn dein Nachziehstapel keine drei Karten mehr umfasst, deckst du die darin noch enthaltenen Karten auf, mischst dann deinen Ablagestapel (der die bereits aufgedeckten Karten nicht umfasst) und machst ihn zum neuen Nachziehstapel, von dem du die noch fehlenden Karten aufdeckst. Sind dann noch immer nicht genügend Karten aufgedeckt, belässt du es bei den bereits aufgedeckten Karten.
	\end{Spacing}}}
\end{tikzpicture}
\hspace{-0.6cm}
\begin{tikzpicture}
	\card
	\cardstrip
	\cardbanner{banner/white.png}
	\cardicon{icons/coin.png}
	\cardprice{4}
	\cardtitle{Berater}
	\cardcontent{Wenn dein Nachziehstapel keine drei Karten mehr umfasst, deckst du die darin noch enthaltenen Karten auf, mischst dann deinen Ablagestapel und machst ihn zum neuen Nachziehstapel, von dem du die noch fehlenden Karten aufdeckst. Sind dann noch immer nicht genügend Karten aufgedeckt, belässt du es bei den bereits aufgedeckten Karten. Unabhängig davon, wie viele Karten du aufgedeckt hast, wählt dein linker Nachbar eine davon aus, die du ablegst. Die verbleibenden Karten nimmst du auf die Hand.}
\end{tikzpicture}
\hspace{-0.6cm}
\begin{tikzpicture}
	\card
	\cardstrip
	\cardbanner{banner/white.png}
	\cardicon{icons/coin.png}
	\cardprice{4}
	\cardtitle{Platz}
	\cardcontent{Zuerst ziehst du eine Karte. Du erhältst zwei weitere Aktionen, die du spielen darfst, nachdem du alle Anweisungen dieser Karte (soweit möglich) erfüllt hast. Dann darfst du eine Geldkarte (auch ein \emph{MEISTERSTÜCK}) ablegen. Du kannst die Karte ablegen, die du eben gezogen hast, falls es sich um eine Geldkarte handelt. Wenn du eine Geldkarte abgelegt hast, nimmst du dir eine Münze. Karten, die mehreren Kartentypen angehören, von denen einer \emph{GELD} ist (wie der \emph{HAREM} aus Die Intrige), sind Geldkarten.}
\end{tikzpicture}
\hspace{-0.6cm}
\begin{tikzpicture}
	\card
	\cardstrip
	\cardbanner{banner/white.png}
	\cardicon{icons/coin.png}
	\cardprice{4}
	\cardtitle{\scriptsize{Steuereintreiber}}
	\cardcontent{Du darfst eine Geldkarte aus deiner Hand entsorgen. Karten, die mehreren Kartentypen angehören, von denen einer GELD ist (wie der \emph{HAREM} aus \emph{Die Intrige}), sind Geldkarten. Wenn du eine Geldkarte entsorgst, muss jeder Mitspieler, der mindestens fünf Karten in der Hand hat, die gleiche Karte aus seiner Hand auf seinen Ablagestapel legen oder seine Handkarten vorzeigen, damit die anderen Spieler sehen, dass er diese Karte nicht auf der Hand hat. Nimm dir für die entsorgte Karte eine Geldkarte, die bis zu \coin[3] mehr kostet als die entsorgte Geldkarte, und lege sie oben auf deinen Nachziehstapel. Wenn dein Nachziehstapel aufgebraucht ist, wird dies die einzige Karte deines Nachziehstapels. Du musst dir keine teurere Geldkarte nehmen, sondern darfst dir auch eine gleich teure oder preiswertere Geldkarte nehmen.}
\end{tikzpicture}
\hspace{-0.6cm}
\begin{tikzpicture}
	\card
	\cardstrip
	\cardbanner{banner/white.png}
	\cardicon{icons/coin.png}
	\cardprice{4+}
	\cardtitle{Herold}
	\cardcontent{\tiny{\begin{Spacing}{1}
	\vspace{1em}
	Wenn du diese Karte kaufst, darfst du mehr dafür zahlen. Wenn du das tust legst du pro \coin[1], das du überzahlst, eine beliebige Karte deines Ablagestapels oben auf deinen Nachziehstapel. Du darfst dir dafür die Karten deines Ablagestapels ansehen, was normalerweise nicht möglich ist. Allerdings darfst du nicht zuerst deinen Ablagestapel durchsehen, um zu entscheiden, wie viel du überzahlen willst. Sobald du überzahlt hast, musst du die entsprechende Anzahl an Karten in beliebiger Reihenfolge oben auf deinen Nachziehstapel legen, sofern möglich. Wenn du so viel überzahlst, dass du mehr Karten auf deinen Nachziehstapel legen müsstest, als sich in deinem Ablagestapel be nden, legst du einfach alle Karten deines Ablagestapels in beliebiger Reihenfolge auf deinen Nachziehstapel. Falls du den \emph{HEROLD} kaufst, ohne zu überzahlen, darfst du deinen Ablagestapel nicht durchsehen.

	\medskip

	Wenn du diese Karte in der Aktionsphase spielst, ziehst du zuerst eine Karte. Du erhältst dann eine weitere Aktion, die du spielen darfst, nachdem du alle Anweisungen dieser Karte (soweit möglich) erfüllt hast. Dann deckst du die oberste Karte deines Nachziehstapels auf. Wenn es sich um eine Aktionskarte handelt, \emph{musst} du sie spielen. Die Karte zu spielen, verbraucht keine Aktion. Karten, die mehreren Kartentypen angehören, von denen einer AKTION ist (z. B. \emph{GROSSE HALLE} aus \emph{Die Intrige}), sind Aktionskarten. Alle anderen Kartentypen werden auf den Nachziehstapel zurückgelegt, ohne sie zu spielen.

	\medskip

	\emph{Hinweis:} Wenn durch den \emph{HEROLD} eine Dauer-Karte (aus \emph{Seaside}) ausgespielt wird, wird der \emph{HEROLD} dennoch am Ende der Runde wie gewohnt abgelegt, da er nicht gebraucht wird, um an etwas zu erinnern.
	\end{Spacing}}}
\end{tikzpicture}
\hspace{-0.6cm}
\begin{tikzpicture}
	\card
	\cardstrip
	\cardbanner{banner/gold.png}
	\cardicon{icons/coin.png}
	\cardprice{3+}
	\cardtitle{\footnotesize{Meisterstück}}
	\cardcontent{Dies ist eine Geldkarte mit dem Wert \coin[1], wie Kupfer. Wenn du sie kaufst, nimmst du dir ein Silber pro \coin[1], das du überzahlst. Falls du zum Beispiel \coin[6] für das \emph{MEISTERSTÜCK} zahlst, erhältst du drei Silber. Das \emph{MEISTERSTÜCK} ist eine Geldkarte und wird regeltechnisch auch so behandelt.}
\end{tikzpicture}
\hspace{-0.6cm}
\begin{tikzpicture}
	\card
	\cardstrip
	\cardbanner{banner/white.png}
	\cardicon{icons/coin.png}
	\cardprice{5}
	\cardtitle{Bäcker}
	\cardcontent{Wenn du diese Karte spielst, ziehst du eine Karte, darfst eine weitere Aktionskarte ausspielen und nimmst dir eine Münze.

	\medskip

	Wird ein Spiel mit dieser Karte gespielt, erhält jeder Spieler zu Beginn des Spiels eine Münze. Das gilt auch für Partien mit dem \emph{SCHWARZMARKT}, bei denen der \emph{BÄCKER} sich im \emph{SCHWARZMARKT}-Stapel befindet.}
\end{tikzpicture}
\hspace{-0.6cm}
\begin{tikzpicture}
	\card
	\cardstrip
	\cardbanner{banner/white.png}
	\cardicon{icons/coin.png}
	\cardprice{5}
	\cardtitle{\scriptsize{Kaufmannsgilde}}
	\cardcontent{Wenn diese Karte im Spiel ist (d.h. du hast sie in der Aktionsphase gespielt), darfst du eine weitere Karte kaufen und hast zusätzlich \coin[1] zur Verfügung. Jedes Mal, wenn du eine Karte kaufst, nimmst du dir eine Münze (1 Münze, wenn du eine Karte kaufst; 2 Münzen, wenn du zwei Karten kaufst etc.). Denke daran, dass du Münzen nur einsetzen kannst, bevor du Karten kaufst: Du darfst also diese Münze nicht sofort aufwenden. Diese Anweisung ist kumulativ: Wenn du zwei \emph{KAUFMANNSGILDEN} ausgespielt hast, bringt dir jede Karte, die du kaufst, zwei Münzen ein. Wenn du jedoch eine \emph{KAUFMANNSGILDE} mehrfach spielst, aber nur eine im Spiel hast – wie beim \emph{THRONSAAL} (\emph{DOMINION®}) oder \emph{KÖNIGSHOF} (\emph{Blütezeit}) –, bekommst du beim Kauf einer Karte nur eine Münze.}
\end{tikzpicture}
\hspace{-0.6cm}
\begin{tikzpicture}
	\card
	\cardstrip
	\cardbanner{banner/white.png}
	\cardicon{icons/coin.png}
	\cardprice{5}
	\cardtitle{Metzger}
	\cardcontent{Zuerst nimmst du dir 2 Münzen. Dann darfst du eine Karte aus deiner Hand entsorgen und eine beliebige Anzahl an Münzen einsetzen (auch 0 Münzen). Da du den \emph{METZGER} nicht mehr auf der Hand hast, kannst du diese Karte nicht entsorgen. Allerdings kannst du eine andere \emph{METZGER}-Karte entsorgen. Wenn du eine Karte entsorgt hast, nimmst du dir eine Karte, deren Kosten höchstens der Summe aus den Kosten der entsorgten Karte und der Anzahl der eingesetzten Münzen entsprechen darf. So könntest du beispielsweise ein \emph{ANWESEN} entsorgen und sechs Münzen zahlen, um dir eine \emph{PROVINZ} zu nehmen; oder du könntest einen weiteren \emph{METZGER} entsorgen und null Münzen zahlen, um dir ein \emph{HERZOGTUM} zu nehmen. Die Münzen, die du einsetzt, werden in den Vorrat zurückgelegt und dürfen in der Kaufphase nicht mehr benutzt werden, um andere Karten zu kaufen.}
\end{tikzpicture}
\hspace{-0.6cm}
\begin{tikzpicture}
	\card
	\cardstrip
	\cardbanner{banner/white.png}
	\cardicon{icons/coin.png}
	\cardprice{5}
	\cardtitle{Hellseherin}
	\cardcontent{Das Gold und die Fluchkarten stammen aus dem Vorrat und werden auf den Ablagestapel gelegt. Wenn kein Gold mehr vorhanden ist, erhältst du keins. Sind nicht mehr genügend Fluchkarten da, teilst du sie reihum aus, beginnend mit deinem linken Nachbarn. Jeder Spieler, der eine Fluchkarte erhalten hat, muss eine Karte vom Nach- ziehstapel ziehen. Falls ein Spieler keine Fluchkarte bekommen hat – weil nicht genügend Fluchkarten da waren oder aus einem anderen Grund –, zieht er keine Karte. Nutzt ein Spieler den \emph{WACHTURM} (\emph{Blütezeit}), um den Fluch zu entsorgen, hat er dennoch eine Fluchkarte erhalten und zieht deshalb eine Karte. Nutzt ein Spieler den \emph{FAHRENDEN HÄNDLER} (\emph{Hinterland}), um stattdessen ein Silber zu nehmen, hat er keine Fluchkarte erhalten und zieht infolgedessen auch keine Karte.}
\end{tikzpicture}
\hspace{-0.6cm}
\begin{tikzpicture}
	\card
	\cardstrip
	\cardbanner{banner/white.png}
	\cardicon{icons/coin.png}
	\cardprice{5}
	\cardtitle{\footnotesize{Wandergeselle}}
	\cardcontent{Zunächst nennst du eine Karte. Dabei muss es sich nicht um den Namen einer Karte handeln, die in diesem Spiel verwendet wird. Dann deckst du solange Karten vom Nachziehstapel auf, bis du drei Karten aufgedeckt hast, die \emph{nicht} dem von dir genannten Namen entsprechen. Nimm diese Karten auf die Hand und lege die anderen ab. Sollte dabei der Nachziehstapel aufgebraucht werden, bevor du auf drei entsprechende Karten gestoßen bist, mischst du deinen Ablagestapel und nutzt ihn als neuen Nachziehstapel. Wenn du beim weiteren Aufdecken noch immer nicht auf drei Karten mit einem anderen als dem genannten Namen kommst, beendest du das Aufdecken. Nimm die gefundenen Karten anderen Namens auf die Hand und lege den Rest auf deinen Ablagestapel.}
\end{tikzpicture}
\hspace{-0.6cm}
\begin{tikzpicture}
	\card
	\cardstrip
	\cardbanner{banner/white.png}
	\cardtitle{\scriptsize{Empfohlene 10er Sätze\qquad}}
	\cardcontent{\emph{Kunsthandwerk} (Die Gilden + \textit{Basisspiel}):\\
	Steinmetz, Berater, Bäcker, Wandergeselle, Kaufmannsgilde, \textit{Laboratorium}, \textit{Keller}, \textit{Werkstatt}, \textit{Jahrmarkt}, \textit{Geldverleiher}

	\smallskip

	\emph{Rechtschaffen und anständig} (Die Gilden + \textit{Basisspiel}):\\
	Metzger, Bäcker, Leuchtenmacher, Arzt, Wahrsager, \textit{Miliz}, \textit{Dieb}, \textit{Geldverleiher}, \textit{Gärten}, \textit{Dorf}

	\smallskip

	\emph{Des Guten zuviel} (Die Gilden + \textit{Basisspiel}):\\
	Platz, Meisterstück, Leuchtenmacher, Steuereintreiber, Herold, \textit{Bibliothek}, \textit{Umbau}, \textit{Abenteurer}, \textit{Markt}, \textit{Kanzler}

	\smallskip

	\emph{Nenne diese Karte} (Die Gilden + \textit{Die Intrige}):\\
	Bäcker, Arzt, Platz, Berater, Meisterstück, \textit{Burghof}, \textit{Wunschbrunnen}, \textit{Harem}, \textit{Tribut}, \textit{Adelige}

	\smallskip

	\emph{Geschäftstricks} (Die Gilden + \textit{Die Intrige}):\\
	Steinmetz, Herold, Wahrsager, Wandergeselle, Metzger, \textit{Große Halle}, \textit{Adelige}, \textit{Verschwörer}, \textit{Maskerade}, \textit{Kupferschmied}

	\smallskip

	\emph{Entscheidungen, Entscheidungen} (Die Gilden + \textit{Die Intrige}):\\
	Kaufmannsgilde, Leuchtenmacher, Meisterstück, Steuereintreiber, Metzger, \textit{Brücke}, \textit{Handlanger}, \textit{Bergwerk}, \textit{Anbau}, \textit{Herzog}}
\end{tikzpicture}
\hspace{-0.6cm}
\begin{tikzpicture}
	\card
	\cardstrip
	\cardbanner{banner/white.png}
	\cardtitle{\footnotesize{Neue Regeln (1/2)}\qquad}
	\cardcontent{\tiny{\begin{Spacing}{1}
	\vspace{1em}
	\emph{Die Münzen:} Einige Karten in \emph{Die Gilden} erlauben den Spielern, sich Münzen zu nehmen. Münzen werden immer vom Vorrat genommen, nicht von anderen Spielern. Diese Münzen können – anders als die Geldkarten – über den Zug, in dem ein Spieler sie erhält, hinaus aufbewahrt werden. In der Kaufphase eines Spielers kann dieser, \emph{bevor} er Karten kauft, eine beliebige Anzahl an Münzen einsetzen; jede eingesetzte Münze bringt dem Spieler +\coin[1]. Eingesetzte Münzen werden in den Vorrat zurückgelegt. Die Anzahl der Münzen im Vorrat ist nicht begrenzt: Sollten einmal nicht ausreichend Münzen vorhanden sein, verwenden die Spieler einen beliebigen Ersatz. Zwar handelt es sich um die gleichen Marker/Münzen wie in \emph{Seaside} und \emph{Blütezeit}, allerdings können die durch Anweisungen auf den Karten aus \emph{Die Gilden} erworbenen Münzen ausschließlich in der Kaufphase ausgegeben werden. Sie dürfen weder zum Kauf einer Karte mithilfe der Karte \emph{SCHWARZMARKT} benutzt werden, noch dürfen sie auf ein Piratenschiff-Tableau (\emph{Seaside}) oder die \emph{HANDELSROUTE} (\emph{Blütezeit}) gelegt werden.
	\end{Spacing}}}
\end{tikzpicture}
\hspace{-0.6cm}
\begin{tikzpicture}
	\card
	\cardstrip
	\cardbanner{banner/white.png}
	\cardtitle{\footnotesize{Neue Regeln (2/2)}\qquad}
	\cardcontent{\tiny{\begin{Spacing}{1}
	\vspace{1em}
	\emph{Überzahlen:} Für einige Karten in \emph{Die Gilden} darf man mehr zahlen, als sie eigentlich kosten. Bei diesen Karten steht hinter den Kosten ein \enquote{+}, z.B. \coin[2\textsuperscript{+}]. Wenn ein Spieler vor dem Kauf einen \emph{beliebigen zusätzlichen Betrag} für eine solche Karte zahlt, tritt ein auf der jeweiligen Karte genannter Effekt ein, der von der Höhe der Überzahlung abhängt. Überzahlt werden kann mit allen Geldwerten, mit denen Karten gekauft werden: Geldkarten, Münzen und Geldwerte auf Aktionskarten. Die Karten \emph{TRANK} (\emph{Die Alchemisten}) können ebenfalls zum Überzahlen verwendet werden (allerdings ist das nicht immer sinnvoll). Das zum Überzahlen verwendete Geld ist ausgegeben und kann dann nicht mehr eingesetzt werden. Geldkarten werden abgelegt und Münzen kommen zurück in den Vorrat. Spieler können eine Karte ausschließlich beim Kauf überzahlen, nicht wenn sie diese auf eine andere Weise erhalten. Das \enquote{+} ist lediglich eine Erinnerung: Für eine Karte, die hinter den Kosten das Symbol \enquote{+} aufweist, gelten nach wie vor für alle Zwecke die normalen Kosten.
	Grundsätzlich gibt es keine Wechselwirkungen zwischen Karten, welche die Kartenkosten reduzieren – wie die \emph{BRÜCKE} (\emph{Die Intrige}) oder die \emph{FERNSTRASSE} (\emph{Hinterland}) – und dem \enquote{Überzahle}.

	\emph{Zwei Dinge passieren zur gleichen Zeit:} Wenn einem Spieler zwei Dinge gleichzeitig passieren, entscheidet er selbst, in welcher Reihenfolge sie eintreten. Wenn unterschiedlichen Spielern zwei Dinge gleichzeitig passieren, werden diese reihum in Spielreihenfolge abgehandelt, beginnend mit dem Spieler, der gerade an der Reihe ist.
	\end{Spacing}}}
\end{tikzpicture}
\hspace{0.6cm}

	    % Basic settings for this card set
\renewcommand{\cardcolor}{adventures}
\renewcommand{\cardextension}{Erweiterung VIII}
\renewcommand{\cardextensiontitle}{Abenteuer}

\clearpage
\newpage
\section{\cardextension \ - \cardextensiontitle \ (Rio Grande Games 2015)}

\begin{tikzpicture}
	\card
	\cardstrip
	\cardbanner{banner/goldlightbrown.png}
	\cardicon{banner/coin.png}
	\cardprice{2}
	\cardtitle{\scriptsize{Königliche Münzen}}
	\cardcontent{}
\end{tikzpicture}
\hspace{-1cm}
\begin{tikzpicture}
	\card
	\cardstrip
	\cardbanner{banner/lightbrown.png}
	\cardicon{banner/coin.png}
	\cardprice{2}
	\cardtitle{Rattenfänger}
	\cardcontent{}
\end{tikzpicture}
\hspace{-1cm}
\begin{tikzpicture}
	\card
	\cardstrip
	\cardbanner{banner/white.png}
	\cardicon{banner/coin.png}
	\cardprice{2}
	\cardtitle{Zerstörung}
	\cardcontent{}
\end{tikzpicture}
\hspace{-1cm}
\begin{tikzpicture}
	\card
	\cardstrip
	\cardbanner{banner/orange.png}
	\cardicon{banner/coin.png}
	\cardprice{3}
	\cardtitle{Amulett}
	\cardcontent{}
\end{tikzpicture}
\hspace{-1cm}
\begin{tikzpicture}
	\card
	\cardstrip
	\cardbanner{banner/orange.png}
	\cardicon{banner/coin.png}
	\cardprice{3}
	\cardtitle{Ausrüstung}
	\cardcontent{}
\end{tikzpicture}
\hspace{-1cm}
\begin{tikzpicture}
	\card
	\cardstrip
	\cardbanner{banner/orangeblue.png}
	\cardicon{banner/coin.png}
	\cardprice{3}
	\cardtitle{\scriptsize{Karawanenwächter}}
	\cardcontent{}
\end{tikzpicture}
\hspace{-1cm}
\begin{tikzpicture}
	\card
	\cardstrip
	\cardbanner{banner/lightbrown.png}
	\cardicon{banner/coin.png}
	\cardprice{3}
	\cardtitle{Kundschafter}
	\cardcontent{}
\end{tikzpicture}
\hspace{-1cm}
\begin{tikzpicture}
	\card
	\cardstrip
	\cardbanner{banner/orange.png}
	\cardicon{banner/coin.png}
	\cardprice{3}
	\cardtitle{Verlies}
	\cardcontent{}
\end{tikzpicture}
\hspace{-1cm}
\begin{tikzpicture}
	\card
	\cardstrip
	\cardbanner{banner/lightbrown.png}
	\cardicon{banner/coin.png}
	\cardprice{4}
	\cardtitle{Duplikat}
	\cardcontent{}
\end{tikzpicture}
\hspace{-1cm}
\begin{tikzpicture}
	\card
	\cardstrip
	\cardbanner{banner/white.png}
	\cardicon{banner/coin.png}
	\cardprice{4}
	\cardtitle{Elster}
	\cardcontent{}
\end{tikzpicture}
\hspace{-1cm}
\begin{tikzpicture}
	\card
	\cardstrip
	\cardbanner{banner/white.png}
	\cardicon{banner/coin.png}
	\cardprice{4}
	\cardtitle{Geizhals}
	\cardcontent{}
\end{tikzpicture}
\hspace{-1cm}
\begin{tikzpicture}
	\card
	\cardstrip
	\cardbanner{banner/white.png}
	\cardicon{banner/coin.png}
	\cardprice{4}
	\cardtitle{Hafenstadt}
	\cardcontent{}
\end{tikzpicture}
\hspace{-1cm}
\begin{tikzpicture}
	\card
	\cardstrip
	\cardbanner{banner/white.png}
	\cardicon{banner/coin.png}
	\cardprice{4}
	\cardtitle{Kurier}
	\cardcontent{}
\end{tikzpicture}
\hspace{-1cm}
\begin{tikzpicture}
	\card
	\cardstrip
	\cardbanner{banner/lightbrown.png}
	\cardicon{banner/coin.png}
	\cardprice{4}
	\cardtitle{\footnotesize{Transformation}}
	\cardcontent{}
\end{tikzpicture}
\hspace{-1cm}
\begin{tikzpicture}
	\card
	\cardstrip
	\cardbanner{banner/white.png}
	\cardicon{banner/coin.png}
	\cardprice{4}
	\cardtitle{Wildhüter}
	\cardcontent{}
\end{tikzpicture}
\hspace{-1cm}
\begin{tikzpicture}
	\card
	\cardstrip
	\cardbanner{banner/orange.png}
	\cardicon{banner/coin.png}
	\cardprice{5}
	\cardtitle{Brückentroll}
	\cardcontent{}
\end{tikzpicture}
\hspace{-1cm}
\begin{tikzpicture}
	\card
	\cardstrip
	\cardbanner{banner/lightbrowngreen.png}
	\cardicon{banner/coin.png}
	\cardprice{5}
	\cardtitle{Ferne Lande}
	\cardcontent{}
\end{tikzpicture}
\hspace{-1cm}
\begin{tikzpicture}
	\card
	\cardstrip
	\cardbanner{banner/orange.png}
	\cardicon{banner/coin.png}
	\cardprice{5}
	\cardtitle{Geisterwald}
	\cardcontent{}
\end{tikzpicture}
\hspace{-1cm}
\begin{tikzpicture}
	\card
	\cardstrip
	\cardbanner{banner/white.png}
	\cardicon{banner/coin.png}
	\cardprice{5}
	\cardtitle{\tiny{Geschichtenerzähler}}
	\cardcontent{}
\end{tikzpicture}
\hspace{-1cm}
\begin{tikzpicture}
	\card
	\cardstrip
	\cardbanner{banner/white.png}
	\cardicon{banner/coin.png}
	\cardprice{5}
	\cardtitle{\scriptsize{Kunsthandwerker}}
	\cardcontent{}
\end{tikzpicture}
\hspace{-1cm}
\begin{tikzpicture}
	\card
	\cardstrip
	\cardbanner{banner/lightbrown.png}
	\cardicon{banner/coin.png}
	\cardprice{5}
	\cardtitle{\scriptsize{Königliche Kutsche}}
	\cardcontent{}
\end{tikzpicture}
\hspace{-1cm}
\begin{tikzpicture}
	\card
	\cardstrip
	\cardbanner{banner/gold.png}
	\cardicon{banner/coin.png}
	\cardprice{5}
	\cardtitle{Relikt}
	\cardcontent{}
\end{tikzpicture}
\hspace{-1cm}
\begin{tikzpicture}
	\card
	\cardstrip
	\cardbanner{banner/white.png}
	\cardicon{banner/coin.png}
	\cardprice{5}
	\cardtitle{Riese}
	\cardcontent{}
\end{tikzpicture}
\hspace{-1cm}
\begin{tikzpicture}
	\card
	\cardstrip
	\cardbanner{banner/gold.png}
	\cardicon{banner/coin.png}
	\cardprice{5}
	\cardtitle{Schatz}
	\cardcontent{}
\end{tikzpicture}
\hspace{-1cm}
\begin{tikzpicture}
	\card
	\cardstrip
	\cardbanner{banner/orange.png}
	\cardicon{banner/coin.png}
	\cardprice{5}
	\cardtitle{Sumpfhexe}
	\cardcontent{}
\end{tikzpicture}
\hspace{-1cm}
\begin{tikzpicture}
	\card
	\cardstrip
	\cardbanner{banner/white.png}
	\cardicon{banner/coin.png}
	\cardprice{5}
	\cardtitle{\scriptsize{Verlorene Stadt}}
	\cardcontent{}
\end{tikzpicture}
\hspace{-1cm}
\begin{tikzpicture}
	\card
	\cardstrip
	\cardbanner{banner/lightbrown.png}
	\cardicon{banner/coin.png}
	\cardprice{5}
	\cardtitle{Weinhändler}
	\cardcontent{}
\end{tikzpicture}
\hspace{-1cm}
\begin{tikzpicture}
	\card
	\cardstrip
	\cardbanner{banner/orange.png}
	\cardicon{banner/coin.png}
	\cardprice{6}
	\cardtitle{Gefolgsmann}
	\cardcontent{}
\end{tikzpicture}
\hspace{-1cm}
\begin{tikzpicture}
	\card
	\cardstrip
	\cardbanner{banner/white.png}
	\cardtitle{Ereignisse (1/7)\quad}
	\cardcontent{}
\end{tikzpicture}
\hspace{-1cm}
\begin{tikzpicture}
	\card
	\cardstrip
	\cardbanner{banner/white.png}
	\cardtitle{Ereignisse (2/7)\quad}}
	\cardcontent{}
\end{tikzpicture}
\hspace{-1cm}
\begin{tikzpicture}
	\card
	\cardstrip
	\cardbanner{banner/white.png}
	\cardtitle{Ereignisse (3/7)\quad}}
	\cardcontent{}
\end{tikzpicture}
\hspace{-1cm}
\begin{tikzpicture}
	\card
	\cardstrip
	\cardbanner{banner/white.png}
	\cardtitle{Ereignisse (4/7)\quad}}
	\cardcontent{}
\end{tikzpicture}
\hspace{-1cm}
\begin{tikzpicture}
	\card
	\cardstrip
	\cardbanner{banner/white.png}
	\cardtitle{Ereignisse (5/7)\quad}}
	\cardcontent{}
\end{tikzpicture}
\hspace{-1cm}
\begin{tikzpicture}
	\card
	\cardstrip
	\cardbanner{banner/white.png}
	\cardtitle{Ereignisse (6/7)\quad}}
	\cardcontent{}
\end{tikzpicture}
\hspace{-1cm}
\begin{tikzpicture}
	\card
	\cardstrip
	\cardbanner{banner/white.png}
	\cardtitle{Ereignisse (7/7)\quad}}
	\cardcontent{}
\end{tikzpicture}
\hspace{-1cm}
\begin{tikzpicture}
	\card
	\cardstrip
	\cardbanner{banner/white.png}
	\cardicon{banner/coin.png}
	\cardprice{2}
	\cardtitle{Kleinbauer}
	\cardcontent{}
\end{tikzpicture}
\hspace{-1cm}
\begin{tikzpicture}
	\card
	\cardstrip
	\cardbanner{banner/white.png}
	\cardicon{banner/coin.png}
	\cardprice{3\textsuperscript{*}}
	\cardtitle{Soldat}
	\cardcontent{}
\end{tikzpicture}
\hspace{-1cm}
\begin{tikzpicture}
	\card
	\cardstrip
	\cardbanner{banner/white.png}
	\cardicon{banner/coin.png}
	\cardprice{4\textsuperscript{*}}
	\cardtitle{Flüchtling}
	\cardcontent{}
\end{tikzpicture}
\hspace{-1cm}
\begin{tikzpicture}
	\card
	\cardstrip
	\cardbanner{banner/white.png}
	\cardicon{banner/coin.png}
	\cardprice{5\textsuperscript{*}}
	\cardtitle{Schüler}
	\cardcontent{}
\end{tikzpicture}
\hspace{-1cm}
\begin{tikzpicture}
	\card
	\cardstrip
	\cardbanner{banner/lightbrown.png}
	\cardicon{banner/coin.png}
	\cardprice{6\textsuperscript{*}}
	\cardtitle{Lehrer}
	\cardcontent{}
\end{tikzpicture}
\hspace{-1cm}
\begin{tikzpicture}
	\card
	\cardstrip
	\cardbanner{banner/white.png}
	\cardicon{banner/coin.png}
	\cardprice{2}
	\cardtitle{Page}
	\cardcontent{}
\end{tikzpicture}
\hspace{-1cm}
\begin{tikzpicture}
	\card
	\cardstrip
	\cardbanner{banner/white.png}
	\cardicon{banner/coin.png}
	\cardprice{3\textsuperscript{*}}
	\cardtitle{Schatzsucher}
	\cardcontent{}
\end{tikzpicture}
\hspace{-1cm}
\begin{tikzpicture}
	\card
	\cardstrip
	\cardbanner{banner/white.png}
	\cardicon{banner/coin.png}
	\cardprice{4\textsuperscript{*}}
	\cardtitle{Krieger}
	\cardcontent{}
\end{tikzpicture}
\hspace{-1cm}
\begin{tikzpicture}
	\card
	\cardstrip
	\cardbanner{banner/white.png}
	\cardicon{banner/coin.png}
	\cardprice{5\textsuperscript{*}}
	\cardtitle{Held}
	\cardcontent{}
\end{tikzpicture}
\hspace{-1cm}
\begin{tikzpicture}
	\card
	\cardstrip
	\cardbanner{banner/orange.png}
	\cardicon{banner/coin.png}
	\cardprice{6\textsuperscript{*}}
	\cardtitle{Champion}
	\cardcontent{}
\end{tikzpicture}
\hspace{-1cm}
\begin{tikzpicture}
	\card
	\cardstrip
	\cardbanner{banner/white.png}
	\cardicon{}
	\cardprice{}
	\cardtitle{\scriptsize{Empfohlene 10er Sätze\qquad}}
	\cardcontent{\emph{Sanfte Einführung} (+ \underline{Ereignisse}): 
	\\
	\smallskip
	\\
	\underline{Spähtrupp}, Amulett, Duplikat, Ferne Lande, Gefolgsmann, Hafenstadt, Rattenfänger, Riese, Schatz, Verlies, Wildhüter
	\\
	\smallskip
	\\
	\emph{Profi-Einführung} (+ \underline{Ereignisse}): 
	\\
	\smallskip
	\\
	\underline{Mission}, \underline{Planung}, Elster, Geisterwald, Karawanenwächter, Kleinbauer (+ Eintausch-Karten), Königliche Münzen, Sumpfhexe, Verlorene Stadt, Weinhändler, Zerstörung, Transformation
	\\
	\smallskip
	\\
	\emph{Auf höchster Ebene} (Abenteuer + \underline{Ereignisse} + \textit{Basisspiel}):
	\\
	\smallskip
	\\
	\underline{Training}, Ausrüstung, Geizhals, Kundschafter, Verlies, Verlorene Stadt, \textit{Markt}, \textit{Miliz}, \textit{Spion}, \textit{Thronsaal}, \textit{Werkstatt}
	\\
	\smallskip
	\\
	\emph{Verzerrte Größen} (Abenteuer + \underline{Ereignisse} + \textit{Basisspiel}):
	\\
	\smallskip
	\\
	\underline{Freudenfeuer}, \underline{Überfall}, Amulett, Duplikat, Kurier, Riese, Schatz, \textit{Bürokrat}, \textit{Dieb}, \textit{Gärten}, \textit{Geldverleiher}, \textit{Hexe}
	\\}
\end{tikzpicture}
\hspace{1cm}
	    % Basic settings for this card set
\renewcommand{\cardcolor}{empires}
\renewcommand{\cardextension}{Erweiterung IX}
\renewcommand{\cardextensiontitle}{Empires}
\renewcommand{\seticon}{empires.png}

\clearpage
\newpage
\section{\cardextension \ - \cardextensiontitle \ (Rio Grande Games 2016)}

\begin{tikzpicture}
	\card
	\cardstrip
	\cardbanner{banner/white.png}
	\cardicon{icons/coin.png}
	\cardprice{2}
	\cardtitle{Feldlager}
	\cardcontent{Du darfst ein \emph{GOLD} oder ein \emph{DIEBESGUT} aus der Hand aufdecken. Wenn du das nicht kannst oder möchtest, legst du diese Karte zur Seite und legst sie zu Beginn deiner Aufräumphase zurück in den Vorrat. Sollte dort zu diesem Zeitpunkt bereits ein \emph{DIEBESGUT} offen liegen, muss nun erst wieder das zurückgelegte \emph{FELDLAGER} genommen werden, bevor das \emph{DIEBESGUT} genommen werden darf.}
\end{tikzpicture}
\hspace{-0.6cm}
\begin{tikzpicture}
	\card
	\cardstrip
	\cardbanner{banner/white.png}
	\cardicon{icons/coin.png}
	\cardprice{2}
	\cardtitle{Patrizier}
	\cardcontent{Du musst die oberste Karte deines Nachziehstapels aufdecken. Wenn der Stapel aufgebraucht ist, mischst du deinen Ablagestapel und legst ihn als Nachziehstapel bereit. Wenn auch dort keine Karten liegen, erhältst du nichts.

	\medskip

	Nur eine Karte, die mehr als \coin[5] kostet, darfst du auf die Hand nehmen. Ob die Karte außerdem noch Kosten in Form von Schulden \hex aufweist, ist dabei unerheblich (z.B. \emph{REICHTUM} darf auf die Hand genommen werden, \emph{STADTVIERTEL} nicht).}
\end{tikzpicture}
\hspace{-0.6cm}
\begin{tikzpicture}
	\card
	\cardstrip
	\cardbanner{banner/white.png}
	\cardicon{icons/coin.png}
	\cardprice{2}
	\cardtitle{Siedler}
	\cardcontent{Auch wenn du weißt, dass sich kein \emph{KUPFER} in deinem Ablagestapel befindet, darfst du ihn ansehen. Du musst kein \emph{KUPFER} auf die Hand nehmen, wenn du das nicht möchtest.}
\end{tikzpicture}
\hspace{-0.6cm}
\begin{tikzpicture}
	\card
	\cardstrip
	\cardbanner{banner/white.png}
	\cardicon{icons/coin.png}
	\cardprice{3}
	\cardtitle{\footnotesize{Bauernmarkt}}
	\cardcontent{Diese Karte beinhaltet den neuen Typ SAMMLUNG, d.h. hier kommen die Siegpunktmarker zum Einsatz.

	\medskip

	Wenn diese Karte das erste Mal ausgespielt wird, legt der Spieler einen \victorypointtoken-Marker auf den BAUERNMARKT-Vorratsstapel und erhält dann +\coin[1] für den gerade gelegten Marker. Wird die Karte zum zweiten, dritten und vierten Mal ausgespielt, legt der Spieler jeweils einen weiten Marker auf den Stapel und erhält +\coin[2], +\coin[3] bzw. +\coin[4], egal welcher Spieler die vorherigen Marker auf den Stapel gelegt hat. Wird die Karte danach erneut ausgespielt, nimmt der Spieler die 4 \victorypointtoken-Marker (dafür aber kein \coin) und muss den ausgespielten \emph{BAUERNMARKT} entsorgen.

	\medskip

	Danach beginnt der Vorgang wieder von vorn und wird fortgesetzt, falls der Vorratsstapel leer ist.}
\end{tikzpicture}
\hspace{-0.6cm}
\begin{tikzpicture}
	\card
	\cardstrip
	\cardbanner{banner/white.png}
	\cardicon{icons/coin.png}
	\cardprice{3}
	\cardtitle{Gladiator}
	\cardcontent{Wenn du mindestens 1 Handkarte hast, musst du diese aufdecken. Wenn dein linker Mitspieler keine Karte mit gleichem Namen aufdecken kann oder will (z.B. auch, wenn du keine Handkarte aufdecken konntest, weil du keine hast), erhältst du zusätzlich +\coin[1] . Sind noch Karten auf dem \emph{GLADIATOR}-Vorratsstapel vorhanden, musst du eine entsorgen. Deckt der Mitspieler eine Karte mit gleichem Namen auf, erhältst du nur +\coin[2] und darfst keinen \emph{GLADIATOR} entsorgen.}
\end{tikzpicture}
\hspace{-0.6cm}
\begin{tikzpicture}
	\card
	\cardstrip
	\cardbanner{banner/white.png}
	\cardicon{icons/coin.png}
	\cardprice{3}
	\cardtitle{Katapult}
	\cardcontent{Wenn du mindestens 1 Handkarte hast, musst du auch eine entsorgen. Kostet die entsorgte Karte \coin[3] oder mehr, nimmt sich jeder Mitspieler (beginnend bei deinem linken Nachbarn) einen \emph{FLUCH}. Karten mit Schulden kosten nur dann \coin[3] oder mehr, wenn sie zusätzlich zu etwaigen Schulden-Kosten mindestens \coin[3] kosten. Ist die entsorgte Karte eine Geldkarte muss jeder Mitspieler - unabhängig von den Kosten der Karte - seine Handkarten auf 3 reduzieren.}
\end{tikzpicture}
\hspace{-0.6cm}
\begin{tikzpicture}
	\card
	\cardstrip
	\cardbanner{banner/white.png}
	\cardicon{icons/coin.png}
	\cardprice{3}
	\cardtitle{\footnotesize{Wagenrennen}}
	\cardcontent{Nimm deine aufgedeckte Karte nach dem Vergleich der Kosten mit der aufgedeckten Karte deines linken Mitspielers auf die Hand. Der Mitspieler legt seine aufgedeckte Karte zurück auf den Nachziehstapel.

	Kosten beide Karten gleich viel oder kostet die Karte des Mitspielers mehr, erhältst du nichts. Kostet deine Karte mehr erhältst du +1 \coin und +1 \victorypointtoken-Marker. Hast entweder du oder dein linker Mitspieler (auch nach dem eventuellen Mischen des Ablagestapels) keine Karte zum Aufdecken, erhältst du nichts.}
\end{tikzpicture}
\hspace{-0.6cm}
\begin{tikzpicture}
	\card
	\cardstrip
	\cardbanner{banner/orange.png}
	\cardicon{icons/coin.png}
	\cardprice{3}
	\cardtitle{Zauberin}
	\cardcontent{Spieler, die mit einer Reaktionskarte wie dem \emph{BURGGRABEN} (aus dem Basisspiel) reagieren möchten, müssen dies tun, sobald die \emph{ZAUBERIN} ausgespielt wurde, auch wenn der Angriff sie erst in ihrem nächsten Zug betrifft.

	\medskip

	Jeder Mitspieler erhält in seinem nächsten Zug für die erste gespielte Aktionskarte + 1 Karte sowie + 1 Aktion, darf aber den eigentlichen Effekt der Karte beim Ausspielen nicht durchführen. Anweisungen, die sich auf einen anderen Zeitpunkt im Spiel beziehen (z.B. die beim Kauf der Karte zum Tragen kommen), werden nicht beeinflusst.

	Um anzuzeigen, dass die erste ausgespielte Aktionskarte von der \emph{ZAUBERIN} beeinflusst wird, empfehlen wir, diese beim Ausspielen quer auszulegen. Karten, die bereits ausgespielt wurden (z.B. Dauerkarten wie das \emph{ARCHIV}), werden zu Beginn des Zuges normal abgehandelt und nicht von der \emph{ZAUBERIN} beeinflusst. Spielt ein Spieler in seiner Aktionsphase keine Aktionskarte aus, dafür aber in seiner Kaufphase eine \emph{KRONE} (kombinierte Aktions- und Geldkarte), kommt der Effekt der \emph{ZAUBERIN} zum Tragen, da es sich um eine Aktionskarte handelt, auch wenn diese in der Kaufphase ausgespielt wurde. Normalerweise kann der Spieler die + 1 Aktion zu diesem Zeitpunkt nicht nutzen, es sei denn, er kauft zum Beispiel eine \emph{VILLA}.}
\end{tikzpicture}
\hspace{-0.6cm}
\begin{tikzpicture}
	\card
	\cardstrip
	\cardbanner{banner/gold.png}
	\cardicon{icons/coin.png}
	\cardprice{4}
	\cardtitle{Felsen}
	\cardcontent{Wenn du diese Karte in deiner Kaufphase nimmst oder entsorgst, nimm ein \emph{SILBER} und lege es auf deinen Nachziehstapel. Wenn du diese Karte zu einem anderen Zeitpunkt (auch während des Zuges eines anderen Spielers) nimmst oder entsorgst, nimm ein \emph{SILBER} auf die Hand.}
\end{tikzpicture}
\hspace{-0.6cm}
\begin{tikzpicture}
	\card
	\cardstrip
	\cardbanner{banner/white.png}
	\cardicon{icons/coin.png}
	\cardprice{4}
	\cardtitle{Opfer}
	\cardcontent{Wenn die entsorgte Karte eine kombinierte Karte ist, erhältst du die Boni aller entsprechenden Typen dieser Karte. Entsorgst du eine Karte, die keinem der angegebenen Typen entspricht (z.B. einen \emph{FLUCH}), erhältst du nichts.}
\end{tikzpicture}
\hspace{-0.6cm}
\begin{tikzpicture}
	\card
	\cardstrip
	\cardbanner{banner/white.png}
	\cardicon{icons/coin.png}
	\cardprice{4}
	\cardtitle{Tempel}
	\cardcontent{Es dürfen nur Karten mit unterschiedlichem Namen entsorgt werden, z.B. ein \emph{KUPFER} und ein \emph{ANWESEN}.

	Auch wenn der \emph{TEMPEL}-Vorratsstapel leer ist, legst du einen \victorypointtoken-Marker auf den leeren Platz. Das kann relevant werden, wenn durch Anweisungen auf anderen Karten ein \emph{TEMPEL} in den Vorrat zurückgelegt wird (z.B. durch den \emph{BOTSCHAFTER} aus \emph{Seaside}) nehmen darf.

	\medskip

	Wenn du einen \emph{TEMPEL} nimmst, nimmst du auch alle \victorypointtoken-Marker, die zu diesem Zeitpunkt auf dem Vorratsstapel liegen.}
\end{tikzpicture}
\hspace{-0.6cm}
\begin{tikzpicture}
	\card
	\cardstrip
	\cardbanner{banner/white.png}
	\cardicon{icons/coin.png}
	\cardprice{4}
	\cardtitle{Villa}
	\cardcontent{Wenn du diese Karte in deiner Aktionsphase nimmst (z.B. durch die \emph{INGENIEURIN}), nimm sie sofort auf die Hand und erhalte + 1 Aktion. Dadurch kannst du z.B. die gerade genommene \emph{VILLA} sofort ausspielen. Wenn du diese Karte in deiner Kaufphase nimmst (z.B. indem du sie kaufst), nimm sie auf die Hand und kehre sofort in die Aktionsphase zurück, wo du + 1 Aktion hast. Hast du die Aktionsphase erneut komplett abgeschlossen, kehrst du wieder zur Kaufphase zurück. Hier kannst du weitere Geldkarten ausspielen (und z.B. die \emph{ARENA} kommt wieder zum Tragen). Wenn du diese Karte während des Zuges eines Mitspielers nimmst, nimmst du die Karte auf die Hand und erhältst zwar + 1 Aktion, kannst diese aber nicht nutzen, da es nicht dein Zug ist. Es ist möglich, mehrmals pro Zug (z.B. durch das Nehmen mehrerer \emph{VILLEN}) in die Aktionsphase zurückzukehren. Dies bedeutet aber nicht, dass du an den \enquote{Beginn deines Zuges} zurückkehrst. Anweisungen, die sich darauf beziehen, haben keine Auswirkung.}
\end{tikzpicture}
\hspace{-0.6cm}
\begin{tikzpicture}
	\card
	\cardstrip
	\cardbanner{banner/orange.png}
	\cardicon{icons/coin.png}
	\cardprice{5}
	\cardtitle{Archiv}
	\cardcontent{Lege die obersten drei Karten deines Nachziehstapels zur Seite und schau sie dir an. Nimm eine der Karten sofort auf die Hand und lege die anderen Karten unter dieses \emph{ARCHIV}. Spielst du zwei \emph{ARCHIVE}, lege die Karten für die nächsten Züge unter das jeweils ausgespielte \emph{ARCHIV}. Hast du nicht genügend Karten, um drei Karten zur Seite zu legen, legst du nur so viele wie möglich zur Seite. Das \emph{ARCHIV} wird in dem Spielzug abgelegt, in dem die letzte zur Seite gelegte Karte des jeweiligen \emph{ARCHIVS} auf die Hand genommen wurde.}
\end{tikzpicture}
\hspace{-0.6cm}
\begin{tikzpicture}
	\card
	\cardstrip
	\cardbanner{banner/gold.png}
	\cardicon{icons/coin.png}
	\cardprice{5}
	\cardtitle{Diebesgut}
	\cardcontent{Nimm dir jedes Mal, wenn du diese Karte spielst, einen \victorypointtoken-Marker und lege ihn bei dir ab.}
\end{tikzpicture}
\hspace{-0.6cm}
\begin{tikzpicture}
	\card
	\cardstrip
	\cardbanner{banner/white.png}
	\cardicon{icons/coin.png}
	\cardprice{5}
	\cardtitle{Emsiges Dorf}
	\cardcontent{Du darfst deinen Ablagestapel auch dann durchsehen, wenn du weißt, dass du keine \emph{SIEDLER} darin hast. Du darfst die Reihenfolge der Karten in deinem Ablagestapel nicht verändern.}
\end{tikzpicture}
\hspace{-0.6cm}
\begin{tikzpicture}
	\card
	\cardstrip
	\cardbanner{banner/white.png}
	\cardicon{icons/coin.png}
	\cardprice{5}
	\cardtitle{Forum}
	\cardcontent{Wenn du diese Karte kaufst, erhältst du + 1 Kauf. Du kannst beispielsweise mit \coin[\hspace{-0.3em}13] und nur einem freien Kauf, zuerst diese Karte kaufen und dann mit dem zusätzlichen Kauf noch eine \emph{PROVINZ}.}
\end{tikzpicture}
\hspace{-0.6cm}
\begin{tikzpicture}
	\card
	\cardstrip
	\cardbanner{banner/white.png}
	\cardicon{icons/coin.png}
	\cardprice{5}
	\cardtitle{Gärtnerin}
	\cardcontent{Ist diese Karte im Spiel und du nimmst eine Punktekarte – egal in welcher Spielphase – nimmst du dir einen \victorypointtoken-Marker und legst ihn bei dir ab. Wenn du mehrere Karten nimmst, nimmst du dir für jede genommene Karten einen \victorypointtoken-Marker. Hast du mehrere \emph{GÄRTNERINNEN} im Spiel, nimmst du dir für jede \emph{GÄRTNERIN} pro genommener Karte einen \victorypointtoken-Marker.

	Wenn du z.B. eine \emph{GÄRTNERIN} auf eine \emph{KRONE} spielst, befindet sich die \emph{GÄRTNERIN} trotzdem nur einmal im Spiel und du darfst dir pro genommener Karte nur einen \victorypointtoken-Marker nehmen.}
\end{tikzpicture}
\hspace{-0.6cm}
\begin{tikzpicture}
	\card
	\cardstrip
	\cardbanner{banner/white.png}
	\cardicon{icons/coin.png}
	\cardprice{5}
	\cardtitle{\footnotesize{Handelsplatz}}
	\cardcontent{Zu den Aktionskarten, die du zu diesem Zeitpunkt im Spiel hast zählen alle Aktionskarten, die du ausgespielt hast, Dauerkarten, die sich aus vergangenen Zügen im Spiel befinden und Reservekarten (aus \emph{Abenteuer}), die du in diesem Zug bereits aufgerufen hast. Wenn du diese Karte außerhalb deines Zuges nimmst, hast du keine Aktionskarten im Spiel und du darfst dir keine \victorypointtoken-Marker nehmen.}
\end{tikzpicture}
\hspace{-0.6cm}
\begin{tikzpicture}
	\card
	\cardstrip
	\cardbanner{banner/whitegold.png}
	\cardicon{icons/coin.png}
	\cardprice{5}
	\cardtitle{Krone}
	\cardcontent{Diese Karte ist eine kombinierte Aktions- und Geldkarte. Wenn du sie in deiner Aktionsphase ausspielst, darfst du eine Aktionskarte von deiner Hand wählen und ausspielen. Du nimmst die gewählte Karte nicht wieder auf die Hand, sondern spielst die Aktion ein zweites Mal. Dafür benötigst du keine weiteren Aktionen. Wählst du eine \emph{KRONE}, musst du diese auch als Aktionskarte ausspielen (und dann darfst du bis zu zwei weitere Aktionskarten jeweils zweimal spielen).

	Spielst du diese Karte in deiner Aktionsphase als Geldkarte aus (z.B. durch den \emph{GESCHICHTENERZÄHLER} aus \emph{Abenteuer}), darfst du trotzdem eine Aktionskarte zweimal ausspielen.

	\medskip

	Spielst du diese Karte in deiner Kaufphase, darfst du eine beliebige Geldkarte von deiner Hand wählen, sie ausspielen und zweimal ausführen. Wählst du eine \emph{KRONE}, spielst du diese aus und dann eine weitere Geldkarte von der Hand zweimal und dann noch eine Geldkarte zweimal.}
\end{tikzpicture}
\hspace{-0.6cm}
\begin{tikzpicture}
	\card
	\cardstrip
	\cardbanner{banner/white.png}
	\cardicon{icons/coin.png}
	\cardprice{5}
	\cardtitle{Legionär}
	\cardcontent{Mitspieler, die auf das Ausspielen dieser Karte mit einer Reaktionskarte reagieren möchten, müssen dies tun, bevor du dich entscheidest, ob du ein \emph{GOLD} aufdeckst oder nicht.

	\medskip

	Mitspieler, die bereits zwei oder weniger Karten auf der Hand haben, müssen keine Karte ablegen, müssen gleichwohl aber eine Karte ziehen.}
\end{tikzpicture}
\hspace{-0.6cm}
\begin{tikzpicture}
	\card
	\cardstrip
	\cardbanner{banner/gold.png}
	\cardicon{icons/coin.png}
	\cardprice{5}
	\cardtitle{Vermögen}
	\cardcontent{Diese Karte ist eine Geldkarte mit zusätzlichen Anweisungen. Sie hat den Wert \coin[6]. Außerdem erhältst du + 1 Kauf.

	\medskip

	Wenn du diese Karte ablegst (in der Regel in deiner Aufräumphase), nimm \hex[6] vom Vorrat. Dann kannst du sofort beliebig viele \hex (auch mehr als die \hex[6], die du durch das Ablegen dieser Karte erhalten hast) zurückzahlen.

	\medskip

	Wenn du diese Karte nicht ablegst (z.B. wenn du sie stattdessen entsorgst), erhältst du keine \hex. Wenn du diese Karte zweimal ausgespielt hast (z.B. durch eine \emph{KRONE}), erhältst du trotzdem nur \hex[6], da du nur eine Karte ablegst.}
\end{tikzpicture}
\hspace{-0.6cm}
\begin{tikzpicture}
	\card
	\cardstrip
	\cardbanner{banner/white.png}
	\cardicon{icons/coin.png}
	\cardprice{5}
	\cardtitle{Wilde Jagd}
	\cardcontent{Wählst du die erste Option, lege einen \victorypointtoken-Marker vom Vorrat auf den \emph{WILDE-JAGD}-Vorratsstapel.

	\medskip

	Wählst du die zweite Option und der \emph{ANWESEN}-Vorratsstapel ist leer (d.h. du kannst dir kein \emph{ANWESEN} nehmen), darfst du dir die \victorypointtoken-Marker vom \emph{WILDE-JAGD}-Vorratsstapel nicht nehmen. Du darfst aber diese Option trotzdem wählen.

	\medskip

	Ist der \emph{WILDE-JAGD}-Vorratsstapel leer, funktioniert das Ausspielen dieser Karte trotzdem in der beschriebenen Weise weiter. Nutzt die Platzhalterkarte, um den Vorratsstapel zu markieren.}
\end{tikzpicture}
\hspace{-0.6cm}
\begin{tikzpicture}
	\card
	\cardstrip
	\cardbanner{banner/gold.png}
	\cardicon{icons/coin.png}
	\cardprice{5}
	\cardtitle{Zauber}
	\cardcontent{Wenn du diese Karte ausspielst und dich für die zweite Option entscheidest, darfst du (musst aber nicht) sofort, wenn du die \emph{nächste} Karte in deinem Zug kaufst, eine Karte mit anderem Namen nehmen, die \emph{exakt so viel} kostet, wie die gekaufte Karte. Dann erst nimmst du die gekaufte Karte. Das kann wichtig bei Karten sein, die Anweisungen beim Nehmen einer Karte beinhalten.

	Spielst du mehrere \emph{ZAUBER} in einem Zug, darfst du dir für die nächste gekaufte Karte mehrere Karten mit anderem Namen als die gekaufte aber gleichen Kosten nehmen. Die Karten, die du nimmst müssen zwar einen anderen Namen als die Gekaufte haben, dürfen aber untereinander alle den gleichen Namen haben.}
\end{tikzpicture}
\hspace{-0.6cm}
\begin{tikzpicture}
	\card
	\cardstrip
	\cardbanner{banner/white.png}
	\cardicon{icons/hex.png}
	\cardprice{\textcolor{white}{4}}
	\cardtitle{Ingenieurin}
	\cardcontent{Du darfst dir keine Karte nehmen, die mehr als \coin[4] kostet oder mit \hex in den Kosten hat. Nimm die gewählte Karte.

	Dann darfst du diese \emph{INGENIEURIN} entsorgen. Wenn du das tust, nimm eine weitere Karte, die bis zu \coin[4] kostet. Dies kann die gleiche Karte wie die erste sein oder eine andere.}
\end{tikzpicture}
\hspace{-0.6cm}
\begin{tikzpicture}
	\card
	\cardstrip
	\cardbanner{banner/white.png}
	\cardicon{icons/hex.png}
	\cardprice{\textcolor{white}{8}}
	\cardtitle{\miniscule{KöniglicherSchmied}}
	\cardcontent{Du musst, nachdem du 5 Karten nachgezogen hast, alle deine Handkarten vorzeigen und jedes \emph{KUPFER}, das du zu diesem Zeitpunkt auf der Hand hast, ablegen.}
\end{tikzpicture}
\hspace{-0.6cm}
\begin{tikzpicture}
	\card
	\cardstrip
	\cardbanner{banner/white.png}
	\cardicon{icons/hex.png}
	\cardprice{\textcolor{white}{8}}
	\cardtitle{Lehnsherr}
	\cardcontent{Wähle eine Karte vom Vorrat, die zu diesem Zeitpunkt bis zu \coin[5] kostet, d.h. du darfst keine Karte eines leeren Stapels, eine nicht sichtbare Karte eines gemischten Stapels oder eine Karte eines Nicht-Vorratsstapels wählen.

	\medskip

	Behandle nun den ausgespielten \emph{LEHNSHERR}, wie die gewählte Karte (und nicht mehr als \emph{LEHNSHERR}) – bis sie nicht mehr im Spiel ist. Das heißt du befolgst alle Anweisungen der anderen Karte. Auch nimmt der \emph{LEHNSHERR} den Namen, die Kosten und den Typ der gewählten Karte an, bis er nicht mehr im Spiel ist. Als Dauerkarte bleibt dieser \emph{LEHNSHERR} ebenso im Spiel, wie er als Reservekarte (aus \emph{Abenteuer}) zur Seite gelegt wird. Spielst du diesen \emph{LEHNSHERR} auf einen \emph{THRONSAAL} (aus dem \emph{Basisspiel}), wählst du beim ersten Ausspielen die Karte, die dieser \emph{LEHNSHERR} ab sofort ist – beim zweiten Ausspielen ist er damit wieder genau diese Karte – du darfst keine andere Karte wählen. Erst mit dem Ausspielen des \emph{LEHNSHERRN} nimmt er Typ und Namen der gewählten Karte an – d.h. du darfst ihn nicht als \emph{KRONE} in deiner Kaufphase spielen, da er selbst keine Geldkarte ist und nicht in der Kaufphase ausgespielt werden darf.}
\end{tikzpicture}
\hspace{-0.6cm}
\begin{tikzpicture}
	\card
	\cardstrip
	\cardbanner{banner/white.png}
	\cardicon{icons/hex.png}
	\cardprice{\textcolor{white}{8}}
	\cardtitle{Stadtviertel}
	\cardcontent{Du musst deine Handkarten aufdecken. Für jede Aktionskarte (auch ggf. kombinierte), die du aufdeckst, ziehst du eine Karte nach.}
\end{tikzpicture}
\hspace{-0.6cm}
\begin{tikzpicture}
	\card
	\cardstrip
	\cardbanner{banner/white.png}
	\cardicon{icons/coin.png}
	\cardprice{8}
	\cardiconaddition{icons/hex.png}
	\cardpriceaddition{\textcolor{white}{8}}
	\cardtitle{\quad Reichtum}
	\cardcontent{Es werden nur alle \coin verdoppelt, die du vor dem Ausspielen dieser Karte ausgespielt hast und nur, wenn du in diesem Zug noch keinen \emph{REICHTUM} ausgespielt hast. Für jedes weitere Ausspielen eines \emph{REICHTUMS} erhältst du nur + 1 Kauf.}
\end{tikzpicture}
\hspace{-0.6cm}
\begin{tikzpicture}
	\card
	\cardstrip
	\cardbanner{banner/green.png}
	\cardtitle{Schlösser (1/2)\quad}
	\cardcontent{\emph{Schloss-Karten:} Der Schloss-Stapel ist ein gemischter Vorratsstapel. Alle Schlösser werden nach Kosten sortiert auf dem Vorratsstapel bereitgelegt (die teuerste zuunterst).

	\bigskip

	\emph{Bescheidenes Schloss:} Spielst du sie in deiner Kaufphase aus, ist sie \coin[1] wert. Bei Spielende erhältst du pro Karte, die den Typ \emph{SCHLOSS} beinhaltet, einen \victorypointtoken-Marker.

	\medskip

	\emph{Verfallendes Schloss:} Diese Karte ist zu Spielende 1 \victorypoint wert – wie ein \emph{ANWESEN}. Wenn du diese Karte während des Spiels nimmst, nimm dir einen \victorypointtoken-Marker sowie ein \emph{SILBER} vom Vorrat. Wenn du diese Karte während des Spiels entsorgst, nimm dir einen weiteren \victorypointtoken-Marker sowie ein \emph{SILBER} vom Vorrat.

	\medskip

	\emph{Kleines Schloss:} Spielst du sie in deiner Aktionsphase aus, entsorge dieses \emph{KLEINE SCHLOSS} oder eine andere \emph{SCHLOSS}-Karte aus deiner Hand. Wenn du das tust, nimm dir die \emph{SCHLOSS}-Karte vom Vorratsstapel, die zu diesem Zeitpunkt oben liegt. Dies kann eine teurere sein, als die, die du entsorgst. Du musst die Kosten nicht bezahlen. Bei Spielende ist diese Karte 2 \victorypoint wert.}
\end{tikzpicture}
\hspace{-0.6cm}
\begin{tikzpicture}
	\card
	\cardstrip
	\cardbanner{banner/green.png}
	\cardtitle{Schlösser (2/2)\quad}
	\cardcontent{\emph{Spukschloss:} Diese Karte ist zu Spielende 2 \victorypoint wert. Wenn du diese Karte während deines Zuges nimmst (kaufst oder auf andere Art und Weise nimmst), nimm dir ein \emph{GOLD} vom Vorrat. Ist kein \emph{GOLD} mehr im Vorrat, erhältst du nichts. Außerdem (egal ob du ein \emph{GOLD} nehmen kannst oder nicht) müssen alle Mitspieler mit 5 oder mehr Handkarten 2 Handkarten auf ihren Nachziehstapel zurücklegen. Da diese Karte keine Angriffskarte ist, dürfen die Mitspieler keine Reaktionskarte spielen.

	\smallskip

	\emph{Reiches Schloss:} Spielst du sie in deiner Aktionsphase aus, lege beliebig viele Punktekarten (auch ggf. kombinierte) aus deiner Hand ab. Pro abgelegter Karte erhältst du +\coin[2]. Bei Spielende ist diese Karte 3 \victorypoint wert.

	\smallskip

	\emph{Ausgedehntes Schloss:} Wenn du diese Karte kaufst oder auf andere Art und Weise nimmst, nimm ein \emph{HERZOGTUM} oder drei \emph{ANWESEN}. Bei Spielende ist diese Karte 4 \victorypoint wert.

	\smallskip

	\emph{Prunkschloss:} Wenn du diese Karte kaufst oder auf andere Art und Weise nimmst, zeige deine Handkarten vor. Nimm einen \victorypointtoken-Marker vom Vorrat für jede Punktekarte (auch ggf. kombinierte), die du zu diesem Zeitpunkt auf der Hand oder im Spiel hast.

	\smallskip
 
	\emph{Königsschloss:} Bei Spielende erhältst du pro Karte, die den Typ \emph{SCHLOSS} beinhaltet (inklusive dieser Karte) 2 \victorypoint.}
\end{tikzpicture}
\hspace{-0.6cm}
\begin{tikzpicture}
	\card
	\cardstrip
	\cardbanner{banner/white.png}
	\cardtitle{Katapult/Felsen\qquad}
	\cardcontent{Spielvorbereitung: Legt auf diese Karte 5 Felsen und oben darauf 5 Katapulte.

	\bigskip

	Es darf immer nur die oberste Karte des Stapels genommen oder gekauft werden.}
\end{tikzpicture}
\hspace{-0.6cm}
\begin{tikzpicture}
	\card
	\cardstrip
	\cardbanner{banner/white.png}
	\cardtitle{\scriptsize{Gladiator/Reichtum}\qquad}
	\cardcontent{Spielvorbereitung: Legt auf diese Karte 5 Reichtum und oben darauf 5 Gladiatoren. 

	\bigskip

	Es darf immer nur die oberste Karte des Stapels genommen oder gekauft werden.}
\end{tikzpicture}
\hspace{-0.6cm}
\begin{tikzpicture}
	\card
	\cardstrip
	\cardbanner{banner/white.png}
	\cardtitle{\scriptsize{Siedler/Emsiges Dorf}\qquad}
	\cardcontent{Spielvorbereitung: Legt auf diese Karte 5 Emsige Dörfer und oben darauf 5 Siedler. 

	\bigskip

	Es darf immer nur die oberste Karte des Stapels genommen oder gekauft werden.}
\end{tikzpicture}
\hspace{-0.6cm}
\begin{tikzpicture}
	\card
	\cardstrip
	\cardbanner{banner/white.png}
	\cardtitle{\scriptsize{Patrizier/Handelsplatz}\qquad}
	\cardcontent{Spielvorbereitung: Legt auf diese Karte 5 Handelsplätze und oben darauf 5 Patrizier. 

	\bigskip

	Es darf immer nur die oberste Karte des Stapels genommen oder gekauft werden.}
\end{tikzpicture}
\hspace{-0.6cm}
\begin{tikzpicture}
	\card
	\cardstrip
	\cardbanner{banner/white.png}
	\cardtitle{\scriptsize{Feldlager/Diebesgut}\qquad}
	\cardcontent{Spielvorbereitung: Legt auf diese Karte 5 Diebesgut und oben darauf 5 Feldlager. 

	\bigskip

	Es darf immer nur die oberste Karte des Stapels genommen oder gekauft werden.}
\end{tikzpicture}
\hspace{-0.6cm}
\begin{tikzpicture}
	\card
	\cardstrip
	\cardbanner{banner/white.png}
	\cardtitle{Ereignisse (1/4)\quad}
	\cardcontent{Ereignisse können nur in der Kaufphase erworben werden. Dies benötigt 1 Kauf sowie genügend (vorher ausgespielte) Geldwerte. Erwirbst du ein Ereignis mit Schulden \hex, nimmst du die entsprechende Anzahl \hex-Marker an dich. Die Kosten (\hex \coin) sind auf jedem Ereignis oben links zu finden. Sobald du ein Ereignis erwirbst, führst du die darauf beschriebene Anweisung aus. Du nimmst das Ereignis aber \emph{nicht} an dich.

	\bigskip
 
	\emph{Aufstieg:} Wenn du keine Aktionskarte entsorgst passiert nichts weiter.

	\medskip
 
	\emph{Erforschen:} Jeder Erwerb eines \emph{ERFORSCHEN} gibt dir den Kauf zurück, den du für den Erwerb benötigt hast. Mit \coin[7] und 1 Kauf kannst du zum Beispiel 2 \emph{ERFORSCHEN} erwerben und dann eine Karte kaufen oder ein Ereignis für \coin[3] erwerben.}
\end{tikzpicture}
\hspace{-0.6cm}
\begin{tikzpicture}
	\card
	\cardstrip
	\cardbanner{banner/white.png}
	\cardtitle{Ereignisse (2/4)\quad}
	\cardcontent{\emph{Steuer:} Auf jeden Vorratsstapel (d.h. alle Königreichkarten, Fluchkarten und Basiskarten, nicht Ereignisse und Landmarken) wird in der Spielvorbereitung 1 \hex-Marker gelegt. Spieler, die eine Karte von einem Stapel kaufen, auf dem \hex-Marker liegen, müssen alle Marker des Stapels nehmen. Nimmt ein Spieler eine Karte auf andere Art und Weise (d.h. er kauft sie nicht), werden eventuelle \hex-Marker auf die nächste Karte des Vorratsstapels gelegt. Wenn du dieses Ereignis erwirbst, legst du 2 \hex-Marker auf einen beliebigen Vorratsstapel – egal ob dort zu diesem Zeitpunkt bereits \hex-Marker liegen oder nicht.

	\medskip
 
	\emph{Bankett:} Du kannst dieses Ereignis auch kaufen, wenn der \emph{KUPFER}-Vorratsstapel aufgebraucht ist.

	\medskip
 
	\emph{Versalztes Land:} Wenn die entsorgte Karte eine Anweisung beinhaltet, die eintritt, wenn diese Karte entsorgt wird, musst du diese Anweisung ausführen.

	\medskip
 
	\emph{Ritual:} Wenn du keinen \emph{FLUCH} nehmen kannst (z.B. weil der Vorratsstapel leer ist), passiert nichts. Es werden nur die \coin-Kosten gezählt – für \hex-Kosten oder \potion-Kosten (aus \emph{Alchemisten}) erhältst du nichts.}
\end{tikzpicture}
\hspace{-0.6cm}
\begin{tikzpicture}
	\card
	\cardstrip
	\cardbanner{banner/white.png}
	\cardtitle{Ereignisse (3/4)\quad}
	\cardcontent{\emph{Glücksfall:} Wenn weniger als 3 \emph{GOLD} im Vorrat sind, nimm dir die restlichen \emph{GOLD}.

	\medskip
 
	\emph{Eroberung:} Pro \emph{SILBER}, das du in diesem Zug genommen hast (inklusive der 2 \emph{SILBER} durch diese Karte), nimm dir einen \victorypointtoken-Marker vom Vorrat. Dies ist kumulativ. Erwirbst du z.B. eine \emph{EROBERUNG} und erhältst dafür 2 \victorypointtoken-Marker (für die beiden SILBER durch diese Karte) und dann noch eine \emph{EROBERUNG}, für die du 2 \emph{SILBER} nehmen kannst, erhältst du für die zweite \emph{EROBERUNG} schon 4 \victorypointtoken-Marker. Sind nicht genügend \emph{SILBER} im Vorrat, nimmst du dir so viele wie möglich. Dann erhältst du aber auch entsprechend weniger \victorypointtoken-Marker.

	\medskip
 
	\emph{Beherrschen:} Ist der \emph{PROVINZ}-Vorratsstapel leer oder du kannst aus einem anderen Grund keine \emph{PROVINZ} nehmen, hat dieses Ereignis keine Auswirkung.

	\medskip
 
	\emph{Hochzeit:} Den \victorypointtoken-Marker nimmst du in jedem Fall – auch wenn der \emph{GOLD}-Vorratsstapel leer ist.}
\end{tikzpicture}
\hspace{-0.6cm}
\begin{tikzpicture}
	\card
	\cardstrip
	\cardbanner{banner/white.png}
	\cardtitle{Ereignisse (4/4)\quad}
	\cardcontent{\emph{Siegeszug:} Wenn du ein \emph{ANWESEN} nimmst, nimmst du für jede Karte, die du in diesem Zug bereits genommen hast (inklusive dem \emph{ANWESEN} jedoch nicht für Ereignisse), einen \victorypointtoken-Marker. Wenn du kein \emph{ANWESEN} nehmen kannst (z.B. weil der Vorratsstapel leer ist), passiert nichts.

	\medskip
 
	\emph{Schlacht:} Du kannst dieses Ereignis auch erwerben wenn der \emph{HERZOGTUM}-Vorratsstapel leer ist. Die bis zu ausgewählten 5 Karten verbleiben in deinem Ablagestapel. Die restlichen Karten mischst du in deinen Nachziehstapel.

	\medskip
 
	\emph{Spende:} Befinden sich unter den entsorgten Karten welche, die Anweisungen beinhalten, die beim Entsorgen ausgeführt werden, musst du diese ausführen, bevor du die restlichen Karten mischst. Die \emph{SPENDE} wird erst nach dem Zug, in dem sie erworben wird, ausgeführt (d.h. zwischen zwei Zügen). Damit hat zum Beispiel die \emph{BESESSENHEIT} (aus \emph{Alchemisten}) auf diese Anweisung keine Auswirkung.}
\end{tikzpicture}
\hspace{-0.6cm}
\begin{tikzpicture}
	\card
	\cardstrip
	\cardbanner{banner/green.png}
	\cardtitle{\footnotesize{Landmarken (1/8)}\quad}
	\cardcontent{Einige Landmarken enthalten Anweisungen für die Spielvorbereitung (unterhalb der Trennlinie). Spielt ihr mit einer dieser Karten, beachtet dies in der Spielvorbereitung.

	\medskip
 
	Darfst du dir auf Grund einer Anweisung \victorypointtoken-Marker von einer Landmarkenkarte oder einem Vorratsstapel nehmen und dort sind zu diesem Zeitpunkt keine \victorypointtoken-Marker vorhanden, erhältst du nichts. Sind die zu Spielbeginn platzierten \victorypointtoken-Marker aufgebraucht, werden keine neuen \victorypointtoken-Marker platziert.

	\bigskip
 
	\emph{Aquädukt:} Wenn du eine Geldkarte von einem Vorratsstapel nimmst, auf dem ein oder mehrere \victorypointtoken-Marker liegen (auch ggf. kombinierte Karten oder \emph{KUPFER}, wenn dort durch Anweisungen auf Karten oder Ereignissen \victorypointtoken-Marker platziert wurden), nimm einen \victorypointtoken-Marker und lege ihn hierher auf das \emph{AQUÄDUKT}.

	\smallskip
 
	Wenn du eine Punktekarte (auch ggf. kombinierte) nimmst, nimm dir alle \victorypointtoken{\ }-Marken, die zu diesem Zeitpunkt hier auf dem \emph{AQUÄDUKT} liegen.

	\smallskip
 
	Wenn du eine kombinierte Geld- und Punktekarte nimmst, kannst du dich entscheiden, in welcher Reihenfolge du die Anweisungen ausführst.}
\end{tikzpicture}
\hspace{-0.6cm}
\begin{tikzpicture}
	\card
	\cardstrip
	\cardbanner{banner/green.png}
	\cardtitle{\footnotesize{Landmarken (2/8)}\quad}
	\cardcontent{\emph{Arena:} Beginnst du (z.B. durch die \emph{VILLA}) in deinem Zug mehrfach mit deiner Kaufphase, kannst du die \emph{ARENA} mehrfach nutzen.

	\medskip
 
	\emph{Badehaus:} Egal ob du eine Karte kaufst oder auf andere Art und Weise nimmst (bzw. nehmen musst) – erhältst du in diesem Fall keine \victorypointtoken{\ }-Marker vom \emph{BADEHAUS}. Wer ein Ereignis erwirbt, nimmt damit keine Karte und kann – insofern keine andere Karte genommen wurde - 2 \victorypointtoken-Marker von hier nehmen.

	\medskip
 
	\emph{Basilika:} Für jede Karte die du kaufst, nimmst du 2 \victorypointtoken-Marker von der \emph{BASILIKA}, falls du zu diesem Zeitpunkt mindestens \coin[2] ausgespielt aber noch nicht verbraucht hast. Hast du beispielsweise \coin[4] und 3 Käufe, kannst du ein \emph{KUPFER} kaufen (\coin[4] übrig), dir 2 \victorypointtoken-Marker nehmen, ein \emph{ANWESEN} kaufen (\coin[2] übrig), dir 2 \victorypointtoken-Marker nehmen und ein weiteres \emph{ANWESEN} kaufen (\emph{0} übrig) – für den letzten Kauf erhältst du keine \victorypointtoken-Marker.}
\end{tikzpicture}
\hspace{-0.6cm}
\begin{tikzpicture}
	\card
	\cardstrip
	\cardbanner{banner/green.png}
	\cardtitle{\footnotesize{Landmarken (3/8)}\quad}
	\cardcontent{\emph{Bollwerk:} Hier werden alle Geldkarten (auch ggf. kombinierte) ausgewertet, die im Spiel benutzt wurden (auch ggf. Geldkarten, die im \emph{SCHWARZMARKT} (aus Basisspiel \emph{Special Edition} bzw. \emph{Promokarte}) enthalten waren). Haben zwei oder mehrere Spieler die gleiche höchste Anzahl einer Geldkarte, erhalten alle diese Spieler 5 \victorypointtoken-Marker.

	\medskip
 
	\emph{Brunnen:} Du erhältst entweder 15 \victorypoint oder 0 \victorypoint. Es gibt keinen Extra-Bonus, wenn du mehr als 10 \emph{KUPFER} besitzt.

	\medskip
 
	\emph{Entweihter Schrein:} Immer wenn du eine beliebige Aktionskarte nimmst und auf dem entsprechenden Vorratsstapel ein oder mehrere \victorypointtoken-Marker liegen (egal ob sie dort auf Grund der Anweisung auf dieser Landmarken-Karte oder einer anderen Karte, Ereignis oder Landmarken-Karte liegen), nimm einen \victorypointtoken-Marker von dort und lege ihn hierher auf den \emph{ENTWEIHTEN SCHREIN}.

	\smallskip
 
	Nur wenn du einen \emph{FLUCH} kaufst (nicht, wenn du ihn auf andere Art und Weise nimmst), nimmst du alle \victorypointtoken-Marker, die zu diesem Zeitpunkt hier liegen.

	In der Spielvorbereitung legt ihr auf jeden Vorratsstapel, der den Typ AKTION, nicht aber den Typ SAMMLUNG (also nicht auf die Karten \emph{BAUERNMARKT}, \emph{TEMPEL} und \emph{WILDE JAGD}) beinhaltet, 2 \victorypointtoken-Marker.}
\end{tikzpicture}
\hspace{-0.6cm}
\begin{tikzpicture}
	\card
	\cardstrip
	\cardbanner{banner/green.png}
	\cardtitle{\footnotesize{Landmarken (4/8)}\quad}
	\cardcontent{\emph{Gebirgspass:} Diese Landmarken-Karte wird genau einmal pro Spiel ausgeführt – nämlich nach Beendigung des Zuges, in dem ein Spieler die erste \emph{PROVINZ} aus dem Vorrat nimmt. Entsorgt vorher ein Spieler bereits eine \emph{PROVINZ} (z.B. durch das Ereignis \emph{VERSALZTES LAND}), hat jener Spieler diese \emph{PROVINZ} aber nicht vorher genommen und erfüllt deshalb diese Bedingung auch noch nicht. In einem Spiel, indem keine \emph{PROVINZ} genommen wird, findet diese Landmarken-Karte keine Anwendung.

	\smallskip
 
	Der \emph{GEBIRGSPASS} wird zwischen zwei Zügen ausgeführt und kann damit z.B. von der \emph{BESESSENHEIT} (aus \emph{Alchemisten}) nicht beeinflusst werden. Der Mitspieler links von dem Spieler, der die erste \emph{PROVINZ} genommen hat, beginnt mit einem Gebot oder passt. Ein Gebot besteht aus einer Anzahl \hex zwischen \hex[1] und \hex[\hspace{-0.25em}40]. Der nächste Spieler muss mindestens \hex[1] mehr bieten als der vorherige oder passen. Ein Gebot von \hex[\hspace{-0.25em}40] kann nicht überboten werden. Haben alle Spieler ein Gebot abgegeben oder gepasst, bzw. wurde bereits das Höchstgebot von \hex[\hspace{-0.25em}40] erreicht, erhält der Spieler mit dem höchsten Gebot die entsprechende Anzahl \hex-Marker sowie 8 \victorypointtoken-Marker. Passen alle Spieler, erhält keiner etwas.}
\end{tikzpicture}
\hspace{-0.6cm}
\begin{tikzpicture}
	\card
	\cardstrip
	\cardbanner{banner/green.png}
	\cardtitle{\footnotesize{Landmarken (5/8)}\quad}
	\cardcontent{\emph{Grabmal:} Dies funktioniert auch außerhalb deines Zuges (z.B. mit dem \emph{TRICKSER} aus \emph{Intrige}) oder wenn du eine Karte entsorgst, die nicht deine eigene ist (z.B. durch das Ereignis \emph{VERSALZTES LAND}).

	\medskip
 
	\emph{Kolonnaden:} Wenn du eine Aktionskarte kaufst (nicht, wenn du sie auf andere Art und Weise nimmst), musst du eine Karte mit dem gleichen Namen bereits im Spiel haben, um 2 \victorypointtoken-Marker von hier zu erhalten. Karten eines Stapels haben nicht unbedingt alle den gleichen Namen (z.B. bei gemischten Stapeln).

	\medskip
 
	\emph{Labyrinth:} Dies kann nur einmal pro Zug eines Spielers eintreten, nämlich genau in dem Moment, in dem er die zweite Karte in seinem Zug nimmt. Nimmt er außerhalb seines Zuges zwei Karten, erhält er nichts.

	\medskip
 
	\emph{Mauer:} Hast du mehr als 15 Karten in deinem Kartensatz, bekommst für jede Karte darüber hinaus 1 \victorypoint. Spieler, die zum Beispiel 27 Karten im Kartensatz haben, erhalten -12 \victorypoint, Spieler mit 14 Karten im Kartensatz erhalten keinen \victorypoint Abzug. Die Gesamtpunktzahl kann damit auch negativ sein.}
\end{tikzpicture}
\hspace{-0.6cm}
\begin{tikzpicture}
	\card
	\cardstrip
	\cardbanner{banner/green.png}
	\cardtitle{\footnotesize{Landmarken (6/8)}\quad}
	\cardcontent{\emph{Museum:} Auch Karten, die vom gleichen Stapel stammen, aber unterschiedliche Namen haben (z.B. gemischte Stapel), werden mit jeweils 2 \victorypoint abgerechnet.

	\medskip
 
	\emph{Obelisk:} Es zählen alle Karten des gewählten Stapels, auch wenn sie unterschiedliche Namen haben (z.B. bei gemischten Stapeln).

	\smallskip
 
	Zu Spielbeginn ermittelt ihr einen zufälligen Stapel, der den Typ AKTION beinhaltet (auch ggf. kombinierte Karten) und zum Vorrat gehört. \emph{RUINEN} (aus \emph{Dark Ages}) können bestimmt werden, ebenfalls der Stapel, der auch als Bannstapel für die \emph{JUNGE HEXE} (aus \emph{Reiche Ernte}) genutzt wird. Dazu zählen jedoch nicht die Eintausch- und Preiskarten (aus \emph{Reiche Ernte}), da diese nicht zum Vorrat gehören.

	\medskip
 
	\emph{Obstgarten:} Du erhältst keinen zusätzlichen Bonus, wenn du zum Beispiel von einer Aktionskarte 6 Exemplare besitzt, d.h. du erhältst für eine Aktionskarte, von der du 3 Exemplare besitzt genauso 4 \victorypoint wie für eine, von der du 7 Exemplare besitzt.}
\end{tikzpicture}
\hspace{-0.6cm}
\begin{tikzpicture}
	\card
	\cardstrip
	\cardbanner{banner/green.png}
	\cardtitle{\footnotesize{Landmarken (7/8)}\quad}
	\cardcontent{\emph{Palast:} Wenn du bei Spielende beispielsweise 7 \emph{KUPFER}, 5 \emph{SILBER} und 2 \emph{GOLD} in deinem Kartensatz hast, erhältst du 6 \victorypoint, da du zwei komplette Sätze aus je 1 \emph{KUPFER}, \emph{SILBER} und \emph{GOLD} besitzt. Hättest du noch ein drittes \emph{GOLD}, würdest du 9\victorypoint  erhalten.

	\medskip
 
	\emph{Räuberfestung:} Hast du bei Spielende zum Beispiel 3 \emph{SILBER} und 1 \emph{GOLD} in deinem Kartensatz, werden dir 8 \victorypoint abgezogen. Die Gesamtpunktzahl kann damit auch negativ sein.

	\medskip
 
	\emph{Schlachtfeld:} Du erhältst 2 \victorypointtoken-Marker von hier, egal ob du die Punktekarte (auch ggf. kombinierte) kaufst oder auf andere Art und Weise nimmst. Dies funktioniert auch außerhalb deines Zuges. Falls mehrere Spieler eine Punktekarte nehmen, wird dies in Spielerreihenfolge (beginnend bei dem Spieler links des aktuellen Spielers) getan.

	\medskip
 
	\emph{Triumphbogen:} Wenn du bei Spielende beispielsweise 7 \emph{VILLEN} und 4 \emph{WILDE JAGDEN} (und keine andere (auch ggf. kombinierte) Aktionskarte häufiger) in deinem Kartensatz hast, erhältst du 12 \victorypoint (d.h. 3 \victorypoint für jede der 4 \emph{WILDE JAGDEN}). Hast du neben 7 \emph{VILLEN} auch 7 \emph{WILDE JAGDEN}, erhältst du für beide zusammen 21 \victorypoint.}
\end{tikzpicture}
\hspace{-0.6cm}
\begin{tikzpicture}
	\card
	\cardstrip
	\cardbanner{banner/green.png}
	\cardtitle{\footnotesize{Landmarken (8/8)}\quad}
	\cardcontent{\emph{Turm:} Der Vorratsstapel muss leer sein. Ein gemischter Stapel, bei dem nur eine Sorte Karten fehlt, zählt nicht. Die Vorratsstapel mit Punktekarten zählen ebenfalls nicht, ein leerer Fluch-Stapel aber schon.

	\medskip
 
	\emph{Wolfsbau:} Du bekommst keine Minuspunkte durch den \emph{WOLFSBAU}, wenn du von einer Karte gar keine bzw. zwei oder mehr Stück in deinem kompletten Kartensatz besitzt. Hast du zum Beispiel einen \emph{FLUCH} in deinem Nachziehstapel und einen in deinem Ablagestapel, hast du insgesamt zwei \emph{FLÜCHE} und erhältst keine Minuspunkte durch den \emph{WOLFSBAU}. Die Gesamtpunktzahl kann negativ sein.}
\end{tikzpicture}
\hspace{-0.6cm}
\begin{tikzpicture}
	\card
	\cardstrip
	\cardbanner{banner/white.png}
	\cardtitle{\scriptsize{Empfohlene 10er Sätze\qquad}}
	\cardcontent{\emph{Basis Einführung} (\underline{Ereignisse und Landmarken-Karten}):\\
	\underline{Turm}, \underline{Hochzeit}, Schlösser (alle 8 bzw. 12 Schlosskarten), Wagenrennen, Stadtviertel, Ingenieurin, Bauernmarkt, Forum, Legionär, Patrizier/Handelsplatz, Opfer, Villa

	\smallskip

	\emph{Fortgeschrittene Einführung} (\underline{Ereignisse und Landmarken-Karten}):\\
	\underline{Arena}, \underline{Triumphbogen}, \underline{Hochzeit}, \underline{Spende}, Archiv, Vermögen, Katapult/Felsen, Krone, Zauberin, Gladiator/Reichtum, Gärtnerin, Königlicher Schmied, Siedler/Emsiges Dorf, Tempel

	\smallskip

	\emph{Alles in Maßen} (Empires + \underline{Ereignisse und Landmarken-Karten} + \textit{Basisspiel}):\\
	\underline{Obstgarten}, \underline{Glücksfall}, Zauberin, Forum, Legionär, Lehnsherr, Tempel, \textit{Keller}, \textit{Bibliothek}, \textit{Umbau}, \textit{Dorf}, \textit{Werkstatt}

	\smallskip

	\emph{Silberne Kugeln} (Empires + \underline{Ereignisse und Landmarken-Karten} + \textit{Basisspiel}):\\
	\underline{Aquädukt}, \underline{Eroberung}, Katapult/Felsen, Zauber, Bauernmarkt, Gärtnerin, Patrizier/Handelsplatz, \textit{Bürokrat}, \textit{Gärtner}, \textit{Laboratorium}, \textit{Markt}, \textit{Geldverleiher}}
\end{tikzpicture}
\hspace{-0.6cm}
\begin{tikzpicture}
	\card
	\cardstrip
	\cardbanner{banner/white.png}
	\cardtitle{\scriptsize{Empfohlene 10er Sätze\qquad}}
	\cardcontent{\emph{Köstliche Folter} (Empires + \underline{Ereignisse und Landmarken-Karten} + \textit{Intrige}):\\
	\underline{Arena}, \underline{Bankett}, Schlösser (alle 8 bzw. 12 Schlosskarten), Krone, Gärtnerin, Opfer, Siedler/Emsiges Dorf, \textit{Baron}, \textit{Brücke}, \textit{Harem}, \textit{Eisenhütte}, \textit{Kerkermeister}

	\smallskip

	\emph{Buddy-Prinzip} (Empires + \underline{Ereignisse und Landmarken-Karten} + \textit{Intrige}):\\
	\underline{Aussaat}, \underline{Wolfsbau}, Archiv, Vermögen, Katapult/Felsen, Ingenieurin, Forum, \textit{Maskerade}, \textit{Bergwerk}, \textit{Adlige}, \textit{Handlanger}, \textit{Handelsposten}

	\smallskip

	\emph{Kontrollbereich} (Empires + \underline{Ereignisse und Landmarken-Karten} + \textit{Abenteuer}):\\
	\underline{Bankett}, \underline{Bollwerk}, Vermögen, Katapult/Felsen, Zauber, Krone, Bauernmarkt, \textit{Königliche Münzen}, \textit{Page}, \textit{Relikt}, \textit{Schatz}, \textit{Weinhändler}

	\smallskip

	\emph{Kein Geld, keine Probleme} (Empires + \underline{Ereignisse und Landmarken-Karten} + \textit{Abenteuer}):\\
	\underline{Räuberfestung}, Archiv, Feldlager/Diebesgut, Königlicher Schmied, Tempel, Villa, \textit{Mission}, \textit{Verlies}, \textit{Duplikat}, \textit{Gefolgsmann}, \textit{Kleinbauer}, \textit{Transformation}}
\end{tikzpicture}
\hspace{0.6cm}

		% Basic settings for this card set
\renewcommand{\cardcolor}{nocturne}
\renewcommand{\cardextension}{Erweiterung X}
\renewcommand{\cardextensiontitle}{Nocturne}
\renewcommand{\seticon}{nocturne.png}

\clearpage
\newpage
\section{\cardextension \ - \cardextensiontitle \ (Rio Grande Games 2018)}

\begin{tikzpicture}
	\card
	\cardstrip
	\cardbanner{banner/white.png}
	\cardicon{icons/coin.png}
	\cardprice{2}
	\cardtitle{Druidin}
	\cardcontent{In der Spielvorbereitung werden die obersten 3 \emph{Gaben} aufgedeckt neben dem \emph{Druidinnen}-Stapel zur Seite gelegt. Diese 3 \emph{Gaben} werden in diesem Spiel \emph{ausschließlich} für die \emph{DRUIDIN} verwendet. Verwendet ihr weitere SEGEN-Karten im Spiel, besteht der entsprechende \emph{Gaben}-Stapel aus den restlichen neun \emph{Gaben}.
		
	\medskip
		
	Wenn du die \emph{DRUIDIN} ausspielst, wähle eine der drei zur Seite gelegten \emph{Gaben}, empfange sofort die entsprechende \emph{Gabe}, lasse die \emph{Gabe} aber zur Seite gelegt (auch wenn die \emph{Gabe} eigentlich sagt, du sollst sie bis zu deiner Aufräumphase aufbewahren).}
\end{tikzpicture}
\hspace{-0.6cm}
\begin{tikzpicture}
	\card
	\cardstrip
	\cardbanner{banner/white.png}
	\cardicon{icons/coin.png}
	\cardprice{2}
	\cardtitle{\footnotesize{Fährtensucher}}
	\cardcontent{In der Spielvorbereitung erhält jeder Spieler ein ERBSTÜCK \emph{BEUTEL} und dafür ein \emph{KUPFER} weniger.
		
	\medskip
		
	Wenn du diese Karte ausspielst und danach mehr als eine Karte nimmst (kaufst oder auf andere Art und Weise nimmst), darfst du für jede genommene Karte neu entscheiden, ob du sie auf deinen Nachziehstapel legst. Wenn du die durch den \emph{FÄHRTENSUCHER} empfangene \emph{Gabe} ausführst und dadurch eine Karte nimmst (z.B. ein \emph{SILBER} durch \emph{GESCHENK DES BERGES}), darfst du diese auf deinen Nachziehstapel legen.}
\end{tikzpicture}
\hspace{-0.6cm}
\begin{tikzpicture}
	\card
	\cardstrip
	\cardbanner{banner/white.png}
	\cardicon{icons/coin.png}
	\cardprice{2}
	\cardtitle{Fee}
	\cardcontent{In der Spielvorbereitung erhält jeder Spieler ein ERBSTÜCK \emph{ZIEGE} und dafür ein \emph{KUPFER} weniger.
		
	\medskip
		
	Wenn du die \emph{FEE} nicht entsorgst, erhältst du die \emph{Gabe} gar nicht. Wenn du die abgelegte \emph{Gabe} laut Anweisung bis zu deiner Aufräumphase behalten sollst, lege sie vor dir ab, merke dir, dass du sie zweimal empfängst und lege sie in deiner Aufräumphase ab.}
\end{tikzpicture}
\hspace{-0.6cm}
\begin{tikzpicture}
	\card
	\cardstrip
	\cardbanner{banner/blue.png}
	\cardicon{icons/coin.png}
	\cardprice{2}
	\cardtitle{\footnotesize{Getreuer Hund}}
	\cardcontent{Die Reaktion kann entweder im eigenen Zug (wenn die Karte zu einem anderen Zeitpunkt außer der Aufräumphase abgelegt wird) oder während des Zugs eines Mitspielers, wenn z.B. durch einen Angriff die Karte abgelegt werden muss, zum Tragen kommen. Wenn du die Reaktion nutzen möchtest, lege diesen \emph{GETREUEN HUND} zur Seite, anstatt ihn abzulegen und nimm ihn am Ende des Zuges wieder auf die Hand (wenn es dein eigener Zug war, nach dem Ziehen der neuen Kartenhand).
		
	\medskip

	Um die Reaktion des \emph{GETREUEN HUNDES} zu nutzen, muss er sich nicht zwingend auf der Hand befinden. Muss er zum Beispiel auf Grund der \emph{NACHTWACHE} direkt aus dem Nachziehstapel abgelegt werden, darfst du die Reaktion nutzen. Musst du den \emph{GETREUEN HUND} auf den Ablagestapel legen, ohne ihn im spieltechnischen Sinn abzulegen (z.B. nach dem Kauf oder durch \emph{LUMPENSAMMLER} aus \emph{Dark Ages}), passiert nichts. Du kannst den \emph{GETREUEN HUND} aber nur außerhalb der Aufräumphase ablegen (bzw. zur Seite legen), wenn du durch eine Anweisung dazu aufgefordert wirst.}
\end{tikzpicture}
\hspace{-0.6cm}
\begin{tikzpicture}
	\card
	\cardstrip
	\cardbanner{banner/black.png}
	\cardicon{icons/coin.png}
	\cardprice{2}
	\cardtitle{Kloster}
	\cardcontent{Diese Karte ist eine Nachtkarte und kann nur in der Nachtphase ausgespielt werden. Wenn du zum Beispiel bis zu dem Zeitpunkt, an dem du das \emph{KLOSTER} ausspielst, drei Karten genommen hast, darfst du 0 bis 3 Karten entsorgen – du darfst Handkarten und/oder \emph{KUPFER}, die sich gerade im Spiel befinden, entsorgen, in jeder beliebigen Kombination. Eingetauschte Karten (z.B. eine \emph{VAMPIRIN}, die für eine \emph{FLEDERMAUS} eingetauscht wurde) zählen nicht als \enquote{genommen}.}
\end{tikzpicture}
\hspace{-0.6cm}
\begin{tikzpicture}
	\card
	\cardstrip
	\cardbanner{banner/orangeblack.png}
	\cardicon{icons/coin.png}
	\cardprice{2}
	\cardtitle{Wächterin}
	\cardcontent{Diese Karte ist eine Nachtkarte und kann nur in der Nachtphase ausgespielt werden. Wenn du diese Karte nimmst, nimm sie direkt auf die Hand, anstatt sie auf deinen Ablagestapel zu legen. Da die Nachtphase nach der Kaufphase kommt, kannst du diese Karte in dem Zug ausspielen, in der du sie gekauft oder auf andere Weise vorher im Zug genommen hast.
		
	\medskip
		
	Solange sich diese Karte im Spiel befindet, bist du von allen ausgespielten Angriffskarten deiner Mitspieler nicht betroffen (auch nicht, wenn du das möchtest). Lege die Karte in der Aufräumphase deines nächsten Zuges ab.}
\end{tikzpicture}
\hspace{-0.6cm}
\begin{tikzpicture}
	\card
	\cardstrip
	\cardbanner{banner/orange.png}
	\cardicon{icons/coin.png}
	\cardprice{3}
	\cardtitle{\footnotesize{Geheime Höhle}}
	\cardcontent{In der Spielvorbereitung erhält jeder Spieler ein ERBSTÜCK \emph{WUNDERLAMPE} und dafür ein \emph{KUPFER} weniger.
		
	\medskip
		
	Wenn du nicht genau 3 Karten ablegst, wird die \emph{GEHEIME HÖHLE} in der Aufräumphase abgelegt. Wenn du genau 3 Karten ablegst, bleibt die \emph{GEHEIME HÖHLE} bis zum Ende des nächsten Zuges im Spiel und du erhältst zu Beginn des nächsten Zuges +\coin[3] . Du kannst wählen, 3 Karten abzulegen, auch wenn du weniger als 3 Karten auf der Hand hast und alle deine Handkarten ablegen – den Bonus erhältst du jedoch nicht und du legst die \emph{GEHEIME HÖHLE} am Ende des Zuges ab. Hast du mehr als 3 Karten auf der Hand, musst du entweder genau 3 Karten oder gar keine ablegen.}
\end{tikzpicture}
\hspace{-0.6cm}
\begin{tikzpicture}
	\card
	\cardstrip
	\cardbanner{banner/orangeblack.png}
	\cardicon{icons/coin.png}
	\cardprice{3}
	\cardtitle{Geisterstadt}
	\cardcontent{Diese Karte ist eine Nachtkarte und kann nur in der Nachtphase ausgespielt werden. Wenn du diese Karte nimmst, nimm sie direkt auf die Hand, anstatt sie auf den Ablagestapel zu legen. Da die Nachtphase nach der Kaufphase kommt, kannst du diese Karte in dem Zug ausspielen, in der du sie gekauft oder auf andere Weise vorher im Zug genommen hast.}
\end{tikzpicture}
\hspace{-0.6cm}
\begin{tikzpicture}
	\card
	\cardstrip
	\cardbanner{banner/white.png}
	\cardicon{icons/coin.png}
	\cardprice{3}
	\cardtitle{Kobold}
	\cardcontent{Im Spiel befinden sich der ausgespielte \emph{KOBOLD} selbst, andere in diesem Zug ausgespielte Karten, Dauerkarten aus vorherigen Zügen sowie aufgerufene Karten (aus \emph{Abenteuer}). Nicht im Spiel befinden sich bereits entsorgte sowie zur Seite gelegte Karten.}
\end{tikzpicture}
\hspace{-0.6cm}
\begin{tikzpicture}
	\card
	\cardstrip
	\cardbanner{banner/black.png}
	\cardicon{icons/coin.png}
	\cardprice{3}
	\cardtitle{Nachtwache}
	\cardcontent{Diese Karte ist eine Nachtkarte und kann nur in der Nachtphase ausgespielt werden. Wenn du diese Karte nimmst, nimm sie direkt auf die Hand, anstatt sie auf den Ablagestapel zu legen. Da die Nachtphase nach der Kaufphase kommt, kannst du diese Karte in dem Zug ausspielen, in der du sie gekauft oder auf andere Weise vorher im Zug genommen hast.}
\end{tikzpicture}
\hspace{-0.6cm}
\begin{tikzpicture}
	\card
	\cardstrip
	\cardbanner{banner/white.png}
	\cardicon{icons/coin.png}
	\cardprice{3}
	\cardtitle{Narr}
	\cardcontent{In der Spielvorbereitung erhält jeder Spieler ein ERBSTÜCK \emph{GLÜCKSTALER} und dafür ein \emph{KUPFER} weniger.
		
	\medskip
		
	Wenn du bereits den \emph{Zustand} \emph{IM WALD VERIRRT} vor dir liegen hast, passiert nichts. Wenn du \emph{IM WALD VERIRRT} nicht hast, erhalte es (wenn es gerade bei einem Spieler liegt, gibt er es dir), lege es vor dir ab und decke dann die obersten 3 \emph{Gaben} des \emph{Gaben}-Stapels auf. Empfange die \emph{Gaben} in einer von dir festgelegten Reihenfolge (diese musst du nicht zu Beginn festlegen – du kannst eine \emph{Gabe} empfangen und dann die nächste wählen usw.) und lege sie direkt nach Empfang ab bzw. bewahre sie bis zu deiner Aufräumphase auf, wenn eine \emph{Gabe} dies erfordert. Der \emph{Zustand} bleibt vor dir liegen, bis ein anderer Spieler ihn mit Hilfe des \emph{NARREN} erhält.}
\end{tikzpicture}
\hspace{-0.6cm}
\begin{tikzpicture}
	\card
	\cardstrip
	\cardbanner{banner/black.png}
	\cardicon{icons/coin.png}
	\cardprice{3}
	\cardtitle{Wechselbalg}
	\cardcontent{Immer wenn ihr mit der Königreichkarte \emph{WECHSELBALG} spielt und sich noch \emph{WECHSELBALG}-Karten im Vorrat befinden, darfst du, wenn du eine Karte nimmst, die in diesem Moment mindestens \coin[3] kostet, die genommene Karte in einen \emph{WECHSELBALG} eintauschen. Lege die genommene Karte zurück auf den entsprechenden Stapel (Anweisungen, die beim Nehmen der Karte eintreten, treten noch ein), nimm einen \emph{WECHSELBALG} und lege ihn auf den Ablagestapel. Karten, die gar kein \coin oder weniger als \coin[3] sowie z.B. \potion (aus \emph{Alchemisten}) oder \hex (aus \emph{Empires}) kosten, dürfen nicht getauscht werden, da sie niemals mehr als \coin[3] kosten, egal wie hoch die zusätzlichen Kosten in Form von \potion oder \hex sind. So darf z. B. eine \emph{VERWANDLUNG} (aus \emph{Alchemisten}) nicht getauscht werden, da sie nur \potion aber kein \coin kostet. Der \emph{ALCHEMIST} (aus \emph{Alchemisten}) hingegen darf eingetauscht werden, da dieser \coin[3] \potion kostet.
		
	\medskip
		
	Diese Karte ist eine Nachtkarte und kann nur in der Nachtphase ausgespielt werden. Wenn du das tust, entsorge diesen \emph{WECHSELBALG} und nimm dir für eine Karte, die du im Spiel hast (das können Aktions-, Geld- und/oder Nachtkarten sein), eine gleiche Karte vom entsprechenden Stapel.}
\end{tikzpicture}
\hspace{-0.6cm}
\begin{tikzpicture}
	\card
	\cardstrip
	\cardbanner{banner/white.png}
	\cardicon{icons/coin.png}
	\cardprice{4}
	\cardtitle{Attentäter}
	\cardcontent{Wenn du diese Karte ausspielst, decke die oberste \emph{Plage} auf. Jeder Mitspieler führt die Anweisung darauf (im Uhrzeigersinn, beginnend bei deinem linken Nachbarn) aus. Lege die \emph{Plage} danach auf den separaten \emph{Plagen}-Ablagestapel.}
\end{tikzpicture}
\hspace{-0.6cm}
\begin{tikzpicture}
	\card
	\cardstrip
	\cardbanner{banner/black.png}
	\cardicon{icons/coin.png}
	\cardprice{4}
	\cardtitle{Exorzistin}
	\cardcontent{Diese Karte ist eine Nachtkarte und kann nur in der Nachtphase ausgespielt werden. Wenn du das tust, entsorgst du eine beliebige Handkarte. Kostet die entsorgte Karte mehr als eine oder mehrere der ERSCHEINUNGEN (\emph{IRRLICHT} \coin[0] , \emph{TEUFELCHEN} \coin[2] , \emph{GEIST} \coin[4] ), nimmst du eine der billigeren ERSCHEINUNGEN und legst sie auf deinen Ablagestapel. Du darfst auch eine Karte entsorgen, die nicht weniger als eine der ERSCHEINUNGEN kostet (z.B. einen \emph{FLUCH} oder ein \emph{KUPFER}), nimmst dir dafür aber keine ERSCHEINUNG.}
\end{tikzpicture}
\hspace{-0.6cm}
\begin{tikzpicture}
	\card
	\cardstrip
	\cardbanner{banner/green.png}
	\cardicon{icons/coin.png}
	\cardprice{4}
	\cardtitle{Friedhof}
	\cardcontent{In der Spielvorbereitung erhält jeder Spieler ein ERBSTÜCK \emph{ZAUBERSPIEGEL} und dafür ein \emph{KUPFER} weniger.
		
	\medskip
		
	Wenn du einen \emph{FRIEDHOF} nimmst, entsorge 0 bis 4 Karten aus deiner Hand.}
\end{tikzpicture}
\hspace{-0.6cm}
\begin{tikzpicture}
	\card
	\cardstrip
	\cardbanner{banner/white.png}
	\cardicon{icons/coin.png}
	\cardprice{4}
	\cardtitle{Konklave}
	\cardcontent{Du darfst eine Aktionskarte aus deiner Hand ausspielen, von der du gerade keine gleiche im Spiel hast. Du kannst eine Karte ausspielen, die du in diesem Zug gespielt, aber bereits entsorgt hast. Dauerkarten aus vergangenen Zügen befinden sich im Spiel und dürfen entsprechend nicht ausgespielt werden. Nicht ausgespielt werden darf eine weitere \emph{KONKLAVE}, da eine solche sich bereits im Spiel befindet. Wenn du eine regelgerechte Aktionskarte gespielt hast, erhältst du eine weitere Aktion – diese hat keine Limitationen, du kannst damit jede beliebige Aktionskarte aus deiner Hand ausspielen.}
\end{tikzpicture}
\hspace{-0.6cm}
\begin{tikzpicture}
	\card
	\cardstrip
	\cardbanner{banner/white.png}
	\cardicon{icons/coin.png}
	\cardprice{4}
	\cardtitle{\footnotesize{Minnesängerin}}
	\cardcontent{Um die \emph{Gabe} zu empfangen, decke die oberste \emph{Gabe} des \emph{Gaben}-Stapels auf, empfange die \emph{Gabe} und lege sie (außer die \emph{Gabe} sagt dir etwas anderes) auf den separaten \emph{Gaben}-Ablagestapel.}
\end{tikzpicture}
\hspace{-0.6cm}
\begin{tikzpicture}
	\card
	\cardstrip
	\cardbanner{banner/white.png}
	\cardicon{icons/coin.png}
	\cardprice{4}
	\cardtitle{Schäferin}
	\cardcontent{In der Spielvorbereitung erhält jeder Spieler ein ERBSTÜCK \emph{WEIDELAND} und dafür ein \emph{KUPFER} weniger.
		
	\medskip
		
	Wenn du zum Beispiel 3 Punktekarten (auch ggf. kombinierte) ablegst, ziehst du 6 Karten nach.}
\end{tikzpicture}
\hspace{-0.6cm}
\begin{tikzpicture}
	\card
	\cardstrip
	\cardbanner{banner/white.png}
	\cardicon{icons/coin.png}
	\cardprice{4}
	\cardtitle{Seeliges Dorf}
	\cardcontent{Decke die \emph{Gabe} auf, bevor du dich entscheidest, ob du sie sofort oder erst zu Beginn deines nächsten Zuges empfangen möchtest. Wenn du sie erst in deinem nächsten Zug empfangen möchtest, lege sie vor dir ab und lege sie nach Empfang (oder in der Aufräumphase nach Empfang, wenn die \emph{Gabe} dies anweist) auf den separaten \emph{Gaben}-Ablagestapel.}
\end{tikzpicture}
\hspace{-0.6cm}
\begin{tikzpicture}
	\card
	\cardstrip
	\cardbanner{banner/black.png}
	\cardicon{icons/coin.png}
	\cardprice{4}
	\cardtitle{\scriptsize{Teufelswerkstatt}}
	\cardcontent{Diese Karte ist eine Nachtkarte und kann nur in der Nachtphase ausgespielt werden. Es zählen alle Karten, die bisher in diesem Zug genommen wurden, auch in der Nachtphase bis zum Ausspielen dieser \emph{TEUFELSWERKSTATT}. Du kannst dich nicht entscheiden, einen anderen Bonus zu nehmen – wenn du 2 oder mehr Karten genommen hast, musst du ein \emph{TEUFELCHEN} nehmen. Du darfst nicht stattdessen ein \emph{GOLD} oder eine Karte, die bis zu \coin[4] kostet, nehmen. Eingetauschte Karten (z.B. \emph{VAMPIRIN} für eine \emph{FLEDERMAUS}) gelten nicht als \enquote{genommen}.}
\end{tikzpicture}
\hspace{-0.6cm}
\begin{tikzpicture}
	\card
	\cardstrip
	\cardbanner{banner/white.png}
	\cardicon{icons/coin.png}
	\cardprice{4}
	\cardtitle{\scriptsize{Totenbeschwörer}}
	\cardcontent{In der Spielvorbereitung werden die 3 ZOMBIES aufgedeckt auf den Müllstapel gelegt. Der \emph{TOTENBESCHWÖRER} kann damit mindestens einen der 3 ZOMBIES spielen, da diese von Spielbeginn an im Müll liegen. Im Verlauf des Spiels können weitere Aktionskarten, die im Müll landen, gespielt werden.
		
	\medskip
		
	Spiele eine Aktionskarte, die mit der Vorderseite nach oben im Müll liegt und keine Dauerkarte ist, lasse sie dort und drehe sie für diesen Zug mit der Bildseite nach unten – damit kann jede Karte des Müllstapels maximal einmal pro Zug gespielt werden. Am Ende des Zuges wird die Karte wieder umgedreht. Die so gespielte Aktionskarte befindet sich nicht \enquote{im Spiel} und verbleibt auf jeden Fall im Müllstapel, auch wenn sie durch die Anweisung eigentlich anderswo hingelegt werden würde.}
\end{tikzpicture}
\hspace{-0.6cm}
\begin{tikzpicture}
	\card
	\cardstrip
	\cardbanner{banner/white.png}
	\cardicon{icons/coin.png}
	\cardprice{5}
	\cardtitle{Folterknecht}
	\cardcontent{Du musst das \emph{TEUFELCHEN} nehmen, wenn du keine anderen Karten im Spiel hast. Wenn du andere Karten im Spiel hast, decke die oberste \emph{Plage} auf. Jeder Mitspieler führt die Anweisung darauf (im Uhrzeigersinn, beginnend bei deinem linken Nachbarn) aus. Lege die \emph{Plage} danach auf den separaten \emph{Plagen}-Ablagestapel.}
\end{tikzpicture}
\hspace{-0.6cm}
\begin{tikzpicture}
	\card
	\cardstrip
	\cardbanner{banner/gold.png}
	\cardicon{icons/coin.png}
	\cardprice{5}
	\cardtitle{Götze}
	\cardcontent{Wichtig ist, wie viele \emph{GÖTZEN} du in diesem Moment im Spiel hast, nicht wie viele du in diesem Zug gespielt hat (es gibt Karten, z.B. \emph{FALSCHGELD} aus \emph{Dark Ages}, durch die diese Zahl unterschiedlich sein kann).}
\end{tikzpicture}
\hspace{-0.6cm}
\begin{tikzpicture}
	\card
	\cardstrip
	\cardbanner{banner/white.png}
	\cardicon{icons/coin.png}
	\cardprice{5}
	\cardtitle{Heiliger Hain}
	\cardcontent{Du musst die \emph{Gabe} empfangen. Mit Ausnahme von \emph{GESCHENK DES FELDES} und \emph{GESCHENK DES WALDES}, die +\coin[1] geben, können alle Mitspieler wählen, ob sie die \emph{Gabe} ebenfalls empfangen wollen oder nicht. Bei \emph{GESCHENK DES FLUSSES} zieht jeder Mitspieler, der sich für die \emph{Gabe} entscheidet, am Ende \emph{deines} Zuges eine FELDES und GESCHENK DES WALDES, die + Karte.}
\end{tikzpicture}
\hspace{-0.6cm}
\begin{tikzpicture}
	\card
	\cardstrip
	\cardbanner{banner/orangeblack.png}
	\cardicon{icons/coin.png}
	\cardprice{5}
	\cardtitle{Krypta}
	\cardcontent{Diese Karte ist eine Nachtkarte und kann nur in der Nachtphase ausgespielt werden. Sie bleibt bis zur Aufräumphase des Zuges im Spiel, in dem die letzte zur Seite gelegte Geldkarte auf die Hand genommen wird. Lege die Geldkarten verdeckt unter die \emph{KRYPTA}. Du darfst dir die Karten jederzeit anschauen, deine Mitspieler aber nicht. Du darfst dich auch entscheiden, keine Karte zur Seite zu legen – dann legst du die \emph{KRYPTA} am Ende des Zuges ab.}
\end{tikzpicture}
\hspace{-0.6cm}
\begin{tikzpicture}
	\card
	\cardstrip
	\cardbanner{banner/white.png}
	\cardicon{icons/coin.png}
	\cardprice{5}
	\cardtitle{Puka}
	\cardcontent{In der Spielvorbereitung erhält jeder Spieler ein ERBSTÜCK \emph{VERFLUCHTES GOLD} und dafür ein \emph{KUPFER} weniger.}
\end{tikzpicture}
\hspace{-0.6cm}
\begin{tikzpicture}
	\card
	\cardstrip
	\cardbanner{banner/orangeblack.png}
	\cardicon{icons/coin.png}
	\cardprice{5}
	\cardtitle{Schuster}
	\cardcontent{Diese Karte ist eine Nachtkarte und kann nur in der Nachtphase ausgespielt werden. Außerdem ist sie eine Dauerkarte und wird erst in der Aufräumphase deines nächsten Zuges abgelegt.}
\end{tikzpicture}
\hspace{-0.6cm}
\begin{tikzpicture}
	\card
	\cardstrip
	\cardbanner{banner/orangeblack.png}
	\cardicon{icons/coin.png}
	\cardprice{5}
	\cardtitle{Sündenpfuhl}
	\cardcontent{Diese Karte ist eine Nachtkarte und kann nur in der Nachtphase ausgespielt werden. Wenn du diese Karte nimmst, nimm sie direkt auf die Hand, anstatt sie auf den Ablagestapel zu legen. Da die Nachtphase nach der Kaufphase kommt, kannst du diese Karte in dem Zug ausspielen, in der du sie gekauft oder auf andere Weise vorher im Zug genommen hast.}
\end{tikzpicture}
\hspace{-0.6cm}
\begin{tikzpicture}
	\card
	\cardstrip
	\cardbanner{banner/white.png}
	\cardicon{icons/coin.png}
	\cardprice{5}
	\cardtitle{\scriptsize{Tragischer Held}}
	\cardcontent{Wenn du nach dem Ziehen der + 3 Karten 8 oder mehr Karten auf der Hand hast und diesen \emph{TRAGISCHEN HELDEN} entsorgen musst, erhältst du trotzdem +1 Kauf. Kannst du den \emph{TRAGISCHEN HELDEN} nicht entsorgen (weil du ihn zum Beispiel mit Hilfe des \emph{THRONSAALS} (aus dem \emph{Basisspiel}) zweimal gespielt und bereits entsorgt hast), nimmst du trotzdem eine Geldkarte.}
\end{tikzpicture}
\hspace{-0.6cm}
\begin{tikzpicture}
	\card
	\cardstrip
	\cardbanner{banner/black.png}
	\cardicon{icons/coin.png}
	\cardprice{5}
	\cardtitle{Vampirin}
	\cardcontent{Diese Karte ist eine Nachtkarte und kann nur in der Nachtphase ausgespielt werden.
		
	\medskip
		
	Führe die drei Anweisungen in der vorgegebenen Reihenfolge aus. Decke zuerst die oberste \emph{Plage} auf. Jeder Mitspieler führt die Anweisung darauf (im Uhrzeigersinn, beginnend bei deinem linken Nachbarn) aus. Lege die \emph{Plage} danach auf den separaten \emph{Plagen}-Ablagestapel. Nimm dir dann eine Karte, die bis zu 5 kostet (außer einer VAMPIRIN) und tausche dann die \emph{VAMPIRIN} in eine \emph{FLEDERMAUS} ein.}
\end{tikzpicture}
\hspace{-0.6cm}
\begin{tikzpicture}
	\card
	\cardstrip
	\cardbanner{banner/white.png}
	\cardicon{icons/coin.png}
	\cardprice{5}
	\cardtitle{\scriptsize{Verfluchtes Dorf}}
	\cardcontent{Wenn du bereits 6 oder mehr Handkarten hast, ziehe keine Karten. Wenn du das \emph{VERFLUCHTE DORF} nimmst, empfängst du eine \emph{Plage}; da dies oft in deiner Kaufphase der Fall ist, haben einige der \emph{Plagen} keine Auswirkung auf dich.}
\end{tikzpicture}
\hspace{-0.6cm}
\begin{tikzpicture}
	\card
	\cardstrip
	\cardbanner{banner/whiteblack.png}
	\cardicon{icons/coin.png}
	\cardprice{5}
	\cardtitle{Werwolf}
	\cardcontent{Diese Karte ist sowohl eine Aktionsals auch eine Nachtkarte und kann entsprechend in der Aktions- oder der Nachtphase ausgespielt werden. Spielst du den \emph{WERWOLF} in der Aktionsphase, ziehst du 3 Karten; spielst du ihn in der Nachtphase, empfängt jeder Mitspieler die nächste \emph{Plage}. Decke die oberste \emph{Plage} auf. Jeder Mitspieler führt die Anweisung darauf (im Uhrzeigersinn, beginnend bei deinem linken Nachbarn) aus. Lege die \emph{Plage} danach auf den separaten \emph{Plagen}-Ablagestapel.}
\end{tikzpicture}
\hspace{-0.6cm}
\begin{tikzpicture}
	\card
	\cardstrip
	\cardbanner{banner/orangeblack.png}
	\cardicon{icons/coin.png}
	\cardprice{6}
	\cardtitle{Plünderer}
	\cardcontent{Diese Karte ist eine Nachtkarte und kann nur in der Nachtphase ausgespielt werden. Mitspieler können auf das Ausspielen dieser Angriffskarte mit einer entsprechenden Reaktionskarte reagieren. Hast du zum Beispiel 3 \emph{KUPFER}, 1 \emph{SILBER} und 1 \emph{PLÜNDERER} im Spiel, müssen alle Mitspieler mit 5 oder mehr Handkarten ein \emph{KUPFER}, ein \emph{SILBER} oder einen \emph{PLÜNDERER} (nach ihrer eigenen Wahl) ablegen oder ihre Kartenhand vorzeigen, wenn sie keine dieser Karten auf der Hand haben.}
\end{tikzpicture}
\hspace{-0.6cm}
\begin{tikzpicture}
	\card
	\cardstrip
	\cardbanner{banner/gold.png}
	\cardtitle{Gaben\qquad}
	\cardcontent{Alle 12 \emph{Gaben} sind jeweils nur 1 x im Spiel enthalten. Sie dürfen nur erhalten bzw. empfangen werden, wenn eine Königreichkarte mit dem Typ SEGEN dies anweist.
		
	\smallskip
		
	Die Anweisung einer \emph{Gabe} wird erst ausgeführt, sobald sie \emph{empfangen} wird. Nach dem Empfang der \emph{Gabe} wird diese auf einen separaten \emph{Gaben}-Ablagestapel gelegt. \emph{Gaben} werden niemals Bestandteil des Kartensatzes eines Spielers.
	
	\medskip
	
	\emph{Geschenk des Mondes:} Wenn dein Ablagestapel leer ist, hat diese Gabe für dich keine Auswirkung.
	
	\medskip
	
	\emph{Geschenk des Flusses:} Ziehe die zusätzliche Karte erst, nachdem du deine Kartenhand für den nächsten Zug gezogen hast.
	
	\medskip
	
	\emph{Geschenk des Himmels:} Wenn du weniger als drei Handkarten hast, darfst du alle deine Handkarten ablegen, erhältst dafür aber kein \emph{GOLD}. Hast du drei oder mehr Karten auf der Hand, musst du genau 3 Karten oder gar keine ablegen. Nur wenn du genau 3 Karten ablegst, nimmst du ein \emph{GOLD}.}
\end{tikzpicture}
\hspace{-0.6cm}
\begin{tikzpicture}
	\card
	\cardstrip
	\cardbanner{banner/purple.png}
	\cardtitle{Plagen (1/4)\qquad}
	\cardcontent{Alle 12 \emph{Plagen} sind jeweils nur 1x im Spiel enthalten. Sie werden nur erhalten bzw. empfangen, wenn eine Königreichkarte mit dem Typ UNHEIL dies anweist.
		
	\smallskip
		
	Die Anweisung einer \emph{Plage} wird erst ausgeführt, sobald sie \emph{empfangen} wird. Nach dem Empfang der \emph{Plage} wird diese auf einen separaten \emph{Plagen}-Ablagestapel gelegt. Sie werden niemals Bestandteil des Kartensatzes eines Spielers.
	
	\medskip
	
	\emph{Elend:} Wenn dich diese \emph{Plage} zum dritten (oder vierten, fünften...) Mal in einem Spiel trifft, passiert nichts. Du bleibst bei \emph{DOPPELT ELENDIG}. Pro Spieler ist eine \emph{Zustandskarte} \emph{ELENDIG/DOPPELT ELENDIG} im Spiel enthalten. Jeder Spieler erhält maximal einen solchen \emph{Zustand} – sie werden nicht zwischen den Spielern ausgetauscht.
	
	\begin{itemize}
		\item[\rightarrow] \emph{Zustand} \emph{ELENDIG:} Solange dieser Zustand vor dir liegt, ist dieser – 2 \victorypoint wert.\\
		\item[\rightarrow] \emph{Zustand} \emph{DOPPELT ELENDIG:} Solange dieser Zustand vor dir liegt, ist dieser – 4 \victorypoint wert.
	\end{itemize}
	}
\end{tikzpicture}
\hspace{-0.6cm}
\begin{tikzpicture}
	\card
	\cardstrip
	\cardbanner{banner/purple.png}
	\cardtitle{Plagen (2/4)\qquad}
	\cardcontent{\emph{Furcht:} Du musst eine Aktions- oder Geldkarte deiner Wahl ablegen, wenn du mindestens eine auf der Hand hast. Du zeigst deine Kartenhand nur vor, wenn du keines von beiden auf der Hand hast.
		
	\medskip
		
	\emph{Heuschrecken:} Wenn du etwas anderes als ein \emph{ANWESEN} oder ein \emph{KUPFER} entsorgst, musst du eine billigere Karte desselben Typs (\emph{GELD}, \emph{AKTION}, \emph{ANGRIFF}, \emph{NACHT}, \emph{PUNKTE} etc.) nehmen, sofern es eine gibt. Bei Karten mit mehreren Typen muss es mindestens eine Übereinstimmung im Typ geben.
	
	\smallskip
	
	Wenn du z.B. einen \emph{SCHUSTER} (Typen: \emph{NACHT} \& \emph{DAUER}, Wert: \coin[5] ) entsorgst, kannst du dir eine \emph{GEHEIME HÖHLE} (Typen: \emph{AKTION} \& \emph{DAUER}; Wert: \coin[3] ) nehmen.
	
	\medskip
	
	\emph{Krieg:} Findest du in deinem Nachziehstapel keine Karte, die \coin[3] oder \coin[4] kostet (auch nach dem Mischen des Ablagestapels), entsorge keine Karte und lege alle aufgedeckten Karten auf den Ablagestapel.}
\end{tikzpicture}
\hspace{-0.6cm}
\begin{tikzpicture}
	\card
	\cardstrip
	\cardbanner{banner/purple.png}
	\cardtitle{Plagen (3/4)\qquad}
	\cardcontent{\emph{Neid:} Hast du bereits den \emph{Zustand} \emph{GETÄUSCHT}/\emph{NEIDISCH} vor dir liegen (egal mit welcher Seite nach oben), passiert nichts. Hast du ihn noch nicht vor dir liegen, nimm ihn und lege ihn mit der Seite \emph{NEIDISCH} nach oben vor dir ab. Lege den \emph{Zustand} zu Beginn deiner nächsten Kaufphase zurück und führe den \emph{Zustand} \emph{NEIDISCH} aus. Pro Spieler ist eine Zustandskarte \emph{GETÄUSCHT}/\emph{NEIDISCH} im Spiel enthalten. Jeder Spieler erhält maximal einen solchen \emph{Zustand} – sie werden nicht zwischen den Spielern ausgetauscht.
		
	\begin{itemize}
		\item[\rightarrow] \emph{Zustand Neidisch:} Lege den \emph{Zustand} zu Beginn deiner nächsten Kaufphase zurück. \emph{GOLD} und \emph{SILBER} sind ab diesem Moment bis zum Ende deines Zuges \coin[1] wert. Andere Geldkarten sind nicht betroffen.
	\end{itemize}
	
	\medskip
	
	\emph{Schlechtes Omen:} Normalerweise führt diese \emph{Plage} dazu, dass dein Nachziehstapel nur aus 2 \emph{KUPFER} besteht und der Rest auf dem Ablagestapel liegt. Hast du nur 1 \emph{KUPFER}, liegt diese als einzige Karte auf dem Nachziehstapel. Hast du kein \emph{KUPFER}, ist dein Nachziehstapel leer – zeige deinen Ablagestapel vor, um dies nachzuweisen.}
\end{tikzpicture}
\hspace{-0.6cm}
\begin{tikzpicture}
	\card
	\cardstrip
	\cardbanner{banner/purple.png}
	\cardtitle{Plagen (4/4)\qquad}
	\cardcontent{\emph{Täuschung:} Hast du bereits den \emph{Zustand} \emph{GETÄUSCHT}/\emph{NEIDISCH} vor dir liegen (egal mit welcher Seite nach oben), passiert nichts. Hast du ihn noch nicht vor dir liegen, nimm ihn und lege ihn mit der Seite \emph{GETÄUSCHT} nach oben vor dir ab. Lege den \emph{Zustand} zu Beginn deiner nächsten Kaufphase zurück und führe den \emph{Zustand} \emph{GETÄUSCHT} aus. Pro Spieler ist eine Zustandskarte \emph{GETÄUSCHT}/\emph{NEIDISCH} im Spiel enthalten. Jeder Spieler erhält maximal einen solchen \emph{Zustand} – sie werden nicht zwischen den Spielern ausgetauscht.
		
	\begin{itemize}
		\item[\rightarrow] \emph{Zustand Getäuscht:} Lege den \emph{Zustand} zu Beginn deiner nächsten Kaufphase zurück. Ab diesem Moment bis zum Ende deines Zuges darfst du keine Aktionskarten kaufen.
	\end{itemize}
	}
\end{tikzpicture}
\hspace{-0.6cm}
\begin{tikzpicture}
	\card
	\cardstrip
	\cardbanner{banner/white.png}
	\cardtitle{Zustände\qquad}
	\cardcontent{\tiny {\emph{Elendig:} Nimm den \emph{Zustand} \emph{ELENDIG}, wenn du das erste Mal die \emph{Plage} \emph{ELEND} empfängst. Solange dieser Zustand vor dir liegt, ist dieser – 2 \victorypoint wert.
		
	\medskip
		
	\emph{Doppelt Elendig:} Drehe \emph{ELENDIG} auf \emph{DOPPELT ELENDIG} um, wenn du das zweite Mal die \emph{Plage} \emph{ELEND} empfängst. Solange dieser Zustand vor dir liegt, ist dieser – 4 \victorypoint wert.
	
	\medskip
	
	\emph{Getäuscht:} Nimm den \emph{Zustand} \emph{GETÄUSCHT}, wenn dieser (oder \emph{NEIDISCH}) nicht bereits vor dir liegt, wenn du die \emph{Plage} \emph{TÄUSCHUNG} empfängst. Lege den \emph{Zustand} zu Beginn deiner nächsten Kaufphase zurück. Ab diesem Moment bis zum Ende deines Zuges darfst du keine Aktionskarten kaufen.
	
	\medskip
	
	\emph{Im Wald verirrt:} Nimm den \emph{Zustand} \emph{IM WALD VERIRRT}, wenn dieser nicht bereits vor dir liegt, wenn du die Königreichkarte \emph{NARR} ausspielst. Der \emph{Zustand} bleibt vor dir liegen, bis ein anderer Spieler ihn mit Hilfe des \emph{NARREN} erhält. Solange der \emph{Zustand} vor dir liegt, kannst du dessen Anweisung zu Beginn jedes Zuges anwenden (optional).
	
	\medskip
	
	\emph{Neidisch:} Nimm den \emph{Zustand} \emph{NEIDISCH}, wenn dieser (oder \emph{GETÄUSCHT}) nicht bereits vor dir liegt, wenn du die \emph{Plage} \emph{NEID} empfängst. Lege den \emph{Zustand} zu Beginn deiner nächsten Kaufphase zurück. \emph{GOLD} und \emph{SILBER} sind ab diesem Moment bis zum Ende deines Zuges \coin[1] wert. Andere Geldkarten sind nicht betroffen.\\}
	}
\end{tikzpicture}
\hspace{-0.6cm}
\begin{tikzpicture}
	\card
	\cardstrip
	\cardbanner{banner/white.png}
	\cardicon{icons/coin.png}
	\cardprice{6\textsuperscript{*}}
	\cardtitle{Wunsch}
	\cardcontent{Diese Karte wird nur verwendet, wenn die Königreichkarte \emph{KOBOLD} und/oder \emph{GEHEIME HÖHLE} (\rightarrow ERBSTÜCK \emph{WUNDERLAMPE}) verwendet wird.
		
	\medskip
		
	Diese Karte kann nicht gekauft werden – sie kann nur durch die Anweisung auf der Königreichkarte \emph{KOBOLD} oder dem ERBSTÜCK \emph{WUNDERLAMPE} genommen werden.
	
	\medskip
	
	Du darfst nur dann eine Karte nehmen, wenn du den \emph{WUNSCH} auf seinen Stapel zurückgelegt hast. Tust du das nicht, z.B. weil du ihn mit Hilfe des \emph{THRONSAALS} (aus dem \emph{Basisspiel}) doppelt ausgespielt hast und ihn beim zweiten Mal nicht mehr zurücklegen kannst, darfst du keine Karte nehmen. Nimmst du eine Karte, die normalerweise woanders hingelegt werden würde (z.B. \emph{NOMADENCAMP} aus \emph{Hinterland}), nimm sie trotzdem auf die Hand.}
\end{tikzpicture}
\hspace{-0.6cm}
\begin{tikzpicture}
	\card
	\cardstrip
	\cardbanner{banner/black.png}
	\cardicon{icons/coin.png}
	\cardprice{2\textsuperscript{*}}
	\cardtitle{Fledermaus}
	\cardcontent{Diese Karte kann nicht gekauft werden – sie kann nur durch die Anweisung auf der Königreichkarte \emph{VAMPIRIN} genommen werden.
		
	\medskip
		
	Diese Karte ist eine Nachtkarte und kann nur in der Nachtphase ausgespielt werden. Wenn du diese \emph{FLEDERMAUS} in eine \emph{VAMPIRIN} eintauschst, lege die \emph{FLEDERMAUS} zurück auf ihren Stapel. Ist keine \emph{VAMPIRIN} mehr im Vorrat vorhanden, kannst du die \emph{FLEDERMAUS} nicht eintauschen, du kannst sie aber trotzdem ausspielen und Karten entsorgen.}
\end{tikzpicture}
\hspace{-0.6cm}
\begin{tikzpicture}
	\card
	\cardstrip
	\cardbanner{banner/gold.png}
	\cardtitle{Erbstücke (1/2)\qquad}
	\cardcontent{ERBSTÜCKE werden ausschließlich in der Spielvorbereitung verteilt – und nur, wenn Königreichkarten im Spiel verwendet werden, die ein entsprechendes ERBSTÜCK erfordern. Alle nicht benötigten ERBSTÜCKE kommen in diesem Spiel nicht zum Einsatz. ERBSTÜCKE sind Geldkarten (\emph{WEIDELAND} ist zusätzlich eine Punktekarte) mit einer zusätzlichen Anweisung und ersetzen in der Spielvorbereitung jeweils 1 \emph{KUPFER}:
		
	\medskip
		
	\emph{Wunderlampe:} Dieses ERBSTÜCK wird nur verwendet, wenn die Königreichkarte \emph{GEHEIME HÖHLE} verwendet wird.\\
	Die ausgespielte \emph{WUNDERLAMPE} selbst zählt als eine der 6 Karten, wenn du von ihr genau 1 Karte im Spiel hast. Auch wenn du die \emph{WUNDERLAMPE} entsorgst, erhältst du das \coin[1] für diesen Zug.
	
	\medskip
	
	\emph{Zauberspiegel:} Dieses ERBSTÜCK wird nur verwendet, wenn die Königreichkarte \emph{FRIEDHOF} verwendet wird.\\
	Du darfst diesen \emph{ZAUBERSPIEGEL} nur entsorgen, wenn du dies durch die Anweisung einer anderen Karte tun darfst – der \emph{ZAUBERSPIEGEL} selbst gibt dir dazu nicht das Recht. Solltest du aber eine Möglichkeit haben, diesen \emph{ZAUBERSPIEGEL} zu entsorgen, darfst du eine Aktionskarte aus der Hand ablegen und einen \emph{GEIST} von seinem Stapel nehmen.}
\end{tikzpicture}
\hspace{-0.6cm}
\begin{tikzpicture}
	\card
	\cardstrip
	\cardbanner{banner/gold.png}
	\cardtitle{Erbstücke (2/2)\qquad}
	\cardcontent{\emph{Beutel:} Dieses ERBSTÜCK wird nur verwendet, wenn die Königreichkarte \emph{FÄHRTENSUCHER} verwendet wird.
		
	\medskip
		
	\emph{Weideland:} Dieses ERBSTÜCK wird nur verwendet, wenn die Königreichkarte \emph{SCHÄFERIN} verwendet wird.\\
	Als Geldkarte ausgespielt, ist \emph{WEIDELAND} \coin[1] wert. Zusätzlich bringt sie pro \emph{ANWESEN} im Kartensatz eines Spielers 1 \victorypoint .
	
	\medskip
	
	\emph{Ziege:} Dieses ERBSTÜCK wird nur verwendet, wenn die Königreichkarte \emph{FEE} verwendet wird. Das Entsorgen einer Handkarte ist optional.
	
	\medskip
	
	\emph{Glückstaler:} Dieses ERBSTÜCK wird nur verwendet, wenn die Königreichkarte \emph{NARR} verwendet wird.
	
	\medskip
	
	\emph{Verfluchtes Gold:} Dieses ERBSTÜCK wird nur verwendet, wenn die Königreichkarte \emph{PUKA} verwendet wird.}
\end{tikzpicture}
\hspace{-0.6cm}
\begin{tikzpicture}
	\card
	\cardstrip
	\cardbanner{banner/white.png}
	\cardicon{icons/coin.png}
	\cardprice{3}
	\cardtitle{Zombies}
	\cardcontent{ZOMBIES werden nur im Spiel verwendet und in der Spielvorbereitung in den Müll gelegt, wenn die Königreichkarte \emph{TOTENBESCHWÖRER} im Spiel verwendet wird.
		
	\medskip
		
	\emph{Zombie-Lehrling:} Nur, wenn du eine Karte aus der Hand entsorgst, ziehst du 3 Karten nach und erhältst + 1 Aktion.
	
	\medskip
	
	\emph{Zombie-Maurer:} Du musst, auch wenn du eine Karte entsorgt hast, keine Karte nehmen. Du kannst auch nur eine Karte entsorgen und nichts weiter tun. Du kannst, wenn du eine Karte nimmst, auch eine Karte mit gleichen Kosten oder eine billigere nehmen, auch eine gleiche wie die entsorgte Karte.
	
	\medskip
	
	\emph{Zombie-Spion:} Ziehe eine Karte, bevor du dir die oberste Karte des Nachziehstapels ansiehst.}
\end{tikzpicture}
\hspace{-0.6cm}
\begin{tikzpicture}
	\card
	\cardstrip
	\cardbanner{banner/white.png}
	\cardicon{icons/coin.png}
	\cardprice{0\textsuperscript{*}}
	\cardtitle{Irrlicht}
	\cardcontent{Diese ERSCHEINUNG wird nur verwendet, wenn die Königreichkarte \emph{EXORZISTIN} und/oder eine beliebige Königreichkarte mit dem Typ SEGEN (\rightarrow \emph{GESCHENK DES SUMPFES}) verwendet wird.
		
	\medskip
		
	Diese Karte kann nicht gekauft werden – sie kann nur durch die Anweisung auf der Königreichkarte \emph{EXORZISTIN} oder der \emph{Gabe} \emph{GESCHENK DES SUMPFES} genommen werden.
	
	\medskip
	
	Kostet die aufgedeckte Karte nicht \coin[2] oder weniger, lege sie auf deinen Nachziehstapel zurück.}
\end{tikzpicture}
\hspace{-0.6cm}
\begin{tikzpicture}
	\card
	\cardstrip
	\cardbanner{banner/white.png}
	\cardicon{icons/coin.png}
	\cardprice{2\textsuperscript{*}}
	\cardtitle{Teufelchen}
	\cardcontent{Diese ERSCHEINUNG wird nur verwendet, wenn mindestens eine der Königreichkarten \emph{EXORZISTIN}, \emph{TEUFELSWERKSTATT} und/oder \emph{FOLTERKNECHT} verwendet wird.
		
	\medskip
		
	Diese Karte kann nicht gekauft werden – sie kann nur durch eine Anweisung auf den Königreichkarten \emph{EXORZISTIN}, \emph{TEUFELSWERKSTATT} oder \emph{FOLTERKNECHT} genommen werden.
	
	\medskip
	
	Du darfst eine Aktionskarte aus deiner Hand ausspielen, von der du gerade keine gleiche im Spiel hast. Du kannst eine Karte ausspielen, die du in diesem Zug gespielt, aber bereits entsorgt hast. Dauerkarten aus vergangenen Zügen befinden sich im Spiel und dürfen entsprechend nicht ausgespielt werden. Nicht ausgespielt werden darf ein weiteres \emph{TEUFELCHEN}, da ein solches sich bereits im Spiel befindet.}
\end{tikzpicture}
\hspace{-0.6cm}
\begin{tikzpicture}
	\card
	\cardstrip
	\cardbanner{banner/orangeblack.png}
	\cardicon{icons/coin.png}
	\cardprice{4\textsuperscript{*}}
	\cardtitle{Geist}
	\cardcontent{\tiny{Diese ERSCHEINUNG wird nur verwendet, wenn eine der Königreichkarten \emph{EXORZISTIN} und/oder \emph{FRIEDHOF} (\rightarrow ERBSTÜCK \emph{ZAUBERSPIEGEL}) verwendet wird.
			
	\medskip
			
	Diese Karte kann nicht gekauft werden – sie kann nur durch die Anweisung auf der Königreichkarte \emph{EXORZISTIN} oder dem ERBSTÜCK \emph{ZAUBERSPIEGEL} genommen werden.
	
	\medskip
	
	Diese Karte ist eine Nachtkarte und kann nur in der Nachtphase ausgespielt werden. Wenn dein Nachziehstapel aufgebraucht ist, bevor du eine Aktionskarte aufdeckst, lege die bereits aufgedeckten Karten zur Seite, mische deinen Ablagestapel und lege ihn als neuen Nachziehstapel bereit. Wenn du trotzdem keine Aktionskarte findest, lege alle aufgedeckten Karten ab und es passiert nichts weiter. Lege in diesem Fall den \emph{GEIST} am Ende des Zuges ab (normalerweise erst in der Aufräumphase des nächsten Zuges). Wenn du eine Aktionskarte findest, musst du sie zusammen mit diesem \emph{GEIST} zur Seite legen und zu Beginn deines nächsten Zuges zweimal ausspielen – dies ist nicht optional. Ist die zur Seite gelegte Aktionskarte zusätzlich eine Dauerkarte, bleibt auch der \emph{GEIST} solange im Spiel, bis die Dauerkarte abgelegt wird.
	
	\medskip
	
	Solltest du zu Beginn deines Zuges mehrere Dauerkarten mit \enquote{Zu Beginn des Zuges}-Anweisungen im Spiel haben, entscheidest du, in welcher Reihenfolge du sie abhandelst. Sobald du die Aktionskarte abwickelst, musst du sie hintereinander zweimal ausspielen – du darfst keine andere Anweisung dazwischen abhandeln. Spiele die Aktionskarte aus, führe ihre Anweisungen aus und spiele sie ein zweites Mal aus. Dies verbraucht keine freien oder zusätzlich durch + x Aktionen erhaltenen Aktionen. Sollte sich die Aktionskarte selbst entsorgen, führe ihre Anweisungen trotzdem ein zweites Mal aus, auch wenn sie nicht mehr im Spiel ist.}\\}
\end{tikzpicture}
\hspace{-0.6cm}
\begin{tikzpicture}
	\card
	\cardstrip
	\cardbanner{banner/white.png}
	\cardtitle{\scriptsize{Spielvorbereitung (1/3)}\qquad}
	\cardcontent{Verwendet ihr eine oder mehrere Königreichkarten mit einem gelben ERBSTÜCK-Banner, erhält jeder Spieler ein entsprechendes ERBSTÜCK und dafür ein \emph{KUPFER} weniger. Spielt ihr zum Beispiel mit den Königreichkarten \emph{FEE} und \emph{NARR}, erhält jeder Spieler zu Beginn des Spiels 3 \emph{ANWESEN}, 5 \emph{KUPFER}, 1 \emph{GLÜCKSTALER} und 1 \emph{ZIEGE}.
		
	\medskip
		
	Verwendet ihr eine oder mehrere Königreichkarten mit dem Typ SEGEN, mischt ihr alle \emph{Gaben} und legt sie neben dem Vorrat bereit (\emph{Gaben} sind kein Teil des Vorrats). Legt außerdem den ERSCHEINUNGS-Stapel \emph{IRRLICHT} neben dem Vorrat bereit.
	
	\medskip
	
	Verwendet ihr eine oder mehrere Königreichkarten mit dem Typ UNHEIL, mischt ihr alle \emph{Plagen} und legt sie neben dem Vorrat bereit (\emph{Plagen} sind kein Teil des Vorrats). Legt außerdem die \emph{Zustände} \emph{ELENDIG}/\emph{DOPPELT ELENDIG} sowie \emph{NEIDISCH}/\emph{GETÄUSCHT} neben dem Vorrat bereit.}
\end{tikzpicture}
\hspace{-0.6cm}
\begin{tikzpicture}
	\card
	\cardstrip
	\cardbanner{banner/white.png}
	\cardtitle{\scriptsize{Spielvorbereitung (2/3)}\qquad}
	\cardcontent{Verwendet ihr folgende Königreichkarten, beachtet bitte die entsprechende Vorbereitung:
		
	\medskip
		
	\underline{\emph{Gaben \& Plagen}}\\
	\emph{DRUIDIN:} Legt die 3 obersten \emph{Gaben} aufgedeckt zur Seite – direkt neben den \emph{DRUIDINNEN}-Stapel. Verwendet ihr einen der empfohlenen Königreichkartensätze mit der \emph{DRUIDIN}, legt die entsprechend vorgegebenen \emph{Gaben} aufgedeckt zur Seite.
	
	\medskip
	
	\underline{\emph{Zombies}}\\
	\emph{TOTENBESCHÖRER:} Legt die 3 ZOMBIES offen in den Müll.
	
	\medskip
	
	\underline{\emph{Zustände}}\\
	\emph{NARR:} Legt den \emph{Zustand} \emph{IM WALD VERIRRT} neben dem Vorrat bereit.}
\end{tikzpicture}
\hspace{-0.6cm}
\begin{tikzpicture}
	\card
	\cardstrip
	\cardbanner{banner/white.png}
	\cardtitle{\scriptsize{Spielvorbereitung (3/3)}\qquad}
	\cardcontent{\underline{\emph{Erscheinungen}}\\
	\emph{EXORZISTIN:} Legt alle 3 ERSCHEINUNGS-Stapel (\emph{GEIST}, \emph{TEUFELCHEN}, \emph{IRRLICHT}) neben dem Vorrat bereit.\\
	\emph{FRIEDHOF:} Legt den \emph{GEIST}-Stapel neben dem Vorrat bereit.\\
	\emph{TEUFELSWERKSTATT} und/oder \emph{FOLTERKNECHT}: Legt den \emph{TEUFELCHEN}-Stapel neben dem Vorrat bereit.
	
	\medskip
	
	\underline{\emph{Wünsche}}\\
	\emph{KOBOLD:} Legt den \emph{WUNSCH}-Stapel neben dem Vorrat bereit.\\
	\emph{GEHEIME HÖHLE:} Legt den \emph{WUNSCH}-Stapel neben dem Vorrat bereit.
	
	\medskip
	
	\underline{\emph{Fledermäuse}}\\
	\emph{VAMPIRIN:} Legt den \emph{FLEDERMAUS}-Stapel neben dem Vorrat bereit.}
\end{tikzpicture}
\hspace{-0.6cm}
\begin{tikzpicture}
	\card
	\cardstrip
	\cardbanner{banner/white.png}
	\cardtitle{\footnotesize{Neue Regeln (1/5)}\qquad}
	\cardcontent{\emph{Neue Spielphase \enquote{Nacht}:} In \emph{Nocturne} gibt es einen neuen Kartentyp – die NACHT-Karten. In Spielen, in denen mindestens 1 Königreichkarte dieses Typs verwendet wird, schließt sich unmittelbar an die Kaufphase (vor der Aufräumphase) die Nachtphase an – in dieser Phase dürfen ausschließlich Karten des Typs NACHT ausgespielt werden. Es darf eine beliebige Anzahl Nachtkarten ausgespielt werden.
		
	\medskip
		
	\emph{Erbstücke:} In \emph{Nocturne} gibt es Königreichkarten, die ein gelbes Banner tragen. Verwendet ihr im Spiel eine oder mehrere dieser Karten, ersetzt jeder Spieler in der Spielvorbereitung ein \emph{KUPFER} durch das oder die entsprechenden ERBSTÜCKE.}
\end{tikzpicture}
\hspace{-0.6cm}
\begin{tikzpicture}
	\card
	\cardstrip
	\cardbanner{banner/white.png}
	\cardtitle{\footnotesize{Neue Regeln (2/5)}\qquad}
	\cardcontent{\emph{Segen \& Gaben\textsuperscript{*}:} In \emph{Nocturne} gibt es Königreichkarten mit dem Typ SEGEN. Verwendet ihr im Spiel eine oder mehrere dieser Karten, werden alle \emph{Gaben} zu Beginn des Spiels gemischt und als verdeckter Stapel neben dem Vorrat bereitgelegt. Sie gehören nicht zum Vorrat.
		
	\medskip
		
	Die Königreichkarten mit dem Typ SEGEN enthalten Anweisungen, die Spielern in irgendeiner Art und Weise \emph{Gaben} einbringen. Die Anweisung \enquote{Empfange eine \emph{Gabe}} bedeutet, dass der Spieler die oberste Karte des \emph{Gaben}-Stapels aufdeckt und die Anweisung darauf befolgt. Empfangene \emph{Gaben} werden mit Ausnahme von \emph{GESCHENK DES FELDES}, \emph{GESCHENK DES WALDES} und \emph{GESCHENK DES FLUSSES} direkt auf einen separaten \emph{Gaben}-Ablagestapel gelegt. Die vorgenannten \emph{Gaben} werden nach Empfang vor dem Spieler bis zu dessen Aufräumphase abgelegt und dann erst auf den Ablagestapel gelegt. Ist der \emph{Gaben}-Stapel leer, wird der \emph{Gaben}-Ablagestapel gemischt und als neuer \emph{Gaben}-Stapel bereitgelegt.}
\end{tikzpicture}
\hspace{-0.6cm}
\begin{tikzpicture}
	\card
	\cardstrip
	\cardbanner{banner/white.png}
	\cardtitle{\footnotesize{Neue Regeln (3/5)}\qquad}
	\cardcontent{\emph{Unheil \& Plagen\textsuperscript{*}:} In \emph{Nocturne} gibt es Königreichkarten mit dem Typ UNHEIL. Verwendet ihr im Spiel eine oder mehrere dieser Karten, werden alle \emph{Plagen} zu Beginn des Spiels gemischt und als verdeckter Stapel neben dem Vorrat bereitgelegt. Sie gehören nicht zum Vorrat.
		
	\medskip
		
	Die Königreichkarten mit dem Typ UNHEIL enthalten Anweisungen, die Spielern in irgendeiner Art und Weise \emph{Plagen} bescheren. Die Anweisung \enquote{Empfange eine \emph{Plage}} bedeutet, dass der Spieler die oberste Karte des \emph{Plagen}-Stapels aufdeckt und die Anweisung darauf befolgt. Die Anweisung \enquote{Alle Mitspieler empfangen die nächste \emph{Plage}} bedeutet, dass die oberste \emph{Plage} aufgedeckt wird und alle Mitspieler die Anweisung derselben Karte (im Uhrzeigersinn) befolgen müssen. Empfangene \emph{Plagen} werden immer direkt auf einen separaten \emph{Plagen}-Ablagestapel gelegt. Sobald alle \emph{Plagen} empfangen wurden, wird der \emph{Plagen}-Ablagestapel gemischt und als neuer \emph{Plagen}-Stapel bereitgelegt.}
\end{tikzpicture}
\hspace{-0.6cm}
\begin{tikzpicture}
	\card
	\cardstrip
	\cardbanner{banner/white.png}
	\cardtitle{\footnotesize{Neue Regeln (4/5)}\qquad}
	\cardcontent{\emph{Zustände\textsuperscript{*}:} In \emph{Nocturne} gibt es drei \emph{Plagen} und eine Königreichkarte, die Spielern einen \emph{Zustand} verschaffen. \emph{Zustände} sind Karten, die vor einem Spieler abgelegt werden und eine zusätzliche Regel beinhalten. Es gibt zwei \emph{Zustände}, die einen einzelnen Zug betreffen und dann zurückgelegt werden (\emph{GETÄUSCHT} und \emph{NEIDISCH}), zwei \emph{Zustände}, die die Punktewertung beeinflussen (\emph{ELENDIG} und \emph{DOPPELT ELENDIG}) und einen \emph{Zustand}, der alle Züge eines Spielers beeinflusst, bis ein anderer Spieler den \emph{Zustand} erhält (\emph{IM WALD VERIRRT}). Die Zustände \emph{GETÄUSCHT}/\emph{NEIDISCH} sowie \emph{ELENDIG}/\emph{DOPPELT ELENDIG} sind jeweils auf einer Karte (Vorder- und Rückseite) – die jeweils gültige Seite liegt oben. Ein \emph{Zustand} ist nur so lange gültig, wie die entsprechende Karte vor einem Spieler liegt.}
\end{tikzpicture}
\hspace{-0.6cm}
\begin{tikzpicture}
	\card
	\cardstrip
	\cardbanner{banner/white.png}
	\cardtitle{\footnotesize{Neue Regeln (5/5)}\qquad}
	\cardcontent{\textsuperscript{*} \emph{Zustände}, \emph{Plagen} und \emph{Gaben} gehören nicht zum Vorrat und sind keine \enquote{Karten} im Sinne des Spiels. Sie werden nicht beachtet, wenn es darum geht \enquote{eine Karte zu nehmen} oder wenn alle \enquote{Karten im Spiel} betrachtet werden. Sie werden ebenso wie die Ereignisse aus \emph{Abenteuer} und die Landmarken aus \emph{Empires} niemals in das Kartendeck eines Spielers integriert.
		
	\medskip
		
	\emph{Die Dauerkarten:} In \emph{Nocturne} gibt es Dauerkarten, die z.B. bereits aus \emph{Seaside} und \emph{Abenteuer} bekannt sind. Die orangefarbenen Dauerkarten beinhalten Anweisungen, die zu einem späteren Zeitpunkt ausgeführt werden. Sie werden normalerweise nicht in der Aufräumphase des Zuges abgelegt, in dem sie ausgespielt wurden, sondern bleiben bis zur Aufräumphase des Zuges, in dem sie letztmals eine Wirkung haben, im Spiel. Wird eine Dauerkarte mehrfach ausgespielt (z.B. durch den \emph{THRONSAAL} aus dem \emph{Basisspiel}), bleibt die verursachende Karte ebenfalls solange im Spiel, bis die Dauerkarte abgelegt wird.
	
	\medskip
	
	Um anzuzeigen, dass eine Dauerkarte in der aktuellen Aufräumphase noch nicht abgelegt wird, wird sie in eine eigene Reihe oberhalb der restlichen ausgespielten Karten gelegt.}
\end{tikzpicture}
\hspace{-0.6cm}
\begin{tikzpicture}
	\card
	\cardstrip
	\cardbanner{banner/white.png}
	\cardtitle{\scriptsize{Empfohlene 10er Sätze\qquad}}
	\cardcontent{\emph{Abenddämmerung:}\\
	Folterknecht, Getreuer Hund, Kloster, Nachtwache, Narr (\rightarrow Im Wald verirrt), Schäferin, Schuster, Seliges Dorf, Sündenpfuhl, Tragischer Held

	\medskip

	\emph{Mitternacht:}\\
	Druidin (\rightarrow Geschenk des Feuers, \rightarrow Geschenk des Sumpfes, \rightarrow Geschenk des Windes), Exorzistin (\rightarrow Geist, Teufelchen, \rightarrow Irrlicht), Geheime Höhle (\rightarrow Wunsch), Kobold, Konklave, Krypta, Plünderer, Puka, Teufelswerkstatt (\rightarrow Teufelchen), Verfluchtes Dorf

	\medskip

	\emph{Nachtschicht} (+ \underline{Basisspiel 2. Edition}): \\
	Druidin (\rightarrow Geschenk der Erde, \rightarrow Geschenk des Feuers, \rightarrow Geschenk des Waldes), Exorzistin (\rightarrow Geist, \rightarrow Teufelchen, \rightarrow Irrlicht), Geisterstadt, Götze, Nachtwache, \underline{Banditin, Gärten, Mine, Schmiede, Wilddiebin}

	\medskip

	\emph{Müßiggang} (+ \underline{Basisspiel 2. Edition}): \\
	Konklave, Minnesängerin, Teufelswerkstatt (\rightarrow Teufelchen), Tragischer Held, Verfluchtes Dorf, \underline{Geldverleiher, Händlerin, Keller, Markt, Vorbotin}}
\end{tikzpicture}
\hspace{-0.6cm}
\begin{tikzpicture}
	\card
	\cardstrip
	\cardbanner{banner/white.png}
	\cardtitle{\scriptsize{Empfohlene 10er Sätze\qquad}}
	\cardcontent{\emph{Das neue Schwarz} (+ \underline{Seaside}): \\
	Geheime Höhle (\rightarrow Wunsch), Geisterstadt, Plünderer, Schuster, Sündenpfuhl, \underline{Außenposten, Hafen, Handelsschiff, Karawanenwächter, Taktiker}
	
	\medskip
	
	\emph{Luftschloss} (+ \underline{Empires}): \\
	Exorzistin (\rightarrow Geist, \rightarrow Teufelchen, \rightarrow Irrlicht), Friedhof, Narr (\rightarrow Im Wald verirrt), Schäferin, Wechselbalg, \underline{Archiv, Katapult/Felsen, Ingenieurin, Schlösser, Tempel - Landmarke: Grabmal}
	
	\medskip
	
	\emph{Puka-Possen} (+ \underline{Empires}): \\
	Attentäter, Fee, Geisterstadt, Getreuer Hund, Puka, \underline{Forum, Gärtnerin, Opfer, Siedler/Emsiges Dorf, Wagenrennen – Ereignis: Bankett}}
\end{tikzpicture}
\hspace{0.6cm}

	    % Basic settings for this card set
\renewcommand{\cardcolor}{promo}
\renewcommand{\cardextension}{Promokarte}
\renewcommand{\cardextensiontitle}{}
\renewcommand{\seticon}{empty.png}

\clearpage
\newpage
\section{\cardextension}

\begin{tikzpicture}
	\card
	\cardstrip
	\cardbanner{banner/white.png}
	\cardtitle{Platzhalter\quad}
\end{tikzpicture}
\hspace{-0.6cm}
\begin{tikzpicture}
	\card
	\cardstrip
	\cardbanner{banner/white.png}
	\cardicon{icons/coin.png}
	\cardprice{4}
	\cardtitle{Gesandter}
	\cardcontent{Die aufgedeckten Karten legst du zunächst offen vor dir aus. Kannst du auch nach dem Mischen deines Ablagestapels nur weniger als 5 Karten aufdecken, deckst du nur so viele auf, wie möglich. Dann wählt der Spieler links von dir eine dieser offenen Karten. Die gewählte Karte legst du auf deinen Ablagestapel, die restlichen Karten nimmst du auf die Hand.}
\end{tikzpicture}
\hspace{-0.6cm}
\begin{tikzpicture}
	\card
	\cardstrip
	\cardbanner{banner/white.png}
	\cardicon{icons/coin.png}
	\cardprice{3}
	\cardtitle{\footnotesize{Schwarzmarkt}}
	\cardcontent{\miniscule{Vor dem Spiel mit der Aktionskarte Schwarzmarkt muss der dazugehörige Schwarzmarkt-Stapel zusammengestellt werden. Die Spieler einigen sich welche Karten verwendet werden sollen.

	\medskip

	Für die Zusammenstellung ist Folgendes zu beachten:
	\begin{itemize}
	\item Der Schwarzmarkt-Stapel muss aus mindestens 15 Karten bestehen.
	\item Es dürfen nur Königreichkarten verwendet werden, die nicht im Vorrat sind.
	\item Jede Karte darf nur einmal verwendet werden.
	\end{itemize}
	Die Spieler dürfen sich die Karten vor dem Spiel ansehen. Dann werden die verwendeten Karten gemischt und verdeckt neben dem Vorrat bereit gelegt. Der Schwarzmarkt-Stapel ist nicht Teil des Vorrats. Er wird weder für die Spielende-Bedingung beachtet, noch für andere Zwecke, die auf den Vorrat Bezug nehmen, z.B. Aktionskarten, die erlauben eine Karte zu nehmen (Werkstatt).

	\medskip

	Spielst du den Schwarzmarkt aus, musst du zunächst die obersten 3 Karten vom Schwarzmarkt-Stapel aufdecken. Nun darfst du eine dieser Karten kaufen. Es handelt sich hierbei um einen Kauf in der Aktionsphase, d. h. du darfst sowohl Geldkarten als auch virtuelles Geld verwenden. Der Kauf läuft also in gleicher Weise ab, wie in der Kaufphase. (Der einzige Unterschied ist, dass du nicht wie üblich eine Karte aus dem Vorrat kaufst, sondern eine der 3 aufgedeckten Karten vom Schwarzmarktstapel.) Die nicht gekauften Karten legst du in beliebiger Reihenfolge verdeckt unter den Schwarzmarkt-Stapel zurück. Du musst deinen Mitspielern nicht zeigen in welcher Reihenfolge du die Karten zurücklegst. Die ausgelspielten Geldkarten lässt du bis zur Aufräumphase vor dir liegen. Dieser Kauf verbraucht nicht deinen freien Kauf, du darfst also in der Kaufphase (mindestens) eine weitere Karte kaufen. Nicht verwendetes virtuelles Geld und auch überzählige Münzen ausgespielter Geldkarten stehen dir in der Kaufphase zur Verfügung.

	\medskip

	Wenn du den Schwarzmarkt ausspielst, aber keine Karte kaufen willst oder kannst (z.B. weil der Schwarzmarktstapel leer ist oder du nicht genügend Geld hast), erhältst du trotzdem +\coin[2] für die Kaufphase.}}
\end{tikzpicture}
\hspace{-0.6cm}
\begin{tikzpicture}
	\card
	\cardstrip
	\cardbanner{banner/gold.png}
	\cardicon{icons/coin.png}
	\cardprice{5}
	\cardtitle{Geldversteck}
	\cardcontent{Das Geldversteck ist eine Geldkarte mit dem Wert \coin[2], wie ein Silber. In der Aufräumphase wird das Geldversteck wie üblich abgelegt. Immer wenn ein Spieler seinen Ablagestapel mischt, führt er folgende Schritte aus:
	\begin{noindlist}
	\item Er sucht zunächst alle Geldversteck-Karten aus seinem Ablagestapel und legt sie beiseite.
	\item Dann mischt er die verbliebenen Karten des Ablagestapels.
	\item Nun sortiert er die Karten Geldversteck wieder an beliebigen Stellen seiner Wahl (auch ganz oben oder unten) in den Stapel ein. Dabei darf er Karten des Stapels abzählen, aber nicht ansehen.
	\item Zuletzt legt er den Stapel als neuen Nachziehstapel bereit.
	\end{noindlist}}
\end{tikzpicture}
\hspace{-0.6cm}
\begin{tikzpicture}
	\card
	\cardstrip
	\cardbanner{banner/white.png}
	\cardicon{icons/coin.png}
	\cardprice{4}
	\cardtitle{Carcassonne}
	\cardcontent{\emph{Errata:} Der Kartentext ist falsch, es sollte \enquote{Wenn du zu Beginn deiner Aufräumphase nicht mehr als eine weitere Aktionskarte im Spiel hast\dots} statt \enquote{Wenn du zu Beginn deiner Aufräumphase nur noch eine weitere Aktionskarte im Spiel hast\dots} heißen.

	\medskip

	Zuerst ziehst du immer eine Karte nach und erhältst +2 Aktionen. Wenn du zu Beginn deiner Aufräumphase die Karte Carcassonne und nicht mehr als eine weitere Aktionskarte im Spiel hast, darfst du dich entscheiden, Carcassonne oben auf deinen Nachziehstapel zu legen oder wie üblich auf den Ablagestapel zu legen. Hast du die Karte Carcassonne genau zweimal im Spiel und ansonsten keine weitere Aktionskarte im Spiel, darfst du eine oder beide Karten Carcassonne auf deinen Nachziehstapel legen.}
\end{tikzpicture}
\hspace{-0.6cm}
\begin{tikzpicture}
	\card
	\cardstrip
	\cardbanner{banner/white.png}
	\cardicon{icons/coin.png}
	\cardprice{4}
	\cardtitle{\scriptsize{Befestigtes Dorf}}
	\cardcontent{Zuerst ziehst du immer eine Karte nach und erhältst +2 Aktionen. Wenn du zu Beginn deiner Aufräumphase die Karte Befestigtes Dorf und nicht mehr als eine weitere Aktionskarte im Spiel hast, darfst du dich entscheiden, Befestigtes Dorf oben auf deinen Nachziehstapel zu legen oder wie üblich auf den Ablagestapel zu legen. Hast du die Karte Befestigtes Dorf genau zweimal im Spiel und ansonsten keine weitere Aktionskarte im Spiel, darfst du eine oder beide Karten Befestigtes Dorf auf deinen Nachziehstapel legen.}
\end{tikzpicture}
\hspace{-0.6cm}
\begin{tikzpicture}
	\card
	\cardstrip
	\cardbanner{banner/white.png}
	\cardicon{icons/coin.png}
	\cardprice{5}
	\cardtitle{Gouverneur}
	\cardcontent{Zuerst erhältst du +1 Aktion. Dann wählst du eine der folgenden Optionen:
	\begin{noindlist}
	\item Du ziehst 3 Karten und jeder andere Spieler zieht 1 Karte.
	\item Du nimmst dir 1 Gold und jeder andere Spieler nimmt sich 1 Silber.
	\item Du darfst 1 Karte aus deiner Hand entsorgen und nimmst dir 1 Karte, die genau
	\coin[2] mehr kostet als die entsorgte Karte und jeder andere Spieler darf 1 Karte entsorgen, die genau \coin[1] mehr kostet als die entsorgte Karte.
	\end{noindlist}
	Geh nach der Reihenfolge, beginnend mit dir selbst. Die Karten werden vom Vorrat genommen und auf den Ablagestapel gelegt. Sind im Vorrat keine Karten mehr übrig, können keine entsprechenden Karten genommen werden. Wählst du z.B. die zweite Option und es ist nur noch 1 Silber im Vorrat, bekommt es der Spieler links von dir und die anderen Spieler erhalten nichts. Bei der dritten Option nimmst du dir nur dann 1 Karte, wenn du vorher 1 Karte entsorgt hast und wenn 1 Karte mit den genauen geforderten Kosten im Vorrat verfügbar ist. Wenn du 1 Karte entsorgst, musst du dir 1 Karte nehmen, sofern möglich. Du kannst den ausgespielten Gouverneur nicht entsorgen, da er sich nicht mehr auf deiner Hand befindet. Du kannst aber einen anderen Gouverneur aus deiner Hand entsorgen.}
\end{tikzpicture}
\hspace{-0.6cm}
\begin{tikzpicture}
	\card
	\cardstrip
	\cardbanner{banner/white.png}
	\cardicon{icons/coin.png}
	\cardprice{8}
	\cardtitle{Prinz}
	\cardcontent{\tiny{\begin{Spacing}{1}
	\vspace{1em}
	Wenn du dich entscheidest, einen Prinzen zu spielen, legst du ihn sofort zur Seite. Er befindet sich damit allerdings nicht im Spiel. Anschließend wählst du eine Aktionskarte von deiner Hand, die zu diesem Zeitpunkt maximal \coin[4] kostet und legst diese ebenfalls zur Seite.

	\medskip

	Auch Karten mit Kosten von \coin[0], wie die Preiskarten aus Reiche Ernte sowie Karten mit Kosten von \coin[X] + aus Die Gilden dürfen mit Hilfe des Prinzen zur Seite gelegt werden. Karten, deren Kosten einen Trank enthalten, dürfen dagegen nicht zur Seite gelegt werden.

	\medskip

	Zu Beginn eines Zuges musst du die zur Seite gelegte Aktionskarte spielen. Sie verbraucht dabei nicht deine freie Aktion. Sobald du die Aktionskarte ablegen musst, legst du sie stattdessen wieder zur Seite. Wenn du die Aktionskarte während deines Spielzugs aus dem Spielbereich entfernst (z.B. sie entsorgen musst) und sie dementsprechend in der Aufräumphase nicht mehr im Spiel ist, darfst du die Karte nicht wieder zur Seite legen. Die Wirkung des Prinzen wird sofort aufgehoben.

	\medskip

	Wenn du zu Beginn deines Zuges mehrere Karten spielen musst (z.B. Aktionskarten durch mehrere Prinzen oder Dauerkarten aus Seaside), darfst du selbst entscheiden, in welcher Reihenfolge du sie ausspielst. Die Karte \emph{PRINZ} muss zur Seite gelegt werden, damit sie einen Effekt hat. Den Prinzen zum Beispiel auf einen Thronsaal zu spielen erlaubt dir nicht, zwei Karten zur Seite zu legen, da du den Prinzen nur einmal zur Seite legen kannst. Alle zur Seite gelegten Prinzen und Aktionskarten gehören zum Kartensatz eines Spielers.
	\end{Spacing}}}
\end{tikzpicture}
\hspace{-0.6cm}
\begin{tikzpicture}
	\card
	\cardstrip
	\cardbanner{banner/white.png}
	\cardtitle{Sauna/Eisloch\qquad}
	\cardcontent{Spielvorbereitung: Legt auf diese Karte 5 Eisloch und oben darauf 5 Sauna.

	\medskip

	Es darf immer nur die oberste Karte des Stapels genommen oder gekauft werden.}
\end{tikzpicture}
\hspace{-0.6cm}
\begin{tikzpicture}
	\card
	\cardstrip
	\cardbanner{banner/white.png}
	\cardicon{icons/coin.png}
	\cardprice{4}
	\cardtitle{Sauna}
	\cardcontent{\tiny{\begin{Spacing}{1}
	\vspace{1em}
	\emph{Siehe auch die Hinweise zur Karte Eisloch!}

	\medskip

	Wenn du die Sauna ausspielst, ziehst du zuerst eine Karte und bekommst +1 Aktion. Du kannst dann sofort ein Eisloch aus deiner Hand ausspielen. Das verbraucht keine deiner Aktionen, einschließlich der Aktion, die die Sauna gewährt. Du darfst ein Eisloch auf diese Weise nur direkt nach dem Ausspielen der Sauna spielen, nicht, wenn du zwischendurch eine andere Aktionskarte ausgespielt hast, selbst wenn du eine Sauna im Spiel hast.

	\medskip

	Solange die Sauna im Spiel ist, darfst du jedes Mal, wenn du ein Silber ausspielst, eine Karte aus deiner Hand entsorgen. Wenn du das gleiche Silber mehrmals spielst, wie z.B. mit dem Falschgeld (Dominion - Dark Ages) oder der Krone (Dominion - Empires), darfst du jedes Mal eine Karte entsorgen, wenn du das Silber spielst.

	\medskip

	Wenn du ein Silber ausspielst, kannst du dir jedes Mal überlegen, ob du eine Karte entsorgen möchtest, du musst diese Entscheidung nicht einmal für den gesamten Zug treffen. Wenn du mehrere Saunen im Spiel hast und ein Silber ausspielst, kannst du für jede Sauna, die du im Spiel hast, eine Karte aus deiner Hand entsorgen. Du kannst dich jedes Mal immer noch dazu entschließen, keine Karte zu entsorgen.

	\medskip

	Wenn die Sauna das Spiel verlässt, weil sie zum Beispiel mit der Prozession (Dominion - Dark Ages) entsorgt wurde, kann ihr Effekt nicht mehr genutzt werden.
	\end{Spacing}}}
\end{tikzpicture}
\hspace{-0.6cm}
\begin{tikzpicture}
\card
	\cardstrip
	\cardbanner{banner/white.png}
	\cardicon{icons/coin.png}
	\cardprice{5}
	\cardtitle{Eisloch}
	\cardcontent{Wenn du das Eisloch ausspielst, ziehst du zuerst 3 Karten. Du kannst dann sofort eine Sauna aus deiner Hand ausspielen. Das verbraucht keine deiner Aktionen, und du erhältst trotzdem die +1 Aktion der Sauna, wenn du sie auf diese Weise spielst.

	\medskip

	Du darfst eine Sauna auf diese Weise nur direkt nach dem Ausspielen des Eislochs spielen, nicht, wenn du zwischendurch eine andere Aktionskarte ausgespielt hast, selbst wenn du ein Eisloch im Spiel hast.

	\medskip

	\emph{Folgendes gilt sowohl für das Eisloch als auch für die Sauna:}

	\medskip

	Du kannst die Sauna und das Eisloch durch die Effekte der jeweils anderen Karte abwechselnd spielen, wobei du nur die Aktion für die erste ausgespielte Karte verbrauchst. Du kannst damit fortfahren, bis du nach dem Ausspielen der einen Karte die entsprechende andere Karte nicht mehr auf der Hand hast.

	\medskip

	Wenn du eine Sauna ausspielst, kannst du nicht sofort eine weitere Sauna aus deiner Hand ausspielen, ohne eine Aktion zu verbrauchen. Das gleiche gilt für das Ausspielen eines Eislochs nach einem anderen Eisloch.}
\end{tikzpicture}
\hspace{-0.6cm}
\begin{tikzpicture}
	\card
	\cardstrip
	\cardbanner{banner/white.png}
	\cardtitle{Ereignisse\qquad}
	\cardcontent{\tiny{\begin{Spacing}{1}
	\vspace{1em}
	\emph{Einladung:} Wenn du das Ereignis kaufst, nimmst du dir vom Vorrat eine Aktionskarte, die bis zu \coin[4] kostet und legst sie offen zur Seite. Wenn du sie beiseite gelegt hast, dann spielst du die Aktionskarte zu Beginn des nächsten Zuges aus. Das Ausspielen verbraucht nicht deine Standardaktion für den Zug. Um dich daran zu erinnern, dass du die Karte in deinem nächsten Zug ausspielst, kannst du sie seitwärts oder diagonal drehen, und sie dann richtig herum drehen, sobald du sie ausspielst.

	\medskip

	Wenn du die Aktionskarte bewegst, nachdem du sie genommen, aber bevor du sie zur Seite gelegt hast (z.B. indem du sie mit dem Wachturm (Dominion – Blütezeit) auf den Nachziehstapel legst), dann wird die Einladung zu der Aktionskarte den \enquote{Anschluss verlieren} und nicht in der Lage sein, sie zur Seite zu legen; in diesem Fall wirst du sie zu Beginn deines nächsten Zuges nicht ausspielen.

	\medskip

	Wenn du die Einladung nutzt, um ein Nomadencamp (Dominion – Hinterland) zu nehmen, wird die Einladung wissen, dass das Nomadencamp auf deinem Nachziehstapel zu finden ist, so dass du es in diesem Fall zur Seite legst (sofern du es nicht über einen anderen Effekt an eine andere Stelle verschoben hast).

	\medskip

	\emph{Errata:} Der letzte Satz auf der Karte müsste heißen: \enquote{Wenn du das tust, spiele sie zu Beginn deines nächsten Zuges.}
	\end{Spacing}}}
\end{tikzpicture}
\hspace{-0.6cm}
\begin{tikzpicture}
	\card
	\cardstrip
	\cardbanner{banner/white.png}
	\cardicon{icons/coin.png}
	\cardprice{5}
	\cardtitle{Höflinge}
	\cardcontent{
	Decke eine Karte aus deiner Hand auf. Zähle dann die Typen, denen diese  Karte  angehört  –  also  Aktion,  Geld,  Reaktion,  Angriff,  Punkte,  Fluch  etc.  Pro  Typ,  dem  die  Karte  angehört,  entscheidest  du  dich  für  eine  der vier angegebenen Optionen. Dabei darfst du keine der Optionen doppelt auswählen. 
	
		\medskip

	Wenn du zum Beispiel eine \emph{PATROUILLE} aus \emph{Ergänzung - Die Intrige} (Aktion) aufdeckst, darfst du eine Option auswählen, deckst du einen \emph{KARAWANENWÄCHTER} aus \emph{Abenteuer} (Aktion – Dauer – Reaktion) auf, darfst du 3 unterschiedliche Optionen wählen. Entscheidest du dich für das \emph{Gold}, legst du dieses auf den Ablagestapel. Kannst du keine Handkarte aufdecken, erhältst du nichts.}
\end{tikzpicture}
\hspace{-0.6cm}
\begin{tikzpicture}
	\card
	\cardstrip
	\cardbanner{banner/white.png}
	\cardicon{icons/coin.png}
	\cardprice{4}
	\cardtitle{Abbruch}
	\cardcontent{Entsorgen ist nicht optional.
	
			\medskip
			
	Entsorgst du eine Karte, die \coin[0] kostet, oder hast du keine Karte mehr auf der Hand, die du entsorgen könntest, passiert sonst nichts. 

			\medskip
			
	Entsorgst du eine Karte, die \coin[1] oder mehr kostet, nimmst du dir zuerst eine billigere Karte, anschließend ein \emph{GOLD}. Die Karten müssen aus dem Vorrat genommen und in der Reihenfolge, in der sie genommen wurden, auf den Ablagestapel gelegt werden, d.h. das \emph{GOLD} zuletzt. Zwar wird fast immer eine billigere Karte im Vorrat vorhanden sein, da \emph{KUPFER} und \emph{FLUCH} \coin[0] kosten, sollte dies aber einmal nicht der Fall sein, darfst du dir dennoch ein \emph{GOLD} nehmen. Sollte kein \emph{GOLD} mehr im Vorrat vorhanden sein, darfst du dir dennoch die billigere Karte nehmen. 

			\medskip
	
	Karten, die nur Kosten in Form von \potion\ (wie die \emph{VERWANDLUNG} aus \emph{Die Alchemisten}) oder \hex\ aufweisen (wie die \emph{INGENIEURIN} aus \emph{Empires}), kosten nicht \coin[1] oder mehr.}
\end{tikzpicture}
\hspace{-0.6cm}
\begin{tikzpicture}
	\card
	\cardstrip
	\cardbanner{banner/orange.png}
	\cardicon{icons/coin.png}
	\cardprice{3}
	\cardtitle{\scriptsize{Schweriner Dom}}
	\cardcontent{Du kannst keine, eine, zwei oder drei Karten aus deiner Hand verdeckt zur Seite legen, darfst sie aber ansehen.
	
			\medskip
			
	Unabhängig davon, wie viele Karten du zur Seite gelegt hast, kannst du zu Beginn deines nächsten Zuges eine Karte entsorgen.
	
			\medskip
			
	Die Karte, die du entsorgst, kann eine Karte sein, die du zur Seite gelegt hast, oder eine, die du bereits auf der Hand hattest.
	
			\medskip
			
	Spielst du mehrere Male den \emph{SCHWERINER DOM} (oder einen \emph{SCHWERINER DOM} mehrfach, wie z.B. mittels \emph{THRONSAAL} aus dem \emph{Basisspiel}), darfst du entsprechend viele Sätze von jeweils bis zu drei Karten verdeckt zur Seite legen. Zu Beginn deines nächsten Zuges verfährst du folgendermaßen: Nimm dir einen Satz der zur Seite gelegten Karten auf die Hand, anschließend darfst du eine Karte entsorgen, dann wiederhole diese Schritte, bis du alle Sätze auf die Hand genommen hast. Die Reihenfolge, in der du die einzelnen Sätze auf die Hand nimmst, ist frei wählbar.}
\end{tikzpicture}
\hspace{-0.6cm}
\begin{tikzpicture}
	\card
	\cardstrip
	\cardbanner{banner/orange.png}
	\cardicon{icons/coin.png}
	\cardprice{6}
	\cardtitle{\scriptsize{Kapitän Tobias}}
	\cardcontent{\miniscule{\begin{Spacing}{1}
	\vspace{1em}
	Du wählst eine Aktionskarte aus dem Vorrat, die keine Dauerkarte und keine Befehlskarte* ist, und bis zu \coin[\miniscule{4}] kostet, spielst sie und lässt sie im Vorrat liegen. Zu Beginn deines nächsten Zuges wiederholst Du diesen Vorgang; dabei darfst du eine andere Karte auswählen, aber auch dieselbe, sofern diese noch im Vorrat vorhanden ist.
	
	Es kann nur eine Karte aus dem Vorrat gespielt werden, die sichtbar ist und oben auf einem Stapel liegt; weder kann eine Karte von einem leeren Stapel gespielt werden, noch eine Karte von einem gemischten Vorratsstapel, die noch nicht aufgedeckt wurde, oder bereits vergriffen ist, und auch keine Karte, die nicht zum Vorrat gehört (wie bspw.\ der \emph{SÖLDNER} aus \emph{Dark Ages}).
	
	Wenn es in dem Zug, in dem du \emph{KAPITÄN TOBIAS} spielst, im Vorrat keine Aktionskarte gibt, die keine Dauerkarte und keine Befehlskarte ist und bis zu \coin[\miniscule{4}] kostet, bleibt \emph{KAPITÄN TOBIAS} trotzdem im Spiel und du versuchst, zu Beginn deines nächsten Zuges eine solche Karte zu spielen.
	
		Wenn \emph{KAPITÄN TOBIAS} eine Karte spielt, die eine Dauerkarte spielt, beeinflusst das nicht, in welcher Runde \emph{KAPITÄN TOBIAS} aus dem Spiel genommen wird.
	
	\medskip
	
	Die gespielte Aktionskarte bleibt im Vorrat; versucht irgendein Effekt, diese Karte zu bewegen (bspw.\ die \emph{INSEL} aus \emph{Seaside}, die beim Spielen auf dein Insel-Tableau gelegt wird), wird das Bewegen nicht ausgeführt.
	
	\emph{KAPITÄN TOBIAS} kann eine Karte spielen, die sich selbst entsorgt, wenn sie gespielt wird; immer wenn diese Karte überprüft, ob sie entsorgt wurde (wie das \emph{BERGWERK} aus \emph{Die Intrige}), so gilt sie als nicht entsorgt; wenn sie nicht überprüft, ob sie entsorgt wurde (wie die \emph{SCHAUSPIELTRUPPE} aus \emph{Renaissance}), funktioniert sie wie gewohnt.
		
	Karten, die normalerweise andere Karten aus dem Vorrat bewegen, können sich selbst bewegen, wenn sie mittels \emph{KAPITÄN TOBIAS} gespielt werden; bspw.\ kann die \emph{WERKSTATT} aus dem \emph{Basisspiel} sich selbst nehmen und die \emph{Herumtreiberin} aus \emph{Ergänzung - Die Intrige} kann sich selbst entsorgen.
	
	Da die gespielte Karte nicht im Spiel ist, haben Fähigkeiten, die an die Bedingung "`Solange diese Karte im Spiel ist"' geknüpft sind (wie bspw.\ beim \emph{HALSABSCHNEIDER} aus \emph{Blütezeit}), keine Auswirkungen.
	
	\medskip
	
	*Befehlskarten sind ein neuer Kartentyp, der in den Errata 2019 eingeführt wurde, um zu verhindern, dass manche Fähigkeiten in Endlosschleife gespielt werden können. Befehlskarten sind Emulatoren, die Karten aus dem Vorrat spielen, sie aber dort belassen (bspw.\ \emph{VOGELFREIE} aus \emph{Dark Ages} und der \emph{LEHNSHERR} aus \emph{Empires}).
	\end{Spacing}}}
\end{tikzpicture}
\hspace{-0.6cm}
\begin{tikzpicture}
	\card
	\cardstrip
	\cardbanner{banner/white.png}
	\cardtitle{\scriptsize{Spielvorbereitung}\qquad}
	\cardcontent{Promo-Karten nach Belieben zum Aufbau des Königreiches verwenden.}
\end{tikzpicture}
\hspace{0.6cm}

	    % Basic settings for this card set
\renewcommand{\cardcolor}{basicgame}
\renewcommand{\cardextension}{Template}
\renewcommand{\cardextensiontitle}{Template}

\clearpage
\newpage
\section{\cardextension \ - \cardextensiontitle}

\begin{tikzpicture}
	\card
	\cardstrip
	\cardbanner{banner/white.png}
	\cardicon{banner/coin.png}
	\cardprice{}
	\cardtitle{}
	\cardcontent{}
\end{tikzpicture}
\hspace{-1cm}
\begin{tikzpicture}
	\card
	\cardstrip
	\cardbanner{banner/gold.png}
	\cardicon{banner/coin.png}
	\cardprice{}
	\cardtitle{}
	\cardcontent{}
\end{tikzpicture}
\hspace{-1cm}
\begin{tikzpicture}
	\card
	\cardstrip
	\cardbanner{banner/purple.png}
	\cardicon{banner/coin.png}
	\cardprice{}
	\cardtitle{}
	\cardcontent{}
\end{tikzpicture}
\hspace{-1cm}
\begin{tikzpicture}
	\card
	\cardstrip
	\cardbanner{banner/blue.png}
	\cardicon{banner/coin.png}
	\cardprice{}
	\cardiconaddition{banner/hex.png}
	\cardpriceaddition{}
	\cardtitle{}
	\cardcontent{}
\end{tikzpicture}
\hspace{-1cm}
\begin{tikzpicture}
	\card
	\cardstrip
	\cardbanner{banner/green.png}
	\cardicon{banner/coin.png}
	\cardprice{}
	\cardiconaddition{banner/potion.png}
	\cardpriceaddition{}
	\cardtitle{}
	\cardcontent{}
\end{tikzpicture}
\hspace{-1cm}
\begin{tikzpicture}
	\card
	\cardstrip
	\cardbanner{banner/white.png}
	\cardtitle{\scriptsize{Empfohlene 10er Sätze\qquad}}
	\cardcontent{\emph{Name:}
	\\
	Karten ...
	\\
	\smallskip
	\\
	\emph{Name:}
	\\
	Karten ...
	\\
	\smallskip
	\\
	\emph{Name:}
	\\
	Karten ...
	\\
	\smallskip
	\\
	\emph{Name:}
	\\
	Karten ...
	\\
	\smallskip
	\\
	\emph{Name:}
	\\
	Karten ...
	\\
	\smallskip
	\\
	\emph{Name:}
	\\
	Karten ...
	\\}
\end{tikzpicture}
\hspace{1cm}
	\end{center}
\end{document}
